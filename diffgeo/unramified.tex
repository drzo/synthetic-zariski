\subsection{Definition}

\begin{definition}
For any types $X,Y$, a map $f:X\to Y$ is called formally unramified if for any closed proposition $P$ such that $\neg\neg P$ the following square has at most one lift:
 \begin{center}
      \begin{tikzcd}
        P\ar[r]\ar[d] & X\ar[d,"f"] \\
       1 \ar[r]\ar[ru,dashed] & Y
      \end{tikzcd}
    \end{center}
\end{definition}

\begin{definition}
A map $f:X\to Y$ is unramified if $X$ and $Y$ are schemes and $f$ is formally unramified.
\end{definition}

It is clear that formally étale maps are formally unramified. As usual a type $X$ is called formally unramified if the map from $X$ to $1$ is unramified, and a map is formally unramified if and only if all its fibers are formally unramified.

\begin{lemma}
\label{equivalence-unramified}
Having at most one lift against the following class of maps is equivalent:
\begin{enumerate}[(i)]
\item Maps of the form $\Spec(A/N)\to \Spec(A)$ for $A$ f.g. algebra and $N$ a nilpotent ideal.
\item Maps of the form $P\to 1$ for $P$ a closed dense proposition.
\item Closed dense embeddings of types.
\item Maps of the form $(\epsilon=0)\to 1$ for $\epsilon:R$ such that $\epsilon^2=0$.
\item Maps of the form $\Spec(A/(a)) \to \Spec(A)$ for $A$ f.g. algebra and $a:A$ such that $a^2=0$.
\end{enumerate}
\end{lemma}

\begin{proof}
Same as \cref{equivalence-etale}.
\end{proof}

\begin{lemma}
A map is formally unramified if and only if all its fibers are formally unramified.
\end{lemma}

\begin{proof}
By \cref{lifting-defined-fiberwise} and the characterisation (ii) from the previous lemma.
\end{proof}

\begin{lemma}
A type $X$ is formally unramified if and only if for any $x,y:X$ the type $x=y$ is formally étale.
\end{lemma}
\begin{proof}
A type $X$ is formally unramified iff for any closed dense proposition $P$ the fibers of the canonical map in:
\[X\to X^P\]
are propositions. This is equivalent to the induced maps in:
\[(x=y)\to (x=y)^P\]
being equivalences for all $x,y:X$, i.e. all types $x=y$ being formally étale.
\end{proof}

In the langage of modalities, this means that formally unramified types are precisely formally étale-separated types. So being formally unramified is a (non-lex) modality, so that:

\begin{proposition}
We have the following stability results:
\begin{itemize}
\item If $X$ is any type and for all $x:X$ we have a formally unramified type $Y_x$, then:
\[\prod_{x:X}Y_x\]
is formally unramified. 
\item  If $X$ is formally unramified and for all $x:X$ we have a formally unramified type $Y_x$, then:
\[\sum_{x:X}Y_x\]
is formally unramified.
\end{itemize}
\end{proposition}

\subsection{Examples}

Formally étale types are formally unramified.

\begin{proposition}
Any proposition is formally unramified.
\end{proposition}

This means any embedding is formally unramified.

\begin{proposition}
Subtype of formally étale types are formally unramified.
\end{proposition}

\begin{proof}
By the previous example and stability by dependent sum.
\end{proof}

We will see all examples are of this form.

\subsection{Unramified schemes}

\begin{proposition}\label{characterisation-unramified-schemes}
A scheme $X$ is unramified if and only if any of the following propositions hold:
  \begin{enumerate}[(i)]
  \item For all $x,y:X$, the proposition $x=y$ is open.
  \item For all $x,y:X$, the proposition $x=y$ is $\neg\neg$-stable.
  \item For all $x:X$, we have $T_x(X)=0$.
  \item For all infinitesimal pointed type $(D,*)$ (meaning that for all $x:D$ we have $\neg\neg(x=*)$), any map from $D$ to $X$ is constant.
  \end{enumerate}
\end{proposition}

\begin{proof}
First we prove that the four propositions are equivalent:

(i) implies (ii) because open propositions are $\neg\neg$-stable.

(ii) implies (iv) because for any $f:D\to X$ and $x:D$ we have $\neg\neg(x=*)$ so that $\neg\neg(f(x)=f(*))$ and finally $f(x)=f(*)$.

(iv) implies (iii) by taking $D=\mathbb{D}(1)$.

(iii) implies (i) because for any $x:X$ there an open affine $U$ such that $x\in U$. Then $x=y$ is equivalent to $(y\in U)\land x=_U y$, but $x=_Uy$ is decidable by \cref{decidable-of-tangent} and open propositions are stable by $\Sigma$.

Now we check they are equivalent to being unramified:

(ii) implies unramified, indeed we need to check that for $x,y:X$ and $P$ closed dense such that $P\to x=y$, we have $x=y$. But $\neg\neg P$ so that $\neg\neg(x=y)$, and by (ii) we have $x=y$.

Unramified implies (iii) because it implies having at most one lifting against any closed dense subtype, so that by considering $1\subset \mathbb{D}(1)$ it implies having at most one tangent vector.
\end{proof}

\subsection{Separated formally unramified schemes have decidable equality}

\begin{proposition}\label{closed-notnot-stable-decidable}% TODO: put this in context, maybe deduce the above from this
	Let $P$ be a closed, $\neg \neg$-stable proposition. Then $P$ is decidable.
\end{proposition}
\begin{proof}
	Let $P$ be the proposition $I = 0$ where $I$ is a finitely generated ideal of $R$.
	We claim $(I^2 = 0) \to (I = 0)$. Indeed, if $I^2 = 0$, then no element of $I$
	can be invertible, so $I$ is not not zero, and since $P$ is $\neg \neg$-stable,
	$I = 0$. Hence $I$ is
	generated by a single idempotent $e$ of $R$.
	Since $R$ is local, $e$ is either $0$ or $1$. Since $P$ is equivalent to $e = 0$,
	$P$ is decidable.
\end{proof}

\begin{proposition}
Any separated unramified scheme has decidable equality.
\end{proposition}

\begin{proof}
By \cref{characterisation-unramified-schemes} its identity types are $\not\not$-stable. We conclude by \cref{closed-notnot-stable-decidable}.
\end{proof}

\subsection{Unramified maps between schemes}

\begin{lemma}
\label{kernel-is-tangent-of-fibers}
For any map $f:X\to Y$ and $x:X$, we have that:
\[
\mathrm{Ker}(df_x) = T_{(x,\refl_{f(x)})}(\mathrm{fib}_f(f(x)))
\]
\end{lemma}
\begin{proof}
This holds because:
\[
(\mathrm{fib}_f(f(x)),(x,\refl_{f(x)}))
\]
is the pullback of:
\[
(X,x) \to (Y,f(y)) \leftarrow (1,*)
\]
in pointed types, applied using $(\mathbb{D}(1),0)$.
\end{proof}

\begin{proposition}
A map between schemes is unramified if and only if its differentials are injective. 
\end{proposition}
\begin{proof}
The map $df_x$ is injective if and only if its kernel is $0$. By \cref{kernel-is-tangent-of-fibers}, this means that $df_x$ is injective for all $x:X$ if and only if:
\[
\prod_{x:X}T_{(x,\refl_{f(x)})}(\mathrm{fib}_f(f(x)))=0
\]
On the other hand having fibers with trivial tangent space is equivalent to:
\[
\prod_{y:Y}\prod_{x:X}\prod_{p:f(x)=y} T_{(x,p)}(\mathrm{fib}_f(y)) = 0
\]
Both are equivalent by path elimination on $p$.
\end{proof}
