\subsection{Neighborhoods for sets}

\begin{definition}
Let $X$ be a set with $x:X$. The $n$-order neighborhood $N_n(x)$ is defined as the set of $y:X$ such that the exists a f.g. ideal $I$ such that $I^{n+1}=0$ and:
\[I=0 \to x=y\]
\end{definition}

We write:
\[N_\infty(x) = \cup_{n:\N} N_n(x)\]

\begin{lemma}
For any set $X$ and $x$, we have that:
\[N_\infty(x) \simeq \sum_{y:Y} \mathrm{Et}(x=y)\]
\end{lemma}

\begin{proof}
By \cref{etale-replacement-proposition}.
\end{proof}

\subsection{Recap on modules and infinitesimal disks}

\begin{lemma}\label{tangent-infinitesimal-disks}
For any finitely presented $R$-module $M$, we have a natural iso:
\[T_0\D(M) = M^\star\]
\end{lemma}

\begin{proof}
This is a direct application of \cref{equivalence-module-infinitesimal}.
\end{proof}

\begin{lemma}\label{neighborhood-tangent-correspondence}
A linear map between finitely copresented module:
\[f:M\to N\]
is injective (resp. an iso) iff the corresponding pointed map:
\[\D(M^\star) \to \D(N^\star)\]
is an embedding (resp. an equivalence).
\end{lemma}

\begin{proof}
The case of isomorphisms is a direct consequence of \cref{equivalence-module-infinitesimal}.

Assume $f$ injective, then $f^\star$ is surjective the induced map:
\[R\oplus N^\star \to R\oplus M^\star\]
is surjective as well, giving an embedding of affine schemes. 

Conversely assume the pointed map:
\[\D(M^\star) \to \D(N^\star)\]
is an embedding then the induced map:
\[T_0\D(M^\star) \to T_0\D(N^\star)\]
is a injective as well, but it is equivalent to the map:
\[M\to N\]
by \cref{tangent-infinitesimal-disks}.
\end{proof}

\begin{lemma}\label{neighborhood-tangent-correspondence-smooth}
A linear map between finitely copresented module:
\[f:M\to N\]
is surjective if and only if the corresponding pointed map:
\[\D(M^\star) \to \D(N^\star)\]
merely has a section preserving $0$.
\end{lemma}

\begin{proof}
We know that \cref{equivalence-module-infinitesimal} we know that:
\[\D(M^\star) \to \D(N^\star)\]
merely having a section preserving $0$ is equivalent to:
\[M\to N\]
merely having a section. But since any finitely copresented module is projective, this is equivalent to $f$ being surjective.
\end{proof}

\subsection{Formally unramified maps between sets}

\begin{proposition}\label{unramified-sets-map}
Let $f:X\to Y$ be a map between sets. The following are equivalent:
\begin{enumerate}[(i)]
\item The map $f$ is formally unramified.
\item For all $x:X$, the induced map:
\[N_\infty(f) : N_\infty(x)\to N_\infty(f(x))\]
is an embedding.
\item For all $x:X$, the induced map:
\[N_1(f) : N_1(x)\to N_1(f(x))\]
is an embedding.
\end{enumerate}
\end{proposition}

\begin{proof}
Let us assume (i) to prove (ii). Assume given $x$ with $y,z\in N_\infty(x)$ such that $f(y)=f(z)$, we want to prove $y=z$. But we have $\mathrm{Et}(y=z)$ so there is a closed dense $P$ such that $P\to y=z$, and then $y$ and $z$ are both lifting of the same square: 
 \begin{center}
      \begin{tikzcd}
        P\ar[r]\ar[d] & X\ar[d,"f"] \\
       1 \ar[r]\ar[ru,dashed] & Y
      \end{tikzcd}
    \end{center}
so they are equal by (i). 

Let us assume (ii) to prove (iii). Assume given $x:X$ and $y,z\in N_1(x)$ such that $f(y)=f(z)$. Then $y,z\in N_\infty(x)$ so by (ii) they are equal.

Let us assume (iii) to prove (i). Assume given two lifts $x,y:X$ to a square:
 \begin{center}
      \begin{tikzcd}
        \epsilon=0\ar[r]\ar[d] & X\ar[d,"f"] \\
       1 \ar[r]\ar[ru,dashed] & Y
      \end{tikzcd}
    \end{center}
    with $\epsilon^2=0$. Then $\epsilon=0 \to x=y$ so that $x\in N_1(y)$ and since $f(x)=f(y)$, we conclude by (iii) that we have $x=y$.
\end{proof}

\begin{corollary}\label{unramified-map-between-scheme-alt}
Let $f:X\to Y$ be a map between schemes. Then the following are equivalent:
\begin{enumerate}[(i)]
\item The map $f$ is unramified. 
\item For all $x:X$, the induced map:
\[df : T_x(X)\to T_{f(x)}(Y)\]
is injective.
\end{enumerate}
\end{corollary}

\begin{proof}
By \cref{unramified-sets-map} with \cref{neighborhood-tangent-correspondence}
\end{proof}

\subsection{Formally smooth maps between sets}

Here we do not find an equivalence, but just an implication. This might be possible to correct.

\begin{proposition}\label{smooth-sets-map}
Let $f:X\to Y$ be a map between sets. Assume that $X$ is formally smooth and that for all $x:X$, the induced map:
\[N_1(f) : N_1(x)\to N_1(f(x))\]
merely has a section sending $f(x)$ to $x$. Then $f$ is formally smooth.
\end{proposition}

\begin{proof}
Assume given $\epsilon:R$ such that $\epsilon^2=0$ and try to merely find a lift in:
 \begin{center}
      \begin{tikzcd}
        \epsilon=0\ar[r,"\phi"]\ar[d] & X\ar[d,"f"] \\
       1 \ar[r,"y"]\ar[ru,dashed] & Y
      \end{tikzcd}
    \end{center}
    Since $X$ is formally smooth we merely have an $x:X$ such that:
\[\prod_{p:\epsilon=0} \phi(p)=x\]
and therefore:
\[ \epsilon=0 \to y=f(x)\]
This means that we can factor the square:
 \begin{center}
      \begin{tikzcd}
        \epsilon=0\ar[r,"\phi"]\ar[d] & N_1(x) \ar[d]\ar[r]& X\ar[d,"f"] \\
       1 \ar[r,"y"]\ar[ru,dashed] & N_1(f(x)) \ar[r]&Y
      \end{tikzcd}
    \end{center}
    where we can find a lift because the middle arrow has a section sending $f(x)$ to $x$.
\end{proof}

Note that it is immediate from the definition of smoothness that smooth maps induce surjections on tangent spaces. We have a converse when the domain is smooth.

\begin{corollary}\label{smooth-schemes-iff-submersion}
Let $f:X\to Y$ be a map between schemes with $X$ smooth. Then the following are equivalent:
\begin{enumerate}[(i)] 
\item The map $f$ is smooth.
\item For all $x:X$, the induced map:
\[df : T_x(X)\to T_{f(x)}(Y)\]
is surjective.
\end{enumerate}
\end{corollary}

\begin{proof}
It is straightforward to prove that (i) implies (ii), even without any assumption on $X$ and $Y$, by considering the diagram:
 \begin{center}
      \begin{tikzcd}
        1\ar[r]\ar[d] & X\ar[d,"f"] \\
       \mathbb{D}(1) \ar[r] \ar[r]&Y
      \end{tikzcd}
    \end{center}
To prove that (ii) implies (i) we use \cref{smooth-sets-map} with \cref{neighborhood-tangent-correspondence-smooth}.
\end{proof}

\subsection{Formally étale maps between sets}

\begin{proposition}\label{etale-sets-map}
Let $f:X\to Y$ be a map between sets. Assume that $X$ is formally smooth. Then the following are equivalent:
\begin{enumerate}[(i)]
\item The map $f$ is formally étale. 
\item For all $x:X$, the induced map:
\[N_\infty(f) : N_\infty(x)\to N_\infty(f(x))\]
is an equivalence.
\item For all $x:X$, the induced map:
\[N_1(f) : N_1(x)\to N_1(f(x))\]
is an equivalence.
\end{enumerate}
\end{proposition}

\begin{proof}
Assume (i) to prove (ii). This is a general propriety of lex modalities that modal maps induces equivalence of the modal disks.

Assume (ii) to prove (iii). For all $x:X$, the inverse to the map: 
\[N_\infty(f) : N_\infty(x)\to N_\infty(f(x))\]
has to preserve first-order neighbourhood so it induces an equivalence as in (iii).

Finally we assume (iii) to prove (i). By \cref{unramified-sets-map} we already know that $f$ is unramified. We see that it is smooth by using \cref{smooth-sets-map}.
\end{proof}

\begin{corollary}
Let $f:X\to Y$ be a map between schemes. Assume $X$ is smooth. Then the following are equivalent:
\begin{enumerate}[(i)]
\item The map $f$ is étale. 
\item For all $x:X$, the induced map:
\[df : T_x(X)\to T_{f(x)}(Y)\]
is an iso.
\end{enumerate}
\end{corollary}

\begin{proof}
By \cref{etale-sets-map}  with \cref{neighborhood-tangent-correspondence}.
\end{proof}
