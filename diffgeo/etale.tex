
\subsection{Lifting Properties}

\begin{example}
  Let us assume $2\neq 0$ in $R$ and look at the following square:
  \begin{center}
    \begin{tikzcd}
      1\ar[r]\ar[d] & \Spec R[X,X^{-1}][Y]/(Y^2)\ar[d,"\pi"] \\
      \D(1)\ar[r] & \Spec R[X,X^{-1}]
    \end{tikzcd}
  \end{center}
  As the bottom map, we choose the inclusion of $\{x:R^\times \mid (x-1)^2=0 \}$.
  Then, for any choice of the top map, there is a unique lift in this square:
  \begin{center}
    \begin{tikzcd}
      1\ar[r]\ar[d] & \Spec R[X,X^{-1}][Y]/(Y^2)\ar[d,"\pi"] \\
      \D(1)\ar[r]\ar[ru,dashed] & \Spec R[X,X^{-1}]
    \end{tikzcd}
  \end{center}
\end{example}

A non-fintely generated version of the following definition is usually used to define \emph{formally étale} maps of schemes and then it is subsequently noted,
that formally étale maps of finite presentation are étale.
Since all of our schemes are of finite presentations, this should be a correct definition of étale morphism.

\begin{definition}
  \begin{enumerate}[(a)]
  \item   A map $f:X\to Y$ is \notion{formally étale},
    if for all finitely presented $R$-algebras $A$ and all finitely generated nilpotent ideal $N\subseteq A$,
    and all squares like below, there is a unique lift:
    \begin{center}
      \begin{tikzcd}
        \Spec (A/N)\ar[r]\ar[d,"\iota",swap] & X\ar[d,"f"] \\
        \Spec A \ar[r]\ar[ru,dashed] & Y
      \end{tikzcd}
    \end{center}
    -- where $\iota:\Spec (A/N)\to \Spec A$ is induced by the quotient map.
  \item A map $f:X\to Y$ of schemes is \notion{étale}, if it is formally étale.
  \end{enumerate}
\end{definition}

\begin{lemma}%
  A map $f:X\to Y$ is formally étale,
  if and only if it has unique lifts against left maps of the form $\Spec A/(a)\to \Spec A$,
  where $a:A$ with $a^2=0$.
\end{lemma}

\begin{lemma}%
  \label{nilpotent-ideal-not-not-dense}
  Let $A$ be a finitely presented $R$-algebra and $N\subseteq A$ be finitely generated nilpotent.
  Then for $V\colonequiv \Spec (A/N)\subseteq \Spec A$ the following holds:
  \begin{enumerate}[(a)]
  \item For all $x:\Spec A$, $\neg\neg V(x)$.
  \item If $V=\emptyset$, then $\Spec A=\emptyset$.
  \end{enumerate}
\end{lemma}

\begin{proof}
  The generators of $n_1,\dots,n_l$ of $N$ are nilpotent functions.
  \begin{enumerate}[(a)]
  \item We have to show a $\neg\neg$-stable proposition,
    so we can assume $n_1(x)=0,\dots,n_l(x)=0$, but then any $x:\Spec A$ is an element of $V=V(n_1,\dots,v_l)$.
  \item Assume $V=\emptyset$ and $x:\Spec A$.
        We want to show the $\neg\neg$-stable proposition $\emptyset$,
        so we can assume $V(x)$, which is a contradiction.
  \end{enumerate}
\end{proof}

\begin{proposition}%
  Let $P$ be a $\neg\neg$-stable proposition,
  then $P\to 1$ is formally étale.
\end{proposition}

\begin{proof}
  Direct application of \cref{nilpotent-ideal-not-not-dense}.
\end{proof}

\begin{proposition}%
  The map $\Bool\to 1$ is étale.
\end{proposition}

\begin{proof}
  We have to extend maps $f:\Spec (A/(a))\to \Bool$, with $a^2=0$.
  Since $\Bool\subseteq R$, the map $f$ yields an element $f:A/(a)$
  and we have a lift $\tilde{f}:A$ with $f=\tilde{f}+ab$.
  By \cref{nilpotent-ideal-not-not-dense},
  we have for any $x:\Spec A$, that $\neg\neg(\tilde{f}(x)=0)$ or $\neg\neg(\tilde{f}(x)=1)$.

  By Z-choice or computation, we find a $n:\N$,
  such that $\tilde{f}^n(x)=0$ or $\neg\neg(\tilde{f}^n(x)=1)$.
  With the map $1-\_:R\to R$, we can achieve the same for $1$.
\end{proof}

\begin{proposition}%
	Let $X$ be an affine scheme and suppose $X \to 1$ is étale. Then $X$ has decidable equality.
\end{proposition}

\begin{proof}
	Let $X = \Spec R[X_1,\ldots,X_n]/(f_1,\ldots,f_l)$, and suppose $a : X$.
	Since $X$ is étale, the tangent space $T_a X$ has a unique point.
	This tangent space has an explicit description as the spectrum of 
	$R[U_1,\ldots,U_n]/(J_f U^T)$. Since the tangent space has a unique point,
	this ring is $R$. This means for each $i = 1,\ldots,n$, we can find
	$g_i : (f_1,\ldots,f_l)$ with $\partial_{X_j} g_i(a) = \delta_{ij}$.
	This means we can factor $g_i = (X_i-a_i)q_i$ with $q_i(a) = 1$.
	Now for $b : X$ we claim that either $a = b$ or $a \ne b$.
	For each $i$, we have $q_i(b) + (q_i(a) - q_i(b)) = 1$, so either
	$q_i(b)$ is invertible or $q_i(a) - q_i(b)$ is invertible.
	In the first case, $b_i = a_i$ since $g_i(b) = 0$.
	In the second case, $a \ne b$. 
	By finite choice, we may suppose for each $i$ we purely know one of the alternatives.
	Thus we either know $b_i = a_i$ for all $i$, so that $a = b$, or that $a \ne b$.
\end{proof}
