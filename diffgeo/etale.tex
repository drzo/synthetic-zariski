\subsection{Lifting Properties}

\begin{lemma}%
  If a map $f$ has the left lifting property with respect to $l:A_0\to A_1$ and $l':A_1\to A_2$,
  then $f$ has the left lifting property with respect to $l'\circ l$.
\end{lemma}

\begin{proof}
  TODO
\end{proof}

\begin{lemma}
\label{lifts-equivalence-pointwise}
Assume given a commutative square of the form:
 \begin{center}
    \begin{tikzcd}
      A \ar[r,"g"]\ar[d,swap,"u"]& X\ar[d,"p"]\\
      B \ar[r,swap,"f"]& Y
    \end{tikzcd}
  \end{center}
Then the following are equivalent:
\begin{enumerate}[(i)]
\item A lift of the square.
\item For all $b:B$ a lift of:
 \begin{center}
    \begin{tikzcd}
      \mathrm{fib}_u(b)\ar[d]\ar[r,"g"] & X\ar[d,"p"]\\
      1 \ar[r,swap,"f(b)"]& Y
    \end{tikzcd}
  \end{center}
 \item For all $b:B$ a lift of:
  \begin{center}
    \begin{tikzcd}
      \mathrm{fib}_u(b)\ar[r,"g"]\ar[d] & \mathrm{fib}_p(f(b))\ar[d]\\
      1 \ar[r] & 1
    \end{tikzcd}
  \end{center}
\end{enumerate}
\end{lemma}

\begin{proof}
We have that (ii) is equivalent to (iii) by definition of the fiber.

We can assume that the square is of the form:
 \begin{center}
    \begin{tikzcd}
   \sum_{b:B}P_b\ar[d]\ar[r] & \sum_{y:Y} Q_y\ar[d]\\
     B \ar[r,swap,"f"]& Y
    \end{tikzcd}
  \end{center}
  where the top map is:
  \[
  \lambda (b,p). (f(b),g(b,p))
  \]
  for some $g:\prod_{b:B} P_b \to Q_{f(b)}$. A lift of this square is the same as an inhabitant of:
 % \[
 % h : B \to \sum_{y:Y}Q_y
 % \]
  %such that $\pi_Y\circ h = f$ and for all $b:B$ and $p:P(b)$ we have $(f(b),g(b,p)) = h(b)$. But this is equivalent to a giving:
  %\[
  %h: \prod_{b:B}Q_{f(b)}
  %\]
  % such that for all $b:B$ and $p:P(b)$ we have $g(b,p) = h(b)$. This in turn is the same as an inhabitant of:
  \[
  \prod_{b:B} \sum_{q:Q_{f(b)}} \prod_{p:P_b} g(b,p)=q
  \]
  which is equivalent to (iii).
\end{proof}

\begin{lemma}
\label{pointwise-lift-is-enough}
Assume given maps $u:A\to B$ and $p:X\to Y$.
Then if $p$ has the right lifting property (resp. at most one lift) against $\mathrm{fib}_u(b)\to 1$ for all $b:B$, then it has the right lifting property (resp. at most one lift) against $u$.
\end{lemma}

\begin{proof}
By \cref{lifts-equivalence-pointwise} a lift of a square
 \begin{center}
    \begin{tikzcd}
      A \ar[r]\ar[d,swap,"u"]& X\ar[d,"p"]\\
      B \ar[r]& Y
    \end{tikzcd}
  \end{center}
  is equivalent a family of lifts of squares of the form: 
   \begin{center}
    \begin{tikzcd}
      \mathrm{fib}_u(b)\ar[d]\ar[r] & X\ar[d,"p"]\\
      1 \ar[r]& Y
    \end{tikzcd}
  \end{center}
  But a product of contractible types (resp. propositions) is itself contractible (resp. a proposition) so we can conclude. 
\end{proof}

\begin{lemma}
\label{lifting-defined-fiberwise}
A map $p$ has the right lifting property (resp. at most one lift) against a map $P\to 1$ if and only all the fibers of $p$ have this property against $P\to 1$.
\end{lemma}

\begin{proof}
This is an immediate consequence of the universal property of the fibers.
\end{proof}

\subsection{\'Etale maps}

\begin{example}
  Let us assume $2\neq 0$ in $R$ and look at the following square:
  \begin{center}
    \begin{tikzcd}
      1\ar[r]\ar[d] & \A^1\setminus\{0\}\ar[d,"x\mapsto x^2"] \\
      \D(1)\ar[r] & \A^1\setminus\{0\}
    \end{tikzcd}
  \end{center}
  As the bottom map, we choose the inclusion of $\{x:R^\times \mid (x-1)^2=0 \}$.
  Then, for any choice of the top map, there is a unique lift in this square:
  \begin{center}
    \begin{tikzcd}
      1\ar[r]\ar[d] & \A^1\setminus\{0\}\ar[d,"x\mapsto x^2"] \\
      \D(1)\ar[r]\ar[ru,dashed] & \A^1\setminus\{0\}
    \end{tikzcd}
  \end{center}
\end{example}

A non-finitely generated version of the following definition is usually used
to define \emph{formally étale} maps of schemes
\footnote{In \cite{EGAIV3}[§17], the definition of formally étale maps ranges over arbitrary ideals, but uses the same lifting condition as below.}
and then it is subsequently noted,
that formally étale maps of finite presentation are étale.
Since all of our schemes are of finite presentations, this should be a correct definition of étale morphism:

\begin{definition}
  \begin{enumerate}[(a)]
  \item   A map $f:X\to Y$ is \notion{formally étale},
    if for all finitely presented $R$-algebras $A$ and all finitely generated nilpotent ideals $N\subseteq A$,
    and all squares like below, there is a unique lift:
    \begin{center}
      \begin{tikzcd}
        \Spec (A/N)\ar[r]\ar[d,"\iota",swap] & X\ar[d,"f"] \\
        \Spec A \ar[r]\ar[ru,dashed] & Y
      \end{tikzcd}
    \end{center}
    -- where $\iota:\Spec (A/N)\to \Spec A$ is induced by the quotient map.
  \item A map $f:X\to Y$ of schemes is \notion{étale}, if it is formally étale.
  \end{enumerate}
\end{definition}

Another way to phrase our definition of formally étale maps would be to say,
that they are the maps with the right lifting property (\cite{modalities}[definition 1.45]) with respect to ``left`` maps of the form $\Spec (A/N)\to \Spec A$.
We will use some well known, general closure properties of left maps, starting with closure under composition:

\begin{lemma}%
\label{decomposition-nilpotent}
For $A$ an algebra and $N$ a nilpotent ideal in $A$, the map:
\[A
\to A/N
\] 
can be factored as maps of the form:
\[
B\to B/(b)
\]
where $b:B$ is such that $b^2=0$.
\end{lemma}

\begin{proof}
TODO
\end{proof}

\begin{lemma}
\label{equivalence-etale}
Having the right lifting property against the following class of maps is equivalent:
\begin{enumerate}[(i)]
\item Maps of the form $\Spec(A/N)\to \Spec(A)$ for $A$ f.g. algebra and $N$ a nilpotent ideal.
\item Maps of the form $P\to 1$ for $P$ a closed dense proposition.
\item Closed dense embeddings of types.
\item Maps of the form $(\epsilon=0)\to 1$ for $\epsilon:R$ such that $\epsilon^2=0$.
\item Maps of the form $\Spec(A/(a)) \to \Spec(A)$ for $A$ f.g. algebra and $a:A$ such that $a^2=0$.
\end{enumerate}
\end{lemma}

\begin{proof}
(i) implies (ii) because closed dense proposition are of the form $\Spec(R/N)$ for $N$ a nilpotent ideal.

(ii) implies (iii) because of \cref{pointwise-lift-is-enough}.

(iii) implies (iv) because $\epsilon=0$ is closed dense when $\epsilon$ is nilpotent.

(iv) implies (v) because of \cref{pointwise-lift-is-enough}, as the fiber of $\Spec(A/(a)) \to \Spec(A)$ over $x$ is $a(x)=0$.

(v) implies (i) because of \cref{decomposition-nilpotent}.
\end{proof}

\begin{lemma}
A map is formally étale if and only if all its fibers are formally étale.
\end{lemma}

\begin{proof}
By \cref{lifting-defined-fiberwise} and the characterisation (ii) from the previous lemma.
\end{proof}

\begin{lemma}%
  \label{nilpotent-ideal-not-not-dense}
  Let $A$ be a finitely presented $R$-algebra and $N\subseteq A$ be finitely generated nilpotent.
  Then for $V\colonequiv \Spec (A/N)\subseteq \Spec A$ the following holds:
  \begin{enumerate}[(a)]
  \item For all $x:\Spec A$, $\neg\neg V(x)$.
  \item If $V=\emptyset$, then $\Spec A=\emptyset$.
  \end{enumerate}
\end{lemma}

\begin{proof}
  \begin{enumerate}[(a)]
  \item
    Let $x : \Spec A$ be given.
    The generators $n_1,\dots,n_l$ of $N$ are nilpotent functions,
    so in particular the elements $n_1(x), \dots, n_l(x)$ of $R$
    are not not zero.
    This means precisely $\neg\neg V(x)$.
  \item Assume $V=\emptyset$ and $x:\Spec A$.
        We want to show the $\neg\neg$-stable proposition $\emptyset$,
        so we can assume $V(x)$, which is a contradiction.
  \end{enumerate}
\end{proof}

\begin{proposition}%
  Let $P$ be a $\neg\neg$-stable proposition,
  then $P\to 1$ is formally étale.
\end{proposition}

\begin{proof}
  Direct application of \cref{nilpotent-ideal-not-not-dense}.
\end{proof}

\begin{proposition}%
  The map $\Bool\to 1$ is étale.
\end{proposition}

\begin{proof}
  We have to extend maps $f:\Spec (A/(a))\to \Bool$, with $a^2=0$.
  Since $\Bool\subseteq R$, the map $f$ yields an element $f:A/(a)$
  and we have a lift $\tilde{f}:A$ with $f=\tilde{f}+ab$.
  By \cref{nilpotent-ideal-not-not-dense},
  we have for any $x:\Spec A$, that $\neg\neg(\tilde{f}(x)=0)$ or $\neg\neg(\tilde{f}(x)=1)$.

  By Z-choice or computation, we find a $n:\N$,
  such that $\tilde{f}^n(x)=0$ or $\neg\neg(\tilde{f}^n(x)=1)$.
  With the map $1-\_:R\to R$, we can achieve the same for $1$.
\end{proof}

\begin{proposition}%
	\label{decidable-of-tangent}
	Let $X$ be an affine scheme and $a : X$ a point. Suppose the tangent space
	$T_a X$ has a unique point. Then for any $b : X$, equality $a = b$ is decidable.
\end{proposition}

\begin{proof}
	Given that $T_a X = \{0\}$, we also have $T^\star_a X = \mm_a / \mm_a^2 = 0$
	by \cref{maximal-cotangent}.
	So $\mm_a^2 = \mm_a$. By \cite[Lemma II.4.6]{lombardi-quitte}
	(proved using Nakayama's lemms, or the determinant trick), $\mm_a$ is generated
	by a single idempotent $e$ of $A$. We have $e(b)$ idempotent in $R$,
	so since $R$ is local it is either 0 or 1. We have $e(a) = 0$, so if $e(b) = 1$
	we have $a \ne b$. If $e(b) = 0$, then $a = b$ since $b$ is in the order 0
	neighborhood of $a$.
\end{proof}

\begin{proposition}%
	Let $X$ be a separated \'{e}tale scheme. Then $X$ has decidable equality.
\end{proposition}

\begin{proof}
	Let $a, b : X$. Since $X$ is separated, $a = b$ is closed.
	We claim it is also open.
	To see this, pick an affine open neighbourhood $U$ of $a$.
	Now $a = b$ is equivalent to $(b \in U) \wedge (a = b)$.
	Since open propositions are closed under $\Sigma$,
	it suffices to show that $a = b$ is open assuming $b \in U$.
	In this case we can apply \cref{decidable-of-tangent}.
\end{proof}

\begin{proposition}% TODO: put this in context, maybe deduce the above from this
	Let $P$ be a closed, $\neg \neg$-stable proposition. Then $P$ is decidable.
\end{proposition}
\begin{proof}
	Let $P$ be the proposition $I = 0$ where $I$ is a finitely generated ideal of $R$.
	We claim $(I^2 = 0) \to (I = 0)$. Indeed, if $I^2 = 0$, then no element of $I$
	can be invertible, so $I$ is not not zero, and since $P$ is $\neg \neg$-stable,
	$I = 0$. Hence $I$ is
	generated by a single idempotent $e$ of $R$.
	Since $R$ is local, $e$ is either $0$ or $1$. Since $P$ is equivalent to $e = 0$,
	$P$ is decidable.
\end{proof}

\subsection{Unramified maps}

\begin{definition}
For any types $X,Y$, a map $f:X\to Y$ is called formally unramified if for any closed proposition $P$ such that $\neg\neg P$ the following square has at most one lift:
 \begin{center}
      \begin{tikzcd}
        P\ar[r]\ar[d] & X\ar[d,"f"] \\
       1 \ar[r]\ar[ru,dashed] & Y
      \end{tikzcd}
    \end{center}
\end{definition}

\begin{definition}
A map $f:X\to Y$ is unramified if $X$ and $Y$ are schemes and $f$ is formally unramified.
\end{definition}

It is clear that formally étale maps are formally unramified. As usual a type $X$ is called formally unramified if the map from $X$ to $1$ is unramified, and a map is formally unramified if and only if all its fibers are formally unramified.

\begin{lemma}
\label{equivalence-unramified}
Having at most one lift against the following class of maps is equivalent:
\begin{enumerate}[(i)]
\item Maps of the form $\Spec(A/N)\to \Spec(A)$ for $A$ f.g. algebra and $N$ a nilpotent ideal.
\item Maps of the form $P\to 1$ for $P$ a closed dense proposition.
\item Closed dense embeddings of types.
\item Maps of the form $(\epsilon=0)\to 1$ for $\epsilon:R$ such that $\epsilon^2=0$.
\item Maps of the form $\Spec(A/(a)) \to \Spec(A)$ for $A$ f.g. algebra and $a:A$ such that $a^2=0$.
\end{enumerate}
\end{lemma}

\begin{proof}
Same as \cref{equivalence-etale}.
\end{proof}

\begin{lemma}
A map is formally unramified if and only if all its fibers are formally unramified.
\end{lemma}

\begin{proof}
By \cref{lifting-defined-fiberwise} and the characterisation (ii) from the previous lemma.
\end{proof}

\begin{lemma}
A type $X$ is formally unramified if and only if for any $x,y:X$ and any dense closed proposition $P$ we have:
\[
(P\to x=y)\to x=y
\]
\end{lemma}

\begin{proposition}
A scheme $X$ is unramified if and only if any of the following propositions hold:
  \begin{enumerate}[(i)]
  \item For all $x,y:X$, the proposition $x=y$ is open.
  \item For all $x,y:X$, the proposition $x=y$ is $\neg\neg$-stable.
  \item For all $x:X$, we have $T_x(X)=0$.
  \item For all infinitesimal pointed type $(D,*)$ (meaning that for all $x:D$ we have $\neg\neg(x=*)$), any map from $D$ to $X$ is constant.
  \end{enumerate}
\end{proposition}

\begin{proof}
First we prove that the four propositions are equivalent:

(i) implies (ii) because open propositions are $\neg\neg$-stable.

(ii) implies (iv) because for any $f:D\to X$ and $x:D$ we have $\neg\neg(x=*)$ so that $\neg\neg(f(x)=f(*))$ and finally $f(x)=f(*)$.

(iv) implies (iii) by taking $D=\mathbb{D}(1)$.

(iii) implies (i) because for any $x:X$ there an open affine $U$ such that $x\in U$. Then $x=y$ is equivalent to $(y\in U)\land x=_U y$, but $x=_Uy$ is decidable by \cref{decidable-of-tangent} and open propositions are stable by $\Sigma$.

Now we check they are equivalent to being unramified:

(ii) implies unramified, indeed we need to check that for $x,y:X$ and $P$ closed dense such that $P\to x=y$, we have $x=y$. But $\neg\neg P$ so that $\neg\neg(x=y)$, and by (ii) we have $x=y$.

Unramified implies (iii) because it implies having at most one lifting against any closed dense subtype, so that by considering $1\subset \mathbb{D}(1)$ it implies having at most one tangent vector.
\end{proof}

\begin{example}
Any proposition is formally unramified, so any embedding of types is formally unramified.

Finite sets are unramified, as their identity types are decidable and therefore $\neg\neg$-stable.

The affine line $\A^1$ is not unramified, as its identity types are not $\neg\neg$-stable. Same for $\mathbb{D}(1)$.
\end{example}

\begin{lemma}
\label{kernel-is-tangent-of-fibers}
For any map $f:X\to Y$ and $x:X$, we have that:
\[
\mathrm{Ker}(df_x) = T_{(x,\refl_{f(x)})}(\mathrm{fib}_f(f(x)))
\]
\end{lemma}
\begin{proof}
This holds because:
\[
(\mathrm{fib}_f(f(x)),(x,\refl_{f(x)}))
\]
is the pullback of:
\[
(X,x) \to (Y,f(y)) \leftarrow (1,*)
\]
in pointed types, applied using $(\mathbb{D}(1),0)$.
\end{proof}

\begin{proposition}
A map between schemes is unramified if and only if its differentials are injective. 
\end{proposition}
\begin{proof}
The map $df_x$ is injective if and only if its kernel is $0$. By \cref{kernel-is-tangent-of-fibers}, this means that $df_x$ is injective for all $x:X$ if and only if:
\[
\prod_{x:X}T_{(x,\refl_{f(x)})}(\mathrm{fib}_f(f(x)))=0
\]
On the other hand having fibers with trivial tangent space is equivalent to:
\[
\prod_{y:Y}\prod_{x:X}\prod_{p:f(x)=y} T_{(x,p)}(\mathrm{fib}_f(y)) = 0
\]
Both are equivalent by path elimination on $p$.
\end{proof}

\subsection{Smooth morphisms}

\begin{definition}
A morphism $f:X\to Y$ is smooth if for any f.p. $R$-algebra $A$ and f.g. nilpotent ideal $N$ in $A$, any commutative square:
 \begin{center}
      \begin{tikzcd}
        \Spec(A/N)\ar[r]\ar[d] & X\ar[d,"f"] \\
       \Spec(A) \ar[r]\ar[ru,dashed] & Y
      \end{tikzcd}
    \end{center}
merely has a lift.
\end{definition}

Note that here since we do not have: 
\[
\prod_{x:A} ||B(x)|| \to ||\prod_{x:A}B(x)||
\]
we do not seem to have an analogue to \cref{equivalence-etale} or \cref{equivalence-unramified}. Similarly it does not seem that maps with smooth fibers are always smooth, although I did not find a counter-example yet.

\begin{proposition}
Smooth unramified maps are étale.
\end{proposition}

\begin{lemma}
A proposition is smooth if and only if it is étale. %Any $\neg\neg$-stable proposition is smooth. In particular open propositions are smooth. % are smooth.
\end{lemma}

\begin{proof}
All propositions are unramified.
\end{proof}

\begin{lemma}
The scheme $\A^n$ is smooth for any $n$.
\end{lemma}

\begin{proof}
We need to prove that there merely exists a dotted lift in any:
 \begin{center}
      \begin{tikzcd}
        A/N & R[X_1,\cdots,X_n]\ar[l]\ar[ld,dashed] \\
        A \ar[u]& 
      \end{tikzcd}
    \end{center}
    It is enough to choose a lift for each $X_i$.
\end{proof}

What about $\mathbb{P}^n$?

\begin{lemma}
The scheme $\D(1)$ is not smooth.
\end{lemma}

\begin{proof}
If it were smooth we would be able to merely find a dotted lift in:
 \begin{center}
      \begin{tikzcd}
        R[X]/(X^2) & R[X]/(X^2)\ar[l]\ar[ld,dashed] \\
        R[X]/(X^3) \ar[u]& 
      \end{tikzcd}
    \end{center}
\end{proof}

\begin{lemma}
The scheme $\Spec(R[X,Y]/(XY))$ is not smooth.
\end{lemma}

\begin{proof}
If it were smooth we could merely find a dotted lift in:
 \begin{center}
      \begin{tikzcd}
        R[X]/(X^2) & R[X,Y]/(XY)\ar[l]\ar[ld,dashed] \\
        R[X]/(X^3)\ar[u] & 
      \end{tikzcd}
    \end{center}
    where the top map send both $X$ and $Y$ to $X$. This can't be done.
\end{proof}

\begin{lemma}
The map:
\[
p:\Spec(R[X,Y]/(XY))\to \A^1
\] 
corresponding to the map in:
\[
R[X,Y]/(XY) \r R[X]
\]
sending $X$ to $X$ and $Y$ to $0$ is not smooth.
\end{lemma}

\begin{proof}
If it were smooth all its fibers would be smooth, i.e. for all $z:R$ the scheme $\Spec(R[X]/(zX))$ would be smooth. This would imply that for any $z:R$ such that $z^2=0$ we merely have a dotted lift to:
 \begin{center}
      \begin{tikzcd}
        R/z & R[X]/(X^2)\ar[l]\ar[ld,dashed] \\
        R \ar[u]& 
      \end{tikzcd}
    \end{center} 
    where the top map send $X$ to $1$. Such a lift is an $x:R$ such that $x=1$ modulo $z$, and $zx=0$, but then $x=(1+bz)$ and $0=zx=z+bz^2=z$. So we would get that $z^2=0$ implies $z=0$ for all $z:R$, which is not possible.
\end{proof}

I think in the traditional setting this map has smooth fibers, but not here.

