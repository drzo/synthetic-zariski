
\subsection{Smooth maps}

\begin{definition}
A morphism $f:X\to Y$ is formally smooth for any closed dense proposition $P$ the square:
 \begin{center}
      \begin{tikzcd}
        P\ar[r]\ar[d] & X\ar[d,"f"] \\
       1 \ar[r]\ar[ru,dashed] & Y
      \end{tikzcd}
    \end{center}
merely has a lift.
\end{definition}

\begin{lemma}
For any morphism  $f:X\to Y$ the following are equivalent:
\begin{enumerate}[(i)]
\item The map $f$ is formally smooth. 
\item For any $\epsilon:R$ such that $\epsilon^2=0$, there merely exists a dotted lift to any square:
 \begin{center}
      \begin{tikzcd}
        \epsilon=0\ar[r]\ar[d] & X\ar[d,"f"] \\
       1 \ar[r]\ar[ru,dashed] & Y
      \end{tikzcd}
    \end{center}
\end{enumerate}
\end{lemma}

\begin{proof}
It is clear that (i) implies (ii). Conversely any map $P\to 1$ for $P$ dense closed can be decomposed as:
\[P_n \to P_{n-1} \to P_1\to 1\]
where:
\begin{itemize}
\item For all $k$ we have that $P_k$ is the spectrum of a local ring so it has choice.
\item For all $k$ the map:
\[P_{k}\to P_{k-1}\]
is of the form:
\[\Spec(A/a)\to \Spec(A)\]
where $a^2=0$.
\end{itemize}
Then by (ii) we can merely find a lift pointwise to $P_k\to P_{k-1}$, and since $P_{k-1}$ has choice we merely get a global lift. By iterating we merely get a lift to $P\to 1$.
\end{proof}

\begin{remark}
The usual definition for formally smooth is to ask for lifting in:
 \begin{center}
      \begin{tikzcd}
        \Spec(A/N)\ar[r]\ar[d] & X\ar[d,"f"] \\
       \Spec(A) \ar[r]\ar[ru,dashed] & Y
      \end{tikzcd}
    \end{center}
    with $N$ nilpotent. Note that here since we do not have: 
\[
\prod_{x:A} ||B(x)|| \to ||\prod_{x:A}B(x)||
\]
we do not have a direct analogue to \cref{equivalence-etale} or \cref{equivalence-unramified}.
\end{remark}

Our definition of formal smoothness is convenient because it implies:

\begin{lemma}
A map is formally smooth if and only if its fibers are formally smooth.
\end{lemma}
\begin{proof}
The type of filler for:
 \begin{center}
      \begin{tikzcd}
        P\ar[r]\ar[d] & X\ar[d,"f"] \\
       1 \ar[r,"y"]\ar[ru,dashed] & Y
      \end{tikzcd}
    \end{center}
   is equivalent to the type of filler for:
    \begin{center}
      \begin{tikzcd}
        P\ar[r]\ar[d] & \mathrm{fib}_f(y) \\
       1 \ar[ru,dashed] & 
      \end{tikzcd}
    \end{center}
\end{proof}

\begin{remark}
Formally smooth and formally unramified implies formally étale.
\end{remark}

\subsection{Examples}

We give a few examples and counter-examples:

\begin{lemma}
The scheme $\A^n$ is smooth for any $n$.
\end{lemma}

\begin{proof}
We need to prove that there merely exists a dotted lift in any:
 \begin{center}
      \begin{tikzcd}
        R/N & R[X_1,\cdots,X_n]\ar[l]\ar[ld,dashed] \\
        R \ar[u]& 
      \end{tikzcd}
    \end{center}
    It is enough to choose a lift for each $X_i$.
\end{proof}

\begin{lemma}
A type covered by finitely many formally smooth subtype is formally smooth.
\end{lemma}
\begin{proof}
We assume $P_1,\cdots,P_n:X\to \Prop$ covering $X$, i.e. for all $x:X$ we have:
\[
\prod_{x:X}P_1(x)\lor\cdots\lor P_n(x)
\]
such that $\Sigma_{x:X}P_i(x)$ is formally smooth for all $i$.

Assume given:
   \begin{center}
      \begin{tikzcd}
        P\ar[r,"\phi"]\ar[d] &X \\
       1 \ar[ru,dashed] & 
      \end{tikzcd}
    \end{center}
   then we have:
   \[\prod_{p:P} P_1(\phi(p))\lor\cdots\lor P_n(\phi(p))\]
   so that for some $i$ we have:
      \[\prod_{p:P} P_i(\phi(p))\]
as $P$ is closed. So $\phi$ factors through $\Sigma_{x:X}P_i(x)$ which is formally smooth and we can conclude. %Then we can simply lift in $P_i$, which is assumed    
\end{proof}

\begin{corollary}
The scheme $\mathbb{P}^n$ is smooth for any $n$.
\end{corollary}

\begin{lemma}
The scheme $\D(1)$ is not smooth.
\end{lemma}

\begin{proof}
If it were smooth, for any $\epsilon$ with $\epsilon^3=0$ we would be able to prove $\epsilon^2=0$. Indeed would merely is a dotted lift in:
 \begin{center}
      \begin{tikzcd}
        R/(\epsilon^2)& R[X]/(X^2)\ar[l,"\epsilon"]\ar[ld,dashed] \\
        R \ar[u]& 
      \end{tikzcd}
    \end{center}
    that is a $r:R$ such that $(\epsilon+r\epsilon^2)^2=0$. Then $\epsilon^2=0$.
\end{proof}

\begin{lemma}
The scheme $\Spec(R[X,Y]/(XY))$ is not smooth.
\end{lemma}

\begin{proof}
If it were smooth, for any $\epsilon$ with $\epsilon^3=0$ we would be able to prove $\epsilon^2=0$. Indeed would merely is a dotted lift in:
 \begin{center}
      \begin{tikzcd}
        R/(\epsilon^2) & R[X,Y]/(XY)\ar[l]\ar[ld,dashed] \\
        R\ar[u] & 
      \end{tikzcd}
    \end{center}
    where the top map send both $X$ and $Y$ to $\epsilon$. Then we have $r,r':R$ such that $(\epsilon+r\epsilon^2)(\epsilon+r'\epsilon^2)=0$ so that $\epsilon^2=0$. %This can't be done.
\end{proof}

\begin{lemma}
The map:
\[
p:\Spec(R[X,Y]/(XY))\to \A^1
\] 
corresponding to the map in:
\[
R[X,Y]/(XY) \to R[X]
\]
sending $X$ to $X$ and $Y$ to $0$ is not smooth.
\end{lemma}

\begin{proof}
If it were smooth all its fibers would be smooth, i.e. for all $z:R$ the scheme $\Spec(R[X]/(zX))$ would be smooth. This would imply that for any $\epsilon:R$ such that $\epsilon^2=0$ we merely have a dotted lift to:
 \begin{center}
      \begin{tikzcd}
        R/\epsilon & R[X]/(\epsilon X)\ar[l]\ar[ld,dashed] \\
        R \ar[u]& 
      \end{tikzcd}
    \end{center} 
    where the top map send $X$ to $1$. Such a lift gives an $r:R$ such that $\epsilon(1+r\epsilon)=0$, so that $\epsilon=0$.
\end{proof}

I think in the traditional setting this map has smooth fibers, but not here. 

\subsection{Stability property}

Now we give stability properties for formally smooth types.

\begin{proposition}
\label{smoothSurjective}
The image of a formally smooth type by any map is formally smooth.
\end{proposition}
\begin{proof}
We assume $X$ formally smooth and $p:X\to Y$ surjective. Then for any $P$ closed dense and diagram:
 \begin{center}
      \begin{tikzcd}
      P \ar[rd,dotted]\ar[d]\ar[r]& Y\\
      1 \ar[r,dotted,swap,"x"]& X\ar[u,swap,"p"]
      \end{tikzcd}
    \end{center} 
    by choice for closed propositions we merely get the dotted diagonal, and since $X$ is formally smooth we get the dotted $x$, and then $p(x)$ gives a lift.
\end{proof}

\begin{lemma}
If $X$ is a type satifying choice and for all $x:X$ we have a formally smooth type $Y_x$, then:
\[\prod_{x:X}Y_x\]
is formally smooth.
\end{lemma}

So for example formally smooth types are stable by finite products. 

\begin{lemma}
If $X$ is a formally smooth type and for all $x:X$ we have a formally smooth type $Y_x$, then:
\[\sum_{x:X}Y_x\]
is formally smooth.
\end{lemma}

Formally smooth types are not stable by identity types (e.g. identity types in $\A^1$ are not smooth, otherwise they would be closed and étale, i.e. decidable).

\subsection{Equivalence with the usual definition for maps between schemes}

Here we prove that our definition coincide with the usual one for maps with scheme fibers.

\begin{lemma}\label{lifting-is-torsor}
Let $X$ be a scheme and $\epsilon:R$ such that $\epsilon^2=0$. Then the type of liftings of:
 \begin{center}
      \begin{tikzcd}
      \epsilon=0 \ar[d]\ar[r,"\phi"]& X\\
      1 & 
      \end{tikzcd}
    \end{center} 
is an $M$-pseudotorsor where:
\[M = \Hom_{R/\epsilon}\big(\prod_{p:\epsilon=0}T^\star_{\phi(p)}(X),(\epsilon)\big)\]
\end{lemma}

\begin{proof}
TODO
\end{proof}

\begin{lemma}\label{M-is-wqc}
The $R$-module $M$ from the previous lemma is wqc.
\end{lemma}

\begin{proof}
For any $p:\epsilon=0$ we have that $T^\star_{\phi(p)}(X)$ is a finitely presented $R$-module, so that:
\[N = \prod_{p:\epsilon=0}T^\star_{\phi(p)}(X)\]
is a finitely presented $R/\epsilon$-module. Assume a presentation:
\[
(R/\epsilon)^m \to (R/\epsilon)^n\to N\to 0
\]
then we have an exact sequence of $R$-modules:
\[
0\to \Hom_{R/\epsilon}(N,(\epsilon)) \to (\epsilon)^n\to (\epsilon)^m
\]
but $(\epsilon)$ is wqc so that $\Hom_{R/\epsilon}(N,(\epsilon))$ is the kernel of a map between wqc $R$-modules and it is wqc.
\end{proof}

\begin{proposition}
Let $p:X\to Y$ be a map which fibers are schemes. Then $p$ merely having lifts against the following classes of maps is equivalent:
\begin{enumerate}[(i)]
\item The maps $\epsilon=0\to 1$ where $\epsilon^2=0$.
\item The maps $P\to 1$ where $P$ is closed dense (i.e. $p$ being formally smooth).
\item The maps $\Spec(A/a)\to \Spec(A)$ where $A$ fp $R$-algebra and $a^2=0$.
\item The maps $\Spec(A/N)\to \Spec(A)$ where $A$ fp $R$-algebra and $N$ fg nilpotent ideal.
\end{enumerate}
\end{proposition}

\begin{proof}
It is enough to prove that (i) implies (iii), as any map in (iv) is a composite of maps in (iii). Assume a diagram:
 \begin{center}
      \begin{tikzcd}
      \Spec(A/a) \ar[d]\ar[r]& X\ar[d,"p"]\\
      \Spec(A) \ar[r] & Y
      \end{tikzcd}
    \end{center} 
    with $a^2=0$, we try to merely find a lift. By \cref{lifting-is-torsor}, we know that the type of lifts over $x:\Spec(A)$ is an $M_x$-pseudotorsor. By hypothesis (i) this is in fact an $M_x$-torsor. Mere existence of a lift for the diagram is then precisely a mere section of the dependent torsor $(x:\Spec(A))\mapsto M_x$, i.e. a proof that it is merely trivial. But $M_x$ is wqc by \cref{M-is-wqc} so that $H^1(\Spec(A),M)=0$ by \cite{draft}[Theorem 8.3.6] and any $M$-torsor is merely trivial, meaning we merely have a lift.
\end{proof}

\subsection{Smooth schemes}

\begin{lemma}
Assume given a smooth affine scheme:
\[X = \Spec(R[X_1,\cdots,X_n]/P_1,\cdots,P_m)\]
such that for all $x:X$ the Jacobian:
\[J(x) : R^n \to R^m\]
is surjective. Then $X$ is smooth.
\end{lemma}

\begin{proof}
TODO
\end{proof}

We want to show some kind of converse. We start with a simple case, when $m=1$. First an auxiliary lemma.

\begin{lemma}\label{order-smooth-add-one}
Assume given:
\[P : R[X_1,\cdots,X_n]\]
such that:
\[\Spec(R[X_1,\cdots,X_n]/P)\] 
is smooth. Then for all $x:R^n$ such that $P(x)=0$ and $k>1$, we have that:
\[N_{k-1}(P) : N_{k-1}(x) \to N_{k-1}(0)\]
being zero implies that:
\[N_{k}(P) : N_{k}(x) \to N_{k}(0)\]
is zero as well.
\end{lemma}

\begin{proof}
Assume given $x:R^n$ such that $P(x)=0$, using a translation we can assume $x=0$. Then:
\[N_{k-1}(P) = 0\]
means that $P=0$ modulo $(X_1,\cdots,X_n)^k$.

Assume given $\epsilon_1,\cdots,\epsilon_n:R$ such that:
\[(\epsilon_1,\cdots,\epsilon_n)^{k+1}=0\]
then we consider the lift in:
 \begin{center}
      \begin{tikzcd}
      R/(\epsilon_1,\cdots,\epsilon_n)^k & R[X_1,\cdots,X_n]/P\ar[l,swap,"X_i\mapsto \epsilon_i"]\ar[ld,dotted]\\
      R\ar[u] &
      \end{tikzcd}
    \end{center} 
    which gives $\delta_1,\cdots,\delta_n: (\epsilon_1,\cdots,\epsilon_n)^k$ such that:
    \[P(\epsilon_1+\delta_1,\cdots,\epsilon_n+\delta_n) = 0\]
    Since $P=0$ modulo $(X_1,\cdots,X_n)^k$, we have:
\[P = \sum_{i_1,\cdots,i_k} c_{i_1,\cdots,i_k} X_{i_1}\cdots X_{i_k} + Q\]
where $Q=0$ modulo $(X_1,\cdots,X_n)^{k+1}$. By computation we get:
   \[\sum_{i_1,\cdots,i_k} c_{i_1,\cdots,i_k} \epsilon_{i_1}\cdots \epsilon_{i_k} = 0\]
   
   So we know that:
   \[(\epsilon_1,\cdots,\epsilon_n)^{k+1}=0\]
   implies:
   \[\sum_{i_1,\cdots,i_k} c_{i_1,\cdots,i_k} \epsilon_{i_1}\cdots \epsilon_{i_k} = 0\]
   so by sqc we can conclude that all $c_{i_1,\cdots,i_k}$ are zeros, so that $P = 0$ modulo $(X_1,\cdots,X_n)^{k+1}$, which indeed means $N_k(P)=0$.
\end{proof}

\begin{lemma}
Assume given $P:R[X_1,\cdots,X_n]$ such that:
\[\Spec(R[X_1,\cdots,X_n]/P)\]
is smooth and $P\not=0$. Then for all $x:R^n$ the jacobian:
\[J(P)(x) : R^n \to R\]
is surjective.
\end{lemma}

\begin{proof}
Since the jacobian is linear and takes value in $R$, it is enough to prove that it is not equal to zero to conclude that it is surjective. But we have an equivalence:
\[(N_1(x)\to N_1(0)) \simeq \mathrm{Hom}_R(T_x(R^n),T_0(R))\]
sending $N_1(P)$ to $J(P)(x)$ so it is enough to prove that $N_1(P)\not=0$. 

Assume $N_1(P)=0$, then by \cref{order-smooth-add-one} we have $N_k(P)=0$ for all $k$ and then $P=0$, which is a contradiction.
\end{proof}

\begin{remark}
Expecting a full converse is unreasonable, see for example:
\[\Spec(R[X]/ (X-a)(X-b),(X-a)(X-c))\]
which is the point whenever $b\not=c$, so that it is smooth, but has its jacobian going from $R$ to $R^2$ so never surjective.
\end{remark}

\subsection{Smooth schemes have free tangent spaces}

TODO

