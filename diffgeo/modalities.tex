\subsection{Generalities on modalities}

In this section we state useful facts about modalities in HoTT. We assume $L$ a left exact modality and $L'$ the modality corresponding to $L$-separated types, i.e. types with $L$-modal identity types. We have in mind $L$ the formally étale modality and $L'$ the formally unramified modality. Precise references should be added.

\begin{lemma}
If $L$ is accessible (i.e. defined by unique litfting conditions), then $L'$ is defined by the corresponding 'at most one' lifting conditions.
\end{lemma}

\begin{lemma}
The modality $L$ preserves $n$-types for any $n$.
\end{lemma}

The following is one of the many equivalent characterisations of left exact modalities.

\begin{lemma}
For any $X$ the localisation $\eta_X:X\to L(X)$ induces $L$-localisations:
\[x=_Xy \to \eta_X(x)=_{LX}\eta_X(y)\]
\end{lemma}

\begin{proposition}
\label{separatedLocalisationIsQuotient}
A map $f:X\to Y$ is an $L'$-localisation if and only if $f$ is surjective and for all $x,y:X$ the map:
\[ap_f : x=_Xy\to f(x)=_Yf(y)\] 
is an $L$-localisation.
\end{proposition}

\begin{corollary}
The modality $L'$ preserve $n$-types for any $n$.
\end{corollary}

\begin{corollary}
If $X$ is $L'$-modal then:
\[X\to LX\]
is an embedding.
\end{corollary}

This means that the successive localisations:
\[X\to L'X\to LX\]
are the image factorisation of the localisation $X\to LX$.

\subsection{Lifting Properties}

\begin{lemma}%
  If a map $f$ has the left lifting property with respect to $l:A_0\to A_1$ and $l':A_1\to A_2$,
  then $f$ has the left lifting property with respect to $l'\circ l$.
\end{lemma}

\begin{lemma}
\label{lifts-equivalence-pointwise}
Assume given a commutative square of the form:
 \begin{center}
    \begin{tikzcd}
      A \ar[r,"g"]\ar[d,swap,"u"]& X\ar[d,"p"]\\
      B \ar[r,swap,"f"]& Y
    \end{tikzcd}
  \end{center}
Then the following are equivalent:
\begin{enumerate}[(i)]
\item A lift of the square.
\item For all $b:B$ a lift of:
 \begin{center}
    \begin{tikzcd}
      \mathrm{fib}_u(b)\ar[d]\ar[r,"g"] & X\ar[d,"p"]\\
      1 \ar[r,swap,"f(b)"]& Y
    \end{tikzcd}
  \end{center}
 \item For all $b:B$ a lift of:
  \begin{center}
    \begin{tikzcd}
      \mathrm{fib}_u(b)\ar[r,"g"]\ar[d] & \mathrm{fib}_p(f(b))\ar[d]\\
      1 \ar[r] & 1
    \end{tikzcd}
  \end{center}
\end{enumerate}
\end{lemma}

\begin{proof}
We have that (ii) is equivalent to (iii) by definition of the fiber.

We can assume that the square is of the form:
 \begin{center}
    \begin{tikzcd}
   \sum_{b:B}P_b\ar[d]\ar[r] & \sum_{y:Y} Q_y\ar[d]\\
     B \ar[r,swap,"f"]& Y
    \end{tikzcd}
  \end{center}
  where the top map is:
  \[
  \lambda (b,p). (f(b),g(b,p))
  \]
  for some $g:\prod_{b:B} P_b \to Q_{f(b)}$. A lift of this square is the same as an inhabitant of:
 % \[
 % h : B \to \sum_{y:Y}Q_y
 % \]
  %such that $\pi_Y\circ h = f$ and for all $b:B$ and $p:P(b)$ we have $(f(b),g(b,p)) = h(b)$. But this is equivalent to a giving:
  %\[
  %h: \prod_{b:B}Q_{f(b)}
  %\]
  % such that for all $b:B$ and $p:P(b)$ we have $g(b,p) = h(b)$. This in turn is the same as an inhabitant of:
  \[
  \prod_{b:B} \sum_{q:Q_{f(b)}} \prod_{p:P_b} g(b,p)=q
  \]
  which is equivalent to (iii).
\end{proof}

\begin{lemma}
\label{pointwise-lift-is-enough}
Assume given maps $u:A\to B$ and $p:X\to Y$.
Then if $p$ has the right lifting property (resp. at most one lift) against $\mathrm{fib}_u(b)\to 1$ for all $b:B$, then it has the right lifting property (resp. at most one lift) against $u$.
\end{lemma}

\begin{proof}
By \cref{lifts-equivalence-pointwise} a lift of a square
 \begin{center}
    \begin{tikzcd}
      A \ar[r]\ar[d,swap,"u"]& X\ar[d,"p"]\\
      B \ar[r]& Y
    \end{tikzcd}
  \end{center}
  is equivalent a family of lifts of squares of the form: 
   \begin{center}
    \begin{tikzcd}
      \mathrm{fib}_u(b)\ar[d]\ar[r] & X\ar[d,"p"]\\
      1 \ar[r]& Y
    \end{tikzcd}
  \end{center}
  But a product of contractible types (resp. propositions) is itself contractible (resp. a proposition) so we can conclude. 
\end{proof}

\begin{lemma}
\label{lifting-defined-fiberwise}
A map $p$ has the right lifting property (resp. at most one lift) against a map $P\to 1$ if and only all the fibers of $p$ have this property against $P\to 1$.
\end{lemma}

\begin{proof}
This is an immediate consequence of the universal property of the fibers.
\end{proof}