Recall from \cite{draft} that for a type $X$ with $x : X$, we say that $y : X$ is in the
\emph{formal neighborhood} of $x$ if $\neg \neg (x = y)$.
We will write $N_\infty(x)$ for the type of $y : X$ such that $\neg \neg(x = y)$.

\begin{definition}
For a type $X$ and $k : \N$, we say $x, y : X$ are \notion{neighbors of order $k$}
if $\neg \neg (x = y)$ and for all $U \subseteq X$ open containing $x$ (and hence $y$), 
and $q_0,\ldots,q_k : U \to R$,
we have $(q_0(x) - q_0(y))\cdots (q_k(x)-q_k(y)) = 0$.
This is equivalent to $q_0(y)\cdots q_k(y) = 0$ for all $q_i : U \to R$ such that $q_i(x) = 0$.
We write $N_k(x)$ for the type of all order $k$ neighbors of $x$.
\end{definition}
Note that this defines a symmetric and reflexive relation on $X$. It is functorial in the
sense that if $x$, $y$ are neighbors of order $k$ and $f : X \to Y$ is some map,
then $f(x)$ and $f(y)$ are also neigbors of order $k$.
The relation is not transitive, but we do have that if $x$, $y$ are neighbors of order
$k$ and $y$, $z$ are neighbors of order $l$, then $x$ and $z$ are neighbors of order
$k+l$. If $X$ is a scheme, then $x,y$ are order $0$ neighbors if and only if $x = y$.

\begin{lemma}\label{neighbor-criterion}
Let $X = \Spec A$ be an affine scheme and suppose $f_1,\ldots,f_n$ generate $A$.
Then $x$ and $y$ are neighbors of order $k$ iff 
$(f_1(x) - f_1(y))^{e_1} \cdots (f_n(x) - f_n(y))^{e_n} = 0$ for any $e_1,\ldots,e_n : \N$
with $e_1 + \ldots + e_n = k+1$.
\end{lemma}
\begin{proof}
The forward implication is direct, so let us consider the reverse.
We may suppose $f_i(x) = 0$, since $f_i - f_i(x)$ also forms a generating set of $A$.
In particular $f_i(y)$ is nilpotent, so not not zero.
So not not every $f_i(y)$ is zero. This shows that not not $x = y$, since
points of an affine scheme are separated by functions $X \to R$.

Now let $U \subseteq X$ be an open neighborhood of $x$ and $q_0,\ldots,q_k : U \to R$
with $q_i(x) = 0$.
We can write $U$ as $D(g_1,\ldots,g_m)$ with $g_j : A$.
Since $x \in U$ we have $x : D(g_j)$ for some $j$.
Write $g = g_j$.
Now $q_i$ restricts to a map $D(g) \to R$, which corresponds to an element
$p_i : A_g$. Write $p_i = a_i/{g^l}$ with $a_i : A$.
We want to show $a_0(y)\cdots a_k(y) = 0$.
By assumption we can write each $a_i$ as a polynomial in $f_1,\ldots,f_n$
with zero constant term. Thus $a_0(y)\cdots a_k(y)$ is a sum of monomials of degree
at least $k+1$ in $q_1,\ldots,q_n$. Since each monomial vanishes, so does the sum.
\end{proof}

As a corollary, we have that $N_k(x)$ is an affine scheme
for any point $x$ of a scheme; 
indeed it is a closed subscheme of any open affine neighborhood of $x$.

\begin{lemma}\label{formal-nbhd-union}
Let $X$ be a scheme and $x, y : X$. If $y$ is in the formal neighborhood of $x$,
then there merely exists $k : \N$ such that $y$ is in $N_k(x)$.
\end{lemma}
\begin{proof}
Without loss of generality $X = \Spec A$ is affine. Pick generators
$f_1,\ldots,f_n$ of $A$. Then $\neg \neg(f_i(x)=f_i(y))$.
So $f_i(x)-f_i(y)$ is nilpotent for each $i$.
Say $(f_i(x)-f_i(y))^{k_i+1} = 0$.
Then $x$ and $y$ are order $k_1+\ldots+k_n$ neighbors, since
if $e_1+\ldots+e_n = k_1+\ldots+k_n+1$, we have
$e_i \ge k_i+1$ for some $i$.
\end{proof}

\begin{definition}
For $X$ a type and $x : X$ a point, the stalk $\OO_x$ is the (filtered) 
colimit of $U \to R$ over open neighbourhoods $x \in U \subseteq X$.
\end{definition}

As a warning, note that we cannot expect $\OO_x$ to be finitely presented.
If $X = \Spec A$ is affine, then $\OO_x$ is the localisation of $A$ away from the kernel
of $x : A \to R$.
There is a natural map $\OO_x \to R$ of $R$-algebras, evaluating a germ at $x$.

\begin{lemma}
The map $\OO_x \to R$ reflects invertible elements. In particular $\OO_x$ is a local ring.
\end{lemma}
\begin{proof}
Consider $x \in U \subseteq X$, $f : U \to R$. Suppose $f(x)$ is invertible.
Then $\{y : U \mid f(y) \ne 0 \}$ is also an open neighborhood of $x$,
and $f$ is invertible on it. Hence $f$ is invertible in $\OO_x$.
\end{proof}
\begin{definition}
The kernel of the evaluation map $\OO_x \to R$ is the `maximal ideal' $\mm_x$.
\end{definition}

\begin{lemma}\label{nbhd-is-spec}
If $X$ is a scheme with $x : X$, then $N_k(x)$ is the
spectrum of $\OO_x / \mm_x^{k+1}$. In particular the latter is finitely presented
over $R$.
\end{lemma}
\begin{proof}
The map from $N_k(x)$ neighboorhood can be described directly, by evaluating
at $y$. To see this is an equivalence, we suppose without loss of generality that
$X = \Spec A$ is affine. Then $\OO_x / \mm_x^{k+1} = A / (\mm_x \cap A)^{k+1}$,
essentially since if $f : A$ with $f(x) \ne 0$, then we have that
$f$ is invertible in $A / (\mm_x \cap A)^{k+1}$, by the formula
$(1 - g)(1 + g + \ldots + g^k) = 1$ modulo $g^{k+1}$.
The spectrum of the latter is clearly the order $k$ neighborhood of $x$.
\end{proof}

\begin{lemma}
For $V$ a finitely presented $R$-module, the disk $\D(V)$ is the
first order neighborhood of $0$ in $V^\star$.
\end{lemma}
\begin{proof}
By \cref{functions-on-module}, the algebra $V^\star \to R$ is generated by
elements $f \mapsto f(v)$ for $v : V$. The result follows from \cref{neighbor-criterion}.
\end{proof}

\begin{lemma}\label{maximal-cotangent}
Let $X$ be a scheme and $x : X$ a point. 
Then $N_1(x)$ is equivalent to the
disk $\D(T^\star_x X)$. Moreover, we have an isomorphism
of $R$-modules
\[ \mm_x / \mm_x^2 \simeq T^\star_p X. \]
\end{lemma}
This means that $x : X$ and $0 : T_p X$ have the same first order neighborhoods,
which aligns well with an intuitive understanding of tangent spaces.
\begin{proof} %TODO think about simplifying this proof!!
The ring of functions $N_1(x) \to R$ is equivalent to
$\OO_x / \mm_x^2$, which is equivalent to the square-zero extension
$R \oplus \mm_x / \mm_x^2$. Any $r : R$ determines an endomorphism of
$\mm_x / \mm_x^2$ (multiplication by $r$), and hence an endomorphism
of $\OO_x / \mm_x^2$, and hence an endomorphism of $N_1(x)$, which
by abuse of notation we write $y \mapsto (1-r)x + ry$. 
The defining property is that
for $f : N_1(x) \to R$, we have $f((1-r)x + ry) = (1-r)f(x) + rf(y)$.
Given $y : N_1(x)$,
we define a tangent vector $\D(1) \to_\pt (X,x)$ by $t \mapsto (1-t)x + ty$.
This defines a map $N_1(x) \to_\pt (T_x X, 0)$. It lifts to
a map $N_1(x) \to_\pt \D(T^\star_x X)$ by functoriality of $N_1$.

Next, we define a map $(N_1(x) \to R) \to \D(T^\star_x X) \to R$.
Suppose $f : N_1(x) \to R$ and $v : \D(T^\star_x X)$.
Then $v$ restricts to a map $\D(1) \to_\pt N_1(x)$
(as does any tangent vector),
so we have $f \circ v : \D(1) \to_\pt (R,f(x))$.
By SQC, we can write $f(v(t)) = f(x) + ct$ for a well-defined $c : R$.
We claim the map $f \mapsto v \mapsto f(x) + c$ respects multiplication;
it is clear that it is $R$-linear in $f$. Thus suppose $f, f' : N_1(x) \to R$.
Note that $(ff')(v(t)) = (f(x) + ct)(f'(x)+c't) = (ff')(x) + (f(x)c' + f'(x)c)t$,
since $t^2 = 0$. It remains to show that
$(ff')(x) + f(x)c' + f'(x)c = (f(x) + c)(f'(x) + c')$, i.e. that
$cc' = 0$. This follows from the fact that $v$ is in the first order neighbourhood
of $0$.

This defines maps back and forth between
$N_1(x)$ and $\D(T^\star_x X)$. We omit the verification that they are inverse to
each other. The claim that $\mm_x / \mm_x^2 = T^\star_p X$ follows from the
fact that a module can be recovered from its square-zero extension.
\end{proof}

The axiom of synethetic quasi-coherence applies only to finitely presented $R$-algebras,
and would be false for general $R$-algebras. Surprisingly, it is still true in 
the following special case.
\begin{definition} 
For $X$ a type and $x : X$ a point, let $\widehat \OO_x$ denote the
completion of $\OO_x$ at the ideal $\mm_x$. That is, $\widehat \OO_x$ is 
the inverse limit of 
$R \leftarrow \OO_x / \mm_x \leftarrow \OO_x / \mm_x^2 \leftarrow \ldots$
\end{definition}
For example, if $X = \A^n$, then $\widehat \OO_x$ is the power series ring
in $n$ variables.
\begin{lemma}%TODO attribute this to Ingo?
For $X$ a scheme and $p : X$ a point, $\widehat \OO_p$ is the ring of functions
on the formal neighborhood $N_\infty(p)$ of $p$. Conversely,
$N_\infty(p)$ is the spectrum of $\widehat \OO_p$.
\end{lemma}
\begin{proof}
Without loss of generality we assume $X$ is affine. Since $N_\infty(p)$ is the
sequential colimit of $N_k(p)$ over $k : \N $ by \cref{formal-nbhd-union}, the
ring of functions $N_\infty(p) \to R$ is indeed the limit of the rings of
functions $N_k(p) \to R$, which is $\OO_p / \mm_p^{k+1}$ by \cref{nbhd-is-spec}.

It remains to show that any $R$-algebra homomorphism $\widehat \OO_p \to R$
is given by evaluation at some $y : N_k(p)$. That is, given
$f : \widehat \OO_p \to R$, we need to show that $f$ factors through
$\OO_p / \mm_p^k$ for some $k$.
Let $\mm_p$ be generated by $X_1, \ldots X_n$.
We claim that $\neg \neg (f(X_i) = 0)$.
Indeed suppose $f(X_i) \ne 0$, so that $f(X_i)$ is invertible. Say
$yf(X_i) = 1$ with $y : R$. Now $Z \coloneqq 1 + yX_i + y^2X_i^2 + \ldots$
is a well-defined element of $\widehat \OO_p$, with
$Z = 1 + yX_i Z$. Hence $f(Z) = 1 + y f(X_i)f(Z) = 1 + f(Z)$.
This means $1 = 0$ in $\widehat \OO_p$, which means $1 = 0$ in $R$, which is impossible.

Thus $\neg \neg (f(X_i) = 0)$, so $f(X_i)$ is nilpotent in $R$.
Say $f(X_i)^{k_i+1} = 0$. Then $f(\mm_x^{k+1}) = 0$ where $k = k_1 + \ldots + k_n$.
So $f$ factors through $\OO_x / \mm_x^{k+1}$, as needed.
\end{proof}
