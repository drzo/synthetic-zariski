
\subsection{Reduced schemes}

There is a \emph{candidate} definition of reduced schemes.
The analogue to the classical definition, that an affine scheme is reduced,
if its algebra of functions is reduced, is expected to be useless in the synthetic setup.

\begin{definition}
  An affine scheme $X=\Spec A$ is \notion{reduced},
  if for all functions $f:A$, nilpotency implies $\neg\neg (f=0)$.
\end{definition}

An alternative, stronger criterion would be that if $f : A$ is nilpotent, then
$f = r_1 a_1 + \ldots + r_n a_n$ with $r_i : R$ nilpotent and $a_i : A$.

\begin{example}
  \begin{enumerate}[(a)]
  \item $\D(1)$ is not reduced.
    The algebra of functions is $R+\varepsilon R$ and we know that $\varepsilon$ is nilpotent and non-zero.
  \item $\A^1$ is reduced. To see this, let $f:R[X]$ be nilpotent.
    Then all coefficients of $f$ are nilpotent and since we proof a double-negation,
    we can assume they are zero.
  \item This example might be a reason to reject this definition of reduced:
    Let $\varepsilon:R$ be nilpotent.
    Then for any nilpotent function $f:R/\varepsilon$, there merely is $g:R$ with $g+(\varepsilon)=f$.
    By nilpotency of $f$ we have a $k:\N$ such that $g^k=\varepsilon\cdot h$ for some $h:R$.
    So there is $l:\N$ such that $g^{k\cdot l}=\varepsilon^l\cdot h^l=0$ in $R$, so $\neg\neg (g=0)$ and $\neg\neg (f=0)$.
  \end{enumerate}
\end{example}

