
\subsection{Reduced schemes}
There is a \emph{candidate} definition of reduced schemes.
The analogue to the classical definition, that an affine scheme is reduced,
if its algebra of functions is reduced, is expected to be useless in the synthetic setup.

\begin{definition}
  An affine scheme $X=\Spec A$ is \notion{reduced},
  if for all functions $f:A$, nilpotency implies $\neg\neg (f=0)$.
\end{definition}

\begin{example}
  \begin{enumerate}[(a)]
  \item $\D(1)$ is not reduced.
    The algebra of functions is $R+\varepsilon R$ and we know that $\varespilon$ is nilpotent and non-zero.
  \item $\A^1$ is reduced. To see this, let $f:R[X]$ be nilpotent.
    Then all coefficients of $f$ are nilpotent and since we proof a double-negation,
    we can assume they are zero.
  \end{enumerate}
\end{example}