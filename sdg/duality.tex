{
\newcommand{\cat}[1]{\mathcal{#1}}
\newcommand{\theory}[1]{\mathbf{#1}}
\newcommand{\Psh}{\mathop{PSh}}
\newcommand{\Set}{\mathsf{Set}}
\newcommand{\TT}{\theory{T}}
\newcommand{\EE}{\theory{E}}
\newcommand{\Com}{\theory{Com}}
\newcommand{\diffquot}[2]{\frac{\Delta{#1}}{\Delta{#2}}}

We postulate the existence of a commutative ring \(R\) which satisfies the following axioms:
\begin{description}
  \item[\textbf{(Locality)}]
    The ring \(R\) is local.
  \item[\textbf{(Differentiability)}]
    The ring \(R\) is a Fermat ring.
  \item[\textbf{(Duality)}]
    For any finitely presented \(\EE_R\)-algebra \(A\), the canonical homomorphism
    \[...\]
    is an isomorphism of \(\EE_R\)-algebras.
\end{description}

\begin{lemma}
  Let \(A\) be a finitely presented \(\EE_R\)-algebra.
  Then \(\Spec_{\EE_R}(A)=\emptyset\) if and only if \(A=0\).
\end{lemma}
\begin{proof}
  Like in synthetic algebraic geometry.
  The if direction uses locality of \(R\).
\end{proof}

\begin{lemma}
  The ring \(R\) is a denial field. That is, if \(r:R\) does not vanish it is invertible.
\end{lemma}
\begin{proof}
  Using the previous lemma and lemma ?? on quotients of algebras over Fermat theories, the proof of the corresponding statement from synthetic algebraic geometry carries over.
\end{proof}

}
