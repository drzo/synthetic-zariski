{
\newcommand{\cat}[1]{\mathcal{#1}}
\newcommand{\theory}[1]{\mathbf{#1}}
\newcommand{\Psh}{\mathop{PSh}}
\newcommand{\Set}{\mathsf{Set}}
\newcommand{\TT}{\theory{T}}
\newcommand{\EE}{\theory{E}}
\newcommand{\Com}{\theory{Com}}

\subsection{Endomorphism Theories}

% \begin{remark}
%   Given an object \(R\) of a category \(\cat C\) with specified finite powers, theory morphisms \(\TT\to\EE_R\) are in bijection with \(\TT\)-algebra structures on \(R\).
% \end{remark}

Suppose \(\TT\) is any algebraic theory (for example, the theory of rings) and let \(\cat A\) be any full subcategory of the category of \(\TT\)-algebras in the category of sets (think of finitely presented algebras).
We write \(\cat L\) for the opposite category of \(\cat A\) and \(\ell A\) for the formal dual of a \(\TT\)-algebra \(A\) in \(\cat A\).
Moreover, we suppose \(\cat L\) is equipped with a Grothendieck topology (for example, the Zariski topology).
Then let \(\cat T\) be the sheaf topos over \(\cat L\), denote the Yoneda embedding \(\cat L\to\cat T\) by \(Y\) and the sheafification functor \(\Psh(\cat L)\to\cat T\) by \(s\).
Finally, we write \(F_\TT(n)\) for the free \(\TT\)-algebra on \(n\) generators and then consider the \emph{generic \(\TT\)-algebra} \(R\colonequiv sY\ell(F_{\TT}(1))\) in \(\cat T\).

\begin{proposition}
  The endomorphism theory of \(R\) is equivalent to the algebraic theory \(\TT\).
\end{proposition}
\begin{proof}
  For two natural numbers \(n,m:\N\) we calculate:
  \begin{align*}
    \EE_R(n,m)
    &= \cat T(R^n,R^m)\\
    &= \cat T(sY\ell(F_{\TT}(n)),sY\ell(F_{\TT}(m)))\\
    &= \Psh(\cat L)(Y\ell(F_{\TT}(n)),Y\ell(F_{\TT}(m)))\\
    &= \cat L(\ell(F_{\TT}(n)),\ell(F_{\TT}(m)))\\
    & = \cat A(F_{\TT}(m),F_{\TT}(n))\\
    & = \Alg{\TT}(F_{\TT}(m),F_{\TT}(n))\\
    & = \Alg{\TT}(\TT(m,\_),\TT(n,\_))\\
    & = [\TT,\Set](\TT(m,\_),\TT(n,\_))\\
    & = \TT(n,m)
  \end{align*}
Here, we used the fact that the free \(\TT\)-algebra on \(n\) generators is given by the representable functor \(\TT(n,\_)\).
The last equality is just the Yoneda embedding for copresheaves.
\end{proof}


\subsection{Fermat Algebras}

\begin{definition}
  A \emph{Fermat ring (object)} is a commutative ring object whose endomorphism theory is a Fermat theory.
\end{definition}

\begin{example}
  The structure sheaf in the small Zariski topos of a scheme is a Fermat ring?
\end{example}

\begin{example}
  The affine line in the big Zariski topos is a Fermat ring?
\end{example}

\begin{example}
  The real line in the Dubuc topos is a Fermat ring?
\end{example}



}
