\subsection{Generalities on delooping in $T$-sheaves}

\begin{definition}
A type $A$ is $T$-connected if:
\[\forall(x,y:A).\ \propTrunc{x=y}_T\]
\end{definition}

The key intuition for the next lemma is that both $A$ and $B$ are deloopings of the same group in the topos of sheaves.

\begin{lemma}\label{deloopings-equivalence}
Assume $A,B$ pointed $T$-connected sheaves. Then any pointed map $f:A\to_\pt B$ inducing an equivalence:
\[\Omega f : \Omega A \simeq \Omega B\]
is itself an equivalence.
\end{lemma}

\begin{proof}
First we prove that is an embedding. So we have to prove that for all $x,y:A$ the map:
\[\mathrm{ap}_f : x=y \to f(x)=f(y)\]
is an equivalence. Since $A$ and $B$ are sheaves so are their identity types, so the map being an equivalence is a modal proposition and we can assume $x$ and $y$ are the basepoint, in which case it is part of the hypothesis.

Now we prove it is surjective, indeed for any $x:B$ we have that $\fib_f(x)$ is a modal propositions so when proving it is inhabited we can assume $x$ is the basepoint of $B$ and give the basepoint of $A$ as a pre-image.
\end{proof}


\subsection{$\Aut(M_{n+1}(R)) = \Aut(\bP^n) = \PGL_{n+1}(R)$}

\begin{proposition}\label{Aut-MnR-PGL}
The map:
\[\alpha:\PGL_{n+1}(R)\to\Aut(M_{n+1}(R))\]
\[P\mapsto PMP^{-1}\]
is an equivalence.
\end{proposition}

\begin{proof}
TODO
\end{proof}

\begin{proposition}\label{Aut-Pn-PGL}
The map:
\[\beta:\PGL_{n+1}(R)\to\Aut(\bP^n)\]
\[X\mapsto PX\]
is an equivalence.
\end{proposition}

\begin{proof}
\cite{TODO}
\end{proof}


\subsection{The Severi-Brauer construction is an equivalence}

\begin{proposition}\label{right-ideal-is-equivalence}
The map:
\[\RI:\AZ_n\to\SB_n\]
is an equivalence.
\end{proposition}

\begin{proof}
By \cref{deloopings-equivalence} it is enough to prove that the top map in the triangle:
\begin{center}
\begin{tikzcd}
\Aut(M_n(R))\ar[rr,"\Omega\RI"] & & \Aut(\bP^n) \\
& \PGL_{n+1}(R)\ar[ru,swap,"\alpha"]\ar[lu,"\beta"] & \\
\end{tikzcd}
\end{center}
is an equivalence. But since the two other maps in the triangle are equivalences by \cref{Aut-MnR-PGL} and \cref{Aut-Pn-PGL}, it is enough to prove that the triangle commutes. To do this we need to check that for all $P:\PGL_{n+1}(R)$ we have that:
\[\delta^{-1}\circ \mathrm{ap}_\RI(\alpha(P))\circ\delta = \beta(P)\]
in $\Aut(\bP^n)$, with $\delta$ defined in \cref{right-ideal-of-matrices-are-projective}. So we need to prove the following square commutes:
\begin{center}
\begin{tikzcd}
\RI(M_{n+1}(R))\ar[rr,"I\mapsto PIP^{-1}"]&& \RI(M_{n+1}(R)) \\
\bP^n\ar[u,"\delta"]\ar[rr,swap,"X\mapsto PX"]&& \bP^n\ar[u,swap,"\delta"] 
\end{tikzcd}
\end{center}
where $\mathrm{ap}_\RI$ was computed using path induction.

We can check that:
\[\delta(Y) = \{M:M_n(R)\ |\ \forall A,B:R^{n+1}.\ A^tX\cdot B^tM = B^tX\cdot A^tM\}\]
To check that:
\[P\delta(X)P^-{1} = \delta(PX)\]
we just need to check an inclusion as both are free module of the same dimension. Assume $M\in\delta(X)$, to check that $PMP^{-1}\in\delta(PX)$ we need to check that for all $A,B:R^{n+1}$ we have that:
\[A^tPX\cdot B^tPMP^{-1} = B^tX\cdot A^tPMP^{-1}\]
but since $M\in\delta(X)$ we have that:
\[(P^tA)X\cdot (P^tB)^tM = (P^tB)^tX\cdot (P^tA)^tM\]
which gives us what we want.
\end{proof}