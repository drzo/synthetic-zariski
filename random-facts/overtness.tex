A type~$X$ is \emph{overt} iff~$X$-indexed sums preserve openness, that is iff
for every open~$U \subseteq X$ the proposition~``$U$ is inhabited'' is open again. The
following proposition emerged at the 2024 Dagstuhl meeting, prompted by and
jointly with Andrej Bauer and Martín Escardó:

\begin{proposition}\label{r-overt}
The following statements are equivalent.
\begin{enumerate}
\item The ring~$R$ is overt.
\item For every polynomial~$f : R[X]$, the proposition that~$f$ has an
anti-zero (a number~$x$ such that~$f(x) \neq 0$) is open.
\item The ring~$R$ is infinite in the sense that for every natural number~$n$,
there are~$n$ pairwise distinct elements of~$R$.
\item The ring~$R$ is infinite in the sense that for every finite
list~$x_1,\ldots,x_n$ of elements of~$R$, there is an element~$y$
distinct from all of the~$x_i$.
\end{enumerate}
\end{proposition}

\begin{proof}Statement~2 is just the special instance of Statement~1 for the
case~$U = D(f)$. Conversely, Statement~1 follows from Statement~2 because
an arbitrary open of~$R$ is of the form~$\bigcup_{i=1}^n D(f_i)$ and because
finite disjunctions of open propositions are open. Trivially, Statement~4
implies Statement~3.

To verify that Statement~3 implies Statement~2, let~$f : R[X]$ be a polynomial.
By definition, there is an upper bound~$n$ of the formal degree of~$f$. By
assumption, there are~$n+1$ pairwise distinct numbers~$r_0,\ldots,r_n$. Then the
statement that~$f$ has an antizero is equivalent to the finite
disjunction~$\bigvee_{i=0}^n (f(r_i) \neq 0)$: The ``if'' direction is trivial,
and for the ``only if'' direction, assume to the contrary that~$f(r_0) = \ldots
= f(r_n) = 0$. Then~$f$ can be factored as~$f(X) = (X-r_1)\cdots(X-r_n) \cdot
c$. Because~$f(r_0) = 0$, we have~$c = 0$ and hence~$f = 0$. This is a
contradiction to~$f$ admitting an antizero.

To verify that Statement~2 implies Statement~4, let numbers~$x_1,\ldots,x_n$
of~$R$ be given. Up to double negation, the monic
polynomial~$(X-x_1)\cdots(X-x_n)+1$ has a zero. Hence up to double negation,
the polynomial~$f(X) = (X-x_1)\cdots(X-x_n)$ has an antizero. By assumption,
this statement is open and therefore double negation stable, hence~$f$ actually
has an antizero.
\end{proof}

\begin{remark}The equivalent conditions of Proposition~\ref{r-overt} are
satisfied in case the external base ring~$k$ contains, for every natural
number~$n$, elements~$x_1,\ldots,x_n$ whose pairwise differences are
invertible.\end{remark}

\begin{proposition}If~$R$ is overt, then every open neighborhood of~$0$ in~$R$
is infinite in the sense that for every finite list~$x_1,\ldots,x_n$ of
elements, there is an element~$y$ distinct from all the~$x_i$.
\end{proposition}

\begin{proof}Let~$U \subseteq R$ be an open neighborhood of~$0$. Then there is
a polynomial~$f : R[X]$ such that~$0 \in D(f) \subseteq U$.
Let~$x_1,\ldots,x_n$ be elements of~$U$. Up to double
negation, the polynomial~$(X-x_1)\cdots(X-x_n) \cdot X \cdot f + 1$ has a zero. Such a zero is an
element~$y$ which is distinct from all the~$x_i$ (and from~$0$). So up to
double negation, the polynomial~$(X-x_1)\cdots(X-x_n) \cdot X \cdot f$ has an
antizero. Because~$R$ is overt, the existence of an antizero is (open and
hence) double negation stable so that we can conclude that there actually is an
antizero.
\end{proof}
