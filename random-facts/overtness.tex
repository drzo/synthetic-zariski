A type~$X$ is \emph{overt} iff~$X$-indexed sums preserve openness, that is iff
for every open~$U \subseteq X$ the proposition~``$U$ is inhabited'' is open again. The
following proposition emerged at the 2024 Dagstuhl meeting, prompted by and
jointly with Andrej Bauer and Martín Escardó:

\begin{proposition}\label{r-overt}
The following statements are equivalent.
\begin{enumerate}
\item The ring~$R$ is overt.
\item For every polynomial~$f : R[X]$, the proposition that~$f$ has an
anti-zero (a number~$x$ such that~$f(x) \neq 0$) is open.
\item The ring~$R$ is infinite in the sense that for every natural number~$n$,
there are~$n$ pairwise distinct elements of~$R$.
\item The ring~$R$ is infinite in the sense that for every finite
list~$x_1,\ldots,x_n$ of elements of~$R$, there is an element~$y$
distinct from all of the~$x_i$.
\end{enumerate}
\end{proposition}

\begin{proof}Statement~2 is just the special instance of Statement~1 for the
case~$U = D(f)$. Conversely, Statement~1 follows from Statement~2 because
an arbitrary open of~$R$ is of the form~$\bigcup_{i=1}^n D(f_i)$ and because
finite disjunctions of open propositions are open. Trivially, Statement~4
implies Statement~3.

To verify that Statement~3 implies Statement~2, let~$f : R[X]$ be a polynomial.
By definition, there is an upper bound~$n$ of the formal degree of~$f$. By
assumption, there are~$n+1$ pairwise distinct numbers~$r_0,\ldots,r_n$. Then the
statement that~$f$ has an antizero is equivalent to the finite
disjunction~$\bigvee_{i=0}^n (f(r_i) \neq 0)$: The ``if'' direction is trivial,
and for the ``only if'' direction, assume to the contrary that~$f(r_0) = \ldots
= f(r_n) = 0$. Then~$f$ can be factored as~$f(X) = (X-r_1)\cdots(X-r_n) \cdot
c$. Because~$f(r_0) = 0$, we have~$c = 0$ and hence~$f = 0$. This is a
contradiction to~$f$ admitting an antizero.

To verify that Statement~2 implies Statement~4, let numbers~$x_1,\ldots,x_n$
of~$R$ be given. Up to double negation, the monic
polynomial~$(X-x_1)\cdots(X-x_n)+1$ has a zero. Hence up to double negation,
the polynomial~$f(X) = (X-x_1)\cdots(X-x_n)$ has an antizero. By assumption,
this statement is open and therefore double negation stable, hence~$f$ actually
has an antizero.
\end{proof}

\begin{remark}The equivalent conditions of Proposition~\ref{r-overt} are
satisfied in case the external base ring~$k$ contains, for every natural
number~$n$, elements~$x_1,\ldots,x_n$ whose pairwise differences are
invertible.\end{remark}

\begin{proposition}If~$R$ is overt, then every open neighborhood of~$0$ in~$R$
is infinite in the sense that for every finite list~$x_1,\ldots,x_n$ of
elements, there is an element~$y$ distinct from all the~$x_i$.
\end{proposition}

\begin{proof}Let~$U \subseteq R$ be an open neighborhood of~$0$. Then there is
a polynomial~$f : R[X]$ such that~$0 \in D(f) \subseteq U$.
Let~$x_1,\ldots,x_n$ be elements of~$U$. Up to double
negation, the polynomial~$(X-x_1)\cdots(X-x_n) \cdot X \cdot f + 1$ has a zero. Such a zero is an
element~$y$ which is distinct from all the~$x_i$ (and from~$0$). So up to
double negation, the polynomial~$(X-x_1)\cdots(X-x_n) \cdot X \cdot f$ has an
antizero. Because~$R$ is overt, the existence of an antizero is (open and
hence) double negation stable so that we can conclude that there actually is an
antizero.
\end{proof}

\begin{proposition}\label{finite-compact}
Let~$A$ be a finitely presented~$R$-algebra. If furthermore~$A$ is finitely
generated as an~$R$-module, then~$X = \Spec(A)$ is compact (in the sense
that~$X$-indexed products of opens are open).\end{proposition}

\begin{proof}Let~$A = R[X_1,\ldots,X_k]/(q_1,\ldots,q_t)$. As~$A$ is finitely
generated as an~$R$-module, there are monic polynomials~$f_1,\ldots,f_k$ of
positive degree such that~$f_\ell(X_\ell) = 0$ in~$A$. Hence~$\Spec(A)$ is a
closed subset of~$\prod_{\ell=1}^k \Spec(R[X_\ell]/(f_\ell))$. As closed
subsets of compact sets are compact (XXX supply reference) and finite products
of compact sets are compact (XXX supply reference), we are reduced to the
situation that~$A = R[X]/(f)$ where~$f = \sum_{j=0}^n a_{n-j} X^j$ is a monic
polynomial of positive degree~$n$. In this case~$X$ is the set of zeros of~$f$
and it suffices to prove: For every finite list~$g_1,\ldots,g_m : R[X]$ of
polynomials, the proposition that
\begin{equation}\label{finite-compact-eq}
  \tag{$\dagger$}
  \forall(u : R).\ \bigl(f(u) = 0 \Rightarrow \bigvee_{i=1}^m g_i(u) \neq 0\bigr)
\end{equation}
is open. To this end, we consider the polynomial
\[ p(U_1,\ldots,U_n,T) \vcentcolon= \prod_{j=1}^n \sum_{i=1}^m g_i(U_j) T^{i-1}. \]
Regarded as a polynomial in~$T$, its coefficients are symmetric in the~$U_i$.
By the fundamental theorem on symmetric polynomials, there are
polynomials~$h_0,\ldots,h_m : R[A_0,\ldots,A_{n-1}]$ such that
\[ p(U_1,\ldots,U_n,T) = \sum_{i=1}^m h_i(e_1(\vec U),\ldots,e_n(\vec U)) T^{i-1}. \]
We claim that proposition~\eqref{finite-compact-eq} is equivalent to the
disjunction
\begin{equation}\label{finite-compact-eq2}
  \tag{$\ddagger$}
  \bigvee_{i=1}^m (h_i(a_1,\ldots,a_n) \neq 0).
\end{equation}
Assume~Proposition~\eqref{finite-compact-eq}. As
Proposition~\eqref{finite-compact-eq2} is negative and hence double
negation stable, we may assume that~$f$ splits into linear factors: $f(X) =
\prod_{j=1}^n (X-u_j)$. By assumption, for every~$j \in \{1,\ldots,n\}$ we
have~$\bigvee_{i=1}^m (g_i(u_j) \neq 0)$. Hence
\begin{equation}\label{finite-compact-eq3}
  \tag{$\star$}
  1 \in \bigcap_{j=1}^n \bigl(g_i(u_j)\bigr)_{i=1}^m = c\Bigl(\sum_{i=1}^m g_i(u_j) T^{i-1}\Bigr) =
  c(p) = \bigl(h_i(a_1,\ldots,a_n)\bigr)_{i=1}^m,
\end{equation}
so Proposition~\eqref{finite-compact-eq2} holds. Here~$c$ refers to the radical
content of a polynomial, the radical of the ideal generated by its
coefficients, and the second equality is
by~\cite[Proposition~1]{banaschewski-vermeulen:radical}.
% https://core.ac.uk/reader/82176380

Conversely, assume Proposition~\eqref{finite-compact-eq2} and let~$u : R$ be a
zero of~$f$. As the claim that~$\bigvee_{i=1}^m (g_i(u) \neq 0)$ is double
negation stable, we may assume that~$f$ splits into linear factors,~$f(X) =
\prod_{j=1}^n (X-u_j)$, with~$u_1 = u$. By~\eqref{finite-compact-eq3}, we
have~$1 \in \bigl(g_i(u_1)\bigr)_{i=1}^m$ as desired.
\end{proof}

\begin{definition}A proposition~$p$ holds \emph{foo-locally} if and only if
there are numbers~$a_1,\ldots,a_n : R$ such that for every
partition~$\{1,\ldots,n\} = I \mathop{\dot\cup} J$, if all the~$a_i$ with~$i
\in I$ are zero and all the~$a_j$ with~$j \in J$ are invertible, then~$p$
holds.
\end{definition}

\begin{definition}A scheme is \emph{finite} if and only if it of the
form~$\Spec(A)$ where~$A$ is finitely presented as an~$R$-algebra and finitely
generated as an~$R$-module.\end{definition}

\begin{definition}A scheme is \emph{quasi-finite} if and only if foo-locally,
it is finite.
\end{definition}

\begin{proposition}Finite schemes are quasi-finite and
compact.\end{proposition}

\begin{proof}Compactness is by Proposition~\ref{finite-compact} and
quasi-finiteness is immediate.
\end{proof}

XXX Question: Does the converse hold? Classically it is
\href{https://math.stackexchange.com/questions/4674878/does-quasi-finite-and-\%C3\%A9tale-locally-closed-enough-to-imply-finite}{well-known}.
Need to check issue \#6.
