The following should be one part of showing that autormorphisms of $\mathbb P^n$ are
given by $\PGL_{n+1}(R)$.

\begin{theorem}
Let $f : \mathbb P^n \to \mathbb P^n$ be an arbitrary map.
Suppose $f$ is not not in $\PGL_{n+1}(R)$. Then $f$ is in $\PGL_{n+1}(R)$.
\end{theorem}
\begin{proof}
First note that $f$ sends any $n+1$ points in general position to $n+1$ points in general position.
This is because to be in general position means that a determinant is invertible, which
is a negative property, and any map in $\PGL$ preserves the property of being in general position.

Let $e_i$ be the point of $\mathbb P^n$ with zeroes in all coordinates except the $i$th.
Since $e_0, \cdots, e_n$ are in general position, so are $f(e_0), \cdots, f(e_n)$.
Thus we can find $f' \in \PGL$ with $f'(e_i) = f(e_i)$. Replacing $f$ by $f'^{-1} \circ f$,
we may assume that $f(e_i) = e_i$. Now if $f$ is in $\PGL$, it must be given by a diagonal matrix,
so $f$ is not not given by a diagonal matrix.

Write $U_i \subseteq \mathbb P^n$ for the standard affine patch of $\mathbb P^n$
consisting of points whose $i$th coordinate is invertible. 
We have that $f(U_i) \subseteq U_i$, since to be in $U_i$ is a negative property
and this containment holds if $f$ is given by a diagonal matrix.
Now $f$ restricts to a map $U_i \to U_i$ which by SQC is given by $n+1$
polynomials in $n$ variables. Homogenising these polynomials,
we see that for $x = [X_0 : \cdots : X_n] \in U_i$,
$f(x)$ is given by polynomials
$p_{i0},\cdots,p_{in} \in R[X_0, \cdots, X_n]$ homogeneous of some degree $d_i$,
so that $f(x) = [p_{i0} : \cdots : p_{in}]$,
where $p_{ii} = X_i^{d_i}$.

Since $f(e_i) = e_i$, we have that the coefficient of $X_i^{d_i}$ in $p_{ij}$ is
zero for $i \ne j$. We also have that the coefficient of $X_i^{d_i-1} X_j$ in $p_{ij}$
is invertible, since this holds when $f$ is given by a diagonal matrix 
(in this case the coefficient is the ratio of diagonal entries).
We also know that $p_{ij}p_{kl} = p_{il}p_{kj}$ for all $i, j, k, l$ since
the descriptions of $f$ on all the patches match up.

We claim that $p_{ij}$ is a unit multiple of $X_i^{d_i-1} X_j$. 
To this end, we claim that $p_{ij}$ is a sum of monomials which contain neither
$X_j^2$ nor $X_k$ for $k \ne i$, $k \ne j$. 
We prove both of these claims separately but using the same idea. The idea is that of
fixing a monomial ordering, and using the fact that if $g$, $h$ are monomials with $g$ `pseudomonic'
in the sense that $g$ has invertible leading coefficient,
then any bound on the degree of $gh$ gives a bound on the degree of $h$.
We may assume $i \ne j$ in either case.
\begin{enumerate}
\item $X_j^2$: consider the equation $p_{ij}p_{ji} = X_i^{d_i} X_j^{d_j}$. Consider some
monomial ordering which is lexicographic first on the degree of $X_j$ and then on $X_i$.
Here $p_{ji}$ is pseudomonic of degree $X_j^{d_j-1}X_i$.
Thus $p_{ij}$ has degree at most $X_j X_i^{d_i-1}$ 
(and indeed that coefficient is assumed invertible). So $X_j^2$ cannot appear in $p_{ij}$.
\item $X_k$ where $k \ne i$, $k \ne j$: consider the equation $p_{ij} p_{jk} = p_{ik} X_j^{d_j}$.
Consider some monomial ordering which is lexicographic first on the degree of $X_k$
and then on $X_j$. By the above, $p_{jk}$ is pseudomonic of degree $X_j^{d_j-1} X_k$,
and the degree of $p_{ik}$ is at most $X_k X_j^{d_i-1}$.
Thus the degree of $p_{ij}$ is at most $X_k X_j^{d_i-1} X_j^{d_j} / (X_j^{d_j-1}X_k) = X_j^{d_i}$.
Thus $X_k$ cannot appear.
\end{enumerate}

Given that $p_{ij}$ is a unit multiple of $X_i^{d_i-1} X_j$, it is direct that
$f$ is given by a diagonal matrix. This finishes the proof.
\end{proof}
