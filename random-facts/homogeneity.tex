\begin{definition}
  A \emph{graded} $R$-algebra $S$ is an $R$-algebra $S$ together with the datum of a direct sum decomposition
  \begin{equation*}
    S = \oplus_{n \in \Z} S_n
  \end{equation*}
  as an $R$-module such that $S_k \cdot S_\ell \subseteq S_{k + l}$
  for every $k$, $\ell \in \Z$.  We identify each $S_n$ for $n \in \Z$
  with its image in $S$.  The elements of $S_n$ are called \emph{homogeneous} of degree $n$.
\end{definition}

\begin{remark}
  The datum of a grading of an $R$-algebra $S$ thus gives us an
  essentially finite decomposition\footnote{That means there merely are $i,j:\Z$ such that $u_n=0$ if $n<i$ or $n>j$.}
  $u = \sum_{n \in \Z} u_n$ for every
  $u \in S$ where each $u_n$ is homogeneous of degree $n$.
  Furthermore, the decomposition of a homogeneous element $u$ is just $u = u$.
\end{remark}

\begin{proposition}
  \label{Spectrum of graded algebra}
  Let $S$ be a graded $R$-algebra and let $X \coloneqq \Spec S$.  For each $t \in R^\times$ and each $p \in X$ we define
  \begin{equation*}
    t \cdot p \coloneqq \left(u \mapsto \sum_{n \in \Z} p(u_n) t^n\right) \in \Spec S.
  \end{equation*}
  This defines an operation of the multiplicative group $R^\times$ on $X$.
\end{proposition}

\begin{proof}
  That $1 \cdot p = p$ for every $p \in X$ follows from
  $p(u) = \sum_{n \in \Z} p(u_n)\cdot 1^n$.  That
  $r \cdot (r' \cdot p) = (r \cdot r') \cdot p$ follows from the
  homogeneity of every $u_n$.
\end{proof}

\begin{theorem}
  The construction in \cref{Spectrum of graded algebra} yields an
  identification between the type of graded $R$-algebras and the type
  of affine schemes together with an action of the multiplicative
  group $R^\times$.
\end{theorem}

\begin{proof}
  We give the converse construction.  Let $X = \Spec S$ be an affine
  scheme together with the datum $R^\times \times X \to X$ of an
  action of the multiplicative group $R^\times$.  By synthetic quasi-coherence, the function $R^\times \times X \to X$ yields a homomorphism
  \begin{equation*}
    \alpha(t)\colon S \to S[t, t^{-1}] = S \otimes R[t, t^{-1}], u \mapsto \sum_{n \in Z} u_n t^n
  \end{equation*}
  of $R$-algebras where the sum on the right hand side is essentially
  finite.  That $1 \cdot p = p$ for all $p \in X$ is equivalent to
  $\alpha(1) = \id_S$, which, in turn, yields
  $u = \sum_{n \in \Z} u_n$.  That $t \cdot (t' \cdot p) = (t \cdot t') \cdot p$ is equivalent to
  $(\alpha(t) \otimes \id_{R[t, t^{-1}]}) \circ \alpha(t') = \alpha(t \cdot t')$, from which $\alpha(t) (u_n) = u_n t^n$ follows.
\end{proof}

\begin{remark}
  Let $S$ be a graded $R$-algebra.  The action of $R^\times$ on $\Spec S$ induces a natural action of $R^\times$ on the $R$-algebra of functions
  on $\Spec S$ given by
  \begin{equation*}
    R^{\Spec S} \times R^\times \to R^{\Spec S}, (f, t) \mapsto (p \mapsto f(t \cdot p)).
  \end{equation*}
  Under the identification $R^{\Spec S} = S$ given by synthetic quasicoherence, this gives the action
  \begin{equation*}
    S \times R^\times \to S, (u, t) \mapsto \sum_{n \in \Z} u_n t^n
  \end{equation*}
  of $R^\times$ on the $R$-algebra $S$.
\end{remark}

\begin{example}
  Let $V$ be a finitely presented $R$-module.  The symmetric algebra
  $\Sym^* V$ of $V$ over $R$ is naturally graded.  In particular, the
  affine scheme
  \begin{equation*}
    V\spcheck \coloneqq \Spec \Sym^* V
  \end{equation*}
  carries a natural action by the multiplicative group $R^\times$.
\end{example}

\begin{remark}
  We write $V\spcheck$ as
  \begin{equation*}
    \Spec \Sym^* V = \Hom_{\Alg R}(\Sym^* V, R) = \Hom_{\Mod R}(V, R)
  \end{equation*}
  by the universal property of $\Sym^* V$.  In particular, $V\spcheck$
  carries the structure of a (finitely copresented) $R$-module.
  Moreover, the natural action of $R^\times$ on the left hand side
  corresponds to scalar multiplication on the right hand side.
\end{remark}

From the above, we can deduce that $V$ is reflexive:

\begin{theorem}
  Let $V$ be a finitely presented $R$-module.  Set
  \begin{equation*}
    V^{\vee\vee} \coloneqq \Hom_{\Mod R}(V\spcheck, R).
  \end{equation*}
  The natural map
  \begin{equation*}
    V \to V^{\vee\vee}, f \mapsto (p \mapsto p(f))
  \end{equation*}
  is an isomorphism of $R$-modules.  In particular, $V$ is reflexive
  and the dual of every finitely copresented $R$-module (which is
  always the dual of a finitely presented $R$-module) is finitely
  presented.
\end{theorem}

\begin{proof}
  The identification
  \begin{equation*}
    \Sym^* V \to R^{\Spec \Sym^* V} = R^{V\spcheck}
  \end{equation*}
  is an identity of $R$-algebras with an $R^\times$-action.  In
  particular, the homogeneous elements of degree $1$ on the left hand
  side correspond to the homogeneous elements of degree $1$ on the
  right hand side.  This yields an isomorphism
  \begin{equation*}
    \phi\colon V \to \Hom_{R^\times}(V\spcheck, R) \coloneqq \{u\colon V\spcheck \to R \mid \forall t \in R^\times\forall v \in V\colon u(vt) = u(v) t\},
    v \mapsto (p \mapsto p(v))
  \end{equation*}
  of $R$-modules.  As the image of $\phi$ lies inside
  $V^{\vee\vee} \subseteq \Hom_{R^\times}(V\spcheck, R)$ we
  actually have $V^{\vee\vee} = \Hom_{R^\times}(V\spcheck, R)$ and
  the theorem is proven.
\end{proof}

\begin{remark}
  It follows from the above that for a finitely copresented $R$-module
  $V\spcheck$, a ($1$-)homogenous map $V\spcheck \to R$ is already
  $R$-linear (principle of microlinearity).
\end{remark}

%%% Local Variables:
%%% mode: latex
%%% TeX-master: "main"
%%% End:
