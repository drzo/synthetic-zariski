We follow the notations and setting for Synthetic Algebraic Geometry \cite{draft}.
In particular, $R$ denotes the generic local ring and $\Gm$ is the multiplicative group of units of $R$.

In Synthetic Algebraic Geometry, a scheme is defined as a set satisfying some property \cite{draft}. In particular
the projective space $\bP^n$ can be defined to be the quotient of $R^{n+1}\setminus\{0\}$ by the
equivalence relation $a\sim b$ which expresses that $a$ and $b$ are proportional, %i.e. $a_ib_j=a_jb_i$,
which is equal to $\Sigma_{r:\Gm}ar = b$. We can then prove \cite{draft}
that this set is a scheme. This definition goes back to \cite{Kock74}.

 In this setting, a map of schemes is simply an arbitrary set theoretic map. An application of this work is to show
 that the maps $\bP^n\rightarrow \bP^m$ are given by $m+1$ homogeneous polynomials of the same degree in $n+1$ variables.

\medskip


There is another definition of $\bP^n$ which uses ``higher'' notions. Let $B\Gm$ be the delooping
of $\Gm$. It can be defined as the type of lines $\Sigma_{M:\Mod{R}}\|{M=R^1}\|$. Over $B\Gm$ we have the
family of sets
$$T_n(l) = l^{n+1}\setminus\{0\}$$
Note that we use the same notation for an element $l : B\Gm$,
its underlying $R$-module and its underlying set.
An equivalent definition of $\bP^n$ is then
$$
\bP^n = \sum_{l:B\Gm}T_n(l)
$$
That is, we replaced the quotient operation, here a set of orbits for a free group action, by a sum type over the delooping of this group
\cite{Sym}.

The standard line bundles (or twisting sheaves) on $\bP^n$ can then be constructed as follows.
We define $l^{\otimes n}:B\Gm$ for $l:B\Gm$ by
$l^{\otimes 0} = R^1$,
$l^{\otimes (n+1)} = l^{\otimes n}\otimes l$ for $n \geqslant 0$
and $l^{\otimes (n-1)} = l^{\otimes n}\otimes l^{\vee}$ for $n \leqslant 0$.
Then we define the line bundle $O(d):\bP^n\rightarrow B\Gm$ by $O(d)(l,s) = l^{\otimes d}$.

\medskip

 Connected to this definition of $\bP^n$, we can note the following equalities.

\begin{proposition}\label{end}
  $$\prod_{l:B\Gm}l^n\rightarrow l \;\;\;=\;\;\; \Hom(R^n,R)$$
\end{proposition}

\begin{proof}
We rewrite $\Hom(R^n,R)$, the set of $R$-module morphism, as
$$
\sum_{\alpha:R^n\rightarrow R}\prod_{\lambda:\Gm}\prod_{x:R^n}\alpha(\lambda x) = \lambda \alpha(x)
$$
This follows from the axioms of Synthetic Algebraic Geometry: the map $\alpha$ has to be a polynomial. And a polynomial
satisfying $\alpha(\lambda x) = \lambda \alpha(x)$ for $\lambda$ in $R^{\times}$ has to be linear.

\medskip

It is then a general fact that if we have a pointed connected groupoid $(A,a)$ and a family of
sets $T(x)$ for $x:A$, then $\prod_{x:A}T(x)$ is the set of fixedpoints of $T(a)$ for the $(a=a)$ action
\cite{Sym}.
\end{proof}

We will use the following remark, proved in \cite{draft}.

\begin{lemma}\label{ext}
  Any map $R^{n+1}\setminus\{0\}\rightarrow R$ can be uniquely extended to a map $R^{n+1}\rightarrow R$ for $n>0$.
\end{lemma}

We will also use the following proposition, already noticed in \cite{draft}.

\begin{proposition}\label{const}
 Any map from $\bP^n$ to $R$ is constant.
\end{proposition}

\begin{proof}
  Since $\bP^n$ is a quotient of $R^{n+1}\setminus\{0\}$, the set $\bP^n\rightarrow R$ is
  the set of maps $\alpha:R^{n+1}\setminus\{0\}\rightarrow R$
  such that $\alpha(\lambda x) = \alpha(x)$ for all $\lambda$ in $\Gm$. These are exactly the constant maps
  using Lemma \ref{ext} \cite{draft}.
\end{proof}

\begin{proposition}\label{aut}
$$\prod_{l:B\Gm}T_n(l)\rightarrow T_n(l) \;\;=\;\; GL_{n+1}$$
\end{proposition}

\begin{proof}
  This follows from two remarks.

  The first remark is that it follows from Lemma \ref{ext} that the restriction map
$$
(l^{n+1}\rightarrow l)\rightarrow ((l^{n+1}\setminus\{0\})\rightarrow l)
$$
is a bijection for $n>0$ if $l:B\Gm$.

\medskip

The second remark is that a linear map $u:R^{m}\rightarrow R^{m}$ such that
$$
x\neq 0~\rightarrow~u(x)\neq 0
$$
is exactly an element of $GL_{m}$.

We show this by induction on $m$. For $m=1$ we have $u(1)\neq 0$ iff $u(1)$ invertible.

For $m>1$, we look at $u(e_1) = \Sigma \alpha_ie_i$ with $e_1,\dots,e_m$ basis of $R^m$.
We have one $\alpha_i$ invertible. By composing $u$ with an element in $GL_m$, we can then
assume that $u(e_1) = e_1+v_1$ and $u(e_i) = v_i$, for $i>1$, with $v_1,\dots,v_m$ in $Re_2+\dots+Re_m$.
We can then conclude by induction.
\end{proof}

We can generalize \Cref{end}
and get a result related to \Cref{aut} as follows.
 
\begin{lemma}\label{hom}
  \begin{enumerate}[(i)]
    \item
      \[  \prod_{l:B\Gm}l^n\rightarrow l^{\otimes d} \;\;=\;\; (R[X_1, \dots, X_n])_d \]
      That is,
      every element of the left-hand side is given by
      a unique homogeneous polynomial of degree $d$ in $n$ variables.
    \item
      An element in
      $$\prod_{l:B\Gm}T_n(l)\rightarrow T_m(l^{\otimes d})$$
      is given by $m+1$ homogeneous polynomials $p = (p_0,\dots,p_m)$ of degree $d$ such that
      $x\neq 0$ implies $p(x)\neq 0$.
  \end{enumerate}
\end{lemma}

\begin{proof}
We show the first item. Following \cite{Sym}, this product is the set of maps $\alpha:R^n\rightarrow R^{\otimes d}$
which are invariant by the $\Gm$-action $(\alpha\cdot r)(x) = r^d\alpha(r^{-1} x)$. In the setting of \cite{draft},
these are exactly the maps given by homogeneous polynomials of degree $d$.
\end{proof}
