We follow the notations and setting for Synthetic Algebraic Geometry \cite{draft}.
In particular, $R$ denotes the generic
local ring and $GL_1$ is the multiplicative group of unit of $R$.

In Synthetic Algebraic Geometry, a scheme is defined as a set satisfying some property \cite{draft}. In particular
the projective space $\bP^n$ can be defined to be the quotient of $R^{n+1}-\{0\}$ by the
equivalence relation $a\equiv b$ which expresses that $a$ and $b$ are proportional, i.e.
$a_ib_j=a_jb_i$, which is equal to $\Sigma_{r:GL_1}ar = b$. We can then prove \cite{draft}
that this set is a scheme.

There is another definition of $\bP^n$ which uses ``higher'' notions. Let $BGL_1$ be the delooping
of $GL_1$, which can be defined as the type of line bundle $\Sigma_{M:R-Mod}\|{M=R^1}\|$.
An equivalent definition of $\bP^n$ is then \cite{Sym}
$$
\bP^n = \Sigma_{l,p:BGL_1}l^{n+1}-\{0\}
$$
That is, we replaced the quotient operation by a sum type.

\medskip

Connected to this definition, we can note the following equalities.

\begin{proposition}\label{end}
  $$\prod_{(l,p):BGL_1}l^n\rightarrow l ~~=~~ Hom(R^n,R)$$
\end{proposition}

\begin{proof}
We rewrite $Hom(R^n,R)$, the set of $R$-module morphism, as
$$
\Sigma_{\alpha:R^n\rightarrow R}\prod_{\lambda:GL_1}\prod_{x:R^n}\alpha(\lambda x) = \lambda \alpha(x)
$$
This follows from the axioms of Synthetic Algebraic Geometry: the map $\alpha$ has to be a polynomial. And a polynomial
satisfying $\alpha(\lambda x) = \lambda \alpha(x)$ for $\lambda$ in $R^{\times}$ has to be linear.

\medskip

It is then a general fact that if we have a pointed connected groupoid $A,a$ and a family of
sets $T(x)$ for $x:A$, then $\prod_{x:A}T(x)$ is the set of fixedpoints of $T(a)$ for the $(a=a)$ action
\cite{Sym}.
\end{proof}

We will use the following remark, proved in \cite{draft}.

\begin{lemma}\label{ext}
  Any map $R^{n+1}-\{0\}\rightarrow R$ can be uniquely extended to a map $R^{n+1}\rightarrow R$ for $n>0$.
\end{lemma}

A similar reasoning proves the following.

\begin{proposition}\label{const}
 Any map from $\bP^n$ to $R$ is constant.
\end{proposition}

\begin{proof}
  $\bP^n\rightarrow R$ is equivalent to
  $$\prod_{(l,p):BGL_1}l^{n+1}-\{0\}\rightarrow R$$
  and, using \cite{Sym}, this is the set of maps $\alpha:R^{n+1}-\{0\}\rightarrow R$
  such that $\alpha(\lambda x) = \alpha(x)$ for all $\lambda$ in $GL_1$. These are exactly the constant maps
  using Lemma \ref{ext} \cite{draft}.
\end{proof}

\begin{proposition}\label{aut}
$$\prod_{(l,p):BGL_1}(l^{n+1}-\{0\})\rightarrow (l^{n+1}-\{0\}) ~~=~~ GL_{n+1}$$
\end{proposition}

\begin{proof}
  This follows from two remarks.

  The first remark is that it follows from Lemma \ref{ext} that the restriction map
$$
(l^{n+1}\rightarrow l)\rightarrow ((l^{n+1}-\{0\})\rightarrow l)
$$
is a bijection for $n>0$ if $(l,p):BGL_1$.

\medskip

The second remark is that a linear map $u:R^{m}\rightarrow R^{m}$ such that
$$
x\neq 0~\rightarrow~u(x)\neq 0
$$
is exactly an element of $GL_{m}$.

We show this by induction on $m$. For $m=1$ we have $u(1)\neq 0$ iff $u(1)$ invertible.

For $m>1$, we look at $u(e1) = \Sigma \alpha_ie_i$ with $e_1,\dots,e_m$ basis of $R^m$.
We have one $\alpha_i$ invertible. By composing $u$ with an element in $GL_m$, we can then
assume that $u(e_1) = e_1+v_1$ and $u(e_i) = v_i$, for $i>1$, with $v_1,\dots,v_m$ in $Re_2+\dots+Re_m$.
We can then conclude by induction.
\end{proof}

We define $l^{\otimes n}:BGL_1$ for $l:BGL_1$ by $l^{\otimes 0} = R^1$ and $l^{\otimes (n+1)} = l^{\otimes n}\otimes l$.

We define the line bundle $O(n):\bP^n\rightarrow BGL_1$ by $O(n)(l,s) = l^{\otimes (-n)}$ if $n\geq 0$ and 
and $O(n)(l,s) = O(-n)(l,s)$ if $n\geq 0$.

\medskip

 A similar argument as the one used for Proposition \ref{aut} gives also the following result.
 
 \begin{lemma}\label{hom}
   An element in
  $$\prod_{(l,p):BGL_1}(l^{n+1}-\{0\})\rightarrow ((l^{\otimes d})^{m+1}-\{0\})$$
  is given by $m+1$ homogeneous polynomials $p = (p_0,\dots,p_m)$ of degree $d$ such that
  $x\neq 0$ implies $p(x)\neq 0$.
\end{lemma}
