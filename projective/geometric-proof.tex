A geometric property of $\bP^n$:

\begin{lemma}\label{constant-functions-Pn-minus-points}
  Let $n>1$ and $p\neq q$ be points of $\bP^n$, then all functions
  \begin{center}
  \begin{enumerate}[(i)]
  \item $\bP^n\setminus\{p\}\to \Z$
  \item $\bP^n\setminus\{p,q\}\to \Z$
  \item $\bP^n\setminus\{p\}\to R$
  \item $\bP^n\setminus\{p,q\}\to R$
  \end{enumerate}
  \end{center}
  are constant.
\end{lemma}

\begin{proof}
  We start with (iv).
  Let $f:\bP^n\setminus\{p,q\}\to R$.
  For the charts $U_0=\{[x_0:\dots:x_n]\mid x_0\neq 0\}$ and $U_1=\{[x_0:\dots:x_n]\mid x_0\neq 0\}$, we can assume $p\in U_0, p\notin U_1$ and $q\in U_1, q\notin U_0$.
  Then $f_{|U_0\setminus\{p\}}$ can be extended to $U_0$ by \Cref{ext} and an analogous extension exisits on $U_1$.
  These extensions glue with $f$ to a function $\widetilde{f}:\bP^n\to R$ which agrees with $f$ on $\bP^n\setminus\{p,q\}$.
  By \Cref{const}, $\widetilde{f}$ is constant and therefore $f$ is constant.
  This carries over to functions $\bP^n\setminus\{p,q\}\to \mathrm{Bool}$ since $\mathrm{Bool}\subseteq R$ and thus also to any $\bP^n\setminus\{p,q\}\to \Z$,
  which shows (ii).
  (i) and (iii) follow from (ii) and (iv).
\end{proof}
We proceed by extendending \Cref{Matthias} to subspaces of $\bP^n$ which can be constructed like $\bP^1$:

\begin{lemma}\label{line-bundle-on-line}
  Let $M\subseteq R^{n+1}$ be a submodule with $\|M=R^2\|$.
  Then $\Gr(1,M)\subseteq \bP^n$ and the map
  \begin{align*}
    \Z\times \KR &\to (\Gr(1,M)\to \KR) &\\
    (d,l_0) &\mapsto (L\mapsto L^{\otimes d}) &\text{ for $d\geq 0$} \\
    (d,l_0) &\mapsto (L\mapsto \Hom_{\Mod{R}}(L^{\otimes d},R)) &\text{ for $d< 0$} 
  \end{align*}
  is an equivalence.
\end{lemma}

\begin{proof}
  We prove a proposition, so we have an $R$-linear isomorphism $\phi:R^2\to M$ and for each $d:\Z$, we get a commutative triangle:
  \begin{center}
  \begin{tikzcd}
    \Gr(1,M)\ar[rr,"\OO(d)"] && \KR \\
    & \Gr(1,R^2)\ar[lu,"\phi"]\ar[ur,swap,"\OO(d)"] &
  \end{tikzcd}
\end{center}
by restricting $\phi$ to each line in $\Gr(1,R^2)$.
This shows that the map from \Cref{Matthias} and from the statement are equal as maps to $(V:\Mod{R})\times \|V=R^2\| \times (V\to\KR))$,
which proves the claim.
\end{proof}

\begin{theorem}
  The map
  \begin{align*}
    \Z\times \KR &\to (\bP^n\to \KR) \\
    (d,l_0) &\mapsto (x\mapsto l_0\otimes \OO(d)(x))
  \end{align*}
  is an equivalence.
\end{theorem}

\begin{proof}
  It is enough to show that the map is surjective, by the same reasoning as in the proof of \Cref{Matthias}.
  Let $L:\bP^n\to \KR$. First we determine the degree of $L$.
  Let $p\neq q$ be points in $\bP^n$ and $M\subseteq R^{n+1}$ be the span of $p$ and $q$ as submodules of $R^{n+1}$.
  Then $\|M=R^2\|$ and we can use the inverse $i$ of the map in \Cref{line-bundle-on-line} to define $d\colonequiv \pi_1(i(L_{|\Gr(1,M)}))$.
  The integer $d$ is independent of the choice of $p$ and $q$:
  If we let $p$ vary, we get a function of type $\bP^n\setminus\{q\}\to \Z$ which is constant by \Cref{constant-functions-Pn-minus-points}.
  The same applies for $q$
  and the two subsets $\bP^n\setminus\{p\}$ and $\bP^n\setminus\{q\}$ cover $\bP^n$.

  In the following we consider only $L$ such that $d$ as constructed above is $0$.
  This means that on each line $\Gr(1,M)$, $L$ will be constant.
  So for $p,x:\bP^n$,
  and $x\neq p$ we can construct an equality $P_x:L(x)=L(p)$  by restricting $L$ to $\Gr(1,\langle x,p\rangle)$ and applying \Cref{line-bundle-on-line}.
  So we have $P:(x:\bP^n\setminus\{p\})\to L(x)=L(p)$ and for $q\neq p$ we can construct
  $Q:(y:\bP^n\setminus\{q\})\to L(y)=L(q)$ analogously.
  
  The claim follows if we show that $L$ is constant on all of $\bP^n$.
  Since, overall, we show the proposition that the map from the statement merely has a preimage,
  we can assume $a:L(p)=R^1$ and $b:L(q)=R^1$ and get:
  \[ \left((x:\bP^n\setminus\{p,q\})\mapsto a^{-1}P_x^{-1}Q_xb\right) : \bP^n\setminus\{p,q\} \to R^\times\]
  which is constantly $\lambda$ by \Cref{constant-functions-Pn-minus-points}.
  So $P$ and $Q$ can be corrected using $\lambda,a$ and $b$ to yield a global proof of constancy of $L$.
\end{proof}