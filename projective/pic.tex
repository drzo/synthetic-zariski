 Let $A$ be a commutative {\em connected} ring\footnote{If $e(1-e) = 0$ then $e=0$ or $e=1$.}.

The following result is standard.

 \begin{lemma}\label{stand}
   An invertible element of $A[X,1/X]$ can be written $X^N\Sigma a_nX^n$ with $N$ in $\Z$
   and $a_0$ unit and $a_n$ nilpotent if $n\neq 0$.
 \end{lemma}
 
 Using this Lemma we deduce the following.

\begin{lemma}\label{nilpotent}
  Any invertible element of $A[X,1/X]$ can be written uniquely as a product
  $uX^l(1+a)(1+b)$ with $l$ in $\Z$, $u$ in $A^{\times}$ and $a$ (resp. $b$)
  polynomial in $A[X]$ (resp. $1/XA[1/X]$) with only nilpotent coefficients.
\end{lemma}

\begin{proof}
  Write $\Sigma v_nX^n$ the invertible element of $A[X,1/X]$.
  W.l.o.g. we can assume that the polynomial is of the form $1 + \Sigma v_nX^n$ with
  all $v_n,~n\in \Z$ nilpotent.
  We let $J$ be the ideal generated by these nilpotent elements.
  We have some $N$ such that $J^N = 0$.
  
  We first multiply by the inverse of $1 + \Sigma_{n<0}v_nX^n$, making all coefficients of
  $X^n,~n<0$ in $J^2$.
  We keep doing this until all these elements are $0$.
  %We then do the same, killing all coefficients of $X^n$ for $n>0$.
  We have then written the invertible polynomials on the form $(1+a)(1+b)$.

  Such a decomposition is unique: if we have $(1+a)(1+b)$ in $A^{\times}$ with $a = \Sigma_{n\geqslant 0}a_nX^n$
  and $b = \Sigma_{n<0}b_nX^n$ then we have $a_n = 0$ for $n>0$ and $b_n = 0$ for $n<0$.
\end{proof}

\begin{corollary}\label{Pic1}
  We have $\prod_{L:\bP^1\rightarrow B\Gm}\Sigma_{p:\Z}\|L = O(p)\|$
\end{corollary}

\begin{proof}
A line bundle $L([x_0,x_1])$ on $\bP^1$ is trivial on each affine charts $x_0\neq 0$ and $x_1\neq 0$ by Corollary \ref{c1}, so
it is characterised by an invertible Laurent polynomial on $R$, and the result follows from Lemma \ref{nilpotent}.
\end{proof}

We can then state the following strengthening.

\begin{proposition}\label{Matthias}
  The map $B\Gm\times\Z\rightarrow (\bP^1\rightarrow B\Gm)$
  which associates to $l_0,d$ the map $x\mapsto l_0\otimes O(d)(x)$ is an equivalence.
\end{proposition}

\begin{proof}
  Corollary \ref{Pic1} shows that this map is essentially surjective.
  We can then use  Corollary 8.8.2 in \cite{HoTT} to conclude using Proposition \ref{const}.
\end{proof}

 It is a curious remark that $B\Gm\rightarrow B\Gm$ is also equivalent
 to $B\Gm\times \Hom(\Gm,\Gm) = B\Gm\times\Z$.

\begin{corollary}\label{Matthias1}
  We have $\prod_{L:\bP^1\rightarrow B\Gm}\prod_{x:R}L([1:x]) = L([0:1])$.
\end{corollary}

\section{Line bundles on $\bP^n$}

We can now reformulate Quillen argument for Theorem 2' \cite{Quillen} in our setting.

\begin{proposition}\label{trivial}
  If we have $V:\bP^n\rightarrow B\Gm$ we have ${\prod_{s:R^n}V([1:s]) = V([1:0])}$.
\end{proposition}

\begin{proof}
  We define $L:\A^{n-1}\rightarrow {B\Gm}^{\bP^1}$ by $L~s~[x_0:x_1] = V([x_0:x_1:x_0s])$.
  We apply Corollary \ref{Matthias1} and we get
  $$V([1:s]) = L~s~[1:r] = L~s~[1:1] = L~s~[0:1] = V([0:1:0])$$
\end{proof}

 Note that the use of Corollary \ref{Matthias1} replaces the use of the ``Quillen patching''
 \cite{lombardi-quitte} introduced in \cite{Quillen}.

\medskip

Let $T$ be the ring of polynomials $u = \Sigma_p u(p)X^p$ with
$X^p = X_0^{p_0}\dots X_n^{p_n}$ with $\Sigma p_i = 0$. We write $T_l$ the subring
of $u$ which contains only monomials $X^p$ with $p_i\geqslant 0$ if $i\neq l$
and $T_{lm}$ the subring of $u$ 
which contains only monomials $X^p$ with $p_i\geqslant 0$ if $i\neq l$ and $i\neq m$.

Note that $T_l$ is the polynomial ring $T_l = R[X_0/X_l,\dots,X_n/X_l]$.

A line bundle on $\bP^n$ is given by compatible line bundles on each $Sp(T_l)$.

By Proposition \ref{trivial}, a line bundle on $\bP^n$ is trivial on each $Sp(T_l)$.
So it is determined by $t_{ij}$ invertible in $T_i[X_i/X_j] = T_j[X_j/X_i] = T_{ij}$
such that $t_{ik} = t_{ij}t_{jk}$ and $t_{ii} = 1$. W.l.o.g we can assume, using Lemma \ref{stand}, that
$t_{ij} = (X_i/X_j)^N u_{ij}$, for some fixed $N$, where $t_{ij}(p)$ is invertible for $p = 0$
and all other coefficients $t_{ij}(p)$ for $p\neq 0$
are nilpotent. The result $\Pic(\bP^n) = \Z$ will then follow from the following result.

\begin{proposition}
  There exists $s_i$ invertible in $T_i$ such that $u_{ij} = s_i/s_j$ 
\end{proposition}

\begin{proof}
%  (We follow essentially David's argument.)
  Each $u_{ij}$ is such that $u_{ij}(p)$ unit for $p=0$ and
  all $u_{ij}(p)$ nilpotent for $p\neq 0$.

  Like in the proof Lemma \ref{nilpotent}, we can change $u_{01}$ so that
  we have $u_{01}(p) = 0$ if $p\neq 0$ and $p_0\geqslant 0$ or $p_1\geqslant 0$ by multiplying $u_{01}$ by a unit in $T_0$ and a unit in $T_1$.
  
  We claim then that, in this case, $u_{01}$ has to be a unit. For this we show that $u_{01}(p) = 0$
  if $p_l>0$ for each $l\neq 0,1$.
  This is obtained by looking at the relation $u_{01}= u_{0l}u_{l1}$. Let $L$ is the ideal generated by
  coefficients $u_{0l}(p)$ and $u_{1l}(p)$ with $p_l>0$ and $I$
  the ideal generated by all nilpotent coefficients of $u_{0l}$ and $u_{1l}$.
  Thanks to the form of $u_{01}$ we must have $L\subseteq LI$ and so $L=0$ by Nakayma.

  This implies that all coefficients $u_{01}(p)$ such that $p_l>0$ are $0$.

  Since this holds for each $l>1$ we have that $u_{01}$ is a unit in $R$.

  W.l.o.g. we can assume $u_{01}= 1$. We then have $u_{0l} = u_{1l}$ in $T_{0l}\cap T_{1l} = T_l$
  and we take $s_l = u_{0l} = u_{1l}$.
\end{proof}

\begin{corollary}
  $\Pic(\bP^n) = \Z$.
\end{corollary}

We can then strengthen this result, with the same reasoning as in Proposition \ref{Matthias}.

\begin{theorem}\label{Matthias2}
  The map $B\Gm\times\Z\rightarrow (\bP^n\rightarrow B\Gm)$
  which associates to $l_0,d$ the map $x\mapsto l_0\otimes O(d)(x)$ is an equivalence.
\end{theorem}

We deduce from this a characterisation of the maps $\bP^n\rightarrow\bP^m$.% using Lemma \ref{hom}.

\begin{corollary}\label{map}
  A map $\bP^n\rightarrow\bP^m$ is given by $m+1$ homogeneous polynomials $p = (p_0,\dots,p_m)$ on $R^{n+1}$
  of the same   degree $d$ such that $x\neq 0$ implies $p(x)\neq 0$.
\end{corollary}

\begin{proof}
Write $T_n(l)$ for $l^{n+1}\setminus\{0\}$. We have $\bP^n = \Sigma_{l:B\Gm}T_n(l)$ and so
$$
\bP^n\rightarrow\bP^m = \sum_{s:\bP^n\rightarrow B\Gm}\prod_{x:\bP^n}T_m((s~x).\pi_1)
$$
Using Theorem \ref{Matthias2}, this is equal to
$$
\sum_{l_0:B\Gm}\sum_{d:\Z}\prod_{l:B\Gm}T_n(l)\rightarrow T_m(l_0\otimes l^{\otimes d})
$$
and, as for Lemma \ref{hom}, this is the set of tuple of $m+1$ polynomials in $R[X_0,\dots,X_n]$ homogenenous
of degree $d$, sending $x\neq 0$ to $p(x)\neq 0$, and quotiented by proportionality.
\end{proof}

We deduce the characterisation of $\Aut(\bP^n)$. This is a
remarkable result, since the automorphisms are in this framework only bijections of sets.

\begin{corollary}
  $\Aut(\bP^n)$ is $PGL_{n+1}$.
\end{corollary}

We also have the following application of computation of cohomology groups \cite{draft}.

\begin{corollary}
A function $\bP^n\rightarrow\bP^m$ is constant if $n>m$.
\end{corollary}

\begin{proof}
We proved in \cite{draft} that cohomology groups can be computed as Cech cohomology for any
finite open acyclic covering. It follows from this that $H^n(\bP^n,R)\neq 0$.
By Corollary \ref{map}, a map $\bP^n\rightarrow\bP^m$ is given by $m+1$ non zero polynomials
$p(x) = (p_0(x),\dots,p_m(x))$ homogeneous of the same degree $d\geqslant 0$ and such that $x\neq 0$ implies $p(x)\neq 0$.
This means that $\bP^n$ is covered by $m+1$ open subsets $U_i(x)$ defined by $p_i(x)\neq 0$.
I claim that we should have $d=0$.

 If $q(x)$ is a non zero homogeneous polynomial of degree $d>0$, the open $q(x)\neq 0$ defines an {\em affine}
 and hence acyclic \cite{draft}, open subset of $\bP^n$ (see e.g. Exercise 3.5 in \cite{Hartshorne}).
 It follows that the covering $U_0,\dots,U_m$ is acyclic if $d>0$. But this contradicts $H^n(\bP^n,R)\neq 0$.

 Hence $d=0$ and the map is constant.
\end{proof}
