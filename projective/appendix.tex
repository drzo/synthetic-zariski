\section*{Appendix 1: Horrock's Theorem}

We present an alternative constructive proof of the the following special case of Horrocks Theorem \cite{Lam,lombardi-quitte},
for a commutative ring $A$. 

\begin{lemma}\label{Horrocks}
  If an ideal of $A[X]$ divides a principal ideal $(f)$ with $f$ monic then it is itself a principal ideal.
\end{lemma}

Let $I$ and $J$ be such that $I\cdot J = (f)$. We can then write $f = \Sigma u_iv_i$ with $u_i$ in $I$ and
$v_i$ in $J$. We then have $I = (u_1,\dots,u_n)$ and $J=(v_1,\dots,v_n)$.
The strategy of the proof is to build comaximal monoids $S_1,\dots,S_l$ in $A$ \cite{lombardi-quitte} such
that $I$ is generated by a monic polynomial in each $A_{S_j}[X]$.

\subsection{Formal computation of gcd}

 We start by describing a general technique introduced in \cite{lombardi-quitte}.

If we have a list $u_1,\dots,u_n$ of polynomials over a field we can compute the gcd
so that $(g) = (u_1,\dots,u_n)$ and $g$ is $0$ or a monic polynomial.

In general if we are now over a ring $R$, we can interpret this computation formally as
follows. We build a binary tree of root $R$. At each node of the tree we have a f.p. extension $A$
of $R$. If we want to decide whether an element $a$ in $R$ is invertible or $0$
\footnote{A priori it is an element
of $A$, but we can always assume that this element comes from an element of $R$.}
we open two branches: one with $A\rightarrow A/(a)$ (intuitively we force $a$ to be $0$)
and the other with $A\rightarrow A_a = A[1/a]$ (intuitively we force $a$ to be invertible).

In this way we have at each leaf a f.p. extension $R\rightarrow A$ and in $A$ we have
$g$, a monic polynomial in $A[X]$ or $0$, such that $(g) = (u_1,\dots,u_n)$ in $A[X]$.
Over each branch we have a list of elements $a_1,\dots,a_n$ of $R$ that we force to be
invertible, and a list of elements $b_1,\dots,b_m$ of $R$ that we force to be $0$.
We associate to this branch the multiplicative monoid generated by $a_1\dots a_n$
and $1 + (b_1,\dots,b_m)$. In this way, we build a list of monoids $S_1,\dots,S_l$
that are {\em comaximal} \cite{lombardi-quitte}: if $s_i$ in $S_i$ then $1 = (s_1,\dots,s_l)$.

\subsection{Application to Horrocks' Theorem}

We assume $f = \Sigma u_iv_i$ and $fp_{ij} = u_iv_j$ with $\Sigma p_{ii} = 1$
in $A[X]$. The goal is to build comaximal monoids $S_1,\dots,S_l$ with $I$ generated by
a monic polynomial in $A_{S_j}[X]$.

We first build a binary tree which corresponds to the formal computation of the gcd of
$u_1,\dots,u_n$ as described above. To each branch we associate an element that
we force to be invertible and a list of elements $b_1,\dots,b_m$ that we force to be $0$.
We write $S$ for the multiplicative monoid generated by $a$ and $1 + (b_1,\dots,b_m)$.
We also have a monic polynomial $\gamma$ in $A_S[X]$ 
such that $I = (\gamma)$ in $A_S[X]/(b_1,\dots,b_m)$.

 Note that $I = (u_1,\dots,u_n)$ contains $f$.

\begin{lemma}
  If $p$ is a polynomial in $I$ which is monic in $A_S[X]/(b_1,\dots,b_m)$ of degree $<deg(f)$
  then there exists $h$ monic in
  $A_S[X]$ and in $(u_1,\dots,u_n)$ and such that $p=h$ mod $(b_1,\dots,b_m)$.
\end{lemma}

\begin{proof}
  (Same proof as in Lam \cite{Lam}.) Let $N$ be the degree of $f$.
  If $I$ also contains a polynomial $q$ which is monic
  mod. $L$ of degree $N-1$, we can kill all coefficients (in $L$) of degree $\geqslant N$
  using $f$, and we get that $I$ also contains a monic polynomial of degree $N-1$
  and equal to $q$ mod. $L$.
  Similarly $I$ will also contain a monic polynomial of degree $N-2$, and so on, until
  we get $h$ monic in $(u_1,\dots,u_n)$ and equal to $p$ mod. $L$.
\end{proof}

By this Lemma, we get a monic polynomial $h$ in $(u_1,\dots,u_n)$ in $A_S[X]$
and such that $I=(h)$ in $A_S[X]/(b_1,\dots,b_m)$.

\begin{lemma}
 $I = (h)$ in $A_{S}[X]$.
\end{lemma}

\begin{proof}
  Let $L$ be $(b_1,\dots,b_m)$ in $A_S[X]$.
  Since $I$ contains $I\cap L$ and $I\cdot J = (f)$ with $f$ regular, we can find $K$
  such that $I\cdot K = I\cap L$.
  We then have $I\cdot K = 0$ mod. $L$ and hence $K = 0$ mod. $L$ since $I$ contains $f$
  which is monic.
  This means $I\cap L = I\cdot L$. Then we have $I = (h) + I\cdot L$.
  The result then follows from the fact that $h$ is monic and from Nakayama, as in Lam \cite{Lam}:
  the module $M = I/(h)$ is a finitely generated module over $A_S$ and satisfies
  $M\subseteq ML$.
\end{proof}

\begin{corollary}
  We can find comaximal elements $s_1,\dots,s_l$ such that $I$ is principal and generated by a
  monic polynomial in each $A_{s_j}[X]$. Since these monic polynomials are uniquely determined
  we can patch these generators and get that $I$ is principal in $A[X]$\footnote{If $A$ is not
  connected, the generator of $I$ may not be monic: if $e(1-e)=1$ then the ideal $(eX+(1-e))$
  divides the ideal $(X)$.}.
\end{corollary}


\section*{Appendix 2: Quillen Patching}

We reproduce the argument in Quillen's paper \cite{Quillen}, as simplified in \cite{lombardi-quitte}.
This technique of Quillen Patching has been replaced by the equivalence in Proposition \ref{Matthias}.

If $P$ and $Q$ are two idempotent matrix of the same size, let us write $P\simeq Q$ for expressing that $P$ and $Q$ presents
the same projective module (which means that there are similar, which is in this case is the same as being equivalent).

If we have a projective module on $A[X]$, presented by a matrix $P(X)$, this module is extended
precisely when we have $P(X)\simeq P(0)$.

\begin{lemma}
  If $S$ is a multiplicative monoid of $A$ and $P(X)\simeq P(0)$ on $A_S[X]$ then there exists
  $s$ in $S$ such that $P(X+sY)\simeq P(X)$ in $A[X]$.
\end{lemma}

\begin{lemma}
  The set of $s$ in $A$ such that $P(X+sY)\simeq P(X)$ is an ideal of $A$.
\end{lemma}

\begin{corollary}
  If we have $M$ projective module of $A[X]$ and $S_1,\dots,S_n$ comaximal multiplicative monoids of $A$
  such that each $M\otimes_{A[X]} A_{S_i}[X]$ is extended from $A_{S_i}$ then $M$ is extended from $A$.
\end{corollary}

Let us reformulate in synthetic term this result. Let $A$ be a f.p. $R$-algebra and $L:\Spec(A)\rightarrow B\Gm^{\A^1}$.
Then $L$ corresponds to a projective module of rank $1$ on $A[X]$. We can form
$$T(x) = \prod_{r:R}L~x~r = L~x~0$$
and $\|T(x)\|$ expresses that $L~x$ defines a trivial line bundle on $\A^1 = \Spec(R[X])$.
It is extended exactly when we have
$\|{\prod_{x:\Spec(A)}T(x)}\|$. We can then use Zariski local choice to state.

\begin{proposition}\label{c2}
  We have the implication $(\prod_{x:\Spec(A)}\|T(x)\|)\rightarrow \|\prod_{x:\Spec(A)}T(x)\|$.
\end{proposition}

\newpage

\section*{Appendix 3: Classical argument}

We reproduce a message of Brian Conrad in MathOverflow \cite{conrad-mathoverflow-16324}.

\medskip

``We know that the Picard group of projective $(n-1)$-space over a field $k$ is $\Z$
generated by $\OO(1)$.
This underlies the proof that the automorphism group of such a projective space is $\PGL_n(k)$.
But what is the automorphism group of $\bP^{n-1}(A)$ for a general ring $A$? Is it $\PGL_n(A)$?
It's a really important fact that the answer is yes.
But how to prove it? It's a shame that this isn't done in Hartshorne.

By an elementary localization, we may assume $A$ is local.
In this case we claim that $\Pic(\bP^{n-1}(A))$ is infinite cyclic generated by $\OO(1)$.
Since this line bundle has the known $A$-module of global sections,
it would give the desired result if true by the same argument as in the field case.
And since we know the Picard group over the residue field, we can twist
to get to the case when the line bundle is trivial on the special fiber. How to do it?

\medskip

 Step 0: The case when $A$ is a field. Done.

 \medskip

 Step 1: The case when $A$ is Artin local.
 This goes via induction on the length, the case of length $0$ being Step $0$
 and the induction resting on cohomological results for projective space over the residue field.

  \medskip

 Step 2: The case when $A$ is complete local noetherian ring. This goes
 using Step 1 and the theorem on formal functions (formal schemes in disguise).

  \medskip

 Step 3: The case when $A$ is local noetherian.
 This is faithfully flat descent from Step 2 applied over $A~\widehat{}$

 \medskip
 
 Step 4: The case when $A$ is local:
 descent from the noetherian local case in Step 3 via direct limit arguments.

\medskip
 
QED''
