\section*{Appendix 1: Quillen Patching}

We reproduce the argument in Quillen's paper \cite{Quillen}, as simplified in \cite{lombardi-quitte}.
This technique of Quillen Patching has been replaced by the equivalence in Proposition \ref{Matthias}.

If $P$ and $Q$ are two idempotent matrix of the same size, let us write $P\simeq Q$ for expressing that $P$ and $Q$ presents
the same projective module (which means that there are similar, which is in this case is the same as being equivalent).

If we have a projective module on $A[X]$, presented by a matrix $P(X)$, this module is extended
precisely when we have $P(X)\simeq P(0)$.

\begin{lemma}
  If $S$ is a multiplicative monoid of $A$ and $P(X)\simeq P(0)$ on $A_S[X]$ then there exists
  $s$ in $S$ such that $P(X+sY)\simeq P(X)$ in $A[X]$.
\end{lemma}

\begin{lemma}
  The set of $s$ in $A$ such that $P(X+sY)\simeq P(X)$ is an ideal of $A$.
\end{lemma}

\begin{corollary}
  If we have $M$ projective module of $A[X]$ and $S_1,\dots,S_n$ comaximal multiplicative monoids of $A$
  such that each $M\otimes_{A[X]} A_{S_i}[X]$ is extended from $A_{S_i}$ then $M$ is extended from $A$.
\end{corollary}

Let us reformulate in synthetic term this result. Let $A$ be a f.p. $R$-algebra and $L:Sp(A)\rightarrow B\Gm^{\A^1}$.
Then $L$ corresponds to a projective module of rank $1$ on $A[X]$. We can form
$$T(x) = \prod_{r:R}L~x~r = L~x~0$$
and $\|T(x)\|$ expresses that $L~x$ defines a trivial line bundle on $\A^1 = Sp(R[X])$.
It is extended exactly when we have
$\|{\prod_{x:Sp(A)}T(x)}\|$. We can then use Zariski local choice to state.

\begin{proposition}\label{c2}
  We have the implication $(\prod_{x:Sp(A)}\|T(x)\|)\rightarrow \|\prod_{x:Sp(A)}T(x)\|$.
\end{proposition}

\newpage

\section*{Appendix 2: Classical argument}

We reproduce a message of Brian Conrad in mathoverflow \cite{conrad-mathoverflow-16324}.

\medskip

``We know that the Picard group of projective $(n-1)$-space over a field $k$ is $\Z$
generated by $O(1)$.
This underlies the proof that the automorphism group of such a projective space is $PGL_n(k)$.
But what is the automorphism group of $\bP^{n-1}(A)$ for a general ring $A$? Is it $PGL_n(A)$?
It's a really important fact that the answer is yes.
But how to prove it? It's a shame that this isn't done in Hartshorne.

By an elementary localization, we may assume $A$ is local.
In this case we claim that $\Pic(\bP^{n-1}(A))$ is infinite cyclic generated by $O(1)$.
Since this line bundle has the known $A$-module of global sections,
it would give the desired result if true by the same argument as in the field case.
And since we know the Picard group over the residue field, we can twist
to get to the case when the line bundle is trivial on the special fiber. How to do it?

\medskip

 Step 0: The case when $A$ is a field. Done.

 \medskip

 Step 1: The case when $A$ is Artin local.
 This goes via induction on the length, the case of length $0$ being Step $0$
 and the induction resting on cohomological results for projective space over the residue field.

  \medskip

 Step 2: The case when $A$ is complete local noetherian ring. This goes
 using Step 1 and the theorem on formal functions (formal schemes in disguise).

  \medskip

 Step 3: The case when $A$ is local noetherian.
 This is faithfully flat descent from Step 2 applied over $A~\widehat{}$

 \medskip
 
 Step 4: The case when $A$ is local:
 descent from the noetherian local case in Step 3 via direct limit arguments.

\medskip
 
QED''
