A line bundle on a type $X$ is a map $X\rightarrow \KR$.

%For $BGL_1$ we take $\Sigma_{M:\Mod{R}}\|M = R\|$.

\medskip

 A line bundle $L$ on $Sp(A)$ will define a f.p. $A$-module $\prod_{x:Sp(A)}L(x)$ \cite{draft}.
It is presented by a matrix $P$.
Since this f.p. module is locally free, we can find $Q$ such that $PQP = P$ and
$QPQ = Q$ \cite{lombardi-quitte}. We then have $Im(P) = Im(PQ)$ and this is a projective module of rank $1$. We can then assume $P$ square matrix and
$P^2 = P$ and the matrix $I-P$ can  be seen as listing the generators of this module.

If $M$ is a natrix we write $\Delta_l(M)$ for the ideal generated by the $l\times l$ minors of
$M$. We have $\Delta_1(I-P) = 1$ and $\Delta_2(I-P) = 0$, since this projective module is of rank $1$.

The module is free exactly if we can find a column vector $X$ and a line vector $Y$ such that
$XY = I-P$. We then have $YX = 1$, since if $r = YX$ we have $I-P = XYXY = rXY = r(I-P)$ and
hence $r = 1$ since $\Delta_1(I-P) = 1$.
%We then
%$YX = 1$ since if we write $r = YX$ we have $Q^2 = XYXY = rXY = rQ$ and so $(1-r)Q = 0$ which implies $1 = r$ since $Q$
%is of rank $1$.



\medskip


The line bundle on $Sp(A)$ is trivial on $D(f)$ if, and only if, the module $M\otimes A[1/f][X]$ is free, which
is equivalent to the fact that we can find $X$ and $Y$ such that $YX = (f^N)$ and $XY = f^N(I-P)$ for some $N$.

We can then apply Horrocks Theorem \ref{Horrocks} in Synthetic Algebraic Geometry for the ring $R$.

\begin{proposition}
  If we have $L:\A^1\rightarrow \KR$ which is trivial on some $D(f)$ where $f$ in $R[X]$ is monic
  then $L$ is trivial on $\A^1$.
\end{proposition}

\begin{corollary}\label{c1}
  If we have $L:\bP^1\rightarrow \KR$ then we have
  $$\|{\prod_{r:R}L([1:r]) = L([1:0])}\|\,\,\,\,\,\,\,\,\,\,\,\,\,\|{\prod_{r:R}L([r:1]) = L([0:1])}\|$$
\end{corollary}

\begin{proof}
Indeed, by Zariski local choice \cite{draft}, the line bundle $L$ is locally trivial. In particular, it is trivial
in a neighborhood of the point $[0,1]$, which gives us $f$ monic in $R[X]$ such that $L$ restricted to $A^1$
is trivial on $D(f)$ and we can apply the previous Proposition.
\end{proof}


