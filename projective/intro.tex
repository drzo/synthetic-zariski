Grothendieck advocated for a functor of points approach to schemes early in his
project of foundation of algebraic geometry (see the introduction of \cite{EGAIV1}).
In this approach, a scheme is defined as a special kind of (covariant) set valued functor
on the category of finitely presented commutative ring. This functor should in particular
be a sheaf w.r.t. Zariski topology. As a typical example, the projective space $\bP^n$
is the functor, which to a ring $A$ associates the set of finitely presented factor direct
sub-module of $A^{n+1}$ \cite{Demazure,Eisenbud,Jantzen}.

In the 70s, Anders Kock suggested to use the language of higher-order logic \cite{Church40}
to describe the Zariski topos, the collection of sheaves for Zariski topology \cite{Kock74,kockreyes}.
This allows for
a more suggestive and geometrical description of schemes, that has now seen as special kind
of types satisfying some properties and a map of schemes in this setting is just any map.
There is in particular a ``generic
local ring'' $R$, which associates to $A$ its underlying set. As described in \cite{kockreyes}
the projective space $\bP^n$ is then the set of lines in $R^{n+1}$.

A natural question is we can we show in this setting that the automorphism group of $\bP^n$
is  $PGL_{n+1}(R)$?
More generally, can we show that any map $\bP^n\rightarrow \bP^m$ is given by $m+1$ homogeneous
polynomials of same degree in $n+1$ variables?
It is possible from this to deduce the corresponding result about $\bP^n$ as defined
as functor of points (but the maps are now {\em natural transformations}) or about $\bP^n$ as
defined as a scheme (but the maps are now {\em maps of schemes}).
(This result, though fundamental, is surprisingly not in \cite{Hartshorne}.)
One goal of this paper is to present such a proof.

In \cite{draft}, we presented an axiomatisation of the Zariski {\em higher topos} \cite{lurie-htt},
using instead of the language of higher-order logic the language of dependent type theory
with univalence \cite{hott}. The first axiom is that we have a local ring $R$. We define
then an affine scheme to be a type of the form $Sp(A) = Hom_{R-alg}(A,R)$ for some finitely presented
$R$-algebra $A$. The second axiom, inspired from the work of Ingo Blechschmidt \cite{ingo-thesis},
states that the evaluation map $A\rightarrow R^{Sp(A)}$ is a bijection. The last axiom states
that each $Sp(A)$ satisfies some form of local choice \cite{draft}. We can then define a notion
of {\em open} proposition, with the corresponding notion of open subset, and define a scheme as a type
covered by a finite number of open subsets that are affine schemes. We can then in particular define
$\bP^n$ as in \cite{kockreyes} and show that it is a scheme.
We show in this setting, dependent type theory with univalence extended with these 3 axioms,
the above result about maps between $\bP^n$ and $\bP^m$ and the result about automorphisms of $\bP^n$.

Interestingly, though these results are
about the Zariski $1$-topos, the proof makes use of types that
are not (homotopy) sets (in the sense of \cite{hott}),
since it proceeds in characterizing $\bP^n\rightarrow\KR$, where $\KR$ is the delooping
(thus a type which is not a set) of the multiplicative group of units of $R$.
More technically, we also use such higher types as an alternative of the technique
of Quillen patching \cite{Quillen,lombardi-quitte,lam}.



%% Schemes as special kind of sheaf for Zariski topos.

%% Even nicer in a type theoretic framework

%% Anders Kock property of Zariski topos.

%% Zariski topos higher logic

%% Definition of $\bP^n$ as a set of lines in $R^{n+1}$ coincides with the definition
%% of projective as functor of points (Demazure? Eisenbud?)

%% ``Geometric'' definition

%% Meyers, Blechschmidt use of type theory with univalence

%% Axiomatisation of the Zariski (higher) topos

%% A scheme is defined as a type satisfying some property and a map of schemes is {\em any} function
%% between the corresponding types


