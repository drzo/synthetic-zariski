\begin{definition}
  A sequence is a diagram of shape $\mathbb N_{\leq}$.
\end{definition} 

\begin{lemma}
  $B$ is a countably presented Boolean algebra iff 
  it merely is the colimit of a sequence of finitely presented Boolean algebras.% $(B_n)_{n\in\mathbb N}$.
\end{lemma} 
%Alternatively, we can define a countably presented Boolean algebra as an inductive limit of finite Boolean algebras
%that are given by a sequence of finite sets and maps between then.
\begin{proof}
  First, assume a sequence of finitely presented Boolean algebras. 
  We need to show that the colimit is a countably presented Boolean algebra. 
  \begin{itemize}
    \item The set of generators for the colimit is the colimit of the sets of generators. 
    \item The set of relations for the colimit is the union of the sets of relations. 
      After all, any expression $f$ that becomes $0$ somewhere in the sequence will will be coprojected to $0$
      in the colimit. And as any equality that holds in the colimit uses finitely many elements, 
      it must already hold somewhere in the sequence. 
  \end{itemize}
  Note that both colimits over countably many finite sets are countable. 
  Hence the colimit is countably represented. 

  Conversely, given a countably presented Boolean algebra $B$, we need to give a sequence and show it's colimit is $B$. 
  For our sequence, we assume we have an enumeration of the generators of $B$. 
  We let $G_n$ be given by the first $n$ generators. 
  Let $R_n$ be the relations involving these generators, 
  of which there are only finitely many. 
  We define $B_n = G_n/R_n$, which is a finitely presented Boolean algebra. 
  The embedding of the first $n$ generators into the first $m$ generators gives us 
  a map $B_n \to B_m$ whenever $n\leq m$. 
  Because these morphisms are compatible, this defines a sequence of Boolean algebras. 
  We claim the colimit of this sequence is $B$. 

  Any element in $B$ can be expressed as Boolean combination of finitely many generators, 
  which must occur in some $B_n$, and thus in the colimit. 
  Whenever the images of two elements in the colimit are equal, they are already equal in some $B_m$, 
  hence it follows from a finite subset of the relations for $B$ that the elements are equal, 
  hence the elements are equal in $B$. Thus we have an embedding from $B$ into the colimit. 

  Any element in the colimit already appears in some $B_n$, and hence is 
  a finite expression using generators from $B$, thus occurs in $B$ is as well. 
  Suppose two elements in the colimit correspond in this manner to the same element in $B$. 
  Then their equality follows from the relations of $B$. 
  By compactness in the meta-theory, their equality must follow from a finite subset of the relations from $B$, 
  hence there is some $B_m$ where both elements are equal, and they are equal in the colimit as well. 
  Thus the colimit embeds into $B$. 

  We conclude that $B$ and the colimit are isomorphic Boolean algebras. 
\end{proof} 









\begin{definition}
  We call an object (countably) compact if for every sequence $A = colim A_n$, any map $K \to A$ 
  factors through some map $K \to A_n$. 
  %compact means every filtered colimit, we only care about sequences. 
\end{definition}

\begin{lemma}
  Finitely presented algebras are compact in the category of algebras. 
\end{lemma}  
\begin{proof}
  Note that a map $K \to A$ is uniquely determined by it's values on the generators
  of which there are finitely many if $K$ is finitely presented. 
  Any finite subset of $A$ must at some point occur in some $A_n$, 
  thus we can assign the generators of $K$ a value in $A_n$. 
  This induces a map $K \to A_n$. 
\end{proof}



\begin{lemma}
  If $A \to B$ is injective between countably presented Boolean algebras, 
  we can write it as colimit of injections between finitely presented Boolean algebras. 
\end{lemma}
