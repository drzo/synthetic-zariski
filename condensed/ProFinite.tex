%%% \begin{definition}
%%%   A sequence is a diagram of shape $\mathbb N_{\leq}$.
%%% \end{definition} 
%
%x%\begin{lemma}
%x%  $B$ is a countably presented Boolean algebra iff 
%x%  it merely is the colimit of a sequence of finitely presented Boolean algebras.% $(B_n)_{n\in\mathbb N}$.
%x%\end{lemma} 
%x%%%Alternatively, we can define a countably presented Boolean algebra as an inductive limit of finite Boolean algebras
%x%%%that are given by a sequence of finite sets and maps between then.
%x%\begin{proof}
%x%  First, assume a sequence of finitely presented Boolean algebras. 
%x%  We need to show that the colimit is a countably presented Boolean algebra. 
%x%  \begin{itemize}
%x%    \item The set of generators for the colimit is the colimit of the sets of generators. 
%x%    \item The set of relations for the colimit is the union of the sets of relations. 
%x%      After all, any expression $f$ that becomes $0$ somewhere in the sequence will will be coprojected to $0$
%x%      in the colimit. And as any equality that holds in the colimit uses finitely many elements, 
%x%      it must already hold somewhere in the sequence. 
%x%  \end{itemize}
%x%  Note that both colimits over countably many finite sets are countable. 
%x%  Hence the colimit is countably represented. 
\begin{proof}
  Consider $\langle G \rangle /R$ a countable presentation of a Boolean algebra $B$. 
  We will show there exists a diagram of shape $\mathbb N$ taking values in Boolean algebras 
  with $\langle G\rangle / R$ as the colimit.
  \paragraph{The diagram}
  Now let $R_n$ be the first $n$ terms in $R$. 
  Note that each of these finitely many terms uses only finitely many symbols from $G$.
  Let $G_n$ be the finite set of terms used in $R_n$, unioned with the finite set of the first $n$ elements of $G$. 
  Define for each $n\in\mathbb N$ the finitely presented Boolean algebra $B_n = G_n /R_n$. 
  If $n\leq m$, then \Cref{rmkMorphismsOutOfQuotient} gives us a map $B_n \to B_m$ 
  as $G_n \subseteq G_{n+1}$ and $R_n \subseteq R_{n+1}$. 
  Thus $(B_n)_{n\in \mathbb N}$ gives us a diagram of shape $\mathbb N$
  with values in finitely presented algebras. 

  \paragraph{The colimit}
  As $G_n\subseteq G$ and $R_n \subseteq R$, 
  \Cref{rmkMorphismsOutOfQuotient} also gives us a map $B_n\to \langle G \rangle /R$. 
  We claim the resulting cocone is a colimit. 

  Suppose we have a cocone $C$ on the diagram $(B_n)_{n\in\mathbb N}$. 
  We need to show that there exists a map $\langle G \rangle / R\to C$ and
  we need to show this map is unique as map between cocones. 
  \begin{itemize}
    \item To show there exists a map $\langle G \rangle / R \to C$, 
      we use remark \Cref{rmkMorphismsOutOfQuotient} again. 
      Let $g\in G$ be the $n$'th element of $G$, 
      note that $g\in G_n$, and consider the image of $g$ under the map $B_n \to C$. 
      This procedure defines a function from $G$ to the underlying set of $C$. 
      Let $\phi \in R$ be the $n$'th element of $R$, 
      note that $\phi \in R_n$, and the map $B_n \to C$ must send $\phi$ to $0$. 
      Thus the function from $G$ to the underlying set of $C$ also sends $\phi$ to $0$. 
      This thus defines a map $\langle G \rangle / R \to C$. 
    \item To show uniqueness, consider that any map of cocones must take the same values 
      on all $g\in G_n$ for all $n\in\mathbb N$, but by \Cref{rmkMorphismsOutOfQuotient}
      this uniquely defines a map. 
  \end{itemize}
%  By \Cref{rmkMorphismsOutOfQuotient} a map $\langle G \rangle /R \to C$.
%  By \Cref{rmkMorphismsOutOfQuotient} such maps are uniquely determined by their values on $G$. 
%  to the underlying set of $C$. 
%  Let $g\in G$ be the $n$'th element of $G$. 
%  We 
%
%
%
%
%  We let $G_n$ be given by the first $n$ generators. 
%  Let $R_n$ be the relations involving these generators, 
%  of which there are only finitely many. 
%  We define $B_n = G_n/R_n$, which is a finitely presented Boolean algebra. 
%  The embedding of the first $n$ generators into the first $m$ generators gives us 
%  a map $B_n \to B_m$ whenever $n\leq m$. 
%  Because these morphisms are compatible, this defines a sequence of Boolean algebras. 
%  We claim the colimit of this sequence is $B$. 
%
%  Any element in $B$ can be expressed as Boolean combination of finitely many generators, 
%  which must occur in some $B_n$, and thus in the colimit. 
%  Whenever the images of two elements in the colimit are equal, they are already equal in some $B_m$, 
%  hence it follows from a finite subset of the relations for $B$ that the elements are equal, 
%  hence the elements are equal in $B$. Thus we have an embedding from $B$ into the colimit. 
%
%  Any element in the colimit already appears in some $B_n$, and hence is 
%  a finite expression using generators from $B$, thus occurs in $B$ is as well. 
%  Suppose two elements in the colimit correspond in this manner to the same element in $B$. 
%  Then their equality follows from the relations of $B$. 
%  By compactness in the meta-theory, their equality must follow from a finite subset of the relations from $B$, 
%  hence there is some $B_m$ where both elements are equal, and they are equal in the colimit as well. 
%  Thus the colimit embeds into $B$. 
%
%  We conclude that $B$ and the colimit are isomorphic Boolean algebras. 
\end{proof} 









\begin{definition}
  We call an object $K$ (countably) compact if for every sequence $A = colim A_n$, we have
  $A^K = colim A_n^K$.
  %compact means every filtered colimit, we only care about sequences. 
\end{definition}

\begin{lemma}
  Finitely presented algebras are compact in the category of algebras. 
\end{lemma}  
%% \begin{proof}
%%   Note that a map $K \to A$ is uniquely determined by it's values on the generators
%%   of which there are finitely many if $K$ is finitely presented. 
%%   Any finite subset of $A$ must at some point occur in some $A_n$, 
%%   thus we can assign the generators of $K$ a value in $A_n$. 
%%   This induces a map $K \to A_n$. 
%% \end{proof}

 The following uses Dependent Choice.

\begin{lemma}
  If $A \to B$ is injective between countably presented Boolean algebras, 
  we can write it as colimit of injections between finitely presented Boolean algebras. 
\end{lemma}
