\subsection{Countably Presented Boolean Algebras}
We will use countably presented (c.p.) boolean algebras in the following, for more definitions and notation see \Cref{A-cp-boolean-algebras}.

\begin{definition}
  A countably presented Boolean algebra $B$ is a Boolean algebra such that there merely are 
  %decidable 
  countable sets $I,J$, 
  a set of generators $g_i,~{i\in I}$ and a set $f_j,~{j\in J}$ of Boolean expressions over these generators 
  such that $B$ is equivalent to the quotient of the free Boolean algebra over the generators by the relations
  $f_j=0$. 
\end{definition} 
If $I,J$ are finite, we call $B$ a finitely presented Boolean algebra. 

\begin{example}
  \begin{enumerate}[(i)]
  \item $2$ is the Boolean algebra given by the empty set of generators and no relations. 
    It's underlying set is $\{0,1\}$. 
  \item   The trivial Boolean algebra given by the empty set of generators and the relation $\{1\}$, 
    It's underlying set contains only one element and we have $0=1$ in the trivial Boolean algebra. 
  \item\label{ExampleBAunderCantor}   $C = \langle \N \rangle $ is the Boolean algebra given by $\N$ as set of generators and no relations. 
  \item\label{ExampleBAunderNinfty}
    Denote $\N_{(co)fin}$ for the set of subsets of $\N$ which are finite or co-finite. 
    Under the interpretation of $\wedge = \cap , \vee = \cup, 0 = \emptyset, 1 = \N$, and $\neg$ 
    as the set-theoretic complemented (denoted $(\cdot)^C$). 
    
    These operations are well-defined on $\N_{(co)fin}$ 
    and they give the structure of a Boolean algebra. For a proof that this algebra is coutably presented, see \Cref{N-co-fin-cp}.
  \end{enumerate}
\end{example}

\begin{remark}
  We will denote $B_\infty$ for $\langle G \rangle/ \langle R\rangle $ as above, 
  and by the above result, we have that any $b:B_\infty$ can be written 
  either as $\bigvee_{i\in I_0} p_i$ or as $\bigwedge_{i\in I_0} \neg p_i$ for some finite $I_0\subseteq \N$. 
\end{remark}


\begin{remark}
  As Boolean algebras are rings, any relation of the form $f=g$ with both $f,g$ Boolean expressions 
  can be written as $h=0$ with $h=f-g$ a Boolean expression. 
\end{remark} 
We can express a countably presented Boolean algebra as the colimit of a finitely presented Boolean algebra. 
This is the formulation closer to \cite{Scholze}.

\input{Profinite}

\medskip

In SAG, we deal with a fixed commutative ring $R$. For this project, the role of $R$ is taken over by 
the Boolean algebra $2 = 1+1$. Note that we don't need to postulate an alternative for the \textbf{Loc} axiom. 
We write $\Boole$ the type of countably presented Boolean algebras.
Note that as each Boolean algebra is a Set, we $\Boole$ is a subtype of $hSet$.
Also, as being countable is a notion independent of universes, $\Boole$ is independent of universes.
Finally, note that $\Boole$ has a natural category structure. 

\medskip

\begin{definition}
  For $B$ a countably presented Boolean algebra, we define $Sp(B)$ as the set of Boolean morphisms from $B$ to $2$. 
\end{definition}

\medskip

An example of an element of $\Boole$ is the free algebra $C$ on countably many generators. The corresponding set $Sp(C)$
is then Cantor space $2^{\N}$.

Another example is the algebra $B_{\infty}$ generated by $p_n$ with relations $p_np_m = 0$ for $n\neq m$. The corresponding
set $Sp(B_{\infty})$ is the set $\Noo$ of binary sequences with at most one element $\neq 0$.

\begin{axiomNum}[Stone duality]
  For any countably presented Boolean algebra $B$, the evaluation map   $B\rightarrow  2^{Sp(B)}$ is an isomorphism.
\end{axiomNum} 

\begin{definition}
  We define the predicate on types $\isSt$ by 
  \begin{equation}
    \isSt(X) = \sum\limits_{B : Boole} X = Sp(B)
  \end{equation} 
  A type $X$ is called \textit{Stone} if $\isSt(X)$ is inhabited.
\end{definition}

Stone types will take over the role of affine scheme from \cite{draft}, 
and we repeat some results here. 
Analogously to Lemma 3.1.2, 
for $X$ Stone, we have $X = Sp(2^X)$. 
%
Proposition 2.2.1 now says that 
$Sp$ gives an equivalence 
\begin{equation}
   Hom_{\Boole} (A, B) = (Sp(B) \to Sp(A))
\end{equation}
By \cite{HoTT, p TODO}, it follows that 
$Sp$ is an embedding from $\Boole$ to any universe of types.
Its image, $\Stone$ also has a natural category structure.
The map $Sp$ defines then an anti-equivalence of categories between $\Boole$ and $\Stone$.


%% THE PROOF NEEDS TO BE REWRITTEN

%% \begin{proof}
%% %  We'll show that $is-Stone$ is contractible as soon as it is inhabited. 
%%   Let $(B,p), (B',p'): \isSt(X)$. 
  
%% %  As $Sp(B) = X = Sp(B')$, by Stone duality we have $$B \simeq 2^{Sp(B)} \simeq 2^{Sp(B')} \simeq B'.$$
%%   Furthermore, 
%%   for $SD: 2^{Sp(B')} = 2^{Sp(B)} \to B' = B$ conjugation by Stone duality, the map 
%%   $$SD \circ tr_{2^-} : Sp(B') = Sp(B) \to B' = B$$
%%   has inverse $tr_{Sp}$ by a path induction argument. 
%%   Therefore 
%%   $$
%%     \sum\limits_{B' : BA} Sp(B') = Sp(B)  
%%     \simeq 
%%     \sum\limits_{B' : BA} B' = B  
%%     $$
%%     as the latter is contractible, so is the former. 
%%     Hence $(B,refl) = (B', p' \cdot p ^{-1})$ and $(B,p) = (B',p')$. 
%%     Thus $\isSt(X)$ is contractible for any $X$. 
%% \end{proof} 

%\begin{remark}
%  By Stone duality, if $Sp(B) \simeq  Sp(B')$, then 
%  Furthermore, for any Boolean algebra $B$, we have that the composite
%
%  \begin{equation} \begin{tikzcd}
%    \sum\limits_{B' : BA} Sp(B') = Sp(B) \arrow[r, "tr_{2^\_}"] & 
%    \sum\limits_{B' : BA} 2^{Sp(B')} = 2^{Sp(B)} \arrow[r , "SD(B')^{-1} \cdot \_ \cdot SD(B)"]&
%    \sum\limits_{B' : BA} B' = B  
%   % \arrow[r, "tr_{Sp}"] & 
%   % \sum\limits_{B' : BA} Sp(B') = Sp(B) 
%  \end{tikzcd}   \end{equation} 
%  is an equivalence. As the latter type is contractible, being Stone is a proposition. 
%\end{remark} 


Any Stone set has a natural topology, where basic open are decidable subsets.

\medskip

\begin{proposition}
Any map $f: Sp(B)\rightarrow\N$ is uniformely continuous.
\end{proposition}  

\begin{proof}
  For each natural number $n$, the fiber $f^{-1}(n)$ is a decidable
  subset of $Sp(B)$. Via the isomorphism $B\rightarrow 2^{Sp(B)}$, this corresponds to an element $e_n$ of $B$. We have
  $e_ne_m = 0$. Furthermore the quotient $B'$ of $B$ by the relations $e_n = 0$ is such that $Sp(B') = 0$ and hence
  $1 = 0$ in $B'$, so we have $N$ such that $1 = \vee_{i<N}e_i$.
\end{proof}

\medskip

In formal/point-free topology, we consider that a Boolean algebra $B$ represents a Stone space $Sp(B)$ and a map
$Sp(B') \to Sp(B)$ is represented by a map $B\rightarrow B'$; the map $Sp(B')\to Sp(B)$ is then said to be
{\em formally surjective} if the corresponding map $B\to B'$ is injective. In the topos of light condensed sets,
this becomes a true duality.

\begin{proposition}\label{PropMarkov}
Markov's Principle holds, if we have $\neg \forall_n\alpha(n) = 0$ then we have $\exists_n \alpha(n) = 1$.
\end{proposition}

\begin{proof}
  Let $B$ be the Boolean algebra presented by $\alpha(n)$. We have $Sp(B) = \emptyset$ and hence by duality
  $B$ is trivial, which means that we have $n$ such that $\alpha(n) = 1$.
\end{proof}


\begin{axiomNum}[Surjections are Formal Surjections]
  A map $f:Sp(B')\to Sp(B)$ is surjective iff the corresponding map $B \to B'$ is injective.
\end{axiomNum} 

Another way to state this axiom is that epimorphisms in the category $\Stone$ are exactly the surjective maps.

Yet another formulation is $(\neg \neg X)\rightarrow \propTrunc{X}$ for $X$ Stone space. If we think of an algebra
in $\Boole$ as a proposition theory, this expresses a form of {\em completness}: any non inconsistent theory has
a model. 

\medskip

An example of a surjective map (since it is an epimorphism, since it corresponds to a monomorphism via the anti-equivalence
between $\Stone$ and $\Boole$) is the map sum of the maps $\Noo\rightarrow \Noo$ sending $n$ to $2n$ (resp. $n$ to $2n+1$).
This map has no section. This shows that $\Noo$ is not projective.

Here is another way to formulate this result.

\begin{proposition}
  LLPO is a consequence of Axioms $1$ and $2$.
\end{proposition}

Conversely, {\em with} Dependent Choice, LLPO implies Axiom 2, since it implies completeness of propositional logic.


A consequence of this characterisation of surjective maps is the following.

\begin{proposition}
  The image of any map between two Stone types is Stone.
\end{proposition}

Here is an example showing how to use this axiom. A closed subset of a Stone set is given by a countable
intersection of decidable subset.

\begin{proposition}
  Let $f:X'\rightarrow X$ a surjective map and $F_n$ a decreasing sequence of closed subsets of $X'$ such that
  each restriction $f_{|F_n}$ is surjective. Then if $F = \cap_n F_n$ the restriction $f_{|F}$ is still surjective.
\end{proposition}

\begin{proof}
  Dually, we have an injective map $i:B\rightarrow B'$ with an increasing sequence $I_n$ of ideals of $B'$ such that
  $b = 0$ if $i(b) = 0$ mod. $I_n$. The subset $F$ corresponds to the ideal $I = \cup_n I_n$. If $i(b) = 0$ mod. $I$
  then we have $i(b) = 0$ mod. $I_n$ for some $n$ and $b = 0$. 
\end{proof}





\begin{axiomNum}[Local choice]
  Whenever $X$ Stone and $E\twoheadrightarrow X$ surjective, then there is some $Y$ Stone,
    a surjection $Y \twoheadrightarrow X$ and a map $Y\to E$ such that the following diagram commutes:
    \begin{equation}\begin{tikzcd}
      E \arrow[d,""',two heads]\\
      X & \arrow[l, "", two heads] Y\arrow[lu, ""']
    \end{tikzcd}\end{equation}  
\end{axiomNum} 




The last axiom is Dependent Choice.

\begin{axiomNum}[Dependent Choice]
  Given a family of types $E_n$ and $R_n:E_n\rightarrow E_{n+1}\rightarrow {\mathcal U}$ such that
  for all $n$ and $x:E_n$ there exists $y:E_{n+1}$ with $p:R_n~x~y$ then given $x_0:E_0$ there exists
  $u:\Pi_{n:\N}E_n$ and $v:\Pi_{n:\N}R_n~(u~n)~(u~(n+1))$ and $u~0 = x_0$.
\end{axiomNum}

\medskip

One basic result about the category $\Boole$, the existence of retraction for non empty closed subset inclusion
holds only {\em non} constructively and in our setting we can prove the following.


\begin{proposition}
 It is not the case that for all closed proposition $p$ the inclusion $1+p\rightarrow 1+1$ has a retraction.
\end{proposition}

\begin{proof}
  This implies that all closed propositions are decidable and the proposition $x=\infty$ for $x$ in $\Noo$ is a
  closed proposition which is not decidable.
\end{proof}

\medskip

We can define the set $\Closed$ of closed propositions, where a proposition is closed iff it is equivalent to
the proposition $\forall_n \alpha(n) = 0$ for some $\alpha$ in $2^{\N}$.

\begin{theorem}
  Monomorphisms in $\Stone$ are classified by $\Closed$.
\end{theorem}

\medskip

We have seen that $\Noo$ is not projective. Using Local and Dependent Choice, David noticed that Scholze's argument
about $\ints[\Noo]$ cannot be made internal.

\begin{theorem}
   $\ints[\Noo]$ is {\em not} projective in the category of Abelian Groups.
\end{theorem}



\subsection{Countably presented algebras as sequential colimits}

\begin{definition}
  A sequence in a category is a diagram of shape $\N$, 
  where $\N$ carries the natural structure of a poset. 
\end{definition}
\begin{lemma}\label{lemProFinitePresentation}
  For every countably presented Boolean algebra $B$
  there merely exists a sequence of finitely presented Boolean algebras 
  whose colimit in the category of Boolean algebras is $B$. 
\end{lemma}
\begin{proof}
  Consider $\langle G \rangle \langle\langle R \rangle\rangle$ a countable presentation of a Boolean algebra $B$. 
  We will show there exists a diagram of shape $\N$ taking values in Boolean algebras 
  with $\langle G\rangle / R$ as the colimit.
  \paragraph{The diagram}
  Let $R_n$ be the first $n$ terms in $R$. 
  Note that each of these finitely many terms uses only finitely many symbols from $G$.
  Let $G_n$ be the finite set of terms used in $R_n$, unioned with the finite set of the first $n$ elements of $G$. 
  Define for each $n\in\N$ the finitely presented Boolean algebra $B_n = \langle G_n \rangle  \langle R_n \rangle$. 
  If $n\leq m$, then \Cref{rmkMorphismsOutOfQuotient} gives us a map $B_n \to B_m$ 
  as $G_n \subseteq G_{n+1}$ and $R_n \subseteq R_{n+1}$. 
  Thus $(B_n)_{n\in \N}$ gives us a diagram of shape $\N$
  with values in finitely presented algebras. 

  \paragraph{The colimit}
  As $G_n\subseteq G$ and $R_n \subseteq R$, 
  \Cref{rmkMorphismsOutOfQuotient} also gives us a map $B_n\to \langle G \rangle \langle R \rangle$. 
  We claim the resulting cocone is a colimit. 

  Suppose we have a cocone $C$ on the diagram $(B_n)_{n\in\N}$. 
  We need to show that there exists a map $\langle G \rangle / R\to C$ and
  we need to show this map is unique as map between cocones. 
  \begin{itemize}
    \item To show there exists a map $\langle G \rangle / R \to C$, 
      we use remark \Cref{rmkMorphismsOutOfQuotient} again. 
      Let $g\in G$ be the $n$'th element of $G$, 
      note that $g\in G_n$, and consider the image of $g$ under the map $B_n \to C$. 
      This procedure defines a function from $G$ to the underlying set of $C$. 
      Let $\phi \in R$ be the $n$'th element of $R$, 
      note that $\phi \in R_n$, and the map $B_n \to C$ must send $\phi$ to $0$. 
      Thus the function from $G$ to the underlying set of $C$ also sends $\phi$ to $0$. 
      This thus defines a map $\langle G \rangle / R \to C$. 
    \item To show uniqueness, consider that any map of cocones $\langle G \rangle / \langle R \rangle \to C$ 
      must take the same values on all $g\in G_n$ for all $n\in\N$. 
      Now all $g\in G$ occur in some $G_n$, so any map of cocones $\langle G \rangle /  \langle R \rangle \to C$ 
      takes the same values for all $g\in G$. 
      \Cref{rmkMorphismsOutOfQuotient} now tell us that these values uniquely determine the map. 
  \end{itemize}
\end{proof}
\begin{remark}
  Conversely, any colimit of a sequence of finite Boolean algebras 
  is a countably presented Boolean algebra with 
  as underlying sets of generators and relations the countable union of the finite sets of 
  generators and relations, which are both countable. 
\end{remark}
\begin{lemma}\label{lemFinitelyPresentedBACompact}
  For any finitely presented Boolean algebra $A$,
  and any sequence $(B_n)_{n:\N}$ of Boolean algebras with colimit $B$
  we have that the set $B^A$ is the colimit of the sequence of sets $(B_n^A)_{n:\N}$. 
\end{lemma}  
\begin{proof}
  First note that $B^A$ forms a cocone on $(B_n^A)_{n:\N}$ 
  because any map $A \to B_n$ induces a map $A \to B$. 
  Let $C$ be a cocone on $(B_n^A)_{n:\N}$. 
  We shall show there is an unique morphism of cocones $B^A \to C$. 
  \begin{itemize}
    \item For existence, let $f:B^A$. 
      As $A$ is finitely presented, we write $A = \langle G \rangle / \langle R \rangle$ with $G$ finite.
      By \Cref{rmkMorphismsOutOfQuotient}, $f$ is uniquely determined by it's values on $g\in G$. 
      As $G$ is finite, so is it's image $f(G)\subseteq B$. 
      But any finite subset of $B$ already occurs in $B_n$ for some $n\in\N$. 
      Consequently, the image of $f$ is already contained in some $B_n$. 
      Thus there is some $f_n:(B_n^A)$ such that postcomposing 
      $f_n$ with the map $B_n \to B$ gives back $f$. 
      The image of $f_n$ under the map $(B_n^A) \to C$ is how we define the image of $f$. 
      This is well-defined by the cocone conditions on $C$. 
    \item 
      For uniqueness, by function extensionality maps $B^A \to C$ are uniquely determined by their values on 
      $f:B^A$. By the above, the value of $f$ is uniquely determined by it's value on $B_n$ for 
      any $n$ with the image of $f$ in $B_n$. Thus there is at most one morphism of cocones $B^A \to C$. 
  \end{itemize}
\end{proof}
\begin{remark}\label{rmkEqualityColimit}
  In the above proof, we used that any element $b\in B$ already occurs in some $B_n$. 
  However, please note that it is not necessarily the case that it occurs uniquely in $B_n$, 
  there might be multiple elements in $B_n$ which can all be sent to $b$ in the end. 

  In case our sequence comes from the construction in \Cref{lemProFinitePresentation}, 
  we can see that whenever there are two elements in 
  $B_n$ corresponding to $b\in B$, they will become equal in $B_m$ for some $m\geq n$. 
  The reason is that if $b \sim_{\langle R \rangle} c$, there is a finite subset $R_0 \subseteq R$ such that 
  $b\sim_{\langle R_0 \rangle} c$, which will occur in some $R_m$. 

  One could wonder whether this property holds for general colimits of sequences. 
  In general, if we assume $B$ is the colimit of an arbitrary sequence $(B_n)_{n:\N}$, 
  and there exist some $B_n$ with two elements corresponding to the same element in $B$, 
  Theorem 7.4 from \cite{SequentialColimitHoTT} says that there merely exists some $m\geq n$
  such that they are already equal in $B_m$. 
\end{remark}

%For our next lemma on this presentation of sequences we need the axiom of dependent choice. 
%\begin{axiomNum}[Dependent choice]\label{axDependentChoice}
%  Given a family of types $(E_n)_{n:\N}$ and 
%  a relation 
%  $R_n:E_n\rightarrow E_{n+1}\rightarrow {\mathcal U}$ such that
%  for all $n$ and $x:E_n$ there exists $y:E_{n+1}$ with $p:R_n~x~y$ 
%  then given $x_0:E_0$ there exists
%  $u:\Pi_{n:\N}E_n$ and $v:\Pi_{n:\N}R_n~(u~n)~(u~(n+1))$ and $u~0 = x_0$.
%\end{axiomNum}
\begin{lemma}[Using dependent choice]\label{lemDecompositionOfColimitMorphisms}
  Let $B,C$ be countably presented Boolean algebras, 
  and suppose we have a morphism $f:B\to C$.
  There exists sequences of finitely presented Boolean algebras 
  $(B_n)_{n:\N}, (C_n)_{n:\N}$ with colimits $B,C$ respectively
  and compatible maps of Boolean algebras $f_n:B_n \to C_n$, 
  such that $f$ is the induced morphism $B\to C$.
\end{lemma}
\begin{proof}
  Let $(B_n)_{n:\N}, (C_n)_{n:\N}$ be 
  sequences of finitely presented Boolean algebras with colimits $B$ and $C$. 
  We will take a subsequence of $(C_n)_{n:\N}$, using the axiom of dependent choice above. 

  Our family of types $E_k$ as in \Cref{axDependentChoice} 
  will be strictly increasing sequences $(n_i)_{i\leq k}$ of natural numbers together with a finite family of maps 
  $(f_i: B_{i} \to C_{n_i})_{i\leq k}$ such that
  for all $0\leq i<k$ the following diagram commutes:
  \begin{equation}\label{eqnDecompositionOfColimitMorphisms}
    \begin{tikzcd}
      B_{i} \arrow[r] \arrow[d, "f_i"]& B_{{i+1}} \arrow[r] \arrow[d,"f_{i+1}"]& B \arrow[d,"f"] \\
      C_{n_i} \arrow[r] & C_{n_{i+1}} \arrow[r] & C 
    \end{tikzcd}
  \end{equation}
  Our relation $R_k$ will tell whether the second sequence extends the first one. 
%
  By \Cref{lemFinitelyPresentedBACompact} 
  there exists some $n_0:\N$ 
  such that $B_0 \to B \to C$ factors as 
  \begin{equation}
    \begin{tikzcd}
      B_{0} \arrow[r] \arrow[d, "f_0"]& B \arrow[d,"f"] \\
      C_{n_0} \arrow[r] & C 
    \end{tikzcd}
  \end{equation}
  Because our goal is a proposition, we can untracate this existence to data. 
  This data will form our $x_0:E_0$. %from \Cref{axDependentChoice}. 
%
  Now suppose we have $(f_i: B_{i} \to C_{n_i})_{i\leq k}$ for some $k\geq 0$ 
  such that
  for all $0\leq i<k$ the diagram of \Cref{eqnDecompositionOfColimitMorphisms} commutes.
  We shall show that in this case there exists an $n_{k+1}, f_{k+1}$ 
  making the same diagram commute for $i = k$. 
  Consider $B_{{k}+1}\to B \to C$. By the same argument as for $B_0$, we have a factorization 
  \begin{equation}
    \begin{tikzcd}
    B_{k+1} \arrow[r]  \arrow[d,"f'_{k+1}"]& B \arrow[d,"f"]\\
    C_{n'_{k+1}} \arrow[r] & C
    \end{tikzcd}
  \end{equation}
  Note that we may assume $n'_{k+1} > n_k$.
  Note that it is not necessarily the case that 
  $f'_{k+1}$ is compatibly with $f_k$, meaning the left square in the following diagram needn't commute:
  \begin{equation}
    \begin{tikzcd}
      B_{k} \arrow[r] \arrow[d, "f_k"]& B_{{k+1}}  \arrow[r] \arrow[d,"f'_{k+1}"] & B \arrow[d,"f"] \\
      C_{n_k} \arrow[r] & C_{n'_{k+1}} \arrow[r]  & C 
    \end{tikzcd}
  \end{equation}
  However, both $f'_{k+1}, f_k$ induce the same map $B_{k} \to C$. 
  Recall by \Cref{rmkMorphismsOutOfQuotient} this map is induced by it's value on finitely many elements. 
  By \Cref{rmkEqualityColimit}, it follows there is an $n_{k+1} \geq {n'_{k+1}}$ 
  such that for $f_{k+1}$ the composition of $f'_{k+1}:B_{k+1} \to C_{n'_{k+1}}$ and 
  the map $C_{n'_{k+1}} \to C_{n_{k+1}}$, the following diagram does commute:
  \begin{equation}
    \begin{tikzcd}
      B_{k} \arrow[d,"f_k"]\arrow[r] & B_{{k+1}} \arrow[rd, "f_{k+1}"] \arrow[rr] & & B \arrow[d,"f"] \\
      C_{n_k} \arrow[r] & C_{n'_{k+1}} \arrow[r] & C_{n_{k+1}} \arrow[r] & C 
    \end{tikzcd}
  \end{equation}
  Now by dependent choice for the above $x_0, R_n, E_n$, we get a sequence $(f_i:B_i \to C_{n_i})$  for some 
  strictly increasing sequence $n_i$ of natural numbers. 
  Note that for such a sequence $(n_i)_{i:\N}$, 
  $(C_{n_i})_{i:\N}$ converges to $C$. Also $(B_i)_{i:\N}$ still converges to $B$. 
  Futhermore, by construction the map that sequence $f_i$ induces from $B \to C$ shares all values with $f$
  and thus is equal to $f$. 
  Thus our sequence $f_i$ is as required. 
\end{proof}
\begin{remark}\label{rmkEpiMonoFactorizationCommutes}
  For $f,(f_i)_{i:\N}$ as above, whenever $f_n(x) = 0$, we have $f_{n+1}(x \circ \iota_{n,n+1}) = 0$
  for $\iota_{n,n+1}$ the map $A_n \to A_{n+1}$. 
  By \Cref{rmkMorphismsOutOfQuotient}, $\iota_{n,n+1}$ induces a map $A_n/Ker(f_n)\to A_{n+1}/Ker(f_{n+1})$. 
  This induced map is such that the following diagram commutes:
  \begin{equation}\begin{tikzcd}
    A_n \arrow[d, two heads] \arrow[r, "\iota_{n,n+1}"] & A_{n+1} \arrow[d,two heads]\\
    A_n /Ker(f_n) \arrow[d,hook] \arrow[r] & A_{n+1} /Ker(f_{n+1}) \arrow[d,hook] \\
    B_n \arrow[r] & B_{n+1}
  \end{tikzcd}\end{equation}  
  As the induced maps be epi's / mono's  is epi /mono, the colimit of the sequence 
  $A_n / Ker(f_n)$ will fit into an epi-mono factorization of $f$ and thus be iso to $A/Ker(f)$. 
  Thus the epi-mono factorization of the colimit is the colimit of the epi-mono factorizations. 
\end{remark}
\begin{remark}\label{rmkIsoEpiMonoMapColimit}
  Whenever $f:B \to C$ is an iso, any sequence with $B$ as colimit, also has $C$ as colimit. 
  Thus any iso can be represented this way as sequence of iso's. 
  Conversely, any sequence of isomorphisms induces an isomorphism of their colimits. 

  It follows from \Cref{rmkEpiMonoFactorizationCommutes} that when $f$ is epi/mono, 
  we can say that $f$ can be induced by a sequence 
  $(f_i)_{i\in \N}$ with all $f_i$ epi/mono. 
\end{remark}



