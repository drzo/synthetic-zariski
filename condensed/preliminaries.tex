\subsection{Countably Presented Boolean Algebras}
We will use the type of countably presented (c.p.) boolean algebras $\Boole$,
for more definitions and notation see \Cref{A-cp-boolean-algebras}.

\begin{definition}
  A countably presented Boolean algebra $B$ is a Boolean algebra such that there merely are 
  %decidable 
  countable sets $I,J$, 
  a set of generators $g_i,~{i\in I}$ and a set $f_j,~{j\in J}$ of Boolean expressions over these generators 
  such that $B$ is equivalent to the quotient of the free Boolean algebra over the generators by the relations
  $f_j=0$. 
\end{definition} 
If $I,J$ are finite, we call $B$ a finitely presented Boolean algebra. 

\begin{example}
  \label{boolean-algebra-examples}
  \begin{enumerate}[(i)]
  \item $2$ is the Boolean algebra given by the empty set of generators and no relations. 
    It's underlying set is $\{0,1\}$. 
  \item   The trivial Boolean algebra given by the empty set of generators and the relation $\{1\}$, 
    It's underlying set contains only one element and we have $0=1$ in the trivial Boolean algebra. 
  \item\label{ExampleBAunderCantor}   $C = \langle \N \rangle $ is the Boolean algebra given by $\N$ as set of generators and no relations. The corresponding Stone space $2^{\N}$ is \notion{Cantor space}.
  \item\label{ExampleBAunderNinfty}
    Denote $\N_{(co)fin}$ for the set of subsets of $\N$ which are finite or co-finite. 
    Under the interpretation of $\wedge = \cap , \vee = \cup, 0 = \emptyset, 1 = \N$, and $\neg$ 
    as the set-theoretic complemented (denoted $(\cdot)^C$). 
    
    These operations are well-defined on $\N_{(co)fin}$ 
    and they give the structure of a Boolean algebra. For a proof that this algebra is coutably presented, see \Cref{N-co-fin-cp}.
\rednote{Duplication}    Another example is the algebra $B_{\infty}$ generated by $p_n$ with relations $p_np_m = 0$ for $n\neq m$. The corresponding
set $Sp(B_{\infty})$ is the set $\Noo$ of binary sequences with at most one element $\neq 0$.

  \end{enumerate}
\end{example}

\begin{remark}
  We will denote $B_\infty$ for $\langle G \rangle/ \langle R\rangle $ as above, 
  and by the above result, we have that any $b:B_\infty$ can be written 
  either as $\bigvee_{i\in I_0} p_i$ or as $\bigwedge_{i\in I_0} \neg p_i$ for some finite $I_0\subseteq \N$. 
\end{remark}


\begin{remark}
  As Boolean algebras are rings, any relation of the form $f=g$ with both $f,g$ Boolean expressions 
  can be written as $h=0$ with $h=f-g$ a Boolean expression. 
\end{remark} 

We can express a countably presented Boolean algebra as the colimit of a finitely presented Boolean algebra. 
This is the formulation closer to \cite{Scholze}. We will see this presentation in \Cref{secBooleAsColimits}



