\subsection{The Cauchy reals}
The goal of this section is to introduce the real numbers in a constructive setting, 
following the definition given in \cite{Bishop} with some small adaptations. 
We will later use this definition to show that the interval $[0,1]$ is compact Hausdorff in the sense 
of \Cref{dfnCompactHausdorff}. 

We will assume we are given natural and rational numbers, with decidable (in)equalities
working as expected. 

\begin{definition}
  A \textbf{Cauchy sequence} is a sequence $x : \N \to \mathbb Q$ such that
  for any $n,m:\N$, we have %$0\leq x_n \leq 1$ and 
$|x_n-x_m| \leq (\frac12)^n + (\frac12)^m$. 
\end{definition}
\begin{remark}
  If $x$ is a cauchy sequence and $q$ a rational number, the 
  sequence $(x-q)_n = (x_n-q)$ is also Cauchy.
\end{remark}

Following \cite{Bishop}, we define inequality relations between Cauchy sequences and
rational numbers. 
\begin{definition}
  For $x$ a Cauchy sequence and $q$ a rational number, we define 
  \begin{itemize}
%    \item $x> q = \Sigma(n:\N) x_n > q + {\frac12}^n$. %for some $n:\N$. 
%    \item $x< q = \Sigma(n:\N) x_n < q - {\frac12}^n$. %for some $n:\N$. 
    \item $x\leq  q = \Pi_{n:\N} x_n \leq q+(\frac12)^n$. 
    \item $x\geq  q = \Pi_{n:\N} x_n \geq q-(\frac12)^n$. 
  \end{itemize}
\end{definition}
%\begin{lemma}
%  For $x$ Cauchy and $q$ rational, we have that 
%  $x\leq q$ iff for each $n:\N$, we have a $N_n:\N$ with 
%  $x_m> q-(\frac12)^n$ for all $m \geq N_n$. 
%\end{lemma}
\begin{lemma}\label{ComparisonLemma}
  For $x$ a Cauchy sequence and $q$ a rational number, we have
  $x \leq q \vee x \geq q$. 
\end{lemma}
\begin{proof}
  For rational numbers, we have decidable inequalities, 
  therefore $\geq 0 \vee q \leq 0$. 
  It follows that 
  $ \forall (n:\N) \forall (m:\N) q \geq -(\frac12)^n \vee q \leq (\frac12)^m$. 
  Now by \Cref{TODO}, we may conclude 
  $ (\forall (n:\N) q \geq -(\frac12)^n ) \vee (\forall (m:\N) q \leq (\frac12)^m)$
  as required.
\end{proof}


%%%\begin{definition}
%%%  A Cauchy sequence $x$ is \textbf{nonnegative} if $x_n \geq -(\frac12)^n$
%%%  for every $n:\N$. 
%%%  $x$ is \textbf{nonpositive} if $x_n \leq (\frac12)^n$
%%%  for every $n:\N$. 
%%%\end{definition} 
%%%%\begin{lemma}
%%%%  A Cauchy sequence is nonnegative iff there exists an $N$ such that $x_n \geq -(\frac12)^N$
%%%%  for all $n\geq N$. 
%%%%  A Cauchy sequence is nonpositive iff there exists an $N$ such that $x_n \leq (\frac12)^N$
%%%%  for all $n\geq N$. 
%%%%\end{lemma}
%%%%\begin{proof}
%%%%  Assume $x$ is nonnegative. Thus for every $n:\N$, we have $x_n\geq -(\frac12)^n \geq -(\frac12)^0$. 
%%%%  Thus $N$ can taken to be $0$. 
%%%%%
%%%%  Conversely, as $x$ is Cauchy, we have
%%%%  for all $m :\N$ that  
%%%%%  \begin{equation}- (\frac12)^m -(\frac12)^n \leq    x_m-x_n \leq (\frac12)^m + (\frac12)^n \end{equation}
%%%%  \begin{equation}- (\frac12)^m -(\frac12)^n \leq    x_n-x_m \leq (\frac12)^m + (\frac12)^n \end{equation}
%%%%  If in addition there is an $N$ such that whenever $m\geq N$, we have 
%%%%  $x_m \geq -(\frac12)^N$, so $-x_m \leq (\frac12)^N$, 
%%%%  so $x_n -x_m \leq x_n - (\frac12)^N$. 
%%%%  Therefore, 
%%%%  \begin{equation}- (\frac12)^m -(\frac12)^n \leq    x_n-x_m \leq x_n-(\frac12)^N \end{equation}
%%%%  Thus 
%%%%  \begin{equation}- (\frac12)^m -(\frac12)^n  + (\frac12)^N \leq x_n \end{equation}
%%%%  As $N \geq N$, we have in particular 
%%%%  \begin{equation}- (\frac12)^N -(\frac12)^n  + (\frac12)^N \leq x_n \end{equation}
%%%%  \begin{equation} - (\frac12)^n  \leq x_n \end{equation}
%%%%  thus $x$ is nonnegative. 
%%%%
%%%%  The nonpositive case goes similar. 
%%%%\end{proof}   
%%% 
%%%
%%%\begin{lemma}
%%%  A Caucy sequence is nonnegative or nonpositive. 
%%%\end{lemma}

%\begin{lemma}
%  For any Cauchy sequence $p$, we have 
%  $(\forall (n:\N) p_n \leq (\frac12)^n) \vee (\forall (n:\N) p_n \geq -(\frac12)^n)$. 
%\end{lemma}
%\begin{proof}
%We 
%\end{proof}

\begin{definition}
Given two Cauchy sequences $p = (p_n)_{n\in\N}, q=(q_n)_{n\in\N}$, 
we define the proposition $p \sim_C  q$ as 
\begin{equation}
  p \sim_C q : = \forall (n,m : \N) ((| p_n - q_m| \leq  (\frac12)^n + (\frac12)^m))
\end{equation}
\end{definition}

%\begin{remark}
%  Note that $p\sim_C q$ is equivalent to 
%\begin{equation}
%  \forall (n : \N) | p_n - q_n| \leq  (\frac12)^{n-1}
%\end{equation}
%The equivalence doesn't hold, unless you cut off initial segments (which shouldn't matter). 
%\end{remark} 

\begin{definition}
  The type of \textbf{Cauchy reals} is given by 
  the type of Cauchy sequences modulo $\sim_C$.
\end{definition}

We claim that the inequality in \Cref{TODO} extends to a well-defined 
inequality between Cauchy reals and rational numbers. 

Furthermore, we claim that 
$\Pi_{x:\mathbb R} \Pi_{q:\mathbb Q} x \leq q \vee x \geq q$. 

%\begin{lemma}
%  For any Cauchy real $r$ any Cauchy sequence $p$ representing $r$, 
%  we have 
%  \begin{equation}
%    (\forall (n:\N) p_n \leq (\frac12)^n) \vee (\forall (n:\N) p_n \geq (\frac12)^n)
%  \end{equation}
%
%\end{lemma}

\begin{definition}
  A Cauchy sequence in the interval is a Cauchy sequence $x$ such that 
  for any $n:\N$, we have $0\leq x_n \leq 1$. 
 % 
  The interval of Cauchy reals is given by the type of Cauchy sequences in the interval 
  modulo $\sim_C$. We denote it by $[0,1]$. 
\end{definition}  

