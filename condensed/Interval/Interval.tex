\subsection{The interval}
The goal of this section is to define the interval $[0,1]_\mathbb R$ as a scheme. 
We assume $\mathbb N, \mathbb Q$ have been defined in HoTT
with linear propostional order relations $<,\leq, > ,\geq$ playing nicely together 
and standard algebraic operations. 
From these, we can define the subtype $\mathbb Q_{>0}=\sigma _{q : \mathbb Q} (q>0)$, 
and the absolute-value function $|\cdot|$ on $\mathbb Q$. 

\begin{definition}
  A pre-Cauchy sequence is a sequence of rational numbers $(q_n)_{n: \mathbb N}$ with $0 \leq q_n \leq 1$ 
  for all $n:\mathbb N$
%  together with a term of type
  such that for every $\epsilon: \mathbb Q_{>0}$, we have an $N:\mathbb N$, 
  such that whenever $n,m \geq N$, we have 
\begin{equation}
%  \forall \epsilon : \mathbb Q_{>0} \Sigma N : \mathbb N \forall m,n : \mathbb N (m,n \geq N) \to 
  | q_n - q_m | \leq \epsilon
\end{equation} 
\end{definition}

\begin{definition}
Given two pre-Cauchy sequences $p = (p_n)_{n\in\mathbb N}, q=(q_n)_{n\in\mathbb N}$, 
we define the proposition $p \sim_C  q$ as 
%for all $\epsilon : \mathbb Q_{>0}$ there exists an $N :\mathbb N$ such that whenever $n \geq N$, we have
\begin{equation}
  p \sim_C q : = \forall (\epsilon : \mathbb Q_{>0} )\exists ( N :\mathbb N) \forall (n : \mathbb N) ((n \geq N) \to 
  (| p_n - q_n| \leq  \epsilon))
\end{equation}
\end{definition}
Note that $\sim_C$ defines an equivalence relation on pre-Cauchy sequences. 
\begin{definition}
We define the type of Cauchy sequences as the type of pre-Cauchy sequences quotiented by $\sim_C$. 
\end{definition}

%\begin{definition}
%  A binary sequence consists of an initial segment $I \subseteq \mathbb N$
%  and a function $x:I \to 2$. 
%If $I$ is (in)finite, we call the binary sequence (in)finite as well. 
%\end{definition} 
%
%For $x$ a finite binary sequence and $y$ any binary sequence, 
%we'll denote $(x,y)$ for their concatenation, 
%and $\overline x$ for the infinite sequence repeating $x$. 
%
Consider the relation $\sim_s$ on $2^{\mathbb N}$, 
such that for any finite binary sequence $x$, we have 
$$(x,1,\overline 0) \sim_s (x ,0, \overline 1).$$
\begin{proposition}
  Cauchy sequences are in isomorphism with $2^\mathbb N / \sim_s$. 
\end{proposition} 
\begin{proof}
  For $\alpha: 2^\mathbb N$, define the rational sequence $bin(\alpha)$ by 
  \begin{equation} (bin (\alpha))_n :  = \sum\limits_{0 \leq i \leq n} \alpha(i) / 2^{i+1}\end{equation}  
  This sequence is pre-Cauchy and whenever $\alpha\sim_s \beta$, we have 
  $bin(\alpha) \sim_C bin(\beta)$. 
  Therefore $bin$ induces a function from $2^\mathbb N / \sim_s$ to Cauchy sequences. 
  
  \medskip 

  Conversely, assume we are given a pre-Cauchy sequence $p$. 
  We will define a binary sequence $\alpha  = c(p): 2^\mathbb N$.
  Consider any $i:\mathbb N$, and suppose $\alpha(j)$ has been defined for $0 \leq j<i$. 

  Let $\epsilon = (\frac12)^{i^2}$. %Placeholder value.
  Let $N$ be such that for $n,m \geq N$, we have $|p_n - p_m| < \epsilon$. 
  Consider $\widetilde p_i = p_N - \sum\limits_{0\leq j < i} \frac {\alpha(j)}{2^j}$.
  As the order on $\mathbb Q$ is total, we can define 
  \begin{equation}
    \alpha(i) = \begin{cases}
    1 \text{ if } \widetilde p_i > ((\frac12)^{i} - \epsilon) \\
    0 \text{ otherwise } 
    \end{cases} 
  \end{equation}  
  This is a binary sequence, hence $c$ defines a function from pre-Cauchy sequences to $2^\mathbb N$.
  We claim that whenever $p \sim_C q$, we have $c(p) \sim_s c(q)$. 

  Suppose that $c(p)(n) = 0$ and $c(p)(n) = 1$, we shall show that in this case, 
  for $m > n$ we'll have $c(p)(m) =1 , c(p)(m) = 0$. 

\end{proof}


