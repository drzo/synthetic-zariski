
\begin{definition}
  A \textbf{Cauchy sequence} is a sequence $x : \mathbb N \to \mathbb Q$ such that
  for any $n,m:\mathbb N$, we have %$0\leq x_n \leq 1$ and 
$|x_n-x_m| \leq (\frac12)^n + (\frac12)^m$. 
\end{definition}

\begin{definition}
Given two Cauchy sequences $p = (p_n)_{n\in\mathbb N}, q=(q_n)_{n\in\mathbb N}$, 
we define the proposition $p \sim_C  q$ as 
\begin{equation}
  p \sim_C q : = \forall (\epsilon : \mathbb Q_{>0} )\exists ( N :\mathbb N) \forall (n : \mathbb N) ((n \geq N) \to 
  (| p_n - q_n| \leq  \epsilon))
\end{equation}
\end{definition}

\begin{definition}
  The type of \textbf{Cauchy reals} is given by 
  the type of Cauchy sequences modulo $\sim_C$.
\end{definition}

\begin{definition}
  A Cauchy sequence in the interval is a Cauchy sequence $x$ such that 
  for any $n:\mathbb N$, we have $0\leq x_n \leq 1$. 
 % 
  The interval of Cauchy reals is given by the type of Cauchy sequences in the interval 
  modulo $\sim_C$. 
\end{definition}  

\begin{lemma}
  For $p$ a Cauchy sequence, we have $p\leq 0 \vee p \geq 0 $. 
\end{lemma}


We want to show that the interval of Cauchy reals are a scheme. 
Informally, to any binary sequence $\alpha : \mathbb N \to 2$, 
we can associate a Cauchy sequence 
\begin{equation}\label{eqnBinaryEncode}
  n\mapsto \sum\limits_{i = 0 }^n \frac {\alpha(i)}{2^{i+1}}
\end{equation}
and we are going to give a closed relation on Cantor space such that 
two binary sequences are equivalent iff they correspond to the same Cauchy reals. 
%
First, we'll need some notation.
\begin{definition}
Given a binary sequence $\alpha:\mathbb N \to 2$ and a natural number $n : \mathbb N$  
we denote $\alpha|_n: \mathbb N_{\leq n} \to 2$ for the 
restriction of $\alpha$ to a finite sequence of length $n$. 
We denote $\overline 0, \overline 1$ for the binary sequences which are constantly $0$ and $1$ respectively. 
We denote $0,1$ for the sequences of length $1$ hitting $0,1$ respectively. 
If $x$ is a finite sequence and $y$ is any sequence, denote $x\cdot y$ for their concatenation. 
\end{definition} 
Now we'll give a definition for when two finite binary sequences of length $n$ correspond 
to real numbers whose distance is $\leq (\frac12)^n$.
Basically, we want for every finite sequence $z$ that 
$(z \cdot 0 \cdot \overline 1)$ and  $(z \cdot 1 \cdot \overline 0)$ are equivalent. 

\begin{definition}
Now let $n:\mathbb N$ and $x,y:\mathbb N_{\leq n} \to 2$ be two sequences of length $n$. 
We say $x,y$ are near if we have an $m:\mathbb N$ with $m\leq n$
and some $a: \mathbb N_{\leq m} \to 2$, 
such that one of $(a \cdot 0 \cdot \overline 1)|_n,  ( a \cdot 1 \cdot \overline 0)|_n$
is equal to $x$ and the other is equal to $y$. 
We denote $\text{near}_n(x,y)$ if $x,y$ are near. 
%
To be precise, we define 
\begin{equation}
  \text{near}_n(x,y) = 
\Sigma(m:\mathbb N) m \leq n \wedge 
  \Sigma (a : Fin_m \to 2) 
\bigg( \big( (x,y) = 
((a \cdot 0 \cdot \overline 1)|_n,  ( a \cdot 1 \cdot \overline 0)|_n)
\big)
\bigvee 
\big(
  (y,x) = 
((a \cdot 0 \cdot \overline 1)|_n,  ( a \cdot 1 \cdot \overline 0)|_n)
\big)
\bigg)
\end{equation}
\end{definition}
\begin{remark}
Remark that when $x,y$ are near, $m$ and $a$ as above are unique. 
Thus $\text{near}_n(x,y)$ is a proposotion. 
%
Furthermore, to check whether $x,y$ are near, we need only make $n$ comparisons, 
thus $\text{near}_n(x,y)$ is decidable. 
%
Note that in the above definition, we allow $m = n$ and therefore $x$ is near to itself for any finite sequence $x$. 
Furthermore, we have defined nearness to be symmetric. 
However, it is not a transtive relation. 
After all, the sequence $010$ and $011$ are near and the sequence $011$ and $100$ are near, 
but $010$ is not near to $100$. 
\end{remark}
\begin{definition}
  We define the following relation on Cantor space for $\alpha, \beta: 2^\mathbb N$.
  \begin{equation}
    \alpha \sim_t \beta = \forall (n : \mathbb N) 
    \text{near}_n(\alpha|_n, \beta|_n)
  \end{equation}
\end{definition}
\begin{lemma}
  $\sim_t$ is a closed equivalence relation. 
\end{lemma}
\begin{proof}
   Let $\alpha, \beta, \gamma : 2^\mathbb N$. 
   As the dependent product of propositions is a proposition, $\alpha \sim_t\beta$ is a proposition. 
   %
   Furthermore, the closedness follows from decidability of $\text{near}_n(\alpha|_n, \beta|_n)$. 
   One could define $\gamma(n) = 1$ iff $\text{near}_n(\alpha|_n, \beta|_n)$
   
   As nearness is reflexive and symmetric, so is $\sim_t$. 

   Now suppose $\alpha \sim_t \beta$ and $\beta\sim_t \gamma$. 
   We claim that $\alpha \sim_t \gamma$. 

   Let $n:\mathbb N$, we need to show that 
   $\text{near}_n(\alpha|_n , \gamma|_n)$. 
   Let $(a,m)$ witness that $\text{near}_n(\alpha|_n, \beta|_n)$.
   and let $(b, k)$ witness that $\text{near}_n(\beta|_n, \gamma|_n)$
   We will make a case distinction on whether one of $m,k$ is equal to $n$, or
   both are strictly smaller than $n$. 
   \begin{itemize}
     \item 
       If $m=n$, we have that $\alpha|_n = \beta|_n$, and therefore 
       \begin{equation}
         \text{near}_n(\beta|_n, \gamma|_n) \leftrightarrow \text{near}_n(\alpha|_n, \gamma|_n)
       \end{equation} 
       The above also holds if $k = n$.
     \item 
       If $m< n$, we have that $\alpha(m+1) \neq \beta(m+1)$, thus 
       $\alpha|_l \neq \beta|_l$ for all $l>m$, 
       but we still have $\text{near}_l(\alpha|_l, \beta|_l)$ for these $l$. 
       Therefore $(\alpha, \beta)$ or $(\beta, \alpha)$ must be of the form
       $(a \cdot 0 \cdot \overline 1, a \cdot 1 \cdot \overline 0)$. 
       WLOG, we assume $\alpha = a \cdot 0 \cdot \overline 1$, and thus 
       $\beta = a \cdot 1 \cdot \overline 0$ (if not, we could do bitflips). 

       As $k<n$ also, by the same argument there is some $b$ such that one of 
       $(\beta,\gamma), (\gamma, \beta)$
       is equal to $(b\cdot 0 \cdot \overline 1, b \cdot 1 \cdot \overline 0)$. 
       However, $\beta$ is also of the form $a \cdot 1 \cdot \overline 0$, and 
       thus cannot also be of the form $b \cdot 0 \cdot \overline 1$. 
       Therefore we must have 
       $\beta = b\cdot 1 \cdot \overline 0$ and 
       $\gamma= b\cdot 0 \cdot \overline 1$. 

       But now $b \cdot 1 \cdot \overline 0 = a \cdot 1 \cdot \overline 0$, 
       The lengths of $a,b$ cannot be unequal, and by decidablity of natural numbers, 
       $a,b$ have the same length and it follows that $ a = b$. 
       Therefore $ \alpha = \gamma$, so $\alpha \sim_t\gamma$.
   \end{itemize}

   We conclude that $\sim_t$ is a closed equivalence relation. 
\end{proof}

\begin{theorem}
  The interval of Cauchy reals is isomorphic to $2^\mathbb N / \sim_t$. 
\end{theorem} 
\begin{proof}
  \Cref{eqnBinaryEncode} gives a function from Cantor space to Cauchy sequences in the interval. 
  By a standard argument, this functions sends $\sim_t$ equivalent binary sequences to $\sim_C$ equivalent 
  Cauchy sequences, and thus defines a morphism from $2^\mathbb N/\sim_t$ to the interval of Cauchy reals. 

  For the other function, we will rely on Lemma \cref{TODO}
\end{proof}

