\subsection{Anti-equivalence of $\Boole$ and $\Stone$}

\begin{remark}\label{SpIsAntiEquivalence}
Stone types will take over the role of affine scheme from \cite{draft}, 
and we repeat some results here. 
Analogously to Lemma 3.1.2 of \cite{draft}, 
for $X$ Stone, Stone duality tells us that $X = Sp(2^X)$. 
%
Proposition 2.2.1 of \cite{draft} now says that 
$Sp$ gives an equivalence 
\begin{equation}
   Hom_{\Boole} (A, B) = (Sp(B) \to Sp(A))
\end{equation}
Therefore $\isSt$ is a proposition.
Equivalently, 
%By Definition 4.6.1 %defines an embedding 
%of \cite{hott}, 
this means that 
$Sp$ is an embedding from $\Boole$ to any universe of types.
Its image, $\Stone$ also has a natural category structure.
By the above and Lemma 9.4.5 of \cite{hott}, 
the map $Sp$ defines a dual equivalence of categories between $\Boole$ and $\Stone$.
\end{remark}

\begin{lemma}\label{SpectrumEmptyIff01Equal}
  For $B:\Boole$, we have $0=_B1$ iff $\neg Sp(B)$.
\end{lemma}
\begin{proof}
  Note that whenever $0=1$ in $B$, there is no map $B\to 2$ respecting both $0$ and $1$ as $0\neq 1$ in $2$. 
  Thus $\neg Sp(B)$ whenever $0=1$ in $B$. 
  % 
  Conversely, if $\neg Sp(B)$, then $Sp(B) = \emptyset$, which is also the spectrum of the trivial Boolean algebra. 
  As $Sp$ is an embedding, $B$ is equivalent to the trivial Boolean algebra, and $0=_B1$. 
\end{proof}
%\begin{corollary}\label{MoreConcreteCompleteness}
%  By the above and propositional completeness, we have that $||Sp(B)||$ iff $0\neq_B1$. 
%\end{corollary}


\begin{remark}\label{StoneClosedUnderPullback}
%  By \Cref {SpIsAntiEquivalence} and the fact that that countably presented Boolean algebras form a 
%  finitely cocomplete category (\Cref{CoCompletenessBoole}), the category of Stone spaces is complete. 
  By \Cref {SpIsAntiEquivalence} and the fact that that countably presented Boolean algebras are closed under pushouts, 
  the category of Stone spaces is closed under pullbacks. 
\end{remark}

We conclude this section on the anti-equivalence of Stone and $\Boole$ by a relating surjections to injections. 
This theorem is actually equivalent to completeness of propositional logic, which we'll discuss in 
\Cref{NotesOnAxioms}. 
\begin{theorem}\label{FormalSurjectionsAreSurjections}
  Let $f:A\to B$ be a map of countably presented Boolean algebras. 
  If $f$ is injective, then the corresponding map $(\cdot) \circ f : Sp(B) \to Sp(A)$ is surjective. 
\end{theorem}

\begin{proof}
  Assume $f$ injective and let $x:Sp(A)$.
  By \Cref{StoneClosedUnderPullback}, we have the following pullback square in the category of Stone spaces:
%  In particular, $\sum\limits_{y:Sp(B)} y\circ f = x$ is Stone:
  \begin{equation}\begin{tikzcd}
    \sum\limits_{y:Sp(B)} y\circ f = x \arrow[d] \arrow[r] \arrow["\lrcorner"{pos=0.125}, phantom, dr] 
    & \top \arrow[d,"x"]\\
    Sp(B) \arrow[r,"(\cdot) \circ f"] & Sp(A)
  \end{tikzcd}  \end{equation}
  By propositional completeness and \Cref{SpectrumEmptyIff01Equal},
  to show that a Stone space is merely inhabited, it is sufficient to show that $0\neq 1$ in the underlying Boolean algebra. 
  Consider therefore the dual to the above square in the category of countably presented Boolean algebras,
  where $Sp(P) \simeq  (\sum\limits_{y:Sp(B)} y \circ f = x)$:
  \begin{equation}\begin{tikzcd}
    A \arrow[d,"x"'] \arrow[r,hook,"f"] \arrow[rd,phantom,"\ulcorner"{pos=0.125}] & B\arrow[d]\\
    2 \arrow[r] & P
  \end{tikzcd}\end{equation} 
  Following \Cref{BoolePushouts}, the pushout $P$ is given by $B/R$ with $R$ the relations $f(a) -x(a)$ 
  where $a$ ranges over the generators of $A$.
  Note that $x(a) \in \{0,1\}$. 
  If $x(a)=0$, then $f(a)-x(a) = f(a)$, 
  and if $x(a) = 1$, then $f(a) -x(a) = \neg f(a) = f(\neg a)$. 
  So $R$ contains elements of the form $f(b)$ where $b$ ranges over some countable $B\subseteq A$, 
  such that $x(b) = 0$ for $b\in B$. 
  
  We shall show that $0\neq_P 1$. 
  Note that $0=_P 1$ iff 
  $1 = \bigvee_{r \in R_0} r$ for some $R_0\subseteq R$ finite. 
  By the above, this means that 
  $$1= \bigvee_{b\in B_0} f(b) = f(\bigvee_{b\in B_0} b)$$ 
  for some $B_0\subseteq B$ finite. 
  As $f$ is injective, we must have $\bigvee_{b\in B_0} b  = 1$. 
  However, 
  $$x(\bigvee_{b\in B_0} B ) = \bigvee_{b \in B_0} x(b) = \bigvee_{b\in B_0} 0 = 0$$
  And as $x(1) = 1 \neq_2 0$, we get a contradiction. Therefore $0\neq_P 1$ as required. 
\end{proof}  
The converse to the above theorem is true as well, regardless of propositional completeness:
\begin{lemma}
If $f:A\to B$ is a map in $\Boole$ and $(\cdot) \circ f :Sp(B) \to Sp(A)$ is surjective, 
$f$ is injective. 
\end{lemma}
\begin{proof}
  Suppose precomposition with $f$ is surjective. 
  Let $a:A$ be such that $f(a)= 0$. 
  By assumption, for every $x:A\to 2$, there is a $y:B\to 2$ with $y\circ f = x$. 
  Consequentely $x(a) = y(f(a)) = y(0) = 0$. 
  So $x(a) = 0$ for every $x:Sp(A)$. 
  Thus $Sp(A) = Sp(A/\{a\})$, and as $Sp$ is an embedding, 
  $A \simeq A/\{a\}$, and $a = 0$ in $A$. 
  So whenever $f(a) = 0$, we have $a=0$. Thus $f$ is injective. 
\end{proof}
