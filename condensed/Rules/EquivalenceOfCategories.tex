\subsection{Anti-equivalence of $\Boole$ and $\Stone$}

\begin{remark}\label{SpIsAntiEquivalence}
Stone types will take over the role of affine scheme from \cite{draft}, 
and we repeat some results here. 
Analogously to Lemma 3.1.2 of \cite{draft}, 
for $X$ Stone, Stone duality tells us that $X = Sp(2^X)$. 
%
Proposition 2.2.1 of \cite{draft} now says that 
$Sp$ gives an equivalence 
\begin{equation}
   Hom_{\Boole} (A, B) = (Sp(B) \to Sp(A))
\end{equation}
Therefore $\isSt$ is a proposition.
Equivalently, 
%By Definition 4.6.1 %defines an embedding 
%of \cite{hott}, 
this means that 
$Sp$ is an embedding from $\Boole$ to any universe of types.
Its image, $\Stone$ also has a natural category structure.
By the above and Lemma 9.4.5 of \cite{hott}, 
the map $Sp$ defines a dual equivalence of categories between $\Boole$ and $\Stone$.
\end{remark}

\begin{lemma}\label{SpectrumEmptyIff01Equal}
  For $B:\Boole$, we have $0=_B1$ iff $\neg Sp(B)$.
\end{lemma}
\begin{proof}
  Note that whenever $0=1$ in $B$, there is no map $B\to 2$ respecting both $0$ and $1$ as $0\neq 1$ in $2$. 
  Thus $\neg Sp(B)$ whenever $0=1$ in $B$. 
  % 
  Conversely, if $\neg Sp(B)$, then $Sp(B) = \emptyset$, which is also the spectrum of the trivial Boolean algebra. 
  As $Sp$ is an embedding, $B$ is equivalent to the trivial Boolean algebra, and $0=_B1$. 
\end{proof}

%\begin{corollary}\label{MoreConcreteCompleteness}
%  By the above and propositional completeness, we have that $||Sp(B)||$ iff $0\neq_B1$. 
%\end{corollary}


\begin{remark}\label{StoneClosedUnderPullback}
%  By \Cref {SpIsAntiEquivalence} and the fact that that countably presented Boolean algebras form a 
%  finitely cocomplete category (\Cref{CoCompletenessBoole}), the category of Stone spaces is complete. 
  By \Cref {SpIsAntiEquivalence} and the fact that that countably presented Boolean algebras are closed under pushouts, 
  the category of Stone spaces is closed under pullbacks. 
\end{remark}

We conclude this section on the anti-equivalence of Stone and $\Boole$ by a relating surjections to injections. 
This theorem is actually equivalent to completeness of propositional logic, which we'll discuss in 
\Cref{NotesOnAxioms}. 
\begin{theorem}\label{FormalSurjectionsAreSurjections}
  Let $f:A\to B$ be a map of countably presented Boolean algebras. 
  If $f$ is injective, then the corresponding map $(\cdot) \circ f : Sp(B) \to Sp(A)$ is surjective. 
\end{theorem}

\begin{proof}
  Assume $f$ injective and let $x:Sp(A)$.
  By \Cref{FiberConstruction}, we have that $\sum\limits_{y:Sp(B)} y\circ f = Sp(B/R) $
  for $R=f(G)$ for some countable $G\subseteq A$ with $x(g) = 0$ for all $g\in G$. 
  By propositional completeness and \Cref{SpectrumEmptyIff01Equal}, 
  it's sufficient to show that $0\neq_{B/R}1$. 
  Note that $0=_{B/R} 1$ iff 
  $1 = \bigvee R_0$ for some $R_0\subseteq R$ finite. 
  But then $$1 = \bigvee f(G_0) = f(\bigvee  G_0)$$ for some $G_0\subseteq G$ finite. 
  And as $f$ is injective, $\bigvee G_0 = 1$. 
  However, 
  $$
  x(\bigvee G_0) = 
  x(\bigvee_{g\in G_0} g ) = \bigvee_{g \in G_0} x(g) = \bigvee_{g\in G_0} 0 = 0$$
  And as $x(1) = 1 \neq_2 0$, we get a contradiction. Therefore $0\neq_{B/R} 1$ as required. 
\end{proof}  
The converse to the above theorem is true as well, regardless of propositional completeness:
\begin{lemma}\label{SurjectionsAreFormalSurjections}
If $f:A\to B$ is a map in $\Boole$ and $(\cdot) \circ f :Sp(B) \to Sp(A)$ is surjective, 
$f$ is injective. 
\end{lemma}
\begin{proof}
  Suppose precomposition with $f$ is surjective. 
  Let $a:A$ be such that $f(a)= 0$. 
  By assumption, for every $x:A\to 2$, there is a $y:B\to 2$ with $y\circ f = x$. 
  Consequentely $x(a) = y(f(a)) = y(0) = 0$. 
  So $x(a) = 0$ for every $x:Sp(A)$. 
  Thus $Sp(A) = Sp(A/\{a\})$, and as $Sp$ is an embedding, 
  $A \simeq A/\{a\}$, and $a = 0$ in $A$. 
  So whenever $f(a) = 0$, we have $a=0$. Thus $f$ is injective. 
\end{proof}
