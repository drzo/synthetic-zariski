In this section, we will introduce the basic rules we'll use in this paper. 
We will first state our axioms, and then draw out some first consequences.
Most notably, 
we will see that Markov's principle (\Cref{MarkovPrinciple}) and the 
lesser limited principle of omniscience (\Cref{LLPO}) can be shown. 
%There are equivalent axiom systems we could have stated instead, which we discuss in \Cref{NotesOnAxioms}.
\subsection{Axioms}\label{Axioms}
In this section, we will state our axioms. 
In \Cref{NotesOnAxioms}, we will discuss alternative versions of our axiom system. 
\begin{axiomNum}[Stone duality]
  For any countably presented Boolean algebra $B$, the evaluation map $B\rightarrow  2^{Sp(B)}$ is an isomorphism.
\end{axiomNum} 

\begin{axiomNum}[Propositional completeness]
  For $S:\Stone$, we have that $\neg \neg S \leftrightarrow || S ||$
\end{axiomNum}

%\begin{axiomNum}[Surjections are formal Surjections]
%  A map $f:Sp(B')\to Sp(B)$ is surjective iff the corresponding map $B \to B'$ is injective.
%\end{axiomNum} 
%
%\begin{lemma}\label{LemSurjectionsFormalToCompleteness}
% For $S:\Stone$, we have that $\neg \neg S \to || S ||$
%\end{lemma}
%\begin{proof}
%  First, assume that surjections are formal surjections. 
%  Let $B:\Boole$ and suppose $\neg \neg Sp(B)$. 
%  %Note that if $0=1$ in $B$, then $Sp(B) =\emptyset$, meaning $\neg Sp(B)$. 
%  %Therefore, we have $0\neq 1$ in $B$. 
%  We will show that the map $f:2\to B$ is injective. 
%  Let $f:2 \to B$, note that if $f(0) = f(1)$ then $0=1$ in $B$, 
%  If $0=1$ in $B$, there are no maps $B\to 2$ preserving $0$ and $1$, thus $\neg Sp(B)$. 
%  This is a contradiction with $\neg \neg Sp(B)$. Thus we may conclude that $f(0)\neq f(1)$. 
%  Hence by case distinction on $2$ we can show $f$ we have that $f x = f y$ implies $ x= y$. Thus 
%  $f$ is injective thus the map $Sp(B) \to Sp(2) = 1$ is surjective, thus $Sp(B)$ is merely inhabited. 
%\end{proof} 
%Actually, we will see in \Cref{CorDoubleNegToAx2} that the converse is also true. 

%\begin{axiomNum}[Local choice]
%  Whenever $S$ Stone and $E\twoheadrightarrow S$ surjective, then there is some $T$ Stone,
%    a surjection $T \twoheadrightarrow S$ and a map $T\to E$ 
%    such that the following diagram commutes:
%    \begin{equation}\begin{tikzcd}
%      E \arrow[d,""',two heads]\\
%      S & \arrow[l, "", two heads, dashed] T\arrow[lu, ""',dashed ]
%    \end{tikzcd}\end{equation}  
%\end{axiomNum} 
\begin{axiomNum}[Local choice]
  Whenever we have $S:\Stone$, $E,F$ arbitrary types, a map $f:S \to F$ and a 
  surjection $e:E \twoheadrightarrow F$, 
  there exists a Stone space $T$, a surjective map $T\twoheadrightarrow S$ and an arrow $T\to E$ making the following diagram commute:
    \begin{equation}\begin{tikzcd}
      T \arrow[d,dashed, two heads ] \arrow[r,dashed]&  E \arrow[d,""',two heads, "e"]\\
      S  \arrow[r, "f"] & F
    \end{tikzcd}\end{equation}  
\end{axiomNum}


\begin{axiomNum}[Dependent choice]\label{axDependentChoice}
  Given a family of types $(E_n)_{n:\N}$ and 
  a relation 
  $R_n:E_n\rightarrow E_{n+1}\rightarrow {\mathcal U}$ such that
  for all $n$ and $x:E_n$ there exists $y:E_{n+1}$ with $p:R_n~x~y$ 
  then given $x_0:E_0$ there exists
  $u:\Pi_{n:\N}E_n$ and $v:\Pi_{n:\N}R_n~(u_n)~(u_{n+1})$ and $u_0 = x_0$.
\end{axiomNum}
