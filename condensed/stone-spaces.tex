\subsection{Definitions}
\begin{definition}
  For $B$ a countably presented Boolean algebra, we define $Sp(B)$ as the set of Boolean morphisms from $B$ to $2$. 
\end{definition}
\begin{definition}
  We define the predicate on types $\isSt$ by 
  \begin{equation}
    \isSt(X) := \sum\limits_{B : Boole} X = Sp(B)
  \end{equation} 
  A type $X$ is called \textit{Stone} if $\isSt(X)$ is inhabited.
\end{definition}

\begin{definition}
  We define the predicate on types $\isSt$ by 
  \begin{equation}
    \isSt(X) = \sum\limits_{B : Boole} X = Sp(B)
  \end{equation} 
  A type $X$ is called \textit{Stone} if $\isSt(X)$ is inhabited.
\end{definition}

Stone types will take over the role of affine scheme from \cite{draft}, 
and we repeat some results here. 
Analogously to Lemma 3.1.2, 
for $X$ Stone, we have $X = Sp(2^X)$. 
%
Proposition 2.2.1 now says that 
$Sp$ gives an equivalence 
\begin{equation}
   Hom_{\Boole} (A, B) = (Sp(B) \to Sp(A))
\end{equation}
By \cite{HoTT, p TODO}, it follows that 
$Sp$ is an embedding from $\Boole$ to any universe of types.
Its image, $\Stone$ also has a natural category structure.
The map $Sp$ defines then an anti-equivalence of categories between $\Boole$ and $\Stone$.

\subsection{Examples of Stone types}
\begin{example}\label{ExampleBAunderEmpty}
  \begin{enumerate}[(i)]
  \item   The dual to the trivial Boolean algebra from \Cref{ExampleBAunderEmpty} is the empty set 
    $\emptyset$, 
    as $0\neq 1$ in $2$, but $0=1$ in the trivial Boolean algebra, 
    there can be no functions preserving both $0$ and $1$ 
    from the trivial Boolean algebra to $2$. 
  \item   The dual to $C$ from \Cref{boolean-algebra-examples} \ref{ExampleBAunderCantor} is called Cantor space 
    and denoted $2^\N$. 
    By \Cref{rmkMorphismsOutOfQuotient}, terms of $2^\N$ 
    correspond to set-theoretic functions $\N \to 2$. 
    We also call such functions binary sequences. 
  \item The dual to $\N_{(co)fin}$ from \Cref{boolean-algebra-examples} \ref{ExampleBAunderNinfty} is called 
    $\N_\infty$. By \Cref{rmkMorphismsOutOfQuotient}, terms of $\N_\infty$ 
    correspond to functions $\alpha: \N \to 2$ such that $\alpha(n) \wedge \alpha(m) = 0$ 
    whenever $n \neq m$. This means that $\alpha(n) = 1$ for at most one $n\in\N$. 
    There is an embedding $\N \to \N_\infty$ sending $n$ to the unique sequence $\chi_n$
    which sends $n$ to $1$. 
    There is furthermore a term $\infty:\N_\infty$ which is the sequence which is constantly $0$. 
  \end{enumerate}
\end{example}

