\begin{abstract}
In synthetic algebraic geometry (SAG) \cite{draft}, we study finitely presented algebras over a commutative ring. 
In this work, we study countably presented Boolean algebras instead. 
Where the finitely presented algebras over a commutative ring induce a Zariski topos, 
%the opposite category of these 
the countably presented Boolean algebras induce the topos of light condensed sets \cite{TODO}. 
\cite{draft} proposes an axiomatization of the Zariski topos in univalent homotopy type theory \cite{HoTT}. 
In this work, we propose a similar axiomatization for light condensed sets. 
% Furthermore, spectra of countably presented Boolean algebras correspond to quotients of Cantor space
% which is cool because reasons
\end{abstract} 

\begin{definition}
  A countably presented Boolean algebra $B$ is a Boolean algebra such that there mererly are countable sets $I,J$, 
  a set of generators $g_i,~{i\in I}$ and a set $f_j,~{j\in J}$ of Boolean expressions over these generators 
  such that $B$ is equivalent to the quotient of the free Boolean algebra over the generators by the relations
  $f_j=0$.
\end{definition} 

Alternatively, we can define a countably presented Boolean algebra as an inductive limite of finite Boolean algebras
that are given by a sequence of finite sets and maps between then.

 \medskip

In SAG, we deal with a fixed commutative ring $R$. For this project, the role of $R$ is taken over by 
the Boolean algebra $2 = 1+1$. Note that we don't need to postulate an alternative for the \textbf{Loc} axiom. 
We write $\Boole$ the type of countably presented Boolean algebras.



\medskip





\begin{definition}
  For $B$ a countably presented Boolean algebra, we define $Sp(B)$ as the set of Boolean morphisms from $B$ to $2$. 
%  We assign a pointwise Boolean algebra structure on $Sp(B)$. 
  %This is not necessarily a Boolean algebra, structure, consider the free BA on 1 generator (the diamond).
  %The maps send the generator to either 0 or 1, but the pointwise conjunction of these maps sends both the generator 
  % and it's negation to 0, which shouldn't happen.
\end{definition}

\medskip

An example of element of $\Boole$ is the free algebra $C$ on countably many generators. The corresponding set $Sp(C)$
is then Cantor space $2^{\N}$.

Another example is the algebra $B_{\infty}$ generated by $p_n$ with relations $p_np_m = 0$ for $n\neq m$. The corresponding
set $Sp(B_{\infty})$ is the set $\Noo$ of binary sequences with at most one element $\neq 0$.

\begin{axiom}[Stone duality]
  For any countably presented Boolean algebra $B$, the canonical map   $B\rightarrow  2^{Sp(B)}$ is an isomorphism.
\end{axiom} 

\begin{definition}
  A type $X$ is called \textit{Stone} if there exists $B$ in $\Boole$ such that $X = Sp(B)$. 
\end{definition}

\begin{remark}
  By Stone duality, if $Sp(B) \simeq  Sp(B')$, then $$B \simeq 2^{Sp(B)} \simeq 2^{Sp(B')} \simeq B'.$$
  Furthermore, for any Boolean algebra $B$, we have that the composite
  $$ 
  \sum\limits_{B' : \Boole} Sp(B') = Sp(B) \to 
  \sum\limits_{B' : \Boole} 2^{Sp(B')} = 2^{Sp(B)} \to
  \sum\limits_{B' : \Boole} B' = B 
  $$ 
  is an equivalence. As the latter type is contractible, for any type $X$ the type $\Sigma_{B:\Boole}X = Sp(B)$ is a proposition.
\end{remark} 

It thus follows from the first axiom that $Sp$ is an embedding from $\Boole$ to any universe of types. We call
$\Stone$ its image. Both groupoids $\Boole$ and $\Stone$ have a natural category structure.
The map $Sp$ defines then an anti-equivalence of categories between these two categories.


Any Stone set has a natural topology, where basic open are decidable subsets.

\medskip

\begin{proposition}
Any map between $f: Sp(B)\rightarrow\N$ is uniformely continuous.
\end{proposition}  

\begin{proof}
  For each natural number $n$, the fiber $f^{-1}(n)$ is a decidable
  subset of $Sp(B)$. Via the isomorphism $B\rightarrow 2^{Sp(B)}$, this corresponds to an element $e_n$ of $B$. We have
  $e_ne_m = 0$. Furthermore the quotient $B'$ of $B$ by the relations $e_n = 0$ is such that $Sp(B') = 0$ and hence
  $1 = 0$ in $B'$, so we have $N$ such that $1 = \vee_{i<n}e_i$.
\end{proof}

\medskip

In formal/point-free topology, we consider that a Boolean algebra $B$ represents a Stone space $Sp(B)$ and a map
$Sp(B') \to Sp(B)$ is represented by a map $B\rightarrow B'$; the map $Sp(B')\to Sp(B)$ is then said to be
{\em formally surjective} if the corresponding map $B\to B'$ is injective. In the topos of light condensed sets,
this becomes a true duality.

\begin{axiom}[Surjections are Formal Surjections]
  A map $f:Sp(B')\to Sp(B)$ is surjective iff the corresponding map $B \to B'$ is injective.
\end{axiom} 

Another way to state this axiom is that epimorphisms in the category $\Stone$ are exactly the surjective maps.

\medskip

An example of a surjective map (since it is an epimorphism, since it corresponds to a monomorphism via the anti-equivalence
between $\Stone$ and $\Boole$) is the map sum of the maps $\Noo\rightarrow \Noo$ sending $n$ to $2n$ (resp. $n$ to $2n+1$).
This map has no section. This shows that $\Noo$ is not projective.

Another way to formulate this result is the following.

\begin{proposition}
  LLPO is a consequence of Axioms $1$ and $2$.
\end{proposition}

Here is another consequence of this characterisation of surjective maps.

\begin{proposition}
  The image of any map between two Stone types is Stone.
\end{proposition}

\begin{axiom}[Local choice]
  Whenever $X$ Stone and $E\twoheadrightarrow X$ surjective, then there is some $Y$ Stone,
    a surjection $Y \twoheadrightarrow X$ and a map $Y\to E$ such that the following diagram commutes:
    \begin{equation}\begin{tikzcd}
      E \arrow[d,""',two heads]\\
      X & \arrow[l, "", two heads] Y\arrow[lu, ""']
    \end{tikzcd}\end{equation}  
\end{axiom} 




The last axiom is Dependent Choice.

\begin{axiom}[Dependent Choice]
  Given a family of types $E_n$ and $R_n:E_n\rightarrow E_{n+1}\rightarrow {\mathcal U}$ such that
  for all $n$ and $x:E_n$ there exists $y:E_n$ with $p:R~x~y$ then given $x_0:E_0$ there exists
  $u:\Pi_{n:\N}E_n$ and $v:\Pi_{n:\N}R~(u~n)~(u~(n+1))$ and $u~0 = x_0$.
\end{axiom}

\medskip

We recall two basic results about the category $\Boole$, which holds constructively using Dependent Choice,
that can now be formulate as results about $\Stone$.

\begin{proposition}
  And injective map in $\Stone$ has a retraction. Any surjective {\em open} map in $\Stone$ has a section.
\end{proposition}

\medskip

We can define the set $\Closed$ of closed propositions, where a proposition is closed iff it is equivalent to
the proposition $\forall_n \alpha(n) = 0$ for some $\alpha$ in $2^{\N}$.

\begin{theorem}
  Monomorphisms in $\Stone$ are classified by $\Closed$.
\end{theorem}

\medskip

We have seen that $\Noo$ is not projective. Using Local and Dependent Choice, we can show the following.
We can follow the argument presented by Scholze in lecture 3 in the present axiomatic setting.

\begin{theorem}
  $\ints[\Noo]$ is projective in the category of Abelian Groups.
\end{theorem}

