\subsection{Intersection of closed in compact Hausdorff}
\begin{lemma}
  In a compact Hausdorff, closed sets are closed under intersection. 
\end{lemma}
\begin{proof}
  
\end{proof}
\begin{lemma}
  For $S$ Stone, $D\subseteq S$ decidable, $\sim$ a closed equivalence relation on $S$,
  the set $\{x:S | \exists y : D (x\sim y)\}$ is closed. 
\end{lemma}
\begin{proof}
  
\end{proof}



\begin{lemma}
  For $S$ Stone, $D\subseteq S$ decidable, 
  $\sim$ a decidable equivalence relation on $S$,
  the set $\{x:S | \exists y : D (x\sim y)\}$ is closed. 
\end{lemma}
\begin{proof}
  Let $B = 2^S$, so $S = Sp(B)$. 
  As $D$ is decidable, 
%  there is some $b:B$ such that $D(y) := (y(b) = 1)$. 
  there is some $n:\mathbb N$ such that $D(y)$ only depends on $y|_n$. 

  As $\sim$ is decidable, there is a finite set $I_0\subseteq \mathbb N$,
  such that $x\sim y = \prod_{i:I_0} x(i) = y(i)$. 

  Thus 
  $$
   \exists (y : D) (x\sim y) = 
  || \Sigma(y:2^\mathbb N) y(b) = 1 \wedge \prod(i:I_0) x(i) = y(i)||
  $$
\end{proof}



\begin{lemma}
  Let $S$ Stone, then $D\subseteq S$ is closed iff 
  $D\subseteq S\subseteq 2^{\mathbb N}$ is closed. 
\end{lemma}
\begin{proof}
  Follows immediately from countable intersection of basic clopen. 
\end{proof}




%%  Let $A,B\subseteq X$ be two closed subsets of a compact Hausdorff space $X = S/ \sim$. 
%%  If we know that closed subsets contain are exactly those containing their limits this is very easy right? 
%%  Then any sequence has it's limit both in $A$ and $B$. 
%%\begin{lemma}
%%  Whenever $x_n$ is a convergent sequence, so is $f(x_n)$. 
%%\end{lemma}
%%\begin{proof}
%%  Follows immediately from \Cref{sequenceConvergentIffLimit}.
%%\end{proof}
%%
%%
%%\begin{lemma}
%%  In a compact Hausdorff, whenever a subset $A$ contains all of its limit points, it is closed. 
%%\end{lemma}
%%
%%\begin{proof}
%%  Suppose $A\subseteq X$ contains all of it's limit points. We will show that $f^{-1}(A)$ is closed. 
%%  Let $(x_n)_{n:\mathbb N}$ be a sequence in $f^{-1}(A)$ with limit $l$, 
%%  then 
%%  $(f(x_n))_{n:\mathbb N}$ is a sequence in $A$ with limit $f(l)$. 
%%  $A$ contains $f(l)$ by assumption. 
%%  Therefore $l\in f^{-1}(A)$. 
%%  Thus every sequence in $f^{-1}(A)$ with a limit has its limit in $f^{-1}(A)$. 
%%\end{proof}
%%
%%\begin{lemma}
%%  In a Stone space, whenever a subset $A$ contains all of its limit points, it is closed. 
%%\end{lemma}
%%\begin{proof}
%%  Let $A \subseteq S$ contain all of it's limit points. 
%%  We will show $A$ is a countable intersection of decidable subsets of $S$, hence closed. 
%%  As $S$ is a subset of Cantor space, we may assume it is Cantor space. 
%%  Thus $A$ is a set of binary sequences. 
%%
%%  We will denote $D_n$ be the set of initial segments of length $n$ occuring in $A$. 
%%  We claim this is well defined, that's not a problem, as it's the image of an operation. 
%%
%%  Counterexample : $A = \{ \overline 0 | p\}$ which contains all of it's limit points
%%  (any sequence in $A$ must be $\overline 0$ constantly, which has a limit if the sequence exists in $A$). 
%%  However, $D_n$ is not decidable. 
%%  Also $A$ is not the intersection of countably many decidable sets I believe. 
%%  Unless off course $p$ is of the form $\alpha=0$, but those are not the only propositions.
%%  For example, the proposition $\beta\neq 0$ cannot be written in that form for general $\beta$. 
%%\end{proof}
