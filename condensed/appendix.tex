\subsection{Countably Presented Boolean Algebras: Definitions}
\label{A-cp-boolean-algebras}
\begin{definition}
  The language of Boolean algebras consists of
  \begin{itemize}
    \item Two symbols for constants $0,1$. 
    \item Two binary function symbols $\wedge, \vee$ called meet and join respectively. 
    \item One unary function symbol $\neg$ called negation. 
  \end{itemize}
\end{definition}
\begin{definition}
  The algebraic theory of Boolean algebras is as follows:
  \begin{itemize}
    \item We have axioms regarding order, sometimes called identity and absorption:
    \begin{itemize}
      \item $0 \wedge a = 0$. 
      \item $1 \wedge a = a$. 
      \item $0 \vee a = a$. 
      \item $1 \vee a = 1$. 
      \item $a \vee (a \wedge b) = a$
      \item $a \wedge (a \vee b) = a$. 
    \end{itemize}
    \item We have axioms regarding negations:
      \begin{itemize}
        \item $a \vee \neg a = 1$. 
        \item $a \wedge \neg a = 0$. 
        \item $\neg 1 = 0$. 
        \item $\neg \neg a = a$. 
      \end{itemize}
    \item We have axioms regarding commutativity and associativity:
      \begin{itemize}
        \item $a \vee b = b \vee a$. 
        \item $a \wedge b = b \wedge a$. 
        \item $(a \vee b) \vee c = a \vee (b \vee c)$. 
        \item $(a \wedge b) \wedge c = a \wedge (b \wedge c)$. 
      \end{itemize}
    \item We have axioms regarding distributivity:
      \begin{itemize}
        \item $a \vee (b \wedge c) = ( a \vee b) \wedge (a \vee c)$. 
        \item $a \wedge (b \vee c) = (a \wedge b) \vee ( a\wedge c)$. 
      \end{itemize}
    \item We have axioms called deMorgan's laws:
      \begin{itemize}
        \item $\neg (a \vee b ) = (\neg a) \wedge (\neg b)$. 
        \item $\neg (a \wedge b) = (\neg a) \vee (\neg b)$. 
      \end{itemize}
  \end{itemize}
\end{definition}
\begin{definition}
  Some language we will use is as follows:
  \begin{itemize}
    \item A Boolean algebra is a set-theoretic model of the theory of Boolean algebras. 
      It consists of a set $B$ and terms $0_B,1_B: B, \neg_B : B \to B, \vee_B, \wedge_B : B \times B \to B$ 
      satisfying the theory of Boolean algebras. 
      Sometimes we will leave out the subscript $(\cdot)_B$. 
    \item A morphism of Boolean algebra is a function between such sets preserving the operations above. 
      If $A,B$ are Boolean algebras, we write $A \to B$ for the type of such morphisms. 
%    \item We denote $id_B$ for the identity morphism $B \to B$. 
    \item If we have have two morphisms $f:A\to B , g : B \to A$ such that 
      $f\circ g (b) = b, g \circ f (a) = a$ for all $a:A, b:B$, 
      we say $A$ and $B$ are isomorphic and write $A \simeq B$. 
  \end{itemize}
\end{definition}
\begin{remark}
  We sometimes write $a \leq b$ and  $b \geq a$. Both of these mean that 
  $a\wedge b = a$, or equivalently $ a\vee b = b$. 
  Interpreting this notation as an order relation, we often think in the following terms:
  \begin{itemize}
    \item $a\wedge b$ is the biggest element $c$ with $c \leq a$ and $c\leq b$.
    \item $a\vee b$ is the smallest element $c$ with $c \geq a$ and $c \geq b$. 
    \item $0$ is the smallest element. 
    \item $1$ is the biggest element. 
  \end{itemize}
\end{remark}
\begin{remark}
  For a family of expressions $x_i$ over a 
  finite set of indices $I = \{i_0, \cdots, i_n\}$, we will sometimes denote 
  \begin{itemize}
    \item $\bigvee_{i:I} x_i = x_{i_0} \vee \cdots \vee x_{i_n}$,
      with the understanding that this equals $0$ if $I$ is empty. 
    \item $\bigwedge_{i:I} x_i = x_{i_0} \wedge \cdots \wedge x_{i_n}$,
      with the understanding that this equals $1$ if $I$ is empty. 
  \end{itemize}
\end{remark}
\begin{remark}
  Boolean algebras are in bijective correspondence with Boolean rings, which 
  are rings with the property that $x\cdot x = x$ for all $x$. 
  The underlying set remains the same, and the translation is as follows:
  \begin{itemize}
    \item 
      To go from such a ring to a Boolean algebra, we let 
      $0 = 0, 1 =1 ,  a \vee b = a + b + ( a \cdot b), a\wedge b = a \cdot b, \neg a = a + 1$. 
    \item 
      and to go from a Boolean algebra to a Boolean ring, we let 
      $0 = 0, 1 = 1, a + b = (a \wedge \neg  b) \vee ( \neg a \wedge b), a \cdot b = a\wedge b , -a = a$. 
\end{itemize}
\end{remark}

%\begin{definition}
%  Let $\mathcal L$ be a language, 
%  let $\phi,\psi$ be terms in any language,
%  and suppose that for some index set $I$ we have 
%  for each $i:I$ two terms $a_i,b_i$ of same sort in $\mathcal L$. 
%  Let $E= (a_i = b_i)_{i:I}$. 
%
%  We say that $E$ allows us to rewrite $\phi$ to $\psi$ if there is a finite list of expressions 
%  $\phi = \phi_0 , \phi_1, \phi_2, \cdots, \phi_{n-1}, \phi_n = \psi$, 
%  such that each $\phi_{n+1}$ is given by taking $\phi_n$ and either changing one occurence of $a_i$ to  $b_i$, 
%  or one occurence of $b_i$ to $a_i$ for some $i:I$. 
%\end{definition}


\begin{definition}
  Given any set of symbols $G$, the free Boolean algebra on $G$, denoted $\langle G \rangle$
  is constructed as follows:
  \begin{itemize}
    \item We add $G$ as set of constants to the language of Boolean algebras. 
    \item We consider the set of terms in this language. 
    \item If there merely exists some finite sequence of 
      equalities from the theory of Boolean algebras allowing us to rewrite one term to another, 
      we consider the two terms to be equivalent. 
    \item The free Boolean algebra is then the set of equivalence classes under this relation. 
  \end{itemize}
\end{definition}
\begin{remark}
  The free Boolean algebra is a Boolean algebra. 
\end{remark}
\begin{definition}
  Let $B$ be a Boolean algebra, a subset $I\subseteq B$ is an ideal iff 
  $0\in I$ and whenver $x,y \in I$ and $b\in B$, we have $x\vee y \in I$ and $x \wedge b \in B$. 

  If $I$ is an ideal of $B$, then for $a,b:B$, we write $x\sim_I y$ iff 
  $a - b  = 
  (a \wedge \neg b ) \vee ( \neg a \wedge b) 
  \in I$. 
\end{definition}
\begin{lemma}
  Let $B$ be a Boolean algebra, and let $I\subseteq B$ be an ideal. 
  Then the relation $\sim_I$ is an equivalence relation. 
\end{lemma}
\begin{proof}
  \begin{itemize}
    \item To see $\sim_I$ is reflexive, note that 
      $(x \wedge \neg x) \vee (\neg x \wedge x) = 0 \vee 0 = 0 \in I$. 
    \item To see $\sim_I$ is symmetric, note that 
      $ (a \wedge \neg b) \vee ( \neg a \wedge b) = 
        (b \wedge \neg a) \vee ( \neg b \wedge a)$. 
    \item To see $\sim_I$ is transitive, that 
      whenever $a,b \in I$, so is $ a + b$. 
      Thus if $a -b , b -c \in I$, so is 
      $(a -b) + (b-c) = a -c$.
  \end{itemize}
\end{proof}
\begin{lemma}
  Whenever $B$ is a Boolean algebra and $I\subseteq B$ an ideal, 
  $B/I$ is a Boolean algebra. 
\end{lemma}
\begin{proof}
  We need to show that the Boolean operations respect equivalence classes. 
  \begin{itemize}
    \item 
      Suppose $a \sim_I b$. We shall show that $\neg a \sim_I \neg b$. 
      Thus we need to show that 
      $((\neg a) \wedge (\neg \neg b)) \vee ((\neg \neg a) \wedge \neg b) \in I$, 
      this term is equal to 
      $( a \wedge \neg b) \vee (b \wedge \neg a)$, which is in $I$ as $ a\sim_I b$. 
    \item 
      Suppose $ a\sim _I b $. We shall show that 
      $a \wedge c \sim_I b \wedge c$. 
      Note that by distributivity of $+$ and $\cdot$, we have 
      $(a \cdot c) -( b \cdot c) = (a -b ) \cdot c = (a-b) \wedge c$. 
      By assumption $a -b \in I$, and thus $(a-b) \wedge c \in I$ as well. 
    \item Note that $a \vee b = \neg ( \neg a \wedge \neg b)$, by the above two properties, 
      joins therefore respect equivalence relations. 
  \end{itemize}
\end{proof}
\begin{definition}
  Let $B$ be a Boolean algebra, and let $R\subseteq B$. 
  We denote $\langle R\rangle $ for the ideal generated by $R$, 
  which is the set of expressions 
  $\{(\bigvee_{r \in R_0} r) \wedge b| R_0\subseteq R \text{ finite}, b \in B\}$
\end{definition} 
\begin{remark}
  For $R\subseteq B$ as above, $ \langle R \rangle$ is an ideal.
\end{remark}

\begin{remark}\label{rmkMorphismsOutOfQuotient}
  Let $B$ be a Boolean algebra. 
  
  Suppose $G$ is a set of symbols, to define a morphism 
  $\langle G \rangle \to B$ it is sufficient to define 
  a function from $G$ to the underlying set of $B$. 

  Let $C$ be a Boolean algebra, and let $I\subseteq C$ be an ideal. 
  To define a map $C / I \to B$, it is sufficient to define a map $f:C\to B$
  with $f(i) = 0$ for all $i\in I$. 
  If $I = \langle R\rangle $, it is sufficient that $f(r) = 0$ for all $r\in R$. 
\end{remark}
%\begin{remark}
%  For every element $x$ of $\langle G\rangle /R$ as above,
%  there merely exists a finite subset of generators $G_0\subseteq G$ such that $x$ is expressable 
%  as a Boolean combination of symbols from $G_0$. 
%\end{remark}
\begin{definition}
  A type $X$ is countable if there merely exists a surjection 
  $\N \twoheadrightarrow X$. 
\end{definition}

\begin{definition}
  Let $B$ be a Boolean algebra. 
  If there exist $G, R$ as above such that 
  $B \simeq \langle G \rangle /\langle R \rangle$, 
  we call $G,R$ respectively generators and relations for $B$. 
  %
  If $G,R$ can taken to be finite, we call $B$ finitely presented. 
  If $G,R$ can taken to be countable, we call $B$ countably presented. 
\end{definition}
\begin{remark}
  There is a category of Boolean algebras, which has a subcategory of countably presented Boolean algebras, 
  which has a subcategory of finitely presented Boolean algebras. 
\end{remark}
\begin{remark}\label{rmkBoolePushouts}
  All categories above have pushouts. 
\end{remark}
\begin{definition}
  Let $f:B\to C$ be a Boolean morphism. 
  The set $\{x \in B| fx = 0\}$ is called the kernel of $f$ and denoted 
  $Ker(f)$. 
\end{definition}
\begin{remark}
  For any map of Boolean algebras $f$, the kernel of $f$ is an ideal. 
  Furthermore, we have that $a\sim_{Ker(f)} b$ iff $f(a) = f(b)$. 
\end{remark}
\begin{remark}
  Let $f:B \to C$ be a Boolean morphism. 
  The epi-mono factorization of $f$ exists and is given by  
  $B \twoheadrightarrow B / Ker(f) \hookrightarrow C$.
\end{remark}

\begin{remark}\label{N-co-fin-cp}
  The algebra of co-finite subsets of $\N$, which we defined as $\N_{(co)fin}$ in \Cref{ExampleBAunderNinfty} is
  countably presented.
\end{remark}
\begin{proof}
  We let $f:\langle G \rangle / R \to \N_{(co)fin}$, be the unique morphism of Boolean algebras 
  satisfying $f(p_n) = \{n\}$ for all $n:\N$. As $\{n\} \cap \{m\} = \emptyset$ whenever $n\neq m$, 
  \Cref{rmkMorphismsOutOfQuotient} indeed tells us this is sufficient to define $f$. 

  Now we define $g:\N_{(co)fin} \to \langle G \rangle / R$. 
  On a finite subset $I$, we define $g(I) = \bigvee_{i\in I} p_i$, 
  and on a cofinite subset $J$, we define $g(J) = \bigwedge _{i \in J^C} \neg p_i$. 
  Note that in these cases we indeed have $I,J^C$ are finite, so these are well-defined elements. 

  We need to show that $g$ is a Boolean morphism. 
  \begin{itemize}
    \item 
      By deMorgan's laws, $g$ preserves $\neg$:
      for $I$ finite we have
      \begin{equation}
      \neg g(I) = \neg (\bigvee_{i\in I} p_i) = \bigwedge_{i\in I} \neg p_i = g(I^C)
      \end{equation}
      And for $J$ cofinite, we apply similar reasoning. 
    \item To see that $g$ preserves $\vee$, we need to check three cases
      \begin{itemize}
        \item If both $I,J$ are finite, then 
        \begin{equation} 
          g(I \cup J) = \bigvee_{i\in I \cup J} p_i= \bigvee_{i\in I} p_i \vee \bigvee_{j\in J} p_j 
          = g(I) \vee g(J)
        \end{equation}
        and we're done. 
      \item If both $I,J$ are cofinite, we have
        \begin{equation}
          g(I) \vee g(J) = 
          \bigwedge_{i \in I^C} \neg p_i \vee 
          \bigwedge_{j \in J^C} \neg p_j 
          = 
          \bigwedge_{i\in I^C} 
          \bigwedge_{j \in J^C}(\neg p_i \vee  \neg p_j) 
        \end{equation}
        Now note that in $\langle G \rangle / R$, we have 
        \begin{equation}
          \neg p_i \vee \neg p_j = \neg ( p_i \wedge p_j) = 
          \begin{cases}
            p_i \text{ if } i = j\\
            1 \text{ if } i \neq j  
          \end{cases}
        \end{equation}
        Therefore, we can leave out the case that $i\neq j$ in the calculation of the above meet, and
        \begin{equation}
          \bigwedge_{i\in I^C} 
          \bigwedge_{j \in J^C}(\neg p_i \vee  \neg p_j)  
          = 
          \bigwedge_{i \in (I^C \cap J^C)} \neg p_i
          = 
          \bigwedge_{i \in (I \cup J)^C} \neg p_i 
        \end{equation}
        as $I\cup J$ must also be cofinite, this equals 
          $ g( I \cup J)$. 
        \item 
          If $I$ is finite and $J$ cofinite, we have 
          that $I\cup J$ is cofinite, hence 
          \begin{equation}
            g(I\cup J) = \bigwedge_{k\in (I \cup J)^C} \neg p_k
            = \bigwedge_{k \in (J^C -I)} \neg p_k
          \end{equation}
          Now note that 
          whenever $i\neq k$, we have 
          \begin{equation}
            p_i = (p_i \wedge \neg p_k) \vee (p_i \wedge p_k) = 
            (p_i \wedge \neg p_k) \vee 0 = p_i \wedge \neg p_k
          \end{equation}
          Hence by absorption
          \begin{equation} 
            (p_i \vee \neg p_k)  =
              \begin{cases}
                1 \text{ if } i = k \\
                \neg p_k \text{ if } i \neq k
              \end{cases}
          \end{equation}
          As for all $k\in J^C-I$ and all $i\in I$ we have $k\neq i$, we may thus write
          \begin{equation}
            \bigwedge_{k \in (J^C - I)} \neg p_k = 
            \bigwedge_{k \in (J^C - I)} (\neg p_k \vee (\bigvee_{i\in I} p_i))
          \end{equation}
          But now we can use that adding $1$ in a meet does not change the meet, and see that 
          \begin{equation}
            \bigwedge_{k \in (J^C - I)} (\neg p_k \vee (\bigvee_{i\in I} p_i))
            = 
            \bigwedge_{j \in J^C} (\neg p_j \vee (\bigvee_{i\in I} p_i))
          \end{equation}
          And using distributivity rules, we can see that 
          \begin{equation}
            \bigwedge_{j \in J^C} (\neg p_j \vee (\bigvee_{i\in I} p_i))
            = 
            (\bigwedge_{j \in J^C} \neg p_k) \vee (\bigvee_{i\in I} p_i)
          \end{equation}
          From which we may conclude that $g(I\cup J) = g(I) \cup g(J)$. 
      \end{itemize}
    \item The case for $\wedge$ is completely dual to the case for $\vee$. 
  \end{itemize}
We conclude that $g$ is a Boolean morphism. 
Furthermore, $g$ and $f$ are each other's inverse, thus the Boolean algebras are isomorphic. 
\end{proof}
