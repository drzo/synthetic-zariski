%We present some work in progress, 
%applying lessons about the Zariski topos from synthetic algebraic geometry \cite{draft} 
%to the topos of light condensed sets \cite{Scholze}. 
%Specifically, we 
We propose a variation of
the axiomatization of Zariski (higher) topos in synthetic algebraic geometry
to an axiomatization of (separable) Stone spaces, with Stone duality,
within a univalent homotopy type theory.  This satisfies LLPO, and is reminiscent
of the notion of light condensed sets.

\cite{draft}. 

\medskip

%\section{Axioms} 
We denote the type of countably presented Boolean algebras by $\Boole$.
Given a Boolean algebra $B$, we define $Sp(B)$, the spectrum of $B$ as the set of 
Boolean morphisms from $B$ to $2$. 
 A type of the form $Sp(B)$ for $B:\Boole$ is called Stone.


 Two motivating examples of elements of $\Boole$ are as follows:
 \begin{itemize}
   \item $C$ is the free Boolean algebra on countably many generators $(p_n)_{n\in\mathbb N}$. 
     The corresponding set $Sp(C)$ is Cantor space $2^\mathbb N$. 
   \item The Boolean algebra $ B_\infty$ is given by the quotient of $C$ by the relations $p_n\wedge p_m = 0$ for $n\neq m$. 
 The corresponding set $Sp(B_\infty)$ is $\Noo$. 
% the set of binary-sequences which hit $1$ at most once. 
  \end{itemize} 

\begin{axiom}[Stone duality]
  For any countably presented Boolean algebra, there is an isomorphism $B \simeq 2^{Sp(B)}$
\end{axiom}
It follows from Stone duality that being Stone is a proposition and $Sp$ defines an embedding from $\Boole$ 
to any universe $\mathcal U$. We denote call its image $\Stone$. 

Both $\Stone$ and $\Boole$ have a natural structure of a category, and 
Stone duality gives that these categories are anti-equivalent. 

Furthermore Stone duality gives us that any map from $X:\Stone$ to $\mathbb N$ is uniformly continuous
when we endow $X$ with the topology generated by decidable subsets. 

\begin{axiom}[Surjections are Formal Surjections]
  A map $Sp(B')\to Sp(B)$ is surjective iff the corresponding map $B \to B'$ is injective.
\end{axiom} 
%Note that in the category of Boolean algebras, a map is injective iff it is mono. 
%Hence the above axiom can also be stated as surjections being exactly epimorphisms. 
%Note also that if 1\neq 0, then 2 -> B is injective, hence Sp(B) -> 2 surjective, hence
%Sp(B) inhabited. 
%\begin{axiom}[Inhabited spectra of nontrivial algebras]
%  For any $B:\Boole$ with $1 \neq 0$, $Sp(B)$ is merely inhabited. 
%\end{axiom} 
%\begin{axiom}[Stone truncation]
%  For any $X: \Stone$ we have $\neg \neg X \to  ||X||$.
%\end{axiom} 
In particular, this axiom tells us that if $B$ is nontrivial, 
$Sp(B)$ is merely inhabited.
%
%
The sum of the maps
$\Noo \to \Noo$
%$\mathbb N_\infty \to \mathbb N_\infty$ 
sending $n$ to $2n,2n+1$ respectively
is surjective but has no section. 
This implies that $\Noo$ is not projective and LLPO holds. 
However, the negation of WLPO follows from Axiom 1.  

\begin{axiom}[Local choice]
  Given $X$ Stone, $E,F$ arbitrary types, a map $E \to F$ and $E\twoheadrightarrow F$ surjective, 
  there is some $Y$ Stone,
    a surjection $Y \twoheadrightarrow X$ and a map $Y\to E$ such that the following diagram commutes:
    \begin{equation*}\begin{tikzcd}
      Y \arrow [d, two heads,dashed] \arrow [r,dashed] & E \arrow[d,""',two heads]\\
      X \arrow[r] & F
    \end{tikzcd}\end{equation*}  
\end{axiom} 

\begin{axiom}[Dependent Choice]
  Given a family of types $E_n$ and $R_n:E_n\rightarrow E_{n+1}\rightarrow {\mathcal U}$ such that
  for all $n$ and $x:E_n$ there exists $y:E_n$ with $p:R~x~y$ then given $x_0:E_0$ there exists
  $u:\Pi_{n:\N}E_n$ and $v:\Pi_{n:\N}R~(u~n)~(u~(n+1))$ and $u~0 = x_0$.
\end{axiom}

We are working on the proof that these axioms can be verified in a model of HoTT, similar
to the model presented in \cite{draft}.
