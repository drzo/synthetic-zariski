% latexmk -pdf -pvc main.tex
\documentclass{../util/zariski}


\title{Synthetic Stone Duality}

\begin{document}

\author{Felix Cherubini, Thierry Coquand, Freek Geerligs and Hugo Moeneclaey}

\maketitle

%\begin{abstract}
%In synthetic algebraic geometry (SAG) \cite{draft}, we study finitely presented algebras over a commutative ring. 
%In this work, we study countably presented Boolean algebras instead. 
%Where the finitely presented algebras over a commutative ring induce a Zariski topos, 
%%the opposite category of these 
%the countably presented Boolean algebras induce the topos of light condensed sets \cite{Scholze}. 
%\cite{draft} proposes an axiomatization of the Zariski topos in univalent homotopy type theory \cite{hott}. 
%In this work, we propose similar axioms, which we expect to be modelled by light condensed sets. 
%% Furthermore, spectra of countably presented Boolean algebras correspond to quotients of Cantor space
%% which is cool because reasons
%\end{abstract} 
%
\rednote{The following is a collection of notes on work in progress.}

\rednote{I'm cleaning up, not all results that were in older versions have a place yet.}
\tableofcontents

% Logic and Topology
% - Building blocks, explaining the types of Stone, Boole (and what we mean with countable)
% - Rules, stating the axioms and first consequences, including the omniscience principles
% - Topology on propositions, mentioning under what constructions open propositions are closed
% - Examples of closed/open propositions, explaining why Boole is discrete and Stone Hausdorff
% - Topology on Stone spaces, including the classification of closed spaces.
% - Compact Hausdorff spaces. 
% Directed Univalence
% - Tychonov
% - Directed univalence, link to Phoa
% Cohomology
% - Cohomology and the interval
% Appendix, 
%  - alternative formulations of axiom 2
%  - more details on technical constructions
%  - colimit presentation of countably presented Boolean algebras (I'm not sure where we actually use this)
%  - scott continuity instead of axiom 1 



%\section*{Introduction}
%
Algebraic geometry is the study of solutions of non-linear equations using methods from geometry.
Most prominently, algebraic geometry was essential in the proof of Fermat's last theorem by Wiles and Taylor.
The central geometric objects in algebraic geometry are called \emph{schemes}.
Their basic building blocks are called \emph{affine schemes},
where, informally, an affine scheme corresponds to a solution sets of polynomial equations.
While this correspondence is clearly visible in the functorial approach to algebraic geometry and our synthetic approach,
it is somewhat obfuscated in the most commonly used, topological appraoch.

In recent years,
formalization of the intricate notion of affine schemes
received some attention as a benchmark problem
-- this is, however, \emph{not} a problem addressed by this work.
Instead, we use a synthetic approach to algebraic geometry,
very much alike to that of synthetic differential geometry.
This means, while a scheme in classical algebraic geometry is a complicated compound datum,
we work in a setting where schemes are types,
with an additional property that can be defined within our synthetic theory.

Following ideas of Ingo Blechschmidt and Anders Kock  (\cite{ingo-thesis}, \cite{kock-sdg}[I.12]),
we use a base ring $R$, which is local and satisfies an axiom reminiscent of the Kock-Lawvere axiom.
A more general axiom, is called \emph{synthetic quasi coherence (SQC)} by Blechschmidt and
a version quatifying over external algbras is called the \emph{comprehensive axiom}\footnote{
  In \cite{kock-sdg}[I.12], Kock's ``axiom $2_k$'' could equivalently be Theorem 12.2,
  which is exactly our synthetic quasi coherence axiom, except that it only quantifies over external algebras.
}
by Kock.
The exact concise form of the SQC axiom we use, was noted by David Jaz Myers in 2018 and communicated to the first author.

Before we state the SQC axiom, let us take a step back and look at the basic objects of study in algebraic geometry,
solutions of polynomial equations.
Given a system of polynomial equations
\begin{align*}
  p_1(X_1, \dots, X_n) &= 0\rlap{,} \\
  \vdots\quad\quad\;\;   \\
  p_m(X_1, \dots, X_n) &= 0\rlap{,}
\end{align*}
the solution set
$\{ x : R^n \mid \forall i.\; p_i(x_1, \dots, x_n) = 0 \}$
is in canonical bijection to the set of $R$-algebra homomorphisms
\[ \Hom_R(R[X_1, \dots, X_n]/(p_1, \dots, p_m), R) \]
by identifying a solution $(x_1,\dots,x_n)$ with the homomorphism that maps each $X_i$ to $x_i$.
Conversely, for any $R$-algebra $A$, which is merely of the form $R[X_1, \dots, X_n]/(p_1, \dots, p_m)$,
we define the \emph{spectrum} of $A$ to be
\[
  \Spec A \colonequiv \Hom_R(A, R)
  \rlap{.}
\]
In contrast to classical, non-synthetic algebraic geometry,
where this set needs to be equipped with additional structure,
we postulate axioms that will ensure that $\Spec A$ has the expected geometric properties.
Namely, SQC is the statement that, for all finitely presented $R$-algebras $A$, the canonical map
  \begin{align*}
    A&\xrightarrow{\sim} (\Spec A\to R) \\
    a&\mapsto (\varphi\mapsto \varphi(a))
  \end{align*}
is an equivalence.
A prime example of a spectrum is $\A^1\colonequiv \Spec R[X]$,
which turns out to be the underlying set of $R$.
With the SQC axiom,
\emph{any} function $f:\A^1\to \A^1$ is given as a polynomial with coefficients in $R$.
In fact, all functions between affine schemes are given by polynomials.
Furthermore, for any affine scheme $\Spec A$,
the axiom ensures that
the algebra $A$ can be reconstructed as the algebra of functions $\Spec A \to R$,
therefore establishing a duality between affine schemes and algebras.

The Kock-Lawvere axiom used in synthetic differential geometry,
might be stated as the SQC axiom restricted to (external) \emph{Weil-algebras},
whose spectra correspond to pointed infinitesimal spaces.
These spaces can be used in both, synthetic differential and algebraic geometry,
in very much the same way.

In the accompanying formalization \cite{formalization} of some basic results,
we use a setup which was already proposed by David Jaz Myers
in a conference talk (\cite{myers-talk1, myers-talk2}).
On top of Myers' ideas,
we were able to define schemes, develop some topological properties of schemes,
and construct projective space.

An important, not yet formalized result
is the construction of cohomology groups.
This is where the \emph{homotopy} type theory really comes to bear --
instead of the hopeless adaption of classical, non-constructive definitions of cohomology,
we make use of higher types,
for example the $k$-th Eilenberg-MacLane space $K(R,k)$ of the group $(R,+)$.
As an analogue of classical cohomology with values in the structure sheaf,
we then define cohomology with coefficients in the base ring as:
\[
  H^k(X,R):\equiv \|X\to K(R,k)\|_0
  \rlap{.}
\]
This definition is very convenient for proving abstract properties of cohomology.
For concrete calculations we make use of another axiom,
which we call \emph{Zariski-local choice}.
While this axiom was conceived of for exactly these kind of calculations,
it turned out to settle numerous questions with no apparent connection to cohomology.
One example is the equivalence of two notions of \emph{open subspace}.
A pointwise definition of openness was suggested to us by Ingo Blechschmidt and
is very convenient to work with.
However, classically, basic open subsets of an affine scheme are given
by functions on the scheme and the corresponding open is morally the collection of points where the function does not vanish.
With Zariski-local choice, we were able to show that these notions of openness agree in our setup.

Apart from SQC, locality of the base ring $R$ and Zariski-local choice,
we only use homotopy type theory, including univalent universes, truncations and some very basic higher inductive types.
Roughly, Zariski-local choice states, that any surjection into an affine scheme merely has sections on a \emph{Zariski}-cover.
The latter, internal, notion of cover corresponds quite directly to the covers in the site of the \emph{Zariski topos},
which we use to construct a model of homotopy type theory with our axioms.

More precisely, we can use the \emph{Zariski topos} over any base ring.
Toposes built using other Grothendieck topologies, like for example the étale topology, are not compatible with Zariski-local choice.
We did not explore whether an analogous setup can be used for derived algebraic geometry
\footnote{Here, the word ``derived'' refers to the rings the algebraic geometry is built up from -- instead of the 0-truncated rings we use, ``derived'' algebraic geometry would use simplicial or spectral rings.
  Sometimes, ``derived'' refers to homotopy types appearing in ``the other direction'', namely as the values of the sheaves that used.
  In that direction, our theory is already derived, since we use homotopy type theory.
  Practically that means that we expect no problems when expanding our theory of synthetic schemes to what classic algebraic geometers
  call ``stacks''.
}
-- meaning that the 0-truncated rings we used are replaced by higher rings.
This is only because for a derived approach, would have to work with higher monoids, which is currently infeasible
-- we are not aware of any obstructions for, say, an SQC axiom holding in derived algebraic geometry.

In total, the scope of our theory so far, includes quasi-compact, quasi-separated schemes of finite type over an arbitrary ring.
These are all finiteness assumptions, that were chosen for convenience and include examples like closed subspaces of projective space,
which we want to study in future work, as example applications.
So far, we know that basic internal constructions, like affine schemes, correspond to the correct classical external constructions.
This can be expanded using our model, which is of course also important to ensure the consistency of our setup.


\section{Stone types}
\subsection{Homotopy type theory}

\begin{lemma}%
  \label{kraus-glueing}
  Let $X$ and $I$ be types.
  For $(U_i:X \to \Prop)_{i:I}$ and $P:U_i\to \nType{0}$, we have the following glueing property: \\
  If for each $i:I$ there is a dependent function $s_i:(x:U_i)\to P(x)$ together with
  proofs of equality on intersections $p_{ij}:(x:U_i\cap U_j)\to (s_i(x)=s_j(y))$,
  then there is a globally defined dependent function $s:(x:X) \to P(x)$,
  such that for all $x:X$ and $i:I$ we have $U_i(x) \to s(x)=s_i(x)$
\end{lemma}

\begin{proof}
  We define $s$ pointwise.
  Let $x:X$.
  Using a Lemma of Kraus and the $p_{ij}$, we get a factorization
  \[ \begin{tikzcd}[row sep=0mm]
    \sum_{i:I} U_i(x) \ar[rr, "s_{\pi_1(\_)}(x)"]\ar[rd] & & P(x) \\
    & \|\sum_{i:I} U_i(x)\|_{-1}\ar[ru,dashed] &
  \end{tikzcd} \]
-- which defines a unique value $s(x):P(x)$.
\end{proof}

\subsection{Subtypes and Logic}

We use the notation $\exists_{x:X}P(x)\colonequiv \|\sum_{x:X}P(x)\|$.
We use $+$ for the coproduct of types and for types $A,B$ we write
\[ A\vee B\colonequiv \| A+B \|\rlap{.}\]

We will use subtypes extensively.

\begin{definition}
  Let $X$ be a type.
  A \notion{subtype} of $X$ is a function $U:X\to\Prop$ to the type of propositions.
  We write $U\subseteq X$ to indicate that $U$ is as above.
  If $X$ is a set, a subtype may be called \notion{subset} for emphasis.
\end{definition}

We will freely switch between subtypes $U:X\to\Prop$ and the corresponding embeddings
\[
  \begin{tikzcd}
    \sum_{x:X}U(x) \ar[r,hook] & X
  \end{tikzcd}
  \rlap{.}
\]
In particular, if we write $x:U$ for a subtype $U:X\to\Prop$, we mean that $x:\sum_{x:X}U(x)$ -- but we might silently project $x$ to $X$.

\begin{definition}
  Let $I$ and $X$ be types and $U_i:X\to\Prop$ a subtype for any $i:I$.
  \begin{enumerate}[(a)]
  \item The \notion{union} $\bigcup_{i:I}U_i$ is the subtype $(x:X)\mapsto \exists_{i:I}U_i(x)$.
  \item The \notion{intersection} $\bigcap_{i:I}U_i$ is the subtype $(x:X)\mapsto\prod_{i:I}U_i(x)$.
  \end{enumerate}
\end{definition}

We will use common notation for finite unions and intersections.
The following formula hold:

\begin{lemma}
  Let $I$, $X$ be types, $U_i:X\to\Prop$ a subtype for any $i:I$ and $V,W$ subtypes of $X$.
  \begin{enumerate}[(a)]
  \item Any subtype $P:V\to\Prop$ is a subtype of $X$ given by $(x:X)\mapsto\sum_{x:V}P(x)$.
  \item $V\cap \bigcup_{i:I} U_i=\bigcup (V\cap U_i)$.
  \item If $\bigcup_{i:I}U_i=X$ we have $V=\bigcup_{i:I}U_i\cap V$.
  \item If $\bigcup_{i:I}U_i=\emptyset$, then $U_i=\emptyset$ for all $i:I$.
  \end{enumerate}
\end{lemma}

\subsection{Algebra}

\begin{definition}%
  A commutative ring $R$ is \notion{local} if $1\neq 0$ in $R$ and
  if for all $x,y:R$ such that $x+y$ is invertible, $x$ is invertible or $y$ is invertible.
\end{definition}

\begin{definition}%
  Let $R$ be a commutative ring.
  A \notion{finitely presented} $R$-algebra is an $R$-algebra $A$,
  such that there merely are natural numbers $n,m$ and polynomials $f_1,\dots,f_m:R[X_1,\dots,X_n]$
  and an equivalence of $R$-algebras $A\simeq R[X_1,\dots,X_n]/(f_1,\dots,f_m)$.
\end{definition}

\begin{definition}%
  \label{regular-element}
  Let $A$ be a commutative ring.
  An element $r:A$ is \notion{regular},
  if the multiplication map $r\cdot\_:A\to A$ is injective.
\end{definition}

\begin{lemma}%
  \label{units-products-regular}
  Let $A$ be a commutative ring.
  \begin{enumerate}[(a)]
  \item All units of $A$ are regular.
  \item If $f$ and $g$ are regular, their product $fg$ is regular.
  \end{enumerate}
\end{lemma}

\begin{example}
  The monomials $X^k:A[X]$ are regular.
\end{example}

\begin{lemma}%
  \label{fg-ideal-local-global}
  Let $A$ be a commutative ring and $f_1,\dots,f_n:A$.
  For finitely generated ideals $I_i\subseteq A_{f_i}$,
  such that $A_{f_if_j}\cdot I_i=A_{f_if_j}\cdot I_j$ for all $i,j$,
  there is a finitely generated ideal $I\subseteq A$,
  such that $A_{f_i}\cdot I=I_i$ for all $i$.
\end{lemma}

\begin{proof}
  Choose generators 
  \[ \frac{g_{i1}}{1},\dots,\frac{g_{ik_i}}{1} \]
  for each $I_i$.
  These generators will still generate $I_i$, if we multiply any of them with any power of the unit $\frac{f_i}{1}$.
  Now
  \[ A_{f_if_j}\cdot I_i\subseteq A_{f_if_j}\cdot I_j \]
  means that for any $g_{ik}$, we have a relation
  \[ (f_if_j)^l g_{ik}=\sum_{l}h_{l}g_{jl}\]
  for some power $l$ and coefficients $h_{l}:A$.
  This means, that $f_i^lg_{ik}$ is contained in $I_j$.
  Multiplying $f_i^lg_{ik}$ with further powers of $f_i$ or multiplying $g_{jl}$ with powers of $f_j$ does not change that.
  So we can repeat this for all $i$ and $k$ to arrive at elements $\tilde{g_{ik}}:A$,
  which generate an ideal $I\subseteq A$ with the desired properties.
\end{proof}

\ignore{
    - injective/embedding/-1-truncated map
  pushouts:
    - inclusions are jointly surjective,
    - pushouts of embeddings between sets are sets
  subtypes:
    - embeddings (composition, multiple definitions, relation to injection)  
    - we freely switch between predicates and types
    - subtypes of subtypes are subtypes
  pullbacks:
    - pasting (reference)
    - pullback of subtype = composition
  algebra:
    - free comm algebras, quotients
    - other definitions of polynomials
    - fp closed under: quotients, adjoining variables, tensor products
}

\section{Stone duality}
\subsection{Statement of the axioms}
We always assume there is a commutative ring $R$.
Sometimes we will assume $R$ has additional properties, or, more generally,
axioms hold that involve $R$.
We will always mention which of these axiom are needed to prove each statement,
by listing the shorthands introduced in the axioms below.

\begin{axiom}[Loc]%
  \label{loc}\index{Loc}
  $R$ is a local ring.
\end{axiom}

\begin{axiom}[SQC]%
  \label{sqc}\index{sqc}
  For any finitely presented $R$-algebra $A$, the homomorphism
  \[ a \mapsto (\varphi\mapsto \varphi(a)) : A \to (\Spec A \to R)\]
  is an isomorphism of $R$-algebras.
\end{axiom}

\begin{axiom}[Z-choice]%
  \label{Z-choice}\index{Z-choice}
  Let $A$ be a finitely presented $R$-algebra
  and let $B : \Spec A \to \mU$ be a family of inhabited types.
  Then there merely exists
  a finite list of coprime elements $f_1, \dots, f_n \in A$
  together with dependent functions $s_i : \Pi_{x : D(f_i)} B(x)$.
  As a formula:
  \[ (\Pi_{x : \Spec A} \propTrunc{B(x)}) \to
     \propTrunc{ \Sigma_{n : \N} \Sigma_{f_1, \dots, f_n : A}
      ((f_1, \dots, f_n) = (1)) \times
      \Pi_i \Pi_{x : D(f_i)} B(x) }
     \rlap{.}
  \]
\end{axiom}

\subsection{First consequences}

\begin{proposition}[using \axiomref{sqc}]%
  For all finitely generated $R$-algebras $A$ and $B$ we have
  \[ \Hom(\Spec B, \Spec A)=\Hom_{\Alg{R}}(A,B)\]
  -- where the equality is induced by exponentiation with $R$.
\end{proposition}

\begin{proposition}[using \axiomref{sqc}, \axiomref{loc}]%
  \label{nilpotence-double-negation}\label{non-zero-invertible}\label{generalized-field-property}
  
  \begin{enumerate}[(a)]
  \item An element $x:R$ is nilpotent,
    if and only if $\neg \neg (x=0)$.
  \item An element $x:R$ is invertible,
    if and only if $x\neq 0$.
  \item A vector $x:R^n$ is non-zero,
    if and only if one of its entries is invertible.
  \end{enumerate}
\end{proposition}

\subsection{Anti-equivalence of $\Boole$ and $\Stone$}

\begin{remark}\label{SpIsAntiEquivalence}
Stone types will take over the role of affine scheme from \cite{draft}, 
and we repeat some results here. 
Analogously to Lemma 3.1.2 of \cite{draft}, 
for $X$ Stone, Stone duality tells us that $X = Sp(2^X)$. 
%
Proposition 2.2.1 of \cite{draft} now says that 
$Sp$ gives a natural equivalence 
\begin{equation}
   Hom_{\Boole} (A, B) = (Sp(B) \to Sp(A))
\end{equation}
Therefore $Sp$ is an embedding from $\Boole$ to any universe of types, and $\isSt$ is a proposition.

Its image, $\Stone$ also has a natural category structure.
By the above and Lemma 9.4.5 of \cite{hott}, the map $Sp$ defines a dual equivalence of categories between $\Boole$ and $\Stone$.
\end{remark}

\begin{lemma}\label{SpectrumEmptyIff01Equal}
  For $B:\Boole$, we have $0=_B1$ iff $\neg Sp(B)$.
\end{lemma}
\begin{proof}
  Note that whenever $0=1$ in $B$, there is no map $B\to 2$ respecting both $0$ and $1$ as $0\neq 1$ in $2$. 
  Thus $\neg Sp(B)$ whenever $0=1$ in $B$. 
  % 
  Conversely, if $\neg Sp(B)$, then $Sp(B) = \emptyset$, which is also the spectrum of the trivial Boolean algebra. 
  As $Sp$ is an embedding, $B$ is equivalent to the trivial Boolean algebra, and $0=_B1$. 
\end{proof}

%\begin{corollary}\label{MoreConcreteCompleteness}
%  By the above and propositional completeness, we have that $||Sp(B)||$ iff $0\neq_B1$. 
%\end{corollary}


\begin{remark}\label{StoneClosedUnderPullback}
%  By \Cref {SpIsAntiEquivalence} and the fact that that countably presented Boolean algebras form a 
%  finitely cocomplete category (\Cref{CoCompletenessBoole}), the category of Stone spaces is complete. 
  By \Cref {SpIsAntiEquivalence} and the fact that that countably presented Boolean algebras are closed under pushouts, 
  the category of Stone spaces is closed under pullbacks. 
\end{remark}

We conclude this section on the anti-equivalence of Stone and $\Boole$ by a relating surjections to injections. 
This theorem is actually equivalent to completeness of propositional logic, which we'll discuss in 
\Cref{NotesOnAxioms}. 

\begin{theorem}\label{FormalSurjectionsAreSurjections}
  Let $f:A\to B$ be a map of countably presented Boolean algebras. 
  If $f$ is injective, then the corresponding map $(\cdot) \circ f : Sp(B) \to Sp(A)$ is surjective. 
\end{theorem}

\begin{proof}
  Assume $f$ injective and let $x:Sp(A)$.
  By \Cref{FiberConstruction}, we have that $\left(\sum\limits_{y:Sp(B)} y\circ f = x \right) = Sp(B/R) $
  for $R=f(G)$ for some countable $G\subseteq A$ with $x(g) = 0$ for all $g\in G$. 
  By propositional completeness and \Cref{SpectrumEmptyIff01Equal}, 
  it's sufficient to show that $0\neq_{B/R}1$. 
  Note that $0=_{B/R} 1$ iff 
  $1 =_B \bigvee R_0$ for some $R_0\subseteq R$ finite. 
  But then $$1 = \bigvee f(G_0) = f(\bigvee  G_0)$$ for some $G_0\subseteq G$ finite. 
  And as $f$ is injective, $\bigvee G_0 = 1$. 
  However, 
  $$
  x(\bigvee G_0) = 
  x(\bigvee_{g\in G_0} g ) = \bigvee_{g \in G_0} x(g) = \bigvee_{g\in G_0} 0 = 0$$
  And as $x(1) = 1$, we get a contradiction. Therefore $0\neq_{B/R} 1$ as required. 
\end{proof}  
The converse to the above theorem is true as well, regardless of propositional completeness:
\begin{lemma}\label{SurjectionsAreFormalSurjections}
If $f:A\to B$ is a map in $\Boole$ and $(\cdot) \circ f :Sp(B) \to Sp(A)$ is surjective, 
$f$ is injective. 
\end{lemma}
\begin{proof}
  Suppose precomposition with $f$ is surjective. 
  Let $a:A$ be such that $f(a)= 0$. 
  By assumption, for every $x:A\to 2$, there is a $y:B\to 2$ with $y\circ f = x$. 
  Consequentely $x(a) = y(f(a)) = y(0) = 0$. 
  So $x(a) = 0$ for every $x:Sp(A)$. 
  Thus $Sp(A) = Sp(A/\{a\})$, and as $Sp$ is an embedding, 
  $A \simeq A/\{a\}$, and $a = 0$ in $A$. 
  So whenever $f(a) = 0$, we have $a=0$. Thus $f$ is injective. 
\end{proof}

\subsection{Principles of omniscience}
In constructive mathematics, we do not have general access to the law of excluded middle (LEM).
There are some principles that are weaker than LEM, which can be used to describe 
the proof-theoretic strength of a logical system, called principles of omniscience.
In this section, we will show that two of them (Markov's principle and LLPO) hold, 
and one (WLPO) fails in our system.

\begin{theorem}[The negation of the weak lesser principle of omniscience ($\neg$WLPO)]
  It is not the case that the statement %There is no method which given $\alpha:2^\mathbb N$ decides whether 
  $\forall_{n:\mathbb N} \alpha(n) = 0$ is decidable for general $\alpha:2^\mathbb N$. 
\end{theorem}
\begin{proof}
  Such a decission method would give a function $f:2^\mathbb N \to 2$ such that 
  $f(\alpha) = 0$ iff $\forall_{n:\mathbb N} \alpha (n)= 0$. 
  By Stone duality, there must be some $c:C$ with 
  $f(\alpha) = 0 \iff \alpha(c) = 0$. 
  $c$ is expressable using only finitely many generators $(p_n)_{n\leq N}$. 
  Now consider $\beta,\gamma:\langle C \rangle \to 2$ given by $\beta(p_n) = 0$ for all $n:\mathbb N$ and
  $\gamma(p_n) = 0$ iff $n\leq N$. 
  Note that these functions are equal on $(p_n)_{n\leq N}$, therefore, $\beta(c) = \gamma(c)$. 
  However, $f(\beta) = 0$ and $f(\gamma) = 1$.
  We thus have a contradiction, thus a decission method as required doesn't exist. 
\end{proof}

The following result is due to David W\"arn:
\begin{theorem}[Markov's principle (MP)]\label{MarkovPrinciple}
  For $\alpha:\Noo$, we have that 
  \begin{equation}
    (\neg (\forall_{n:\mathbb N} \alpha (n)= 0)) \to \Sigma_{n:\mathbb N} \alpha (n)= 1
  \end{equation}
\end{theorem}
\begin{proof}
  Assume $\neg (\forall_{n:\mathbb N} \alpha (n)= 0)$.
%  Suppose $\alpha \neq \infty$.%, then it is not the case that $\alpha(n) = 0$ for all $n:\N$. 
  It is sufficient to show that $2/\{\alpha(n)|n\in\N\}$ is the trivial Boolean algebra. 
  It will then follow that there is a finite subset $N_0\subseteq \N$ 
  with $\bigvee_{i:N_0} \alpha(i) = 1$.
  As $\alpha(i) \in \{0,1\}$ and $\alpha(i) = 1$ for at most one $i$, it then follows that 
  there exists an unique $n\in\mathbb N$ with $\alpha(n) = 1$. 

  To show that $2/\{\alpha(n)|n\in\N\}$ is trivial, we will show it has an empty spectrum. 
  Suppose $x: 2 \to 2$ is such that $x(\alpha(n)) = 0$ for every $n:\N$. 
  As $x(1) = 1\neq_2 0$, we must have for every $n:\N$ that $\alpha(n) \neq 1$. 
  But then $\alpha(n) = 0$, contradicting our assumption. 
  We get a contradicition and there thus there are no terms of the spectrum of $2/\{\alpha(n)|n\in\N\}$ as required. 
\end{proof}
\begin{corollary}
  For $\alpha:2^\mathbb N$, we have that 
  \begin{equation}
    (\neg (\forall_{n:\mathbb N} \alpha (n)= 0)) \to \Sigma_{n:\mathbb N} \alpha (n)= 1
  \end{equation}
\end{corollary}
\begin{proof}
  Given $\alpha:2^\mathbb N$, consider the sequence $\alpha':\Noo$ satisfying $\alpha'(n) = 1$ iff 
  $n$ is minimal with $\alpha(n) = 1$. Then apply the above theorem.
\end{proof}

\begin{theorem}[The lesser limited principle of omniscience (LLPO)]\label{LLPO}
  For $\alpha:\N_\infty$, 
  we have that 
  \begin{equation}\label{eqnLLPO}
    \forall_{k:\N} \alpha(2k) = 0  \vee \forall_{k:\N} \alpha(2k+1) = 0
  \end{equation}
\end{theorem}
\begin{proof}
%
%  We first will define a map $f:B_\infty \to B_\infty \times B_\infty$. 
%  Because of \Cref{rmkMorphismsOutOfQuotient}, it is sufficient to define $f$ on $(p_n)_{n:\N}$ with 
%  $f(p_n) \wedge f(p_m) = (0,0)$ for $n\neq m$. 
%  To define $f(p_n)$, we use a case distinction on whether $n$ is odd or even. 
  Define $f:B_\infty \to B_\infty \times B_\infty$ as follows:
  \begin{equation}
    f(p_n) =\begin{cases}
      (p_k,0) \text{ if } n = 2k\\
      (0,p_k) \text{ if } n = 2k+1\\
    \end{cases}
  \end{equation}
  By making a case distinction on $n,m$ being odd or even, 
  we can see that 
  $f(p_n) \wedge f(p_m) = (0,0)$ when $n\neq m$, thus $f$ is well-defined. 
  We also claim it is injective.
  Now let $x:B_\infty$ with $f(x) = 0$. 
  We denote $E,O\subseteq \N$ for the even and odd numbers, 
  and we make a case distincition based on \Cref{BinftyTermsWriting}.
  \begin{itemize}
    \item Suppose 
      $x = \bigvee_{i\in I_0} p_i$. 
      Then 
      $$f(x) = (\bigvee_{i\in I_0 \cap E } p_{\frac i2} , \bigvee_{i\in I_0 \cap O } p_{\frac {i-1}2} ) = (0,0)$$
      But now as $p_j\neq 0$ for all $j\in\N$, we must have $I_0 \cap E = \emptyset = I_0 \cap O$. 
      Thus $I_0= \emptyset$, and $x = 0$. 
    \item Suppose 
%      Let $x$ correspond to a cofinite subset of $\N$. Write 
      $x = \bigwedge_{j\in J} \neg p_j$. % for $J$ finite. 
      We will derive a contradiction. %, from which we can conclude that $x=0$ after all. 
      Note that   
      $$f(x) = (\bigwedge_{j\in J \cap E } \neg p_j , \bigwedge_{j\in J \cap O } \neg p_j )$$
      As $f(x) = (0,0)$, we have that 
      $\bigwedge_{j\in J \cap E } \neg p_j =0$ and
      $\bigwedge_{j\in J \cap O } \neg p_j  = 0$.
      However, any finite meet of negations will correspond to a cofinite set,
      in particular it will not correspond to the empty set, and thus not be $0$.
      Thus $f(x)\neq 0$, contradicting the assumption that $f(x) = 0$, hence $x=0$ ex falso. 
  \end{itemize}
  In both cases, we conclude $x=0$, thus $f$ is injective. 
  By \Cref{FormalSurjectionsAreSurjections}, $f$ corresponds to a surjection 
  $s:\Noo + \Noo \to \Noo$.
  Now let $\alpha : \Noo$, 
  let $x:\Noo + \Noo$ be such that $s x = \alpha$. 
  If $x = inl(\beta)$, 
  for any $k:\N$, we have that 
  $$\alpha (p_{2k+1}) = s(x) (p_{2k+1}) = x(f(p_{2k+1})) = inl(\beta) (0,p_k)  = \beta(0) = 0.$$
  Similarly, if $x = inr(\beta)$, we have $\alpha(p_{2k}) = 0$ for all $k:\N$. 
  Thus $\Cref{eqnLLPO}$ holds for $\alpha$ as required. 
\end{proof}

The use of \Cref{FormalSurjectionsAreSurjections}, and hence of propositional completeness, 
was not trivial in the above proof, as the following shows:
\begin{lemma}
  The above function $f$ does not have a retraction. 
\end{lemma}
\begin{proof}
  Suppose $r:B_\infty \times B_\infty \to B_\infty$ is a retraction of $f$. 
  Note that $r(0,1):B_\infty$ is expressable using only finitely many generators $(p_n)_{n\leq N}$
  Note that $r(0,1) \geq r(0,p_k) = p_{2k+1}$ for all $k:\N$. 
  As a consequence, $r(0,1)$ cannot be of the form $\bigvee_{i\in I_0} p_i$, and by \Cref{BinftyTermsWriting}, 
  $r(0,1)$ corresponds to a cofinite subset of $\N$. % = \bigwedge_{i:I_0} \neg p_i$, where $i\leq N$ for $i\in I_0$. 
  By similar reasoning so does $r(1,0)$.% corresponds to a cofinite subset of $\N$. 
  But the intersection of cofinite subsets is cofinite, while 
  $$r(0,1) \wedge r(1,0) = r( (1,0) \wedge (0,1)) = r(0,0) = 0$$
  which gives a contradiction. Thus no retraction exists. 
\end{proof}

We finish with an equivalent formulation of LLPO:
\begin{lemma}\label{corAlternativeLLPO}
  Let $(\phi_n)_{n:\N}, (\psi_m)_{m:\N}$ be families of decidable propositions indexed over $\N$.
  We then have 
  \begin{equation}
    (\forall_{n:\N} \forall_{m:\N} (\phi_n \vee \psi_m) )
    \leftrightarrow
    ((\forall_{n:\N} \phi_n) \vee (\forall_{m:\N} \psi_m) )
  \end{equation}
\end{lemma}
Note that the above statement implies LLPO as 
$\alpha(2n) =0 \vee \alpha(2m+1) =0$ for all $n,m:\mathbb N$ if $\alpha:\Noo$. 
\begin{proof}
  Note that the implication from right to left in the above equation always holds.
  Assume that for all $m,n:\mathbb N$ we have $\phi_n\vee \psi_m$ 
  Consider the sequence $\alpha:2^\mathbb N$ where $\alpha(2n) = 0$ iff $\phi_n$ and 
  $\alpha(2m+1) = 0$ iff $\psi_m$. 
  Let $\beta:\Noo$ be such that $\beta(i) = 1$ iff $i$ is minimal with $\alpha(i) = 1$
  By LLPO, we have that 
  $\beta$ is $0$ on all odd entries or on all even entries. 
  Suppose that $\beta$ hits $0$ on all odd entries. 
  We will show $\psi_m$ for all $m:\N$. 
  As $\beta(2m+1) = 0$, there is some $l<2m+1$ with $\beta(l) = 1$. 
  As $\beta$ hits $0$ on odd entries, $l$ is even. 
  So $\alpha(2n) = 1$ for $n = \frac{l}2$, meaning that $\neg \phi_n$. 
  By assumption, $\phi_n \vee \psi_m$ holds, hence $\psi_m$ must hold. 
  Thus for all $m:\N$, we have $\psi_m$ if $\beta$ hits $0$ on all odd entries. 
  By a symmetric argument, if $\beta$ hits $0$ on all even entries, we have $\psi_n$ for all $n:\N$. 
  We conclude that 
  $((\forall_{n:\N} \phi_n) \vee (\forall_{m:\N} \psi_m) )$ 
  as required. 
\end{proof}


\section{Topology}
In this section, we will define the types of open and closed propositions. 
These will allow us to define a (synthetic) topology  \cite{SyntheticTopologyLesnik} on any type.
We will study this topology on Stone types in particular.

\subsection{Open and closed propositions}
In this section we will introduce a topology on the type of propositions, and 
study their logical properties.
We think of open and closed propositions respectively as countable disjunctions and conjunctions of decidable propositions.
Such a definition is universe-independent, and can be made internally.
\begin{definition}
  We define the types $\Open, \Closed$ of open and closed propositions as follows:
  \begin{itemize}
    \item 
    A proposition $P$ is open iff there merely exists some $\alpha:2^\N$ such that 
      $P \leftrightarrow \exists_{n:\mathbb N} \alpha(n) = 0$. 
    \item 
    A proposition $P$ is closed iff there merely exists some $\alpha:2^\N$ such that 
      $P \leftrightarrow \forall_{n:\mathbb N} \alpha(n) = 0$. 
  \end{itemize}
\end{definition}

\begin{remark}\label{rmkOpenClosedNegation}
  The negation of an open proposition is closed, 
  and by Markov's principle (\Cref{MarkovPrinciple}), the negation of a closed proposition is open. 
  Also by Markov's principle, we have $\neg \neg P \to P$ whenever $P$ is open or closed. 
  By the negation of WLPO (\Cref{NotWLPO}), 
  not every closed proposition is decidable. 
  Therefore, not every open proposition is decidable. 
  % Both therefore and similarly can be used here, by a similar proof we can show it, or we can use that 
  % if $P$ is closed and $\neg P$ is decidable, so is $\neg \neg P = P$. 
  Every decidable proposition is both open and closed, 
  and in \Cref{ClopenDecidable} we shall see the converse. 
\end{remark}
\begin{lemma}\label{ClosedCountableConjunction}
  Closed propositions are closed under countable conjunctions.
\end{lemma}
\begin{proof}
  Let $(P_n)_{n:\N}$ be a countable family of closed propositions. 
  By countable choice, for each 
  $n:\N$ we have an $\alpha_n:2^\N $ 
  such that $P_n \leftrightarrow \forall_{m:\N} \alpha_n(m)  =0$. 
  Consider a surjection $s:\N \to \N \times \N$.
  Let 
  $$\beta(k) = \alpha_{\pi_0(s(k))}(\pi_1 (s(k))).$$
  Note that $\forall_{k:\N} \beta(k) = 0$ iff 
  $\forall_{m,n:\N}\alpha_m(n) = 0$, which happens iff $\forall_{n:\N} P_n$. 
  Hence the countable conjunction of closed propositions is closed. 
\end{proof} 
\begin{lemma}\label{OpenCountableDisjunction}
  Open propositions are closed under countable disjunctions. 
\end{lemma}
\begin{proof}
  By a replacing the universal with existential quantifiers in the above proof. 
\end{proof}
\begin{corollary}\label{ClopenDecidable}
  If a proposition is both open and closed, it is decidable. 
\end{corollary}
\begin{proof}
  If $P$ is open and closed, $P\vee \neg P$ is open, hence
  equivalent to $\neg \neg (P \vee \neg P)$, which is provable. 
\end{proof}

\begin{lemma}\label{ClosedFiniteDisjunction} 
  Closed propositions are closed under finite disjunctions. 
\end{lemma}
\begin{proof}
  We shall show that 
  $(\forall_{n:\N} \alpha(n) = 0 )\vee (\forall_{n:\N} \beta(n) = 0 )$ is closed for any $\alpha,\beta:2^\N$.
  By \Cref{corAlternativeLLPO}, the statement is equivalent to 
  $ \forall_{n:\N}  \forall_{m:\N}  (\alpha(n) = 0 \vee \beta(m) = 0)$, 
  which is a countable conjunction of decidable propositions, 
  hence closed by \Cref{ClosedCountableConjunction}.
\end{proof}
\begin{lemma}\label{OpenFiniteConjunction}
  Open propositions are closed under finite conjunctions. 
\end{lemma}
\begin{proof}
  We need to show that for any $\alpha,\beta:2^\N$, the following proposition is open:
  \begin{equation}\label{eqnConjunctionOpen}
    (\exists_{n:\N} \alpha(n) = 0 )\wedge(\exists_{n:\N} \beta(n) = 0 )
  \end{equation}
  Consider $\gamma:2^\N$ given by 
  $\gamma(l) = 1$ iff there exist some $k,k'\leq l$ with 
  $\alpha(k) = \beta(k') = 0$. 
  As we only need to check finitely many combinations 
  of $k,k'$, this is a decidable property for each $l:\N$ and $\gamma$ is well-defined. 
  Clearly, $\gamma$ witnesses that the proposition in \Cref{eqnConjunctionOpen} is open.
\end{proof}

\begin{lemma}\label{OpenDependentSums}
  Open propositions are closed under dependent sums.
\end{lemma}
\begin{proof}
  First note that for $D$ a decidable proposition, and $X:D \to \Open$,
  by case splitting on $D$, we can see 
  $\Sigma_{d:D} X(d)$ is open.
%
  Then note that for $P$ an open proposition, 
  there exists a sequence of decidable propositions $A_n$ with 
  $P = \Sigma_{n:\N} A_n $.
%
  So for $Y : P \to Open $, the dependent sum $\Sigma_P Y$ is given by 
  $\Sigma_{n:\N} (\Sigma_{a:A_n} Y(n,a))$. 
  which is a countable sum of open propositions. 
  As $\Sigma_P Y$ is a proposition, it is 
  a countable disjunction of open propositions, 
  hence open by \Cref{OpenCountableDisjunction}.
\end{proof}

%\begin{lemma}
%  Closed propositions are closed under dependent sums. 
%\end{lemma}
%\begin{proof}
%  \rednote{TODO, don't know if it's true}
%\end{proof}
%

%\begin{remark}
%  If $P$ is open, $P \to \bot$ is only open if $P$ is decidable, which is not in general the case. 
%  Thus $\Open$ is not closed under dependent products. Neither is $\Closed$. 
%  However, in \Cref{TODO}, we will see that we can mix. 
%\end{remark}

\subsection{Equality in $\Boole$ and $\Stone$}
\begin{lemma}\label{BooleEqualityOpen}
  Whenever $B:\Boole$, $a,b:B$ the proposition $a=_Bb$ is open. 
\end{lemma}
\begin{proof}
  Let $G,R$ be the generators and relations of $B$. 
  Let $a,b$ be represented by $x,y$ in the free Boolean algebra on $G$. 
  Now let $R_n$ denote the first $n$ elements of $R$. 
  Note that $a=b$ iff there exists some $n:\N$ with $x-y \leq \bigvee_{r\in R_n} r$. 
  Furthermore, inequality is decidable in the free Boolean algebra, hence
  $a=b$ is a countable disjunction of decidable propositions, hence open. 
\end{proof}


\begin{corollary}\label{TruncationStoneClosed}
  Whenever $S:\Stone$, $||S||$ is closed. 
\end{corollary}
\begin{proof}
  By \Cref{SpectrumEmptyIff01Equal}, $\neg S$ is equivalent to $0=_B 1$, which is open by the above. 
  Hence $\neg \neg S$ is a closed proposition, and by propositional completeness, so is $||S||$. 
\end{proof}

\begin{remark}\label{ExplicitTruncationStoneClosed}
  \rednote{New check later}
  The above lemma and corollary actually show that if we have an explicit 
  presentation of a Stone space as $S = Sp(2[G] / R)$, 
  we can construct an explicit sequence $\alpha:2^\N$ such that $||S|| \leftrightarrow \forall_{n:\N} \alpha(n) = 0$. 
\end{remark}


\begin{corollary}\label{PropositionsClosedIffStone}
  A proposition $P$ is closed iff it is a Stone space. 
\end{corollary}
\begin{proof}
  By the above, if $S$ is both a Stone space and a proposition, it is closed. 
  Conversely, note that 
  $$
  (\forall_{n:\N} \alpha(n) = 0 )\leftrightarrow Sp(2/\{\alpha(n)| n:\N\}).
  $$
  The latter is a proposition, as there is at most one Boolean map $2/\{\alpha(n)|n:\N\} \to 2$.
\end{proof}

\begin{lemma}\label{StoneEqualityClosed}
  Whenever $S:\Stone$, and $s,t:S$, the proposition $s=t$ is closed. 
\end{lemma}
\begin{proof}
  Suppose $S= Sp(B)$ and let $G$ be the generators of $B$. 
  Note that $s=t$ iff $s(g) =_2 t(g)$ for all $g:G$. 
  As $G$ is countable, and equality in $2$ is decidable, 
  $s=t$ is a countable conjunction of decidable propositions, hence 
  closed. 
\end{proof}
%
The following question was asked by Bas Spitters at TYPES 2024:
\begin{corollary}
  For $S:\Stone$ and $x,y,z:S$ 
  \begin{equation}\label{Apartness}
  x \neq y \to (x\neq z \vee y \neq z)
  \end{equation}
\end{corollary}
\begin{proof}
  As $x\neq y$, we can show that $\neg ( x = z \wedge y = z)$. 
  This in turn implies $\neg \neg ( x \neq  z \vee y \neq  z)$. 
  As, $x\neq z$ and $y \neq z$ are both open propositions, by \Cref{OpenCountableDisjunction} so is their disjunction. 
  By \Cref{rmkOpenClosedNegation}, that disjunction is double negation stable and \Cref{Apartness} follows. 
\end{proof}
\begin{remark}
  If \Cref{Apartness} holds in a type, we say that it's inequality is an apartness relation. 
  By a similar proof as above, it can be shown that in our setting inequality is an apartness relation 
  as soon as equality is open or closed. 
\end{remark}

\subsection{Types as spaces}
The subobject $\Open$ of the type of propositions induces a topology on every type. 
This is the viewpoint taken in synthetic topology. 
We will follow the terminology of \cite{SyntheticTopologyLesnik}, 
other references include \cite{SyntheticTopologyEscardo, TODOSortOutTaylorsReferences}.
%Defining a topology in this way has some benefits, which we summarize in this section. 

\begin{definition}
  Let $T$ be a type, and let $A\subseteq T$ be a subtype. 
  We call $A\subseteq T$ open or closed iff $A(t)$ is open or closed respectively for all $t:T$.
\end{definition}

\begin{remark}
  It follows immediately that any the pre-image of any map of types sends 
  open subtypes to open subtypes, hence is continuous. 
  This is only relevant for a space if the topology we defined above matches the topology one would expect. 
  In \Cref{StoneClosedSubsets}, we shall see that it resembles the standard topology of Stone spaces.
  In \Cref{IntervalClosedSubsets}, we shall see that it is the standard topology for the unit interval. 
\end{remark}

\begin{lemma}[transitivity of openness]
  Let $T$ be a type, let $V\subseteq T$ open and let $W\subseteq V$ open. 
  Then the composite $W\subseteq V\subseteq T$ is open as well. 
\end{lemma}
\begin{proof}
  Denote $W'\subseteq T$ for the composite. 
  Note that $W'(t) = \Sigma_{v:V(t)} W(v)$. 
  As open propositions are closed under dependent sums (\Cref{OpenDependentSums}), 
  $W'(t)$ is an open proposition, as required. 
\end{proof}

\begin{remark}
  As the true proposition is open and openness is transitive, 
  $\Open$ can be called a dominance according to Proposition 2.25 of \cite{SyntheticTopologyLesnik}
\end{remark}



%\begin{remark}
%  Phao's principle is a special case of directed univalence. 
%\end{remark}
%\begin{proof}
%  \rednote{TODO}
%\end{proof}

\subsection{The topology on Stone spaces}
\begin{theorem}\label{StoneClosedSubsets}
  Let $A\subseteq S$ be a subset of a Stone space. TFAE:
  \begin{enumerate}[(i)]
    \item There exists a map $\alpha_{(\cdot)}:S \to 2^\N$ such that 
      $A (x) \leftrightarrow \forall_{n:\N} \alpha_x(n) = 0$ for any $x:S$. 
    \item There exists some countable family 
      $D_n,~{n:\N}$ 
      of decidable subsets of $S$ with $A = \bigcap_{n:\N} D_n$. 
    \item There exists a Stone space $T$ and some map $T\to S$ whose image is $A$. 
    \item $A$ is closed.
  \end{enumerate}
\end{theorem}
\begin{proof}
\item 
  \begin{itemize}
  \item[$(i)\leftrightarrow (ii)$.] 
    $D_n$ and $\alpha_{(\cdot)}$ can be defined from each other by 
%    Define the decidable subsets of $S$ 
     $D_n(x) \leftrightarrow (\alpha_x(n) = 0)$. Then observe that %$A=\bigcap_{n:\N} D_n$ as 
     \begin{equation}
      (\bigcap_{n:\N} D_n) (x) \leftrightarrow 
      \forall_{n:\mathbb N} (\alpha_x(n) = 0) 
%      \leftrightarrow A s. 
     \end{equation}
   \item[$(ii) \to (iii)$.]
      Let $S=Sp(B)$. 
      By Stone duality, we have $d_n,~n:\N$ terms of $B$ such that $D_n = \{x:S| x(d_n) = 1\}$. 
      Let $C = B/\langle (\neg d_n)_{n:\N}\rangle$.
      Then the map $Sp(C) \to S$ is as desired because
      $$Sp(C) = \{x:S| \forall_{n:\N} x(\neg d_n) =0\}  = \bigcap_{n:\N} D_n.$$
%      By \Cref{SurjectionsAreFormalSurjections}, t
%      The quotient map $B \twoheadrightarrow C$
%      corresponds to a map $\iota:Sp(C) \hookrightarrow  S$. 
%      For $s:S$, $s$ lies in the image of this map iff 
%      for all $n:\N$ we have  $s(\neg d_n) = 0$, 
%      \begin{equation}
%        x\in \iota(Sp(C)) \leftrightarrow x(\neg d_n) = 0 \leftrightarrow x(d_n) = 1 \leftrightarrow x\in D_n
%      \end{equation}
%      Thus the image of $\iota$ is given by $\bigcap_{n:\N} D_n$. 
   \item[$(iii) \to (i)$.] 
     \rednote{TODO 
       The order of untracating is important in this proof, 
     and I struggle a bit with stressing this in a way this is clear (and concise). 
    Check with fresh eyes later. }
      Let $f:T\to S$ be a map between Stone spaces. 
      Assume $S = Sp(A), T = Sp (B)$. 
%      For this proof, we work with explicit presentations for $A,B$. 
%
      Let $G$ be a countable set of generators of $A$. 
      Assume also we have countable sets of generators and relations for $B$. 
%
      Following \Cref{FiberConstruction}, using $G$, for each $x:S$, we can construct 
      a countable set $I_x\subseteq B$ such that $$Sp(B/I_x) = (\Sigma_{y:T} f(y) = x) .$$
      As we saw in \Cref{BooleEqualityOpen} and \Cref{TruncationStoneClosed}, 
      this type being inhabited is a countable conjunction 
      of decidable propositions on the relations for $B/I_x$. 
      As these relations are explicit, we can construct the sequence $\alpha_x$ as required. 

%
%      Recall that the propositional truncation of a Stone is closed, as it is the negation of $0=1$ in the underlying 
%      Boolean algebra, which is open as it f
%
%
%      the core idea of the proof was that the closed proposition corresponds to checking equality in the underlying BA, 
%      which was closed as 
%
%
%
%
%      Note that $x$ in the image of $f$ iff $0\neq_{B/I_x} 1$. 
%      At this point, we have generators and relations of $B/I_x$ as data.
%      Hence using the proof of \Cref{BooleEqualityOpen}, we can construct a sequence 
%      $\alpha_x:2^\N$ such that $0 =_{B/I_x}1\leftrightarrow \exists_{n:\N} \alpha_x(n) = 0$. 
%      And for $\beta_x(n) = 1-\alpha_x(n)$, we conclude that 
%      \begin{equation}
%        x\in f(T) \leftrightarrow \forall_{n:\N} \beta_x(n) = 0
%      \end{equation}
%      Note that we did not use any choice axioms in the proof of this implication,
%      as we untruncated our assumptions before we specified $x$. 
   \item [$(i) \to (iv)$.] By definition.
   \item[$(iv) \to (iii)$.]
     As $A$ is closed, it induces a map $a:S\to \Closed$. 
     We can cover the closed propositions with Cantor space
     by sending 
     $\alpha \mapsto \forall_{n:\mathbb N} \alpha n = 0.$
     Now local choice gives us that there merely exists $T, e, \beta_\cdot$ as follows:
     \begin{equation}
       \begin{tikzcd}
         T \arrow[r,"\beta_\cdot"] \arrow[d, two heads,"e"] & 2^\mathbb N 
         \arrow[d,two heads, "\forall_{n:\mathbb N} (\cdot)n = 0"] \\
         S \arrow[r,"a"] & \Closed
       \end{tikzcd} 
     \end{equation} 
     Define $B(x) \leftrightarrow \forall_{n:\mathbb N} \beta_x(n) = 0$. 
     As $(i) \to (iii)$ by the above, $B$ is the image of some Stone space. 
     Furthermore, note that $A$ is the image of $B$, thus $A$ is the image of some Stone space. 
\end{itemize} 
\end{proof} 
\rednote{Ordering from here on out is WIP}


\begin{corollary}\label{InhabitedClosedSubSpaceClosed}
  For $S:\Stone$ and $A\subseteq S$ closed, we have 
  $\exists_{x:X} A(x)$ is closed. 
\end{corollary}
\begin{proof}
  Let $A$ be the image of a map map $T\to S$ for $T:\Stone$. 
  Then $\exists_{x:S} A(x) \leftrightarrow ||T||$, which is closed by \Cref{TruncationStoneClosed}
\end{proof}

\begin{corollary}\label{ClosedDependentSums}
  Closed propositions are closed under dependent sums. 
\end{corollary}
\begin{proof}
  Let $P:\Closed$ and $Q:P \to \Closed$. 
  Then $\Sigma_{p:P} Q(p) \leftrightarrow \exists_{p:P} Q(p)$.
  As $P$ is Stone by \Cref{PropositionsClosedIffStone}, 
%  As $P$ is Stone by \Cref{PropositionsClosedIffStone}, it is also compact Hausdorff, thus
  \Cref{InhabitedClosedSubSpaceClosed} gives that $\Sigma_{p:P} Q(p)$ is closed. 
\end{proof}
\begin{remark}
  Analogously to \Cref{OpenTransitive} and \Cref{OpenDominance}, it follows that 
  closedness is transitive and $\Closed$ forms a dominance. 
\end{remark}


\begin{lemma}\label{StoneSeperated}
  If $S:\Stone $, and $F,G:S \to \Closed$ be such that $F\cap G = \emptyset$. 
  Then there exists a decidable subset $D:S \to 2$ such $F\subseteq D, G \subseteq \neg D$. 
\end{lemma}
\begin{proof}
  Assume $S = Sp(B)$. 
  By the above theorem, there exists sequences $f_n,g_n:B,~n:\N$ such that 
  $x\in F$ iff $x(f_n) = 1$ for all $n:\N$ and 
  $y\in G$ iff $y(g_m) = 1$ for all $m:\N$. 
%
  Denote $R\subseteq B$ for $\{\neg f_n|n:\N\} \cup\{\neg g_n|n:\N\}$. 
  Note that any inhabitant $Sp(B/R)$ gives a map $x:B\to 2$ such that
  $x(g_n)= x(f_n) = 1$ for all $n:\N$, hence $x\in F \cap G$. 
  As $F\cap G = \emptyset $, it follows that $Sp(B/R)$ is empty.
%
  Thus there exists finite sets $I,J\subseteq \N $ such that 
  $$1 =_B ((\bigvee_{i\in I} \neg f_i) \vee (\bigvee_{j\in J} \neg g_j)).$$
%
  Let $y\in G$. Then $y(\neg g_j) = 0$ for all $j \in J$. 
  Hence 
  $$
  1 = y(1) 
  = 
  y(\bigvee_{i\in I} \neg f_i) = y (\neg (\bigwedge_{i\in I} f_i))
  $$
  Thus $y(\bigwedge_{i\in I} f_i) = 0$. 
  Note that if $x\in F$, we have $x(f_i) = 1$ for all $i\in I$, hence 
  $x(\bigwedge_{i\in I} f_i) = 1$. 
  Thus for $D$ corresponding to $\bigwedge_{i\in I} f_i$, we have that 
  $F\subseteq D, G\subseteq \neg D$ as required. 
\end{proof} 





\appendix
%\section{Technical details}
\section{The equivalence of $B_\infty$ and $\N_{(co)fin}$}
Recall that we defined $B_\infty$ as the quotient of the freely generated algebra 
over $p_n,~n\in\N$ by the relations $\{p_n \wedge p_m | n\neq m\}$. 

\begin{lemma}\label{N-co-fin-cp}
  The Boolean algebra of co-finite subsets of $\N$
  is equivalent to $B_\infty$. 
\end{lemma}
\begin{proof}
  Let $f:B_\infty \to \N_{(co)fin}$ be induced by sending $p_n$ to $\{n\}$. 
  Note that whenever $n\neq m$, we have 
  $f(p_n)\wedge f(p_m) = \{n\} \cap \{m\} = \emptyset$, 
  thus $f$ respects the relations of $B_\infty$ and is well-defined.

  Define $g:N_{(co)fin)} \to B_\infty$ as follows:
  \begin{itemize}
    \item On a finite subset $I$, we define $g(I) = \bigvee_{i\in I} p_i$, 
    \item On a cofinite subset $J$, we define $g(J) = \bigwedge _{i \in J^C} \neg p_i$. 
  \end{itemize}
  Note that in these cases we indeed have $I,J^C$ are finite, so these are well-defined elements. 
  We must show that $g$ is a Boolean morphism. 

  \begin{itemize}
    \item 
      By deMorgan's laws, $g$ preserves $\neg$:
      for $I$ finite we have
      \begin{equation}
      \neg g(I) = \neg (\bigvee_{i\in I} p_i) = \bigwedge_{i\in I} \neg p_i = g(I^C)
      \end{equation}
      And for $J$ cofinite, we apply similar reasoning. 
    \item To see that $g$ preserves $\vee$, we need to check three cases
      \begin{itemize}
        \item If both $I,J$ are finite, then 
        \begin{equation} 
          g(I \cup J) = \bigvee_{i\in I \cup J} p_i= \bigvee_{i\in I} p_i \vee \bigvee_{j\in J} p_j 
          = g(I) \vee g(J)
        \end{equation}
        and we're done. 
      \item If both $I,J$ are cofinite, we have
        \begin{equation}
          g(I) \vee g(J) = 
          \bigwedge_{i \in I^C} \neg p_i \vee 
          \bigwedge_{j \in J^C} \neg p_j 
          = 
          \bigwedge_{i\in I^C} 
          \bigwedge_{j \in J^C}(\neg p_i \vee  \neg p_j) 
        \end{equation}
        Now note that in $B_\infty$, we have 
        \begin{equation}
          \neg p_i \vee \neg p_j = \neg ( p_i \wedge p_j) = 
          \begin{cases}
            \neg p_i \text{ if } i = j\\
            1 \text{ if } i \neq j  
          \end{cases}
        \end{equation}
        Therefore, we can leave out the case that $i\neq j$ in the calculation of the above meet, and
        \begin{equation}
          \bigwedge_{i\in I^C} 
          \bigwedge_{j \in J^C}(\neg p_i \vee  \neg p_j)  
          = 
          \bigwedge_{i \in (I^C \cap J^C)} \neg p_i
          = 
          \bigwedge_{i \in (I \cup J)^C} \neg p_i 
        \end{equation}
        as $I\cup J$ must also be cofinite, this equals 
          $ g( I \cup J)$. 
        \item 
          If $I$ is finite and $J$ cofinite, we have 
          that $I\cup J$ is cofinite, hence 
          \begin{equation}
            g(I\cup J) = \bigwedge_{k\in (I \cup J)^C} \neg p_k
            = \bigwedge_{k \in (J^C -I)} \neg p_k
          \end{equation}
          Now note that 
          whenever $i\neq k$, we have 
          \begin{equation}
            p_i = (p_i \wedge \neg p_k) \vee (p_i \wedge p_k) = 
            (p_i \wedge \neg p_k) \vee 0 = p_i \wedge \neg p_k
          \end{equation}
          Hence by absorption
          \begin{equation} 
            (p_i \vee \neg p_k)  =
              \begin{cases}
                1 \text{ if } i = k \\
                \neg p_k \text{ if } i \neq k
              \end{cases}
          \end{equation}
          As for all $k\in J^C-I$ and all $i\in I$ we have $k\neq i$, we may thus write
          \begin{equation}\label{eqnCofiniteHelper1}
            \bigwedge_{k \in (J^C - I)} \neg p_k = 
            \bigwedge_{k \in (J^C - I)} (\neg p_k \vee (\bigvee_{i\in I} p_i))
          \end{equation}
          We now note that 
          \begin{equation}\label{eqnCofiniteHelper2}
            1=\bigwedge_{i\in I} 1 = \bigwedge_{i\in I} (\neg p_i \vee (\bigvee_{i\in I} p_i)).
          \end{equation}
          Taking the meet of the expressions in \Cref{eqnCofiniteHelper1} and \Cref{eqnCofiniteHelper2}, 
          we see that 
          \begin{equation}
            \bigwedge_{k \in (J^C - I)} \neg p_k = 
            \bigwedge_{j \in J^C} (\neg p_j \vee (\bigvee_{i\in I} p_i))
          \end{equation}
          And using distributivity rules, we can see that 
          \begin{equation}
            \bigwedge_{j \in J^C} (\neg p_j \vee (\bigvee_{i\in I} p_i))
            = 
            (\bigwedge_{j \in J^C} \neg p_k) \vee (\bigvee_{i\in I} p_i)
          \end{equation}
          From which we may conclude that $g(I\cup J) = g(I) \cup g(J)$. 
      \end{itemize}
    \item The case for $\wedge$ is completely dual to the case for $\vee$. 
  \end{itemize}
We conclude that $g$ is a Boolean morphism. 
Furthermore, it is easy to see that $g$ and $f$ are each other's inverse, 
thus the Boolean algebras are isomorphic. 
\end{proof}
\begin{remark}\label{AppendixCofiniteOrFinite}
  As a consequence of the above proof, any $b:B_\infty$ corresponds either to 
  \begin{itemize}
    \item a finite set $I$, in which case $b = \bigvee_{i\in I} p_i$. 
    \item a cofinite set $J$, in which case $b = \bigwedge_{j\in J^C} \neg p_j$. 
  \end{itemize}
  We will call $b$ finite/cofinite respectively. 
\end{remark}
\begin{remark}
Recall that $\Noo$ is defined as the spectrum of $B_\infty$. 
If $\alpha:\Noo$ satisfies $\alpha(p_n) = 1$, then $\alpha(p_m) = 0$ for all $n\neq m$. 
Therefore, for each $n:\N$, there is an unique map $\chi_n$ with $\chi_n(p_n) = 1$. 
There is also the point $\chi_\infty : \Noo$ which is unique 
with the property that $ \chi_\infty(p_n) = 0$ for all $n:\N$. 
\end{remark}
\rednote{Active WIP}
\begin{lemma}
  A decidable subset $D: \Noo \to 2$ corresponds to a cofinite $d:B_\infty$ iff $\chi_\infty\in D$. 
\end{lemma}
\begin{proof}
  Let $D:\Noo \to 2$ correspond to $d:B_\infty$.
  Then $\chi_\infty\in D$ iff $\chi_\infty(d) = 1$. 
%
  By \Cref{AppendixCofiniteOrFinite}, $d$ can have two forms.
  By definition of $\chi_\infty$, we have that 
  $\chi_\infty(\bigvee_{i\in I} p_i) = 0$ and 
  $\chi_\infty(\bigwedge_{j\in J^C} \neg p_j) = 1$. 
  Thus $\chi_\infty(d) = 1$ iff $d$ is cofinite. 
\end{proof}
\begin{lemma}
  If $d:B_\infty$ is of the form $\bigvee_{i\in I} p_i$,
  it corresponds to the decidable set 
  $D = \{\chi_i | i \in I\} \subseteq \Noo$.
\end{lemma}
\begin{proof}
  Clearly whenever $i\in I$, we have $\chi_i(\bigvee_{i\in I} p_i) = 1$. 
  Now suppose $f:B_\infty \to 2$ is such that $f(\bigvee_{i\in I} p_i) = 1$. 
  Then $\bigvee_{i\in I}(f(p_i)) = 1$, hence it is not the case that $f(p_i) = 0$ for all $i\in I$. 
  Now as $I$ is finite and $f(p_i) = 0 \vee f(p_i) = 1$ for all $i\in I$, 
  there must exist some (necessarily unique) $i\in I$ with $f(p_i) = 1$. Hence $f = \chi_i$. 
  Thus $f(d) = 1$ iff there is some $i\in I$ with $f = \chi_i$. 
\end{proof}
\begin{lemma}
  If $d:B_\infty$ is of the form $\bigwedge_{j\in J^C} \neg p_j$,
  it corresponds to the decidable set 
  $D = \neg \{\chi_j | j \in J^C\} \subseteq \Noo$.
\end{lemma}
\begin{proof}
  Clearly whenever $j\in J^C$, we have 
  $\chi_j(\bigwedge_{j\in J^C} \neg p_j) = 0$, so $\chi_j \notin D$. 

\end{proof}


%
%\begin{lemma}
%  For all decidable subsets $D:\Noo\to 2$,
%  with $D$ non-empty, there exists some $n:\N$ with $\chi_n \in D$. 
%\end{lemma}
%\begin{proof}
%  We make a case distinction based on \Cref{AppendixCofiniteOrFinite}. 
%  \begin{itemize}
%    \item 
%      If $D$ corresponds to a finite $d:B_\infty$, but is non-empty, then 
%      $d=\bigvee_{i\in I} p_i$ for $I\subseteq \N$ finite and non-empty. 
%      If $I$ is finite (as in \Cref{dfnFinite}) and non-empty, 
%      $I\simeq Fin_k$ for some $k\neq 0$. 
%      In particular, there is a map $1 \to I$,
%      hence a term $i:I$. 
%      Then $\chi_i(d) = 1$, hence $\chi_i \in D$. 
%    \item 
%      If $D$ corresponds to some cofinite $d:B_\infty$, we have 
%      $d = \bigvee_{i\in I} \neg p_i$ for some $I\subseteq \N$ finite. 
%      Then there is some 
%\end{proof}
%




\section{Cocompleteness of $\Boole$}
\rednote{TODO, is $\Boole$ closed under countable limits? 
  It has finite colimits, as it has pushouts and initial object.
  It should also have sequential colimits (TODO). 
  Is a countable coproduct the sequential colimit of it's initial finite coproducts? 
}
\begin{lemma}\label{BoolePushouts}
  Countably presented Boolean algebras are closed under pushout. 
\end{lemma} 
\begin{proof}
  Let $A,B,C:\Boole$, and suppose $f:A\to B, g:A \to C$ are Boolean morphisms. 
  Let $G_A, G_B,G_C$ be the underlying countable sets of generators for $B,C$ and 
  let $R_A,R_B,R_C$ be the underlying countable sets of relations. 
  Consider $P$ the Boolean algebra generated by $G_B\sqcup G_C$ under the relations 
  $R_B\cup R_C \cup F$ where $F$ is the set of expressions $f(a)-g(a), a\in G_A$.
  
  Note that as the generators of $B$ are included in those of $P$, 
  and all relations of $B$ are included in those of $P$, there is a map $h:B\to P$. 
  Similarly there is a map $i:C\to P$. 
  We now claim that the following is a pushout square:
  \begin{equation}\begin{tikzcd}
    A \arrow[r,"f"] \arrow[d,"g"] & B \arrow[d,"h"]\\
    C \arrow[r,"i"] & P
  \end{tikzcd}\end{equation}  
  Suppose $\beta:B \to X, \gamma:C\to X$ are such that $\beta\circ f = \gamma \circ h$. 
  $\beta,\gamma$ then induce maps on the generators of $P$. 
  These maps respect $F$ as $\beta\circ f=\gamma\circ h$, and they must respect $R_B,R_C$ as they are maps out of $B,C$. 
  Therefore, $\beta,\gamma$ induce a map $e:P\to X$, such that 
  $e(b) = \beta(b)$ for $b:G_B$ and $e(c)=\gamma(c)$ for $c:G_C$. 
  Furthermore, any map $P\to X$ with this property must agree with $e$ on all the generators of $P$, 
  and therefore equal $e$. Thus $e$ is the unique extension $P\to X$. 
  Thus $P$ the above square is indeed a pushout. 
\end{proof}

For some proofs in this paper, 
\rednote{(right now the counter's at two)}
we'd like a very concrete description of the fiber of a map of Stone spaces. 
The following construction turns out to be particularly useful. 
\begin{lemma}\label{FiberConstruction}
  Let $A,B:\Boole$, let $G$ be an explicit countable set of generators for $A$, and let 
  $f:A \to B, x:A\to 2$. 
  Define the countable set 
  \begin{equation}
    G' = \{a | a\in G, x(a) = 0\} \cup \{\neg a | a \in G, x(a) = 1\}
  \end{equation} 
  For $R = f(G')$,
%  Then we can construct a countable set $R\subseteq B$ such that 
  the pushout of $f$ and $x$ is given by $B/R$. 
\end{lemma}  
\begin{proof}
We consider the following pullback in the category of Stone spaces:
  \begin{equation}\begin{tikzcd}
    \sum\limits_{y:Sp(B)} y\circ f = x \arrow[d] \arrow[r] \arrow["\lrcorner"{pos=0.125}, phantom, dr] 
    & \top \arrow[d,"x"]\\
    Sp(B) \arrow[r,"(\cdot) \circ f"] & Sp(A)
  \end{tikzcd}  \end{equation}
Dual to this square, we have the following pushout in the category of Boolean algebras,
where $Sp(P) \simeq  (\sum\limits_{y:Sp(B)} y \circ f = x)$:
  \begin{equation}\begin{tikzcd}
    A \arrow[d,"x"'] \arrow[r,hook,"f"] \arrow[rd,phantom,"\ulcorner"{pos=0.125}] & B\arrow[d]\\
    2 \arrow[r] & P
  \end{tikzcd}\end{equation} 
  Following \Cref{BoolePushouts}, 
  the pushout $P$ is given by $B/R$ with $R$ the relations $f(a) -x(a)$ 
  where $a$ ranges over the generators of $A$.
  Note that $x(a) \in \{0,1\}$. 
  If $x(a)=0$, then $f(a)-x(a) = f(a)$, 
  and if $x(a) = 1$, then $f(a) -x(a) = \neg f(a) = f(\neg a)$. 
  So we can define the subset $G'\subseteq A$ given by 
  \begin{equation}
    G' = \{a | a\in G, x(a) = 0\} \cup \{\neg a | a \in G, x(a) = 1\}
  \end{equation} 
  $G'$ is in bijection with $G$, hence countable. 
  Furthermore, $x(g) = 0$ for all $g\in G'$. 
  And $R = f(G')$.
\end{proof}





%\begin{lemma}\label{BooleCoEqualizers}
%  Countably presented Boolean algebras are closed under coequalizers.
%\end{lemma}
%\begin{proof}
%  Let $f,g:A\to B$ be Boolean morphisms.
%  Define $C = B/R$, where $R$ is given by the relations $fa-ga,~a\in G_A$, for $G_A$ the set of generators of $A$.
%  Suppose that we have a map $x:B\to D$ with $xf = gf$. Then $x$ respects $R$, and thus defines a map $y:C \to D$. 
%  Furthermore, any map $C\to D$ extending $x$ agrees with $y$ on the generators of $C$, 
%  and is thus equal to $y$. Therefore $C$ is the coequalizer of $f,g$. 
%\end{proof}


%%
%%\begin{corollary}\label{CoCompletenessBoole}
%%  The category of countably presented Boolean algebras contains all finite colimits. 
%%\end{corollary}
%%\begin{proof}
%%  Recall that $\Boole$ has an initial object given by $2$. 
%%  By \Cref{BoolePushouts}, 
%%%  it is therefore closed under coproducts. 
%%%  By \Cref{BooleCoEqualizers}, 
%%  it follows that $\Boole$ contains all finite colimits. 
%%\end{proof}


\section{Some notes on our axioms}
\label{NotesOnAxioms}
In \Cref{Axioms}, we have chosen to present propositional completeness as an axiom. 
However, assuming Stone duality, we could have made some other choices, 
and left propositional completeness as a theorem. 
What's more, assuming the axiom of Dependent choice,
the axiom is equivalent to LLPO. 
In this section, we will show these equivalences. 
\rednote{At time of writing, not all the references were cleaned up, and some might change or split, 
so I need to come back here later after all.}

\begin{theorem}\label{AlternativesToAxiom2}
  Assuming Stone duality, the following are equivalent:
  \begin{enumerate}[(i)]
    \item For $S$ Stone, we have $\neg \neg S \to ||S||$. 
    \item For $S$ Stone, we have that $||S||$ is closed. 
    \item A map $f:A \to B$ in $\Boole$ is injective iff the map $(\cdot) \circ f : Sp(B) \to Sp(A)$ is surjective. 
  \end{enumerate}
\end{theorem}
\begin{proof}
  We assume that $S= Sp(B)$. 
  Note the proof of \Cref{SpectrumEmptyIff01Equal} only uses Stone duality. 
  The proof of \Cref{BooleEqualityOpen} only relies on the definition of $\Boole$.
  Hence the argument in \Cref{TruncationStoneClosed}, which shows $(i)\to (ii)$ only relies on Stone duality. 
  Furthermore, the argument that closed propositions are double negation stable (\Cref{rmkOpenClosedNegation})
  only used \Cref{MarkovPrinciple}, which followed from Stone duality as well. 
  Hence if $||S||$ is closed, we have $\neg \neg ||S|| \leftrightarrow ||S||$, thus $(ii) \to (i)$. 
  $(i)\to (iii)$ is \Cref{FormalSurjectionsAreSurjections}. 
  By the above discussion, we also have that $\neg \neg S$ iff $0\neq_B 1$. 
  Note that $0\neq_B 1$ iff the map $2\to B$ is injective. 
  Furthermore, $||S||$ iff the map $S \to \top $ is surjective. 
  Hence $(iii) \to (i)$. 
\end{proof} 

\begin{lemma}\label{LLPOAndDCToCompleteness}
Assuming dependent choice, Stone duality, and that closed propositions are closed under disjunctions, 
we can show propositional completeness. 
\end{lemma}
\begin{proof}
  Let $B:\Boole$ satisfy $0\neq_B 1$. We will show there merely exists a map $B\to 2$. 
  Let $G$ be the set of generators of $B$. 
  We will use dependent choice on the the following $E_n,R_n$:
  \begin{itemize}
    \item 
  Let $E_n$ be the type consisting of 
  \begin{itemize}
    \item A map from the first $n$ generators of $B$ to $2$, denoted $x_n:G_n \to 2$. 
    \item A proposition denoting that $0\neq_{B_n} 1$ for $B_n$ given by:
      \begin{equation}
        B_n := B/\big( \{g|g\in G_n, x_n(g) = 0\} \cup \{ \neg g| g\in G_n, x_n(g) = 1\}\big).
      \end{equation}
  \end{itemize}
  \item 
    And let $R_n:E_n \to E_{n+1} \to \mathcal U$ denote the relation that $x_{n+1}$ extends $x_n$. 
  \end{itemize} 
  Note that $E_0$ is inhabited as $0\neq_B 1$. Assume $E_n$.
%  Now assume $x_n:G_n\to 2$ witnesses $E_n$. 
  As $0\neq_{B_n}1$, for all $g:B_n$, we can show 
%  we have $$\neg ((g =1)  \wedge ((\neg g) = 1)).$$
% % 
%%  Now suppose that $E_n$ is inhabited,  and let $x_n:G_n \to 2$. 
%%  Note that in $B_n$, we have $0\neq 1$ and thus $$\neg ((g =1)  \wedge ((\neg g) = 1))$$
%%  for all $g:B_n$.
%  Therefore, we have 
  $$\neg \neg (( g\neq 1) \vee ((\neg g) \neq 1)).$$
  By \Cref{EqualityIsOpen}, and \Cref{rmkOpenClosedNegation}, 
  (which could be shown using Stone Duality)
  and the assumption that 
  closed statements are closed under disjunction, we have that the above statement is equivalent to 
  $(g \neq 1) \vee ((\neg g) \neq 1)$. 
  This holds in particular for $g$ the $n+1$'th generator of $B$. 
  Therefore, we have that $0\neq 1$ in $B_n/\{g\}$ or in $B_n/\{\neg g\}$. 
  Thus we can extend $x_n$ by letting $x_{n+1}(g) = 0$ or $x_{n+1}(g) = 1$ respectively. 
  
  By dependent choice, we get a map $x:G\to 2$. 
  We claim that for this map $x$, we have $0\neq 1$ in 
  \begin{equation}
    B' := B/\big( \{g|g\in G, x(g) = 0\} \cup \{ \neg g| g\in G, x(g) = 1\}\big).
  \end{equation}
  Note that $B'$ is the colimit of the sequence $B_n$ with projection maps $B_n \to B_{n+1}$. 
  Thus if $0=1$ in $B'$, $0=1$ in some $B_n$, which doesn't happen by assumption. 
  Therefore we have $0\neq 1$ in $B'$. 
  Furthermore, note that $B'$ is equivalent to a Boolean algebra with no generators, 
  as any generator in $B$ is sent to either $0$ or $1$ by the relations in $B'$. 
%
  But now any Boolean algebra with no generators and $0\neq 1$ is isomorphic to $2$. 
  Therefore $B'\simeq 2$, and the projection map $B\to B'$ gives a map $B \to 2$. 
  
\end{proof}

\begin{corollary}
Assuming dependent choice and Stone duality, TFAE:
\begin{enumerate}[(i)]
  \item For $S$ Stone, we have $\neg \neg S \to ||S||$. 
  \item LLPO.
  \item The disjunction of two closed propositions is closed. 
\end{enumerate}
\end{corollary}
\begin{proof}
  $(i) \to (ii)$ is \Cref{LLPO}, $(ii) \to (iii)$ is \Cref{ClosedDisjunction}, 
  and $(iii) \to (i)$ is \Cref{LLPOAndDCToCompleteness}
\end{proof}
\rednote{
  @Hugo, you mentioned that axiom 2 was independent from the other axioms. 
This might be a good place to reference to that proof}

%\subsection{Countably presented algebras as sequential colimits}\label{secBooleAsColimits}

\begin{definition}
  A sequence in a category is a diagram of shape $\N$, 
  where $\N$ carries the natural structure of a poset. 
\end{definition}
\begin{lemma}\label{lemProFinitePresentation}
  For every countably presented Boolean algebra $B$
  there merely exists a sequence of finitely presented Boolean algebras 
  whose colimit in the category of Boolean algebras is $B$. 
\end{lemma}
\begin{proof}
  Consider $\langle G \rangle \langle\langle R \rangle\rangle$ a countable presentation of a Boolean algebra $B$. 
  We will show there exists a diagram of shape $\N$ taking values in Boolean algebras 
  with $\langle G\rangle / R$ as the colimit.
  \paragraph{The diagram}
  Let $R_n$ be the first $n$ terms in $R$. 
  Note that each of these finitely many terms uses only finitely many symbols from $G$.
  Let $G_n$ be the finite set of terms used in $R_n$, unioned with the finite set of the first $n$ elements of $G$. 
  Define for each $n\in\N$ the finitely presented Boolean algebra $B_n = \langle G_n \rangle  \langle R_n \rangle$. 
  If $n\leq m$, then \Cref{rmkMorphismsOutOfQuotient} gives us a map $B_n \to B_m$ 
  as $G_n \subseteq G_{n+1}$ and $R_n \subseteq R_{n+1}$. 
  Thus $(B_n)_{n\in \N}$ gives us a diagram of shape $\N$
  with values in finitely presented algebras. 

  \paragraph{The colimit}
  As $G_n\subseteq G$ and $R_n \subseteq R$, 
  \Cref{rmkMorphismsOutOfQuotient} also gives us a map $B_n\to \langle G \rangle \langle R \rangle$. 
  We claim the resulting cocone is a colimit. 

  Suppose we have a cocone $C$ on the diagram $(B_n)_{n\in\N}$. 
  We need to show that there exists a map $\langle G \rangle / R\to C$ and
  we need to show this map is unique as map between cocones. 
  \begin{itemize}
    \item To show there exists a map $\langle G \rangle / R \to C$, 
      we use remark \Cref{rmkMorphismsOutOfQuotient} again. 
      Let $g\in G$ be the $n$'th element of $G$, 
      note that $g\in G_n$, and consider the image of $g$ under the map $B_n \to C$. 
      This procedure defines a function from $G$ to the underlying set of $C$. 
      Let $\phi \in R$ be the $n$'th element of $R$, 
      note that $\phi \in R_n$, and the map $B_n \to C$ must send $\phi$ to $0$. 
      Thus the function from $G$ to the underlying set of $C$ also sends $\phi$ to $0$. 
      This thus defines a map $\langle G \rangle / R \to C$. 
    \item To show uniqueness, consider that any map of cocones $\langle G \rangle / \langle R \rangle \to C$ 
      must take the same values on all $g\in G_n$ for all $n\in\N$. 
      Now all $g\in G$ occur in some $G_n$, so any map of cocones $\langle G \rangle /  \langle R \rangle \to C$ 
      takes the same values for all $g\in G$. 
      \Cref{rmkMorphismsOutOfQuotient} now tell us that these values uniquely determine the map. 
  \end{itemize}
\end{proof}
\begin{remark}
  Conversely, any colimit of a sequence of finite Boolean algebras 
  is a countably presented Boolean algebra with 
  as underlying sets of generators and relations the countable union of the finite sets of 
  generators and relations, which are both countable. 
\end{remark}
\begin{lemma}\label{lemFinitelyPresentedBACompact}
  For any finitely presented Boolean algebra $A$,
  and any sequence $(B_n)_{n:\N}$ of Boolean algebras with colimit $B$
  we have that the set $B^A$ is the colimit of the sequence of sets $(B_n^A)_{n:\N}$. 
\end{lemma}  
\begin{proof}
  First note that $B^A$ forms a cocone on $(B_n^A)_{n:\N}$ 
  because any map $A \to B_n$ induces a map $A \to B$. 
  Let $C$ be a cocone on $(B_n^A)_{n:\N}$. 
  We shall show there is an unique morphism of cocones $B^A \to C$. 
  \begin{itemize}
    \item For existence, let $f:B^A$. 
      As $A$ is finitely presented, we write $A = \langle G \rangle / \langle R \rangle$ with $G$ finite.
      By \Cref{rmkMorphismsOutOfQuotient}, $f$ is uniquely determined by it's values on $g\in G$. 
      As $G$ is finite, so is it's image $f(G)\subseteq B$. 
      But any finite subset of $B$ already occurs in $B_n$ for some $n\in\N$. 
      Consequently, the image of $f$ is already contained in some $B_n$. 
      Thus there is some $f_n:(B_n^A)$ such that postcomposing 
      $f_n$ with the map $B_n \to B$ gives back $f$. 
      The image of $f_n$ under the map $(B_n^A) \to C$ is how we define the image of $f$. 
      This is well-defined by the cocone conditions on $C$. 
    \item 
      For uniqueness, by function extensionality maps $B^A \to C$ are uniquely determined by their values on 
      $f:B^A$. By the above, the value of $f$ is uniquely determined by it's value on $B_n$ for 
      any $n$ with the image of $f$ in $B_n$. Thus there is at most one morphism of cocones $B^A \to C$. 
  \end{itemize}
\end{proof}
\begin{remark}\label{rmkEqualityColimit}
  In the above proof, we used that any element $b\in B$ already occurs in some $B_n$. 
  However, please note that it is not necessarily the case that it occurs uniquely in $B_n$, 
  there might be multiple elements in $B_n$ which can all be sent to $b$ in the end. 

  In case our sequence comes from the construction in \Cref{lemProFinitePresentation}, 
  we can see that whenever there are two elements in 
  $B_n$ corresponding to $b\in B$, they will become equal in $B_m$ for some $m\geq n$. 
  The reason is that if $b \sim_{\langle R \rangle} c$, there is a finite subset $R_0 \subseteq R$ such that 
  $b\sim_{\langle R_0 \rangle} c$, which will occur in some $R_m$. 

  One could wonder whether this property holds for general colimits of sequences. 
  In general, if we assume $B$ is the colimit of an arbitrary sequence $(B_n)_{n:\N}$, 
  and there exist some $B_n$ with two elements corresponding to the same element in $B$, 
  Theorem 7.4 from \cite{SequentialColimitHoTT} says that there merely exists some $m\geq n$
  such that they are already equal in $B_m$. 
\end{remark}

%For our next lemma on this presentation of sequences we need the axiom of dependent choice. 
%\begin{axiomNum}[Dependent choice]\label{axDependentChoice}
%  Given a family of types $(E_n)_{n:\N}$ and 
%  a relation 
%  $R_n:E_n\rightarrow E_{n+1}\rightarrow {\mathcal U}$ such that
%  for all $n$ and $x:E_n$ there exists $y:E_{n+1}$ with $p:R_n~x~y$ 
%  then given $x_0:E_0$ there exists
%  $u:\Pi_{n:\N}E_n$ and $v:\Pi_{n:\N}R_n~(u~n)~(u~(n+1))$ and $u~0 = x_0$.
%\end{axiomNum}
\begin{lemma}[Using dependent choice]\label{lemDecompositionOfColimitMorphisms}
  Let $B,C$ be countably presented Boolean algebras, 
  and suppose we have a morphism $f:B\to C$.
  There exists sequences of finitely presented Boolean algebras 
  $(B_n)_{n:\N}, (C_n)_{n:\N}$ with colimits $B,C$ respectively
  and compatible maps of Boolean algebras $f_n:B_n \to C_n$, 
  such that $f$ is the induced morphism $B\to C$.
\end{lemma}
\begin{proof}
  Let $(B_n)_{n:\N}, (C_n)_{n:\N}$ be 
  sequences of finitely presented Boolean algebras with colimits $B$ and $C$. 
  We will take a subsequence of $(C_n)_{n:\N}$, using the axiom of dependent choice above. 

  Our family of types $E_k$ as in \Cref{axDependentChoice} 
  will be strictly increasing sequences $(n_i)_{i\leq k}$ of natural numbers together with a finite family of maps 
  $(f_i: B_{i} \to C_{n_i})_{i\leq k}$ such that
  for all $0\leq i<k$ the following diagram commutes:
  \begin{equation}\label{eqnDecompositionOfColimitMorphisms}
    \begin{tikzcd}
      B_{i} \arrow[r] \arrow[d, "f_i"]& B_{{i+1}} \arrow[r] \arrow[d,"f_{i+1}"]& B \arrow[d,"f"] \\
      C_{n_i} \arrow[r] & C_{n_{i+1}} \arrow[r] & C 
    \end{tikzcd}
  \end{equation}
  Our relation $R_k$ will tell whether the second sequence extends the first one. 
%
  By \Cref{lemFinitelyPresentedBACompact} 
  there exists some $n_0:\N$ 
  such that $B_0 \to B \to C$ factors as 
  \begin{equation}
    \begin{tikzcd}
      B_{0} \arrow[r] \arrow[d, "f_0"]& B \arrow[d,"f"] \\
      C_{n_0} \arrow[r] & C 
    \end{tikzcd}
  \end{equation}
  Because our goal is a proposition, we can untracate this existence to data. 
  This data will form our $x_0:E_0$. %from \Cref{axDependentChoice}. 
%
  Now suppose we have $(f_i: B_{i} \to C_{n_i})_{i\leq k}$ for some $k\geq 0$ 
  such that
  for all $0\leq i<k$ the diagram of \Cref{eqnDecompositionOfColimitMorphisms} commutes.
  We shall show that in this case there exists an $n_{k+1}, f_{k+1}$ 
  making the same diagram commute for $i = k$. 
  Consider $B_{{k}+1}\to B \to C$. By the same argument as for $B_0$, we have a factorization 
  \begin{equation}
    \begin{tikzcd}
    B_{k+1} \arrow[r]  \arrow[d,"f'_{k+1}"]& B \arrow[d,"f"]\\
    C_{n'_{k+1}} \arrow[r] & C
    \end{tikzcd}
  \end{equation}
  Note that we may assume $n'_{k+1} > n_k$.
  Note that it is not necessarily the case that 
  $f'_{k+1}$ is compatibly with $f_k$, meaning the left square in the following diagram needn't commute:
  \begin{equation}
    \begin{tikzcd}
      B_{k} \arrow[r] \arrow[d, "f_k"]& B_{{k+1}}  \arrow[r] \arrow[d,"f'_{k+1}"] & B \arrow[d,"f"] \\
      C_{n_k} \arrow[r] & C_{n'_{k+1}} \arrow[r]  & C 
    \end{tikzcd}
  \end{equation}
  However, both $f'_{k+1}, f_k$ induce the same map $B_{k} \to C$. 
  Recall by \Cref{rmkMorphismsOutOfQuotient} this map is induced by it's value on finitely many elements. 
  By \Cref{rmkEqualityColimit}, it follows there is an $n_{k+1} \geq {n'_{k+1}}$ 
  such that for $f_{k+1}$ the composition of $f'_{k+1}:B_{k+1} \to C_{n'_{k+1}}$ and 
  the map $C_{n'_{k+1}} \to C_{n_{k+1}}$, the following diagram does commute:
  \begin{equation}
    \begin{tikzcd}
      B_{k} \arrow[d,"f_k"]\arrow[r] & B_{{k+1}} \arrow[rd, "f_{k+1}"] \arrow[rr] & & B \arrow[d,"f"] \\
      C_{n_k} \arrow[r] & C_{n'_{k+1}} \arrow[r] & C_{n_{k+1}} \arrow[r] & C 
    \end{tikzcd}
  \end{equation}
  Now by dependent choice for the above $x_0, R_n, E_n$, we get a sequence $(f_i:B_i \to C_{n_i})$  for some 
  strictly increasing sequence $n_i$ of natural numbers. 
  Note that for such a sequence $(n_i)_{i:\N}$, 
  $(C_{n_i})_{i:\N}$ converges to $C$. Also $(B_i)_{i:\N}$ still converges to $B$. 
  Futhermore, by construction the map that sequence $f_i$ induces from $B \to C$ shares all values with $f$
  and thus is equal to $f$. 
  Thus our sequence $f_i$ is as required. 
\end{proof}
\begin{remark}\label{rmkEpiMonoFactorizationCommutes}
  For $f,(f_i)_{i:\N}$ as above, whenever $f_n(x) = 0$, we have $f_{n+1}(x \circ \iota_{n,n+1}) = 0$
  for $\iota_{n,n+1}$ the map $A_n \to A_{n+1}$. 
  By \Cref{rmkMorphismsOutOfQuotient}, $\iota_{n,n+1}$ induces a map $A_n/Ker(f_n)\to A_{n+1}/Ker(f_{n+1})$. 
  This induced map is such that the following diagram commutes:
  \begin{equation}\begin{tikzcd}
    A_n \arrow[d, two heads] \arrow[r, "\iota_{n,n+1}"] & A_{n+1} \arrow[d,two heads]\\
    A_n /Ker(f_n) \arrow[d,hook] \arrow[r] & A_{n+1} /Ker(f_{n+1}) \arrow[d,hook] \\
    B_n \arrow[r] & B_{n+1}
  \end{tikzcd}\end{equation}  
  As the induced maps be epi's / mono's  is epi /mono, the colimit of the sequence 
  $A_n / Ker(f_n)$ will fit into an epi-mono factorization of $f$ and thus be iso to $A/Ker(f)$. 
  Thus the epi-mono factorization of the colimit is the colimit of the epi-mono factorizations. 
\end{remark}
\begin{remark}\label{rmkIsoEpiMonoMapColimit}
  Whenever $f:B \to C$ is an iso, any sequence with $B$ as colimit, also has $C$ as colimit. 
  Thus any iso can be represented this way as sequence of iso's. 
  Conversely, any sequence of isomorphisms induces an isomorphism of their colimits. 

  It follows from \Cref{rmkEpiMonoFactorizationCommutes} that when $f$ is epi/mono, 
  we can say that $f$ can be induced by a sequence 
  $(f_i)_{i\in \N}$ with all $f_i$ epi/mono. 
\end{remark}






%
%\section{Topology}
%
\subsection{Closed subtypes}

\begin{definition}%
  \label{closed-proposition}\label{closed-subtype}
  \begin{enumerate}[(a)]
  \item
    A \notion{closed proposition} is a proposition
    which is merely of the form $x_1 = 0 \land \dots \land x_n = 0$
    for some elements $x_1, \dots, x_n \in R$.
  \item
    Let $X$ be a type.
    A subtype $U : X \to \Prop$ is \notion{closed}
    if for all $x : X$, the proposition $U(x)$ is closed.
  \item
    For $A$ a finitely presented $R$-algebra
    and $f_1, \dots, f_n : A$,
    we set
    $V(f_1, \dots, f_n) \colonequiv
    \{\, x : \Spec A \mid f_1(x) = \dots = f_n(x) = 0 \,\}$.
  \end{enumerate}
\end{definition}

Note that $V(f_1, \dots, f_n) \subseteq \Spec A$ is a closed subtype
and we have $V(f_1, \dots, f_n) = \Spec (A/(f_1, \dots, f_n))$.

\begin{proposition}[using \axiomref{sqc}]%
  There is an order-reversing isomorphism of partial orders
  \begin{align*}
    \text{f.g.-ideals}(R) &\xrightarrow{{\sim}} \Omega_{cl} \\
    I &\mapsto (I = (0))
  \end{align*}
  between the partial order of finitely generated ideals of $R$
  and the partial order of closed propositions.
\end{proposition}

\begin{proof}
  For a finitely generated ideal $I = (x_1, \dots, x_n)$,
  the proposition $I = (0)$ is indeed a closed proposition,
  since it is equivalent to $x_1 = 0 \land \dots \land x_n = 0$.
  It is also evident that we get all closed propositions in this way.
  What remains to show is that
  \[ I = (0) \Rightarrow J = (0)
     \qquad\text{iff}\qquad
     J \subseteq I
     \rlap{\text{.}}
  \]
  For this we use synthetic quasicoherence.
  Note that the set $\Spec R/I = \Hom_R(R/I, R)$ is a proposition
  (has at most one element),
  namely it is equivalent to the proposition $I = (0)$.
  Similarly, $\Hom_R(R/J, R/I)$ is a proposition
  and equivalent to $J \subseteq I$.
  But then our claim is just the equation
  \[ \Hom(\Spec R/I, \Spec R/J) = \Hom_R(R/J, R/I) \]
  which holds by \Cref{spec-embedding},
  since $R/I$ and $R/J$ are finitely presented $R$-algebras
  if $I$ and $J$ are finitely generated ideals.
\end{proof}

\begin{lemma}[using \axiomref{sqc}]%
  \label{ideals-embed-into-closed-subsets}
  We have $V(f_1, \dots, f_n) \subseteq V(g_1, \dots, g_m)$
  as subsets of $\Spec A$
  if and only if
  $(g_1, \dots, g_m) \subseteq (f_1, \dots, f_n)$
  as ideals of $A$.
\end{lemma}

\begin{proof}
  The inclusion $V(f_1, \dots, f_n) \subseteq V(g_1, \dots, g_m)$
  means a map $\Spec (A/(f_1, \dots, f_n)) \to \Spec (A/(g_1, \dots, g_m))$
  over $\Spec A$.
  By \Cref{spec-embedding}, this is equivalent to
  a homomorphism $A/(g_1, \dots, g_m) \to A/(f_1, \dots, f_n)$,
  which in turn means the stated inclusion of ideals.
\end{proof}

\begin{lemma}[using \axiomref{loc}, \axiomref{sqc}, \axiomref{Z-choice}]%
  \label{closed-subtype-affine}
  A closed subtype $C$ of an affine scheme $X=\Spec A$ is an affine scheme
  with $C=\Spec (A/I)$ for a finitely generated ideal $I\subseteq A$.
\end{lemma}

\begin{proof}
  By \axiomref{Z-choice} and boundedness,
  there is a cover $D(f_1),\dots,D(f_l)$, such that
  on each $D(f_i)$, $C$ is the vanishing set of functions
  \[ g_1,\dots,g_n:D(f_i)\to R\rlap{.} \]
  By \Cref{ideals-embed-into-closed-subsets},
  the ideals generated by these functions
  agree in $A_{f_i f_j}$,
  so by \Cref{fg-ideal-local-global},
  there is a finitely generated ideal $I\subseteq A$,
  such that $A_{f_i}\cdot I$ is $(g_1,\dots,g_n)$
  and $C=\Spec A/I$.
\end{proof}

\subsection{Open subtypes}

While we usually drop the prefix ``qc'' in the definition below,
one should keep in mind, that we only use a definition of quasi compact open subsets.
The difference to general opens does not play a role so far,
since we also only consider quasi compact schemes later.

\begin{definition}%
  \label{qc-open}
  \begin{enumerate}[(a)]
  \item A proposition $P$ is \notion{(qc-)open}, if there merely are $f_1,\dots,f_n:R$,
    such that $P$ is equivalent to one of the $f_i$ being invertible.
  \item Let $X$ be a type.
    A subtype $U:X\to\Prop$ is \notion{(qc-)open}, if $U(x)$ is an open proposition for all $x:X$.
  \end{enumerate}
\end{definition}

\begin{proposition}[using \axiomref{loc}, \axiomref{sqc}]%
  \label{open-iff-negation-of-closed}
  A proposition $P$ is open
  if and only if
  it is the negation of some closed proposition
  (\Cref{closed-proposition}).
\end{proposition}

\begin{proof}
  Indeed, by \Cref{generalized-field-property},
  the proposition $\inv(f_1) \lor \dots \lor \inv(f_n)$
  is the negation of ${f_1 = 0} \land \dots \land {f_n = 0}$.
\end{proof}

\begin{proposition}[using \axiomref{loc}, \axiomref{sqc}]%
  \label{open-union-intersection}
  Let $X$ be a type.
  \begin{enumerate}[(a)]
  \item The empty subtype is open in $X$.
  \item $X$ is open in $X$.
  \item Finite intersections of open subtypes of $X$ are open subtypes of $X$.
  \item Finite unions of open subtypes of $X$ are open subtypes of $X$.
  \item Open subtypes are invariant under pointwise double-negation.
  \end{enumerate}
  Axioms are only needed for the last statement.
\end{proposition}

In \Cref{open-subscheme} we will see that open subtypes of open subtypes of a scheme are open in that scheme.
Which is equivalent to open propositions being closed under dependent sums.

\begin{proof}[of \Cref{open-union-intersection}]
  For unions, we can just append lists.
  For intersections, we note that invertibility of a product
  is equivalent to invertibility of both factors.
  Double-negation stability
  follows from \Cref{open-iff-negation-of-closed}.
\end{proof}

\begin{lemma}%
  \label{preimage-open}
  Let $f:X\to Y$ and $U:Y\to\Prop$ open,
  then the \notion{preimage} $U\circ f:X\to\Prop$ is open.
\end{lemma}

\begin{proof}
  If $U(y)$ is an open proposition for all $y : Y$,
  then $U(f(x))$ is an open proposition for all $x : X$.
\end{proof}

\begin{lemma}[using \axiomref{loc}, \axiomref{sqc}]%
  \label{open-inequality-subtype}
  Let $X$ be affine and $x:X$, then the proposition
  \[ x\neq y \]
  is open for all $y:X$.
\end{lemma}

\begin{proof}
  We show a proposition, so we can assume $\iota: X\to \A^n$ is a subtype.
  Then for $x,y:X$, $x\neq y$ is equivalent to $\iota(x)\neq\iota(y)$.
  But for $x,y:\A^n$, $x\neq y$ is the open proposition that $x-y\neq 0$.
\end{proof}

The intersection of all open neighborhoods of a point in an affine scheme,
is the formal neighborhood of the point.
We will see in \Cref{intersection-of-all-opens}, that this also holds for schemes.

\begin{lemma}[using \axiomref{loc}, \axiomref{sqc}]%
  \label{affine-intersection-of-all-opens}
  Let $X$ be affine and $x:X$, then the proposition
  \[ \prod_{U:X\to \Open}U(x)\to U(y) \]
  is equivalent to $\neg\neg (x=y)$.
\end{lemma}

\begin{proof}
  By \Cref{open-union-intersection}, $\neg\neg (x=y)$ implies $\prod_{U:X\to \Open}U(x)\to U(y)$.
  For the other implication,
  $\neg (x=y)$ is open by \Cref{open-inequality-subtype}, so we get a contradiction.
\end{proof}

We now show that our two definitions (\Cref{affine-open}, \Cref{qc-open})
of open subtypes of an affine scheme are equivalent.

\begin{theorem}[using \axiomref{loc}, \axiomref{sqc}, \axiomref{Z-choice}]%
  \label{qc-open-affine-open}
  Let $X=\Spec A$ and $U:X\to\Prop$ be an open subtype,
  then $U$ is affine open, i.e. there merely are $h_1,\dots,h_n:X\to R$ such that
  $U=D(h_1,\dots,h_n)$.
\end{theorem}

\begin{proof}
  Let $L(x)$ be the type of finite lists of elements of $R$,
  such that one of them being invertible is equivalent to $U(x)$.
  By assumption, we know
  \[\prod_{x:X}\propTrunc{L(x)}\rlap{.}\]
  So by \axiomref{Z-choice}, we have $s_i:\prod_{x:D(f_i)}L(x)$.
  We compose with the length function for lists to get functions $l_i:D(f_i)\to\N$.
  By \Cref{boundedness}, the $l_i$ are bounded.
  Since we are proving a proposition, we can assume we have actual bounds $b_i:\N$.
  So we get functions $\tilde{s_i}:D(f_i)\to R^{b_i}$,
  by append zeros to lists which are too short,
  i.e. $\widetilde{s}_i(x)$ is $s_i(x)$ with $b_i-l_i(x)$ zeros appended.

  Then one of the entries of $\widetilde{s}_i(x)$ being invertible,
  is still equivalent to $U(x)$.
  So if we define $g_{ij}(x)\colonequiv \pi_j(\widetilde{s}_i(x))$,
  we have functions on $D(f_i)$, such that
  \[
    D(g_{i1},\dots,g_{ib_i})=U\cap D(f_i)
    \rlap{.}
  \]
  By \Cref{affine-open-trans}, this is enough to solve the problem on all of $X$.
\end{proof}

This allows us to transfer one important lemma from affine-opens to qc-opens.
The subtlety of the following is that while it is clear that the intersection of two
qc-opens on a type, which are \emph{globally} defined is open again, it is not clear,
that the same holds, if one qc-open is only defined on the other.

\begin{lemma}[using \axiomref{loc}, \axiomref{sqc}, \axiomref{Z-choice}]%
  \label{qc-open-trans}
  Let $X$ be a scheme, $U\subseteq X$ qc-open in $X$ and $V\subseteq U$ qc-open in $U$,
  then $V$ is qc-open in $X$.
\end{lemma}

\begin{proof}
  Let $X_i=\Spec A_i$ be a finite affine cover of $X$.
  It is enough to show, that the restriction $V_i$ of $V$ to $X_i$ is qc-open.
  $U_i\colonequiv X_i\cap U$ is qc-open in $X_i$, since $X_i$ is qc-open.
  By \Cref{qc-open-affine-open}, $U_i$ is affine-open in $X_i$,
  so $U_i=D(f_1,\dots,f_n)$.
  $V_i\cap D(f_j)$ is affine-open in $D(f_j)$, so by \Cref{affine-open-trans},
  $V_i\cap D(f_j)$ is affine-open in $X_i$.
  This implies $V_i\cap D(f_j)$ is qc-open in $X_i$ and so is $V_i=\bigcup_{j}V_i\cap D(f_j)$.
\end{proof}

\begin{lemma}[using \axiomref{loc}, \axiomref{sqc}, \axiomref{Z-choice}]%
  \label{qc-open-sigma-closed}
  \begin{enumerate}[(a)]
  \item qc-open propositions are closed under dependent sums:
    if $P : \Open$ and $U : P \to \Open$,
    then the proposition $\sum_{x : P} U(x)$ is also open.
  \item Let $X$ be a type. Any open subtype of an open subtype of $X$ is an open subtype of $X$.
  \end{enumerate}
\end{lemma}

\begin{proof}
  \begin{enumerate}[(a)]
  \item Apply \Cref{qc-open-trans} to the point $\Spec R$.
  \item Apply the above pointwise.
  \end{enumerate}
\end{proof}

\begin{remark}
  \Cref{qc-open-sigma-closed} means that
  the (qc-) open propositions constitute a \notion{dominance}
  in the sense of~\cite{rosolini-phd-thesis}.
\end{remark}

The following fact about the interaction of closed and open propositions
is due to David Wärn.

\begin{lemma}%
  \label{implication-from-closed-to-open}
  Let $P$ and $Q$ be propositions
  with $P$ closed and $Q$ open.
  Then $P \to Q$ is equivalent to $\lnot P \lor Q$.
\end{lemma}

\begin{proof}
  We can assume $P = (f_1 = \dots = f_n = 0)$
  and $Q = (\inv(g_1) \lor \dots \lor \inv(g_m))$.
  Then we have:
  \begin{align*}
    (P \to Q) &= \qquad
    \text{\Cref{generalized-field-property} for $g_1, \dots, g_m$}\\
    (P \to \lnot (g_1 = \dots = g_m = 0)) &= \\
    \lnot (f_1 = \dots = f_n = g_1 = \dots = g_m = 0) &= \qquad
    \text{\Cref{generalized-field-property} for $f_1, \dots, f_n, g_1, \dots, g_m$}\\
    (\inv(f_1) \lor \dots \lor \inv(f_n) \lor \inv(g_1) \lor \dots \lor \inv(g_m) &= \qquad
    \text{\Cref{generalized-field-property} for $f_1, \dots, f_n$}\\
    \lnot P \lor Q &
  \end{align*}
\end{proof}


%
%\subsection{Compact Hausdorff}
%\begin{definition}
  Let $S$ be Stone, $C\subseteq S$. 
  Then $C$ is open if it is the countable union of decidable subsets. 
\end{definition}
\begin{lemma}
  For $S$ Stone and $C\subseteq S$, 
  $C$ is closed iff it's complement is open 
  and $C$ is open iff it's complement is closed. 
\end{lemma}



\begin{proof}
  This follows from the fact that the complement of a decidable subset is decidable and 
  \Cref{LemDecidableSubsetsDeMorgan}.
\end{proof}
\begin{lemma}
  For $S$ Stone, any cover by opens merely has a finite subcover.
\end{lemma}
\begin{proof}
  Let $S= \bigcup_{i:I} A_i$ be a cover of $S$ by open sets. 
  Assume furthermore $S= Sp(B)$. 
  As every open is the union of decidable subsets, we may assume $A_i$ decidable, 
  and thus corresponding to points $a_i \in B$. 
  These points are such that $1 = \bigvee_{i:I} a_i$. 
  As $B$ is countably presented, it is countable. 
  Thus $(a_i)_{i:I}$ is a countable set. 
  The morphism $I\to B$ is surjective, and as we're proving a proposition,
  we may use type-theoretic AC to give 
  a countable subset $I_0 \subseteq I$ such that $\bigvee_{i:I_0} a_i = 1$ as well. 
  So $S = \bigcup_{i:I_0} A_i$ for $I_0$ countable. 
\end{proof}

Note that the basic clopens are not the only clopens. 
I.e. not every set that is both a countable intersection of decidable subsets and 
a countable union of decidable subset is itself decidable. 
In $B_\infty$, we can describe the even numbers both as the infinite meet of cofinite sets excluding odd numbers up to $n$
and the join of finite sets including even numbers up to $n$. 
But the even numbers do not themselves for an element of $B_\infty$.
Thus the set of maps $B\to 2$ sending every $\chi_{2n}$ to $1$ is clopen but not decidable. 

%
%\begin{lemma}
%  For $S$ Stone, any collections of closed sets satisfying the finite intersection property 
%  has a merely inhabited intersection. 
%\end{lemma}
%\begin{proof}
%  Suppose that $a_i$ is a collection of points such that $a_i \wedge a_j \neq 0$ whenever $i\neq j$. 
%  We claim that $\bigwedge a_i \neq 0$. 
%  Suppose $\bigwedge a_i = 0$. 
%\end{proof}


\begin{definition}
We define a type $X$ to be compact Hausdorff iff 
$X$ is the quotient of a stone space $S$ by a closed equivalence relation. 
%there is a surjection $e:S\to X$ for $S$ Stone. 
%
A subtype $C\subseteq X$ is closed respectively open iff it's pre-image under the quotient map is. 
\end{definition}

\begin{lemma}
  In a compact Hausdorff, closed sets are closed under intersection. 
\end{lemma}


%\begin{lemma}
%  Let $X$ be a Compact Hausdorff type, 
%  and suppose that $(A_i)_{i:I}$ is a collection of opens in $X$ such that 
%  $X = \bigcup_{i:I} A_i$. 
%  There then exists a finite subset $I_0\subseteq I$ with 
%  $X = \bigcup_{i:I_0} A_i$. 
%\end{lemma} 

\begin{lemma}
  In a Compact Hausdorff space, the complement of an open is closed, and the complement of a closed is open. 
\end{lemma}
\begin{proof}
  Let $e : S \to X$ be the quotient map of a Stone space by a closed equivalence relation. 
  and let $(A_n)_{n:\mathbb N}$ be a countable family of decidable subsets in $S$. 

  First, we claim that 
  $X - \bigcup_{n:\mathbb N} e(A_n)$
  is closed in $X$. 
\end{proof}

\begin{lemma}
  Whenever $X$ is compact Hausdorff, $F_0, F_1$ are closed and disjoint, 
  there exist $G_0, G_1$ disjoint clopen such that 
  $F_i \subseteq X - G_{1-i}$ and $G_0 \cup G_1 = X$. 
  
\end{lemma}

%
%
%%\begin{definition}
  Let $S$ be Stone, $C\subseteq S$. 
  Then $C$ is open if it is the countable union of decidable subsets. 
\end{definition}
\begin{lemma}
  For $S$ Stone and $C\subseteq S$, 
  $C$ is closed iff it's complement is open 
  and $C$ is open iff it's complement is closed. 
\end{lemma}



\begin{proof}
  This follows from the fact that the complement of a decidable subset is decidable and 
  \Cref{LemDecidableSubsetsDeMorgan}.
\end{proof}
\begin{lemma}
  For $S$ Stone, any cover by opens merely has a finite subcover.
\end{lemma}
\begin{proof}
  Let $S= \bigcup_{i:I} A_i$ be a cover of $S$ by open sets. 
  Assume furthermore $S= Sp(B)$. 
  As every open is the union of decidable subsets, we may assume $A_i$ decidable, 
  and thus corresponding to points $a_i \in B$. 
  These points are such that $1 = \bigvee_{i:I} a_i$. 
  As $B$ is countably presented, it is countable. 
  Thus $(a_i)_{i:I}$ is a countable set. 
  The morphism $I\to B$ is surjective, and as we're proving a proposition,
  we may use type-theoretic AC to give 
  a countable subset $I_0 \subseteq I$ such that $\bigvee_{i:I_0} a_i = 1$ as well. 
  So $S = \bigcup_{i:I_0} A_i$ for $I_0$ countable. 
\end{proof}

Note that the basic clopens are not the only clopens. 
I.e. not every set that is both a countable intersection of decidable subsets and 
a countable union of decidable subset is itself decidable. 
In $B_\infty$, we can describe the even numbers both as the infinite meet of cofinite sets excluding odd numbers up to $n$
and the join of finite sets including even numbers up to $n$. 
But the even numbers do not themselves for an element of $B_\infty$.
Thus the set of maps $B\to 2$ sending every $\chi_{2n}$ to $1$ is clopen but not decidable. 

%
%\begin{lemma}
%  For $S$ Stone, any collections of closed sets satisfying the finite intersection property 
%  has a merely inhabited intersection. 
%\end{lemma}
%\begin{proof}
%  Suppose that $a_i$ is a collection of points such that $a_i \wedge a_j \neq 0$ whenever $i\neq j$. 
%  We claim that $\bigwedge a_i \neq 0$. 
%  Suppose $\bigwedge a_i = 0$. 
%\end{proof}


\begin{definition}
We define a type $X$ to be compact Hausdorff iff 
$X$ is the quotient of a stone space $S$ by a closed equivalence relation. 
%there is a surjection $e:S\to X$ for $S$ Stone. 
%
A subtype $C\subseteq X$ is closed respectively open iff it's pre-image under the quotient map is. 
\end{definition}

\begin{lemma}
  In a compact Hausdorff, closed sets are closed under intersection. 
\end{lemma}


%\begin{lemma}
%  Let $X$ be a Compact Hausdorff type, 
%  and suppose that $(A_i)_{i:I}$ is a collection of opens in $X$ such that 
%  $X = \bigcup_{i:I} A_i$. 
%  There then exists a finite subset $I_0\subseteq I$ with 
%  $X = \bigcup_{i:I_0} A_i$. 
%\end{lemma} 

\begin{lemma}
  In a Compact Hausdorff space, the complement of an open is closed, and the complement of a closed is open. 
\end{lemma}
\begin{proof}
  Let $e : S \to X$ be the quotient map of a Stone space by a closed equivalence relation. 
  and let $(A_n)_{n:\mathbb N}$ be a countable family of decidable subsets in $S$. 

  First, we claim that 
  $X - \bigcup_{n:\mathbb N} e(A_n)$
  is closed in $X$. 
\end{proof}

\begin{lemma}
  Whenever $X$ is compact Hausdorff, $F_0, F_1$ are closed and disjoint, 
  there exist $G_0, G_1$ disjoint clopen such that 
  $F_i \subseteq X - G_{1-i}$ and $G_0 \cup G_1 = X$. 
  
\end{lemma}

%%\subsection{Intersection of closed in compact Hausdorff}

\begin{lemma}
  In a compact Hausdorff, closed sets are closed under intersection. 
\end{lemma}
\begin{proof}
  
\end{proof}



\begin{lemma}
  Any Stone space merely is a closed subspace of Cantor space. 
\end{lemma}
\begin{proof}
  Let $S$ be a Stone space, and let it's underlying Boolean algebra $B$ be generated by 
  $(b_n)_{n:\mathbb N}$ under quotient of the relations ${\phi_i}_{i:\mathbb N}$. 
  Then $S = \{ x: 2^\mathbb N | \forall_{i:\mathbb n} x(\phi_i) = 0\}$, 
\end{proof}


\begin{lemma}
  For, $D\subseteq 2^\mathbb N$ decidable, $\sim$ a closed equivalence relation on $2^\mathbb N$,
  the set $$\{x:S | \exists y : D (x\sim y)\}$$ is closed. 
\end{lemma}
\begin{proof}
  Let $x:S$. We need to show that $\exists (y:S) D(y) \wedge x \sim y$ is a closed proposition. 

  Note first that as $\sim $ is closed, $x \sim \cdot $ is a closed subset of $S$. 
  Therefore, $x\sim \cdot = \bigcap_{n:\mathbb N} E_n$ for $(E_n)_{n:\mathbb N}$ a
  countable family of decidable subsets of $S$, without losing generality, 
  we may even assume that $E_n \subseteq E_m$ whenever $m\geq n$. 
  We thus need to show that 
  $$
  \exists (y:S) D(y) \wedge (\bigcap_{n:\mathbb N} E_n)(y) 
  = 
  \exists (y:S) (\bigcap_{n:\mathbb N} D \cap  E_n)(y) 
  $$
  is closed. 
%
  Now we claim that 
  $\exists (y:S) D(y) \wedge E_n(y)$ is closed for all $n:\mathbb N$. 
  There merely exists an $m:\mathbb N$ such that both $D$ and $E_n$ only consider 
  the first $m$ entries of a sequence. 
  
\end{proof}



\begin{lemma}
  For $S$ Stone, $D\subseteq S$ decidable, 
  $\sim$ a decidable equivalence relation on $S$,
  the set $\{x:S | \exists y : D (x\sim y)\}$ is closed. 
\end{lemma}
\begin{proof}
  Let $B = 2^S$, so $S = Sp(B)$. 
  As $D$ is decidable, 
%  there is some $b:B$ such that $D(y) := (y(b) = 1)$. 
  there is some $n:\mathbb N$ such that $D(y)$ only depends on $y|_n$. 

  As $\sim$ is decidable, there is a finite set $I_0\subseteq \mathbb N$,
  such that $x\sim y = \prod_{i:I_0} x(i) = y(i)$. 

  Thus 
  $$
   \exists (y : D) (x\sim y) = 
  || \Sigma(y:2^\mathbb N) y(b) = 1 \wedge \prod(i:I_0) x(i) = y(i)||
  $$
\end{proof}



\begin{lemma}
  Let $S$ Stone, then $D\subseteq S$ is closed iff 
  $D\subseteq S\subseteq 2^{\mathbb N}$ is closed. 
\end{lemma}
\begin{proof}
  Follows immediately from countable intersection of basic clopen. 
\end{proof}




%%  Let $A,B\subseteq X$ be two closed subsets of a compact Hausdorff space $X = S/ \sim$. 
%%  If we know that closed subsets contain are exactly those containing their limits this is very easy right? 
%%  Then any sequence has it's limit both in $A$ and $B$. 
%%\begin{lemma}
%%  Whenever $x_n$ is a convergent sequence, so is $f(x_n)$. 
%%\end{lemma}
%%\begin{proof}
%%  Follows immediately from \Cref{sequenceConvergentIffLimit}.
%%\end{proof}
%%
%%
%%\begin{lemma}
%%  In a compact Hausdorff, whenever a subset $A$ contains all of its limit points, it is closed. 
%%\end{lemma}
%%
%%\begin{proof}
%%  Suppose $A\subseteq X$ contains all of it's limit points. We will show that $f^{-1}(A)$ is closed. 
%%  Let $(x_n)_{n:\mathbb N}$ be a sequence in $f^{-1}(A)$ with limit $l$, 
%%  then 
%%  $(f(x_n))_{n:\mathbb N}$ is a sequence in $A$ with limit $f(l)$. 
%%  $A$ contains $f(l)$ by assumption. 
%%  Therefore $l\in f^{-1}(A)$. 
%%  Thus every sequence in $f^{-1}(A)$ with a limit has its limit in $f^{-1}(A)$. 
%%\end{proof}
%%
%%\begin{lemma}
%%  In a Stone space, whenever a subset $A$ contains all of its limit points, it is closed. 
%%\end{lemma}
%%\begin{proof}
%%  Let $A \subseteq S$ contain all of it's limit points. 
%%  We will show $A$ is a countable intersection of decidable subsets of $S$, hence closed. 
%%  As $S$ is a subset of Cantor space, we may assume it is Cantor space. 
%%  Thus $A$ is a set of binary sequences. 
%%
%%  We will denote $D_n$ be the set of initial segments of length $n$ occuring in $A$. 
%%  We claim this is well defined, that's not a problem, as it's the image of an operation. 
%%
%%  Counterexample : $A = \{ \overline 0 | p\}$ which contains all of it's limit points
%%  (any sequence in $A$ must be $\overline 0$ constantly, which has a limit if the sequence exists in $A$). 
%%  However, $D_n$ is not decidable. 
%%  Also $A$ is not the intersection of countably many decidable sets I believe. 
%%  Unless off course $p$ is of the form $\alpha=0$, but those are not the only propositions.
%%  For example, the proposition $\beta\neq 0$ cannot be written in that form for general $\beta$. 
%%\end{proof}

%
%\subsection{Open propositions}
%\input{OvertlyDiscrete/FactorizationFin}
%
%\section{Analysis}
%
%\subsection{Convergence}
%\input{Convergence/convergenceClosed}
%\paragraph{Extensional convergence }  
\begin{definition}
  Let $B_\infty$ be the Boolean algebra on countably many generators $(p_n)_{n:\mathbb N}$ 
  over the equivalence $p_n\wedge p_m = 0 $ whenever $n \neq m$. 
\end{definition} 
\begin{definition}
  We denote $\Noo$ be the spectrum of $B_\infty$. 
\end{definition} 
\begin{lemma}
  $B_\infty$ is isomorphic with the Boolean algebra of 
  finite/cofinite subsets of $\mathbb N$. 
\end{lemma}
\begin{proof}
  To go from $B_\infty$ to subsets of $\mathbb N$, we send
  the generators $p_n$ to the singleton $\{n\}$, which are clearly finite. 
  We call the induced Boolean operation $f$. 

  To go from finite/cofinite subsets of $\mathbb N$ to $B_\infty$,
  a finite subset $I$ of $\mathbb N$ is sent to the element 
  $\bigvee_{i \in I} p_i$, and a cofinite subset $J$ is sent to the element 
  $\bigwedge_{i \in J^C} \neg p_i$.  
  We call this function $g$ and we need to show that $g$ is a Boolean morphism. 
  \begin{itemize}
    \item By deMorgan's laws, $g$ preserves $\neg$. 
    \item To see that $g$ respects $\vee$, we need to check three cases
      \begin{itemize}
        \item If both $I,J$ are finite, then 
        \begin{equation} 
          g(I \cup J) = \bigvee_{i\in I \cup J} p_i= \bigvee_{i\in I} p_i \vee \bigvee_{j\in J} p_j 
        \end{equation}
      \item If both $I,J$ are cofinite, we have
        \begin{equation}
          g(I) \vee g(J) = 
          \bigwedge_{i \in I^C} \neg p_i \vee 
          \bigwedge_{j \in J^C} \neg p_j 
          = 
          \bigwedge_{i\in I^C} 
          \bigwedge_{j \in J^C}(\neg p_i \vee  \neg p_j) 
        \end{equation}
        Now note that $\neg p_i \vee \neg p_j = \neg ( p_i \wedge p_j)$, which 
        is $1$ if $i \neq j$ and $p_i$ if $i =j$. 
        We can leave $1$ out of the meet, and we are left with the intersection of $I^C$ and $J^C$, so
        \begin{equation}
          g(I) \vee g(J) = 
          \bigwedge_{i \in (I^C \cap J^C)} \neg p_i
          = 
          \bigwedge_{i \in (I \cup J)^C} \neg p_i 
        \end{equation} 
        as the union of $I$ and $J$ is also cofinite, this equals 
          $ g( I \cup J)$. 
        \item If $I$ is finite and $J$ cofinite, we have 
        \begin{equation}
        g(I) \vee g(J) = (\bigvee_{i\in I} p_i) \vee (\bigwedge_{j \in J^C} \neg p_j)
        = \bigwedge_{j \in J^C} (\bigvee_{i \in I}( p_i \vee \neg p_j))
        \end{equation}
        If $i\neq j$, then $p_i\wedge p_j = 0$, hence $\neg p_j \geq p_i$ and $p_i \vee \neg p_j  = \neg p_j$
        If $i = j$, then $p_i \vee \neg p_j = 1$.
%        \begin{equation}
%        g(I) \vee g(J) = 
%        = \bigwedge_{j \in J^C} (\bigvee_{i \in I-J}( p_i \vee \neg p_j))
%        \end{equation}
%
%        \item If $I$ is cofinite and $J$ is finite, we have that $I \cup J$ is cofinite.
%        Thus 
%        \begin{equation}
%          g(I \cup J) = \bigwedge_{i \in (I \cup J)^C} \neg p_i
%        \end{equation}
%
      \end{itemize}
    \item The case for $\wedge$ is completely dual to the case for $\vee$. 
  \end{itemize}
We conclude that $g$ is a Boolean morphism. 
Furthermore, $g$ and $f$ are clearly inverses, thus the Boolean algebras are isomorphic. 
\end{proof}

  \begin{lemma}\label{lemBinftyNormalForm}
  Any element of $B_\infty$ can be written as 
  either $\bigvee_{i\in I} p_i$  or
  as $\bigwedge_{j\in J} \neg p_j$ 
  for finite $I,J\subseteq \mathbb N$. 
\end{lemma}
\begin{proof}
  Remark that whenever $n \neq m$, we have that 
  $\neg p_n \geq p_m$ as $p_m \wedge p_n = 0$. 
\end{proof}
There is canonical embedding $\mathbb N \hookrightarrow \Noo$, 
wich sends $n$ to the unique function $\chi_{n}$ sending $p_n$ to $1$. 
We denote $\infty \in \Noo$ for the function which is constantly $0$. 
By \Cref{PropMarkov}, if an element is not $\infty$, 
it comes from the embedding $\mathbb N \hookrightarrow \Noo$. 
\begin{lemma}\label{LemmaOpensContainingInfty}
  Let $U$ be an open subset of $\Noo$ containing $\infty$.
  Then there merely exists an $N:\mathbb N$ such that whenever $n\geq N$, 
  $\chi_n\in U$ as well. 
\end{lemma}
\begin{proof}
  It is sufficient to prove the lemma for $U$ a basic open. 
  Assume $b : B_\infty $ is such that 
  $U = \{ \phi: B_\infty \to 2| \phi(b) = 1\}$.
  Assume furthermore that $\infty \in U$.
%  so $U$ contains the function sending every $p_i$ to $0$. 
  by \Cref{lemBinftyNormalForm}, $b$ can have two forms.
  If $b = \vee_{i\in I} p_i$, then as $\infty(b) = 0$, 
  we must have $I = \emptyset$, and thus $b = 0$, 
  which means $U$ is empty, contradicting $\infty\in U$. 
  Therefore, 
  $b$ must be of the form $\wedge_{j \in J} \neg p_j$. 
  Note that for $N = \max J + 1$, whenever $n>J$, 
  $\chi_n$  sends $b$ to $1$. 
  Thus $\chi_n \in U$ as well, and we are done. 
\end{proof}

\begin{definition}
  Let $\alpha$ be a sequence in $X$, we say that $\alpha$
  is convergent iff there exists an extension. 
  \begin{equation}\begin{tikzcd}
    \mathbb N \arrow[r, "\alpha"] \arrow[d,hook]  & X \\
    \Noo \arrow[ru,dashed]
  \end{tikzcd}\end{equation}  
\end{definition}  



\begin{proposition}
  A sequence is convergent iff it has a limit
\end{proposition}
\begin{proof}
  Let $\alpha$ be convergent, with extension $\overline \alpha$.
  we claim that $\overline \alpha(\infty)$ is a limit of $\alpha$.
  Let $U \subseteq X$ be an open containing $x$. 
  As $\overline\alpha^{-1}(U)$ is an open subset of $\Noo$ containing $\infty$,
  \Cref{LemmaOpensContainingInfty} tells us there exists some $N$ such that $[N,\infty]\subseteq \overline \alpha^{-1}(U)$. 
  Thus there exists an $N$ such that for $n\geq N$, we have $\alpha(n) \in U$, as required. 

  Conversely, suppose $\alpha$ has limit $x$. 
  Assume $X = Sp(B)$, and let $b\in B$. Then $b$ corresponds to a decidable subset $U\subseteq X$.
  For any decidable subset $U \subseteq X$, we have 
  $\alpha^{-1}(U)$ a decidable subset of $\mathbb N$. 
  We claim that $\alpha^{-1}(U)$ is either finite or cofinite. 
  As $U$ is decidable, we can decide wheter $x\in U$. If $x\in U$, $\alpha^{-1}(U)$ is cofinite, as 
  $\alpha(n) \in U$ for all $n \geq N$ for some $N$. 
  If $x\notin U$, we have $x\in U^C$, which is also decidable and therefore $\alpha^{-1}(U^C)$ is cofinite. 
  As $\alpha^{-1}(U) ^ C = \alpha^{-1}(U^C)$, it follows that $\alpha^{-1}(U)$ is finite. 
  Thus $\alpha^{-1}(U)$ is finite or cofinite for any decidable subset $U\subseteq X$. 
  Finite and cofinite subsets of $\mathbb N$ correspond to elements of $B_\infty$. 
  Therefore, $\alpha$ induces a map $B \to B_\infty$, which corresponds to a map 
  $\overline \alpha: \Noo \to X$. 

  We claim that $\overline \alpha$ extends $\alpha$. 
  Denote $\iota$ for the map $\mathbb N \to \Noo$. 
  We need to show that $\overline \alpha \circ \iota = \alpha$. 
  First note that by definition, we have that $(\overline \alpha \circ \iota)^{-1}(U) = \alpha^{-1}(U)$ 
  for any decidable $U\subseteq X$. 





\end{proof}




%
%\subsection{The interval}
%%The goal of this section is to define the interval $[-2,2]_\mathbb R$ as a scheme. 
We assume $\N, \mathbb Q$ have been defined in HoTT
with linear propostional order relations $<,\leq, > ,\geq$ playing nicely together 
and standard algebraic operations. 
From these, we can define the subtype $\mathbb Q_{>0}=\sum_{q : \mathbb Q} (q>0)$, 
and the absolute-value function $|\cdot|$ on $\mathbb Q$. 

\begin{definition}
  A pre-Cauchy sequence is a sequence of rational numbers $(q_n)_{n: \N}$ with $-2 \leq q_n \leq 2$ 
  for all $n:\N$
%  together with a term of type
  such that for every $\epsilon: \mathbb Q_{>0}$, we have an $N_\epsilon:\N$, 
  such that whenever $n,m \geq N_\epsilon$, we have 
\begin{equation}
%  \forall \epsilon : \mathbb Q_{>0} \Sigma N : \N \forall m,n : \N (m,n \geq N) \to 
  | q_n - q_m | \leq \epsilon
\end{equation} 
\end{definition}

\begin{definition}
Given two pre-Cauchy sequences $p = (p_n)_{n\in\N}, q=(q_n)_{n\in\N}$, 
we define the proposition $p \sim_C  q$ as 
%for all $\epsilon : \mathbb Q_{>0}$ there exists an $N :\N$ such that whenever $n \geq N$, we have
\begin{equation}
  p \sim_C q : = \forall (\epsilon : \mathbb Q_{>0} )\exists ( N :\N) \forall (n : \N) ((n \geq N) \to 
  (| p_n - q_n| \leq  \epsilon))
\end{equation}
\end{definition}
Note that $\sim_C$ defines an equivalence relation on pre-Cauchy sequences. 
\begin{definition}
We define the type of Cauchy sequences as the type of pre-Cauchy sequences quotiented by $\sim_C$. 
\end{definition}

%\begin{definition}
%  A binary sequence consists of an initial segment $I \subseteq \N$
%  and a function $x:I \to 2$. 
%If $I$ is (in)finite, we call the binary sequence (in)finite as well. 
%\end{definition} 
%
%For $x$ a finite binary sequence and $y$ any binary sequence, 
%we'll denote $(x,y)$ for their concatenation, 
%and $\overline x$ for the infinite sequence repeating $x$. 
%
Denote $T = \{-1,0,1\}$. 
\begin{lemma}
  $T^\N$ is a scheme. 
\end{lemma}
\begin{proof}
  Sketch: partition $2^\N$ as follows: 
  For $\alpha: 2^\N$, we'll make a sequence $\beta: T^\N$.
  consider for each $n$ the $n$'th block of 2 entries in $\alpha$
  if both are $0$, $\beta(n) = 0$. 
  If the first is $1$, $\beta(n) = -1$
  If first is $0$ and the second is $1$, then $\beta(n) = 1$. 
  This is a closed equivalence relation. 
\end{proof} 

Consider the relation $\sim_s$ on $T^{\N}$, 
such that for any finite binary sequence $x$, we have 
\begin{align}
  (x,1,\overline 0) &\sim_t (x ,0, \overline 1) \\
  (x,-1,\overline 0) &\sim_t (x ,0, \overline {-1})\\
  (x,1,\overline {-1}) &\sim_t (x , \overline 0) \\
  (x,-1,\overline {1}) &\sim_t (x , \overline 0) 
\end{align} 
\begin{lemma}
$\sim_t$ induces a closed equivalence relation on $2^\N$. 
\end{lemma}
\begin{proof}
  TODO
\end{proof} 

\begin{proposition}\label{propTernaryCauchy}
  $T^\N/ \sim_t$ is isomorphic to the type of Cauchy sequences. 
\end{proposition} 
\begin{definition}%Construction might be better than definition here, but WIP so who cares. 
  For $\alpha: T^\N$, define the rational sequence $tri(\alpha)$ by 
  \begin{equation} (tri(\alpha))_n :  = \sum\limits_{0 \leq i \leq n} \frac{\alpha(i)} { 2^{i}} \end{equation}  
  This sequence is pre-Cauchy with $N_\epsilon$ given by the first $n$ with $(\frac12)^n<\epsilon$. 
\end{definition}  
%
%  Also, whenever $\alpha\sim_t \beta$, we have 
%  $tri(\alpha) \sim_C tri(\beta)$. 
%  Therefore $tri$ induces a function from $T^\N / \sim_t$ to Cauchy sequences. 
\begin{definition}
  Given a pre-Cauchy sequence $p$, 
  we will define a $T$-sequence $\alpha  = c(p): T^\N$.
  Consider any $i:\N$, and suppose $\alpha(j)$ has been defined for $0 \leq j<i$. 
%
  Let $\epsilon_i = (\frac12)^{i+1}$. %Placeholder value.
  Define $N_i:= N_{\epsilon_i}$. %is such that for $n,m \geq N_i$, we have $|p_n - p_m| < \epsilon_i$. 
%
  Consider 
  \begin{equation}
    \widetilde p_i = p_N - \sum\limits_{0\leq j < i} \frac {\alpha(j)}{2^{j}}.
  \end{equation}
  As the order on $\mathbb Q$ is total, we can define 
  \begin{equation}
    \alpha(i) = \begin{cases}
    \phantom{-} 1  \text{ if } \widetilde p_i \geq    (\frac12)^{i} \\
    -1             \text{ if } \widetilde p_i \leq  - (\frac12)^{i} \\
    \phantom{-} 0 \text{ otherwise } 
    \end{cases} 
  \end{equation}  
\end{definition} 
We shall now prove the following four things: 
\begin{itemize}
  \item 
    $c(tri(\alpha)) \sim_t \alpha$ for any $\alpha: T^n$.
  \item 
    $tri(c(p)) \sim_C p$ for any pre-Caucy sequence $p$. 
  \item 
    Whenever $p \sim_C q$, we have $c(p)\sim_t c(q)$. 
  \item 
    Whenever $\alpha \sim_t \beta$, we have $tri(\alpha) \sim_C tri(\beta)$. 
\end{itemize}
It follows that $c$ and $tri$ are maps between Cauchy sequences and $T^\N /\sim_t$ 
which are each other's inverse, proving Proposition \ref{propTernaryCauchy}
\begin{lemma} $tri(c(p)) \sim_C p$ for any pre-Caucy sequence $p$. 
\end{lemma} 



\begin{proof}
  Let $\epsilon>0$ be given, consider $n:\N$ such that
  $(\frac12)^n < \epsilon$. 
  We claim that for $m\geq N_n$, we have that $|p_m- tri(c(p))_m| < \epsilon$. 

  By definition $p_{N_n} $  
\end{proof} 






%
\begin{definition}
  A \textbf{Cauchy sequence} is a sequence $x : \mathbb N \to \mathbb Q$ such that
  for any $n,m:\mathbb N$, we have %$0\leq x_n \leq 1$ and 
$|x_n-x_m| \leq (\frac12)^n + (\frac12)^m$. 
\end{definition}

\begin{definition}
Given two Cauchy sequences $p = (p_n)_{n\in\mathbb N}, q=(q_n)_{n\in\mathbb N}$, 
we define the proposition $p \sim_C  q$ as 
\begin{equation}
  p \sim_C q : = \forall (\epsilon : \mathbb Q_{>0} )\exists ( N :\mathbb N) \forall (n : \mathbb N) ((n \geq N) \to 
  (| p_n - q_n| \leq  \epsilon))
\end{equation}
\end{definition}

\begin{definition}
  The type of \textbf{Cauchy reals} is given by 
  the type of Cauchy sequences modulo $\sim_C$.
\end{definition}

%\begin{definition}
%  A Cauchy sequence in the interval is a Cauchy sequence $x$ such that 
%  for any $n:\mathbb N$, we have $0\leq x_n \leq 1$. 
% % 
%  The interval of Cauchy reals is given by the type of Cauchy sequences in the interval 
%  modulo $\sim_C$. 
%\end{definition}  

We want to show that the interval of Cauchy reals are a scheme. 
Informally, to any binary sequence $\alpha : \mathbb N \to 2$, 
we can associate a Cauchy sequence 
\begin{equation}n\mapsto \sum\limits_{i = 0 }^n \frac {\alpha(i)}{2^{i+1}}\end{equation}
and we are going to give a closed relation on Cantor space such that 
two binary sequences are equivalent iff they correspond to the same Cauchy reals. 
%
First, we'll need some notation.
\begin{definition}
Given a binary sequence $\alpha:\mathbb N \to 2$ and a natural number $n : \mathbb N$  
we denote $\alpha|_n: \mathbb N_{\leq n} \to 2$ for the 
restriction of $\alpha$ to a finite sequence of length $n$. 
We denote $\overline 0, \overline 1$ for the binary sequences which are constantly $0$ and $1$ respectively. 
We denote $0,1$ for the sequences of length $1$ hitting $0,1$ respectively. 
If $x$ is a finite sequence and $y$ is any sequence, denote $x\cdot y$ for their concatenation. 
\end{definition} 
Now we'll give a definition for when two finite binary sequences of length $n$ correspond 
to real numbers whose distance is $\leq (\frac12)^n$.
Basically, we want for every finite sequence $z$ that 
$(z \cdot 0 \cdot \overline 1)$ and  $(z \cdot 1 \cdot \overline 0)$ are equivalent. 

\begin{definition}
Now let $n:\mathbb N$ and $x,y:\mathbb N_{\leq n} \to 2$ be two sequences of length $n$. 
We say $x,y$ are near if we have an $m:\mathbb N$ with $m\leq n$
and some $a: \mathbb N_{\leq m} \to 2$, 
such that one of $(a \cdot 0 \cdot \overline 1)|_n,  ( a \cdot 1 \cdot \overline 0)|_n$
is equal to $x$ and the other is equal to $y$. 
We denote $\text{near}_n(x,y)$ if $x,y$ are near. 
%
To be precise, we define 
\begin{equation}
  \text{near}_n(x,y) = 
\Sigma(m:\mathbb N) m \leq n \wedge 
  \Sigma (a : Fin_m \to 2) 
\bigg( \big( (x,y) = 
((a \cdot 0 \cdot \overline 1)|_n,  ( a \cdot 1 \cdot \overline 0)|_n)
\big)
\bigvee 
\big(
  (y,x) = 
((a \cdot 0 \cdot \overline 1)|_n,  ( a \cdot 1 \cdot \overline 0)|_n)
\big)
\bigg)
\end{equation}
\end{definition}
%\begin{lemma}
%  For every $n:\mathbb N$, $\text{near}_n$ is an equivalence relation. 
%\end{lemma}
\begin{remark}
Remark that when $x,y$ are near, $m$ and $a$ as above are unique. 
Thus $\text{near}_n(x,y)$ is a proposotion. 
%
Furthermore, to check whether $x,y$ are near, we need only make $n$ comparisons, 
thus $\text{near}_n(x,y)$ is decidable. 
%
Note that in the above definition, we allow $m = n$ and therefore $x$ is near to itself for any finite sequence $x$. 
Furthermore, we have defined nearness to be symmetric. 
However, it is not a transtive relation. 
After all, the sequence $010$ and $011$ are near and the sequence $011$ and $100$ are near, 
but $010$ is not near to $100$. 
\end{remark}
\begin{definition}
  We define the following relation on Cantor space for $\alpha, \beta: 2^\mathbb N$.
  \begin{equation}
    \alpha \sim_t \beta = \forall (n : \mathbb N) 
    \text{near}_n(\alpha|_n, \beta|_n)
  \end{equation}
\end{definition}
\begin{lemma}
  $\sim_t$ is a closed equivalence relation. 
\end{lemma}
\begin{proof}
   Let $\alpha, \beta, \gamma : 2^\mathbb N$. 
   As the dependent product of propositions is a proposition, $\alpha \sim_t\beta$ is a proposition. 
   %
   Furthermore, the closedness follows from decidability of $\text{near}_n(\alpha|_n, \beta|_n)$. 
   One could define $\gamma(n) = 1$ iff $\text{near}_n(\alpha|_n, \beta|_n)$
   
   As nearness is reflexive and symmetric, so is $\sim_t$. 

   Now suppose $\alpha \sim_t \beta$ and $\beta\sim_t \gamma$. 
   We claim that $\alpha \sim_t \gamma$. 

   Let $n:\mathbb N$, we need to show that 
   $\text{near}_n(\alpha|_n , \gamma|_n)$. 
   Let $(a,m)$ witness that $\text{near}_n(\alpha|_n, \beta|_n)$.
   \begin{itemize}
     \item 
   If $m=n$, we have that $\alpha|_n = \beta|_n$, and therefore 
   $\text{near}_n(\beta|_n, \gamma|_n) \leftrightarrow \text{near}_n(\alpha|_n, \gamma|_n)$.
   \item 
     If $m< n$, we have that $\alpha(m+1) \neq \beta(m+1)$, thus 
     $\alpha|_k \neq \beta|_k$ for all $k>m$, 
     but we still have $\text{near}_k(\alpha|_k, \beta|_k)$ for these $k$. 
     Therefore, $\alpha = a \cdot 0 \cdot \overline 1$ and 
  \end{itemize}
   



\end{proof}

\begin{proposition}
  The interval of Cauchy reals is isomorphic to $2^\mathbb N / \sim_t$. 
\end{proposition} 

%%\printindex
%
%\section{Directed Univalence}
%% !TEX encoding = UTF-8 Unicode
\subsection{Subquotient systems}

\begin{definition}
A subquotient pre-system consists of $X$ a type and $U$ a class of propositions.
\end{definition}

\begin{definition}
For $(X,U)$ a subquotient pre-system, we define:
\[Sub_{X,U} = \sum_{Y:\Type} \exists (P : X\to U).\ Y = \Sigma_XP\]
\[SubQ_{X,U} = \sum_{Y:\Type} \exists (P : X\to U)(R: \Sigma_XP\to \Sigma_XP\to U\ \mathrm{equivalence\ relation}).\ Y = (\Sigma_XP)/R\]
\end{definition}

\begin{definition}
We say a class of types $T$ has local choice if for all $X\in T$ and $P:X\to\Type$ such that:
\[\prod_{x:X}\propTrunc{P(x)}\]
there merely exists $Y\in T$ and a surjection:
\[f:Y\to X\]
such that:
\[\prod_{y:Y}P(f(y))\]
\end{definition}

\begin{proposition}\label{lex-sub-pro}
Assume $(X,U)$ a Subquotient pre-system such that:
\begin{itemize}
\item Identity types in $X$ are in $U$.
\item $U$ is closed by $\sum$ and $\top$.
\item $\propTrunc{X}$ and $X\times X = X$.
\item $Sub_{X,U}$ has local choice.
\end{itemize}
Then we $SubQ_{X,U}$ has the following:
\begin{itemize}
\item Stability under finite limits.
\item Stability by quotient by equivalence relation with value in $U$.
\item Local choice.
\end{itemize}
\end{proposition}

\begin{proof}
\begin{itemize}
\item First we check that $SubQ_{X,U}$ has local choice. Since we assume that $Sub_{X.U}$ has local choice and that any type in $SubQ_{X,U}$ is covered by a type in $Sub_{X,U}$ by definition, it is enough to check that $Sub_{X,U}\subset SubQ_{X,U}$ to conclude. But given $S = \Sigma_XP$ in $Sub_{X,U}$ we have that:
\[\Sigma_XP = (\Sigma_XP)/L\]
where:
\[L((x,_),(y,_))= (x=_Xy)\]
and since $x=_Xy$ is assumed to be in $U$ we conclude.

\item Stability by quotient by equivalence relation with value in $U$ is clear.

\item Now we want to check stability under finite limits.

First we check that $U\subset SubQ_{X,U}$. Indeed assume $P\in U$, then with $L$ the trivial relation we have:
\[(X\times P) / L = \propTrunc{X\times P} = P\]
as $\propTrunc{X} = 1$ so that since $\top\in U$ we conclude $P\in SubQ_{X,U}$.

This means that $SubQ_{X,U}$ is stable by identity type, and that $1\in SubQ_{X,U}$.

All that is left is to check stability under $\Sigma$. Assume $S: SubQ_{X,U}$ and $T:S\to SubQ_{X,U}$. Through the fact that $S$ is covered by a type in $Sub_{X,U}$ and local choice for $Sub_{X,U}$ we merely get $S':Sub_{X,U}$, say $S'=\Sigma_XP$ and a surjective map:
\[f:S'\to S\]
such that for all $x:\sum_XP$ we have:
\[T(f(x)) = (\Sigma_XP_x)/R_x\]
so we get a surjective map:
\[\sum_{x:X}\sum_{P(x)}(\Sigma_X P_x)/R_x  \to \sum_ST\]
Then the identity types in $\sum_ST$ are in $U$ as $U$ is stable by $\Sigma$, so it is enough to check that:
\[\sum_{x:X}\sum_{P(x)}(\Sigma_X P_x)/R_x\]
is in $SubQ_{X,U}$ to conclude, as we can then apply the previous bullet-point. But this type is equivalent to:
\[\left(\sum_{(x,x'):X\times X}\Sigma_{P(x)}P_x(x')\right)/ L\]
where:
\[L((x,x'),(y,y')) =\sum_{x=y} R_y(x',y') \]
which is in $SubQ_{X,U}$ as $U$ is stable by $\Sigma$, $x=y$ in in $U$ and $X\times X = X$.
\end{itemize}
\end{proof}

\begin{proposition}\label{coproducts-sub-quo}
Assume $(X,U)$ a subquotient pre-system such that $\bot\in U$ and $X+X = X$. Then $SubQ_{X,U}$ is stable by finite coproducts.
\end{proposition}

\begin{proof}
We have that:
\[\bot = (X\times \bot) / L\]
where $L$ is the unique such equivalence relation. Since $\bot\in U$ we conclude that $\bot\in SubQ_{X,U}$.

Given $S$ and $S'$ in $SubQ_{X,U}$, say:
\[S = (\sigma_XP)/R\]
\[S' = (\sigma_XP')/R'\]
Then we have that:
\[S+S' = \left(\sum_{X+X}[P,P']\right) L\]
where:
\[[P,P'](l(x)) = P(x)\]
\[[P,P'](r(x)) = P'(x)\]
and:
\[L(l(x),l(y)) = R(x,y)\]
\[L(l(x),r(y)) = \bot\]
\[L(r(x),l(y)) = \bot\]
\[L(r(x),r(y)) = R'(x,y)\]
Since $\bot\in U$ and $X+X=X$ we conclude that $S+S'$ is in $SubQ_{X,U}$.
\end{proof}

\begin{proposition}\label{prop-sub-quo}
Assume $(X,U)$ a subquotient pre-system such that $\top\in U$ and for all $S\in Sub_{X,U}$ we have that $\propTrunc{S}\in U$. Then any proposition in $SubQ_{X,U}$ is in $U$. 
\end{proposition}

\begin{proof}
If we have a proposition $S$ in $SubQ_{X,U}$, say:
\[S = (\Sigma_XP)/R\]
then:
\[S = \propTrunc{S} = \propTrunc{\Sigma_XP}\]
and we can conclude.
\end{proof}

\begin{definition}
A subquotient system is a subquotient pre-system obeying the hypothesis of \cref{lex-sub-pro}, \cref{coproducts-sub-quo} and \cref{prop-sub-quo}.
\end{definition}

We just pack all this up in one theorem:

\begin{theorem}\label{stabitity-sub-quo}
Let $(X,U)$ be a subquotient system, then $SubQ_{X,U}$ has the following:
\begin{itemize}
\item Stability under finite limits.
\item Stability under finite coproducts.
\item Stability under quotient by equivalence relations.
\item Local choice.
\end{itemize}
\end{theorem}

We have two main examples in mind.

\begin{example}
The subquotient pre-system $St = (2^\N,\mathrm{Closed})$ is a quotient system. We have that $Sub_{St}$ is the type of stone spaces, and $CHaus = SubQ_{St}$ the type of compact Haussdorf spaces.

Closed propositions are stable by $\Sigma$. TODO 

We also need that for any stone space $S$ we have that $\propTrunc{S}$ is a closed proposition. TODO
\end{example}

\begin{example}
The subquotient pre-system $Od = (\N,\mathrm{Open})$ is a quotient system. We have that $ODisc = SubQ_{Od}$ the type of so-called overtly discrete types.

A key observation is that open propositions are stable by countable disjunctions.

This means open propositions are stable by $\sum$ because we can assume:
\[P = \Sigma_{n:\N} A(n)\]
with $A(n)$ decidable and:
\[Q:P \to \mathrm{Open}\]
Then we have that:
\[\Sigma_PQ = \exists(n:\N).\ \Sigma_{A(n)} Q(n)\]
which is open as $\Sigma_{A(n)} Q(n)$ is open for all $n$, as $A(n)$ is decidable.

Types in $Sub_{Od}$ even have full choice because both $\N$ and decidable propositions have full choice.
\end{example}

So both $ODisc$ and $CHaus$ enjoys the conclusion of \cref{stabitity-sub-quo}.


\subsection{Tychonov}

\begin{proposition}\label{tychonov}
Assume $(X,U)$ and $(Y,C)$ two subquotient system such that:
\begin{itemize}
\item $S\to Y$ is in $SubQ_{Y,C}$ for all $S:Sub_{X,U}$.
\item If $P\in U$ and $Q:P\to C$ then $\prod_{p:P}Q(p) \in C$.
\item If $Q:X\to C$ then $\prod_{x:X}Q(x) \in C$.
\end{itemize}
Then we have the following:
\begin{itemize}
\item If $S:SubQ_{X,U}$ and $T:S\to SubQ_{Y,C}$, then:
\[\prod_{s:S}T(s) \in SubQ_{Y,C}\]
\end{itemize}
\end{proposition}

\begin{proof}
Note that for $S':Sub_{X,U}$ and $Q:S'\to C$ we have that:
\[\prod_{S'}Q\]
is in $C$.

Now we use local choice to get $S':Sub_{X,U}$ with a surjective map:
\[f:S'\to S\]
such that for all $s:S'$ we have:
\[T(f(s)) = (\Sigma_YU_s)/R_s\]

Then the map:
\[\prod_ST \to \prod_{s:S'}(\Sigma_YU_s)/R_s\]
is an embedding, its fiber over $\alpha$ is:
\[\prod_{s,t:S'} \prod_{f(s) =_S f(t)} \alpha(s) = \alpha(t)\]
which is in $C$ by the hypothesis. Therefore it is enough to prove that:
\[\prod_{s:S'}(\Sigma_YP_s)/R_s\]
is in $SubQ_{Y,C}$. 

But this type is the quotient of:
\[\prod_{s:S'}(\Sigma_YP_s)\]
by:
\[L(\alpha,\beta) = \prod_{s:S'} R_s(\alpha(s),\beta(s))\]
which is in $C$, therefore it is enough to to prove that:
\[\prod_{s:S'}(\Sigma_YP_s)\]
is in $SubQ_{Y,C}$.

But this type is equivalent to:
\[\sum_{f:S'\to Y} \prod_{s:S'}P_s(f(s))\]
Since $\prod_{s:S'}P_s(f(s))$ is in $C$, it is enough to prove that:
\[S'\to Y\]
is in $SubQ_{Y,C}$. But this is one of the hypothesis.
\end{proof}

\begin{definition}
Two subquotient systems $A,B$ are called dual if both $(A,B)$ and $(B,A)$ satisfy the hypothesis of \cref{tychonov}.
\end{definition}

\begin{example}
We have that $St = (2^\N,\mathrm{Closed})$ and $Od = (\N,\mathrm{Open})$ are dual quotient systems.

\begin{itemize}
\item We need that if $P$ open and $Q:P\to \mathrm{Closed}$, then $\Sigma_PQ$ is closed. Assume $Q=\Sigma_{n:\N}A(n)$ with $A(n)$ decidable, since open propositions have choice we can assume for $n:\N$ such that $A(n)$ that $Q(n) = \forall_{k:\N} B_n(k)$ with $B_n(k)$ decidable. Then:
\[\Sigma_PQ = \prod_{n,k:\N} \prod_{A(n)} B_n(k) \]
which is indeed closed.

It is clear that closed propositions are closed by countable products.

$\Sigma_\N P\to 2^\N$ is compact Hausdorff? Yes indeed, it is even Stone because it is equivalent to:
\[\prod_{k,n:\N} 2^{P(n)}\]
and $2^{P(n)}$ is Stone as $P(n)$ is open, indeed:
\[(\Sigma_\N A)\to 2\] 
for $A$ decidable is a countable product of Stone space, as $2^A$ is Stone for $A$ decidable.

\item First we check that given $S$ Stone, we have that:
\[S\to \N\]
is overtly discrete. Indeed identity types in $S\to \N$ are closed and there is a surjection from fundamental systems of idempotent in $2^S$ to $S\to \N$, so it is enough to prove that the type fundamental systems of idempotent in a c.p. algebra is overtly discrete. To have this it is enough to prove that countably presented algebra are overtly discrete. TODO
\end{itemize}
\end{example}

When applying \cref{tychonov} to $ODisc$ and $CHaus$ we get Tychonov theorem and its dual.


\subsection{Factorisation}

\begin{proposition}\label{factorisation-subquotient}
Assume given dual subquotient systems $(X,C)$ and $(Y,U)$. such that
such that:
\begin{itemize}
\item Given $S:Sub_{X,U}$ any map:
\[S\to \N\]
merely factors through a finite type.
\item Given $P\in C$ and $Q\in U$ we have that:
\[(P\to Q) \to (\neg P \lor Q)\]
\end{itemize}
Then for $S:SubQ_{X,C}$ and $T:SubQ_{Y,U}$, any map:
\[S\to T\]
merely factors through a finite type.
\end{proposition}

\begin{proof}
We proceed in three steps:
\begin{enumerate}[(i)]
\item First we show the factorisation of any map:
\[S\to T\]
for $S:Sub_{X,C}$ and $T:Sub_{Y,U}$. Indeed we then we merely have $Q:Y\to U$ such that:
\[T = \Sigma_YQ\]
so the map:
\[S\to \Sigma_YQ\]
induces a map:
\[S\to Y\]
which we factor as:
\[S\to \mathrm{Fin}(n) \to Y\]
This gives a factorisation of the starting map as follows:
\[S\to \Sigma_{\mathrm{Fin}(n)} Q \to T\]
Then for any $k:\mathrm{Fin}(n)$ we define:
\[S_k \subset S\]
\[s\in S_k := (f(s) = k)\]
which is decidable and therefore in $C$, so that $\propTrunc{S_k}$ is in $C$. We have that:
\[\propTrunc{S_k} \to Q(k)\]
therefore we have
\[\neg S_k \lor Q(k)\]
Now we just need to remove the $k$ such that $\neg S_k$ to get a factorisation:
\[S\to \mathrm{Fin}(l) \to \Sigma_{\mathrm{Fin}(n)}Q\]
\item We show how to factor any map:
\[S\to \mathrm{Fin}(n)/R\]
where $S:SubQ_{X,U}$ and $R$ in an equivalence relation in $U$. We do this by induction on $n$, if $n=0,1$ it is clear. Using local choice we get:
\begin{center}
\begin{tikzcd}
S'\ar[d,"p"]\ar[r,"f"] & \mathrm{Fin}(n)\ar[d] \\
S\ar[r,"g"] & \mathrm{Fin}(n)/R
\end{tikzcd}
\end{center}
Then for all $k:\mathrm{Fin}(n)$ we define:
\[S'_k \subset S'\]
\[s'\in S'_k := (f(s) = k)\]
which is a decidable cover of $S'$. Then we use:
\[S_k = p(S'_k)\]
which is a cover of $S$ such that:
\[(S_k\cap S_l) \to R(k,l)\]
since $g$ restricted to an $S_i$ has value $[i]$. But since $\propTrunc{S_k\cap S_l}$ is in $C$, this means that:
\[\neg(S_k\cap S_l) \lor R(k,l)\]
If for all $k\not=l$ we have $\neg(S_k\cap S_l)$ then we can conclude by factoring through $\mathrm{Fin}(n)$, otherwise we have $R(k,l)$ for some $k\not=l$ and we have that $\mathrm{Fin}(n)/R$ is equivalent to $\mathrm{Fin}(n-1)/R$ where we have removed $l$. We conclude by induction on $n$.
\item  Now we show how to factor any map:
\[S\to T\]
for $S:SubQ_{X,C}$ and $T:SubQ_{Y,U}$. There is $Q:Y\to U$ and $R$ an $U$-valued equivalence relation on $\Sigma_YQ$ such that:
\[T = (\Sigma_YQ)/R\]
Using local choice we have that:
\begin{center}
\begin{tikzcd}
S'\ar[d]\ar[r] & \Sigma_YQ\ar[d] \\
S\ar[r] & (\Sigma_YQ)/R
\end{tikzcd}
\end{center}
and by factoring the top map using (i), we can get a factorisation:
\[S\to \mathrm{Fin}(n)/R\to (\Sigma_YQ)/R\]
and we conclude using (ii).
\end{enumerate}
\end{proof}

\begin{definition}
Dual subquotient systems $(X,C)$ and $(Y,U)$ obeying the hypothesis of \cref{factorisation-subquotient} are called factorial.
\end{definition}

\begin{remark}
We have that $St = (2^\N,\mathrm{Closed})$ and $Od = (\N,\mathrm{Open})$ are factorial. We need to check the following:
\begin{itemize}
\item Given a stone space $S$, any map $S\to\N$ merely factors through $\N$. This is known.
\item Given $P$ closed and $Q$ open, we have that:
\[(P\to Q) = \neg P \lor Q\]
We just note that as $\neg P \lor Q$ is open it is $\neg\neg$-stable, and the rest is just intuitionistic logic.
\end{itemize}
From this we know that for $S$ compact Hausdorff and $T$ overtly discrete, any map:
\[S\to T\]
merely factors through a finite type. 
\end{remark}

\subsection{Scott continuity}




%
%\appendix
%\section{Appendix}
%\section*{Appendix 1: Quillen Patching}

We reproduce the argument in Quillen's paper \cite{Quillen}, as simplified in \cite{LQ}.
This technique of Quillen Patching has been replaced by the equivalence in Proposition \ref{Matthias}.

If $P$ and $Q$ are two idempotent matrix of the same size, let us write $P\simeq Q$ for expressing that $P$ and $Q$ presents
the same projective module (which means that there are similar, which is in this case is the same as being equivalent).

If we have a projective module on $A[X]$, presented by a matrix $P(X)$, this module is extended
precisely when we have $P(X)\simeq P(0)$.

\begin{lemma}
  If $S$ is a multiplicative monoid of $A$ and $P(X)\simeq P(0)$ on $A_S[X]$ then there exists
  $s$ in $S$ such that $P(X+sY)\simeq P(X)$ in $A[X]$.
\end{lemma}

\begin{lemma}
  The set of $s$ in $A$ such that $P(X+sY)\simeq P(X)$ is an ideal of $A$.
\end{lemma}

\begin{corollary}
  If we have $M$ projective module of $A[X]$ and $S_1,\dots,S_n$ comaximal multiplicative monoids of $A$
  such that each $M\otimes_{A[X]} A_{S_i}[X]$ is extended from $A_{S_i}$ then $M$ is extended from $A$.
\end{corollary}

Let us reformulate in synthetic term this result. Let $A$ be a f.p. $R$-algebra and $L:Sp(A)\rightarrow BGL_1^{\A^1}$.
Then $L$ corresponds to a projective module of rank $1$ on $A[X]$. We can form
$$T(x) = \prod_{r:R}L~x~r = L~x~0$$
and $\|T(x)\|$ expresses that $L~x$ defines a trivial line bundle on $\A^1 = Sp(R[X])$.
It is extended exactly when we have
$\|{\prod_{x:Sp(A)}T(x)}\|$. We can the use Zariski local choice to state.

\begin{proposition}\label{c2}
  We have the implication $(\prod_{x:Sp(A)}\|T(x)\|)\rightarrow \|\prod_{x:Sp(A)}T(x)\|$.
\end{proposition}

\newpage

\section*{Appendix 2: Classical argument}

We reproduce a message of Brian Conrad in mathoverflow.

\medskip

``We know that the Picard group of projective $(n-1)$-space over a field $k$ is $\Z$
generated by $O(1)$.
This underlies the proof that the automorphism group of such a projective space is $PGL_n(k)$.
But what is the automorphism group of $\bP^{n-1}(A)$ for a general ring $A$? Is it $PGL_n(A)$?
It's a really important fact that the answer is yes.
But how to prove it? It's a shame that this isn't done in Hartshorne.

By an elementary localization, we may assume $A$ is local.
In this case we claim that $\Pic(\bP^{n-1}(A))$ is infinite cyclic generated by $O(1)$.
Since this line bundle has the known $A$-module of global sections,
it would give the desired result if true by the same argument as in the field case.
And since we know the Picard group over the residue field, we can twist
to get to the case when the line bundle is trivial on the special fiber. How to do it?

\medskip

 Step 0: The case when $A$ is a field. Done.

 \medskip

 Step 1: The case when $A$ is Artin local.
 This goes via induction on the length, the case of length $0$ being Step $0$
 and the induction resting on cohomological results for projective space over the residue field.

  \medskip

 Step 2: The case when $A$ is complete local noetherian ring. This goes
 using Step 1 and the theorem on formal functions (formal schemes in disguise).

  \medskip

 Step 3: The case when $A$ is local noetherian.
 This is faithfully flat descent from Step 2 applied over $A~\widehat{}$

 \medskip
 
 Step 4: The case when $A$ is local:
 descent from the noetherian local case in Step 3 via direct limit arguments.

\medskip
 
QED''

%
\printbibliography
%
\end{document}
