% latexmk -pdf -pvc main.tex
\documentclass{../util/zariski-small}


\title{Synthetic Stone Duality}

\begin{document}

\author{Felix Cherubini, Thierry Coquand, Freek Geerligs and Hugo Moeneclaey}

\maketitle

%\begin{abstract}
%In synthetic algebraic geometry (SAG) \cite{draft}, we study finitely presented algebras over a commutative ring. 
%In this work, we study countably presented Boolean algebras instead. 
%Where the finitely presented algebras over a commutative ring induce a Zariski topos, 
%%the opposite category of these 
%the countably presented Boolean algebras induce the topos of light condensed sets \cite{Scholze}. 
%\cite{draft} proposes an axiomatization of the Zariski topos in univalent homotopy type theory \cite{hott}. 
%In this work, we propose similar axioms, which we expect to be modelled by light condensed sets. 
%% Furthermore, spectra of countably presented Boolean algebras correspond to quotients of Cantor space
%% which is cool because reasons
%\end{abstract} 
%
\rednote{The following is a collection of notes on work in progress.}
\section{Countably presented Boolean algebras}
\subsection{Definitions}
\begin{definition}
  The language of Boolean algebras consists of
  \begin{itemize}
    \item Two symbols for constants $0,1$. 
    \item Two binary function symbols $\wedge, \vee$ called meet and join respectively. 
    \item One unary function symbol $\neg$ called negation. 
  \end{itemize}
\end{definition}
\begin{definition}
  The algebraic theory of Boolean algebras is as follows:
  \begin{itemize}
    \item We have axioms regarding order, sometimes called identity and absorption:
    \begin{itemize}
      \item $0 \wedge a = 0$. 
      \item $1 \wedge a = a$. 
      \item $0 \vee a = a$. 
      \item $1 \vee a = 1$. 
      \item $a \vee (a \wedge b) = a$
      \item $a \wedge (a \vee b) = a$. 
    \end{itemize}
    \item We have axioms regarding negations:
      \begin{itemize}
        \item $a \vee \neg a = 1$. 
        \item $a \wedge \neg a = 0$. 
        \item $\neg 1 = 0$. 
        \item $\neg \neg a = a$. 
      \end{itemize}
    \item We have axioms regarding commutativity and associativity:
      \begin{itemize}
        \item $a \vee b = b \vee a$. 
        \item $a \wedge b = b \wedge a$. 
        \item $(a \vee b) \vee c = a \vee (b \vee c)$. 
        \item $(a \wedge b) \wedge c = a \wedge (b \wedge c)$. 
      \end{itemize}
    \item We have axioms regarding distributivity:
      \begin{itemize}
        \item $a \vee (b \wedge c) = ( a \vee b) \wedge (a \vee c)$. 
        \item $a \wedge (b \vee c) = (a \wedge b) \vee ( a\wedge c)$. 
      \end{itemize}
    \item We have axioms called deMorgan's laws:
      \begin{itemize}
        \item $\neg (a \vee b ) = (\neg a) \wedge (\neg b)$. 
        \item $\neg (a \wedge b) = (\neg a) \vee (\neg b)$. 
      \end{itemize}
  \end{itemize}
\end{definition}
\begin{definition}
  Some language we will use is as follows:
  \begin{itemize}
    \item A Boolean algebra is a set-theoretic model of the theory of Boolean algebras. 
      It consists of a set $B$ and terms $0_B,1_B: B, \neg_B : B \to B, \vee_B, \wedge_B : B \times B \to B$ 
      satisfying the theory of Boolean algebras. 
      Sometimes we will leave out the subscript $(\cdot)_B$. 
    \item A morphism of Boolean algebra is a function between such sets preserving the operations above. 
      If $A,B$ are Boolean algebras, we write $A \to B$ for the type of such morphisms. 
%    \item We denote $id_B$ for the identity morphism $B \to B$. 
    \item If we have have two morphisms $f:A\to B , g : B \to A$ such that 
      $f\circ g (b) = b, g \circ f (a) = a$ for all $a:A, b:B$, 
      we say $A$ and $B$ are isomorphic and write $A \simeq B$. 
  \end{itemize}
\end{definition}
\begin{remark}
  We sometimes write $a \leq b$ and  $b \geq a$. Both of these mean that 
  $a\wedge b = a$, or equivalently $ a\vee b = b$. 
  Interpreting this notation as an order relation, we often think in the following terms:
  \begin{itemize}
    \item $a\wedge b$ is the biggest element $c$ with $c \leq a$ and $c\leq b$.
    \item $a\vee b$ is the smallest element $c$ with $c \geq a$ and $c \geq b$. 
    \item $0$ is the smallest element. 
    \item $1$ is the biggest element. 
  \end{itemize}
\end{remark}
\begin{remark}
  For a family of expressions $x_i$ over a 
  finite set of indices $I = \{i_0, \cdots, i_n\}$, we will sometimes denote 
  \begin{itemize}
    \item $\bigvee_{i:I} x_i = x_{i_0} \vee \cdots \vee x_{i_n}$,
      with the understanding that this equals $0$ if $I$ is empty. 
    \item $\bigwedge_{i:I} x_i = x_{i_0} \wedge \cdots \wedge x_{i_n}$,
      with the understanding that this equals $1$ if $I$ is empty. 
  \end{itemize}
\end{remark}
\begin{remark}
  Boolean algebras are in bijective correspondence with Boolean rings, which 
  are rings with the property that $x\cdot x = x$ for all $x$. 
  The underlying set remains the same, and the translation is as follows:
  \begin{itemize}
    \item 
      To go from such a ring to a Boolean algebra, we let 
      $0 = 0, 1 =1 ,  a \vee b = a + b + ( a \cdot b), a\wedge b = a \cdot b, \neg a = a + 1$. 
    \item 
      and to go from a Boolean algebra to a Boolean ring, we let 
      $0 = 0, 1 = 1, a + b = (a \wedge \neg  b) \vee ( \neg a \wedge b), a \cdot b = a\wedge b , -a = a$. 
\end{itemize}
\end{remark}

%\begin{definition}
%  Let $\mathcal L$ be a language, 
%  let $\phi,\psi$ be terms in any language,
%  and suppose that for some index set $I$ we have 
%  for each $i:I$ two terms $a_i,b_i$ of same sort in $\mathcal L$. 
%  Let $E= (a_i = b_i)_{i:I}$. 
%
%  We say that $E$ allows us to rewrite $\phi$ to $\psi$ if there is a finite list of expressions 
%  $\phi = \phi_0 , \phi_1, \phi_2, \cdots, \phi_{n-1}, \phi_n = \psi$, 
%  such that each $\phi_{n+1}$ is given by taking $\phi_n$ and either changing one occurence of $a_i$ to  $b_i$, 
%  or one occurence of $b_i$ to $a_i$ for some $i:I$. 
%\end{definition}


\begin{definition}
  Given any set of symbols $G$, the free Boolean algebra on $G$, denoted $\langle G \rangle$
  is constructed as follows:
  \begin{itemize}
    \item We add $G$ as set of constants to the language of Boolean algebras. 
    \item We consider the set of terms in this language. 
    \item If there merely exists some finite sequence of 
      equalities from the theory of Boolean algebras allowing us to rewrite one term to another, 
      we consider the two terms to be equivalent. 
    \item The free Boolean algebra is then the set of equivalence classes under this relation. 
  \end{itemize}
\end{definition}
\begin{remark}
  The free Boolean algebra is a Boolean algebra. 
\end{remark}
\begin{definition}
  Let $B$ be a Boolean algebra, a subset $I\subseteq B$ is an ideal iff 
  $0\in I$ and whenver $x,y \in I$ and $b\in B$, we have $x\vee y \in I$ and $x \wedge b \in B$. 

  If $I$ is an ideal of $B$, then for $a,b:B$, we write $x\sim_I y$ iff 
  $a - b  = 
  (a \wedge \neg b ) \vee ( \neg a \wedge b) 
  \in I$. 
\end{definition}
\begin{lemma}
  Let $B$ be a Boolean algebra, and let $I\subseteq B$ be an ideal. 
  Then the relation $\sim_I$ is an equivalence relation. 
\end{lemma}
\begin{proof}
  \begin{itemize}
    \item To see $\sim_I$ is reflexive, note that 
      $(x \wedge \neg x) \vee (\neg x \wedge x) = 0 \vee 0 = 0 \in I$. 
    \item To see $\sim_I$ is symmetric, note that 
      $ (a \wedge \neg b) \vee ( \neg a \wedge b) = 
        (b \wedge \neg a) \vee ( \neg b \wedge a)$. 
    \item To see $\sim_I$ is transitive, that 
      whenever $a,b \in I$, so is $ a + b$. 
      Thus if $a -b , b -c \in I$, so is 
      $(a -b) + (b-c) = a -c$.
  \end{itemize}
\end{proof}
\begin{lemma}
  Whenever $B$ is a Boolean algebra and $I\subseteq B$ an ideal, 
  $B/I$ is a Boolean algebra. 
\end{lemma}
\begin{proof}
  We need to show that the Boolean operations respect equivalence classes. 
  \begin{itemize}
    \item 
      Suppose $a \sim_I b$. We shall show that $\neg a \sim_I \neg b$. 
      Thus we need to show that 
      $((\neg a) \wedge (\neg \neg b)) \vee ((\neg \neg a) \wedge \neg b) \in I$, 
      this term is equal to 
      $( a \wedge \neg b) \vee (b \wedge \neg a)$, which is in $I$ as $ a\sim_I b$. 
    \item 
      Suppose $ a\sim _I b $. We shall show that 
      $a \wedge c \sim_I b \wedge c$. 
      Note that by distributivity of $+$ and $\cdot$, we have 
      $(a \cdot c) -( b \cdot c) = (a -b ) \cdot c = (a-b) \wedge c$. 
      By assumption $a -b \in I$, and thus $(a-b) \wedge c \in I$ as well. 
    \item Note that $a \vee b = \neg ( \neg a \wedge \neg b)$, by the above two properties, 
      joins therefore respect equivalence relations. 
  \end{itemize}
\end{proof}
\begin{definition}
  Let $B$ be a Boolean algebra, and let $R\subseteq B$. 
  We denote $\langle R\rangle $ for the ideal generated by $R$, 
  which is the set of expressions 
  $\{(\bigvee_{r \in R_0} r) \wedge b| R_0\subseteq R \text{ finite}, b \in B\}$
\end{definition} 
\begin{remark}
  For $R\subseteq B$ as above, $ \langle R \rangle$ is an ideal.
\end{remark}

\begin{remark}\label{rmkMorphismsOutOfQuotient}
  Let $B$ be a Boolean algebra. 
  
  Suppose $G$ is a set of symbols, to define a morphism 
  $\langle G \rangle \to B$ it is sufficient to define 
  a function from $G$ to the underlying set of $B$. 

  Let $C$ be a Boolean algebra, and let $I\subseteq C$ be an ideal. 
  To define a map $C / I \to B$, it is sufficient to define a map $f:C\to B$
  with $f(i) = 0$ for all $i\in I$. 
  If $I = \langle R\rangle $, it is sufficient that $f(r) = 0$ for all $r\in R$. 
\end{remark}
%\begin{remark}
%  For every element $x$ of $\langle G\rangle /R$ as above,
%  there merely exists a finite subset of generators $G_0\subseteq G$ such that $x$ is expressable 
%  as a Boolean combination of symbols from $G_0$. 
%\end{remark}
\begin{definition}
  Let $B$ be a Boolean algebra. 
  If there exist $G, R$ as above such that 
  $B \simeq \langle G \rangle /\langle R \rangle$, 
  we call $G,R$ respectively generators and relations for $B$. 
  %
  If $G,R$ can taken to be finite, we call $B$ finitely presented. 
  If $G,R$ can taken to be countable, we call $B$ countably presented. 
\end{definition}
\begin{remark}
  There is a category of Boolean algebras, which has a subcategory of countably presented Boolean algebras, 
  which has a subcategory of finitely presented Boolean algebras. 
\end{remark}
\begin{remark}
  All categories above have pushouts.
\end{remark}
\begin{definition}
  Let $f:B\to C$ be a Boolean morphism. 
  The set $\{x \in B| fx = 0\}$ is called the kernel of $f$ and denoted 
  $Ker(f)$. 
\end{definition}
\begin{remark}
  For any map of Boolean algebras $f$, the kernel of $f$ is an ideal. 
  Furthermore, we have that $a\sim_{Ker(f)} b$ iff $f(a) = f(b)$. 
\end{remark}
\begin{remark}
  Let $f:B \to C$ be a Boolean morphism. 
  The epi-mono factorization of $f$ exists and is given by  
  $B \twoheadrightarrow B / Ker(f) \hookrightarrow C$.
\end{remark}

\subsection{Countably presented as colimit of finitely presented}
%SeeMain%%%% \begin{definition}
%SeeMain%%%%   A sequence is a diagram of shape $\mathbb N_{\leq}$.
%SeeMain%%%% \end{definition} 
%SeeMain%%
%SeeMain%%x%\begin{lemma}
%SeeMain%%x%  $B$ is a countably presented Boolean algebra iff 
%SeeMain%%x%  it merely is the colimit of a sequence of finitely presented Boolean algebras.% $(B_n)_{n\in\mathbb N}$.
%SeeMain%%x%\end{lemma} 
%SeeMain%%x%%%Alternatively, we can define a countably presented Boolean algebra as an inductive limit of finite Boolean algebras
%SeeMain%%x%%%that are given by a sequence of finite sets and maps between then.
%SeeMain%%x%\begin{proof}
%SeeMain%%x%  First, assume a sequence of finitely presented Boolean algebras. 
%SeeMain%%x%  We need to show that the colimit is a countably presented Boolean algebra. 
%SeeMain%%x%  \begin{itemize}
%SeeMain%%x%    \item The set of generators for the colimit is the colimit of the sets of generators. 
%SeeMain%%x%    \item The set of relations for the colimit is the union of the sets of relations. 
%SeeMain%%x%      After all, any expression $f$ that becomes $0$ somewhere in the sequence will will be coprojected to $0$
%SeeMain%%x%      in the colimit. And as any equality that holds in the colimit uses finitely many elements, 
%SeeMain%%x%      it must already hold somewhere in the sequence. 
%SeeMain%%x%  \end{itemize}
%SeeMain%%x%  Note that both colimits over countably many finite sets are countable. 
%SeeMain%%x%  Hence the colimit is countably represented. 
%SeeMain%\begin{proof}
%SeeMain%  Consider $\langle G \rangle /R$ a countable presentation of a Boolean algebra $B$. 
%SeeMain%  We will show there exists a diagram of shape $\mathbb N$ taking values in Boolean algebras 
%SeeMain%  with $\langle G\rangle / R$ as the colimit.
%SeeMain%  \paragraph{The diagram}
%SeeMain%  Now let $R_n$ be the first $n$ terms in $R$. 
%SeeMain%  Note that each of these finitely many terms uses only finitely many symbols from $G$.
%SeeMain%  Let $G_n$ be the finite set of terms used in $R_n$, unioned with the finite set of the first $n$ elements of $G$. 
%SeeMain%  Define for each $n\in\mathbb N$ the finitely presented Boolean algebra $B_n = G_n /R_n$. 
%SeeMain%  If $n\leq m$, then \Cref{rmkMorphismsOutOfQuotient} gives us a map $B_n \to B_m$ 
%SeeMain%  as $G_n \subseteq G_{n+1}$ and $R_n \subseteq R_{n+1}$. 
%SeeMain%  Thus $(B_n)_{n\in \mathbb N}$ gives us a diagram of shape $\mathbb N$
%SeeMain%  with values in finitely presented algebras. 
%SeeMain%
%SeeMain%  \paragraph{The colimit}
%SeeMain%  As $G_n\subseteq G$ and $R_n \subseteq R$, 
%SeeMain%  \Cref{rmkMorphismsOutOfQuotient} also gives us a map $B_n\to \langle G \rangle /R$. 
%SeeMain%  We claim the resulting cocone is a colimit. 
%SeeMain%
%SeeMain%  Suppose we have a cocone $C$ on the diagram $(B_n)_{n\in\mathbb N}$. 
%SeeMain%  We need to show that there exists a map $\langle G \rangle / R\to C$ and
%SeeMain%  we need to show this map is unique as map between cocones. 
%SeeMain%  \begin{itemize}
%SeeMain%    \item To show there exists a map $\langle G \rangle / R \to C$, 
%SeeMain%      we use remark \Cref{rmkMorphismsOutOfQuotient} again. 
%SeeMain%      Let $g\in G$ be the $n$'th element of $G$, 
%SeeMain%      note that $g\in G_n$, and consider the image of $g$ under the map $B_n \to C$. 
%SeeMain%      This procedure defines a function from $G$ to the underlying set of $C$. 
%SeeMain%      Let $\phi \in R$ be the $n$'th element of $R$, 
%SeeMain%      note that $\phi \in R_n$, and the map $B_n \to C$ must send $\phi$ to $0$. 
%SeeMain%      Thus the function from $G$ to the underlying set of $C$ also sends $\phi$ to $0$. 
%SeeMain%      This thus defines a map $\langle G \rangle / R \to C$. 
%SeeMain%    \item To show uniqueness, consider that any map of cocones must take the same values 
%SeeMain%      on all $g\in G_n$ for all $n\in\mathbb N$, but by \Cref{rmkMorphismsOutOfQuotient}
%SeeMain%      this uniquely defines a map. 
%SeeMain%  \end{itemize}
%SeeMain%%  By \Cref{rmkMorphismsOutOfQuotient} a map $\langle G \rangle /R \to C$.
%SeeMain%%  By \Cref{rmkMorphismsOutOfQuotient} such maps are uniquely determined by their values on $G$. 
%SeeMain%%  to the underlying set of $C$. 
%SeeMain%%  Let $g\in G$ be the $n$'th element of $G$. 
%SeeMain%%  We 
%SeeMain%%
%SeeMain%%
%SeeMain%%
%SeeMain%%
%SeeMain%%  We let $G_n$ be given by the first $n$ generators. 
%SeeMain%%  Let $R_n$ be the relations involving these generators, 
%SeeMain%%  of which there are only finitely many. 
%SeeMain%%  We define $B_n = G_n/R_n$, which is a finitely presented Boolean algebra. 
%SeeMain%%  The embedding of the first $n$ generators into the first $m$ generators gives us 
%SeeMain%%  a map $B_n \to B_m$ whenever $n\leq m$. 
%SeeMain%%  Because these morphisms are compatible, this defines a sequence of Boolean algebras. 
%SeeMain%%  We claim the colimit of this sequence is $B$. 
%SeeMain%%
%SeeMain%%  Any element in $B$ can be expressed as Boolean combination of finitely many generators, 
%SeeMain%%  which must occur in some $B_n$, and thus in the colimit. 
%SeeMain%%  Whenever the images of two elements in the colimit are equal, they are already equal in some $B_m$, 
%SeeMain%%  hence it follows from a finite subset of the relations for $B$ that the elements are equal, 
%SeeMain%%  hence the elements are equal in $B$. Thus we have an embedding from $B$ into the colimit. 
%SeeMain%%
%SeeMain%%  Any element in the colimit already appears in some $B_n$, and hence is 
%SeeMain%%  a finite expression using generators from $B$, thus occurs in $B$ is as well. 
%SeeMain%%  Suppose two elements in the colimit correspond in this manner to the same element in $B$. 
%SeeMain%%  Then their equality follows from the relations of $B$. 
%SeeMain%%  By compactness in the meta-theory, their equality must follow from a finite subset of the relations from $B$, 
%SeeMain%%  hence there is some $B_m$ where both elements are equal, and they are equal in the colimit as well. 
%SeeMain%%  Thus the colimit embeds into $B$. 
%SeeMain%%
%SeeMain%%  We conclude that $B$ and the colimit are isomorphic Boolean algebras. 
%SeeMain%\end{proof} 
%SeeMain%
%SeeMain%
%SeeMain%
%SeeMain%
%SeeMain%
%SeeMain%



\begin{definition}
  We call an object $K$ (countably) compact if for every sequence 
  $(B_n)_{n:\mathbb N}$ with colimit $B$, we have
  that the set $B^K$ is the colimit of the sequence of sets $(B_n^K)_{n:\mathbb N}$.
\end{definition}

\begin{lemma}
  All finitely presented Boolean algebras are compact in the category of Boolean algebras. 
\end{lemma}  
\begin{proof}
  Let $K$ be a finitely presented Boolean algebra, 
  and let $(B_n)_{n:\mathbb N}$ be any sequence of Boolean algebras. 


\end{proof}
%% \begin{proof}
%%   Note that a map $K \to A$ is uniquely determined by it's values on the generators
%%   of which there are finitely many if $K$ is finitely presented. 
%%   Any finite subset of $A$ must at some point occur in some $A_n$, 
%%   thus we can assign the generators of $K$ a value in $A_n$. 
%%   This induces a map $K \to A_n$. 
%% \end{proof}

 The following uses Dependent Choice.

\begin{lemma}
  If $A \to B$ is injective between countably presented Boolean algebras, 
  we can write it as colimit of injections between finitely presented Boolean algebras. 
\end{lemma}


\subsection{Examples of countably presented Boolean algebras}
\begin{example}
  $2$ is the Boolean algebra given by the empty set of generators and no relations. 
    It's underlying set is $\{0,1\}$. 
\end{example}
\begin{example}
  The trivial Boolean algebra given by the empty set of generators and the relation $\{1\}$, 
  It's underlying set contains only one element and we have $0=1$ in the trivial Boolean algebra. 
\end{example}

\begin{example}\label{ExampleBAunderCantor}
  $C = \langle N \rangle $ is the Boolean algebra given by $\mathbb N$ as set of generators and no relations. 
\end{example}
\begin{example}\label{ExampleBAunderNinfty}
  Denote $\mathbb N_{(co)fin}$ for the set of subsets of $\mathbb N$ which are finite or co-finite. 
  Under the interpretation of $\wedge = \cap , \vee = \cup, 0 = \emptyset, 1 = \mathbb N$, and $\neg$ 
  as the set-theoretic complemented (denoted $(\cdot)^C$). 

  These operations are well-defined on $\mathbb N_{(co)fin}$ 
  and they give the structure of a Boolean algebra.
  We claim it is countably presented. 

  Let $G = \{p_n| n:\mathbb N\} $ be a countably infinite set of generator symbols and let 
  $R = \{ p_n \wedge p_m | n\neq m :\mathbb N \}$. 
  We claim that $\langle G \rangle / R \simeq \mathbb N_{(co)fin}$. 

\begin{proof}
  We let $f:\langle G \rangle / R \to \mathbb N_{(co)fin}$, be the unique morphism of Boolean algebras 
  satisfying $f(p_n) = \{n\}$ for all $n:\mathbb N$. As $\{n\} \cap \{m\} = \emptyset$ whenever $n\neq m$, 
  \Cref{rmkMorphismsOutOfQuotient} indeed tells us this is sufficient to define $f$. 

  Now we define $g:\mathbb N_{(co)fin} \to \langle G \rangle / R$. 
  On a finite subset $I$, we define $g(I) = \bigvee_{i\in I} p_i$, 
  and on a cofinite subset $J$, we define $g(J) = \bigwedge _{i \in J^C} \neg p_i$. 
  Note that in these cases we indeed have $I,J^C$ are finite, so these are well-defined elements. 

  We need to show that $g$ is a Boolean morphism. 
  \begin{itemize}
    \item 
      By deMorgan's laws, $g$ preserves $\neg$:
      for $I$ finite we have
      \begin{equation}
      \neg g(I) = \neg (\bigvee_{i\in I} p_i) = \bigwedge_{i\in I} \neg p_i = g(I^C)
      \end{equation}
      And for $J$ cofinite, we apply similar reasoning. 
    \item To see that $g$ preserves $\vee$, we need to check three cases
      \begin{itemize}
        \item If both $I,J$ are finite, then 
        \begin{equation} 
          g(I \cup J) = \bigvee_{i\in I \cup J} p_i= \bigvee_{i\in I} p_i \vee \bigvee_{j\in J} p_j 
          = g(I) \vee g(J)
        \end{equation}
        and we're done. 
      \item If both $I,J$ are cofinite, we have
        \begin{equation}
          g(I) \vee g(J) = 
          \bigwedge_{i \in I^C} \neg p_i \vee 
          \bigwedge_{j \in J^C} \neg p_j 
          = 
          \bigwedge_{i\in I^C} 
          \bigwedge_{j \in J^C}(\neg p_i \vee  \neg p_j) 
        \end{equation}
        Now note that in $\langle G \rangle / R$, we have 
        \begin{equation}
          \neg p_i \vee \neg p_j = \neg ( p_i \wedge p_j) = 
          \begin{cases}
            p_i \text{ if } i = j\\
            1 \text{ if } i \neq j  
          \end{cases}
        \end{equation}
        Therefore, we can leave out the case that $i\neq j$ in the calculation of the above meet, and
        \begin{equation}
          \bigwedge_{i\in I^C} 
          \bigwedge_{j \in J^C}(\neg p_i \vee  \neg p_j)  
          = 
          \bigwedge_{i \in (I^C \cap J^C)} \neg p_i
          = 
          \bigwedge_{i \in (I \cup J)^C} \neg p_i 
        \end{equation}
        as $I\cup J$ must also be cofinite, this equals 
          $ g( I \cup J)$. 
        \item 
          If $I$ is finite and $J$ cofinite, we have 
          that $I\cup J$ is cofinite, hence 
          \begin{equation}
            g(I\cup J) = \bigwedge_{k\in (I \cup J)^C} \neg p_k
            = \bigwedge_{k \in (J^C -I)} \neg p_k
          \end{equation}
          Now note that 
          whenever $i\neq k$, we have 
          \begin{equation}
            p_i = (p_i \wedge \neg p_k) \vee (p_i \wedge p_k) = 
            (p_i \wedge \neg p_k) \vee 0 = p_i \wedge \neg p_k
          \end{equation}
          Hence by absorption
          \begin{equation} 
            (p_i \vee \neg p_k)  =
              \begin{cases}
                1 \text{ if } i = k \\
                \neg p_k \text{ if } i \neq k
              \end{cases}
          \end{equation}
          As for all $k\in J^C-I$ and all $i\in I$ we have $k\neq i$, we may thus write
          \begin{equation}
            \bigwedge_{k \in (J^C - I)} \neg p_k = 
            \bigwedge_{k \in (J^C - I)} (\neg p_k \vee (\bigvee_{i\in I} p_i))
          \end{equation}
          But now we can use that adding $1$ in a meet does not change the meet, and see that 
          \begin{equation}
            \bigwedge_{k \in (J^C - I)} (\neg p_k \vee (\bigvee_{i\in I} p_i))
            = 
            \bigwedge_{j \in J^C} (\neg p_j \vee (\bigvee_{i\in I} p_i))
          \end{equation}
          And using distributivity rules, we can see that 
          \begin{equation}
            \bigwedge_{j \in J^C} (\neg p_j \vee (\bigvee_{i\in I} p_i))
            = 
            (\bigwedge_{j \in J^C} \neg p_k) \vee (\bigvee_{i\in I} p_i)
          \end{equation}
          From which we may conclude that $g(I\cup J) = g(I) \cup g(J)$. 
      \end{itemize}
    \item The case for $\wedge$ is completely dual to the case for $\vee$. 
  \end{itemize}
We conclude that $g$ is a Boolean morphism. 
Furthermore, $g$ and $f$ are each other's inverse, thus the Boolean algebras are isomorphic. 
\end{proof}
\end{example}
\section{Stone spaces}
\newcommand{\isSt}{\mathsf{isStone}}
\subsection{Definitions}
\begin{definition}
  For $B$ a countably presented Boolean algebra, we define $Sp(B)$ as the set of Boolean morphisms from $B$ to $2$. 
\end{definition}
\begin{definition}
  We define the predicate on types $\isSt$ by 
  \begin{equation}
    \isSt(X) := \sum\limits_{B : Boole} X = Sp(B)
  \end{equation} 
  A type $X$ is called \textit{Stone} if $\isSt(X)$ is inhabited.
\end{definition}
\subsection{Examples of Stone types}
\begin{example}\label{ExampleBAunderEmpty}
  The dual to the trivial Boolean algebra from $\Cref{ExampleBAunderEmpty}$ is the empty set $\emptyset$, 
  as $0\neq 1$ in $2$, but $0=1$ in the trivial Boolean algebra, there can be no functions preserving both $0$ and $1$ 
  from the trivial Boolean algebra to $2$. 
\end{example}
\begin{example}
  The dual to $C$ from \Cref{ExampleBAunderCantor} is called Cantor space 
  and denoted $2^\mathbb N$. 
  By \Cref{rmkMorphismsOutOfQuotient}, terms of $2^\mathbb N$ 
  correspond to set-theoretic functions $\mathbb N \to 2$. 
  We also call such functions binary sequences. 
\end{example}
\begin{example}
  The dual to $\mathbb N_{(co)fin}$ from \Cref{ExampleBAunderNinfty} is called 
  $\mathbb N_\infty$. By \Cref{rmkMorphismsOutOfQuotient}, terms of $\mathbb N_\infty$ 
  correspond to functions $\alpha: \mathbb N \to 2$ such that $\alpha(n) \wedge \alpha(m) = 0$ 
  whenever $n \neq m$. This means that $\alpha(n) = 1$ for at most one $n\in\mathbb N$. 
  There is an embedding $\mathbb N \to \mathbb N_\infty$ sending $n$ to the unique sequence $\chi_n$
  which sends $n$ to $1$. 
  There is furthermore a term $\infty:\mathbb N_\infty$ which is the sequence which is constantly $0$. 
\end{example}
\subsection{Axioms}
\begin{axiomNum}[Stone duality]
  For any countably presented Boolean algebra $B$, the evaluation map $B\rightarrow  2^{Sp(B)}$ is an isomorphism.
\end{axiomNum} 

\begin{remark}
  \rednote{Check references in other versions of HoTT, draft}
Stone types will take over the role of affine scheme from \cite{draft}, 
and we repeat some results here. 
Analogously to Lemma 3.1.2 of \cite{draft}, 
for $X$ Stone, Stone duality tells us that $X = Sp(2^X)$. 
%
Proposition 2.2.1 of \cite{draft} now says that 
$Sp$ gives an equivalence 
\begin{equation}
   Hom_{\Boole} (A, B) = (Sp(B) \to Sp(A))
\end{equation}
Therefore $\isSt$ is a proposition.
Equivalently, 
%By Definition 4.6.1 %defines an embedding 
%of \cite{hott}, 
this means that 
$Sp$ is an embedding from $\Boole$ to any universe of types.
Its image, $\Stone$ also has a natural category structure.
By the above and Lemma 9.4.5 of \cite{hott}, 
the map $Sp$ defines a dual equivalence of categories between $\Boole$ and $\Stone$.
\end{remark}

\begin{axiomNum}[Surjections are Formal Surjections]
  A map $f:Sp(B')\to Sp(B)$ is surjective iff the corresponding map $B \to B'$ is injective.
\end{axiomNum} 

\begin{axiomNum}[Local choice]
  Whenever $X$ Stone and $E\twoheadrightarrow X$ surjective, then there is some $Y$ Stone,
    a surjection $Y \twoheadrightarrow X$ and a map $Y\to E$ such that the following diagram commutes:
    \begin{equation}\begin{tikzcd}
      E \arrow[d,""',two heads]\\
      X & \arrow[l, "", two heads] Y\arrow[lu, ""']
    \end{tikzcd}\end{equation}  
\end{axiomNum} 
\begin{axiomNum}[Dependent choice]\label{axDependentChoice}
  Given a family of types $(E_n)_{n:\mathbb N}$ and 
  a relation 
  $R_n:E_n\rightarrow E_{n+1}\rightarrow {\mathcal U}$ such that
  for all $n$ and $x:E_n$ there exists $y:E_{n+1}$ with $p:R_n~x~y$ 
  then given $x_0:E_0$ there exists
  $u:\Pi_{n:\N}E_n$ and $v:\Pi_{n:\N}R_n~(u~n)~(u~(n+1))$ and $u~0 = x_0$.
\end{axiomNum}
\subsection{Omniscience principles}
\rednote{active}
\begin{remark}\label{rmkTrivialBA}
  A Boolean algebra $B / R$ 
  is trivial iff there merely exists some $R_0\subseteq R$ finite with 
  $1 = \bigvee_{r \in R_0} r$ in $B$.
\end{remark}

\begin{theorem}[The negation of the weak limited principle of omniscience]
  It is not the case that for all $\alpha:\Noo$, we can decide whether $\alpha=\infty$.
\end{theorem}
\begin{proof}
  Suppose we 
\end{proof}

\begin{theorem}[The lesser limited principle of omniscience]
\end{theorem}
The following result is due to David W\"arn.
\begin{theorem}[Markov's principle (uses Stone duality)]
  For $\alpha:\mathbb N_\infty$, we have that whenever $\alpha\neq \infty$, 
  there exists some $n:\mathbb N$ such that $\alpha = \chi_n$. 
\end{theorem}
\begin{proof}
  Suppose $\alpha \neq \infty$.%, then it is not the case that $\alpha(n) = 0$ for all $n:\mathbb N$. 
  We will show that $2/\{\alpha(n)|n\in\mathbb N\}$ is the trivial Boolean algebra. 
  Then by \Cref{rmkTrivialBA}, it follows there merely is a finite subset $N_0\subseteq \mathbb N$ 
  such that $\bigvee_{i:N_0}\alpha(i) =1$.
  For this finite set $N_0$, we can decide whether every $\alpha(i) =0$, 
  or there is some $i\in N_0$ such that $\alpha(i) =1$.
  As the join of these $\alpha(i)$ is $1$, we must have that there is some $n\in N_0$ with $\alpha(n) = 1$. 
  As $\alpha(j)=1$ for at most one $j$, we conclude that $\alpha = \chi_n$. 
  %

  To show that $2/\{\alpha(n)|n\in\mathbb N\}$ is trivial, we will show it has an empty spectrum. 
  Suppose $x: 2 \to 2$ is such that $x(\alpha(n)) = 0$ for every $n:\mathbb N$. 
  as $x(1) = 1\neq 0$, we must have for every $n:\mathbb N$ that $\alpha(n) \neq 1$. 
  Thus for every $n:\mathbb N$, we have $\alpha(n) = 0$, contradicting our assumption that $\alpha \neq \infty$. 
  Thus any such $x$ cannot exist and $Sp(2/\{\alpha(n)|n\in\mathbb N\}) = \emptyset$. 
  Thus $2/\{\alpha(n) | n \in \Noo \} = 2^\emptyset$, which is the trivial Boolean algebra. 
\end{proof}

\subsection{Topology}
\begin{definition}
  Let $P$ be a proposition. 
  \begin{itemize}
    \item $P$ is decidable if $P + \neg P$
    \item $P$ is open if there merely exists some $\alpha:\Noo$ such that $P \leftrightarrow \alpha \neq \infty$. 
    \item $P$ is closed if there merely exists some $\alpha:\Noo$ such that $P \leftrightarrow \alpha = \infty$. 
  \end{itemize}
\end{definition}
\begin{definition}
  For $S$ a set and $D\subseteq S$ an arbitrary subset, we call $D$ decidable/open/closed 
  iff for every $x:S$, $D(x)$ is decidable/open/closed. 
\end{definition}

\begin{lemma}
  Let $D\subseteq Sp(B)$. TFAE
  \begin{enumerate}[(i)]
    \item $D$ is decidable. 
    \item There merely exists some $b\in B$ such that $D = \{\alpha:Sp(B) | \alpha(b) = 1\}$. 
    \item There merely is a finite subset $G_0$ of the generators of $B$ such that 
      whenever $\alpha,\beta:Sp(B)$ are such that 
      $\alpha(g) = \beta(g)$ for all $g:G_0$ and $D(\alpha)$, we also have $D(\beta)$. 
    \item $D$ is both open and closed. 
  \end{enumerate}
\end{lemma}
\begin{proof}
  \begin{itemize}
    \item $(i) \to (ii)$.
      First, remark that decidable subsets $D$ give terms $f_D:2^{Sp(B)}$, by sending 
      $f(\alpha) = 1$ if $D(\alpha)$ and $f(\alpha) = 0$ if $\neg D(\alpha)$. 
      Stone duality then gives that this term corresponds to evaluation at some $b:B$. 
      Therefore $D(\alpha) \iff f(\alpha) = 1 \iff  \alpha(b) = 1$, as required. 
    \item $(ii) \to (1)$
      To decide $D(\alpha)$ is the same as to evaluate whether $\alpha(b) = 1$, which is decidable. 
    \item $(ii) \to (iii)$. If there exists a $b:B$ such that $D$ is the indicator of $b$ in the above way, 
      then we need only consider the generators occuring in a finite expression of $b$. 
    \item $(iii) \to (i)$. Because being decidable is a proposition, we can untruncate our assumption. 
      Suppose $D$ only depends on $n$ generators, then we need to only check the values 
  \end{itemize}
\end{proof}
\rednote{End of activity}

\subsection{Compact Hausdorff spaces}
\subsection{Analysis}


%\chapter{Axioms}
%\chapter{Stone Spaces}
%\chapter{Open and closed propositions}
%



\appendix
\rednote{The following is not put in order}
\section*{Introduction}

Algebraic geometry is the study of solutions of non-linear equations using methods from geometry.
Most prominently, algebraic geometry was essential in the proof of Fermat's last theorem by Wiles and Taylor.
The central geometric objects in algebraic geometry are called \emph{schemes}.
Their basic building blocks are called \emph{affine schemes},
where, informally, an affine scheme corresponds to a solution sets of polynomial equations.
While this correspondence is clearly visible in the functorial approach to algebraic geometry and our synthetic approach,
it is somewhat obfuscated in the most commonly used, topological appraoch.

In recent years,
formalization of the intricate notion of affine schemes
received some attention as a benchmark problem
-- this is, however, \emph{not} a problem addressed by this work.
Instead, we use a synthetic approach to algebraic geometry,
very much alike to that of synthetic differential geometry.
This means, while a scheme in classical algebraic geometry is a complicated compound datum,
we work in a setting where schemes are types,
with an additional property that can be defined within our synthetic theory.

Following ideas of Ingo Blechschmidt and Anders Kock  (\cite{ingo-thesis}, \cite{kock-sdg}[I.12]),
we use a base ring $R$, which is local and satisfies an axiom reminiscent of the Kock-Lawvere axiom.
A more general axiom, is called \emph{synthetic quasi coherence (SQC)} by Blechschmidt and
a version quatifying over external algbras is called the \emph{comprehensive axiom}\footnote{
  In \cite{kock-sdg}[I.12], Kock's ``axiom $2_k$'' could equivalently be Theorem 12.2,
  which is exactly our synthetic quasi coherence axiom, except that it only quantifies over external algebras.
}
by Kock.
The exact concise form of the SQC axiom we use, was noted by David Jaz Myers in 2018 and communicated to the first author.

Before we state the SQC axiom, let us take a step back and look at the basic objects of study in algebraic geometry,
solutions of polynomial equations.
Given a system of polynomial equations
\begin{align*}
  p_1(X_1, \dots, X_n) &= 0\rlap{,} \\
  \vdots\quad\quad\;\;   \\
  p_m(X_1, \dots, X_n) &= 0\rlap{,}
\end{align*}
the solution set
$\{ x : R^n \mid \forall i.\; p_i(x_1, \dots, x_n) = 0 \}$
is in canonical bijection to the set of $R$-algebra homomorphisms
\[ \Hom_R(R[X_1, \dots, X_n]/(p_1, \dots, p_m), R) \]
by identifying a solution $(x_1,\dots,x_n)$ with the homomorphism that maps each $X_i$ to $x_i$.
Conversely, for any $R$-algebra $A$, which is merely of the form $R[X_1, \dots, X_n]/(p_1, \dots, p_m)$,
we define the \emph{spectrum} of $A$ to be
\[
  \Spec A \colonequiv \Hom_R(A, R)
  \rlap{.}
\]
In contrast to classical, non-synthetic algebraic geometry,
where this set needs to be equipped with additional structure,
we postulate axioms that will ensure that $\Spec A$ has the expected geometric properties.
Namely, SQC is the statement that, for all finitely presented $R$-algebras $A$, the canonical map
  \begin{align*}
    A&\xrightarrow{\sim} (\Spec A\to R) \\
    a&\mapsto (\varphi\mapsto \varphi(a))
  \end{align*}
is an equivalence.
A prime example of a spectrum is $\A^1\colonequiv \Spec R[X]$,
which turns out to be the underlying set of $R$.
With the SQC axiom,
\emph{any} function $f:\A^1\to \A^1$ is given as a polynomial with coefficients in $R$.
In fact, all functions between affine schemes are given by polynomials.
Furthermore, for any affine scheme $\Spec A$,
the axiom ensures that
the algebra $A$ can be reconstructed as the algebra of functions $\Spec A \to R$,
therefore establishing a duality between affine schemes and algebras.

The Kock-Lawvere axiom used in synthetic differential geometry,
might be stated as the SQC axiom restricted to (external) \emph{Weil-algebras},
whose spectra correspond to pointed infinitesimal spaces.
These spaces can be used in both, synthetic differential and algebraic geometry,
in very much the same way.

In the accompanying formalization \cite{formalization} of some basic results,
we use a setup which was already proposed by David Jaz Myers
in a conference talk (\cite{myers-talk1, myers-talk2}).
On top of Myers' ideas,
we were able to define schemes, develop some topological properties of schemes,
and construct projective space.

An important, not yet formalized result
is the construction of cohomology groups.
This is where the \emph{homotopy} type theory really comes to bear --
instead of the hopeless adaption of classical, non-constructive definitions of cohomology,
we make use of higher types,
for example the $k$-th Eilenberg-MacLane space $K(R,k)$ of the group $(R,+)$.
As an analogue of classical cohomology with values in the structure sheaf,
we then define cohomology with coefficients in the base ring as:
\[
  H^k(X,R):\equiv \|X\to K(R,k)\|_0
  \rlap{.}
\]
This definition is very convenient for proving abstract properties of cohomology.
For concrete calculations we make use of another axiom,
which we call \emph{Zariski-local choice}.
While this axiom was conceived of for exactly these kind of calculations,
it turned out to settle numerous questions with no apparent connection to cohomology.
One example is the equivalence of two notions of \emph{open subspace}.
A pointwise definition of openness was suggested to us by Ingo Blechschmidt and
is very convenient to work with.
However, classically, basic open subsets of an affine scheme are given
by functions on the scheme and the corresponding open is morally the collection of points where the function does not vanish.
With Zariski-local choice, we were able to show that these notions of openness agree in our setup.

Apart from SQC, locality of the base ring $R$ and Zariski-local choice,
we only use homotopy type theory, including univalent universes, truncations and some very basic higher inductive types.
Roughly, Zariski-local choice states, that any surjection into an affine scheme merely has sections on a \emph{Zariski}-cover.
The latter, internal, notion of cover corresponds quite directly to the covers in the site of the \emph{Zariski topos},
which we use to construct a model of homotopy type theory with our axioms.

More precisely, we can use the \emph{Zariski topos} over any base ring.
Toposes built using other Grothendieck topologies, like for example the étale topology, are not compatible with Zariski-local choice.
We did not explore whether an analogous setup can be used for derived algebraic geometry
\footnote{Here, the word ``derived'' refers to the rings the algebraic geometry is built up from -- instead of the 0-truncated rings we use, ``derived'' algebraic geometry would use simplicial or spectral rings.
  Sometimes, ``derived'' refers to homotopy types appearing in ``the other direction'', namely as the values of the sheaves that used.
  In that direction, our theory is already derived, since we use homotopy type theory.
  Practically that means that we expect no problems when expanding our theory of synthetic schemes to what classic algebraic geometers
  call ``stacks''.
}
-- meaning that the 0-truncated rings we used are replaced by higher rings.
This is only because for a derived approach, would have to work with higher monoids, which is currently infeasible
-- we are not aware of any obstructions for, say, an SQC axiom holding in derived algebraic geometry.

In total, the scope of our theory so far, includes quasi-compact, quasi-separated schemes of finite type over an arbitrary ring.
These are all finiteness assumptions, that were chosen for convenience and include examples like closed subspaces of projective space,
which we want to study in future work, as example applications.
So far, we know that basic internal constructions, like affine schemes, correspond to the correct classical external constructions.
This can be expanded using our model, which is of course also important to ensure the consistency of our setup.

%

\section{Omniscience principles}
\begin{lemma}\label{LemDecidableSubsetsDeMorgan}
  For $(A_n)$ a family of decidable subsets, we have
    $
    (\bigcup_{n:\mathbb N} A_n)^C
    =
    \bigcap_{n:\mathbb N} (A_n^C)
    $ 
    and 
    $
    \bigcup_{n:\mathbb N} (A_n^C)
    =  
    (\bigcap_{n:\mathbb N} A_n)^C
    $
\end{lemma}

\begin{proof}
\begin{itemize}
  \item 
    Let 
    $x\notin \bigcup_{n:\mathbb N} A_n$. 
    Then for every $n:\mathbb N$, we cannot have $x\in A_n$
    and thus $x\in A_n^C$ by decidability of $A_n$. 
    Thus $x\in \bigcap_{n:\mathbb N} (A_n^C)$. 
    Therefore
    $$
    (\bigcup_{n:\mathbb N} A_n)^C
    \subseteq 
    \bigcap_{n:\mathbb N} (A_n^C).
    $$ 
  \item 
    Suppose that for every $n:\mathbb N$, we have $x\notin A_n$. 
    There does not exist an $n:\mathbb N$ with $x\in A_n$. Thus
    $$
    \bigcap_{n:\mathbb N} (A_n^C)
    \subseteq 
    (\bigcup_{n:\mathbb N} A_n)^C
    $$ 
  \item 
    Suppose there exists some $n$ with $x\in A_n^C$. Then 
    it cannot be the case that $x\in A_m$ for all $m:\mathbb N$.
    Thus
    $$
    \bigcup_{n:\mathbb N} (A_n^C)
    \subseteq 
    (\bigcap_{n:\mathbb N} A_n)^C
    $$ 
  \item 
    Suppose that $x\in (\bigcap_{n:\mathbb N} A_n)^C$. 
    Then define the binary sequence $\alpha$ by $\alpha(i) =1$ iff $i$ is the first index such that 
    $x\notin A_i$. This is well-defined as $A_n$ is decidable for all $n:\mathbb N$. 
    If $\alpha(i) = 0$ for all $i$, then $x\in A_i$ for all $i$. 
%    and it is the case that $x\in A_n$ for all $n:\mathbb N$. 
    Thus under our assumption $x\in (\bigcap_{n:\mathbb N} A_n)^C$, 
    we cannot have that $\alpha(i) = 0$ always. 
    By Markov, there then exists an $i$ such that $\alpha(i) = 1$. 
    Thus $x\notin A_i$ for some $i$. We conclude that. 
    $$
    (\bigcap_{n:\mathbb N} A_n)^C
    \subseteq
    \bigcup_{n:\mathbb N} (A_n^C)
    $$ 
\end{itemize}
\end{proof} 
Note that we only needed decidability for the first and last bullet point, 
and only the last bullet point used countability (and of course Markov's principle). 


\section{Topology}

\subsection{Closed subtypes}

\begin{definition}%
  \label{closed-proposition}\label{closed-subtype}
  \begin{enumerate}[(a)]
  \item
    A \notion{closed proposition} is a proposition
    which is merely of the form $x_1 = 0 \land \dots \land x_n = 0$
    for some elements $x_1, \dots, x_n \in R$.
  \item
    Let $X$ be a type.
    A subtype $U : X \to \Prop$ is \notion{closed}
    if for all $x : X$, the proposition $U(x)$ is closed.
  \item
    For $A$ a finitely presented $R$-algebra
    and $f_1, \dots, f_n : A$,
    we set
    $V(f_1, \dots, f_n) \colonequiv
    \{\, x : \Spec A \mid f_1(x) = \dots = f_n(x) = 0 \,\}$.
  \end{enumerate}
\end{definition}

Note that $V(f_1, \dots, f_n) \subseteq \Spec A$ is a closed subtype
and we have $V(f_1, \dots, f_n) = \Spec (A/(f_1, \dots, f_n))$.

\begin{proposition}[using \axiomref{sqc}]%
  There is an order-reversing isomorphism of partial orders
  \begin{align*}
    \text{f.g.-ideals}(R) &\xrightarrow{{\sim}} \Omega_{cl} \\
    I &\mapsto (I = (0))
  \end{align*}
  between the partial order of finitely generated ideals of $R$
  and the partial order of closed propositions.
\end{proposition}

\begin{proof}
  For a finitely generated ideal $I = (x_1, \dots, x_n)$,
  the proposition $I = (0)$ is indeed a closed proposition,
  since it is equivalent to $x_1 = 0 \land \dots \land x_n = 0$.
  It is also evident that we get all closed propositions in this way.
  What remains to show is that
  \[ I = (0) \Rightarrow J = (0)
     \qquad\text{iff}\qquad
     J \subseteq I
     \rlap{\text{.}}
  \]
  For this we use synthetic quasicoherence.
  Note that the set $\Spec R/I = \Hom_R(R/I, R)$ is a proposition
  (has at most one element),
  namely it is equivalent to the proposition $I = (0)$.
  Similarly, $\Hom_R(R/J, R/I)$ is a proposition
  and equivalent to $J \subseteq I$.
  But then our claim is just the equation
  \[ \Hom(\Spec R/I, \Spec R/J) = \Hom_R(R/J, R/I) \]
  which holds by \Cref{spec-embedding},
  since $R/I$ and $R/J$ are finitely presented $R$-algebras
  if $I$ and $J$ are finitely generated ideals.
\end{proof}

\begin{lemma}[using \axiomref{sqc}]%
  \label{ideals-embed-into-closed-subsets}
  We have $V(f_1, \dots, f_n) \subseteq V(g_1, \dots, g_m)$
  as subsets of $\Spec A$
  if and only if
  $(g_1, \dots, g_m) \subseteq (f_1, \dots, f_n)$
  as ideals of $A$.
\end{lemma}

\begin{proof}
  The inclusion $V(f_1, \dots, f_n) \subseteq V(g_1, \dots, g_m)$
  means a map $\Spec (A/(f_1, \dots, f_n)) \to \Spec (A/(g_1, \dots, g_m))$
  over $\Spec A$.
  By \Cref{spec-embedding}, this is equivalent to
  a homomorphism $A/(g_1, \dots, g_m) \to A/(f_1, \dots, f_n)$,
  which in turn means the stated inclusion of ideals.
\end{proof}

\begin{lemma}[using \axiomref{loc}, \axiomref{sqc}, \axiomref{Z-choice}]%
  \label{closed-subtype-affine}
  A closed subtype $C$ of an affine scheme $X=\Spec A$ is an affine scheme
  with $C=\Spec (A/I)$ for a finitely generated ideal $I\subseteq A$.
\end{lemma}

\begin{proof}
  By \axiomref{Z-choice} and boundedness,
  there is a cover $D(f_1),\dots,D(f_l)$, such that
  on each $D(f_i)$, $C$ is the vanishing set of functions
  \[ g_1,\dots,g_n:D(f_i)\to R\rlap{.} \]
  By \Cref{ideals-embed-into-closed-subsets},
  the ideals generated by these functions
  agree in $A_{f_i f_j}$,
  so by \Cref{fg-ideal-local-global},
  there is a finitely generated ideal $I\subseteq A$,
  such that $A_{f_i}\cdot I$ is $(g_1,\dots,g_n)$
  and $C=\Spec A/I$.
\end{proof}

\subsection{Open subtypes}

While we usually drop the prefix ``qc'' in the definition below,
one should keep in mind, that we only use a definition of quasi compact open subsets.
The difference to general opens does not play a role so far,
since we also only consider quasi compact schemes later.

\begin{definition}%
  \label{qc-open}
  \begin{enumerate}[(a)]
  \item A proposition $P$ is \notion{(qc-)open}, if there merely are $f_1,\dots,f_n:R$,
    such that $P$ is equivalent to one of the $f_i$ being invertible.
  \item Let $X$ be a type.
    A subtype $U:X\to\Prop$ is \notion{(qc-)open}, if $U(x)$ is an open proposition for all $x:X$.
  \end{enumerate}
\end{definition}

\begin{proposition}[using \axiomref{loc}, \axiomref{sqc}]%
  \label{open-iff-negation-of-closed}
  A proposition $P$ is open
  if and only if
  it is the negation of some closed proposition
  (\Cref{closed-proposition}).
\end{proposition}

\begin{proof}
  Indeed, by \Cref{generalized-field-property},
  the proposition $\inv(f_1) \lor \dots \lor \inv(f_n)$
  is the negation of ${f_1 = 0} \land \dots \land {f_n = 0}$.
\end{proof}

\begin{proposition}[using \axiomref{loc}, \axiomref{sqc}]%
  \label{open-union-intersection}
  Let $X$ be a type.
  \begin{enumerate}[(a)]
  \item The empty subtype is open in $X$.
  \item $X$ is open in $X$.
  \item Finite intersections of open subtypes of $X$ are open subtypes of $X$.
  \item Finite unions of open subtypes of $X$ are open subtypes of $X$.
  \item Open subtypes are invariant under pointwise double-negation.
  \end{enumerate}
  Axioms are only needed for the last statement.
\end{proposition}

In \Cref{open-subscheme} we will see that open subtypes of open subtypes of a scheme are open in that scheme.
Which is equivalent to open propositions being closed under dependent sums.

\begin{proof}[of \Cref{open-union-intersection}]
  For unions, we can just append lists.
  For intersections, we note that invertibility of a product
  is equivalent to invertibility of both factors.
  Double-negation stability
  follows from \Cref{open-iff-negation-of-closed}.
\end{proof}

\begin{lemma}%
  \label{preimage-open}
  Let $f:X\to Y$ and $U:Y\to\Prop$ open,
  then the \notion{preimage} $U\circ f:X\to\Prop$ is open.
\end{lemma}

\begin{proof}
  If $U(y)$ is an open proposition for all $y : Y$,
  then $U(f(x))$ is an open proposition for all $x : X$.
\end{proof}

\begin{lemma}[using \axiomref{loc}, \axiomref{sqc}]%
  \label{open-inequality-subtype}
  Let $X$ be affine and $x:X$, then the proposition
  \[ x\neq y \]
  is open for all $y:X$.
\end{lemma}

\begin{proof}
  We show a proposition, so we can assume $\iota: X\to \A^n$ is a subtype.
  Then for $x,y:X$, $x\neq y$ is equivalent to $\iota(x)\neq\iota(y)$.
  But for $x,y:\A^n$, $x\neq y$ is the open proposition that $x-y\neq 0$.
\end{proof}

The intersection of all open neighborhoods of a point in an affine scheme,
is the formal neighborhood of the point.
We will see in \Cref{intersection-of-all-opens}, that this also holds for schemes.

\begin{lemma}[using \axiomref{loc}, \axiomref{sqc}]%
  \label{affine-intersection-of-all-opens}
  Let $X$ be affine and $x:X$, then the proposition
  \[ \prod_{U:X\to \Open}U(x)\to U(y) \]
  is equivalent to $\neg\neg (x=y)$.
\end{lemma}

\begin{proof}
  By \Cref{open-union-intersection}, $\neg\neg (x=y)$ implies $\prod_{U:X\to \Open}U(x)\to U(y)$.
  For the other implication,
  $\neg (x=y)$ is open by \Cref{open-inequality-subtype}, so we get a contradiction.
\end{proof}

We now show that our two definitions (\Cref{affine-open}, \Cref{qc-open})
of open subtypes of an affine scheme are equivalent.

\begin{theorem}[using \axiomref{loc}, \axiomref{sqc}, \axiomref{Z-choice}]%
  \label{qc-open-affine-open}
  Let $X=\Spec A$ and $U:X\to\Prop$ be an open subtype,
  then $U$ is affine open, i.e. there merely are $h_1,\dots,h_n:X\to R$ such that
  $U=D(h_1,\dots,h_n)$.
\end{theorem}

\begin{proof}
  Let $L(x)$ be the type of finite lists of elements of $R$,
  such that one of them being invertible is equivalent to $U(x)$.
  By assumption, we know
  \[\prod_{x:X}\propTrunc{L(x)}\rlap{.}\]
  So by \axiomref{Z-choice}, we have $s_i:\prod_{x:D(f_i)}L(x)$.
  We compose with the length function for lists to get functions $l_i:D(f_i)\to\N$.
  By \Cref{boundedness}, the $l_i$ are bounded.
  Since we are proving a proposition, we can assume we have actual bounds $b_i:\N$.
  So we get functions $\tilde{s_i}:D(f_i)\to R^{b_i}$,
  by append zeros to lists which are too short,
  i.e. $\widetilde{s}_i(x)$ is $s_i(x)$ with $b_i-l_i(x)$ zeros appended.

  Then one of the entries of $\widetilde{s}_i(x)$ being invertible,
  is still equivalent to $U(x)$.
  So if we define $g_{ij}(x)\colonequiv \pi_j(\widetilde{s}_i(x))$,
  we have functions on $D(f_i)$, such that
  \[
    D(g_{i1},\dots,g_{ib_i})=U\cap D(f_i)
    \rlap{.}
  \]
  By \Cref{affine-open-trans}, this is enough to solve the problem on all of $X$.
\end{proof}

This allows us to transfer one important lemma from affine-opens to qc-opens.
The subtlety of the following is that while it is clear that the intersection of two
qc-opens on a type, which are \emph{globally} defined is open again, it is not clear,
that the same holds, if one qc-open is only defined on the other.

\begin{lemma}[using \axiomref{loc}, \axiomref{sqc}, \axiomref{Z-choice}]%
  \label{qc-open-trans}
  Let $X$ be a scheme, $U\subseteq X$ qc-open in $X$ and $V\subseteq U$ qc-open in $U$,
  then $V$ is qc-open in $X$.
\end{lemma}

\begin{proof}
  Let $X_i=\Spec A_i$ be a finite affine cover of $X$.
  It is enough to show, that the restriction $V_i$ of $V$ to $X_i$ is qc-open.
  $U_i\colonequiv X_i\cap U$ is qc-open in $X_i$, since $X_i$ is qc-open.
  By \Cref{qc-open-affine-open}, $U_i$ is affine-open in $X_i$,
  so $U_i=D(f_1,\dots,f_n)$.
  $V_i\cap D(f_j)$ is affine-open in $D(f_j)$, so by \Cref{affine-open-trans},
  $V_i\cap D(f_j)$ is affine-open in $X_i$.
  This implies $V_i\cap D(f_j)$ is qc-open in $X_i$ and so is $V_i=\bigcup_{j}V_i\cap D(f_j)$.
\end{proof}

\begin{lemma}[using \axiomref{loc}, \axiomref{sqc}, \axiomref{Z-choice}]%
  \label{qc-open-sigma-closed}
  \begin{enumerate}[(a)]
  \item qc-open propositions are closed under dependent sums:
    if $P : \Open$ and $U : P \to \Open$,
    then the proposition $\sum_{x : P} U(x)$ is also open.
  \item Let $X$ be a type. Any open subtype of an open subtype of $X$ is an open subtype of $X$.
  \end{enumerate}
\end{lemma}

\begin{proof}
  \begin{enumerate}[(a)]
  \item Apply \Cref{qc-open-trans} to the point $\Spec R$.
  \item Apply the above pointwise.
  \end{enumerate}
\end{proof}

\begin{remark}
  \Cref{qc-open-sigma-closed} means that
  the (qc-) open propositions constitute a \notion{dominance}
  in the sense of~\cite{rosolini-phd-thesis}.
\end{remark}

The following fact about the interaction of closed and open propositions
is due to David Wärn.

\begin{lemma}%
  \label{implication-from-closed-to-open}
  Let $P$ and $Q$ be propositions
  with $P$ closed and $Q$ open.
  Then $P \to Q$ is equivalent to $\lnot P \lor Q$.
\end{lemma}

\begin{proof}
  We can assume $P = (f_1 = \dots = f_n = 0)$
  and $Q = (\inv(g_1) \lor \dots \lor \inv(g_m))$.
  Then we have:
  \begin{align*}
    (P \to Q) &= \qquad
    \text{\Cref{generalized-field-property} for $g_1, \dots, g_m$}\\
    (P \to \lnot (g_1 = \dots = g_m = 0)) &= \\
    \lnot (f_1 = \dots = f_n = g_1 = \dots = g_m = 0) &= \qquad
    \text{\Cref{generalized-field-property} for $f_1, \dots, f_n, g_1, \dots, g_m$}\\
    (\inv(f_1) \lor \dots \lor \inv(f_n) \lor \inv(g_1) \lor \dots \lor \inv(g_m) &= \qquad
    \text{\Cref{generalized-field-property} for $f_1, \dots, f_n$}\\
    \lnot P \lor Q &
  \end{align*}
\end{proof}


\subsection{Compact Hausdorff}
\begin{definition}
  Let $S$ be Stone, $C\subseteq S$. 
  Then $C$ is open if it is the countable union of decidable subsets. 
\end{definition}
\begin{lemma}
  For $S$ Stone and $C\subseteq S$, 
  $C$ is closed iff it's complement is open 
  and $C$ is open iff it's complement is closed. 
\end{lemma}



\begin{proof}
  This follows from the fact that the complement of a decidable subset is decidable and 
  \Cref{LemDecidableSubsetsDeMorgan}.
\end{proof}
\begin{lemma}
  For $S$ Stone, any cover by opens merely has a finite subcover.
\end{lemma}
\begin{proof}
  Let $S= \bigcup_{i:I} A_i$ be a cover of $S$ by open sets. 
  Assume furthermore $S= Sp(B)$. 
  As every open is the union of decidable subsets, we may assume $A_i$ decidable, 
  and thus corresponding to points $a_i \in B$. 
  These points are such that $1 = \bigvee_{i:I} a_i$. 
  As $B$ is countably presented, it is countable. 
  Thus $(a_i)_{i:I}$ is a countable set. 
  The morphism $I\to B$ is surjective, and as we're proving a proposition,
  we may use type-theoretic AC to give 
  a countable subset $I_0 \subseteq I$ such that $\bigvee_{i:I_0} a_i = 1$ as well. 
  So $S = \bigcup_{i:I_0} A_i$ for $I_0$ countable. 
\end{proof}

Note that the basic clopens are not the only clopens. 
I.e. not every set that is both a countable intersection of decidable subsets and 
a countable union of decidable subset is itself decidable. 
In $B_\infty$, we can describe the even numbers both as the infinite meet of cofinite sets excluding odd numbers up to $n$
and the join of finite sets including even numbers up to $n$. 
But the even numbers do not themselves for an element of $B_\infty$.
Thus the set of maps $B\to 2$ sending every $\chi_{2n}$ to $1$ is clopen but not decidable. 

%
%\begin{lemma}
%  For $S$ Stone, any collections of closed sets satisfying the finite intersection property 
%  has a merely inhabited intersection. 
%\end{lemma}
%\begin{proof}
%  Suppose that $a_i$ is a collection of points such that $a_i \wedge a_j \neq 0$ whenever $i\neq j$. 
%  We claim that $\bigwedge a_i \neq 0$. 
%  Suppose $\bigwedge a_i = 0$. 
%\end{proof}


\begin{definition}
We define a type $X$ to be compact Hausdorff iff 
$X$ is the quotient of a stone space $S$ by a closed equivalence relation. 
%there is a surjection $e:S\to X$ for $S$ Stone. 
%
A subtype $C\subseteq X$ is closed respectively open iff it's pre-image under the quotient map is. 
\end{definition}

\begin{lemma}
  In a compact Hausdorff, closed sets are closed under intersection. 
\end{lemma}


%\begin{lemma}
%  Let $X$ be a Compact Hausdorff type, 
%  and suppose that $(A_i)_{i:I}$ is a collection of opens in $X$ such that 
%  $X = \bigcup_{i:I} A_i$. 
%  There then exists a finite subset $I_0\subseteq I$ with 
%  $X = \bigcup_{i:I_0} A_i$. 
%\end{lemma} 

\begin{lemma}
  In a Compact Hausdorff space, the complement of an open is closed, and the complement of a closed is open. 
\end{lemma}
\begin{proof}
  Let $e : S \to X$ be the quotient map of a Stone space by a closed equivalence relation. 
  and let $(A_n)_{n:\mathbb N}$ be a countable family of decidable subsets in $S$. 

  First, we claim that 
  $X - \bigcup_{n:\mathbb N} e(A_n)$
  is closed in $X$. 
\end{proof}

\begin{lemma}
  Whenever $X$ is compact Hausdorff, $F_0, F_1$ are closed and disjoint, 
  there exist $G_0, G_1$ disjoint clopen such that 
  $F_i \subseteq X - G_{1-i}$ and $G_0 \cup G_1 = X$. 
  
\end{lemma}

\subsection{Intersection of closed in compact Hausdorff}

\begin{lemma}
  In a compact Hausdorff, closed sets are closed under intersection. 
\end{lemma}
\begin{proof}
  
\end{proof}



\begin{lemma}
  Any Stone space merely is a closed subspace of Cantor space. 
\end{lemma}
\begin{proof}
  Let $S$ be a Stone space, and let it's underlying Boolean algebra $B$ be generated by 
  $(b_n)_{n:\mathbb N}$ under quotient of the relations ${\phi_i}_{i:\mathbb N}$. 
  Then $S = \{ x: 2^\mathbb N | \forall_{i:\mathbb n} x(\phi_i) = 0\}$, 
\end{proof}


\begin{lemma}
  For, $D\subseteq 2^\mathbb N$ decidable, $\sim$ a closed equivalence relation on $2^\mathbb N$,
  the set $$\{x:S | \exists y : D (x\sim y)\}$$ is closed. 
\end{lemma}
\begin{proof}
  Let $x:S$. We need to show that $\exists (y:S) D(y) \wedge x \sim y$ is a closed proposition. 

  Note first that as $\sim $ is closed, $x \sim \cdot $ is a closed subset of $S$. 
  Therefore, $x\sim \cdot = \bigcap_{n:\mathbb N} E_n$ for $(E_n)_{n:\mathbb N}$ a
  countable family of decidable subsets of $S$, without losing generality, 
  we may even assume that $E_n \subseteq E_m$ whenever $m\geq n$. 
  We thus need to show that 
  $$
  \exists (y:S) D(y) \wedge (\bigcap_{n:\mathbb N} E_n)(y) 
  = 
  \exists (y:S) (\bigcap_{n:\mathbb N} D \cap  E_n)(y) 
  $$
  is closed. 
%
  Now we claim that 
  $\exists (y:S) D(y) \wedge E_n(y)$ is closed for all $n:\mathbb N$. 
  There merely exists an $m:\mathbb N$ such that both $D$ and $E_n$ only consider 
  the first $m$ entries of a sequence. 
  
\end{proof}



\begin{lemma}
  For $S$ Stone, $D\subseteq S$ decidable, 
  $\sim$ a decidable equivalence relation on $S$,
  the set $\{x:S | \exists y : D (x\sim y)\}$ is closed. 
\end{lemma}
\begin{proof}
  Let $B = 2^S$, so $S = Sp(B)$. 
  As $D$ is decidable, 
%  there is some $b:B$ such that $D(y) := (y(b) = 1)$. 
  there is some $n:\mathbb N$ such that $D(y)$ only depends on $y|_n$. 

  As $\sim$ is decidable, there is a finite set $I_0\subseteq \mathbb N$,
  such that $x\sim y = \prod_{i:I_0} x(i) = y(i)$. 

  Thus 
  $$
   \exists (y : D) (x\sim y) = 
  || \Sigma(y:2^\mathbb N) y(b) = 1 \wedge \prod(i:I_0) x(i) = y(i)||
  $$
\end{proof}



\begin{lemma}
  Let $S$ Stone, then $D\subseteq S$ is closed iff 
  $D\subseteq S\subseteq 2^{\mathbb N}$ is closed. 
\end{lemma}
\begin{proof}
  Follows immediately from countable intersection of basic clopen. 
\end{proof}




%%  Let $A,B\subseteq X$ be two closed subsets of a compact Hausdorff space $X = S/ \sim$. 
%%  If we know that closed subsets contain are exactly those containing their limits this is very easy right? 
%%  Then any sequence has it's limit both in $A$ and $B$. 
%%\begin{lemma}
%%  Whenever $x_n$ is a convergent sequence, so is $f(x_n)$. 
%%\end{lemma}
%%\begin{proof}
%%  Follows immediately from \Cref{sequenceConvergentIffLimit}.
%%\end{proof}
%%
%%
%%\begin{lemma}
%%  In a compact Hausdorff, whenever a subset $A$ contains all of its limit points, it is closed. 
%%\end{lemma}
%%
%%\begin{proof}
%%  Suppose $A\subseteq X$ contains all of it's limit points. We will show that $f^{-1}(A)$ is closed. 
%%  Let $(x_n)_{n:\mathbb N}$ be a sequence in $f^{-1}(A)$ with limit $l$, 
%%  then 
%%  $(f(x_n))_{n:\mathbb N}$ is a sequence in $A$ with limit $f(l)$. 
%%  $A$ contains $f(l)$ by assumption. 
%%  Therefore $l\in f^{-1}(A)$. 
%%  Thus every sequence in $f^{-1}(A)$ with a limit has its limit in $f^{-1}(A)$. 
%%\end{proof}
%%
%%\begin{lemma}
%%  In a Stone space, whenever a subset $A$ contains all of its limit points, it is closed. 
%%\end{lemma}
%%\begin{proof}
%%  Let $A \subseteq S$ contain all of it's limit points. 
%%  We will show $A$ is a countable intersection of decidable subsets of $S$, hence closed. 
%%  As $S$ is a subset of Cantor space, we may assume it is Cantor space. 
%%  Thus $A$ is a set of binary sequences. 
%%
%%  We will denote $D_n$ be the set of initial segments of length $n$ occuring in $A$. 
%%  We claim this is well defined, that's not a problem, as it's the image of an operation. 
%%
%%  Counterexample : $A = \{ \overline 0 | p\}$ which contains all of it's limit points
%%  (any sequence in $A$ must be $\overline 0$ constantly, which has a limit if the sequence exists in $A$). 
%%  However, $D_n$ is not decidable. 
%%  Also $A$ is not the intersection of countably many decidable sets I believe. 
%%  Unless off course $p$ is of the form $\alpha=0$, but those are not the only propositions.
%%  For example, the proposition $\beta\neq 0$ cannot be written in that form for general $\beta$. 
%%\end{proof}


\subsection{Open propositions}
\input{OvertlyDiscrete/FactorizationFin}

\section{Analysis}

\subsection{Convergence}
\input{Convergence/convergenceClosed}
\paragraph{Extensional convergence }  
\begin{definition}
  Let $B_\infty$ be the Boolean algebra on countably many generators $(p_n)_{n:\mathbb N}$ 
  over the equivalence $p_n\wedge p_m = 0 $ whenever $n \neq m$. 
\end{definition} 
\begin{definition}
  We denote $\Noo$ be the spectrum of $B_\infty$. 
\end{definition} 
\begin{lemma}
  $B_\infty$ is isomorphic with the Boolean algebra of 
  finite/cofinite subsets of $\mathbb N$. 
\end{lemma}
\begin{proof}
  To go from $B_\infty$ to subsets of $\mathbb N$, we send
  the generators $p_n$ to the singleton $\{n\}$, which are clearly finite. 
  We call the induced Boolean operation $f$. 

  To go from finite/cofinite subsets of $\mathbb N$ to $B_\infty$,
  a finite subset $I$ of $\mathbb N$ is sent to the element 
  $\bigvee_{i \in I} p_i$, and a cofinite subset $J$ is sent to the element 
  $\bigwedge_{i \in J^C} \neg p_i$.  
  We call this function $g$ and we need to show that $g$ is a Boolean morphism. 
  \begin{itemize}
    \item By deMorgan's laws, $g$ preserves $\neg$. 
    \item To see that $g$ respects $\vee$, we need to check three cases
      \begin{itemize}
        \item If both $I,J$ are finite, then 
        \begin{equation} 
          g(I \cup J) = \bigvee_{i\in I \cup J} p_i= \bigvee_{i\in I} p_i \vee \bigvee_{j\in J} p_j 
        \end{equation}
      \item If both $I,J$ are cofinite, we have
        \begin{equation}
          g(I) \vee g(J) = 
          \bigwedge_{i \in I^C} \neg p_i \vee 
          \bigwedge_{j \in J^C} \neg p_j 
          = 
          \bigwedge_{i\in I^C} 
          \bigwedge_{j \in J^C}(\neg p_i \vee  \neg p_j) 
        \end{equation}
        Now note that $\neg p_i \vee \neg p_j = \neg ( p_i \wedge p_j)$, which 
        is $1$ if $i \neq j$ and $p_i$ if $i =j$. 
        We can leave $1$ out of the meet, and we are left with the intersection of $I^C$ and $J^C$, so
        \begin{equation}
          g(I) \vee g(J) = 
          \bigwedge_{i \in (I^C \cap J^C)} \neg p_i
          = 
          \bigwedge_{i \in (I \cup J)^C} \neg p_i 
        \end{equation} 
        as the union of $I$ and $J$ is also cofinite, this equals 
          $ g( I \cup J)$. 
        \item If $I$ is finite and $J$ cofinite, we have 
        \begin{equation}
        g(I) \vee g(J) = (\bigvee_{i\in I} p_i) \vee (\bigwedge_{j \in J^C} \neg p_j)
        = \bigwedge_{j \in J^C} (\bigvee_{i \in I}( p_i \vee \neg p_j))
        \end{equation}
        If $i\neq j$, then $p_i\wedge p_j = 0$, hence $\neg p_j \geq p_i$ and $p_i \vee \neg p_j  = \neg p_j$
        If $i = j$, then $p_i \vee \neg p_j = 1$.
%        \begin{equation}
%        g(I) \vee g(J) = 
%        = \bigwedge_{j \in J^C} (\bigvee_{i \in I-J}( p_i \vee \neg p_j))
%        \end{equation}
%
%        \item If $I$ is cofinite and $J$ is finite, we have that $I \cup J$ is cofinite.
%        Thus 
%        \begin{equation}
%          g(I \cup J) = \bigwedge_{i \in (I \cup J)^C} \neg p_i
%        \end{equation}
%
      \end{itemize}
    \item The case for $\wedge$ is completely dual to the case for $\vee$. 
  \end{itemize}
We conclude that $g$ is a Boolean morphism. 
Furthermore, $g$ and $f$ are clearly inverses, thus the Boolean algebras are isomorphic. 
\end{proof}

  \begin{lemma}\label{lemBinftyNormalForm}
  Any element of $B_\infty$ can be written as 
  either $\bigvee_{i\in I} p_i$  or
  as $\bigwedge_{j\in J} \neg p_j$ 
  for finite $I,J\subseteq \mathbb N$. 
\end{lemma}
\begin{proof}
  Remark that whenever $n \neq m$, we have that 
  $\neg p_n \geq p_m$ as $p_m \wedge p_n = 0$. 
\end{proof}
There is canonical embedding $\mathbb N \hookrightarrow \Noo$, 
wich sends $n$ to the unique function $\chi_{n}$ sending $p_n$ to $1$. 
We denote $\infty \in \Noo$ for the function which is constantly $0$. 
By \Cref{PropMarkov}, if an element is not $\infty$, 
it comes from the embedding $\mathbb N \hookrightarrow \Noo$. 
\begin{lemma}\label{LemmaOpensContainingInfty}
  Let $U$ be an open subset of $\Noo$ containing $\infty$.
  Then there merely exists an $N:\mathbb N$ such that whenever $n\geq N$, 
  $\chi_n\in U$ as well. 
\end{lemma}
\begin{proof}
  It is sufficient to prove the lemma for $U$ a basic open. 
  Assume $b : B_\infty $ is such that 
  $U = \{ \phi: B_\infty \to 2| \phi(b) = 1\}$.
  Assume furthermore that $\infty \in U$.
%  so $U$ contains the function sending every $p_i$ to $0$. 
  by \Cref{lemBinftyNormalForm}, $b$ can have two forms.
  If $b = \vee_{i\in I} p_i$, then as $\infty(b) = 0$, 
  we must have $I = \emptyset$, and thus $b = 0$, 
  which means $U$ is empty, contradicting $\infty\in U$. 
  Therefore, 
  $b$ must be of the form $\wedge_{j \in J} \neg p_j$. 
  Note that for $N = \max J + 1$, whenever $n>J$, 
  $\chi_n$  sends $b$ to $1$. 
  Thus $\chi_n \in U$ as well, and we are done. 
\end{proof}

\begin{definition}
  Let $\alpha$ be a sequence in $X$, we say that $\alpha$
  is convergent iff there exists an extension. 
  \begin{equation}\begin{tikzcd}
    \mathbb N \arrow[r, "\alpha"] \arrow[d,hook]  & X \\
    \Noo \arrow[ru,dashed]
  \end{tikzcd}\end{equation}  
\end{definition}  



\begin{proposition}
  A sequence is convergent iff it has a limit
\end{proposition}
\begin{proof}
  Let $\alpha$ be convergent, with extension $\overline \alpha$.
  we claim that $\overline \alpha(\infty)$ is a limit of $\alpha$.
  Let $U \subseteq X$ be an open containing $x$. 
  As $\overline\alpha^{-1}(U)$ is an open subset of $\Noo$ containing $\infty$,
  \Cref{LemmaOpensContainingInfty} tells us there exists some $N$ such that $[N,\infty]\subseteq \overline \alpha^{-1}(U)$. 
  Thus there exists an $N$ such that for $n\geq N$, we have $\alpha(n) \in U$, as required. 

  Conversely, suppose $\alpha$ has limit $x$. 
  Assume $X = Sp(B)$, and let $b\in B$. Then $b$ corresponds to a decidable subset $U\subseteq X$.
  For any decidable subset $U \subseteq X$, we have 
  $\alpha^{-1}(U)$ a decidable subset of $\mathbb N$. 
  We claim that $\alpha^{-1}(U)$ is either finite or cofinite. 
  As $U$ is decidable, we can decide wheter $x\in U$. If $x\in U$, $\alpha^{-1}(U)$ is cofinite, as 
  $\alpha(n) \in U$ for all $n \geq N$ for some $N$. 
  If $x\notin U$, we have $x\in U^C$, which is also decidable and therefore $\alpha^{-1}(U^C)$ is cofinite. 
  As $\alpha^{-1}(U) ^ C = \alpha^{-1}(U^C)$, it follows that $\alpha^{-1}(U)$ is finite. 
  Thus $\alpha^{-1}(U)$ is finite or cofinite for any decidable subset $U\subseteq X$. 
  Finite and cofinite subsets of $\mathbb N$ correspond to elements of $B_\infty$. 
  Therefore, $\alpha$ induces a map $B \to B_\infty$, which corresponds to a map 
  $\overline \alpha: \Noo \to X$. 

  We claim that $\overline \alpha$ extends $\alpha$. 
  Denote $\iota$ for the map $\mathbb N \to \Noo$. 
  We need to show that $\overline \alpha \circ \iota = \alpha$. 
  First note that by definition, we have that $(\overline \alpha \circ \iota)^{-1}(U) = \alpha^{-1}(U)$ 
  for any decidable $U\subseteq X$. 





\end{proof}





\subsection{The interval}
%The goal of this section is to define the interval $[-2,2]_\mathbb R$ as a scheme. 
We assume $\N, \mathbb Q$ have been defined in HoTT
with linear propostional order relations $<,\leq, > ,\geq$ playing nicely together 
and standard algebraic operations. 
From these, we can define the subtype $\mathbb Q_{>0}=\sum_{q : \mathbb Q} (q>0)$, 
and the absolute-value function $|\cdot|$ on $\mathbb Q$. 

\begin{definition}
  A pre-Cauchy sequence is a sequence of rational numbers $(q_n)_{n: \N}$ with $-2 \leq q_n \leq 2$ 
  for all $n:\N$
%  together with a term of type
  such that for every $\epsilon: \mathbb Q_{>0}$, we have an $N_\epsilon:\N$, 
  such that whenever $n,m \geq N_\epsilon$, we have 
\begin{equation}
%  \forall \epsilon : \mathbb Q_{>0} \Sigma N : \N \forall m,n : \N (m,n \geq N) \to 
  | q_n - q_m | \leq \epsilon
\end{equation} 
\end{definition}

\begin{definition}
Given two pre-Cauchy sequences $p = (p_n)_{n\in\N}, q=(q_n)_{n\in\N}$, 
we define the proposition $p \sim_C  q$ as 
%for all $\epsilon : \mathbb Q_{>0}$ there exists an $N :\N$ such that whenever $n \geq N$, we have
\begin{equation}
  p \sim_C q : = \forall (\epsilon : \mathbb Q_{>0} )\exists ( N :\N) \forall (n : \N) ((n \geq N) \to 
  (| p_n - q_n| \leq  \epsilon))
\end{equation}
\end{definition}
Note that $\sim_C$ defines an equivalence relation on pre-Cauchy sequences. 
\begin{definition}
We define the type of Cauchy sequences as the type of pre-Cauchy sequences quotiented by $\sim_C$. 
\end{definition}

%\begin{definition}
%  A binary sequence consists of an initial segment $I \subseteq \N$
%  and a function $x:I \to 2$. 
%If $I$ is (in)finite, we call the binary sequence (in)finite as well. 
%\end{definition} 
%
%For $x$ a finite binary sequence and $y$ any binary sequence, 
%we'll denote $(x,y)$ for their concatenation, 
%and $\overline x$ for the infinite sequence repeating $x$. 
%
Denote $T = \{-1,0,1\}$. 
\begin{lemma}
  $T^\N$ is a scheme. 
\end{lemma}
\begin{proof}
  Sketch: partition $2^\N$ as follows: 
  For $\alpha: 2^\N$, we'll make a sequence $\beta: T^\N$.
  consider for each $n$ the $n$'th block of 2 entries in $\alpha$
  if both are $0$, $\beta(n) = 0$. 
  If the first is $1$, $\beta(n) = -1$
  If first is $0$ and the second is $1$, then $\beta(n) = 1$. 
  This is a closed equivalence relation. 
\end{proof} 

Consider the relation $\sim_s$ on $T^{\N}$, 
such that for any finite binary sequence $x$, we have 
\begin{align}
  (x,1,\overline 0) &\sim_t (x ,0, \overline 1) \\
  (x,-1,\overline 0) &\sim_t (x ,0, \overline {-1})\\
  (x,1,\overline {-1}) &\sim_t (x , \overline 0) \\
  (x,-1,\overline {1}) &\sim_t (x , \overline 0) 
\end{align} 
\begin{lemma}
$\sim_t$ induces a closed equivalence relation on $2^\N$. 
\end{lemma}
\begin{proof}
  TODO
\end{proof} 

\begin{proposition}\label{propTernaryCauchy}
  $T^\N/ \sim_t$ is isomorphic to the type of Cauchy sequences. 
\end{proposition} 
\begin{definition}%Construction might be better than definition here, but WIP so who cares. 
  For $\alpha: T^\N$, define the rational sequence $tri(\alpha)$ by 
  \begin{equation} (tri(\alpha))_n :  = \sum\limits_{0 \leq i \leq n} \frac{\alpha(i)} { 2^{i}} \end{equation}  
  This sequence is pre-Cauchy with $N_\epsilon$ given by the first $n$ with $(\frac12)^n<\epsilon$. 
\end{definition}  
%
%  Also, whenever $\alpha\sim_t \beta$, we have 
%  $tri(\alpha) \sim_C tri(\beta)$. 
%  Therefore $tri$ induces a function from $T^\N / \sim_t$ to Cauchy sequences. 
\begin{definition}
  Given a pre-Cauchy sequence $p$, 
  we will define a $T$-sequence $\alpha  = c(p): T^\N$.
  Consider any $i:\N$, and suppose $\alpha(j)$ has been defined for $0 \leq j<i$. 
%
  Let $\epsilon_i = (\frac12)^{i+1}$. %Placeholder value.
  Define $N_i:= N_{\epsilon_i}$. %is such that for $n,m \geq N_i$, we have $|p_n - p_m| < \epsilon_i$. 
%
  Consider 
  \begin{equation}
    \widetilde p_i = p_N - \sum\limits_{0\leq j < i} \frac {\alpha(j)}{2^{j}}.
  \end{equation}
  As the order on $\mathbb Q$ is total, we can define 
  \begin{equation}
    \alpha(i) = \begin{cases}
    \phantom{-} 1  \text{ if } \widetilde p_i \geq    (\frac12)^{i} \\
    -1             \text{ if } \widetilde p_i \leq  - (\frac12)^{i} \\
    \phantom{-} 0 \text{ otherwise } 
    \end{cases} 
  \end{equation}  
\end{definition} 
We shall now prove the following four things: 
\begin{itemize}
  \item 
    $c(tri(\alpha)) \sim_t \alpha$ for any $\alpha: T^n$.
  \item 
    $tri(c(p)) \sim_C p$ for any pre-Caucy sequence $p$. 
  \item 
    Whenever $p \sim_C q$, we have $c(p)\sim_t c(q)$. 
  \item 
    Whenever $\alpha \sim_t \beta$, we have $tri(\alpha) \sim_C tri(\beta)$. 
\end{itemize}
It follows that $c$ and $tri$ are maps between Cauchy sequences and $T^\N /\sim_t$ 
which are each other's inverse, proving Proposition \ref{propTernaryCauchy}
\begin{lemma} $tri(c(p)) \sim_C p$ for any pre-Caucy sequence $p$. 
\end{lemma} 



\begin{proof}
  Let $\epsilon>0$ be given, consider $n:\N$ such that
  $(\frac12)^n < \epsilon$. 
  We claim that for $m\geq N_n$, we have that $|p_m- tri(c(p))_m| < \epsilon$. 

  By definition $p_{N_n} $  
\end{proof} 







\begin{definition}
  A \textbf{Cauchy sequence} is a sequence $x : \mathbb N \to \mathbb Q$ such that
  for any $n,m:\mathbb N$, we have %$0\leq x_n \leq 1$ and 
$|x_n-x_m| \leq (\frac12)^n + (\frac12)^m$. 
\end{definition}

\begin{definition}
Given two Cauchy sequences $p = (p_n)_{n\in\mathbb N}, q=(q_n)_{n\in\mathbb N}$, 
we define the proposition $p \sim_C  q$ as 
\begin{equation}
  p \sim_C q : = \forall (\epsilon : \mathbb Q_{>0} )\exists ( N :\mathbb N) \forall (n : \mathbb N) ((n \geq N) \to 
  (| p_n - q_n| \leq  \epsilon))
\end{equation}
\end{definition}

\begin{definition}
  The type of \textbf{Cauchy reals} is given by 
  the type of Cauchy sequences modulo $\sim_C$.
\end{definition}

%\begin{definition}
%  A Cauchy sequence in the interval is a Cauchy sequence $x$ such that 
%  for any $n:\mathbb N$, we have $0\leq x_n \leq 1$. 
% % 
%  The interval of Cauchy reals is given by the type of Cauchy sequences in the interval 
%  modulo $\sim_C$. 
%\end{definition}  

We want to show that the interval of Cauchy reals are a scheme. 
Informally, to any binary sequence $\alpha : \mathbb N \to 2$, 
we can associate a Cauchy sequence 
\begin{equation}n\mapsto \sum\limits_{i = 0 }^n \frac {\alpha(i)}{2^{i+1}}\end{equation}
and we are going to give a closed relation on Cantor space such that 
two binary sequences are equivalent iff they correspond to the same Cauchy reals. 
%
First, we'll need some notation.
\begin{definition}
Given a binary sequence $\alpha:\mathbb N \to 2$ and a natural number $n : \mathbb N$  
we denote $\alpha|_n: \mathbb N_{\leq n} \to 2$ for the 
restriction of $\alpha$ to a finite sequence of length $n$. 
We denote $\overline 0, \overline 1$ for the binary sequences which are constantly $0$ and $1$ respectively. 
We denote $0,1$ for the sequences of length $1$ hitting $0,1$ respectively. 
If $x$ is a finite sequence and $y$ is any sequence, denote $x\cdot y$ for their concatenation. 
\end{definition} 
Now we'll give a definition for when two finite binary sequences of length $n$ correspond 
to real numbers whose distance is $\leq (\frac12)^n$.
Basically, we want for every finite sequence $z$ that 
$(z \cdot 0 \cdot \overline 1)$ and  $(z \cdot 1 \cdot \overline 0)$ are equivalent. 

\begin{definition}
Now let $n:\mathbb N$ and $x,y:\mathbb N_{\leq n} \to 2$ be two sequences of length $n$. 
We say $x,y$ are near if we have an $m:\mathbb N$ with $m\leq n$
and some $a: \mathbb N_{\leq m} \to 2$, 
such that one of $(a \cdot 0 \cdot \overline 1)|_n,  ( a \cdot 1 \cdot \overline 0)|_n$
is equal to $x$ and the other is equal to $y$. 
We denote $\text{near}_n(x,y)$ if $x,y$ are near. 
%
To be precise, we define 
\begin{equation}
  \text{near}_n(x,y) = 
\Sigma(m:\mathbb N) m \leq n \wedge 
  \Sigma (a : Fin_m \to 2) 
\bigg( \big( (x,y) = 
((a \cdot 0 \cdot \overline 1)|_n,  ( a \cdot 1 \cdot \overline 0)|_n)
\big)
\bigvee 
\big(
  (y,x) = 
((a \cdot 0 \cdot \overline 1)|_n,  ( a \cdot 1 \cdot \overline 0)|_n)
\big)
\bigg)
\end{equation}
\end{definition}
%\begin{lemma}
%  For every $n:\mathbb N$, $\text{near}_n$ is an equivalence relation. 
%\end{lemma}
\begin{remark}
Remark that when $x,y$ are near, $m$ and $a$ as above are unique. 
Thus $\text{near}_n(x,y)$ is a proposotion. 
%
Furthermore, to check whether $x,y$ are near, we need only make $n$ comparisons, 
thus $\text{near}_n(x,y)$ is decidable. 
%
Note that in the above definition, we allow $m = n$ and therefore $x$ is near to itself for any finite sequence $x$. 
Furthermore, we have defined nearness to be symmetric. 
However, it is not a transtive relation. 
After all, the sequence $010$ and $011$ are near and the sequence $011$ and $100$ are near, 
but $010$ is not near to $100$. 
\end{remark}
\begin{definition}
  We define the following relation on Cantor space for $\alpha, \beta: 2^\mathbb N$.
  \begin{equation}
    \alpha \sim_t \beta = \forall (n : \mathbb N) 
    \text{near}_n(\alpha|_n, \beta|_n)
  \end{equation}
\end{definition}
\begin{lemma}
  $\sim_t$ is a closed equivalence relation. 
\end{lemma}
\begin{proof}
   Let $\alpha, \beta, \gamma : 2^\mathbb N$. 
   As the dependent product of propositions is a proposition, $\alpha \sim_t\beta$ is a proposition. 
   %
   Furthermore, the closedness follows from decidability of $\text{near}_n(\alpha|_n, \beta|_n)$. 
   One could define $\gamma(n) = 1$ iff $\text{near}_n(\alpha|_n, \beta|_n)$
   
   As nearness is reflexive and symmetric, so is $\sim_t$. 

   Now suppose $\alpha \sim_t \beta$ and $\beta\sim_t \gamma$. 
   We claim that $\alpha \sim_t \gamma$. 

   Let $n:\mathbb N$, we need to show that 
   $\text{near}_n(\alpha|_n , \gamma|_n)$. 
   Let $(a,m)$ witness that $\text{near}_n(\alpha|_n, \beta|_n)$.
   \begin{itemize}
     \item 
   If $m=n$, we have that $\alpha|_n = \beta|_n$, and therefore 
   $\text{near}_n(\beta|_n, \gamma|_n) \leftrightarrow \text{near}_n(\alpha|_n, \gamma|_n)$.
   \item 
     If $m< n$, we have that $\alpha(m+1) \neq \beta(m+1)$, thus 
     $\alpha|_k \neq \beta|_k$ for all $k>m$, 
     but we still have $\text{near}_k(\alpha|_k, \beta|_k)$ for these $k$. 
     Therefore, $\alpha = a \cdot 0 \cdot \overline 1$ and 
  \end{itemize}
   



\end{proof}

\begin{proposition}
  The interval of Cauchy reals is isomorphic to $2^\mathbb N / \sim_t$. 
\end{proposition} 

%\printindex

\printbibliography

\end{document}
