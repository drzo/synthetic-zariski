\begin{lemma}\label{LemDecidableSubsetsDeMorgan}
  For $(A_n)$ a family of decidable subsets, we have
    $
    (\bigcup_{n:\N} A_n)^C
    =
    \bigcap_{n:\N} (A_n^C)
    $ 
    and 
    $
    \bigcup_{n:\N} (A_n^C)
    =  
    (\bigcap_{n:\N} A_n)^C
    $
\end{lemma}

\begin{proof}
\begin{itemize}
  \item 
    Let 
    $x\notin \bigcup_{n:\N} A_n$. 
    Then for every $n:\N$, we cannot have $x\in A_n$
    and thus $x\in A_n^C$ by decidability of $A_n$. 
    Thus $x\in \bigcap_{n:\N} (A_n^C)$. 
    Therefore
    $$
    (\bigcup_{n:\N} A_n)^C
    \subseteq 
    \bigcap_{n:\N} (A_n^C).
    $$ 
  \item 
    Suppose that for every $n:\N$, we have $x\notin A_n$. 
    There does not exist an $n:\N$ with $x\in A_n$. Thus
    $$
    \bigcap_{n:\N} (A_n^C)
    \subseteq 
    (\bigcup_{n:\N} A_n)^C
    $$ 
  \item 
    Suppose there exists some $n$ with $x\in A_n^C$. Then 
    it cannot be the case that $x\in A_m$ for all $m:\N$.
    Thus
    $$
    \bigcup_{n:\N} (A_n^C)
    \subseteq 
    (\bigcap_{n:\N} A_n)^C
    $$ 
  \item 
    Suppose that $x\in (\bigcap_{n:\N} A_n)^C$. 
    Then define the binary sequence $\alpha$ by $\alpha(i) =1$ iff $i$ is the first index such that 
    $x\notin A_i$. This is well-defined as $A_n$ is decidable for all $n:\N$. 
    If $\alpha(i) = 0$ for all $i$, then $x\in A_i$ for all $i$. 
%    and it is the case that $x\in A_n$ for all $n:\N$. 
    Thus under our assumption $x\in (\bigcap_{n:\N} A_n)^C$, 
    we cannot have that $\alpha(i) = 0$ always. 
    By Markov, there then exists an $i$ such that $\alpha(i) = 1$. 
    Thus $x\notin A_i$ for some $i$. We conclude that. 
    $$
    (\bigcap_{n:\N} A_n)^C
    \subseteq
    \bigcup_{n:\N} (A_n^C)
    $$ 
\end{itemize}
\end{proof} 
Note that we only needed decidability for the first and last bullet point, 
and only the last bullet point used countability (and of course Markov's principle). 
