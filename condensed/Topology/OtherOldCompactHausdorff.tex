\begin{definition}
  A Compact Hausdorff space is the quotient of a Stone space by a closed equivalence relation. 
%  We denote $\CHaus$ for the type of compact Hausdorff spaces. 
\end{definition}


\begin{lemma}
Let $S$ is Stone and $\sim $ is closed. 
Then $D\subseteq S/\sim $ is closed iff 
it is the image of the projection map $\pi:S\to S/\sim$ of a closed subset of $S$. 
%it's pre-image under the projection maps $\pi:S\to S /\sim $ is. 
\end{lemma}
\begin{lemma}
%  We always have that if $D\subseteq S/\sim$ is closed, $f^{-1}(D)$ is closed for any map $f$ into $S/\sim$, 
%  in particular for the projection map. 
%
  Suppose $A\subseteq S$ is closed. We will show that $\pi(A)$ is closed. 
  We will define a map $f:S\to \Closed$ respecting $\sim$, such that 
  $f(s) \leftrightarrow \pi(s) \in \pi(A)$.
  Define $f(s) \leftrightarrow \exists_{t:S} (A(t) \wedge (s \sim t))$, 
  which is closed as it is the propositional truncation of the intersection of two closed sets. 
  Also, note that if $s\sim s'$, we have that 
  $f(s) \leftrightarrow f(s')$, hence by univalence $f(s) = f(s')$. 
  Therefore, $f$ induces a map $g:S/\sim \to \Closed$.
  Note that $g(\pi(s))$ iff $\pi(s) \in \pi(A)$. 
  Thus 
  %\rednote{Here I'm hesitant, I miss something} It was the general version of induction principle for quotient. 
  $g(x) \leftrightarrow g(x) \in \pi(A)$. 
  Hence $\pi(A)$ is closed. 

  As $\pi$ is surjective, we have $\pi(\pi^{-1}(D)) = D$ for $D\subseteq S/\sim$. 
  Furthermore, if $D$ is closed, so is $\pi^{-1}(D)$. 
  Thus $D$ is the image of $\pi$ of a closed set iff $D$ is closed. 
\end{lemma}






\begin{lemma}
  Whenever $X$ is compact Hausdorff, $F_0, F_1$ are closed and disjoint, 
  there exist $G_0, G_1$ disjoint clopen such that 
  $F_i \subseteq X - G_{1-i}$ and $G_0 \cup G_1 = X$. 
\end{lemma}






