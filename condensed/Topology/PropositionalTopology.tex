In this section, we will define the types of open and closed propositions. 
These will allow us to define a (synthetic) topology  \cite{SyntheticTopologyLesnik} on any type.
We will later study this topology on Stone types in particular, 
but in this section we will study the type theoretic logic of these propositions.
It is not uncommon \cite{TODO} to define open and closed as countable dis/conjunctions of decidable propositions. 
As we have $2^\N$, this definition can be made internally. 
\begin{definition}
  We define the types $\Open, \Closed$ of open and closed propositions as follows:
  \begin{itemize}
    \item 
    A proposition $P$ is open iff there merely exists some $\alpha:2^\N$ such that 
      $P \leftrightarrow \exists_{n:\mathbb N} \alpha(n) = 0$. 
    \item 
    A proposition $P$ is closed iff there merely exists some $\alpha:2^\N$ such that 
      $P \leftrightarrow \forall_{n:\mathbb N} \alpha(n) = 0$. 
  \end{itemize}
\end{definition}

\begin{remark}\label{rmkOpenClosedNegation}
  The negation of an open proposition is a closed proposition, 
  and by Markov's principle, vice versa. 
%  By Markov's Principle, the negation of a closed proposition is an open proposition.
  Also, by Markov we have $\neg \neg P \to P$ whenever $P$ is open or closed. 
\end{remark}
\begin{lemma}\label{ClosedOpenCountableConDisjunction}
  Closed/Open propositions are closed under countable con-/disjunctions respectively.  
\end{lemma}
\begin{proof}
  Let $(P_n)_{n:\N}$ be a countable family of closed propositions. 
  By countable choice, for each 
  $n:\N$ we have an $\alpha_n:2^\N $ 
  such that $P_n \leftrightarrow \forall_{m:\N} \alpha_n(m)  =0$. 
  Consider a surjection $s:\N \to \N \times \N$.
  Let 
  $$\beta(k) = \alpha_{\pi_0(s(k))}(\pi_1 (s(k))).$$
  Note that $\forall_{k:\N} \beta(k) = 0$ iff 
  $\forall_{m,n:\N}\alpha_m(n) = 0$, which happens iff $\forall_{n:\N} P_n$. 
  Hence the countable conjunction of closed propositions is closed. 
  By a replacing the universal with existential quantifiers, 
  one can show that the countable disjunction of open propositions is open. 
\end{proof} 
\begin{corollary}
  If a proposition is both open and closed, it is decidable. 
\end{corollary}
\begin{proof}
  If $P$ is open and closed, $P\vee \neg P$ is open, hence
  equivalent to $\neg \neg (P \vee \neg P)$, which is provable. 
\end{proof}

\begin{lemma}\label{ClosedDisjunction} 
  Closed/Open propositions are closed under finite dis-/conjunctions respectively. 
\end{lemma}
\begin{proof}
  We shall show that 
  $(\forall_{n:\N} \alpha(n) = 0 )\vee (\forall_{n:\N} \beta(n) = 0 )$ is closed for any $\alpha,\beta:2^\N$.
  By \Cref{corAlternativeLLPO}, the statement is equivalent to 
  $ \forall_{n:\N}  \forall_{m:\N}  (\alpha(n) = 0 \vee \beta(m) = 0)$, 
  which is a countable conjunction of decidable propositions, 
  hence closed by \Cref{ClosedOpenCountableConDisjunction}.
\end{proof}
\begin{lemma}\label{OpenConjunction}
  Open propositions are closed under finite conjunctions. 
\end{lemma}
\begin{proof}
  We need to show that for any $\alpha,\beta:2^\N$, the following proposition is open:
  \begin{equation}\label{eqnConjunctionOpen}
    (\exists_{n:\N} \alpha(n) = 0 )\vee (\exists_{n:\N} \beta(n) = 0 )
  \end{equation}
  Consider $\gamma:2^\N$ given by 
  $\gamma(l) = 1$ iff there exist some $k,k'\leq l$ with 
  $\alpha(k) = \beta(k') = 0$. 
  As we only need to check finitely many combinations 
  of $k,k'$, this is a decidable property for each $l:\N$ and $\gamma$ is well-defined. 
  Clearly, $\gamma$ witnessed that the proposition in \Cref{eqnConjunctionOpen} is open.
\end{proof}

\begin{lemma}
  Closed propositions are closed under $\Pi$-types. 
\end{lemma}
\rednote{TODO}

\begin{lemma}[$\Open$ forms a dominance]
  Open propositions are closed under $\Sigma$-types. 
\end{lemma}
\begin{proof}
  First note that for $D$ a decidable proposition, and $X:D \to \Open$,
  by case splitting on $D$, we can see 
  $\Sigma_{d:D} X(d)$ is open.
%
  Then note that for $P$ an open proposition, 
  there exists a sequence $A_n$ of decidable propositions $A_n$ with 
  $P = \Sigma_{n:\N} A_n $.
%
  So for $Y : P \to Open $, the $\Sigma_P Y$ is given by 
  $\Sigma_{n:\N} (\Sigma_{a:A_n} Y(n,a))$. 
  which is a countable sum of open propositions. 
  As $\Sigma_P Y$ is a proposition, it is 
  a countable disjunction of open propositions, 
  hence open by \Cref{ClosedOpenCountableConDisjunction}.
\end{proof}

