\subsection{Compact Hausdorff types}
\begin{definition}
  A type $X$ is called compact Hausdorff iff there exists some $S:\Stone$ and some 
  equivalence relation $\sim:S \times S \to \Closed$ such that $X \simeq S / \sim$. 
  We denote $\CHaus$ for the type of compact Hausdorff types. 
\end{definition} 

\begin{lemma}\label{CompactHausdorffClosed}
Let $A\subseteq X$ be a subtype of a compact Hausdorff space. 
Let $S, \sim$ be any presentation of $X$. 
Then $A$ is closed iff it is the image of a closed subtype of $S$ under the quotient map. 
\end{lemma}
\begin{proof}
  If $A$ is closed, then it's pre-image under any map is also closed. 
  In particular for $q:S\to X$ the quotient map, $q^{-1}(A)$ is closed. 
  As $q$ is surjective, we have $q(q^{-1}(A)) = A$,
  hence $A$ is the image of a closed subtype of $S$. 
  Conversely, let $B\subseteq S$ be closed. 
%  Then for any $s:S$, the subtype $\{t:S| B(s) \wedge s \sim t\} \subseteq S$ is closed. 
%  Hence by 
  Define $A\subseteq S$ by 
  $$A(s) := \exists_{s:S} (B(t) \wedge s \sim t).$$
  As $B, \sim$ are closed, by \Cref{ClosedCountableConjunction} and \Cref{InhabitedClosedSubSpaceClosed}, 
  we have that $A$  is closed. 
  Also $A$ respects $\sim$, hence induces a map $A': X\to \Closed$.
  Furthermore, $A(q(s))$ iff $q(s)$ is in the image of $B$. 
  Therefore $A'(x)$ iff $x$ is in the image of $B$. 
\end{proof}
\begin{corollary}
  For $X:\CHaus$ a subtype $A\subseteq X$ is closed iff it is the image of 
  a map $T\to X$ for some $T:\Stone$. 
\end{corollary}
\begin{proof}
  Directly from the above and \Cref{StoneClosedSubsets}.
\end{proof}

\begin{corollary}\label{InhabitedClosedSubSpaceClosed}
  For $X:\CHaus$ and $A\subseteq X$ closed, we have 
  $\exists_{x:X} A(x)$ is closed. 
\end{corollary}
\begin{proof}
  Let $A$ be the image of a map map $T\to X$ for $T:\Stone$. 
  Then $\exists_{x:X} A(x) \leftrightarrow ||T||$, which is closed by \Cref{TruncationStoneClosed}
\end{proof}


\begin{corollary}\label{ClosedDependentSums}
  Closed propositions are closed under dependent sums. 
\end{corollary}
\begin{proof}
  Let $P:\Closed$ and $Q:P \to \Closed$. 
  Then $\Sigma_{p:P} Q(p) \leftrightarrow \exists_{p:P} Q(p)$.
  As $P$ is Stone by \Cref{PropositionsClosedIffStone}, it is also compact Hausdorff, thus
  \Cref{InhabitedClosedSubSpaceClosed} gives that $\Sigma_{p:P} Q(p)$ is closed. 
\end{proof}
\begin{remark}
  Analogously to \Cref{OpenTransitive} and \Cref{OpenDominance}, it follows that 
  closedness is transitive and $\Closed$ forms a dominance. 
\end{remark}
\begin{corollary}\label{AllOpenSubspaceOpen}
  For $U\subseteq X$ an open subset of a compact Hausdorff space, we have 
  $\forall_{x:X} U(x)$ open. 
\end{corollary}
\begin{proof}
  As $U$ is open, $\neg U$ is closed. 
  Hence $\exists_{x:X} \neg U(x)$ is closed. 
  Therefore, $\neg (\exists_{x:X} \neg U(x))$ is open. 
  Furthermore, it is equivalent to $\forall_{x:X} \neg \neg U(x)$, 
  which is equivalent to $\forall_{x:X} U(x)$ by \Cref{rmkOpenClosedNegation}.
\end{proof}

\begin{corollary}
  Whenever $X:\Chaus$ and $C_n:S\to \Closed$ closed subsets with $\bigcap_{n:\N} C_n =\emptyset$, there is some $N:\N$ 
  with $\bigcap_{n\leq N} C_n  = \emptyset$. 
\end{corollary}
\begin{proof}
  By \Cref{StoneClosedSubsets}, and \Cref{CompactHausdorffClosed} 
  it is sufficient to prove the above when $C_n$ is decidable and $X:\Stone$.
  In this case we may assume that, 
  $X=Sp(B)$ and $c_n:B$ are such that $C_n = \{x:B\to 2 | x(c_n) = 1\}$. 
  Then the set of maps $B\to 2$ sending all $c_n$ to $1$ is given by 
  $$Sp(B/\{\neg c_n |n:\N\}) \simeq \bigcap_{n:\N} C_n = \emptyset .$$
  Hence 
  $0=1$ in $B/\{\neg c_n|{n:\N}\}$, and there is some $N:\N$ with 
  $\bigvee_{n\leq N} (\neg c_n) = 1$, which also means that 
  $$\emptyset = Sp(B/\{ \neg c_n| n \leq N\})  \simeq \bigcap_{n\leq N} C_n .$$
\end{proof}

\begin{lemma}
  Let $X:\Chaus$ be presented by $S/\sim$. 
  Then $2^X$ is an open sub-Boolean algebra of $2^S$. 
\end{lemma}
Note that we do not claim $2^X$ is countable presented, 
but by \Cref{OpenSubsetEnumerableAreEnumerable}, it will be enumerable. 
\begin{proof}
  Denote $q:S \twoheadrightarrow X$ for the quotient map. 
  This induces an injection of Boolean algebras $2^X \hookrightarrow 2^S$.
  Note that $a:S\to 2$ lies in $2^X$ iff for all $s,t:S$, we have $a(s) = a(t)$ whenever $s\sim t$.
  Note that $a(s) = a(t)$ is decidable and $s\sim t$ is open, hence 
  $(s\sim t) \to (a(s) = a(t))$ is open (\Cref{ImplicationOpenClosed})
  By \Cref{AllOpenSubspaceOpen}, we conclude that 
  $\forall_{s:S} \forall_{t:S} ((s\sim t) \to (a(s) = a(t)))$ is open. 
  Hence $2^S$ is an open subobject of $2^X$. 
\end{proof}
