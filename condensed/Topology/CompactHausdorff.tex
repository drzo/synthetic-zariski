\subsection{Compact Hausdorff types}
\begin{definition}
  A type $X$ is called compact Hausdorff iff there exists some $S:\Stone$ and some 
  equivalence relation $\sim:S \times S \to \Closed$ such that $X \simeq S / \sim$. 
  We denote $\CHaus$ for the type of compact Hausdorff types. 
\end{definition} 

\begin{lemma}\label{CompactHausdorffClosed}
Let $A\subseteq X$ be a subtype of a compact Hausdorff space. 
Let $S, \sim$ be any presentation of $X$. 
Then $A$ is closed iff it is the image of a closed subtype of $S$ under the quotient map. 
\end{lemma}
\begin{proof}
  If $A$ is closed, then it's pre-image under any map is also closed. 
  In particular for $q:S\to X$ the quotient map, $q^{-1}(A)$ is closed. 
  As $q$ is surjective, we have $q(q^{-1}(A)) = A$,
  hence $A$ is the image of a closed subtype of $S$. 
  
  Conversely, let $B\subseteq S$ be closed. 
%  Then for any $s:S$, the subtype $\{t:S| B(s) \wedge s \sim t\} \subseteq S$ is closed. 
%  Hence by 
  Then any $t:X$ belongs to the image of $B$ iff we have that:
  $$ \exists_{s:S} B(s)\land q(s) = t$$
 and by \Cref{ClosedCountableConjunction} and \Cref{InhabitedClosedSubSpaceClosed}, 
  we have that this is closed. 
  \end{proof}

\begin{remark}\label{InhabitedClosedSubSpaceClosedCHaus}
  Let $X:\Chaus$. We have that a subset of $X$ is closed if and only if it is the image of a Stone space. Indeed \Cref{CompactHausdorffClosed} gives the direct proof, and the converse follows from \Cref{InhabitedClosedSubSpaceClosed}
  
  Moreover if $A$ is closed in $X$, considering $q:S\to X$ surjective with $S$ Stone we have that:
  $$ \exists_{s:S}A(q(s)) = \exists_{x:X}A(x)$$
  so that from \Cref{InhabitedClosedSubSpaceClosed} we have that $\exists_{x:X} A(x)$ is closed. 
\end{remark}
%\begin{corollary}
%  For $X:\CHaus$ a subtype $A\subseteq X$ is closed iff it is the image of 
%  a map $T\to X$ for some $T:\Stone$. 
%\end{corollary}
%\begin{proof}
%  Directly from the above and \Cref{StoneClosedSubsets}.
%\end{proof}
\begin{remark}
  It is not the case that every closed subset of a compact Hausdorff space can be written 
  as countable intersection of decidable subsets. 
  In \Cref{UnitInterval}, we shall introduce the unit interval $[0,1]$ as a compact Hausdorff space with many closed 
  subsets, but only two decidable subsets. 
  In \Cref{ConnectedComponent}, we shall actually see that whenever every singleton of a compact Hausdorff space $X$
  can be written as countable intersection of decidable subsets, $X$ is Stone. 
  \rednote{Actually, we'll see that $Sp(2^X)$ and $X$ are bijective sets, 
    which only implies that $X$ is Stone if $2^X:\Boole$, but this depends on our definition of countable, 
see \Cref{CountabilityDiscussion}}
\end{remark}


\begin{corollary}\label{AllOpenSubspaceOpen}
  For $U\subseteq X$ an open subset of a compact Hausdorff space, we have 
  $\forall_{x:X} U(x)$ open. 
\end{corollary}
\begin{proof}
  As $U$ is open, $\neg U$ is closed. 
  Hence $\exists_{x:X} \neg U(x)$ is closed. 
  Therefore, $\neg (\exists_{x:X} \neg U(x))$ is open. 
  Furthermore, it is equivalent to $\forall_{x:X} \neg \neg U(x)$, 
  which is equivalent to $\forall_{x:X} U(x)$ by \Cref{rmkOpenClosedNegation}.
\end{proof}

\begin{lemma}\label{CHausFiniteIntersectionProperty}
  Whenever $X:\Chaus$ and $C_n:S\to \Closed$ closed subsets with $\bigcap_{n:\N} C_n =\emptyset$, there is some $N:\N$ 
  with $\bigcap_{n\leq N} C_n  = \emptyset$. 
\end{lemma}
\begin{proof}
  By \Cref{StoneClosedSubsets}, and \Cref{CompactHausdorffClosed} 
  it is sufficient to prove the above when $C_n$ is decidable and $X:\Stone$.
  In this case we may assume that, 
  $X=Sp(B)$ and $c_n:B$ are such that $C_n = \{x:B\to 2 | x(c_n) = 1\}$. 
  Then the set of maps $B\to 2$ sending all $c_n$ to $1$ is given by 
  $$Sp(B/\{\neg c_n |n:\N\}) \simeq \bigcap_{n:\N} C_n = \emptyset .$$
  Hence 
  $0=1$ in $B/\{\neg c_n|{n:\N}\}$, and there is some $N:\N$ with 
  $\bigvee_{n\leq N} (\neg c_n) = 1$, which also means that 
  $$\emptyset = Sp(B/\{ \neg c_n| n \leq N\})  \simeq \bigcap_{n\leq N} C_n .$$
\end{proof}

\begin{lemma}\label{CHausSeperationOfClosedByOpens}
  Let $X:\CHaus$, and let $A,B:X\to \Closed$ be disjoint. 
  Then there exist $U,V:X\to \Open$ disjoint with $A\subseteq U, B\subseteq V$, 
  and $B\cap U = A \cap V = \emptyset$. 
\end{lemma}
\begin{proof}
  Let $q:S\to X$ be a projection map presenting $X$.
  As $q^{-1}(A), q^{-1}(B)$ are closed, 
  by \Cref{StoneSeperated}, there is some $D:S \to 2$ such that
  $q^{-1}(A) \subseteq D, q^{-1}(B) \subseteq \neg D$. 
  Note that $q(D), q(\neg D)$ are closed by \Cref{CompactHausdorffClosed}. 
  Furthermore, as $q^{-1}(A) \cap \neg D  =\emptyset$, we have that 
  $A\subseteq \neg q (\neg D)$. As $A\cap B = \emptyset$, we have that 
  $A\subseteq \neg q (\neg D) \cap \neg B:= U$.
  Similarly, $B\subseteq \neg  q (D) \cap \neg A:= V$. 
  By definition, $U,V$ are as required. 
\end{proof}

\subsection{When compact Hausdorff spaces are Stone}\label{ConnectedComponent}
\begin{lemma}
  Let $X:\Chaus$ be presented by $S/\sim$. 
  Then $2^X$ is an open sub-Boolean algebra of $2^S$. 
\end{lemma}
Note that we do not claim $2^X$ is countable presented, 
but by \Cref{OpenSubsetEnumerableAreEnumerable}, it will be enumerable. 
\begin{proof}
  Denote $q:S \twoheadrightarrow X$ for the quotient map. 
  This induces an injection of Boolean algebras $2^X \hookrightarrow 2^S$.
  Note that $a:S\to 2$ lies in $2^X$ iff for all $s,t:S$, we have $a(s) = a(t)$ whenever $s\sim t$.
  Note that $a(s) = a(t)$ is decidable and $s\sim t$ is closed, hence 
  $(s\sim t) \to (a(s) = a(t))$ is open (\Cref{ImplicationOpenClosed})
  By \Cref{AllOpenSubspaceOpen}, we conclude that 
  $\forall_{s:S} \forall_{t:S} ((s\sim t) \to (a(s) = a(t)))$ is open. 
  Hence $2^S$ is an open subobject of $2^X$. 
\end{proof}

\begin{definition}
  Let $X:\Chaus$ and $x:X$. 
  We define the connected component of $x$ (denoted $Q_x$)
  as the intersection of all decidable subsets of $X$ containing $x$. 
\end{definition}

\begin{corollary}
  For $X:\Chaus$ and $x:X$, there is a surjection from $\N$ to the decidable subsets of $X$ containing $x$. 
  % connected component $Q_x$ is a countable intersection of decidable subsets of $X$. 
\end{corollary}
\begin{proof}
  By the above, %there are enumarably many decidable subsets of $X$. 
  there is an enumeration $s: \N \to (1 + (X \to 2))$ of all decidable subsets of $X$. 
  Define $B_{(\cdot)}:\N \to (X \to 2)$ by 
  $$B_n = \begin{cases}
    X \text { if } s(n) = inl(*) \\
    A \text { if } s(n) = inr(A) \text { and } A(x) \\
    X \text { if } s(n) = inr(A) \text { and } \neg A(x)
  \end{cases}
  $$
  Clearly, $B_{(\cdot)}$ hits all decidable subsets of $X$ containing $x$. 
\end{proof}

\begin{lemma}\label{ConnectedComponentSubOpenHasDecidableInbetween}
  Let $X:\Chaus, x:X$ and suppose $U\subseteq X$ is open with $Q_x\subseteq U$. 
  Then we have some decidable $E\subseteq X$ with $E(x)$ and $E\subseteq U$. 
\end{lemma}
\begin{proof}
  By the above, we have $Q_x = \bigcap_{n:\N}B_n$ with $x\in B_n$. 
  If $Q_x \subseteq U$, we have that 
  $$Q_x\cap \neg U = \bigcap_{n:\N} (E_n \cap \neg U)$$ is empty. 
  By \Cref{CHausFiniteIntersectionProperty} there is some $N:\N$ with 
  $$(\bigcap_{n\leq N} E_n )\cap \neg U  = \bigcap_{n\leq N} (E_n \cap \neg U) = \emptyset.$$
  Therefore $\bigcap_{n\leq N} E_n \subseteq U$, furthermore a finite intersection of decidable subsets is decidable. 
  As $x\in E_n$ for all $n:\N$, $x\in \bigcap_{n\leq N} E_n$ as well and we're done. 
\end{proof}

\begin{lemma}
Let $X:\Chaus, x:X$ and $A,B:X \to \Closed$ be disjoint and such that $Q_x = A \cup B$. 
Then one of $A,B$ is empty. 
\end{lemma} 
\begin{proof}
TODO
\end{proof}

%\begin{theorem}
%  Let $X:\CHaus$, then $X \simeq Sp(2^X)$ if for all $x:X$, we have $\{x\} = Q_x$. 
%\end{theorem}
%  Note that $\{x\}$ is closed. 
%  By \Cref{StoneClosedSubsets}, if $X$ is Stone, $\{x\}$ must be a countable intersection of
%  decidable subsets, hence $\{x\} = Q_x$. 
%  \rednote{Thus if $2^X:\Boole$ always, the if above becomes an iff.}
%\begin{proof}
%%  We claim that the map $X \to Sp(2^X)$ is surjective. 
%  We claim that $(2^X \to 2) \to X$ is a presentation of $X$. 
%
%
%  Furthermore, if $\{x\} = Q_x$ for all $x:X$, it is also injective. 
%  Let $f:2^X \to 2$ be a Boolean map.
%  Then 
%\end{proof}
