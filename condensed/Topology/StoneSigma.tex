\subsection{Stone spaces are stable under dependent sums}

\rednote{TODO this is a mess right now}

\begin{lemma}
  Let $X:\Chaus$ be presented by $S/\sim$. 
  Then $2^X$ is an open sub-Boolean algebra of $2^S$. 
\end{lemma}
Note that we do not claim $2^X$ is countable presented, 
but by \Cref{OpenSubsetEnumerableAreEnumerable}, it will be enumerable. 
\begin{proof}
  Denote $q:S \twoheadrightarrow X$ for the quotient map. 
  This induces an injection of Boolean algebras $2^X \hookrightarrow 2^S$.
  Note that $a:S\to 2$ lies in $2^X$ iff for all $s,t:S$, we have $a(s) = a(t)$ whenever $s\sim t$.
  Note that $a(s) = a(t)$ is decidable and $s\sim t$ is closed, hence 
  $(s\sim t) \to (a(s) = a(t))$ is open (\Cref{ImplicationOpenClosed})
  By \Cref{AllOpenSubspaceOpen}, we conclude that 
  $\forall_{s:S} \forall_{t:S} ((s\sim t) \to (a(s) = a(t)))$ is open. 
  Hence $2^S$ is an open subobject of $2^X$. 
\end{proof}

\begin{definition}
  Let $X:\Chaus$ and $x:X$. 
  We define the connected component of $x$ (denoted $Q_x$)
  as the intersection of all decidable subsets of $X$ containing $x$. 
\end{definition}

\begin{corollary}
  For $X:\Chaus$ and $x:X$, there is a surjection from $\N$ to the decidable subsets of $X$ containing $x$. 
  % connected component $Q_x$ is a countable intersection of decidable subsets of $X$. 
\end{corollary}
\begin{proof}
  By the above, %there are enumarably many decidable subsets of $X$. 
  there is an enumeration $s: \N \to (1 + (X \to 2))$ of all decidable subsets of $X$. 
  Define $B_{(\cdot)}:\N \to (X \to 2)$ by 
  $$B_n = \begin{cases}
    X \text { if } s(n) = inl(*) \\
    A \text { if } s(n) = inr(A) \text { and } A(x) \\
    X \text { if } s(n) = inr(A) \text { and } \neg A(x)
  \end{cases}
  $$
  Clearly, $B_{(\cdot)}$ hits all decidable subsets of $X$ containing $x$. 
\end{proof}

\begin{lemma}\label{ConnectedComponentSubOpenHasDecidableInbetween}
  Let $X:\Chaus, x:X$ and suppose $U\subseteq X$ is open with $Q_x\subseteq U$. 
  Then we have some decidable $E\subseteq X$ with $E(x)$ and $E\subseteq U$. 
\end{lemma}
\begin{proof}
  By the above, we have $Q_x = \bigcap_{n:\N}B_n$ with $x\in B_n$. 
  If $Q_x \subseteq U$, we have that 
  $$Q_x\cap \neg U = \bigcap_{n:\N} (E_n \cap \neg U)$$ is empty. 
  By \Cref{CHausFiniteIntersectionProperty} there is some $N:\N$ with 
  $$(\bigcap_{n\leq N} E_n )\cap \neg U  = \bigcap_{n\leq N} (E_n \cap \neg U) = \emptyset.$$
  Therefore $\bigcap_{n\leq N} E_n \subseteq U$, furthermore a finite intersection of decidable subsets is decidable. 
  As $x\in E_n$ for all $n:\N$, $x\in \bigcap_{n\leq N} E_n$ as well and we're done. 
\end{proof}

%\begin{lemma}
%Let $X:\Chaus, x:X$ and $A,B:X \to \Closed$ be disjoint and such that $Q_x = A \cup B$. 
%Then one of $A,B$ is empty. 
%\end{lemma} 
%\begin{proof}
%TODO
%\end{proof}

%\begin{theorem}
%  Let $X:\CHaus$, then $X \simeq Sp(2^X)$ if for all $x:X$, we have $\{x\} = Q_x$. 
%\end{theorem}
%  Note that $\{x\}$ is closed. 
%  By \Cref{StoneClosedSubsets}, if $X$ is Stone, $\{x\}$ must be a countable intersection of
%  decidable subsets, hence $\{x\} = Q_x$. 
%  \rednote{Thus if $2^X:\Boole$ always, the if above becomes an iff.}
%\begin{proof}
%%  We claim that the map $X \to Sp(2^X)$ is surjective. 
%  We claim that $(2^X \to 2) \to X$ is a presentation of $X$. 
%
%
%  Furthermore, if $\{x\} = Q_x$ for all $x:X$, it is also injective. 
%  Let $f:2^X \to 2$ be a Boolean map.
%  Then 
%\end{proof}
