\subsection{The topology of Stone spaces}
\rednote{Active section}
\begin{definition}
  For $S$ a set and $D\subseteq S$ an arbitrary subset, we call $D$ decidable/open/closed 
  iff for every $x:S$, $D(x)$ is decidable/open/closed. 
\end{definition}
\begin{remark}
  \begin{itemize}
    \item   \Cref{rmkOpenClosedNegation} implies that the complement of an 
  open subset is closed and vice versa. 
  \item 
    \Cref{OpenAndClosedMeansDecidable} implies that a subset which is both open 
    and closed is decidable. 
  \item 
    As propositions are subsets of $\top = Sp(2)$, 
    we have that open/closed/decidable subsets of $\top$
    are open/closed/decidable propositions. 
  \item By \Cref{EqualityBooleStoneClosedOpen}, singletons in 
    Boolean algebras are open and singletons in Stone spaces are closed. 
\end{itemize}
\end{remark}

%%I moved a lot of stuff around, the following is true, but unnecessary
%% (I thought we needed to add merely in the big circular theorem but we don't. 
%%    \item We will show that $(ii)$ implies $(i)$.
%%      Let $f:Sp(B) \to Sp(B')$
%%      correspond to $u:B'\to B$. 
%%      We know that $u$ factors through it's kernel as 
%%      $B' \twoheadrightarrow B'/Ker(u) \hookrightarrow B$. 
%%      
%%      As equality in $B$ is open, and $B'$ is countable, we can enumerate all 
%%      $e:B'$ with $u(e) = 1_B$. Let $(e_n)_{n:\N}$ be this enumeration. 
%%      Let $E_n\subseteq Sp(B')$ be the corresponding decidable subsets. 
%%
%%      We will show that $f(Sp(B)) = \bigcap_{n:\N} E_n$. 
%%
%%      Let $x\in \bigcap_{n:\N} E_n\subseteq Sp(B') $. 
%%      Then $x(e) = 1_2$ whenever $u(e) = 1_B$. 
%%      In particular, whenever $b:B'$ is such that $u(b) = 0_B$, 
%%      we have $u(1-b) = 1_B$ thus $x(1-b) = 1_2$ and hence $x(b) = 1_B$. 
%%      Therefore for all $b:Ker(u)$ we have $x(b) = 0_B$. 
%%      By remark \Cref{rmkMorphismsOutOfQuotient}, 
%%      $x:B'\to 2$ defines a morphism $y:B'/Ker(u) \to 2$, 
%%      such that the left triangle in \Cref{eqnDecidableIntersectionExtension} commutes.
%%      Furthermore, as $B'/Ker(u) \hookrightarrow B$ is injective, 
%%      as formal surjections are surjective, 
%%      there is a surjection $Sp(B) \twoheadrightarrow Sp(B'/Ker(u))$. 
%%      Hence there merely is some $z:B\to 2$ making the right triangle in 
%%      \Cref{eqnDecidableIntersectionExtension} commute. 
%%      But now $z\circ u = x$ and hence $f(z) = x$. 
%%      We conclude that there merely is some $z:Sp(B)$ such that $fz = x$. 
%%      Thus $x\in f(Sp(B))$. 
%%    % https://q.uiver.app/#q=WzAsNCxbMCwxLCJCJyJdLFsxLDAsIjIiXSxbMSwxLCJCJy9LZXIodSkiXSxbMiwxLCJCIl0sWzAsMSwieCJdLFswLDIsIiIsMix7InN0eWxlIjp7ImhlYWQiOnsibmFtZSI6ImVwaSJ9fX1dLFsyLDMsIiIsMix7InN0eWxlIjp7InRhaWwiOnsibmFtZSI6Imhvb2siLCJzaWRlIjoidG9wIn19fV0sWzIsMSwieSIsMix7InN0eWxlIjp7ImJvZHkiOnsibmFtZSI6ImRvdHRlZCJ9fX1dLFszLDEsInoiLDIseyJzdHlsZSI6eyJib2R5Ijp7Im5hbWUiOiJkYXNoZWQifX19XV0=
%%      \begin{equation}\label{eqnDecidableIntersectionExtension}\begin{tikzcd}
%%        & 2 \\
%%        {B'} & {B'/Ker(u)} & B
%%        \arrow["x", from=2-1, to=1-2]
%%        \arrow[two heads, from=2-1, to=2-2]
%%        \arrow["y"', dotted, from=2-2, to=1-2]
%%        \arrow[hook, from=2-2, to=2-3]
%%        \arrow["z"', dashed, from=2-3, to=1-2]
%%      \end{tikzcd}\end{equation}
%%
%%      Conversely, whenever $x\in f(Sp(B))$, there is some 
%%      $z:B\to 2$ with $z\circ u = x$. 
%%      Thus for all $e:B'$ with $u(e) = 1_B$, we have 
%%      $x(e) = z\circ u ( e) = z (1_B) = 1_2$. 
%%      Hence $x\in E$ for $E$ the decidable set corresponding to $e$. 
%%      Thus for every $n:\N$ we have $x\in E_n$. 
%%      Thus $x\in \bigcap_{n:\N} E_n$. 
%%
%%      Therefore $f(Sp(B)) = \bigcap_{n:\N} E_n$, 
%%      and thus the image of a map between Stone spaces is a countable intersection 
%%      of decidable subsets. 
% Inhabited open subset of enumerable is enumerable. 
% We could also define $A$ enumerable as there existing a surjection $\N \to 1 + A$, 
% then you don't need this inhabited condition.
\begin{lemma}
  Let $A$ be a countable set and let $P:A \to Open$
  be an open subset of $A$. 
  If $||\sum_{a:A} P a||$, then $\sum_{a:A} P a$ is a countable set. 
\end{lemma}
\begin{proof}
  Being countable is a proposition, so we may untruncate our assumptions:
  \begin{itemize}
    \item 
      Because we have $||\sum_{a:A} P a||$, we assume we have  some $x:A$ with $P x$. 
    \item 
      Because $A$ is countable, we may assume we have some 
      enumeration of elements from $A$ given as $(a_n)_{n:\N}$.
  \end{itemize}
  Furthermore, by countable choice, $P$ gives us 
  for each $a:A$ a specific $\alpha_a:2^\N$ 
  such that 
  \begin{equation}
    P a \leftrightarrow \exists_{m:\N} \alpha_a(m) = 1.
  \end{equation}
  Using this data, we will 
  construct a surjection $s:\N \times \N \to \sum_{a:A} P a$. 
  Define 
  \begin{equation}
    s(m,n) = 
    \begin{cases}
      x \text{ if } \alpha_{a_n}(m) = 0 \\
      a_n \text{ if } \alpha_{a_n}(m) = 1
    \end{cases}
  \end{equation}
  To see that $s$ is surjective, we need to show that for each $a:A$ with $P a$ there 
  merely exists some $(m,n):\N \times \N$ 
  with $s(m,n) = a$. 
  \begin{itemize}
    \item 
  Because $(a_n)_{n:\N}$ formed an enumeration, we have that 
  there merely is some $n$ with $a = a_n$. 
  We can use this $n$ as we're proving a proposition. 
    \item 
  As $P a$ there merely is some $m$ with $\alpha_{a_n}(m) = 1$. 
  This $m$ can also be used as we're proving a proposition. 
  \end{itemize}
  Now by the above choice $s(m,n) = a_n = a$, 
  thus for each $a:A$ with $P a$, there exists some $(m,n)$ with 
  $s(m,n) = a$. Thus $s$ is surjective. 

  As $\N \times \N \simeq \N$, we conclude that 
  $\sum_{a:A} P a$ is countable. 
\end{proof}
%\begin{corollary}
%  As countably presented Boolean algebras are countable  
%  and equality in a countably presented Boolean algebra is open, 
%  the kernel of a map between countably presented Boolean algebras is countable. 
%\end{corollary}


%\begin{lemma}\label{lemDecidableEv}
%  Let $D\subseteq Sp(B)$. 
%  Then $D$ is decidable iff we have a $b\in B$ such that 
%  $D = \{x:Sp(B) | x(b) = 1\}$. 
%\end{lemma}
%\begin{proof}
%  Let $D\subseteq Sp(B)$ be decidable. 
%  Define $f_D:2^{Sp(B)}$ by $f_D(x) = 1$ if $D(x)$ and $f_D(x) = 0$ if $\neg D(x)$.
%  Stone duality then gives there is some $b_D:B$ with 
%  $f_D(x) = 1 \leftrightarrow x(b_D) = 1$. 
%  Thus $D(x) \leftrightarrow f_D(x) = 1 \leftrightarrow x(b_D) = 1$ as requried. 
%
%  For the converse, suppose $D  = \{x : Sp(B) | x(b) = 1\}$ then 
%  whenever $y:Sp(B)$, we have $D(y)$ iff $y(b) = 1$, which is decidable
%  as for any $c:2$ we have $c = 0 \vee c = 1$. 
%  Thus $D$ is decidable. 
%\end{proof}
%If $b,D$ are as in the above lemma, we say $D$ corresponds to $b$. 
%
%\begin{lemma}
%  Let $f:S \to T$ be a map of Stone spaces. 
%  Then the image $f(S) \subseteq T$ is a countable 
%  intersection of decidable subsets. 
%\end{lemma}
%\begin{proof}
%  Let $f:Sp(B) \to Sp(B')$
%  correspond to $u:B'\to B$. 
%  We know that $u$ factors through it's kernel as 
%  $B' \twoheadrightarrow B'/Ker(u) \hookrightarrow B$. 
%  
%  As equality in $B$ is open, and $B'$ is countable, we can enumerate all 
%  $e:B'$ with $u(e) = 1_B$. Let $(e_n)_{n:\N}$ be this enumeration. 
%  Let $E_n\subseteq Sp(B')$ be the corresponding decidable subsets. 
%
%  We will show that $f(Sp(B)) = \bigcap_{n:\N} E_n$. 
%
%  Let $x\in \bigcap_{n:\N} E_n\subseteq Sp(B') $. 
%  Then $x(e) = 1_2$ whenever $u(e) = 1_B$. 
%  In particular, whenever $b:B'$ is such that $u(b) = 0_B$, 
%  we have $u(1-b) = 1_B$ thus $x(1-b) = 1_2$ and hence $x(b) = 1_B$. 
%  Therefore for all $b:Ker(u)$ we have $x(b) = 0_B$. 
%  By remark \Cref{rmkMorphismsOutOfQuotient}, 
%  $x:B'\to 2$ defines a morphism $y:B'/Ker(u) \to 2$, 
%  such that the left triangle in \Cref{eqnDecidableIntersectionExtension} commutes.
%  Furthermore, as $B'/Ker(u) \hookrightarrow B$ is injective, 
%  as formal surjections are surjective, 
%  there is a surjection $Sp(B) \twoheadrightarrow Sp(B'/Ker(u))$. 
%  Hence there merely is some $z:B\to 2$ making the right triangle in 
%  \Cref{eqnDecidableIntersectionExtension} commute. 
%  But now $z\circ u = x$ and hence $f(z) = x$. 
%  We conclude that there merely is some $z:Sp(B)$ such that $fz = x$. 
%  Thus $x\in f(Sp(B))$. 
%% https://q.uiver.app/#q=WzAsNCxbMCwxLCJCJyJdLFsxLDAsIjIiXSxbMSwxLCJCJy9LZXIodSkiXSxbMiwxLCJCIl0sWzAsMSwieCJdLFswLDIsIiIsMix7InN0eWxlIjp7ImhlYWQiOnsibmFtZSI6ImVwaSJ9fX1dLFsyLDMsIiIsMix7InN0eWxlIjp7InRhaWwiOnsibmFtZSI6Imhvb2siLCJzaWRlIjoidG9wIn19fV0sWzIsMSwieSIsMix7InN0eWxlIjp7ImJvZHkiOnsibmFtZSI6ImRvdHRlZCJ9fX1dLFszLDEsInoiLDIseyJzdHlsZSI6eyJib2R5Ijp7Im5hbWUiOiJkYXNoZWQifX19XV0=
%  \begin{equation}\label{eqnDecidableIntersectionExtension}\begin{tikzcd}
%    & 2 \\
%    {B'} & {B'/Ker(u)} & B
%    \arrow["x", from=2-1, to=1-2]
%    \arrow[two heads, from=2-1, to=2-2]
%    \arrow["y"', dotted, from=2-2, to=1-2]
%    \arrow[hook, from=2-2, to=2-3]
%    \arrow["z"', dashed, from=2-3, to=1-2]
%  \end{tikzcd}\end{equation}
%
%  Conversely, whenever $x\in f(Sp(B))$, there is some 
%  $z:B\to 2$ with $z\circ u = x$. 
%  Thus for all $e:B'$ with $u(e) = 1_B$, we have 
%  $x(e) = z\circ u ( e) = z (1_B) = 1_2$. 
%  Hence $x\in E$ for $E$ the decidable set corresponding to $e$. 
%  Thus for every $n:\N$ we have $x\in E_n$. 
%  Thus $x\in \bigcap_{n:\N} E_n$. 
%
%  Therefore $f(Sp(B)) = \bigcap_{n:\N} E_n$, 
%  and thus the image of a map between Stone spaces is a countable intersection 
%  of decidable subsets. 
%\end{proof}

\begin{theorem}
  Let $A\subseteq Sp(B)$ be a subset of a Stone space. TFAE:
  \begin{enumerate}[(i)]
    \item There merely exists some $(D_n)_{n:\N}$ countable family 
      of decidable subsets of $Sp(B)$ with $A = \bigcap_{n:\N} D_n$. 
    \item There merely exists some Stone space $T$ and some map $T\to Sp(B)$ 
      whose image is $A$. 
    \item $A$ is closed. 
  \end{enumerate}
\end{theorem}
\begin{proof}
  \begin{itemize}
    \item To see that $(i) \to (ii)$, consider \Cref{LemdecidableEv}
  \end{itemize}
\end{proof} 




%
%
%\begin{lemma}
%  Let $S$ Stone. 
%  Then a subset $D\subseteq S$ is closed iff there merely is some $T$ Stone 
%  and map $f:T\to S$ with image $D$. 
%\end{lemma}
%\begin{proof}
%  Let $f:T\to S$ be a map between Stone spaces. 
%  We will show that $f(T)$ is closed. 
%  For $x:S$, we have that $x\in f(T)$ iff there exists some $y:T$ 
%  with $f(y) = x$. 
%  So 
%  \begin{equation}
%    (f(T))(x) \leftrightarrow || \sum_{y:T} f(y) = x||
%  \end{equation}
%  Note that $\sum_{y:T} f(y) = x$ is a pullback as follows:
%  \begin{equation}
%    \begin{tikzcd}
%      \sum_{y:T} f(y) = x\arrow[rd,"\lrcorner"{pos=0.125},phantom] \arrow[r] \arrow[d]
%      & 1 \arrow[d,"x"] \\
%      T \arrow[r,"f"']  & S
%    \end{tikzcd}
%  \end{equation}
%  As any pullback of Stone spaces is Stone, it follows that 
%  $\sum_{y:T} f(y) = x$ is Stone, and by \Cref{LemInhabitedOfStoneIsClosed}, 
%  so is $||\sum_{y:T} f(y) = x||$. Thus $f(T)(x)$ is a closed proposition and $f(T)$ closed. 
%  We conclude that the image of a map of Stone spaces is closed. 
%
%  For the converse, consider any $D\subseteq Sp(B)$ closed. 
%  We will find a countably generated ideal $\langle R \rangle \subseteq B$ such that 
%  the quotient map $B \to B /\langle R \rangle$ induces a map 
%  $Sp(B/\langle R \rangle) \to Sp(B)$ whose image is $D$. 
%  \rednote{TODO}
%\end{proof}
%
%
%
%\begin{lemma}
%  Let $A\subseteq S$ be a subset of a Stone space. TFAE
%  \begin{itemize}
%    \item $A$ is closed. 
%    \item There merely is a countable set set of decidable subsets $D_n\subseteq S$ with 
%      $A = \bigcap_{n:\N} D_n$. 
%  \end{itemize}
%\end{lemma}
%\begin{proof}
%  We will first show that $\bigcap_{n:\N} D_n$ is closed for $(D_n)_{n:\N}$ 
%  a family of decidable subsets of $S$. 
%  As $D_n$ is decidable, we can define for each $x:S$ a function 
%  $\alpha_x:2^\N$ by 
%  \begin{equation}
%    \alpha_x(n) = 
%    \begin{cases} 
%      0 \text { if } D_n(x)\\
%      1 \text { if } \neg D_n(x)
%    \end{cases}
%  \end{equation} 
%  Now $x\in \bigcap_{n:\N} D_n $ iff $\alpha_x(n) = 0$ for all $n:\N$. 
%  Thus $\bigcap_{n:\N} D_n$ is closed. 
%  \medskip
%
%  For the converse, 
%  Suppose $A$ is closed. So for each $x: S$ there merely exists some 
%  $\alpha:2^\N$ with $A x \leftrightarrow (\forall_{n:\N}\alpha(n) = 0$. 
%  Let $E$ be a dependent family of types over $X$ given by
%  $E x  = \sum\limits_{\alpha:2^\N} 
%  (A x \leftrightarrow \forall_{n:\N} \alpha(n) = 0)$
%  The projection map $\sum\limits_{x: S} E x \to S$ is surjective by our assumption, 
%  thus local choice gives us some Stone space $T$ and arrows as follows:
%  \begin{equation}\begin{tikzcd}
%    \sum\limits_{x:S}E x \arrow[d,""',two heads]\\
%    S & \arrow[l, "", two heads,"s"] T\arrow[lu, ""']
%  \end{tikzcd}\end{equation}  
%  The map $T\to \sum\limits_{x:S} E x $ induces a map $\gamma_{(\cdot)}:T \to 2^\N$, 
%  such that for all $t:T$ we have 
%  \begin{equation}
%    (\forall_{n:\N} \gamma_t(n)  = 0 )\leftrightarrow A ( s (t))
%  \end{equation}
%  Now for $T = Sp(B)$, under Stone duality the map $\gamma_{(\cdot )}$ 
%  corresponds with a map $\langle C \rangle \to B$, 
%  which gives a map $\N \to B$. 
%
%
%
%\end{proof}
%
%
%
%
%\begin{corollary}\label{CorDoubleNegToAx2}
%  If we have for $S:\Stone$ that $\neg \neg S \to ||S||$, 
%  we can conclude that surjections are formal surjections. 
%\end{corollary}
%\begin{proof}
%  \rednote{All the unkonw references are lemmas that still need to be done (or at least linked) without using ax2}
%  Now assume that for $S:\Stone$, we have $\neg \neg S \to ||S||$. 
%  Let $f:B \to B'$ be injective, and let 
%  $g:S' \to S$ correspond to $f$. 
%  To show that $g$ is surjective, we need to show that for any $x:S$, we have $||\sum\limits_{y:S'} g y = x||$. 
%  By our assumption, we can prove this by showing that 
%  \begin{equation}
%    g^{-1}\{x\} := \sum\limits_{y:S'} g y = x
%  \end{equation} is Stone and non-empty. 
%    First note that $\sum\limits_{y:S'} g y = x$ is the pullback of $g$ and the map $1 \to S$ given by $x$.
%      By duality and \Cref{rmkBoolePushouts} the pullback of Stone spaces is a Stone space. 
%   
%
%      By \Cref{lemPointsAreClosed} and \Cref{lemClosedIntersectionOfDecidable}, 
%      $\{x\}$ is a countable intersection of decidable sets $\bigcap_{n:\N} D_n$. 
%      We have that 
%      \begin{equation}
%        g^{-1}\{x\} = g^{-1}(\bigcap_{n:\N} D_n) = 
%        \bigcap_{n:\N} g^{-1}(D_n)
%      \end{equation} 
%      Let $n:\N$. By Stone duality, the decidable subset $D_n$ corresponds to some $b_n:B$. 
%      As $x\in D_n$, we have that $b_n \neq 0$. 
%      As $f$ is injective, we have that $f(b_n) \neq 0$. 
%      Therefore the corresponding decidable subset, which is $g^{-1}(D_n)$ is not empty. 
%
%      Note that $D_n$ is a closed subset of a Stone space, hence Stone by \Cref{LemClosedSubsetOfStoneIsStone}, 
%      and by the same argument as above $g^{-1}(D_n)$ is Stone. By our assumption $g^{-1}(D_n)$ is thus merely inhabited. 
%      Now by \Cref{lemCompactnessCountableIntersection}, this gives that 
%      $\bigcap_{n:\N} g^{-1}(D_n)$ is merely inhabited, as required. 
%\end{proof}
%
