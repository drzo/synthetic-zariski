\subsection{Topology on propositions}
\begin{definition}
  Let $P$ be a proposition. 
  \begin{itemize}
    \item $P$ is decidable if $P + \neg P$
    \item $P$ is open if there merely exists some $\alpha:\Noo$ such that $P \leftrightarrow \alpha \neq \infty$. 
    \item $P$ is closed if there merely exists some $\alpha:\Noo$ such that $P \leftrightarrow \alpha = \infty$. 
  \end{itemize}
\end{definition}



\begin{remark}
  Let us state some alternative definitions of open and closed:
  \begin{itemize}
    \item 
      By Markov, $P$ being open iff there merely exists some $\alpha:\Noo$ such that 
      $P\leftrightarrow \exists_{n:\N} \alpha(n) = 1$. 
    \item 
      $P$ is open iff there merely exists some $\alpha:2^\N$ such that $P\leftrightarrow \exists n \alpha(n) = 1$.
    \item 
      $P$ is closed iff there merely exists some $\alpha:2^\N$ such that $P\leftrightarrow \forall n \alpha(n) = 0$.
\end{itemize}
\end{remark}

\rednote{I want to know what is data and what really only merely exists. I'll mark it in red.}
\begin{definition}
  Let $P$ be a proposition. 
  \begin{itemize}
    \item $P$ is strongly open   if it comes equiped with some  $\alpha:2^\mathbb N$ such that $P \leftrightarrow \forall_{n:\mathbb N} \alpha n = 0 $. 
    \item $P$ is strongly closed if it comes equiped with some  $\alpha:2^\mathbb N$ such that 
      $P \leftrightarrow \exists_{n:\mathbb N} \alpha n = 1$. 
  \end{itemize}
\end{definition}

\begin{remark}\label{rmkOpenClosedNegation}
  The negation of an open proposition is a closed proposition and vice versa. 
  Furthermore, both open and closed propositions are double negation stable by Markov, 
  meaning that for $P$ open or closed, we have $\neg \neg P \to P$.
\rednote{Also holds for the strong versions}
\end{remark}



\begin{lemma}\label{lemOpenClosedDisjunctionConjunction}
  Closed propositions are closed under countable conjunctions %$\Pi$-types 
  and finite disjunctions.
  Open propositions are closed under countable disjunctions %$\Sigma$-types 
  and finite conjunctions. 
\end{lemma}
\begin{proof}
  \begin{itemize}
    \item Let $P,Q$ be closed. We will show $P\vee Q$ closed as well. 
%      Let $\alpha,\beta:2^\N$ be such that 
%      $P\leftrightarrow \forall_{n:\N} \alpha(n) = 0$ and 
%      $Q\leftrightarrow \forall_{n:\N} \beta(n) = 0$. 
%      Let $\gamma:2^\N$ be such that $\gamma(n) = 0$ iff 
%      we have that $\alpha|_k = \infty|_k$ or $\beta|_k = \infty|_k$. 
      % I think you'll still need some choice to infer from \gamma(n) = 0 for all n that 
      % \alpha(n) = 0 for all n or \beta(n) = 0 for all n
      Assume $\alpha,\beta$ witness that $P,Q$ respectively are closed. 
      %$P\leftrightarrow \alpha = \infty, Q \leftrightarrow \beta = \infty$. 
      So $P\vee Q \leftrightarrow 
      ((\forall_{n:\N} \alpha(n) = 0) 
      \vee (\forall_{n:\N} \beta(n) = 0))$
      By \Cref{corAlternativeLLPO}, we have that 
      $P\vee Q \leftrightarrow 
      \forall_{n:\N}  \forall_{m:\N}  (\alpha(n) = 0 \vee \beta(m) = 0)$
      For $s:\N \to \N \times \N$ surjective, 
      this in turn is equivalent to 
      $\forall_{k:\N}  (\alpha(\pi_0(s(k))) = 0 \vee \beta(\pi_1(s(k))) = 0)$. 
      We define a sequence $\gamma:2^\N$ where 
      $\gamma(k) = 0$ iff $(\alpha(\pi_0(s(k))) = 0 \vee \beta(\pi_1(s(k))) = 0)$. 
      As this is decidable, $\gamma$ is well-defined.
      Also
      $P\vee Q \leftrightarrow \forall_{k:\N} \gamma(k) = 0$, 
      thus $P\vee Q$ is closed. 
\rednote{Also holds for the strong versions}
    \item Let $(P_n)_{n:\N}$ be a countable family of closed propositions. 
      We will show that 
      $\forall_{n:\N} P_n$ is closed as well. 
      By countable choice, we have a function which for each 
      $n:\N$ gives some $\alpha_n:\Noo$ 
      such that $P_n \leftrightarrow \alpha_n =\infty$. 
      Thus $(\forall_{n:\N} P_n )\leftrightarrow 
      (\forall_{n:\N}(\alpha_n = \infty))$. 
      Let $s:\N \to \N \times \N$ surjective and let 
      $\beta(k) = \alpha_{\pi_0(s(k))}(\pi_1 (s(k)))$. 
      Note that $\beta = \infty$ iff for every $n:\N$, 
      and every $m:\N$, we have $\alpha_n(m) = 0$, 
      which happens iff $\alpha_n = \infty$. 
      We conclude that $\forall_{n:\N} P_n \leftrightarrow \beta = \infty$.
      Thus closed propositions are closed under countable conjunctions. 
\rednote{Also holds for the strong versions, wouldn't need countable choice.}
    \item 
      Let $P,Q$ be open. Let $\alpha,\beta:2^\N$ be such that 
      $
      P\leftrightarrow \exists_{k:\N}  \alpha(k) = 1, 
      Q\leftrightarrow \exists_{k:\N}  \beta(k) = 1
      $. 
      We will show that $P\wedge Q$ is open as well. 
      Consider $\gamma:2^\N$ given by 
      $\gamma(l) = 1$ iff there exist some $k,k'\leq l$ with 
      $\alpha(k) = \beta(k') = 1$. As we only need to check finitely many combinations 
      of $k,k'$, this is a decidable property and $\gamma$ is well-defined. 
      Clearly we have $(\exists_{l:\N} \gamma(l) = 1) \leftrightarrow (P \wedge Q)$. 
      Thus $P\wedge Q$ is open. 
\rednote{Also holds for the strong versions}
    \item Let $(P_n)_{n:\N}$ be a countable family of open propositions. 
      We will show that $\exists_{n:\N}P_n$ is open as well. 
      By countable choice, we have a function which for each 
      $n:\N$ gives some $\alpha_n:\Noo$ 
      such that $P_n \leftrightarrow \alpha_n \neq \infty$. 
      Thus $(\exists{n:\N} P_n) \leftrightarrow 
      (\exists{n:\N}\alpha_n \neq \infty)$. 
      Let $s:\N \to \N \times \N$ surjective and let 
      $\beta(k) = \alpha_{\pi_0(s(k))}(\pi_1 (s(k)))$. 
      Then let $\gamma(k) = 1$ iff $k$ is minimal with $\beta(k) = 1$. 
      Note that $\beta \neq \infty$ iff there is some $n:\N$, 
      and some $m:\N$, with $\alpha_n(m) = 1$, 
      which happens iff $\alpha_n \neq \infty$. 
      We conclude that $\exists{n:\N} P_n \leftrightarrow \gamma \neq \infty$.
      Thus open propositions are closed under countable disjunctions. 
\rednote{Also holds for the strong versions}
  \end{itemize}   
\end{proof}


%\begin{lemma}\label{OpenClosedImplication}
%  For $P$ open and $Q$ closed or $P$ closed and $Q$ open, we have that $(P\to Q) \leftrightarrow (\neg P \vee Q)$. 
%\end{lemma}
%\begin{proof}
%  Without losing generality, assume $P$ open and $Q$ closed. 
%  The left direction is well-known. 
%  What is also well-known is 
%  $(P \to Q) \to \neg \neg (\neg P \vee Q)$.
%  By \Cref{rmkOpenClosedNegation}, $\neg P$ is closed. 
%  By \Cref{lemOpenClosedDisjunctionConjunction}, it follows that $\neg P \vee Q$ is closed. 
%  By \Cref{rmkOpenClosedNegation} it follows that $\neg \neg (\neg P \vee Q) \to (\neg P \vee Q)$. 
%  Thus $(P \to Q) \to (\neg P \vee Q)$ as required. 
%\end{proof}
\rednote{Changes here}
\begin{corollary}\label{OpenAndClosedMeansDecidable}
  Let $P$ be both open and closed. 
  Then $P$ is decidable. 
\end{corollary} 
\begin{proof}
  As $P$ is open and closed, by \Cref{lemOpenClosedDisjunctionConjunction},  so is $P\vee \neg P$. 
  Therefore, by \Cref{rmkOpenClosedNegation}, we have that $\neg \neg (P\vee \neg P) \to (P\vee \neg P)$. 
  As $\neg \neg (P\vee \neg P)$ is provable, we conclude $P\vee \neg P$, thus $P$ is decidable. 
\rednote{Also holds for the strong versions}
\end{proof}


\rednote{The results on equality might go somewhere else, before compact Hausdorff}
\begin{lemma}\label{EqualityBooleStoneClosedOpen}
  Equality in $B:\Boole$ is open and equality in $X:\Stone$ is closed. 
\rednote{Also holds for the strong versions}
\rednote{Should be seperated into two lemmas}
\end{lemma}
\begin{proof}
  Let $B:\Boole$, and let $a,b:B$. We will show that $a=b$ is closed. 
  Recall that $B = \langle G \rangle / \langle R \rangle$. 
  And that $(a=b):= (a-b \in \langle R \rangle)$, 
  where $\langle R \rangle$ is the set of expressions of the form
  $(\bigvee_{r\in R_0} r) \wedge b$ for $R_0\subseteq R$ finite and $b:\langle G \rangle$ 
  arbitrary. 
  In particular, note that $\langle R \rangle$ is countable. 
  Let $(r_n)_{n:\N}$ enumerate $ \langle R \rangle $
  Also note that equality in $\langle G \rangle$ is decidable. 
  Thus we can define a sequence $\alpha:\Noo$ with $\alpha(k) = 1$ 
  iff $k$ is minimal with $a -b = r_n$. 
  Now $\alpha \neq \infty $ iff $a = b$. Thus $a=b$ is an open proposition. 

  Let $X:\Stone$ and let $x,y:X$. 
  Let $X= Sp(B)$ for $B = \langle G \rangle / \langle R \rangle $. 
  Note that $x=y$ iff $x(g) = y(g)$ for all generators $g:G$. 
  As $G$ is countable, we can enumerate it's elements as $(g_n)_{n:\N}$. 
  As equality in $B$ decidable, we can define a sequence $\alpha:\Noo$ 
  with $\alpha(k) = 1$ iff $k$ is minimal with $x(g_k) \neq y(g_k)$. 
  Now $\alpha(k) = 0$ for all $k$ iff for all $k$ we have $x(g_k) = y(g_k)$ and thus $x(g) = y(g)$ for all generators $g$ 
  and thus $ x= y$. 
  Thus $(x=y) \leftrightarrow \alpha= \infty$ and $ x=y$ is closed. 
\end{proof}

The following question was asked by Bas Spitters at TYPES
\begin{lemma}[Apartness]
  If $X$ is a type with equality either open or closed, and we are given $x,y,z:X$ with $x \neq y$, we have 
  $x\neq z \vee y \neq z$.  
\end{lemma}
\begin{proof}
  By assumption $x\neq z$ and $ y \neq z$ are both closed propositions. 
  By \Cref{lemOpenClosedDisjunctionConjunction}, so is $x \neq z \vee y \neq z$. 
  By \Cref{rmkOpenClosedNegation}, it is thus sufficient to show 
  $(\neg \neg (x \neq z \vee y \neq z))$.
  Note that as $x\neq y$, we can show that $\neg ( x = z \wedge y = z)$. 
  This in turn implies $\neg \neg ( x \neq  z \vee y \neq  z)$, as required. 
\end{proof}

\begin{lemma}\label{LemInhabitedOfStoneIsClosed}
  If $S$ is Stone, then $||S||$ is a \rednote{strongly} closed proposition. 
\end{lemma} 
\begin{proof}
  By \Cref{LemSurjectionsFormalToCompleteness}, it is sufficient to show that 
  $\neg \neg S$ is closed. We will show that $\neg S$ is open. 
  Assume $S= Sp(B)$. We will show that $\neg S$ iff $0=1$ in $B$. 
  By \Cref{EqualityBooleStoneClosedOpen}, $0=1$ is an open proposition, thus $\neg \neg S$ is closed. 
  \begin{itemize}
    \item If $0=1$ in $B$, there is no map $B\to 2$ preserving $0,1$, thus $\neg Sp(B)$. 
    \item If $\neg Sp(B)$, then $B=2^{Sp(B)}=2^\emptyset$, which is a set with only one element. Thus $0=1$ in $B$. 
  \end{itemize}
\end{proof}
In fact, the statement of the above lemma is equivalent to 
$\neg \neg S \to ||S||$, which is in turn equivalent to Axiom 2. 


\rednote{Active from here on}
\begin{lemma}
  Let $A$ be a countable set and let $P:A \to Open$
  be an open subset of $A$. 
  If $||\sum_{a:A} P a||$, then $\sum_{a:A} P a$ is a countable set. 
\end{lemma}
\begin{proof}
  Being countable is a proposition, so we may untruncate our assumptions:
  \begin{itemize}
    \item 
      Because we have $||\sum_{a:A} P a||$, we assume we have  some $x:A$ with $P x$. 
    \item 
      Because $A$ is countable, we may assume we have some 
      enumeration of elements from $A$ given as $(a_n)_{n:\N}$.
  \end{itemize}
  Furthermore, by countable choice, $P$ gives us 
  for each $a:A$ a specific $\alpha_a:2^\N$ 
  such that 
  \begin{equation}
    P a \leftrightarrow \exists_{m:\N} \alpha_a(m) = 1.
  \end{equation}
  Using this data, we will 
  construct a surjection $s:\N \times \N \to \sum_{a:A} P a$. 
  Define 
  \begin{equation}
    s(m,n) = 
    \begin{cases}
      x \text{ if } \alpha_{a_n}(m) = 0 \\
      a_n \text{ if } \alpha_{a_n}(m) = 1
    \end{cases}
  \end{equation}
  To see that $s$ is surjective, we need to show that for each $a:A$ with $P a$ there 
  merely exists some $(m,n):\N \times \N$ 
  with $s(m,n) = a$. 
  \begin{itemize}
    \item 
  Because $(a_n)_{n:\N}$ formed an enumeration, we have that 
  there merely is some $n$ with $a = a_n$. 
  We can use this $n$ as we're proving a proposition. 
    \item 
  As $P a$ there merely is some $m$ with $\alpha_{a_n}(m) = 1$. 
  This $m$ can also be used as we're proving a proposition. 
  \end{itemize}
  Now by the above choice $s(m,n) = a_n = a$, 
  thus for each $a:A$ with $P a$, there exists some $(m,n)$ with 
  $s(m,n) = a$. Thus $s$ is surjective. 

  As $\N \times \N \simeq \N$, we conclude that 
  $\sum_{a:A} P a$ is countable. 
\end{proof}

\subsection{Closed subspaces of Stone spaces}
%\subsection{The topology of Stone spaces}
\rednote{Active section}
\begin{definition}
  For $S$ a set and $D\subseteq S$ an arbitrary subset, we call $D$ decidable/open/closed 
  iff for every $x:S$, $D(x)$ is decidable/open/closed. 
\end{definition}
\begin{remark}
  \begin{itemize}
    \item   \Cref{rmkOpenClosedNegation} implies that the complement of an 
  open subset is closed and vice versa. 
  \item 
    \Cref{OpenAndClosedMeansDecidable} implies that a subset which is both open 
    and closed is decidable. 
  \item 
    As propositions are subsets of $\top = Sp(2)$, 
    we have that open/closed/decidable subsets of $\top$
    are open/closed/decidable propositions. 
  \item By \Cref{EqualityBooleStoneClosedOpen}, singletons in 
    Boolean algebras are open and singletons in Stone spaces are closed. 
\end{itemize}
\end{remark}
\begin{lemma}
  Let $S$ Stone. 
  Then a subset $D\subseteq S$ is closed iff there merely is some $T$ Stone 
  and map $f:T\to S$ with image $D$. 
\end{lemma}
\begin{proof}
  Suppose $D\subseteq S$ is closed. 



  Let $f:T\to S$ be a map between Stone spaces. 
  We will show that $f(T)$ is closed. 
  For $x:S$, we have that $x\in f(T)$ iff there exists some $y:T$ 
  with $f(y) = x$. 
  So $(f(T))(x) \leftrightarrow || \sigma_{y:T} f(y) = x||$. 
  Note that $\sum_{y:T} f(y) = x$ is the pullback as follows:
  \begin{equation}
    \begin{tikzcd}
      \sum_{y:T} f(y) = x\arrow[rd,"\lrcorner"{pos=0.125},phantom] & 1 \arrow[d,"x"] \\
      T \arrow[r,"f"]  S
    \end{tikzcd}
  \end{equation}
  As any pullback of Stone spaces is Stone, it follows that 
  $\sum_{y:T} f(y) = x$ is Stone, and by \Cref{LemInhabitedOfStoneIsClosed}, 
  so is $||\sum_{y:T} f(y) = x||$. Thus $f(T)(x)$ is a closed proposition and $f(T)$ closed. 
  We conclude that the image of a map of Stone spaces is closed. 
\end{proof}

\begin{lemma}
  Let $D\subseteq Sp(B)$. 
  Then $D$ is decidable iff there merely exist some $b\in B$ such that 
  $D = \{x:Sp(B) | x(b) = 1\}$. 
\end{lemma}
\begin{proof}
  Let $D\subseteq Sp(B)$ be decidable. 
  Define $f_D:2^{Sp(B)}$ by $f_D(x) = 1$ if $D(x)$ and $f_D(x) = 0$ if $\neg D(x)$.
  Stone duality then gives there is some $b_D:B$ with 
  $f_D(x) = 1 \leftrightarrow x(b_D) = 1$. 
  Thus $D(x) \leftrightarrow f_D(x) = 1 \leftrightarrow x(b_D) = 1$ as requried. 

  For the converse, suppose $D  = \{x : Sp(B) | x(b) = 1\}$ then 
  whenever $y:Sp(B)$, we have $D(y)$ iff $y(b) = 1$, which is decidable
  as for any $c:2$ we have $c = 0 \vee c = 1$. 
  Thus $D$ is decidable. 
\end{proof}

\begin{lemma}
  Let $A\subseteq S$ be a subset of a Stone space. TFAE
  \begin{itemize}
    \item $A$ is closed. 
    \item There merely is a countable set set of decidable subsets $D_n\subseteq S$ with 
      $A = \bigcap_{n:\mathbb N} D_n$. 
%    \item $\neg A$ is open. 
%    \item There merely is a countable set set of decidable subsets $E_n\subseteq S$ with 
%      $\neg A = \bigcup_{n:\mathbb N} D_n$. 
  \end{itemize}
\end{lemma}
\begin{proof}
  Suppose $A$ is closed. So for each $x: S$ there merely exists some 
  $\alpha:2^\mathbb N$ with $A x \leftrightarrow (\forall_{n:\mathbb N}\alpha(n) = 0$. 
  Let $E$ be a dependent family of types over $X$ given by
  $E x  = \sum\limits_{\alpha:2^\mathbb N} 
  (A x \leftrightarrow \forall_{n:\mathbb N} \alpha(n) = 0)$
  The projection map $\sum\limits_{x: S} E x \to S$ is surjective by our assumption, 
  thus local choice gives us some Stone space $T$ and arrows as follows:
  \begin{equation}\begin{tikzcd}
    \sum\limits_{x:S}E x \arrow[d,""',two heads]\\
    S & \arrow[l, "", two heads,"s"] T\arrow[lu, ""']
  \end{tikzcd}\end{equation}  
  The map $T\to \sum\limits_{x:S} E x $ induces a map $\gamma_{(\cdot)}:T \to 2^\mathbb N$, 
  such that for all $t:T$ we have 
  \begin{equation}
    (\forall_{n:\mathbb N} \gamma_t(n)  = 0 )\leftrightarrow E ( s (t))
  \end{equation}
  Now for $T = Sp(B)$, under Stone duality the map $\gamma_{(\cdot )}$ 
  corresponds with a map $\langle C \rangle \to B$, 
  which gives a map $\mathbb N \to B$. 



\end{proof}




\begin{corollary}\label{CorDoubleNegToAx2}
  If we have for $S:\Stone$ that $\neg \neg S \to ||S||$, 
  we can conclude that surjections are formal surjections. 
\end{corollary}
\begin{proof}
  \rednote{All the unkonw references are lemmas that still need to be done (or at least linked) without using ax2}
  Now assume that for $S:\Stone$, we have $\neg \neg S \to ||S||$. 
  Let $f:B \to B'$ be injective, and let 
  $g:S' \to S$ correspond to $f$. 
  To show that $g$ is surjective, we need to show that for any $x:S$, we have $||\sum\limits_{y:S'} g y = x||$. 
  By our assumption, we can prove this by showing that 
  \begin{equation}
    g^{-1}\{x\} := \sum\limits_{y:S'} g y = x
  \end{equation} is Stone and non-empty. 
    First note that $\sum\limits_{y:S'} g y = x$ is the pullback of $g$ and the map $1 \to S$ given by $x$.
      By duality and \Cref{rmkBoolePushouts} the pullback of Stone spaces is a Stone space. 
   

      By \Cref{lemPointsAreClosed} and \Cref{lemClosedIntersectionOfDecidable}, 
      $\{x\}$ is a countable intersection of decidable sets $\bigcap_{n:\mathbb N} D_n$. 
      We have that 
      \begin{equation}
        g^{-1}\{x\} = g^{-1}(\bigcap_{n:\mathbb N} D_n) = 
        \bigcap_{n:\mathbb N} g^{-1}(D_n)
      \end{equation} 
      Let $n:\mathbb N$. By Stone duality, the decidable subset $D_n$ corresponds to some $b_n:B$. 
      As $x\in D_n$, we have that $b_n \neq 0$. 
      As $f$ is injective, we have that $f(b_n) \neq 0$. 
      Therefore the corresponding decidable subset, which is $g^{-1}(D_n)$ is not empty. 

      Note that $D_n$ is a closed subset of a Stone space, hence Stone by \Cref{LemClosedSubsetOfStoneIsStone}, 
      and by the same argument as above $g^{-1}(D_n)$ is Stone. By our assumption $g^{-1}(D_n)$ is thus merely inhabited. 
      Now by \Cref{lemCompactnessCountableIntersection}, this gives that 
      $\bigcap_{n:\mathbb N} g^{-1}(D_n)$ is merely inhabited, as required. 
\end{proof}


\begin{definition}
  For $S$ a set and $D\subseteq S$ an arbitrary subset, we call $D$ decidable/open/closed 
  iff for every $x:S$, $D(x)$ is decidable/open/closed. 
\end{definition}
\rednote{Here I'm also interested in the data}
\begin{definition}
  Let $S$ be a set and $D\subseteq S$ be a subset. 
  \begin{itemize}
    \item We call $D$ strongly closed if we have a function 
  $\alpha_{\cdot}S\to 2^\mathbb N$ such that $ D(x) \leftrightarrow \forall_{n:\mathbb N} \alpha_x n = 0$. 
    \item We call $D$ strongly open if we have a function 
  $\alpha_{\cdot}S\to 2^\mathbb N$ such that $ D(x) \leftrightarrow \exists_{n:\mathbb N} \alpha_x n = 1$. 
  \end{itemize} 
\end{definition}


\begin{remark}
  Let us restate some results from the previous section in this light. 
  \begin{itemize}
    \item   \Cref{rmkOpenClosedNegation} implies that the complement of an 
  open subset is closed and vice versa. 
  \item 
    \Cref{OpenAndClosedMeansDecidable} implies that a subset which is both open 
    and closed is decidable. 
  \item 
    As propositions are subsets of $\top = Sp(2)$, 
    we have that open/closed/decidable subsets of $\top$
    are open/closed/decidable propositions. 
  \item By \Cref{EqualityBooleStoneClosedOpen}, singletons in 
    Boolean algebras are open and singletons in Stone spaces are closed. 
    \rednote{All of these are still true for the stronger versions of open and closed. }
\end{itemize}
\end{remark}
\begin{theorem}\label{StrongVersionOfEquivalencesOfClosedSubsetsOfStone}
  Let $A\subseteq Sp(B)$ be a subset of a Stone space. TFAE:
  \begin{enumerate}[(i)]
    \item We have some $(D_n)_{n:\N}$ countable family 
      of decidable subsets of $Sp(B)$ with $A = \bigcap_{n:\N} D_n$. 
    \item We have a Stone space $T$ and some map $T\to Sp(B)$ 
      whose image is $A$. 
%    \item For each $x:Sp(B)$, we have a specific sequence $\alpha_x:2^\N$ such that 
%      $A x \leftrightarrow \forall_{n:\N} \alpha_x(n) = 0$. 
    \item $A$ is strongly closed. 
  \end{enumerate}
\end{theorem}
\begin{proof}
  \begin{itemize}
    \item We will show that $(i)$ implies $(ii)$.
      Let $(D_n)_{n:\N}$ be a countable family of decidable subsets of $Sp(B)$. 
      By Stone duality, decidable subsets of $Sp(B)$ correspond to elements of $B$. 
      Thus we get a countable family $(d_n)_{n:\N}$ of elements of $B$ such that 
      such that $D_n = \{x:Sp(B)| x(d_n) = 1\}$. 
      Consider the countably presented boolean algebra 
      $C = B/\langle (\neg d_n)_{n:\N}\rangle$.
      The quotient map provides an epi $B\twoheadrightarrow C$, which 
      corresponds to a mono $Sp(C) \hookrightarrow Sp(B)$. 
      We claim that the image of this map is exactly $\bigcap_{n:\N} D_n$. 
      Note that $x$ is in the image of this map iff for all $n:\N$, we have 
      $x(\neg d_n) = 0$. Note $x(\neg d_n) = 0$ iff $x (d_n) = 1$ iff $x\in D_n$. 
      Thus $x$ is in the image of the map iff for all $n$ we have $x\in D_n$,
      as required. 
      We conclude that the intersection of a countable family of decidable subsets 
      is the image of some Stone space. 
    \item We will show that $(ii)$ implies $(iii)$.
    Let $f:T\to S$ be a map between Stone spaces. 
    We will show that $f(T)$ is closed. 
    For $x:S$, we have that $x\in f(T)$ iff there exists some $y:T$ 
    with $f(y) = x$. 
    So 
    \begin{equation}
      (f(T))(x) \leftrightarrow || \sum_{y:T} f(y) = x||
    \end{equation}
    Note that $\sum_{y:T} f(y) = x$ is a pullback as follows:
    \begin{equation}
      \begin{tikzcd}
        \sum_{y:T} f(y) = x\arrow[rd,"\lrcorner"{pos=0.125},phantom] \arrow[r] \arrow[d]
        & 1 \arrow[d,"x"] \\
        T \arrow[r,"f"']  & S
      \end{tikzcd}
    \end{equation}
    As any pullback of Stone spaces is Stone, it follows that 
    $\sum_{y:T} f(y) = x$ is Stone.
%    , and by \Cref{LemInhabitedOfStoneIsClosed}, 
%    so is $||\sum_{y:T} f(y) = x||$. 
    Recall now that the proof of \Cref{LemInhabitedOfStoneIsClosed} 
    gives a specific sequence $\alpha_x$ witnessing that $||\sum_{y:T} f(y) = x||$ is closed. 

    Thus $f(T)(x)$ is a closed proposition and $f(T)$ closed. 
    We conclude that the image of a map of Stone spaces is closed, 
    and given the Stone spaces we can construct the witness sequences. 
  \item We will show that $(iii)$ implies $(i)$. 
    Let $\alpha_{\cdot}$ witness that $A$ is strongly closed. Define 
    $D_n(x) \leftrightarrow (\alpha_n(x) = 0)$. 
    Note that $D_n$ is well-defined as $\alpha_n(x):2$ and for all $b:2$ we have
    $(b=0)+ (b=1)$. Now $x \in \bigcap_{n:\N} D_n$ iff
    $\alpha_x(n) = 0$ for all $n:\N$, 
    which happens iff $A x$. Thus $A = \bigcap_{n:\N} D_n$. 
  \end{itemize} 
\end{proof} 
\begin{corollary}
  Let $A\subseteq Sp(B)$ be a subset of a Stone space. TFAE:
  \begin{enumerate}[(i)]
    \item There merely exists a countably family $(D_n)_{n:\N}$
      of decidable subsets of $Sp(B)$ with $A = \bigcap_{n:\N} D_n$. 
    \item There merely exists some Stone space $T$ and some map $T\to Sp(B)$ 
      whose image is $A$. 
%    \item For each $x:Sp(B)$, we merely have a sequence $\alpha:2^\N$ such that 
%      $A x \leftrightarrow \forall_{n:\N} \alpha(n) = 0$. 
    \item $A$ is closed. 
  \end{enumerate}
\end{corollary}
\begin{proof}
  \begin{itemize}
    \item $(i) \leftrightarrow (ii)$ follows from propositionally truncating the equivalence from 
      \Cref{StrongVersionOfEquivalencesOfClosedSubsetsOfStone}.
    \item To show that $A$ is closed, we need to show that $A x $ is closed for all $x$, which is a proposition, 
      hence we could untruncate any assumptions and thus by \Cref{StrongVersionOfEquivalencesOfClosedSubsetsOfStone}, 
      any of $(i),(ii)$ imply $(iii)$. 
    \item Assume $A$ is closed. TODO
  \end{itemize}
\end{proof}



