In \Cref{Axioms}, we have chosen to present propositional completeness as an axiom. 
However, assuming Stone duality, we could have made some other choices, 
and left propositional completeness as a theorem. 
What's more, assuming the axiom of Dependent choice,
the axiom is equivalent to LLPO. 
In this section, we will show these equivalences. 
\rednote{At time of writing, not all the references were cleaned up, and some might change or split, 
so I need to come back here later after all.}

\begin{theorem}\label{AlternativesToAxiom2}
  Assuming Stone duality, the following are equivalent:
  \begin{enumerate}[(i)]
    \item For $S$ Stone, we have $\neg \neg S \to ||S||$. 
    \item For $S$ Stone, we have that $||S||$ is closed. 
    \item A map $f:A \to B$ in $\Boole$ is injective iff the map $(\cdot) \circ f : Sp(B) \to Sp(A)$ is surjective. 
  \end{enumerate}
\end{theorem}
\begin{proof}
  We assume that $S= Sp(B)$. 
  Note the proof of \Cref{SpectrumEmptyIff01Equal} only uses Stone duality. 
  The proof of \Cref{EqualityIsOpen} only relies on the definition of $\Boole$.
  Hence $\neg S$ is equivalent to $0=_B 1$, which is open. 
  Hence $\neg \neg S$ is a closed proposition. It is also equivalent to $\neg \neg ||S||$. 
  Therefore $(i) \to (ii)$. 
  Furthermore, \Cref{rmkOpenClosedNegation}
  only used \Cref{MarkovPrinciple}, 
  which followed from Stone duality as well. 
  Hence if $||S||$ is closed, we have $\neg \neg ||S|| \leftrightarrow ||S||$, thus $(ii) \to (i)$. 
  $(i)\to (iii)$ is \Cref{FormalSurjectionsAreSurjections}. 
  By the above discussion, we also have that $\neg \neg S$ iff $0\neq_B 1$. 
  Note that $0\neq_B 1$ iff the map $2\to B$ is injective. 
  Furthermore, $||S||$ iff the map $S \to \top $ is surjective. 
  Hence $(iii) \to (i)$. 
\end{proof} 

\begin{lemma}\label{LLPOAndDCToCompleteness}
Assuming dependent choice, Stone duality, and that closed propositions are closed under disjunctions, 
we can show propositional completeness. 
\end{lemma}
\begin{proof}
  Let $B:\Boole$ satisfy $0\neq_B 1$. We will show there merely exists a map $B\to 2$. 
  Let $G$ be the set of generators of $B$. 
  We will use dependent choice on the the following $E_n,R_n$:
  \begin{itemize}
    \item 
  Let $E_n$ be the type consisting of 
  \begin{itemize}
    \item A map from the first $n$ generators of $B$ to $2$, denoted $x_n:G_n \to 2$. 
    \item A proposition denoting that $0\neq_{B_n} 1$ for $B_n$ given by:
      \begin{equation}
        B_n := B/\big( \{g|g\in G_n, x_n(g) = 0\} \cup \{ \neg g| g\in G_n, x_n(g) = 1\}\big).
      \end{equation}
  \end{itemize}
  \item 
    And let $R_n:E_n \to E_{n+1} \to \mathcal U$ denote the relation that $x_{n+1}$ extends $x_n$. 
  \end{itemize} 
  Note that $E_0$ is inhabited as $0\neq_B 1$. Assume $E_n$.
%  Now assume $x_n:G_n\to 2$ witnesses $E_n$. 
  As $0\neq_{B_n}1$, for all $g:B_n$, we can show 
%  we have $$\neg ((g =1)  \wedge ((\neg g) = 1)).$$
% % 
%%  Now suppose that $E_n$ is inhabited,  and let $x_n:G_n \to 2$. 
%%  Note that in $B_n$, we have $0\neq 1$ and thus $$\neg ((g =1)  \wedge ((\neg g) = 1))$$
%%  for all $g:B_n$.
%  Therefore, we have 
  $$\neg \neg (( g\neq 1) \vee ((\neg g) \neq 1)).$$
  By \Cref{EqualityIsOpen}, and \Cref{rmkOpenClosedNegation}, 
  (which could be shown using Stone Duality)
  and the assumption that 
  closed statements are closed under disjunction, we have that the above statement is equivalent to 
  $(g \neq 1) \vee ((\neg g) \neq 1)$. 
  This holds in particular for $g$ the $n+1$'th generator of $B$. 
  Therefore, we have that $0\neq 1$ in $B_n/\{g\}$ or in $B_n/\{\neg g\}$. 
  Thus we can extend $x_n$ by letting $x_{n+1}(g) = 0$ or $x_{n+1}(g) = 1$ respectively. 
  
  By dependent choice, we get a map $x:G\to 2$. 
  We claim that for this map $x$, we have $0\neq 1$ in 
  \begin{equation}
    B' := B/\big( \{g|g\in G, x(g) = 0\} \cup \{ \neg g| g\in G, x(g) = 1\}\big).
  \end{equation}
  Note that $B'$ is the colimit of the sequence $B_n$ with projection maps $B_n \to B_{n+1}$. 
  Thus if $0=1$ in $B'$, $0=1$ in some $B_n$, which doesn't happen by assumption. 
  Therefore we have $0\neq 1$ in $B'$. 
  Furthermore, note that $B'$ is equivalent to a Boolean algebra with no generators, 
  as any generator in $B$ is sent to either $0$ or $1$ by the relations in $B'$. 
%
  But now any Boolean algebra with no generators and $0\neq 1$ is isomorphic to $2$. 
  Therefore $B'\simeq 2$, and the projection map $B\to B'$ gives a map $B \to 2$. 
  
\end{proof}

\begin{corollary}
Assuming dependent choice and Stone duality, TFAE:
\begin{enumerate}[(i)]
  \item For $S$ Stone, we have $\neg \neg S \to ||S||$. 
  \item LLPO.
  \item The disjunction of two closed propositions is closed. 
\end{enumerate}
\end{corollary}
\begin{proof}
  $(i) \to (ii)$ is \Cref{LLPO}, $(ii) \to (iii)$ is \Cref{ClosedDisjunction}, 
  and $(iii) \to (i)$ is the \Cref{LLPOAndDCToCompleteness}
\end{proof}
\rednote{
  @Hugo, you mentioned that axiom 2 was independent from the other axioms. 
This might be a good place to reference to that proof}
