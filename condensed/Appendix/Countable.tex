\section{Countability}\label{CountabilityDiscussion}
In the system presented in this paper, 
one of the fundamental building blocks are countably presented Boolean algebras. 
There are several definitions of countable, which are not necissarily constructively equivalent. 

\begin{definition}
  A type $T$ is enumerable iff there exists a surjection $\N \to 1 + T$. 
\end{definition}
\begin{definition}
  A type $T$ is strongly countable if
  $T$ is merely isomorphic to $\N$ or if 
  there exists some $k:\N$ with $T \simeq Fin_k$.% or $T\simeq \N$. 
\end{definition}
\begin{definition}
  A type $T$ is subcountable iff it is merely isomorphic to a decidable subset of $\N$. 
\end{definition}

\begin{lemma}
  Every strongly countable type is subcountable. 
\end{lemma}
\begin{proof}
  Note that $Fin_k$ and $\N$ are both isomorphic to a decidable subset of $\N$. 
\end{proof}
\begin{lemma}
  Every subcountable type is enumerable. 
\end{lemma}
\begin{proof}
  For $A\subseteq \N$ decidable, define $f:\N \to 1 + \Sigma_{n:\N} A(n)$ by 
  $$
  f(n) = 
  \begin{cases}
    inl(*) \text{ if } \neg A(n)\\
    inr(n) \text{ if } A(n)
  \end{cases}
  $$
\end{proof} 


\begin{lemma}\label{OpenSubsetNAreSubCountable}
  Any open subset of $\N$ is subcountable. 
\end{lemma} 
\begin{proof}
  Let $A:\N \to \Open$. 
  By countable choice, there exists a map $\alpha_{(\cdot)}:\N \to \Noo$ such that 
  $\exists_{m:\N} \alpha_n(m) = 0 \leftrightarrow A (n)$. 
  Define $B\subseteq \N \times \N$ by 
  $B(m,n) = (\alpha_{n}(m) = 0)$. 
  Note that $A(n) \leftrightarrow || \Sigma_{m:\N } B(m,n) ||$
\end{proof}

\begin{lemma}\label{OpenSubsetEnumerableAreEnumerable}
  Any open subset of an enumerable type is enumerable. 
\end{lemma}
\begin{proof}
  Let $A$ be enumerable and let $P:A \to Open$.
  We will show that $\Sigma_{a:A} P a$ is enumerable. 
  Let $s:\N \to 1 + A$ surjective. 
  Define $s':\N \to Open$ by 
  $$
  s'(n) = 
  \begin{cases}
    \bot \text { if } s(n) = inl(*)\\
    P(a) \text { if } s(n) = inr(a)
  \end{cases}
  $$ 
  By countable choice, we get a map 
  $\alpha_{(\cdot)}: \N \to 2^\N$  such that 
  $(\exists_{m:\N} \alpha_n(m) = 0) \leftrightarrow s'(n)$. 
  Note that $\alpha_n(m) = 1$ iff 
  $s'(n)$ which happens iff $s(n) = inr(a_n)$ for some $a_n:A$ with $P(a_n)$. 
  Therefore, we can define 
  $z:\N \times \N \to 1 + \Sigma_{a:A} P a$ by 
  \begin{equation}
    z(m,n) = 
    \begin{cases}
      inl(*) \text{ if } \alpha_{n}(m) = 0 \\
      a_n  \text{ if } \alpha_{n}(m) = 1 \text{ and $a_n$ as above}
    \end{cases}
  \end{equation}
  Note that if $a:A$ satisfies $P(a)$, there is some $n:\N$ with $s(n) = inr(a)$. 
  And as $P(a)$, there exists some $m:\N$ with $\alpha_n(m) = 1$. 
  Hence $z(m,n) = a$. 
  Thus $z$ is surjective. 
  As $\N \times \N \simeq \N$, we conclude that 
  $\sum_{a:A} P a$ is enumerable. 
\end{proof}

\begin{lemma}
  Any open subset of $\N$ is subcountable. 
\end{lemma}
\begin{proof}
  Let $A:\N \to \Open$. 
  By countable choice, we get a map $\alpha_{\cdot}: \N \to \Noo$ such that 
  $A(n) \leftrightarrow \Sigma_{m:\N} \alpha_n (m) = 1$. 
  Define $B:\N \times \N \to 2$ by $B(m,n) = \alpha_n(m)$. 
  We then have a bijection $\Sigma_{n:\N} A(n) \to \Sigma_{(n,m) : \N \times \N} B(m,n)$ sending 
  $(n,(m,p))$ to $(n,m,p)$.
\end{proof}



\begin{lemma}\label{OpenSubsetOfNNotDecidable}
  It is not the case that for every $P:\N \to \Open$, 
  $||\Sigma_{n:\N} P(n)||$ is decidable.
  % The subset being decidable could be interpreted as 
  %    that P(n) is decidable for all $n:\N$,
  % or that \Sigma_{n:\N} P(n) + \neg \Sigma_{n:\N} P(n) 
  % or that || \Sigma_{n:\N} P(n) || + \neg ||\Sigma_{n:\N} P(n) ||
\end{lemma}
\begin{proof}
  For $p$ any open proposition and $P(n) = p$ constantly, we have 
  $||\Sigma_{n:\N} P(n)||\leftrightarrow p$. 
  As not every open proposition is decidable (\Cref{rmkOpenClosedNegation}), 
  not every $||\Sigma_{n:\N} P(n)||$ is decidable. 
\end{proof}

\begin{lemma}\label{StronglyCountableTruncationDecidable}
  For every strongly countable type $A$, $||A||$ is decidable. 
\end{lemma}
\begin{proof}
  For a proposition, being decidable is a proposition. 
  Hence we may untruncate the definition of strongly open. 
  If $A \simeq \N$ or $A\simeq Fin_k$ for $k\neq 0$, we have $||A||$. 
  If $A \simeq Fin_0$, then $\neg ||A||$. 
\end{proof}

\begin{corollary}
  Not every enumerable type is strongly countable.
\end{corollary}
\begin{proof}
  If every enumerable type is strongly countable, 
  by \Cref{OpenSubsetEnumerableAreEnumerable}, every open subset of open subsets of $\N$ is strongly countable. 
  By \Cref{StronglyCountableTruncationDecidable}, the truncation of the corresponding type is decidable, which 
  contradicts\Cref{OpenSubsetOfNNotDecidable}.
\end{proof}

\begin{remark}
  Every enumerably represented Boolean algebra has an enumerable underlying set. 
  and every enumerable Boolean algebra is enumerably represented. 
\end{remark}
