\paragraph{Topological convergence}
In this section, $X$ is a Stone space. 
\begin{definition}
  A sequence in $X$ is a map $\mathbb N \to X$. 
\end{definition} 
\begin{definition}
  Let $\alpha$ be a sequence in $X$. 
  We say that $x$ is the limit of $\alpha$ iff 
  for any open $U\subseteq X$ containing $x$, 
  there merely is an $N:\mathbb N$ such that for $n \geq N$, we have
  $x_n \in U$. 
\end{definition}

\paragraph{Closed spaces contain their limits}
\begin{lemma}
  Let $x:2^\mathbb N$ and $D\subseteq 2^\mathbb N$ be a decidable subset. 
  Suppose that for each open $U\subseteq X$ with $U(x)$,
  we merely have some $y_U \in D \cap U$. 
  Then $x\in D$. 
\end{lemma}
\begin{proof}
  Because $D$ is a subtype, $x\in D$ is a proposition, and we will use existence whenever we have mere existence.
  Because $D$ is decidable, %there is a finite amount of information $D$ uses, 
  there merely exists an $n:\mathbb N$ such that 
  whenever $x =_n y$, we have $D(x) \leftrightarrow D(y)$. 
  Consider the open $U_n$ given by $x =_n \cdot $.
  By assumption, there merely is some $y\in D\cap U_n$. 
  so $D(y)$ and $x =_n y$, hence $D(x)$.
\end{proof}
\begin{corollary}
  Let $\iota :D\hookrightarrow 2^\mathbb N$ be the inclusion map of a decidable subset, 
  let $\alpha$ be a sequence in $D$, and 
  suppose that $\alpha\circ \iota$ has a limit $x$ in $2^\mathbb N$. 
  Then $x\in D$. 
\end{corollary}
\begin{corollary}
  Using (ii) from \Cref{propClosedAlternativeDefinitions} it follows that any 
  closed subset of a Cantor space contains all of it's limit points. 
\end{corollary}
\begin{remark}
  The converse is not true. It is not the case that if a subset of a Stone space 
  contains its limits, it is necessarily closed. 
  For any propostion $p$, we have the subset of Cantor space given by $A = \{x:2^\mathbb N | p\}$. 
  If $A$ was closed, $p$ would be equivalent to a proposition of the form $\alpha = 0$. 
  However, not all propositions are of this form. So $A$ needn't be closed. 
  But if a sequence in $A$ exists and has a limit, because the sequence exists, $p$ must hold and 
  thus the limit is contained in $A$ also. 
\end{remark}
