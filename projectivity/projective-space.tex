\subsection{Finite isotropy groups}

\begin{definition}
  A group (set) $G$ is a \emph{group of roots of unity} if there
  merely exists a $d\colon \N$ such $g^d = 1$ for all $g\colon G$.
\end{definition}

\begin{example}
  Let $d\colon \N$.  The group
  $\bmu_d \coloneqq \{t \colon \Gm \mid t^d = 1\}$ of \emph{$d$-th
    roots of unity} is a group of roots of unity (and a group scheme).
\end{example}

\begin{example}
  The group $\Gm$ is not a group of roots of units.  If it were, there
  would exist $d\colon \N$ such that $g^d = 1$ for all $g\colon \Gm$.
  This would mean that $R$ is set of zeros of the polynomial
  $T^d - 1$, so $R$ would be not not finite.  But $R$ is not finite.
\end{example}

\begin{definition}
  Let $X$ be a set and $G$ be a group (set).  An operation
  $X \times G \to X$ \emph{has isotropy groups of roots of unity} if
  and only if for all $p \in X$ the subgroup
  \begin{equation*}
    G_p \coloneqq \{g\colon G \mid x \cdot g = x\}
  \end{equation*}
  is a group of roots of unity.
\end{definition}

\begin{definition}\label{irrelevant ideal}
  Let $X \coloneqq \Spec S$ be an affine scheme with an action of the
  multiplicative group $\Gm$, that is $S$ is a finitely generated
  graded $R$-algebra.  The radical ideal generated by
  $\bigoplus_{d \neq 0} S_d$ is the \emph{irrelevant ideal} $S^+$ of
  $S$.  As a radical ideal, it is finitely generated.  Set
  \begin{equation*}
    X^\circ \coloneqq \{p\colon X \mid \exists f\colon S^+. f(p) \neq 0\}.
  \end{equation*}
\end{definition}

\begin{proposition}
  In the situation of \cref{irrelevant ideal}, the set $X^\circ$ is
  $\Gm$-invariant and is the largest open subscheme of $X$ where the
  action of $\Gm$ has isotropy groups of roots of unity.
\end{proposition}

\begin{proof}
  The irrelevant ideal $S^+$ is invariant under the induced action of
  $R^\times$, from which it follows that $X^\circ$ is $\Gm$-invariant.
  As $S^+$ is finitely generated as a radical ideal, $X^\circ$ is a
  finite union of standard-open subsets and thus open.

  Let $p\colon X$ with $p \in X^\circ$.  We have to show that the isotropy group of $p$
  is a group of roots of unity: There exists an $f\colon S^+$ such that
  $f(p) \neq 0$.  We may assume that $f$ is homogeneous, say of degree
  $d$.  For each $g\colon G_p$, we have
  $f(p) = f(p \cdot g) = f(p) \cdot g^d$.  By invertibility of $f(p)$
  it follows that $g^d = 1$.

  Let $f\colon S_0$ and $p\colon X$ with $f(p) \neq 0$.  To prove the
  maximum condition on $X^\circ$, it remains to show that there exists
  a $q \in X$ with $f(q) \neq 0$ such that $\Gm[,q]$ is not a group of roots of
  unity.  For this, we define an $R$-algebra homomorphism
  \begin{equation*}
    q\colon S \to R, g \mapsto
    \begin{cases}
      g(p) & \text{for $g \in S_0$ and} \\
      0 & \text{for $g \in S_d$ where $d \neq 0$.}
    \end{cases}
  \end{equation*}
  Then $f(q) \neq q$ and $\Gm[,q] = \Gm$.
\end{proof}

%%% Local Variables:
%%% mode: latex
%%% TeX-master: "main"
%%% End:
