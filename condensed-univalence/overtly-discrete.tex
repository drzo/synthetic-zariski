\begin{definition}
A type is overtly discrete if it is a sequential colimit of finite types.
\end{definition}

\begin{lemma}\label{overtly-discrete-colimit-finite}
Let $X$ be a type, TFAE:
\begin{enumerate}[(i)]
\item $X$ is overtly discrete.
\item $X$ is a quotient of a decidable subset of $\N$ by a decidable relation.
\end{enumerate}
\end{lemma}

\begin{proof}
\begin{itemize}
\item (i) implies (ii). Assume $X$ is of the form
\[X  = (\Sigma_\N D)/R\]
with $D$ decidable and $R$ open. Using choice for $\Sigma_\N D$ we get:
\[\alpha : (\Sigma_\N D) \to (\Sigma_\N D)\to 2^\N\]
such that:
\[R(x,y) = \exists_{k:\N} \alpha(x,y,k) = 1\]
Then we define:
\[X_n = (\Sigma_{\mathrm{Fin}(n)} D) / L\]
\[L(x,y) = \exists_{k:\mathrm{Fin}(n)} \alpha(x,y,k) = 1\]
We have that $X_n$ is a finite type as it is a decidable quotient of a decidable subset of a finite type. Moreover:
\[\mathrm{colim}_n X_n = X\]
as sequential colimit commutes with quotients by equivalence relations.
\item (ii) implies (i). Indeed consider a sequential colimit of:
\[f_k : \mathrm{Fin}(l_k) \to \mathrm{Fin}(l_{k+1})\]
Then:
\[\mathrm{colim}_k \mathrm{Fin}(l_k)  =  \left(\sum_{k:\N} \mathrm{Fin}(l_k)\right) / L\]
where $L$ is the equivalence relation generated by $(k,x) \sim (k+1,f_k(x))$. But $\sum_{k:\N} \mathrm{Fin}(l_k)$ is a decidable in $\N$ and the equivalence relation generated by a decidable relation on such a type is open.
\end{itemize}
\end{proof}

\begin{remark}
A proposition is overtly discrete if and only if it is open.
\end{remark}

\begin{theorem}
We have the following:
\begin{enumerate}[(i)]
\item Overtly dicrete types are stable under identity type and and sigma type.
\item Overly discrete types are stable under quotient by equivalence relation with value in overtly discrete types.
\item Overtly discrete type are stable under sequential colimits.
\item Overtly discrete types have local choice.
\end{enumerate}
\end{theorem}

\begin{proof}
We proceed as follows:
\begin{enumerate}[(i)]
\item For stability under identity types, we use that sequential colimits commutes with identity types. 

For stability under sigma, sequential colimits commutes with sigma so that by (iii) it is enough to show that overtly discrete types are stable under finite coproduct. But sequential colimits commutes with finite coproduct.

\item Clear from the alternative description in \cref{overtly-discrete-colimit-finite}.

\item TODO

\item By \cref{overtly-discrete-colimit-finite}, we even have a cover of any overtly discrete type by a decidable subset of $\N$, which is an overtly discrete type that has choice.
\end{enumerate}
\end{proof}

\begin{remark}
(ii) implies that the propositional truncation of an overtly discrete type is open.

(iii) implies that overtly discrete types are closed under countable coproducts.
\end{remark}