In \cref{def:schemes} we defined a scheme to be a type $X$ such that $X$
may be covered by finitely many open affine subtypes.
In this section, we will present general properies of schemes and a couple of common constructions for schemes.


\subsection{General Properties}

\begin{lemma}[using \axiomref{sqc},\axiomref{loc}]%
  \label{lem:intersection-of-all-opens}
  Let $X$ be a scheme and $x:X$, then for all $y:X$ the proposition
  \[ \prod_{U:X\to \Open}U(x)\to U(y) \]
  is equivalent to $\neg\neg (x=y)$.
\end{lemma}

\begin{proof}
  By \cref{prop:open-union-intersection},
  open proposition are always double-negation stable,
  which settles one implication.
  For the implication
  \[ \left(\prod_{U:X\to \Open}U(x)\to U(y)\right) \Rightarrow \neg\neg (x=y) \]
  we can assume that $x$ and $y$ are both inside an open affine $U$
  and use that the statement holds for affine schemes by \cref{lem:affine-intersection-of-all-opens}.
\end{proof}

\subsection{Glueing}

\begin{proposition}[using \axiomref{loc},\axiomref{sqc},\axiomref{Z-choice}]%
  Let $X,Y$ be schemes and $f:U\to X$, $g:U\to Y$ be embeddings with open images in $X$ and $Y$,
  then the pushout of $f$ and $g$ is a scheme.
\end{proposition}

\begin{proof}
  As we noted in \cref{MISSING}, such a pushout is always 0-truncated.
  Let $U_1,\dots,U_n$ be a cover of $X$ and $V_1,\dots,V_m$ be a cover of $Y$.
  By \cref{lem:qc-open-trans}, $U_i\cap U$ is open in $Y$,
  so we can use (large) pushout-recursion to construct a subtype $\tilde{U_i}$,
  which is open in the pushout and restricts to $U_i$ on $X$ and $U_i\cap U$ on $Y$.
  Symetrically we define $\tilde{V_i}$ and in total get an open finite cover of the pushout.
  The pieces of this new cover are equivalent to their counterparts in the covers of $X$ and $Y$,
  so they are affine as well.
\end{proof}

\subsection{Subschemes}

\begin{definition}
  Let $X$ be a scheme.
  A \notion{subscheme} of $X$ is a subtype $Y:X\to\Prop$,
  such that $\sum Y$ is a scheme.
\end{definition}

\begin{proposition}[using \axiomref{loc},\axiomref{sqc},\axiomref{Z-choice}]%
  \label{prop:open-subscheme}
  Any open subtype of a scheme is a scheme.
\end{proposition}

\begin{proof}
  Using \cref{thm:qc-open-affine-open}.
\end{proof}

\begin{proposition}[using \axiomref{sqc},\axiomref{loc},\axiomref{Z-choice}]%
  \label{prop:closed-subscheme}
  Any closed subtype $A:X\to \Prop$ of a scheme $X$ is a scheme.
\end{proposition}

\begin{proof}
  Any open subtype of $X$ is also open in $A$.
  So it is enough to show,
  that any affine open $U_i$ of $X$,
  has affine intersection with $A$.
  But $U_i\cap A$ is closed in $U_i$ and therefore affine by \cref{lem:closed-subtype-affine}.
\end{proof}

\subsection{Equality types}

\begin{lemma}%
  Let $X$ be an affine scheme and $x,y:X$,
  then $x=_Xy$ is an affine scheme
  and $((x,y):X\times X)\mapsto x=_Xy$
  is a closed subtype of $X\times X$.
\end{lemma}

\begin{proof}
  Any affine scheme is merely embedded into $\A^n$ for some $n:\N$.
  The proposition $x=y$ for elements $x,y:\A^n$ is equivalent to $x-y=0$,
  which is equivalent to all entries of this vector being zero.
  The latter is a closed proposition.
\end{proof}

\begin{lemma}%
  Let $X$ be a scheme.
  The equality type $x=_Xy$ is a scheme for all $x,y:X$.
\end{lemma}

\begin{proof}
  Let $x,y:X$ and
  $U\subseteq X$ be an affine open containing $x$.
  Then $U(y)\wedge x=y$ is equivalent to $x=y$.
\end{proof}

\subsection{Fiber products}

\begin{lemma}%
  Let $f:X\to Y$ and $U:Y\to\Prop$ open,
  then the \notion{preimage} $U\circ f:X\to\Prop$ is open.
\end{lemma}

\begin{proposition}[using \axiomref{loc},\axiomref{sqc},\axiomref{Z-choice}]%
  \label{prop:fiber-product-scheme}
  Let
  \[
    \begin{tikzcd}
      X & Z\ar[l,swap,"f"]\ar[r,"g"] & Y
    \end{tikzcd}
  \]
  be schemes, then the \notion{pullback} $X\times_Z Y$ is also a scheme.
\end{proposition}

\begin{proof}
  Let $W_1,\dots,W_n$ be a finite affine cover of $Z$.
  The preimages of $W_i$ under $f$ and $g$ are open
  and covered by fintely many affine open $U_{ik}$ and $V_{ij}$ by \cref{prop:open-subscheme}.
  This leads to the following diagram:
  \begin{center}
    \begin{tikzcd}
      X\times_Z Y\ar[rrr]\ar[ddd] & & & Y\ar[ddd] & \\
      & P_{ij}\ar[hook,lu]\ar[rrr,crossing over] &&& V_{ij}\ar[hook,lu]\ar[ddd] \\
      &&&& \\
      X\ar[rrr] & & & Z & \\
      & U_{i}\ar[rrr]\ar[hook,lu]\ar[from=uuu,crossing over] & & & W_i\ar[hook,lu]
    \end{tikzcd}
  \end{center}
  where the front and bottom square are pullbacks by definition.
  By pullback-pasting, the top is also a pullback,
  so all diagonal maps are embeddings.
  
  $P_{ij}$ is open, since it is a preimage of $V_{ij}$,
  which is open in $Y$ by \cref{lem:qc-open-trans}.
  It remains to show, that the $P_{ij}$ cover $X\times_Z Y$ and that $P_{ij}$ is a scheme.
  Let $x:X\times_Z Y$.
  For the image $w$ of $x$ in $W$, there merely is an $i$ such that $w$ is in $W_i$.
  The image of $x$ in $V_i$ merely lies in some $V_{ij}$,
  so $x$ is in $P_{ij}$.

  We proceed by showing that $P_{ij}$ is a scheme.
  Let $U_{ik}$ be a part of the finite affine cover of $U_i$.
  We repeat part of what we just did:
  \begin{center}
    \begin{tikzcd}
      P_{ij}\ar[rrr]\ar[ddd] & & & U_i\ar[ddd] & \\
      & P_{ijk}\ar[hook,lu]\ar[rrr,crossing over] &&& U_{ik}\ar[hook,lu]\ar[ddd] \\
      &&&& \\
      V_{ij}\ar[rrr] & & & W_i & \\
      & V_{ij}\ar[rrr]\ar[equal,lu]\ar[from=uuu,crossing over] & & & W_i\ar[equal,lu]
    \end{tikzcd}
  \end{center}

  So by \cref{lem:affine-fiber-product}, $P_{ijk}$ is affine.
  Repetition of the above shows, that the $P_{ijk}$ are open and cover $P_{ij}$.
\end{proof}

\subsection{Dependent sums}

\begin{theorem}[using \axiomref{loc},\axiomref{sqc},\axiomref{Z-choice}]%
  \label{thm:sigma-scheme}
  Let $X$ be a scheme and for any $x:X$, let $Y_x$ be a scheme.
  Then the dependent sum
  \[ \left((x:X)\times Y_x\right)\equiv \sum_{x:X}Y_x\]
  is a scheme.
\end{theorem}

\begin{proof}
  We start with an affine $X=\Spec A$ and $Y_x=\Spec B_x$.
  Locally on $U_i = D(f_i)$, for a Zariski-cover $f_1,\dots,f_l$ of $X$,
  we have $B_x=\Spec R[X_1,\dots,X_{n_i}]/(g_{i,x,1},\dots,g_{i,x,m_i})$
  with polynomials $g_{i,x,j}$.
  In other words, $B_x$ is the closed subtype of $\A^{n_i}$
  where the functions $g_{i,x,1},\dots,g_{i,x,m_i}$ vanish.
  By \cref{lem:affine-fiber-product}, the product
  \[ V_i\colonequiv U_i\times \A^{n_i}\]
  is affine.
  The type $(x:U_i)\times \Spec B_x\subseteq V_i$ is affine,
  since it is the zero set of the functions
  \[ ((x,y):V_i)\mapsto g_{i,x,j}(y) \]
  Furthermore, $W_i\colonequiv (x:U_i)\times \Spec B_x$
  is open in $(x:X)\times Y_x$,
  since $W_i(x)$ is equivalent to $U_i(\pi_1(x))$,
  which is an open proposition.

  This settles the affine case.
  We will now assume, that
  $X$ and all $Y_x$ are general schemes.
  We pass again to a cover of $X$ by affine open $U_1,\dots,U_n$.
  We can choose the latter cover,
  such that for each $i$ and $x:U_i$, the $Y_{\pi_1(x)}$
  are covered by $l_i$ many open affine pieces $V_{i,x,1},\dots,V_{i,x,l_i}$
  (by \cref{thm:boundedness}).
  Then $W_{i,j}\colonequiv(x:U_i)\times V_{i,x,j}$ is affine by what we established above.
  It is also open.
  To see this, let $(x,y):((x:X)\times Y_x)$.
  We want to show, that $(x,y)$ being in $W_{i,j}$ is an open proposition.
  We have to be a bit careful, since the open proposition
  $V_{i,x,j}$ is only defined, for $x:U_i$.
  So the proposition we are after is $(z:U_i(x,y))\times V_{i,z,j}(y)$.
  But this proposition is open by \cref{lem:qc-open-sigma-closed}.
\end{proof}

\begin{corollary}
  \label{cor:scheme-map-classification}
  Let $X$ be a scheme.
  For any other scheme $Y$ and any map $f:Y\to X$,
  the fiber map
  $(x:X)\mapsto \fib_f(x)$
  has values in the type of schemes $\Sch$.
  Mapping maps of schemes to their fiber maps,
  is an equivalence of types
  \[ \left(\sum_{Y:\Sch}(Y\to X)\right)\simeq (X\to \Sch)\rlap{.}\]
\end{corollary}

\begin{proof}
  By univalence, there is an equivalence
  \[ \left(\sum_{Y:\Type}(Y\to X)\right)\simeq (X\to \Type)\rlap{.} \]
  From left to right, the equivalence is given by turning a $f:Y\to X$ into $x\mapsto \fib_f(x)$,
  from right to left is given by taking the depedent sum.
  So we just have to note, that both constructions preserve schemes.
  From left to right, this is \cref{prop:fiber-product-scheme}, from right to left,
  this is \cref{thm:sigma-scheme}.
\end{proof}

Subschemes are classified by propositional schemes:

\begin{corollary}
  Let $X$ be a scheme.
  $Y:X\to\Prop$ is a subscheme,
  if and only if $Y_x$ is a scheme for all $x:X$.
\end{corollary}

\begin{proof}
  Restriction of \cref{cor:scheme-map-classification}.
\end{proof}
