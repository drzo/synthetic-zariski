In \cref{def:schemes} we defined a scheme to be a type $X$ such that $X$
may be covered by finitely many open affine subtypes.
In this section, we will present a couple of common constructions for schemes.

\subsection{Glueing}
\begin{proposition}[(\axiomref{loc}),(\axiomref{sqc}),(\axiomref{Z-choice})]%
  Let $X,Y$ be schemes and $f:U\to X$, $g:U\to Y$ be embeddings with open images in $X$ and $Y$,
  then the pushout of $f$ and $g$ is a scheme.
\end{proposition}
\begin{proof}
  As we noted in \cref{MISSING}, such a pushout is always 0-truncated.
  Let $U_1,\dots,U_n$ be a cover of $X$ and $V_1,\dots,V_m$ be a cover of $Y$.
  By \cref{lem:qc-open-trans}, $U_i\cap U$ is open in $Y$,
  so we can use (large) pushout-recursion to construct a subtype $\tilde{U_i}$,
  which is open in the pushout and restricts to $U_i$ on $X$ and $U_i\cap U$ on $Y$.
  Symetrically we define $\tilde{V_i}$ and in total get an open finite cover of the pushout.
  The pieces of this new cover are equivalent to their counterparts in the covers of $X$ and $Y$,
  so they are affine as well.
\end{proof}