In \cref{def:schemes} we defined a scheme to be a type $X$ such that $X$
may be covered by finitely many open affine subtypes.
In this section, we will present a couple of common constructions for schemes.

\subsection{Glueing}

\begin{proposition}[(\axiomref{loc}),(\axiomref{sqc}),(\axiomref{Z-choice})]%
  Let $X,Y$ be schemes and $f:U\to X$, $g:U\to Y$ be embeddings with open images in $X$ and $Y$,
  then the pushout of $f$ and $g$ is a scheme.
\end{proposition}

\begin{proof}
  As we noted in \cref{MISSING}, such a pushout is always 0-truncated.
  Let $U_1,\dots,U_n$ be a cover of $X$ and $V_1,\dots,V_m$ be a cover of $Y$.
  By \cref{lem:qc-open-trans}, $U_i\cap U$ is open in $Y$,
  so we can use (large) pushout-recursion to construct a subtype $\tilde{U_i}$,
  which is open in the pushout and restricts to $U_i$ on $X$ and $U_i\cap U$ on $Y$.
  Symetrically we define $\tilde{V_i}$ and in total get an open finite cover of the pushout.
  The pieces of this new cover are equivalent to their counterparts in the covers of $X$ and $Y$,
  so they are affine as well.
\end{proof}

\subsection{Open subschemes}

\begin{proposition}[(\axiomref{loc}),(\axiomref{sqc}),(\axiomref{Z-choice})]%
  \label{prop:open-subscheme}
  Any open subtype of a scheme is a scheme.
\end{proposition}

\begin{proof}
  Using \cref{thm:qc-open-affine-open}.
\end{proof}

\subsection{Closed subschemes}

\begin{proposition}
  \label{prop:closed-subscheme}
  Any closed subtype $A:X\to \Prop$ of a scheme $X$ is a scheme.
\end{proposition}

\begin{proof}
  Any open subtype of $X$ is also open in $A$.
  So it is enough to show,
  that any affine open $U_i$ of $X$,
  has affine intersection with $A$.
  But $U_i\cap A$ is closed in $U_i$ and therefore affine.
\end{proof}

\subsection{Fiber products}

\begin{lemma}
  Let $f:X\to Y$ and $U:Y\to\Prop$ open,
  then the \notion{preimage} $U\circ f:X\to\Prop$ is open.
\end{lemma}

\begin{proposition}[(\axiomref{loc}),(\axiomref{sqc}),(\axiomref{Z-choice})]%
  Let
  \[
    \begin{tikzcd}
      X & Z\ar[l,swap,"f"]\ar[r,"g"] & Y
    \end{tikzcd}
  \]
  be schemes, then the \notion{pullback} $X\times_Z Y$ is also a scheme.
\end{proposition}

\begin{proof}
  Let $W_1,\dots,W_n$ be a finite affine cover of $Z$.
  The preimages of $W_i$ under $f$ and $g$ are open
  and covered by fintely many affine open $U_{ik}$ and $V_{ij}$ by \cref{prop:open-subscheme}.
  This leads to the following diagram:
  \begin{center}
    \begin{tikzcd}
      X\times_Z Y\ar[rrr]\ar[ddd] & & & Y\ar[ddd] & \\
      & P_{ij}\ar[hook,lu]\ar[rrr,crossing over] &&& V_{ij}\ar[hook,lu]\ar[ddd] \\
      &&&& \\
      X\ar[rrr] & & & Z & \\
      & U_{i}\ar[rrr]\ar[hook,lu]\ar[from=uuu,crossing over] & & & W_i\ar[hook,lu]
    \end{tikzcd}
  \end{center}
  where the front and bottom square are pullbacks by definition.
  By pullback-pasting, the top is also a pullback,
  so all diagonal maps are embeddings.
  
  $P_{ij}$ is open, since it is a preimage of $V_{ij}$,
  which is open in $Y$ by \cref{lem:qc-open-trans}.
  It remains to show, that the $P_{ij}$ cover $X\times_Z Y$ and that $P_{ij}$ is a scheme.
  Let $x:X\times_Z Y$.
  For the image $w$ of $x$ in $W$, there merely is an $i$ such that $w$ is in $W_i$.
  The image of $x$ in $V_i$ merely lies in some $V_{ij}$,
  so $x$ is in $P_{ij}$.

  We proceed by showing that $P_{ij}$ is a scheme.
  Let $U_{ik}$ be a part of the finite affine cover of $U_i$.
  We repeat part of what we just did:
  \begin{center}
    \begin{tikzcd}
      P_{ij}\ar[rrr]\ar[ddd] & & & U_i\ar[ddd] & \\
      & P_{ijk}\ar[hook,lu]\ar[rrr,crossing over] &&& U_{ik}\ar[hook,lu]\ar[ddd] \\
      &&&& \\
      V_{ij}\ar[rrr] & & & W_i & \\
      & V_{ij}\ar[rrr]\ar[equal,lu]\ar[from=uuu,crossing over] & & & W_i\ar[equal,lu]
    \end{tikzcd}
  \end{center}

  So by \cref{lem:affine-fiber-product}, $P_{ijk}$ is affine.
  Repetition of the above shows, that the $P_{ijk}$ are open and cover $P_{ij}$.
\end{proof}