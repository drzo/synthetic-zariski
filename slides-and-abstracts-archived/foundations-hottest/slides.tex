% latexmk -pdflatex='xelatex %O %S' -pvc -pdf slides.tex
\documentclass{beamer}

\beamertemplatenavigationsymbolsempty

% für xelatex:
\usepackage{xltxtra}
\usepackage{unicode-math}

% literatur
\usepackage[backend=biber,style=alphabetic]{biblatex}

\addbibresource{../util/literature.bib}

\usepackage{../util/zariski}

\usepackage{csquotes}
\usepackage{cancel}
\usepackage{tabularx}
\usepackage{hyperref}
\usepackage{tikz}
\usetikzlibrary{cd,arrows,shapes,calc,through,backgrounds,matrix,trees,decorations.pathmorphing,positioning,automata}
\usepackage{graphicx}
\usepackage{color}

\usepackage{mathpartir}
\newcommand{\yields}{\vdash}
\newcommand{\cbar}{\, | \,}


% für tabellen
\usepackage{booktabs}

\title{A Foundation for Synthetic Algebraic Geometry}
\author[Author, Anders] 
{Felix Cherubini, Thierry Coquand, Matthias Hutzler}
\date{HoTTEST, Fall 2023}

\begin{document}

\begin{frame}
  \titlepage
\end{frame}

\begin{frame}
  \frametitle{Related work}
  We continue work of Anders Kock and Ingo Blechschmidt using ideas of David Jaz Myers. \\
  \vspace{0.4cm}
  \pause
  This work is part of a larger project with many collaborators. A lot of the things we have figured out are on github:
  \begin{center}
    \url{https://github.com/felixwellen/synthetic-zariski/}
  \end{center}
  In addition to the authors this also contains contributions of
  \begin{center}
    \begin{tabularx}{\textwidth}{XX}
      Peter Arndt &
      Ingo Blechschmidt \\
      Hugo Moeneclaey &
      Marc Nieper-Wißkirchen \\
      David Wärn 
    \end{tabularx}
  \end{center}
\end{frame}

\begin{frame}
  \vspace{1cm}
  \begin{center}
    \begin{tikzpicture} 
      [ 
      node distance=.8cm, 
      circled/.style={draw, ellipse, ultra thick, fill=blue!12},
      edge from parent/.style={very thick,draw=black,-latex},
      plaintext/.style={}
      ]

      \tikzstyle{level 1}=[sibling angle=81,level distance=4cm]

      \node[circled] (hott) {HoTT+Data+Axioms} [counterclockwise from=207]
      child { 
        node[circled] (grp) {$\infty$-Groupoids} 
        edge from parent [-latex] node (grp) 
        {
          \begin{tikzpicture}[scale=0.8, rotate=25]
            \draw[-, red, ultra thick] (0,0) to (1,1);
            \draw[-, red, ultra thick] (1,0) to (0,1);
          \end{tikzpicture}
        } 
      }
      child { node[matrix, circled, inner sep=0pt] (smgrp) 
        {
          \node[circled, minimum height=0.7cm, minimum width=3cm] (mfd) {Schemes};
          \node[below of=mfd, text width=3cm,
          node distance=0.9cm] 
          {``Constructive Zariski-sheaves''}; \\
        }
      }
      ;
    \end{tikzpicture}
  \end{center}
  \vspace{1cm}
  {\footnotesize * Schemes = quasi-compact, quasi-separated schemes of finite type}
\end{frame}

\begin{frame}
  \frametitle{Now: Internal}
  $R$ is a given commutative ring from now on. \\
  \vspace{0.4cm}
  \pause

  $V\coloneq \{(x,y):R^2\mid x^2+y^2=1 \}$ is a geometric object. \\

  \pause
  Adding equations like $0=0$, $x(x^2+y^2)=x$ does not change the type $V$. \\
  \pause
  $V$ is determined by the $R$-algebra $A\coloneq R[X,Y]/(X^2+Y^2-1)$.
  We can recover it: $V=\Hom_{\Alg{R}}(A,R)$. \\
  \vspace{0.4cm}
  \pause
  \begin{definition}
    \begin{enumerate}[(i)]
    \item An $R$-algebra is finitely presented (fp) if it is merely $R[X_1,\dots,X_n]/(P_1,\dots,P_l)$.
    \item $\Spec(A)\coloneq \Hom_{\Alg{R}}(A,R)$ is the \emph{spectrum} of an fp $R$-algebra $A$.
    \item Any $X$ such that there is an $A$ with $X=\Spec(A)$ is called \emph{affine scheme}.
    \end{enumerate}
  \end{definition}
  
  
   
\end{frame}

\begin{frame}
  \frametitle{Classical vs synthetic}

  How can we make the functor
  \[ A \mapsto \Spec(A) \]
  fully faithful?
  \vspace{5mm}

  \begin{tabularx}{\textwidth}{X|X}
    Classical algebraic geometry
    &
    Synthetic algebraic geometry
    \\ \midrule
    Endow $\Spec(A)$ with additional structure:
    \begin{itemize}
      \item
        Zariski topology
      \item
        structure sheaf $\mathcal{O}_{\Spec(A)}$
    \end{itemize}
    &
    \textbf{Axiom (SQC)\footnote{%
    \enquote{Synthetic Quasi-Coherence}, due to Ingo Blechschmidt}.}
    The map
    \begin{align*}
      A \to R^{\Spec A} \\
      a\mapsto (\varphi\mapsto \varphi(a))
    \end{align*}
    is an equivalence
    for any finitely presented $R$-algebra $A$.
  \end{tabularx}
\end{frame}
 
\begin{frame}
  \frametitle{Basic consequences of SQC}

  \[ A \xrightarrow{\sim} R^{\Spec A}\]

  \vspace{5mm}
  \begin{itemize}
    \item
      $\Spec(R[X]) = R$.
      Thus:
      $R[X] \xrightarrow{\sim} R^R$
  \end{itemize}

  \pause
  \vspace{5mm}
  If $\Spec(A) = \varnothing$,
  then $A = R^\varnothing = 0$.

  \vspace{2.5mm}
  \begin{itemize}
    \item
      $\Spec(R/(r)) = (r = 0)$.
      Thus:
      if $r \neq 0$,
      then $r$ is invertible.
    \item
      $\Spec(R[r^{-1}]) = (\text{$r$ is invertible})$.
      Thus:
      if $r$ is not invertible,
      then $r$ is nilpotent.
  \end{itemize}

  \pause
  \vspace{5mm}
  \textbf{Axiom:} The ring $R$ is local.

  \vspace{2.5mm}
  \begin{itemize}
    \item      If $r_1, \dots, r_n : R$ are not all zero,
      then some $r_i$ is invertible.
  \end{itemize}
\end{frame}

\begin{frame}
  \frametitle{An affine scheme}
  Let $f:A$.
  \begin{align*}
    D(f)&\coloneq\Spec(A_f)=\Spec(A[X]/(fX-1)) \\
        &=\{x:\Spec(A)\mid x(f)\text{ invertible}\} 
  \end{align*}
  \pause
  Or: Let $f:\Spec(A)\to R$, then:
  \begin{align*}
    D(f)&=\{x:\Spec(A)\mid f(x)\text{ invertible}\} \\
        &=\{x:\Spec(A)\mid f(x)\neq 0\}
  \end{align*}
  \pause
  $D(f)$ is called a \emph{standard-open}. \\
  \pause
  Any subset which is merely a finite union of $D(f)$s is called \emph{global-open}.
  \vspace{0.2cm}
  \pause
  Let $f_1,\dots,f_n:A$. Then $\Spec(A)=\bigcup_i D(f_i)$ if and only if $(f_1,\dots,f_n)=(1)$.
\end{frame}

\begin{frame}
  \frametitle{Closed and open propositions}

  For $r_1, \dots, r_n : R$ we have the propositions
  \[ V(r_1, \dots, r_n) \coloneq (r_1 = \dots = r_n = 0) \rlap{,}\]
  \[ D(r_1, \dots, r_n) \coloneq (r_1\neq 0\lor \dots \lor r_n\neq 0) \rlap{.}\]

  Then define:
  \[ \mathrm{closedProp} \coloneq
     \sum_{p : \mathrm{hProp}} \exists r_1, \dots, r_n.\, (p = V(r_1, \dots, r_n))
  \]
  \[ \mathrm{openProp} \coloneq
     \sum_{p : \mathrm{hProp}} \exists r_1, \dots, r_n.\, (p = D(r_1, \dots, r_n))
  \]

  \vspace{5mm}
  A \emph{closed subtype} of $X$
  is a map $X \to \mathrm{closedProp}$.\\
  An \emph{open subtype} of $X$
  is a map $X \to \mathrm{openProp}$.
\end{frame}

\begin{frame}
  \frametitle{Zariski-local choice}

  \vspace{2.5mm}
  \textbf{Axiom (Zariski-local choice):}\\
  For every surjective $\pi$, there merely exist local sections $s_i$
  \[ \begin{tikzcd}[ampersand replacement=\&, column sep=small]
    \& E \ar[d, two heads, "\pi"] \\
    D(f_i) \ar[r, hook] \ar[ur, bend left, dashed, "s_i"] \& \Spec(A)
  \end{tikzcd} \]
  with $f_1, \dots, f_n : A$ such that $(f_1,\dots,f_n)=(1)$.

  \vspace{0.4cm}
  \emph{Alternative formulation:} \\

  \textbf{Axiom (Zariski-local choice):}\\
  Let $B:\Spec(A)\to \mathcal U$ be such that $(x:\Spec(A))\to\|B(x)\|$.
  Then there merely are $n:\N$, $f_1, \dots, f_n : A$ such that $(f_1,\dots,f_n)=(1)$ and
  $s_i:(x:D(f_i))\to B(x)$.
\end{frame}

\begin{frame}
  \frametitle{Pointwise-global principle}
  \begin{theorem}
    Let $f:A$.
    \begin{enumerate}[(a)]
    \item A global-open $U\subseteq D(f)$ is global-open in $\Spec(A)$
    \item A subset $U\subseteq \Spec(A)$ is open if and only if it is global-open.
    \end{enumerate}
  \end{theorem}
  \begin{proof}[Proof-Idea]
    Let $U\subseteq \Spec(A)$ be open. 
    \pause
    That means we have
    \[
      t:\prod_{x:\Spec(A)}\Big\|\sum_{n:\N} \sum_{r_1,\dots,r_n:R}U(x)=(r_1\neq 0\vee\dots\vee r_n\neq 0)\Big\|
    \]
    By something called ``boundedness'', we can assume we have a global ``$n:\N$'' and by Zariski-choice we have
    \[
      s_i:(x:D(f_i))\to \sum_{r_1,\dots,r_n:R}U(x)=(r_1\neq 0\vee\dots\vee r_n\neq 0)
    \]
  \end{proof}
\end{frame}

\begin{frame}
  \frametitle{Schemes}
  A type $X$ is a \emph{scheme} if
  there exist $U_1, \dots, U_n : X \to \mathrm{openProp}$
  such that $X = \bigcup_i U_i$
  and every $U_i$ is an affine scheme.

  \pause
  \vspace{5mm}
  \textbf{Example.}
  \emph{Projective $n$-space}:
  \begin{align*}
    \bP^n
    &\coloneq \{x : R^{n+1}\mid x\neq 0\}/\approx\text{ where $(x\approx y)\coloneq \exists \lambda:R. \lambda x= y$} \\
    &= \{\,\text{submodules $L\subseteq R^{n+1}$ such that $\|L=R^1\|$}\,\}
  \end{align*}
  is a scheme, since
  \[
    U_i([x]) \coloneq \text{($x_i$ is invertible)}
  \]
  is an open affine cover.
\end{frame}

\begin{frame}
  \frametitle{Line bundles}

  The type
  \[ \mathrm{Lines} \coloneq \sum_{L : R\text{-Mod}} \lVert L = R^1 \rVert \]
  has a wild group structure:
  \begin{itemize}
    \item
      $L \otimes L'$ is again a line
    \item
      $L^{\vee} \coloneq \Hom(L, R^1)$ is the inverse
  \end{itemize}

  \pause
  \vspace{5mm}
  A \emph{line bundle} on $X$ is a map $X \to \mathrm{Lines}$.

  \textbf{Example.}
  tautological line bundle , $[x]\mapsto R\langle x \rangle:\bP^n\to \mathrm{Lines}$ 

  \pause
  \vspace{5mm}
  The \emph{Picard group} of $X$ is
  \[ \mathrm{Pic}(X) \coloneq \lVert X \to \mathrm{Lines} \rVert_{\text{set}} \rlap{.}\]

  \vspace{5mm}
  (In fact, $\mathrm{Lines} = K(R^{\times}, 1)$
  and $\mathrm{Pic}(X) = H^1(X, R^{\times})$.)
\end{frame}

\begin{frame}
  \vspace{0.25cm}
  For $A : X \to \mathrm{Ab}$, define \emph{cohomology} as:
  \[ H^n(X, A) :\equiv \Big\| \prod_{x:X}K(A_x,n) \Big\|_{\mathrm{set}} \]
  
  \pause
  Good because:
  \begin{itemize}
  \item $\prod$-type.
  \item $\|\_\|_{\mathrm{set}}$ is a modality.
  \item Homotopy group: $H^n(X,A)=\pi_{k}(\prod_{x:X}K(A,n+k))$.
  \end{itemize}

  \pause
  Non-trivial for $X:\mathrm{Set}$ because:

  $X=$ Pushout of sets $U\leftarrow Y\to V$, \\
  Then a ``cohomology class'' $X\to K(A,1)$ is given by:
  \begin{itemize}
  \item Maps $f:U\to K(A,1)$, $g:V\to K(A,1)$.
  \item And $h:(x:Y)\to f(x)=g(x)$, which is essentially a map $Y\to A$,
    if $U$ and $V$ don't have higher cohomology...
  \end{itemize}
\end{frame}

\begin{frame}
  \frametitle{Zariski-Choice and Cohomology}  
  Let $X=\Spec(A)$ and $M:X\to R\text{-Mod}$ such that $((x:D(f))\to M_x)=((x:X)\to M)_f$, then
  \[ H^1(X,M)=0 \]
  \pause
  \textbf{Proof:} For $T: (x:X)\to K(M_x,1)$ we have to show $\|(x:X)\to T_x=\ast\|$.
  \pause
  By connectedness of the $K(M_x,1)$ we have $(x:X)\to \|T_x=\ast\|$.
  \pause
  \textbf{Zariski-local choice} merely gives us covering $f_1,\dots,f_n:A$, such that for each $i$ we have
  \[
    s_i:(x:D(f_i)) \to T_x=\ast
    \rlap{.}
  \]
  \pause
  So for $t_{ij}(x):\equiv s_j(x)^{-1}\cdot s_i(x)$ we have $t_{ij}+t_{jk}=t_{ik}$.
  \pause
  By algebra, we get $u_i:(x:D(f_i))\to M_x$ with $t_{ij}=u_i-u_j$.
  Then the $\tilde{s}_i:\equiv s_i-u_i$ glue to a global trivialization.
\end{frame}

\begin{frame}
  \frametitle{Results of the larger project}
  \begin{itemize}
  \item Vanishing of $H^n(\Spec(A),M)$ for $n>0$ and \v{C}ech-Cohomology.
  \item Serre's  theorem on Affineness: If all $H^1(X,M)$ vanish, then $X$ is affine.
  \item Smooth schemes are locally standard smooth.
  \item Closed subsets of $\bP^n$ are compact as defined by Martín Escardó in synthetic topology.
  \item $\bP^\infty$ is $\A^1$-equivalent to $BR^\times$.
  \item A subcanonical candidate for the synthetic fppf-topology.
  \item Stacks...
  \end{itemize}
\end{frame}

\begin{frame}
  \vspace{0.5cm}
  \begin{center}
    Thank you! \\
  \end{center}
  \pause
  \vspace{1cm}
  If you want to know more:
  \begin{itemize}
  \item There is a workshop in Gothenburg planned for 
    \begin{center}
      11-15 March 2024. 
    \end{center}
  \item
    The github page mentioned above:
    \begin{center}
      \url{https://github.com/felixwellen/synthetic-zariski/} 
    \end{center}
  \item
    A formalization project:
    \begin{center}
      \url{https://github.com/felixwellen/synthetic-geometry/} 
    \end{center}
  \end{itemize}
\end{frame}

\end{document}
