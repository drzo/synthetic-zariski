% latexmk -pdflatex -pvc geerligs.tex
% This is a template for submissions to HoTT 2023.
% The file was created on the basis of easychair.tex,v 3.5 2017/03/15 
%
% Before submitting, rename the resulting pdf file to "yourname.pdf" 
%
\documentclass[letterpaper]{../../util/easychair}
\usepackage{doc}
\usepackage[expansion=true,protrusion=true]{microtype}
%
\newcommand{\easychair}{\textsf{easychair}}
\newcommand{\miktex}{MiK{\TeX}}
\newcommand{\texniccenter}{{\TeX}nicCenter}
\newcommand{\makefile}{\texttt{Makefile}}
\newcommand{\latexeditor}{LEd}

% some stuff from ../util/zarisky.cls:
\RequirePackage{amsmath,amssymb,mathtools}
\newtheorem{axiom}{Axiom}

\RequirePackage{tikz}
\usetikzlibrary{arrows, cd, babel}

% Referenzen
\RequirePackage[backend=biber,style=alphabetic, backref, backrefstyle=none]{biblatex}
\addbibresource{../../util/literature.bib}

%Numbers for axiom in abstract


% content of ../util/zarisky.sty:
% names for types
\newcommand{\R}{\mathbb{R}}
\newcommand{\Z}{\mathbb{Z}}
\newcommand{\N}{\mathbb{N}}
\newcommand{\Bool}{\mathrm{Bool}}
\DeclareMathOperator{\Fin}{Fin}
\newcommand{\unit}{\mathbf{1}}
\newcommand{\two}{\mathbf{2}}
\newcommand{\isContr}{\mathrm{isContr}}
\newcommand{\isProp}{\mathrm{isProp}}
\newcommand{\isSet}{\mathrm{isSet}}
\newcommand{\isEquiv}{\mathrm{isEquiv}}
\newcommand{\qinv}{\mathrm{qinv}}
\newcommand{\mU}{\mathcal{U}}
\newcommand{\Eq}[1]{\mathrm{Eq}_{#1}}
\newcommand{\isNType}[1]{\mathrm{is}\mbox{-}{#1}\mbox{-}\mathrm{type}}
\newcommand{\nType}[1]{#1\mbox{-}\mathrm{Type}}
\newcommand{\Type}{\mathrm{Type}}
\newcommand{\Prop}{\mathrm{Prop}}
\newcommand{\Open}{\mathrm{Open}}
\newcommand{\propTrunc}[1]{\lVert #1 \rVert}

\newcommand{\Boole}{\mathsf{Boole}}
\newcommand{\Stone}{\mathsf{Stone}}
\newcommand{\CHaus}{\mathsf{CHaus}}
\newcommand{\Noo}{\N_{\infty}}
\newcommand{\Closed}{\mathsf{Closed}}
\newcommand{\ints}{\mathbb{Z}}

% names for terms
\newcommand{\id}{\mathrm{id}}
\newcommand{\refl}{\mathrm{refl}}
\newcommand{\pair}{\mathrm{pair}}
\newcommand{\FunExt}{\mathrm{FunExt}}
\newcommand{\transp}{\mathrm{tr}}
\newcommand{\transpconst}{\mathrm{tconst}}
\newcommand{\ua}{\mathrm{ua}}
\newcommand{\fib}{\mathrm{fib}}

% category theory
\newcommand{\Hom}{\mathrm{Hom}}
\newcommand{\Sh}{\mathrm{Sh}}
\newcommand{\yo}{\mathrm{y}}

% algebra
\newcommand{\inv}{\mathrm{inv}}
\newcommand{\divides}{\mathbin{|}}
\DeclareMathOperator{\AbGroup}{Ab}
\DeclareMathOperator{\im}{im}
\DeclareMathOperator{\coker}{coker}
\newcommand{\Tors}[1]{#1\text{-}\mathrm{Tors}}
\newcommand{\Mod}[1]{#1\text{-}\mathrm{Mod}}
\newcommand{\Vect}[2]{#1\text{-}\mathrm{Vect}_{#2}}
\newcommand{\fpMod}[1]{#1\text{-}\mathrm{Mod}_{\text{fp}}}
\newcommand{\Alg}[1]{#1\text{-}\mathrm{Alg}}

% algebraic geometry
\DeclareMathOperator{\Spec}{Spec}
\DeclareMathOperator{\Sch}{\mathrm{Sch}_{qc}}
\newcommand{\A}{\mathbb{A}}
\newcommand{\D}{\mathbb{D}}
\newcommand{\bP}{\mathbb{P}}


% misc
\newcommand{\notion}[1]{\emph{#1}\index{#1}}
\providecommand*\colonequiv{\vcentcolon\mspace{-1.2mu}\equiv}
\newcommand{\ignore}[1]{}
\newcommand{\rednote}[1]{{\color{red}(#1)}}


%
\title{Synthetic Stone Duality 
}

% Authors are joined by \and. 
% Their affiliations are given by \inst, which indexes
% into the list defined using \institute
%
\author{
Felix Cherubini %\inst{1}
% uncomment the following for multiple authors.
\and 
 Thierry Coquand% \inst{2}%
\and 
 Freek Geerligs% \inst{3}%
\thanks{Speaker.}%
\and
 Hugo Moeneclaey %\inst{4}%
}

% Institutes for affiliations are also joined by \and,
\institute{
  University of Gothenburg and Chalmers University of Technology, Gothenburg, Sweden%\\
}
%  \email{felix.cherubini@posteo.de}
%% uncomment the following for multiple authors.
%\and
%  University of Gothenburg\\
%  \email{Thierry.Coquand@cse.gu.se}
%\and
%  University of Gothenburg\\
%  \email{geerligs@chalmers.se}
%\and
%  University of Gothenburg\\
%  \email{hugomo@chalmers.se}
%}

%  \authorrunning{} has to be set for the shorter version of the authors' names;
% otherwise a warning will be rendered in the running heads. When processed by
% EasyChair, this command is mandatory: a document without \authorrunning
% will be rejected by EasyChair
\authorrunning{Cherubini, Coquand, Geerligs and Moeneclaey}

% \titlerunning{} has to be set to either the main title or its shorter
% version for the running heads. When processed by
% EasyChair, this command is mandatory: a document without \titlerunning
% will be rejected by EasyChair
\titlerunning{Synthetic Stone Duality}

\begin{document}
\maketitle
%We present some work in progress, 
%applying lessons about the Zariski topos from synthetic algebraic geometry \cite{draft} 
%to the topos of light condensed sets \cite{Scholze}. 
%Specifically, we 
%We propose an axiomatization of light condensed sets \cite{Scholze}
%within a univalent homotopy type theory, 
%similar to the axiomatization of Zariski (higher) topos in synthetic algebraic geometry 
%\cite{draft}. 
We propose a variation of the axiomatization of Zariski (higher) topos in synthetic algebraic geometry \cite{draft}
to an axiomatization of (separable) Stone spaces, with Stone duality, within a univalent type theory.  
The roles of affine schemes and schemes are taken over by Stone spaces and compact Hausdorff spaces respectively. 
In the theory, we can show that any map between compact Hausdorff spaces is continuous, hence the negation of WLPO holds. 
However, we can show that LLPO and Markov's principle do hold. 
We conjecture that internal results on light condensed sets \cite{Dagur,Scholze,Condensed} can be shown using univalent
type theory extended by these axioms.
Furthermore, this work can be seen as a variation of the work in \cite{XuE13}. We also expect to build a constructive
sheaf model of these axioms, similar to the constructive model of synthetic algebraic geometry presented in \cite{draft}.

\medskip

%\section{Axioms} 
We denote the type of countably presented Boolean algebras by $\Boole$.
Given a Boolean algebra $B$, we define $Sp(B)$, the spectrum of $B$ as the set of Boolean morphisms from $B$ to $2$.  
A type of the form $Sp(B)$ for $B:\Boole$ is called Stone.
%
Two motivating examples of elements of $\Boole$ are as follows:
 \begin{itemize}
   \item $C$ is the free Boolean algebra on countably many generators $(p_n)_{n\in\mathbb N}$. 
     The corresponding set $Sp(C)$ is Cantor space $2^\mathbb N$. 
   \item 
%     The Boolean algebra $B_\infty$ of finite and cofinite subsets of the natural numbers can be presented as 
     The Boolean algebra $ B_\infty$ is %given by 
     the quotient of $C$ by the relations $p_n\wedge p_m = 0$ for $n\neq m$.  
%     A map $B\infty \to 2$ assigns value $1$ to at most one $p_n$. 
     A term of $Sp(B_\infty)$ sends $p_n$ to $1$ for at most one $n$. 
     For this reason, $Sp(B_\infty)$ is denoted $\Noo$. 
%     The element of $\Noo$ sending every $p_n$ to $0$ is denoted $\infty$, 
%     and elements sending $p_n$ to $1$ correspond to $n\in\mathbb N$. 
  \end{itemize} 

\begin{axiom}[Stone duality]
  For any 
  $B:\Boole$, 
  %countably presented Boolean algebra, 
  the evaluation map $B \to 2^{Sp(B)}$ is an isomorphism. 
\end{axiom}
It follows from Stone duality that being Stone is a proposition and $Sp$ defines an embedding from $\Boole$ 
to any universe $\mathcal U$. We denote its image $\Stone$. 
%
%these categories are anti-equivalent. 
%
Any $X:Stone$ has a topology where basic clopens are given by decidable subsets. 
%By Stone duality, for each basic clopen $D$ of $X=Sp(B)$ there is some $b:B$ such that $D$ is
%exactly the set of morphisms sending $b$ to $1$. 
Using Stone duality we can show that any map from a Stone space to $\mathbb N$ is uniformly continuous. 
Both $\Stone$ and $\Boole$ have a natural structure of a category, and 
Stone duality gives that $Sp$ induces a dual equivalence between them. 
%So maps $Sp(B')\to Sp(B)$ correspond to maps $B \to B'$. 

\begin{axiom}[Surjections are Formal Surjections]
  A map $Sp(B')\to Sp(B)$ is surjective iff the corresponding Boolean map $B \to B'$ is injective.
\end{axiom} 
%Note that in the category of Boolean algebras, a map is injective iff it is mono. 
%Hence the above axiom can also be stated as surjections being exactly epimorphisms. 
%Note also that if 1\neq 0, then 2 -> B is injective, hence Sp(B) -> 2 surjective, hence
%Sp(B) inhabited. 
%\begin{axiom}[Inhabited spectra of nontrivial algebras]
%  For any $B:\Boole$ with $1 \neq 0$, $Sp(B)$ is merely inhabited. 
%\end{axiom} 
%\begin{axiom}[Stone truncation]
%  For any $X: \Stone$ we have $\neg \neg X \to  ||X||$.
%\end{axiom} 
Using this axiom, we can show that if $B$ is nontrivial, $Sp(B)$ is merely inhabited.
%
%
Note that the sum of the maps $\Noo \to \Noo$ sending $n$ to $2n,2n+1$ respectively has no section. 
However, we can use the above axiom to show that it is surjective. 
This implies that $\Noo$ is not projective and that LLPO holds. 
However, we can also show the negation of WLPO from Axiom 1.  

\medskip

Analogously to synthetic algebraic geometry, we need an axiom of local choice. 
\begin{axiom}[Local choice]
  Given $X$ Stone, $E,F$ arbitrary types, a map $X \to F$ and $E\twoheadrightarrow F$ surjective, 
  there is some $Y$ Stone,
    a surjection $Y \twoheadrightarrow X$ and a map $Y\to E$ such that the following diagram commutes:
    \begin{equation*}\begin{tikzcd}
      Y \arrow [d, two heads,dashed] \arrow [r,dashed] & E \arrow[d,""',two heads]\\
      X \arrow[r] & F
    \end{tikzcd}\end{equation*}  
\end{axiom} 

We define a type to be compact Haussdorf if it is the quotient of a Stone type by a closed equivalence relation. 
We denote $\CHaus$ for the the type of compact Hausdorff types. 
A motivating example for compact Haussdorf types is the unit interval, which can be given as a quotient of Cantor space. 
To show that such an interval is isomorphic to the standard Cauchy interval, 
we need LLPO and an axiom of dependent choice. 
\begin{axiom}[Dependent Choice]
%  Given a family of types $E_n$ and $R_n:E_n\rightarrow E_{n+1}\rightarrow {\mathcal U}$ such that
%  for all $n$ and $x:E_n$ there exists $y:E_n$ with $p:R~x~y$ then given $x_0:E_0$ there exists
%  $u:\Pi_{n:\N}E_n$ and $v:\Pi_{n:\N}R~(u~n)~(u~(n+1))$ and $u~0 = x_0$.
  Given a sequence of arbitrary types $(X_n)_{n:\mathbb N}$ and surjections $X_n \twoheadrightarrow X_{n-1}$, 
  all the limit projection maps $X \to X_n$ are surjective. 
\end{axiom}


It is important that these two last axioms are stated for {\em arbitrary} types and not types that are only
homotopy sets. One application should be a proof of $H^n(S,\ints) = 0$ for $n>0$, for $S$ Stone,
that we have checked for $n = 1$ (similar to the proof of $H^1(X,R) = 0$ for $X$ affine in the setting of \cite{draft}).
For this, we use the general definition of cohomology group in Homotopy Type Theory \cite{hott}, which refers to
types that are not necessarily homotopy sets.
We also expect to have for $X:\CHaus$ that $H^n(X,\ints)$ coincide with singular cohomology.

 Finally, we checked that all axioms suggested recently by R. Barton and J. Commelin \cite{bc24} follows from our axioms.



%
%We are working on the proof that these axioms can be verified in the interpretation of HoTT in the topos of light condensed sets. We have checked that the results of lectures 2 and 3 of \cite{Scholze} follow from these axioms. 
%In particular, we can prove that $\ints[\Noo]$ is projective in the category of Abelian groups. 
%
%We are working on the proof that these axioms can be verified in a model of HoTT, 
%similar to the model presented in \cite{draft}.
%We are also checking which results of \cite{Scholze} can be shown from these axioms.





\printbibliography

\end{document}
