\documentclass{beamer}

\usepackage[english]{babel} % Juiste afbreekregels en dergelijke!
\usepackage{parskip} % Alinea's beginnen links uitgelijnd en er staat een lege regel tussen alinea's.
\usepackage{amsmath, amssymb, textcomp,amsthm} % Wiskundige symbolen e.d.
\usepackage{color, colortbl} % Kleuren
\usepackage{enumerate} % Voor opsommingen
\usepackage{hyperref} % Voor het weergeven van hyper\textsl{}links die klikbaar zijn
\usepackage{mathrsfs}

\usepackage{makecell}

\usepackage{graphicx}
\usepackage{caption}

\usepackage{mathrsfs}

\usepackage{tikz-cd}
\usepackage{tikz}

\usetikzlibrary{shapes,patterns,positioning}

\usepackage{pgfplots}

\usepackage{ifthen}
\usepackage{calc}

\usetheme{Frankfurt}
% \setbeamercolor{structure}{fg = red}
% \useinnertheme{rectangles}
% \useoutertheme{smoothbars}

\definecolor{red}{rgb}{0.59,0.00,0.12}
\setbeamercolor{structure}{fg=red}


\setbeamertemplate{mini frames}[box]
\newcommand{\alertline}{%
 \usebeamercolor[fg]{normal text}%
 \only{\usebeamercolor[fg]{alerted text}}
 \usebeamercolor[fg]{normal text}
 }
\newcommand{\Type}{\mathcal U}
\newcommand{\Prop}{\mathrm{Prop}}
\newcommand{\Open}{\mathrm{Open}}
\newcommand{\Susp}{\mathrm{Susp}}
\newcommand{\propTrunc}[1]{\lVert #1 \rVert}
\newcommand{\Um}{\mathrm{Um}}
\newcommand{\Boole}{\mathsf{Boole}}
\newcommand{\Stone}{\mathsf{Stone}}
\newcommand{\Noo}{\N_{\infty}}
\newcommand{\Closed}{\mathsf{Closed}}
\newcommand{\isSt}{\mathsf{isStone}}
\newtheorem{remark}{Remark}
\begin{document}
\section{Introduction}

\title{Synthetic Stone Duality}
\date{TYPES 2024}
\author{
Felix Cherubini, Thierry Coquand, Freek Geerligs, Hugo Moeneclaey}
\maketitle
\begin{frame}{Introduction}
  \begin{itemize}
    \item This is work in progress. 
      \pause 
    \item We'll present an axiom system for working with Stone duality. 
      \pause 
    \item The axioms are expected to have a model in a constructive metatheory. 
      (WIP by Coquand, H\"ofer \& Moeneclaey)
      \pause
    \item 
      This model will validate non-constructive principles like LLPO. 
  \end{itemize}
\end{frame}
\begin{frame}{Table of contents}
\tableofcontents
\end{frame}

  \section{Stone spaces}% (as affine schemes)}
\begin{frame}{Countably presented Boolean algebras}
%  \begin{definition}
%    A ring is Boolean if $x^2 = x$ for all $x$. 
%  \end{definition}
%  \begin{remark}
%    Boolean algebras are in 1-1 correspondence with Boolean rings. 
%  \end{remark}
%  \pause
  \begin{definition}
    A Boolean algebra is countably presented if it is a \uncover<3->{quotient of a}
    \pause 
    freely generated Boolean algebra on countably many generators 
  \uncover<3->{by countably many relations.}
  \end{definition}
  \pause 
  \pause
  \begin{definition}
    $\Boole$ is the type of countably presented Boolean algebras. 
  \end{definition}
%%  \pause
%%  \begin{lemma}
%%    A Boolean algebra is countably presented iff 
%%    it is a sequential colimit of finite Boolean algebras. 
%%%
%%%    Countably presented Boolean algebras are colimits of sequences of 
%%%    finitely presented Boolean algebras. 
%%  \end{lemma}
  \pause 
  \begin{example}
    If we have no generators and no relations, we get $2:\Boole$.
  \end{example}
\end{frame}

\begin{frame}{Stone spaces}
  \begin{example}
    If we have no generators and no relations, we get $2:\Boole$.
  \end{example}
  \pause
\begin{definition}
  For $B:\Boole$, we define the spectrum of $B$ as 
  the set of Boolean morphisms from $B$ to $2$. 
  It is denoted $Sp(B)$.
\end{definition}
\pause
\begin{definition}
  Define a predicate on types as follows:
  \pause
  \begin{equation*}
    \isSt(X) := \sum\limits_{B : Boole} X = Sp(B)
  \end{equation*} 
  \pause 
  If $\isSt(X)$ is inhabited, the type $X$ is called Stone.
\end{definition}
\end{frame}

\begin{frame}{Examples}
  \begin{exampleblock}{Cantor space}
    \begin{itemize}
      \item 
        If we have generators $(p_n)_{n:\mathbb N}$ 
        and no relations, we get $C : \Boole$.
    \pause
  \item 
    Boolean maps $C \to 2$ correspond to \pause binary sequences $2^\mathbb N$.
    \end{itemize}
  \end{exampleblock}
  \pause
  \begin{exampleblock}{$\mathbb N_\infty$}
    \begin{itemize}
      \item 
    If we have generators $(p_n)_{n:\mathbb N}$ \pause 
    and relations $p_n \wedge p_m = 0$ for 
    $n \neq m$, \pause   we get $B_\infty:\Boole$. 
    \item 
    \pause
    Boolean maps $B_\infty \to 2$ correspond to \pause 
    binary sequences \pause which are $1$ at most once.
    \pause
    We call $Sp(B_\infty) = \mathbb N_\infty$. 
    \end{itemize}
  \end{exampleblock}
%  \pause
%  \begin{block}{Remark}
%    \begin{itemize}
%      \item 
%    There is a map $\mathbb N \to \mathbb N_\infty$ sending $n$ to the sequence which is $1$ at $n$ and $0$ everywhere else. 
%    \pause
%      \item 
%    There is a term $\infty:\mathbb N_\infty$ corresponding to the sequence which is $0$ everywhere. 
%    \end{itemize}
%  \end{block}`
\end{frame}

\section{Axiom system}
\begin{frame}{Stone duality}
  \begin{block}{Axiom 1 : Stone duality}
    For $B:\Boole$, the evaluation map $B \to 2^{Sp(B)}$ is an isomorphism. 
  \end{block}
  \pause
  \begin{corollary}
    For $S:\Stone$, we have $S = Sp(2^S)$,
    \pause 
    $\isSt(X)$ is a proposition \pause and
    $Sp$ is an embedding $\Boole \hookrightarrow \Type$. 
  \end{corollary}
  \pause
  \begin{definition}
    $\Stone$ is the image of $Sp$. 
  \end{definition}
  \pause
  \begin{remark}
    Both $\Stone$ and $\Boole$ have a natural category structure and $Sp$
    is a dual equivalence between these categories. 
  \end{remark}
\end{frame}



\begin{frame}{Surjections are formal surjections}
  \begin{remark}
    Both $\Stone$ and $\Boole$ have a natural category structure and $Sp$
    is a dual equivalence between these categories. 
  \end{remark}
  \pause
  \begin{block}{Axiom 2: Surjections are formal surjections}
    A map of Stone spaces is surjective iff the corresponding map of Boolean algebras is injective. 
  \end{block}
  \pause
  \begin{block}{Equivalent to Axiom 2}
    For $S:\Stone$, we have $\neg \neg S \to || S||$. 
  \end{block}
\end{frame}






\begin{frame}[fragile]{Local choice}
  %Motivation:Analogy to SAG
  %Maybe should just mention them very quickly in talk.
  \begin{block}{Axiom 3: Local choice}
  Given $S$ Stone, $E$ arbitrary, %a map $S \to F$ and 
  $E\twoheadrightarrow S$ surjective, 
  \uncover<2->{
  there is some $T$ Stone,
a surjection $T \twoheadrightarrow S$ and a map $T\to E$ such that the following diagram commutes}
   \begin{equation*}\begin{tikzcd}
    & E \arrow[d,""',two heads]\\
      \only<2->{
      T \arrow [r, two heads,dashed] \arrow [ur,dashed]
      } 
       & S
    \end{tikzcd}\end{equation*}  
  \end{block}
\end{frame}
\begin{frame}{Dependent choice}
%\begin{block}{Axiom 4: Dependent choice}
%  Given a family of types $(E_n)_{n:\mathbb N}$ and 
%  a relation 
%  $R_n:E_n\rightarrow E_{n+1}\rightarrow {\mathcal U}$ such that
%  for all $n$ and $x:E_n$ there exists $y:E_{n+1}$ with $p:R_n~x~y$ 
%  \pause
%  then given $x_0:E_0$ there exists
%  $u:\Pi_{n:\mathbb N}E_n$ and 
%  $v:\Pi_{n:\mathbb N}R_n~(u~n)~(u~(n+1))$ and $u~0 = x_0$.
%\end{block}
\begin{block}{Axiom 4: Dependent choice}
  Given a sequence of types $(X_n)_{n:\mathbb N}$ with surjections 
  $X_n \to X_{n-1}$, all the limit projection maps $X \to X_n$ are surjective. 
\end{block}
\end{frame}
\section{Omniscience principles}
\begin{frame}{Omniscience principles} 
  \begin{block}{The negation of WLPO}
    It is \textbf{not} the case that for any $\alpha:2^\mathbb N$ we can decide 
    \vspace{-0.2cm}
    $$
    (\forall_{n:\mathbb N} \alpha(n)  = 0 )
    \vee
    \neg (\forall_{n:\mathbb N} \alpha(n)  = 0 )
    $$
  \end{block}
  \pause
  \begin{block}{Markov's principle (observed by David W\"arn) }
    If $\alpha: 2^\mathbb N$ and $\neg 
    \forall_{n:\mathbb N} \alpha(n)  = 0 $, 
    there exists an $n:\mathbb N$ such that $\alpha(n) = 1$. 
  \end{block}
  \pause
  \begin{block}{LLPO (uses surjections are formal surjections)}
    For $\alpha:\mathbb N_\infty$, we have 
    %that the sequence corresponding to
%    $\alpha(2k+1) = 0 $ for all $k:\mathbb N$ or 
%    $\alpha(2k) = 0$ for all $k:\mathbb N$.
%    $0$ on all odd numbers or on all even numbers. 
    \vspace{-0.2cm}
    $$ 
    \forall_{k:\mathbb N} \alpha(2k+1) = 0 \vee
    \forall_{k:\mathbb N} \alpha(2k) = 0
    $$
  \end{block}
\end{frame}


\section{Topological results} % (as schemes)
%\begin{frame}{Topology}
%  \begin{definition}
%%    A proposition $P$ is closed iff there merely exists some $\alpha:\mathbb N _\infty$ with
%%    $$P \leftrightarrow \alpha = \infty$$ 
%%  \end{definition}
%%  \begin{definition}
%%    A proposition $P$ is open iff there merely exists some $\alpha:\mathbb N _\infty$ with
%%    $$P \leftrightarrow \alpha \neq \infty$$ 
%%  \end{definition}
%%  \begin{remark}
%    Let $P$ be a proposition. If there merely exists some 
%    $\alpha:\only<-2,4->{2^\mathbb N} \only<3>{\mathbb N_\infty}$ with 
%    \begin{itemize}
%      \item $P\leftrightarrow \forall n\alpha(n) = 0$, then $P$ is closed. 
%        \pause
%      \item $P\leftrightarrow \exists n\alpha(n) = 1$, then $P$ is open. 
%        \pause
%        \pause
%    \end{itemize}
%\end{definition}
%  \pause
%   \begin{definition}
%     A subset $A\subseteq S$ is open/closed iff $A (x)$ is open/closed for all $x:S$. 
%   \end{definition}
%   \pause
%%  \begin{remark}
%%    Open/closed propopositions are \alert<4->{countable} disjunctions/conjunctions of decidable propositions. 
%%%    Closed propositions are countable conjunctions of decidable propositions, 
%%%    and open propositions countable disjunctions
%%  \end{remark}
%%  \pause 
%%  \pause
%  \begin{lemma}[using local choice]
%    A subset of a Stone space is closed iff 
%    it is a \alert<7->{countable} intersection of decidable subsets.
%  \end{lemma}
%\end{frame}
%
%\begin{frame}{\only<-5>{Closed}\only<6->{Open} propositions} \begin{definition}
%  A proposition is \only<-5>{closed}\only<6->{open} if it is a countable 
%  \only<-5>{conjunction}\only<6->{\textbf{disjunction}} of decidable propositions. 
%\end{definition}
%\pause 
%\begin{block}{
%  \only<-5>{Closed} \only<6->{Open}
%  propositions are expressable internally}
%  A proposition $P$ is \only<-5>{closed}\only<6->{open}
%  iff \pause there exists some $\alpha:2^\mathbb N$ with 
%  $P\leftrightarrow \only<-5>{\forall_{n:\mathbb N}}\only<6->{{\exists}_{n:\mathbb N}}\alpha(n) = 0$.
%\end{block}
%\pause
%%Here you can also talk about Markov and open vs closed. 
%\begin{definition}
%  A subset $A\subseteq S$ is 
%  \only<-5>{closed}\only<6->{open}
%  iff $x\in A$ is 
%  \only<-5>{closed}\only<6->{open}
%  for all $x:S$. 
%\end{definition}
%%Here you can talk about continuity of all fucntions.
%\pause
%%Here you can talk about why the next result matters. 
%\begin{lemma}[using local choice]
%  A subset of a Stone space is 
%  \only<-5>{closed} \only<6->{open}
%  iff 
%  it is a countable \only<-5>{intersection}\only<6->{\textbf{union}} of decidable subsets.
%\end{lemma}
%\end{frame}
\begin{frame}{Topology on propositions}
  %{\only<-3>{Closed}\only<4->{Open} propositions} 
  \begin{definition}
  A proposition is \only<-4>{closed}\only<5->{open} if it is a countable 
  \only<-4>{conjunction}\only<5->{{disjunction}} of decidable propositions. 
  \pause Denote \only<-4>{$Closed$}\only<5->{$Open$} for  
  \only<-4>{closed}\only<5->{open} 
  proposotions. 
\end{definition}
\pause 
\begin{block}{
  \only<-4>{Closed} \only<5->{Open}
  propositions are expressable internally}
  A proposition $P$ is \only<-4>{closed}\only<5->{open}
  iff \pause there exists some $\alpha:2^\mathbb N$ with 
  $P\leftrightarrow \only<-4>{\forall_{n:\mathbb N}}\only<5->{{\exists}_{n:\mathbb N}}\alpha(n) = 0$.
\end{block}
\pause
\pause
\begin{corollary}[Of Markov's principle]
  The negation of a closed proposition is open
  \pause (and vice versa). 
\end{corollary}
\pause 
\begin{corollary}
  Open/closed propositions are double negation stable. 
\end{corollary}
%Here you can also talk about Markov and open vs closed. 
\end{frame}
\begin{frame}[fragile]{Topology on types}
\begin{definition}
  A subset $A\subseteq S$ is 
  \only<-1>{closed}\only<2->{open}
  iff $x\in A$ is 
  \only<-1>{closed}\only<2->{open}
  for all $x:S$. 
\end{definition}
%Here you can talk about continuity of all fucntions.
\pause
\pause
\begin{corollary}
  All functions are continuous with respect to this topology.
\end{corollary}
\pause 
$$
\begin{tikzcd} 
  \only<6->{f^{-1}(A) \arrow[r] \arrow[d, hook] \arrow[rd,"\lrcorner"{pos=0.125}, phantom]}
  &   
  A \arrow[d,hook] \arrow[r] \arrow[rd,"\lrcorner"{pos=0.125}, phantom] & 1 \arrow[d]\\
  \only<5->{T \arrow[r, "f"] } & S \arrow[r] & Open
\end{tikzcd}
$$
\pause 
\pause 
\pause 
%Here you can talk about why the next result matters. 
\begin{lemma}[using local choice]
  A subset of a Stone space is 
  \only<-7>{closed}\only<8->{open}
  iff 
  it is a countable \only<-7>{intersection}\only<8->{{union}} of decidable subsets.
\end{lemma}
\end{frame}


%\begin{frame}{Open and closed propositions}
%\begin{definition}
%  A proposition is closed if it is a countable 
%  {conjunction} of decidable propositions. 
%\end{definition}
%\begin{block}{Closed can be expressed internally}
%  A proposition $P$ is closed %\only<-5>{closed}\only<6->{open}
%  iff \pause there exists some $\alpha:2^\mathbb N$ with 
%  $P\leftrightarrow \forall_{n:\mathbb N}\alpha(n) = 0$.
%\end{block}
%\begin{definition}
%  A proposition is open if it is a countable 
%  {disjunction} of decidable propositions. 
%\end{definition}
%\begin{block}{Open can be expressed internally}
%  A proposition $P$ is open %\only<-5>{closed}\only<6->{open}
%  iff \pause there exists some $\alpha:2^\mathbb N$ with 
%  $P\leftrightarrow \forall_{n:\mathbb N}\alpha(n) = 0$.
%\end{block}
%
%
%
%\end{frame}
%
%\begin{frame}{\only<-5>{Closed}\only<6->{Open} propositions}
%\begin{definition}
%  A proposition is \only<-5>{closed}\only<6->{open} if it is a countable 
%  \only<-5>{conjunction}\only<6->{\textbf{disjunction}} of decidable propositions. 
%\end{definition}
%\pause 
%\begin{block}{
%  \only<-5>{Closed} \only<6->{Open}
%  propositions are expressable internally}
%  A proposition $P$ is \only<-5>{closed}\only<6->{open}
%  iff \pause there exists some $\alpha:2^\mathbb N$ with 
%  $P\leftrightarrow \only<-5>{\forall_{n:\mathbb N}}\only<6->{{\exists}_{n:\mathbb N}}\alpha(n) = 0$.
%\end{block}
%\pause
%\begin{definition}
%  A subset $A\subseteq S$ is 
%  \only<-5>{closed}\only<6->{open}
%  iff $x\in A$ is 
%  \only<-5>{closed}\only<6->{open}
%  for all $x:S$. 
%\end{definition}
%\pause
%\begin{lemma}[using local choice]
%  A subset of a Stone space is 
%  \only<-5>{closed} \only<6->{open}
%  iff 
%  it is a countable \only<-5>{intersection}\only<6->{\textbf{union}} of decidable subsets.
%\end{lemma}
%\end{frame}
%
\section{Related work}
\begin{frame}{Related work}
  \begin{itemize}
    \item There's literature 
      Cantor space, $\mathbb N_\infty$, 
      and synthetic stuff. %Escardo \& Xu worked on a model of Cantor space. 
    \pause 
    \item 
    Clausen \& Scholze have a lecture series on light condensed sets, which include compact Hausdorff spaces. \\
    \pause
    We expect to prove all \textit{internal} 
    results from this series from the axioms we presented. 
  \pause 
    \item 
      Barton \& Commelin have derived a form of ``directed univalence" from some axioms. \pause
      These axioms hold for light condensed sets. \pause
      We expect to validate their axioms in our axiom system. 
      \pause
    \item 
    Coquand, H\"ofer \& Moeneclaey are working on a 
    constructive model of the axioms presented above, 
    based on the model for synthetic algebraic geometry. 
    \pause 
  \item \url{https://github.com/felixwellen/synthetic-zariski/}
\end{itemize}
\end{frame}

%\section{Future work}
%\begin{frame}{Future work}
%  \begin{itemize}
%    \item Clausen \& Scholze have a lecture series on light condensed. 
%      \pause
%    We expect to prove results from this series from the axioms we presented. 
%      \pause
%    \item In particular, we can show that 
%      \pause
%    \item 
%      Barton \& Commelin are working on a condensed type theory. \pause
%      We expect to validate their axioms with our axioms. 
%  \end{itemize}
%\end{frame}


%\begin{frame}
%
%%\section{Definition of CHaus, all functions continuous, classifier for closed subsets of Stone spaces}
%%\section{Omniscience principles}
%%\section{Future goals}
%\begin{itemize}
%  \item Functional analyuysis
%  \item Commelin Barton CHaus and ODisc
%  \item Cohomology groups
%\end{itemize}
%\end{frame}
%
%\begin{frame}
%Abstract: We present an axiomatization for the (higher) topos of light condensed sets in dependent type theory with univalence, similar to the axiomatization of the (higher) Zariski topos in synthetic algebraic geometry. Stone spaces correspond to affine schemes, and compact Hausdorff spaces correspond to schemes. We can show that all maps between compact Hausdorff spaces are continuous, hence the negation of the weak principle of omniscience holds. We can prove the limited principle of omniscience and Markov's principle. We expect to be able to prove from these axioms that the higher cohomology over Z of any Stone spaces is trivial and that the cohomology over Z of any compact Hausdorff spaces is the same as the singular cohomology.
%
%\end{frame}
\appendix 

\begin{frame}{Markov's principle}
  \begin{corollary}
    For $S:\Stone$, we have $S = Sp(2^S)$,
    $\isSt$ is a proposition.
  \end{corollary}
  \pause
  \begin{block}{Markov's principle (observed by David W\"arn)}
    For $\alpha:2^\mathbb N$, we have 
    \vspace{-0.2cm}
    $$\neg (\forall_{n:\mathbb N} \alpha(n) = 0) \to 
    \exists_{n:\mathbb N} \alpha(n) = 1
    $$
  \end{block}
  \begin{proof}
    Suppose $\neg(\forall_{n:\mathbb N}\alpha(n) = 0)$. 
    \pause 
    Consider $B = 2/\{\alpha(n)=0|n:\mathbb N\}$. \\
    \pause 
    We can show that $Sp(B) = \emptyset$. \pause
    By Stone duality, $1=0$ in $B$. 
    \\\pause 
    There exists some finite $N\subseteq \mathbb N$
    with $1\in \{\alpha(n)|n:\mathbb N\}$. 
    \pause \\
    Hence there is some $n:\mathbb N$ with $\alpha(n) = 1$. 
  \end{proof}
\end{frame}


\begin{frame}{LLPO}
%  \begin{block}{Axiom 2: Surjections are formal surjections}
%    A $Sp(B) \to Sp(C)$ is surjective iff 
%    $C\to B$ is injective. 
%%    the corresponding map of Boolean algebras is injective. 
%  \end{block}
  \begin{exampleblock}{$\mathbb N_\infty$}
    If we have generators $(p_n)_{n:\mathbb N}$ 
    and relations $p_n \wedge p_m = 0$ for 
    $n \neq m$, we get $B_\infty:\Boole$.
    We call $Sp(B_\infty) = \mathbb N_\infty$. 
  \end{exampleblock}
  \begin{block}{LLPO}
    For $\alpha:\mathbb N_\infty$, 
    we have 
%    %that the sequence corresponding to
    $\forall_{k:\mathbb N} \alpha(2k+1) = 0 $ or 
    $\forall_{k:\mathbb N} \alpha(2k) = 0$.
    %for all $k:\mathbb N$.
%%    $0$ on all odd numbers or on all even numbers. 
%    \vspace{-0.2cm}
%    $$ 
%    \forall (k:\mathbb N) \alpha(2k+1) = 0 \vee
%    \forall (k:\mathbb N) \alpha(2k) = 0
%    $$
  \end{block}
  \pause 
  \begin{proof}
    Define a map $f:B_\infty \to B_\infty \times B_\infty $ by 
    \pause 
    \vspace{-0.2cm}
    $$
      f(p_n) = \begin{cases}
        (p_k,0) \text{ if } n = 2k \\
        (0,p_k) \text{ if } n = 2k+1
      \end{cases}
    $$
    \vspace{-0.5 cm}
    \\
    \pause
    We can show that $f$ is injective. 
    \pause \\
    Thus $f$ corresponds to a surjection 
    $s:\mathbb N_\infty +\mathbb N_\infty \to \mathbb N_\infty$. \\
    \pause 
%    For any $k:\mathbb N$, we have $s(inl(\beta))(2k+1) = s(inr(\gamma))(2k) = 0$. 
    If $\alpha=s(inl(\beta))$, then $\alpha(2k+1) = 0$. \\
    \pause
    If $\alpha=s(inr(\beta))$, then $\alpha(2k) = 0$. 
%    Now $s(inl(\beta))(2k+1) = 0$ for all $k:\mathbb N$ and 
%    Now $s(inr(\beta))(2k) = 0$ for all $k:\mathbb N$. 
%    As $s$ is surjective, LLPO holds for all $\alpha:\mathbb N_\infty$. 
  \end{proof}
\end{frame}

\end{document}
