\section{Topological spaces}
\begin{frame}{Cohomology}%Light condensed sets.}
%  We expect to be able to show that $[0,1]$ is acyclic. 
  \pause 
    \begin{lemma}
      $H^1(S,\mathbb Z) = 0$ for $S$ Stone.
    \end{lemma}
    \pause 
    We hope to extend this result to 
      $H^n(S,\mathbb Z) = 0$ for $S$ Stone and $n>1$.
      \pause
      \\
      We also expect to show that $H^n(X,\mathbb Z)$ corresponds to the singular cohomology 
      for $X$ a finite CW-complex.\\
\end{frame}
%\section{Unit interval as Compact Hausdorff space}
\begin{frame}{Compact Hausdorff spaces}
  \begin{definition}
    A type is compact Hausdorff iff it is the quotient of a 
    Stone space by a closed equivalence relation. 
  \end{definition}
  \pause
  \begin{lemma}
    Let $X = S / \sim $ be compact Hausdorff and $A\subseteq X$. TFAE
    \begin{itemize}
      \item $A$ is closed. 
      \item the preimage of $A$ in $X$ is closed. 
    \end{itemize}
\end{lemma}
\pause
  \begin{corollary}
    Any map between compact Hausdorff types is 
    continuous in the sense that the 
    %inverse of an open is open. 
%    such that the 
    inverse image of a closed subset is closed. 
  \end{corollary}
\end{frame}


\begin{frame}{Example: the interval}
  \begin{lemma}
    The unit interval is a compact Hausdorff type. 
  \end{lemma}
  \begin{block}{The closed equivalence relation}
  For $\alpha,\beta:2^\mathbb N$, 
  \pause denote 
      $\alpha \sim_n \beta$ iff 
      \pause 
      there is some $m:\mathbb N$ and some 
      initial segment $x:2^m$ such that 
      \pause
      \vspace{-0.2cm}
      $$
      (\alpha|_n,\beta|_n)= 
      ( 
        (x \cdot 0 \cdot \overline 1)|_n , 
        (x \cdot 1 \cdot \overline 0)|_n
        ) 
      $$
      \vspace{-0.4cm}
      or the other way around. 
      \pause 
      Define 
      $$\alpha\sim\beta := \forall_{n:\mathbb N} (\alpha \sim_n \beta)$$
    \end{block}
\end{frame}

\begin{frame}{Anticlassical results on $[0,1]$}
%  \begin{lemma}
%    The unit interval is a compact Hausdorff type. 
%  \end{lemma}
  \begin{lemma}[using LLPO and dependent choice]
%    The unit interval as above is in bijection with the unit interval of Cauchy reals. 
    The unit interval is a compact Hausdorff type. 
  \end{lemma}
  \pause 
    \begin{corollary}
      Any map $[0,1] \to [0,1]$ is continuous. 
    \end{corollary}
    \pause 
    \begin{corollary}
      Any map $[0,1] \to 2$ is constant.
    \end{corollary}
    \pause 
    \begin{lemma}
      $[0,1]$ is acyclic. 
    \end{lemma}
%%  \pause
%%  \begin{proof}
%%    Using LLPO and dependent choice, 
%%    \pause 
%%    there is a standard construction 
%%    to show that the function $2^\mathbb N \to [0,1]_\mathbb R$ given by 
%%    $$\alpha\mapsto \sum\limits_{i:\mathbb N} \frac{\alpha(i)}{2^{i+1}}$$
%%    is surjective.
%%    \pause
%%    This function will respect the equivalence relation on $2^\mathbb N$ given by \pause
%%    $\alpha \sim \beta$ if for each $n:\mathbb N$, 
%%    there is some finite binary sequence $x$ such that 
%%    $\alpha|_n, \beta|_n$ are both of the form 
%%    $(x \cdot 0 \cdot \overline 1)|_n$ or $(x \cdot 1 \cdot \overline 0)|_n$.
%%  \end{proof}
%  %Time permitting:
    \pause
  \begin{proof}%[Proof sketch]
    Consider the relation on $2^\mathbb N$ with
    $\alpha \sim \beta$ if for each $n:\mathbb N$, 
    there is some finite binary sequence $x$ such that 
    $\alpha|_n, \beta|_n$ are both of the form 
    $(x \cdot 0 \cdot \overline 1)|_n$ or $(x \cdot 1 \cdot \overline 0)|_n$.
  \end{proof}
\end{frame}





\section{Proofs of omniscience principles}
\begin{frame}{The negation of WLPO}
  \begin{block}{Axiom 1 : Stone duality}
    For $B:\Boole$, the evaluation map $B \to 2^{Sp(B)}$ is an isomorphism. 
  \end{block}
%  \begin{exampleblock}{Cantor space}
%    Boolean maps $C \to 2$ correspond with binary sequences $2^\mathbb N$.
%  \end{exampleblock}
  \pause
  \begin{block}{The negation of WLPO}
    It is \textbf{not} the case that for any $\alpha:2^\mathbb N$ we can decide 
    \vspace{-0.2cm}
    $$\forall_{n:\mathbb N} \alpha (n) = 0 \vee \neg \forall_{n:\mathbb N} \alpha (n) = 0 $$
  \end{block}
  \pause
  \begin{proof}
    If so, we have an indicator map $b:2^\mathbb N \to 2$. 
%    \pause \\
%    By Stone duality, we have some $b\in C$ such that 
    \vspace{-0.2cm}
    $$(\forall_{n:\mathbb N} \alpha(n) = 0 )\leftrightarrow b(\alpha) = 0$$
    \vspace{-0.8cm}
    \\
    \pause 
    By Stone duality, $b$ corresponds to is a finite Boolean expression, \pause \\
    so by only looking at $\alpha(n)$ for finitely many $n:\mathbb N$, \pause \\
    we can decide whether $\alpha(n) = 0$ for all $n:\mathbb N$.
  \end{proof}
\end{frame}
\begin{frame}{Markov's principle}
  \begin{corollary}
    For $S:\Stone$, we have $S = Sp(2^S)$,
    $\isSt$ is a proposition.
  \end{corollary}
  \pause
  \begin{block}{Markov's principle (observed by David W\"arn)}
    For $\alpha:2^\mathbb N$, we have 
    \vspace{-0.2cm}
    $$\neg (\forall_{n:\mathbb N} \alpha(n) = 0) \to 
    \exists_{n:\mathbb N} \alpha(n) = 1
    $$
  \end{block}
  \begin{proof}
    Suppose $\neg(\forall_{n:\mathbb N}\alpha(n) = 0)$. 
    \pause 
    Consider $B = 2/\{\alpha(n)=0|n:\mathbb N\}$. \\
    \pause 
    We can show that $Sp(B) = \emptyset$. \pause
    By Stone duality, $1=0$ in $B$. 
    \\\pause 
    There exists some finite $N\subseteq \mathbb N$
    with $1\in \{\alpha(n)|n:\mathbb N\}$. 
    \pause \\
    Hence there is some $n:\mathbb N$ with $\alpha(n) = 1$. 
  \end{proof}
\end{frame}


\begin{frame}{LLPO}
%  \begin{block}{Axiom 2: Surjections are formal surjections}
%    A $Sp(B) \to Sp(C)$ is surjective iff 
%    $C\to B$ is injective. 
%%    the corresponding map of Boolean algebras is injective. 
%  \end{block}
  \begin{exampleblock}{$\mathbb N_\infty$}
    If we have generators $(p_n)_{n:\mathbb N}$ 
    and relations $p_n \wedge p_m = 0$ for 
    $n \neq m$, we get $B_\infty:\Boole$.
    We call $Sp(B_\infty) = \mathbb N_\infty$. 
  \end{exampleblock}
  \begin{block}{LLPO}
    For $\alpha:\mathbb N_\infty$, 
    we have 
%    %that the sequence corresponding to
    $\forall_{k:\mathbb N} \alpha(2k+1) = 0 $ or 
    $\forall_{k:\mathbb N} \alpha(2k) = 0$.
    %for all $k:\mathbb N$.
%%    $0$ on all odd numbers or on all even numbers. 
%    \vspace{-0.2cm}
%    $$ 
%    \forall (k:\mathbb N) \alpha(2k+1) = 0 \vee
%    \forall (k:\mathbb N) \alpha(2k) = 0
%    $$
  \end{block}
  \pause 
  \begin{proof}
    Define a map $f:B_\infty \to B_\infty \times B_\infty $ by 
    \pause 
    \vspace{-0.2cm}
    $$
      f(p_n) = \begin{cases}
        (p_k,0) \text{ if } n = 2k \\
        (0,p_k) \text{ if } n = 2k+1
      \end{cases}
    $$
    \vspace{-0.5 cm}
    \\
    \pause
    We can show that $f$ is injective. 
    \pause \\
    Thus $f$ corresponds to a surjection 
    $s:\mathbb N_\infty +\mathbb N_\infty \to \mathbb N_\infty$. \\
    \pause 
%    For any $k:\mathbb N$, we have $s(inl(\beta))(2k+1) = s(inr(\gamma))(2k) = 0$. 
    If $\alpha=s(inl(\beta))$, then $\alpha(2k+1) = 0$. \\
    \pause
    If $\alpha=s(inr(\beta))$, then $\alpha(2k) = 0$. 
%    Now $s(inl(\beta))(2k+1) = 0$ for all $k:\mathbb N$ and 
%    Now $s(inr(\beta))(2k) = 0$ for all $k:\mathbb N$. 
%    As $s$ is surjective, LLPO holds for all $\alpha:\mathbb N_\infty$. 
  \end{proof}
\end{frame}

