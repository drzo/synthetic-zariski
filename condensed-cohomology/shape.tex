Here we start studying the shape modality $\shape$, which is defined as the localisation at $\mathbb{I}$.

\subsection{Deloopings of overtly discrete abelian groups are local}

\begin{lemma}\label{delooping-overtly-discrete-local}
For all $A$ overtly discrete abelian group and any $n:\N$, we have that:
\[B_nA\]
is $\shape$-local.
\end{lemma}

\begin{proof}
We proceed by induction on $n$. 
\begin{itemize}
\item For $n=0$ we have $A = A^\mathbb{I}$, which holds as maps in $\mathbb{I}\to A$ factor through a finite type and maps in $\mathbb{I}\to 2$ are constant. 
\item For $n+1$, by induction hypothesis we know that $B_{n+1}A$ is $\shape$-separated, and we merely have a lift by \cref{vanishing-cohomology-interval}.
\end{itemize}
\end{proof}

%\begin{corollary}\label{cohomology-shape}
%Let $f:X\to Y$ be a shape equivalence and $A$ be an overtly discrete abelian group. Then for all $k$ we have:
%\[f^* : H^k(Y,A) \simeq H^k(X,A)\]
%\end{corollary}

\begin{corollary}\label{cohomology-shape}
Let $X$ be a type and $A$ be an overtly discrete abelian group. Then for all $k$ we have:
\[H^k(X,A) \simeq H^k(\shape X,A)\]
\end{corollary}



\subsection{The shape of the circle is the circle}

\begin{proposition}
We have that:
\[\shape (\R/\Z) = B\Z\]
\end{proposition}

\begin{proof}
The fibers of the map:
\[\R\to \R/\Z\]
are $\Z$-torsors, as is the case for any group quotient. This means that we have a fiber sequence:
\[\R \to \R/\Z \to B\Z\]
We check that the second map is $\shape$-localisation. We have that $B\Z$ is $\shape$-local by \cref{delooping-overtly-discrete-local}. Since $B\Z$ is connected we just need to prove that $\R$ is $\shape$-contractible to conclude. But $0:\R$ and for any $x:\R$ there is $f:\mathbb{I}\to\R$ such that $f(0)=0$ and $f(1)=x$ so we can conclude.
\end{proof}


\subsection{Finite homotopical cell complex are local}

\begin{lemma}\label{finite-homotopy-groups-countably-presented}
Let $X$ be a finite homotopical cell complex, then for any $x:X$ and any $n$ we have that $\pi_n(X,x)$ is a countably presented abelian group.
\end{lemma}

\begin{proof}
TODO maybe find a reference?
\end{proof}

\begin{proposition}\label{homotopical-complex-local}
Let $X$ be a finite homotopical cell complex, then $X$ is $\shape$-local.
\end{proposition}

\begin{proof}
We decompose $X$ as its Postnikov tower:
\[\cdots\to \propTrunc{X}_{n+1}\to \propTrunc{X}_n\to\cdots \to \propTrunc{X}_0\]
First we show by induction on $n$ then $\propTrunc{X}_n$ is $\shape$-local:
\begin{itemize}
\item We have that $\propTrunc{X}_0$ is a finite set so it is $\shape$-local.
\item Assuming $\propTrunc{X}_n$ is $\shape$-local, it is enough to prove that the fibers of the map:
\[\propTrunc{X}_{n+1}\to \propTrunc{X}_n\]
are $\shape$-local. But they merely are of the form $B_n\pi_n(X,x)$ for some $x:X$, but $\pi_n(X,x)$ is overtly discrete by \cref{finite-homotopy-groups-countably-presented} so that $B_n\pi_n(X,x)$ is $\shape$-local by \cref{delooping-overtly-discrete-local}.
\end{itemize}
Therefore the limit of the Postnikov tower is $\shape$-local as a limit of $\shape$-local group, and we can conclude as a finite homotopical CW complex $X$ is the limit of its Postnikov tower (why TODO, maybe optimistic?).
\end{proof}

\begin{remark}
By Anel / Barton "Choice axioms and Postnikov completeness" we know that Postnikov completion and hypercompletion agree in our setting because we have countable choice. Do we have hypercompleteness?
\end{remark}



\subsection{Cellular cohomology for finite topological cell complex}

\begin{definition}
An $n$-dimensional topological cell complex is defined inductively as a type $X$ such that:
\begin{itemize} 
\item If $n=0$ then $X$ is a finite type.
\item For $n+1$, we ask that there merely exists $X_n$ an $n$-dimensional topological cell complex and a pushout square:
\begin{center}
\begin{tikzcd}
X_n \ar[r] & X\\
\mathbb{S}^n\times \mathrm{Fin}(k)\ar[r]\ar[u] & \mathbb{D}^n\times \mathrm{Fin}(k)\ar[u]\\
\end{tikzcd}
\end{center}
\end{itemize}
A finite topological cell complex is a type that is an $n$-dimensional topological cell complex for some $n$.
\end{definition}

By contrast we call the usual HoTT cell complexes homotopical.

\begin{lemma}\label{shape-spheres-disks}
For all $n$ we have that:
\[\shape \mathbb{D}^n = 1\]
\[\shape \mathbb{S}^n = S^n\]
\end{lemma}

\begin{proof}
For any $x:\mathbb{D}^n$ we have a map $f:\mathbb{I}\to\mathbb{D}^n$ such that $f(0)=0$ and $f(1)=x$ se we have that $\shape\mathbb{D}^n = 1$.

For $S^n$ we proceed inductively:
\[\mathbb{S}^{-1} = S^{-1} = 0\]
which is $\shape$-local. 

Otherwise assume $\shape \mathbb{S}^n = S^n$. We have a pushout diagram:
\begin{center}
\begin{tikzcd}
\mathbb{D}^n\ar[r] & \mathbb{S}^{n+1} \\
\mathbb{S}^{n} \ar[u]\ar[r] &\mathbb{D}^n\ar[u]\\
\end{tikzcd}
\end{center}
which is $\shape$-equivalent to the pushout square:
\begin{center}
\begin{tikzcd}
1\ar[r] & S^{n+1} \\
S^n \ar[u]\ar[r] &1\ar[u]\\
\end{tikzcd}
\end{center}
so that we have:
\[\shape \mathbb{S}^{n+1} = \shape S^{n+1}\]
but $S^{n+1}$ is $\shape$-local by \cref{homotopical-complex-local}.
\end{proof}

\begin{lemma}\label{adding-one-cell-shape}
Let $X$ be a type such that $\shape X$ is a finite homotopical $n$-dimensional cell complex. Assume given a pushout square:
\begin{center}
\begin{tikzcd}
X\ar[r] & Y \\
\mathbb{S}^n\times \mathrm{Fin}(k)\ar[u]\ar[r] & \mathbb{D}^n\times \mathrm{Fin}(k)\ar[u]\\
\end{tikzcd}
\end{center}
Then we have a pushout square:
\begin{center}
\begin{tikzcd}
\shape X \ar[r] & \shape Y \\
S^n\times \mathrm{Fin}(k)\ar[u]\ar[r] & \mathrm{Fin}(k)\ar[u]\\
\end{tikzcd}
\end{center}
\end{lemma}

\begin{proof}
By general reasoning on modalities and \cref{shape-spheres-disks} we have that:
\[\shape Y = \shape\left(\shape X \coprod_{S^n\times \mathrm{Fin}(k)} \mathrm{Fin}(k)\right)\]
but since:
\[\shape X \coprod_{S^n\times \mathrm{Fin}(k)} \mathrm{Fin}(k)\]
is a finite homotopical cell complex by hypothesis, is is $\shape$-local by \cref{homotopical-complex-local} and we can conclude.
\end{proof}

\begin{lemma}\label{celullar-topological-cohomology}
Let $X$ be a finite topological cell complex, then $\shape X$ is a finite homotopical cell complex. Moreover we can compute a presentation for $\shape X$ from a presentation for $X$ simply by localising.
\end{lemma}

\begin{proof}
We apply \cref{adding-one-cell-shape} repeatedly.
\end{proof}

\begin{remark}
Given a finite topological cell complex, defining the corresponding finite homotopical cell complex is not obvious, as we need to show the result is independent from the chosen presentation. Using $\shape$ allows to bypass this issue.
\end{remark}

\begin{corollary}
Assume given $X$ a finite topological cell complex and $A$ an overtly discrete abelian group. Then:
\[H^n(X,A)\]
can be computed using the cellular cohomology of the finite homotopical complex $\shape X$.
\end{corollary}

\begin{proof}
Just recall that by \cref{cohomology-shape} we have that $H^n(X,A) = H^n(\shape X,A)$ and conclude by \cref{celullar-topological-cohomology}.
\end{proof}


