The aim here is to copy the abstract covering theory in \cite{cherubini_rijke_2021}[Section 8],
using the $\A^1$-shape defined further down.

\begin{definition}
  Let $\shape_n\colonequiv\shape_{\A^1,n}$ be the nullification at $\A^1$ and $S^{n+1}$.
\end{definition}

\begin{definition}
  The \notion{universal cover} of $X$ is the type $\hat{X}$ obtained by pullback:
  \begin{center}
    \begin{tikzcd}
      \hat{X}\ar[r]\ar[d] & 1\ar[d] \\
      X\ar[r] & \shape_1 X
    \end{tikzcd}
  \end{center}
\end{definition}

We import the notion of étale maps for a modality from \cite{cherubini_rijke_2021}.

\begin{definition}
  Let $\bigcirc$ be a modality (in the sense of the HoTT-Book),
  then $f:X\to Y$ is \notion{$\bigcirc$-étale}, if the naturality square is a pullback:
  \begin{center}
    \begin{tikzcd}
      X\ar[r]\ar[d,"f"] & \bigcirc X\ar[d] \\
      Y\ar[r] & \bigcirc Y
    \end{tikzcd}
  \end{center}
  Moreover, a map $g:X\to Y$ is called \notion{$\bigcirc$-equivalence},
  if $\bigcirc g$ is an equivalence.
  The ($\bigcirc$-equivalences, $\bigcirc$-étale maps) is an orthogonal factorization system. 
\end{definition}

\begin{proposition}
  Let $X$ be a type, then $\hat{X}\to X$ is $\shape_{\A^1,1}$-étale and $\shape_{\A^1,1}\hat{X}=1$.
  In particular, we have the following lifting property:
  \begin{center}
    \begin{tikzcd}
      1\ar[r]\ar[d] & \hat{X}\ar[d] \\
      \A^1\ar[r]\ar[ru,dashed,"\exists!"] & X
    \end{tikzcd}
  \end{center}
\end{proposition}

\begin{proof}
  All statements are true for general modalities.
\end{proof}
