\subsection{Universal $\A^1$-coverings}

The aim here is to copy the abstract covering theory in \cite{cherubini_rijke_2021}[Section 8],
using the $\A^1$-shape defined further down.

\begin{definition}
  Let $\shape_n\colonequiv\shape_{\A^1,n}$ be the nullification at $\A^1$ and $S^{n+1}$.
\end{definition}

\begin{definition}
  The \notion{universal cover} of $X$ is the type $\hat{X}$ obtained by pullback:
  \begin{center}
    \begin{tikzcd}
      \hat{X}\ar[r]\ar[d] & 1\ar[d] \\
      X\ar[r] & \shape_1 X
    \end{tikzcd}
  \end{center}
\end{definition}

We import the notion of $\bigcirc$-étale maps for a modality $\bigcirc$ from \cite{cherubini_rijke_2021}.

\begin{definition}
  Let $\bigcirc$ be a modality (in the sense of the HoTT-Book),
  then $f:X\to Y$ is \notion{$\bigcirc$-étale}, if the naturality square is a pullback:
  \begin{center}
    \begin{tikzcd}
      X\ar[r]\ar[d,"f"] & \bigcirc X\ar[d] \\
      Y\ar[r] & \bigcirc Y
    \end{tikzcd}
  \end{center}
  Moreover, a map $g:X\to Y$ is called \notion{$\bigcirc$-equivalence},
  if $\bigcirc g$ is an equivalence.
  The ($\bigcirc$-equivalences, $\bigcirc$-étale maps) is an orthogonal factorization system. 
\end{definition}

\begin{proposition}
  Let $X$ be a type, then $\hat{X}\to X$ is $\shape_{\A^1,1}$-étale and $\shape_{\A^1,1}\hat{X}=1$.
  In particular, we have the following lifting property:
  \begin{center}
    \begin{tikzcd}
      1\ar[r]\ar[d] & \hat{X}\ar[d] \\
      \A^1\ar[r]\ar[ru,dashed,"\exists!"] & X
    \end{tikzcd}
  \end{center}
\end{proposition}

\begin{proof}
  All statements are true for general modalities.
\end{proof}

\subsection{Torsor and $\A^1$-coverings}

\begin{proposition}
Let $G$ be a group. The following are equivalent:
\begin{enumerate}[(i)]
\item The type $BG$ is $\A^1$-local.
\item For all $X$, any $G$-torsor over $X$ is $\A^1$-étale.
\end{enumerate}
\end{proposition}

\begin{proof}
(i) implies (ii). Let $f:X\to Y$ be $G$-torsor, and $i:A\to B$ an $\A^1$-equivalence. Assume given a square:
  \begin{center}
    \begin{tikzcd}
      A\ar[r,"s"]\ar[d,swap,"i"] & X\ar[d,"f"] \\
      B\ar[r]\ar[dotted,ur] & Y
    \end{tikzcd}
  \end{center}
  We want to prove there is a unique dotted lifting. 
  
  First we prove that there exists such a lifting. We get a $G$-torsor over $B$ by pulling back the one over $Y$. By the commutation of the diagram we know that the torsor is trivial when restricted to $A$, and since $BG$ is $\A^1$-local and $i$ is an $\A^1$-equivalence there is a unique dotted lift in:
    \begin{center}
    \begin{tikzcd}
      A\ar[r]\ar[d,swap,"i"] & BG\\
      B\ar[dotted,ur] & 
    \end{tikzcd}
  \end{center}
  So that the torsor is trivial on $B$ as well. From this we get $h$ making the triangle commute:
    \begin{center}
    \begin{tikzcd}
      & X\ar[d,"f"] \\
      B\ar[r]\ar[ur,"h"] & Y
    \end{tikzcd}
  \end{center}
  By the definition of torsors, we know that there is $g:A\to G$ such that for all $x:A$ we have:
  \[g(x)\cdot h(i(x)) = s(x)\]
  But since $G$ is $\A^1$-local (as $BG$ is) and $i$ is an $\A^1$-equivalence, we can lift this map to:
    \begin{center}
    \begin{tikzcd}
      A\ar[r,"g"]\ar[d,swap,"i"] & G\\
      B\ar[ur,swap,"g'"] & 
    \end{tikzcd}
  \end{center}
  And then $y\mapsto g'(y)\cdot h(y)$ makes the square commutes.
  
  Now assume given $h_1,h_2$ two such liftings, we consider $g:B\to G$ such that for all $y:B$ we have $g(y)\cdot h_1(y) = h_2(y)$. We know that $g$ has constant value $1$ on $A$, bit there is a unique dotted lift in:
      \begin{center}
    \begin{tikzcd}
      A\ar[r,"1"]\ar[d] & G\\
      B\ar[dotted,ur] & 
    \end{tikzcd}
  \end{center}
  so $g$ has constant value $1$ on all of $B$ and we have $h_1=h_2$.
  
  (ii) implies (i). Saying that $BG$ is $\A^1$-local means there is a unique dotted lift in any:
        \begin{center}
    \begin{tikzcd}
      \A^1\ar[r]\ar[d] & BG\\
      1\ar[dotted,ur] & 
    \end{tikzcd}
  \end{center}
Since any $G$-torsor is assumed $\A^1$-étale, by considering the trivial torsor $G\to 1$ and the fact that $\A^1$-étale maps are modal we conclude that $G$ is $\A^1$-local. Then identity types in $BG$ are $\A^1$-local and there is at most one dotted lift.
 
 To show there merely exists a dotted lift, we need to show that any $G$-torsor $P:\A^1\to BG$ merely is trivial. Since we want to prove a proposition, we can assume $t:P(0)$. Consider:
    \begin{center}
    \begin{tikzcd}
      1\ar[r,"{(}0{,}t{)}"]\ar[d,swap,"0"] & \sum_{x:\A^1}P(x)\ar[d]\\
      \A^1\ar[dotted,ur]\ar[r,swap,"id"] & \A^1
    \end{tikzcd}
  \end{center}
 The right map is a $G$-torsor so it is assumed $\A^1$-étale, and then we merely have a dotted lift because the left map is an $\A^1$-equivalence, and this means that the torsor merely is trivial.
 
\end{proof}
