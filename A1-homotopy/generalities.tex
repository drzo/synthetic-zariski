\subsection{About torsors and modalities}

We assume given a modality $\bigcirc$ such that for any $X$ the localisation:
\[X\to \bigcirc X\]
is surjective (e.g. $\A^1$-localisation).

\begin{proposition}\label{torsor-and-modalities}
Let $G$ be a group. The following are equivalent:
\begin{enumerate}[(i)]
\item The type $BG$ is $\bigcirc$-modal.
\item For all $X$, any $G$-torsor over $X$ is $\bigcirc$-étale.
\end{enumerate}
\end{proposition}

\begin{proof}
(i) implies (ii). Let $f:X\to Y$ be $G$-torsor, and $i:A\to B$ an $\bigcirc$-equivalence. Assume given a square:
  \begin{center}
    \begin{tikzcd}
      A\ar[r,"s"]\ar[d,swap,"i"] & X\ar[d,"f"] \\
      B\ar[r]\ar[dotted,ur] & Y
    \end{tikzcd}
  \end{center}
  We want to prove there is a unique dotted lifting. 
  
  First we prove that there exists such a lifting. We get a $G$-torsor over $B$ by pulling back the one over $Y$. By the commutation of the diagram we know that the torsor is trivial when restricted to $A$, and since $BG$ is $\bigcirc$-modal and $i$ is an $\bigcirc$-equivalence there is a unique dotted lift in:
    \begin{center}
    \begin{tikzcd}
      A\ar[r]\ar[d,swap,"i"] & BG\\
      B\ar[dotted,ur] & 
    \end{tikzcd}
  \end{center}
  So that the torsor is trivial on $B$ as well. From this we get $h$ making the triangle commute:
    \begin{center}
    \begin{tikzcd}
      & X\ar[d,"f"] \\
      B\ar[r]\ar[ur,"h"] & Y
    \end{tikzcd}
  \end{center}
  By the definition of torsors, we know that there is $g:A\to G$ such that for all $x:A$ we have:
  \[g(x)\cdot h(i(x)) = s(x)\]
  But since $G$ is $\bigcirc$-modal (as $BG$ is) and $i$ is an $\bigcirc$-equivalence, we can lift this map to:
    \begin{center}
    \begin{tikzcd}
      A\ar[r,"g"]\ar[d,swap,"i"] & G\\
      B\ar[ur,swap,"g'"] & 
    \end{tikzcd}
  \end{center}
  And then $y\mapsto g'(y)\cdot h(y)$ makes the square commutes.
  
  Now assume given $h_1,h_2$ two such liftings, we consider $g:B\to G$ such that for all $y:B$ we have $g(y)\cdot h_1(y) = h_2(y)$. We know that $g$ has constant value $1$ on $A$, but there is a unique dotted lift in:
      \begin{center}
    \begin{tikzcd}
      A\ar[r,"1"]\ar[d] & G\\
      B\ar[dotted,ur] & 
    \end{tikzcd}
  \end{center}
  so $g$ has constant value $1$ on all of $B$ and we have $h_1=h_2$.
  
  (ii) implies (i). Saying that $BG$ is $\bigcirc$-modal means there is a for any $\bigcirc$-connected type $X$ there is a unique dotted lift in any:
        \begin{center}
    \begin{tikzcd}
      X\ar[r]\ar[d] & BG\\
      1\ar[dotted,ur] & 
    \end{tikzcd}
  \end{center}
Since any $G$-torsor is assumed $\bigcirc$-étale, by considering the trivial torsor $G\to 1$ and the fact that $\bigcirc$-étale maps are modal we conclude that $G$ is $\bigcirc$-modal. Then identity types in $BG$ are $\bigcirc$-modal and there is at most one dotted lift.
 
 To show there merely exists a dotted lift, we need to show that any $G$-torsor $P:X\to BG$ merely is trivial. We know that $X$ is merely inhabited as the localisation:
 \[X\to \bigcirc X = 1\]
 is assumed to be surjective. Since we want to prove a proposition and we know that $X$ is merely inhabited, and we can assume $x:X$. Since torsors are merely inhabited we can assume $t:P(x)$. Consider:
    \begin{center}
    \begin{tikzcd}
      1\ar[r,"{(}x{,}t{)}"]\ar[d,swap,"0"] & \sum_{x:X}P(x)\ar[d]\\
      X\ar[dotted,ur]\ar[r,swap,"id"] & X
    \end{tikzcd}
  \end{center}
 The right map is a $G$-torsor so it is assumed $\bigcirc$-étale, and then we merely have a dotted lift because the left map is an $\bigcirc$-equivalence, and this means that the torsor merely is trivial.
 
\end{proof}

\subsection{About $\A^1$-localisation}

\begin{lemma}\label{A1-replacement-surjective}
Any $\shape_{\A^1}$-connected map is surjective. In particular, for any type $X$ the map:
\[\eta_X:X\to \shape_{\A^1}X\]
is surjective.
\end{lemma}

\begin{proof}
This holds because any $\shape_{\A^1}$-connected type is merely inhabited, as any proposition is $\A^1$-local so that for any $X$ we have a map:
\[\shape_{\A^1}X \to \propTrunc{X}\]
\end{proof}

\begin{lemma}
If $X$ is path-connected then so is $\shape_{\A^1}X$.
\end{lemma}

\begin{proof}
Immediate from \cref{A1-replacement-surjective}.
\end{proof}

\begin{lemma}\label{colimit-shape}
If colimits indexed by $I$ exists in HoTT (e.g. pushouts, sequential colimits, quotients of group actions), and we have a map of $I$-indexed diagrams:
\[f_i : X_i \to Y_i\]
such that for all $i:I$ the map $f_i$ is a $\shape_{\A^1}$-equivalence, then the induced map: 
\[\mathrm{colim}_{i:I} X_i \to \mathrm{colim}_{i:I} Y_i\]
is a $\shape_{\A^1}$-equivalence.
\end{lemma}

\begin{proof}
%Same as \cref{colimit-shape}.
For any $\A^1$-local type $Z$, we have that:
\[(\mathrm{colim}_{i:I} Y_i) \to Z \simeq \mathrm{lim}_{i:I} (Y_i \to Z)\]
\[\simeq \mathrm{lim}_{i:I} (X_i \to Z) \simeq (\mathrm{colim}_{i:I} X_i) \to Z\]
which implies what we want.
\end{proof}

We often use this lemma with the $\shape_{\A^1}$-equivalences:
\[\eta_X : X \to \shape_{\A^1} X\]
Next lemma says that $\A^1$-pullbacks can be computed as plain pullbacks for $\A^1$-étale maps.

\begin{lemma}
Assume a pulllback square:
  \begin{center}
    \begin{tikzcd}
      A\ar[r]\ar[d] & X\ar[d] \\
       B\ar[r] &  Y
    \end{tikzcd}
  \end{center}
  where the right map is $\A^1$-étale, then it is an $\A^1$-pullback, meaning the square:  
  \begin{center}
    \begin{tikzcd}
      \shape_{\A^1}A\ar[r]\ar[d] & \shape_{\A^1}X\ar[d] \\
       \shape_{\A^1}B\ar[r] &  \shape_{\A^1}Y
    \end{tikzcd}
  \end{center}
  is a pullback square.
\end{lemma}

\begin{proof}
This is \cite{cherubini_rijke_2021}[Corollary 5.2]. We give an alternative proof. Any such pullback square is of the form:
  \begin{center}
    \begin{tikzcd}
      \sum_{b:B}P(\shape_{\A^1}g(\eta_A(b)))\ar[r]\ar[d] & \sum_{y:Y}P(\eta_Y(y))\ar[d] \\
       B\ar[r,swap,"g"] &  Y
    \end{tikzcd}
  \end{center}
for some $P: \shape_{\A^1}Y \to {\mathcal U}_{\A^1}$, by \cite{cherubini_rijke_2021}[Corollary 5.5]. Applying $\A^1$-localisation to this square gives:
  \begin{center}
    \begin{tikzcd}
      \sum_{b:\shape_{\A^1}B}P(\shape_{\A^1}g(b)))\ar[r]\ar[d] & \sum_{y:\shape_{\A^1}Y}P(y)\ar[d] \\
       \shape_{\A^1}B\ar[r,swap,"\shape_{\A^1}g"] &  \shape_{\A^1}Y
    \end{tikzcd}
  \end{center}
  which is a pullback square.
\end{proof}

\begin{corollary}
Any fiber sequence:
\[X\to Y\to Z\]
with the second map $\A^1$-étale is an $\A^1$-fiber sequence.
\end{corollary}

\begin{lemma}
A map is $\A^1$-étale if and only if it induces equivalences of $\A^1$-disks. 
\end{lemma}

\begin{proof}
This is a \cite{cherubini_rijke_2021}[Proposition 3.7], using surjectivity from \cref{A1-replacement-surjective}.
\end{proof}

\begin{lemma}
Given a span:
\[ A \to B \leftarrow X\]
where $A$ and $B$ are $\A^1$-local, we have an equivalence:
\[\shape_{\A^1} (A\times_BX) \simeq A\times_B\shape_{\A^1}X\]
\end{lemma}

\begin{proof}
Since $A\times_B\shape_{\A^1}X$ is $\A^1$-local, as a limit of $\A^1$-local type, it is enough to check that the map:
\[A\times_BX \to A\times_B\shape_{\A^1}X\]
has $\A^1$-contractible fibers to conclude. But its fibers are equivalent to fibers of the map:
\[X\to \shape_{\A^1}X\]
which are indeed $\A^1$-contractible.
\end{proof}

\begin{remark}
This implies that colimits in $\A^1$-local types, which are given by: 
\[\mathrm{colim}^{\A^1}_{i:I} X_i =  \shape_{\A^1} \mathrm{colim}_{i:I}X_i\]
are universal, despite $\A^1$-local types not forming a topos for lack of a universe. This holds for any localisation, in fact it holds for any orthogonal factorisation system where the left class is stable by pullback. 
\end{remark}