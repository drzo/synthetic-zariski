The aim here is to copy the abstract covering theory in \cite{cherubini_rijke_2021}[Section 8],
using the $\A^1$-shape defined further down.

We introduce yet another name for the coreduction, deRham stack, infinitesimal shape or crystalline modality:

\begin{definition}
  For any type $X$, let $\widetilde{X}$ denote the formally étale replacement of $X$.
\end{definition}

\begin{definition}
  Let $\A^\times\colonequiv \A^1\setminus\{0\}$.
\end{definition}

\begin{lemma}
  Let $(X,*)$ be a pointed type with a multiplicative left action of $R$,
  such that for all $x:X$, we have $0\cdot x=*$ and $1\cdot x=x$.
  Then $\shape_{\A^1}X=1$.
\end{lemma}

\begin{proof}
  For all $x:X$, we can construct the map
  \[
    \gamma_x\colonequiv (r:\A^1)\mapsto r\cdot x : \A^1\to X
  \]
  By modal induction, this yields a contraction of $\shape_{\A^1}X$ to $\sigma_X(*)$.
\end{proof}

\begin{proposition}
  $\shape_{\A^1}\A^\times=\widetilde{\A^\times}$.
\end{proposition}

\begin{proof}
  Use Zariski choice in the situation
  \begin{center}
    \begin{tikzcd}
      & \A^\times\ar[d,->>] \\
      \A^1\ar[r] & \widetilde{\A^\times} 
    \end{tikzcd}
  \end{center}
  Glue the resulting sections under a $\neg\neg$ to a section $\A^1\to\A^\times$.
  The section is merely of the form $x\mapsto a_0+\sum_{i=1}^na_ix^i$ with $a_0\neq 0$ and nilpotent $a_i$ for $i>0$.
  So the original map $\A^1\to \A^\times$ is constantly $\widetilde{a_o}$.
\end{proof}

\begin{conjecture}
  $\shape_{\A^1}\bP^1=\Susp(\widetilde{\A^\times})$
\end{conjecture}

\begin{definition}
  Let $\shape_n\colonequiv\shape_{\A^1,n}$ be the nullification at $\A^1$ and $S^{n+1}$.
\end{definition}

\begin{definition}
  The \notion{universal cover} of $X$ is the type $\hat{X}$ obtained by pullback:
  \begin{center}
    \begin{tikzcd}
      \hat{X}\ar[r]\ar[d] & 1\ar[d] \\
      X\ar[r] & \shape_1 X
    \end{tikzcd}
  \end{center}
\end{definition}

We import the notion of étale maps for a modality from \cite{cherubini_rijke_2021}.

\begin{definition}
  Let $\bigcirc$ be a modality (in the sense of the HoTT-Book),
  then $f:X\to Y$ is \notion{$\bigcirc$-étale}, if the naturality is a pullback:
  \begin{center}
    \begin{tikzcd}
      X\ar[r]\ar[d,"f"] & \bigcirc X\ar[d] \\
      Y\ar[r] & \bigcirc Y
    \end{tikzcd}
  \end{center}
  Moreover, a map $g:X\to Y$ is called \notion{$\bigcirc$-equivalence},
  if $\bigcirc g$ is an equivalence.
  The ($\bigcirc$-equivalences, $\bigcirc$-étale maps) is an orthogonal factorization system. 
\end{definition}

\begin{proposition}
  Let $X$ be a type, then $\hat{X}\to X$ is $\shape_{\A^1,1}$-étale and $\shape_{\A^1,1}\hat{X}=1$.
  In particular, we have the following lifting property:
  \begin{center}
    \begin{tikzcd}
      1\ar[r]\ar[d] & \hat{X}\ar[d] \\
      \A^1\ar[r]\ar[ru,dashed,"\exists!"] & X
    \end{tikzcd}
  \end{center}
\end{proposition}

\begin{proof}
  All statements are true for general modalities.
\end{proof}

\begin{example}
  The fibers of a universal cover naturally carry a group structure:
  \begin{center}
    \begin{tikzcd}
      \Omega(\Susp(\widetilde{\A^\times}))\ar[r]\ar[d] & \hat{\bP^1}\ar[r]\ar[d] & 1\ar[d] \\
      1\ar[r] & \bP^1\ar[r] & \Susp(\widetilde{\A^\times}) 
    \end{tikzcd}
  \end{center}
\end{example}