The goal here is to prove that any quasi-projective scheme is $\A^1$-equivalent to an affine scheme. We roughly follow [Homotopy Algebraic $K$-theory by Weibel].

\begin{definition}
An affine $\A^1$-replacement for a scheme $X$ consists of an affine scheme $W$ with an $\A^1$-connected map:
\[W\to X\]
\end{definition}

Typical affine $\A^1$-replacement are vector bundles or torsors over a vector bundle.

\begin{lemma}
There exists an affine $\A^1$-replacement for $\bP^n$.
\end{lemma}

\begin{proof}
Consider $W$ the type of projection of $R^{n+1}$ of rank $1$. There is a map:
\[p:W\to \bP^n\] 
sending a projection to its image.

The type $W$ is an affine scheme because it is equivalent to the type of square $n+1$ matrices $M$ such that $M^2=M$ and $M$ characteristic polynomial is $X^n(X-1)$. 

Now we need to check that the fibers of $p$ are $\A^1$-connected. Since giving a projection is equivalent to giving its image and its kernel, any fiber of $p$ is merely equivalent to the type of complements for a line in $R^{n+1}$. So all fibers are merely equivalent and we can just check that the fiber over $[1:0:\cdots:0]$ is $\A^1$-connected. This fiber is the type of matrices where the first line is of the form $(1,a_1,\cdots,a_n)$ and the rest is $0$. This is equivalent to $\A^n$ which is indeed $\A^1$-connected.
\end{proof}

\begin{lemma}
Let $p:X\to Y$ be an affine map between schemes. Then the pullback of an affine $\A^1$-replacement for $Y$ along $p$ is an affine $\A^1$-replacement for $X$.
\end{lemma}

\begin{proof}
Immediate, as affine schemes are closed under dependent product.
\end{proof}

Next proposition could be called Jouanolou's trick.

\begin{proposition}
Any quasi-projective scheme (defined as closed in open in projective space) merely has an affine $\A^1$-replacement.
\end{proposition}

\begin{proof}
The previous two lemmas cover projective schemes. TODO extend to open in projective, I think Felix knows how.
\end{proof}

\begin{remark}
This result can be extended to any scheme with an ample family of line bundles (known as Jouanolou-Thomason Theorem).
\end{remark}


