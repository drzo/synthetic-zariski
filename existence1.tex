\newcommand{\inc}{\mathsf{inc}}
\newcommand{\inl}{\mathsf{inl}}
\newcommand{\inr}{\mathsf{inr}}
%\newcommand{\UU}{\mathcal{U}}
\newcommand{\II}{\mathbf{I}}
\newcommand\norm[1]{\left\lVert #1 \right\rVert}


\newcommand{\Gm}{\mathsf{G_m}}
\newcommand{\ext}{\mathsf{ext}}
\newcommand{\patch}{\mathsf{patch}}
\newcommand{\cov}{\mathsf{cov}}
\newcommand{\isSheaf}{\mathsf{isSheaf}}
\newcommand{\Fib}{\mathsf{Fib}}

\newcommand{\BB}{\mathcal{B}}
\newcommand{\CC}{\mathcal{C}}
\newcommand{\UU}{\mathcal{U}}

In this section, we present models of the 3 axioms, models that are inspired by type theory. The first model is a sheaf model, and does not
take into account universe (and univalence). This model is built
as an {\em internal} model inside a presheaf model. Rather surprisingly, the same method
works if we start from a(n effective) model of univalence. We can localise at a family of open modalities, and  obtain
in this way a model of type theory with univalence satisfying the
3 axioms for Zariski topos. It is then natural to ask how the global section operation behaves for this model, and we show that
it satisfies a property similar to Zariski local choice. 

\subsection{$1$-topos model}

We first build a $1$-sheaf model of the 3 axioms. This model will be formulated as a model of {\em dependent type theory}.
This will be an {\em internal} model inside a presheaf model. While the presheaf model covers universes, this internal model
will not cover universes (and univalence), and we need a refined notion of model in the next section to cover these extensions.

For any small category $\CC$ we can form the presheaf model of type theory over the base category $\CC$.

A context is interpreted as a presheaf over $\CC$. A type over a context $\Gamma$ is interpreted as
a presheaf over the category of elements of $\Gamma$, and an element of this type as a global section
of this presheaf. For any set theoretic universe $\UU$, we have a corresponding presheaf which to
any object $X$ associates the set of $\UU$-small types over $Yo(X)$, and this gives the interpretion
of universes. 

\medskip

We look at the special case where $\CC$ is the opposite of the category of finitely presented $k$-algebra for a fixed
ring $k$.

    In this model we have a presheaf $R(A) = Hom(k[X],A)$ which has a ring structure.

    We have the truth value presheaf $\Omega$ with $\Omega(A)$ is the set of sieves on $A$.
    (There is a predicative version of this object if we are working in a predicative metatheory.)

    We have a dependent type $[p]$ for $p:\Omega$, with $[p]$ subpresheaf of the unit presheaf $\top$.

    We also have a presheaf $Cov$, where $Cov(A)$ is the set of finite sequences $f_1,\dots,f_n$ that are comaximal: $1$ is in the
    ideal generated by $f_1,\dots,f_n$.

    We define $Cov\rightarrow\Omega,~c\mapsto p_c$ by defining $p_c$ to be the proposition $\inv(f_1)\vee\dots\vee\inv(f_n)$
    for $c = f_1,\dots,f_c$. In particular, for $n=0$, this is the proposition $\perp$.

    We can then define the property $\isSheaf$ of being a sheaf for a type $T$ as being the fact that
    the constant map $T\rightarrow T^{[p_c]}$ are isomorphism for all $c:Cov$.

    \medskip

    One can check directly that $\Pi_{x:A}B$ is a sheaf if $B$ is a family of sheaves over $A$
    and that $\Sigma_{x:A}B$ is a sheaf if $A$ is a sheaf and $B$ is a family of sheaves. We hence get a model of type theory
    by interpreting a type as a sheaf. A context is still interpreted as a presheaf, while a type is now a dependent presheaf
    with a proof that it is {\em modal}. An element is a section of the underlying dependent presheaf.

    \medskip

    Any representable presheaf is a sheaf, and in this new model $R$ is a local ring. In particular, we have $\neg (0 = 1)$.
    
    \medskip

    However the type of sheaves in given universe $\Sigma_{X:\UU}[\isSheaf(X)]$ is not a sheaf.

    We present another model which both interprets universe and {\em univalence} and {\em propositional truncation}.

\subsection{Models of univalence}

The constructive models of univalence are presheaf models parametrised by an interval object $\II$
(presheaf with two global distinct elements $0$ and $1$ and which is tiny) and a classifier object
$\Phi$ for cofibrations.

 This is over a base category $\BB$.
 
 If we have another category $\CC$, we automatically get a new model of univalent type theory by
 changing $\BB$ to $\BB\times\CC$.

 A particular case is if $\CC$ is the opposite of the category of f.p. $k$-algebras, where $k$ is a
 fixed commutative ring.

 We have the presheaf $R$ defined by $R(J,A) = Hom(k[X],A)$ where $J$ object of $\BB$ and $A$ object of $\CC$.

  The presheaf $\Gm$ is defined by $\Gm(J,A) = Hom(k[X,1/X],A) = A^{\times}$ set of invertible elements of $A$.

\subsection{Propositional truncation}

    We start by giving a simpler interpretation of propositional truncation. This will simplify
    the proof of the validity of Zariski local choice in the model.

    We work in the presheaf model over a base category $\BB$ which interprets univalent type theory,
    with a presheaf $\Phi$ of cofibrations. The interpretation of the propositional
    truncation $\norm{T}$ {\em does not} require the use of the interval $\II$.

    We recall that in the models, to be contractible can be formulated as having an operation
    $\ext(\psi,v)$ which extends any partial element $v$ of extent $\psi$ to a total element.

    The (new) remark is then that to be a (h)proposition can be formulated as having instead
    an operation $\ext(u,\psi,v)$ which, now {\em given}
    an element $u$, extends any partial element $v$ of extent $\psi$ to a total element.

\medskip    

Propositional truncation is defined as follows. An element of $\norm{T}$ is either of the form
$\inc(a)$ with $a$ in $T$, or of the form $\ext(u,\psi,v)$ where $u$ is in $\norm{T}$ and $\psi$
in $\Phi$ and $v$ a partial element of extent $\psi$.

In this definition, the special constructor $\ext$ is a ``constructor with restrictions'' which
satisfies $\ext(u,\psi,v) = v$ on the extent $\psi$.

\subsection{Covering}

We want to force $e_c = \norm{\inv(f_1)+\dots+\inv(f_n)}$ for $f_1,\dots,f_n$ in $Cov$.

Note that we have $e_c\rightarrow p_c$, with $p_c$ is a {\em strict} proposition $\inv(f_1)\vee \dots\vee\inv(f_n)$.
If we work in an effective metatheory, without choices, we cannot expect to have $p_c\rightarrow e_c$.

\begin{lemma}
  The constant map $\sigma:R\rightarrow R^{e_c}$ is an isomorphism.
\end{lemma}

\begin{proof}
  It is enough to define a map $\patch:R^{e_c}\rightarrow R$ such that $\patch(\sigma(r)) = r$ since
  $e_c$ is a hproposition.

  If we are at a stage $A$ and $f_1,\dots,f_n$ comaximal in $A$, and $u:R^{e_c}(A)$ we define $\patch(u)$ as follows.
  Note that $e_c$ becomes inhabited in each $A[1/f_i]$. We get then a family of element $u_i$ in each $A[1/f_i]$, with $u_i$ compactible
  with $u_j$. We can then patch these elements together to an element in $A$.
\end{proof}

(Since $R$ is a strict hset, the canonical map $R^{p_c}\rightarrow R^{e_c}$ is an isomorphism.)

We now consider the model which, intuitively, forces all these hpropositions $e_c$ to be contractible.
This is obtained by restricting ourseleves to {\em modal types} $T$, the types
$T$ such that the constant map $T\rightarrow T^{e_c}$ is an {\em equivalence}.

\medskip

We have just seen that the strict hset $R$ is such a modal type.

The same reasoning actually shows that each representable presheaf, in particular $\Gm$, is a modal
(strict) hset.

\medskip

We get a model of type theory with univalence where a type is interpreted as a modal type.
In this model $\perp$ is interpreted as $1=0$. We can check that if $T$ is modal then
$T^{1=0}$ is contractible (using the covering with $n=0$), since this is equivalent to $(T^{\perp})^{1=0}$.

\medskip

If $P$ is a proposition, then $P$ is modal if, and only if,
we have $P^{e_c}\rightarrow P$. (Thus $\perp$ is not modal, since we don't have $\neg\neg(1=0)$; indeed we have instead $\neg(1=0)$ in
the original presheaf model.)
Since $P$ is a
proposition, $P^{e_c}$ is equivalent to $P^{\inv(f_1)}\times \dots\times P^{\inv(f_n)}$.

\medskip

We can use this characterisation of modal proposition to define the interpretation of
$\norm{T}(A)$ in the {\em sheaf} model.

An element in $\norm{T}(A)$ is
\begin{enumerate}
\item either $\inc(a)$ with $a$ in $T(A)$,
\item or of the form $\ext(u,\psi,v)$ with $u$ in $\norm{T}(A)$ and $\psi$ in $\Phi$ and
  $v$ in $\norm{T}(A)$ of extent $\psi$,
\item or of the form $\cov(f_1,u_1,\dots,f_n,u_n)$ with $f_1,\dots,f_n$ in $Cov(A)$ and $u_i$ in $\norm{T}(A[1/f_i])$.
\end{enumerate}

If $\alpha:A\rightarrow B$ we define the restriction operation inductively as follows.
\begin{enumerate}
\item $\inc(a)|\alpha = \inc(a|\alpha)$
\item $\ext(u,\psi,v)|\alpha = \ext(u|\alpha,\psi,v|\alpha)$
\item $\cov(f_1,u_1,\dots,f_n,u_n)|\alpha = \cov(\alpha f_1,u_1|\alpha_1,\dots,\alpha f_n,u_n|\alpha_n)$ where
  $\alpha_i$ is the composition $A\rightarrow B\rightarrow B[1/\alpha f_i]$.
\end{enumerate}

If $P$ is a modal proposition and $u:T\rightarrow P$ we can define a map $v:\norm{T}\rightarrow P$
such that $v\circ\inc = u$ as a strict equality.

We can then define $l:\Pi_{f_1,\dots,f_n:R}\norm{\inv(f_1)+\dots+\inv(f_n)}$.

\subsection{Zariski local choice}

If $c = f_1,\dots,f_n$ is a covering of $A$ and $P:Sp(A)\rightarrow \UU$ we define
$\Pi_c P$ to be $\Pi_{D(f_1)}P\times\Pi_{D(f_n)}P$. In this way, $\Pi_c$ is an operation
$\UU^{Sp(A)}\rightarrow\UU$.

We prove local choice: if $A$ is a f.p. algebra over $R$ then we have a map
$$
l:(\Pi_{x:Sp(A)}\norm{P})\rightarrow \norm{\Sigma_{c:Cov(A)}\Pi_cT}
$$
At a stage $B$ a f.p. algebra over $R$ is given by $B\rightarrow A$ and we have $Yo(B).Sp(A)$ isomorphic
to $Yo(A)$.

For defining the map $l$, we define $l(v)$ by induction on $v$.
The element $v$ is in $(\Pi_{x:Sp(A)}\norm{P})(B)$, which can be seen as
an element of $\norm{P}(A)$. If it is $\inc(u)$ we associate $\inc(1,u)$ with the covering $1$ of $A$.
If it is $\ext(u,\psi,v)$ the image is $\ext(l(u),\psi,l(v))$.
If it is $\cov(f_1,u_1,\dots,f_n,u_n)$ we have by induction $l(u_i)$ in $\norm{\Sigma_{c:Cov(A[1/f_i])}\Pi_c P}$.
We can then conclude using the law $\norm{A}\times \norm{B}\rightarrow\norm{A\times B}$.

\medskip

The fact that $R$ is local holds like in the $1$-topos case, and similarly for the fact that
$A\rightarrow R^{Sp(A)}$ is an isomorphism for $A$ f.p. over $R$.

\medskip

 This model is a  simplified version of the sheaf model considered in \cite{CRS21}. It contains however the expected notion
of descent data. The following Lemma illustrates what is going on. If $p$ is a proposition, a partial element of a type $T$ of
extent $p$ is an element of $T^p$.

\begin{lemma}
  Let $p_0,p_1,p_2$ be propositions. The type $T^{p_0\vee p_1\vee p_2}$ is equivalent to the type of tuples $u_0,u_1,u_2$
  where $u_i$ is a partial element of extent $p_i$ together with paths $u_{ij}:u_i =_T u_j$ of extent $p_i\wedge p_j$
  satisfying the cocycle condition $u_{01}\cdot u_{12} = u_{02}$ on the extend $p_0\wedge p_1\wedge p_2$.
\end{lemma}

We can generalize this as follows. If $c = f_1,\dots,f_n:Cov(A)$  and $T$ is a presheaf defined at level $A$, we define
$D_c(T)(A)$, a descent data for $T$ for the covering $c$, as the type of family $u_K(i):T[1/f_K]$ where $K$ is a nonempty
finite subset of $1,\dots,n$ and $i$ an element of $\II^K$ such that $\vee_p (i(p) = 1)$ and $u_K(i) = u_L(i_{|L})$
if $K = L,p$ and $i(p) = 0$. We can then check that $D_c(T)$ is equivalent to $T^{inv(f_1)\vee\dots\vee inv(f_n)}$.
So $T$ is a sheaf iff the canonical map $T\rightarrow D_c(T)$ is an equivalence.

\medskip

If $T$ is a hset ($0$-type), we recover the usual patching condition: if we have $u_i:T[1/f_i]$ with an equality $u_i = u_j$ on
$T[1/f_if_j]$ we can find $u$ in $T(A)$ such that $u = u_i$ on $T[1/f_i]$.

If $T$ is a $1$-type, we recover the usual patching condition: if we have $u_i:T[1/f_i]$ with an equality $e_{ij}:u_i = u_j$ on
$T[1/f_if_j]$, with the cocycle condition $e_{ij}\cdot e_{jk} = e_{ik}$,
we can find $u$ in $T(A)$ with $e_i:u = u_i$ on $T[1/f_i]$ such that $e_{ij}\cdot e_j = e_i$.

%% This follows from the following Lemma, which holds in any effective model of type theory.

%% \begin{lemma}
%%   Let $p_i,~i<n$ be propositions. The type $T^{\vee p_i}$ is equivalent to the type of descent data $u_{l_0\dots l_m}(i_0,\dots,i_m)$
%%   of extend $p_{l_0}\wedge\dots\wedge p_{l_m}$
%%   with $l_0<\dots<l_m$ satisfying the compatibility conditions of \cite{CRS21}.
%% \end{lemma}

%% For $n = 3$ we give $u_{021}(i_0,i_1,i_2)$ with $u_{012}(0,i_1,i_2) = u_{12}(i_1,i_2),~u_{012}(i_0,0,i_2) = u_{02}(i_0,i_2)$
%% and $u_{012}(i_0,i_1,0) = u_{01}(i_0,i_1)$ and $u_{pq}(1,0) = u_p(1)$ and $u_{pq}(0,1) = u_q(1)$. This is one way to express
%% the cocycle condition.


\subsection{Global sections and Zariski global choice}

If $T$ is a sheaf defined at level $A$, we let $\Box_A T$ the type of global sections.
If $c = f_1,\dots,f_n$ is in $Cov(A)$ we let $\Box_c T$ be the type $\Box_{A[1/f_1]}T\times\dots\times\Box_{A[1/f_n]}T$.

Using these notations, we can state the principle of Zariski global choice
$$
(\Box \norm{T})\leftrightarrow \norm{\Sigma_{c:Cov(k)}\Box_c T}
$$

This is valid in the present model.

Using this principle, we can show that $\Box K(\Gm,1)$ is equal to the type of projective modules of rank $1$ over $k$
and that each $\Box K(R,n)$ for $n>0$ is contractible.
                                                                                  




 
