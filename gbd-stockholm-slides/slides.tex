% latexmk -pdflatex='xelatex %O %S' -pvc -pdf slides.tex
\documentclass{beamer}

% für xelatex:
\usepackage{xltxtra}
\usepackage{unicode-math}

% literatur
\usepackage[backend=biber,style=alphabetic]{biblatex}

\addbibresource{../util/literature.bib}

\usepackage{../util/zariski}

\usepackage{csquotes}
\usepackage{hyperref}
\usepackage{tikz}
\usetikzlibrary{cd,arrows,shapes,calc,through,backgrounds,matrix,trees,decorations.pathmorphing,positioning,automata}
\usepackage{graphicx}
\usepackage{color}

\usepackage{mathpartir}
\newcommand{\yields}{\vdash}
\newcommand{\cbar}{\, | \,}


% für tabellen
\usepackage{booktabs}

\title[ConCoh]
{Synthetic Algebraic Geometry}
\author[Author, Anders] 
{Ingo Blechschmidt, Felix Cherubini, Thierry Coquand, Matthias Hutzler, Hugo Moeneclaey, Josselin Poiret and David Wärn}
\institute
{Working on various sub-projects}

\begin{document}

\date{}
\begin{frame}
  \titlepage
\end{frame}

\begin{frame}
  \frametitle{Acknowledgements and Subprojects}
  The approach to cohomology we use was proposed by \textbf{Michael Shulman} in 2013 and was later worked out by \textbf{Floris van Doorn}.
  It is known to many people in the field. \\ ~\\
  \pause
  The approach to synthetic algebraic geometry is based on work by  \textbf{Ingo Blechschmidt}, \textbf{Anders Kock} and \textbf{David Jaz Myers}. \\  ~\\
  \pause
  Currently, there are the following projects:
  \begin{table}
    \centering
    \begin{tabular}{lp{7.5cm}}
      Foundations & Felix, Matthias, Thierry \\
      Proper Schemes & David, Felix, Matthias, Thierry \\
      Differential Geometry & David, Felix, Hugo, Matthias \\
      \v{C}ech Cohomology & David, Felix, Ingo \\
      Formalization & Felix, Joisselin, Matthias \\
    \end{tabular}
  \end{table}
  
\end{frame}

\begin{frame}
  \begin{center}
    \begin{tikzpicture} 
      [ 
      node distance=.8cm, 
      circled/.style={draw, ellipse, ultra thick, fill=blue!12},
      edge from parent/.style={very thick,draw=black,-latex},
      plaintext/.style={}
      ]

      \tikzstyle{level 1}=[sibling angle=81,level distance=4cm]

      \node[circled] (hott) {HoTT+Axioms} [counterclockwise from=207]
      child { 
        node[circled] (grp) {$\infty$-Groupoids} 
        edge from parent [-latex] node (grp) 
        {
          \begin{tikzpicture}[scale=0.8, rotate=25]
            \draw[-, red, ultra thick] (0,0) to (1,1);
            \draw[-, red, ultra thick] (1,0) to (0,1);
          \end{tikzpicture}
        } 
      }
      child { node[matrix, circled, inner sep=0pt] (smgrp) 
        {
          \node[circled, minimum height=0.7cm, minimum width=3cm] (mfd) {Schemes};
          \node[below of=mfd, text width=3cm, 
          node distance=0.9cm] 
          {``Cubical Zariski-sheaves''}; \\
        }
      }
      ;
    \end{tikzpicture}
  \end{center}
  ~\\
  ~\\
  {\footnotesize Schemes = quasi-compact, quasic-separated schemes of finite type}
\end{frame}

\begin{frame}
  \frametitle{Synthetic algebraic geometry}
  \textbf{Axiom:} We have a local, commutative ring $R$. \\
  \pause
  We define $\Spec(A)$ for a finitely presented $R$-algebra $A$ by
  \[ \Spec(A):\equiv \Hom_{R\mathrm{-algebra}}(A,R)\]
  \pause
  \textbf{Axiom (synthetic quasi-coherence/(SQC)):} \\
  For any finitely presented $R$-algebras $A$, the evaluation map
  \[ r\mapsto (f\mapsto f(r)):A\to R^{\Spec(A)} \]
  is an equivalence.
  \pause
  \textbf{Standard open} subsets of $\Spec(A)$ are given for each $f:A$:
  \[ D(f):\equiv \{ x:\Spec(A) \mid x(f)\text{ is invertible}\}\]
\end{frame}

\begin{frame}
  \frametitle{Cohomology of sheaves}
  Let $X$ be a type and $\mathcal F:X\to \mathrm{Ab}$ a dependent abelian group on $X$. \\
  \pause
  The $n$-th cohomology group of $\mathcal F$ is
  \[ H^n(X,\mathcal F):\equiv\left\|\prod_{x:X}K(\mathcal F_x,n)\right\|_0 \]
  \pause
  Properties: \\
  \pause
  The $H^n(X,\mathcal F)$ are all abelian groups. \\
  \pause
  Functoriality, covariant in $\mathcal F$, contravariant in $X$. \\
  \pause
  Some long exact sequence for coefficients. \\
  \pause
  We have a Mayer-Vietoris-Lemma and more generally correspondence with \v{C}ech-Cohomology, for nice enough spaces. \\
\end{frame}

\begin{frame}
  \frametitle{Zariski-Choice and Cohomology}  
  Let $X=\Spec(A)$ and $M:X\to R\text{-Mod}$ such that $((x:D(f))\to M_x)=((x:X)\to M)_f$, then
  \[ H^1(X,M)=0 \]
  \pause
  \textbf{Proof:} Let $|T|: H^1(X,M)\equiv \|(x:X)\to K(M_x,1)\|_0$ and from that $(x:X)\to \|T_x=M_x\|$.
  \pause
  Our third axiom, \textbf{Zariski-local choice}, merely gives us coprime $f_1,\dots,f_n:A$, such that for each $i$ we have
  \[
    s_i:(x:D(f_i)) \to T_x=M_x
    \rlap{.}
  \]
  \pause
  So for $t_{ij}(x):\equiv s_j(x)^{-1}\cdot s_i(x)$ we have $t_{ij}+t_{jk}=t_{ik}$.
  \pause
  By algebra, we get $u_i:(x:D(f_i))\to M_x$ with $t_{ij}=u_i-u_j$.
  Then the $\tilde{s}_i:\equiv s_i-u_i$ glues to a global trivialization.
\end{frame}

\end{document}
