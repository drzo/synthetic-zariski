% latexmk -pdfxe -pvc main.tex
\documentclass{zariski}

\begin{document}

\tableofcontents

\section{Axioms}

We always assume there is a commutative ring $R$.
Sometimes we will assume $R$ has additional properties, or, more generally,
axioms hold that involve $R$.
We will always mention which of these axiom are needed to prove each statement,
by listing the shorthands introduced in the axioms below.

\begin{axiom}[Loc]%
  \label{loc}
  $R$ is a local ring.
\end{axiom}

\begin{axiom}[SQC]%
  \label{sqc}
  For any finitely presented $R$-algebra $A$, the homomorphism
  \[ a \mapsto (\varphi\mapsto \varphi(a)) : A \to (\Spec A \to R)\]
  is an isomorphism of $R$-algebras.
\end{axiom}

\begin{axiom}[Z-choice]%
  \label{Z-choice}
  Let $A$ be a finitely presented $R$-algebra
  and let $B : \Spec A \to \mU$ be a family of inhabited types.
  Then there merely exists
  a finite list of coprime elements $f_1, \dots, f_n \in A$
  together with dependent functions $s_i : \Pi_{x : D(f_i)} B(x)$.
  As a formula:
  \[ (\Pi_{x : \Spec A} \propTrunc{B(x)}) \to
     \propTrunc{ \Sigma_{n : \N} \Sigma_{f_1, \dots, f_n : A}
      ((f_1, \dots, f_n) = (1)) \times
      \Pi_i \Pi_{x : D(f_i)} B(x) }
     \rlap{.}
  \]
\end{axiom}

\ignore{
  AUGSBURG IDEAS:

  * Spec k[[X]] = D
  * Pushout of two schemes along open subscheme is scheme
  * for V : X \to R-Vect-n \bP(V) is scheme
  * P as space of line (in some vector space)
  * open/closed subsets are D(f) with values in a line bundle
  * define separated schemes, compare with 'inequality is an apartness relation'
  * separated => inequality is apartness  
  * define intersection (of curves) as pullbacks
  * define finite schemes
  * show good intersections are finite schemes?
  * show that dim (R^(intersection)) is the right thing (in examples)
    e.g. for curves in 2-dim. projective space, it should be the product of the degrees
  * irreducibility: (X irred :<=> (X=A∪B => X=A ∨ X=B)) shown for A^1, counterexample D(XY)
  
  }

  
\ignore{

(fp) Affine schemes
  
  local ring

  (Loc) R is a fixed local ring
  
  finitely presented algebra
  
  Spec

  _ : A -> R^(Spec A)
  _ : X -> Spec (R^X)

  coupled

  (SQC) Synthetic quasi coherence

  External justification

  is-affine

}

\section{Affine schemes}
\subsection{Affine-open subtypes}

We only talk about affine schemes of finite type, i.e. schemes of the form $\Spec A$ (\cref{spec}),
where $A$ is a finitely presented algebra.

\begin{definition}%
  A type $X$ is \notion{(qc-)affine},
  if there is a finitely presented $R$-algebra $A$, such that $X=\Spec A$.
\end{definition}

\begin{proposition}%
  Let $X$ be a type.
  The type of all finitely presented $R$-algebras $A$, such that $X=\Spec A$, is a proposition.
\end{proposition}

When we write ``$\Spec A$'' we implicitly assume $A$ is a finitely presented $R$-algebra.
Recall from \cref{standard-open-subset}
that the standard open subset $D(f) \subseteq \Spec A$
is given by $D(f)(x)\colonequiv \inv(f(x))$.

\begin{example}[using \axiomref{loc}, \axiomref{sqc}]
  For $a_1, \dots, a_n : R$, we have
  \[ D((X - a_1) \cdots (X - a_n)) = \A^1 \setminus \{ a_1, \dots, a_n \} \rlap{.}\]
  Indeed,
  for any $x : \A^1$,
  $((X - a_1) \dots (X - a_n))(x)$ is invertible if and only if
  $x - a_i$ is invertible for all $i$.
  But by \cref{non-zero-invertible}
  this means $x \neq a_i$ for all $i$.
\end{example}

\begin{definition}%
  \label{affine-open}
  Let $X=\Spec A$.
  A subtype $U:X\to\Prop$ is called \notion{affine-open},
  if one of the following logically equivalent statements holds:
  \begin{enumerate}[(i)]%
  \item $U$ is the union of finitely many affine standard opens.
  \item There are $f_1,\dots,f_n:A$ such that
    \[U(x) \Leftrightarrow \exists_{i} f_i(x)\neq 0 \]
  \end{enumerate}
\end{definition}

We sometimes write $D(f_1, \dots, f_n) \colonequiv D(f_1) \cup \dots \cup D(f_n)$
for a finite union of standard opens.
Note that in general, affine-open subtypes do not need to be affine
-- this is why we use the dash ``-''.

We will introduce a more general definition of open subtype in \cref{qc-open}
and show in \cref{qc-open-affine-open}, that the two notions agree on affine schemes.

\begin{proposition}
  Let $X = \Spec A$ and $f : A$.
  Then $D(f) = \Spec A[f^{-1}]$.
\end{proposition}

\begin{proof}
  \[ D(f) =
     \sum_{x : X} D(f)(x) =
     \sum_{x : \Spec A} \inv(f(x)) =
     \sum_{x : \Hom_{\Alg{R}}(A, R)} \inv(x(f)) =
     \Hom_{\Alg{R}}(A[f^{-1}], R) =
     \Spec A[f^{-1}]
     \]
\end{proof}

Affine-openness is transitive in the following sense:

\begin{lemma}%
  \label{affine-open-trans}
  Let $X=\Spec A$ and $D(f)\subseteq X$ be a standard open.
  Any affine-open subtype $U$ of $D(f)$ is also affine-open in $X$.
\end{lemma}

\begin{proof}
  It is enough to show the statement for $U=D(g)$, $g:A_f$.
  Then
  \[ g=\frac{h}{f^k}\rlap{.}\]
  Now $D(hf)$ is an affine-open in $X$,
  that coincides with $U$: \\
  Let $x:X$, then $(hf)(x)$ is invertible, if and only if both $h(x)$ and $f(x)$ are invertible.
  The latter means $x:D(f)$, so we can interpret $x$ as a homorphism from $A_f$ to $R$.
  Then $x:D(g)$ means $x(g)$ is invertible, which is equivalent to $x(h)$ being invertible,
  since $x(f)^k$ is invertible anyway.
\end{proof}

\begin{lemma}[using \axiomref{loc}, \axiomref{sqc}]%
  \label{standard-open-empty}
  Let $X=\Spec A$ be an affine scheme and $D(f)\subseteq X$ a standard open,
  then $D(f)=\emptyset$, if and only if, $f$ is nilpotent.
\end{lemma}

\begin{proof}
  Since $D(f)=\Spec A_f$, by \cref{weak-nullstellensatz}, we know $D(f)=\emptyset$,
  if and only if, $A_f=0$.
  The latter is equivalent to $f$ being nilpotent.
\end{proof}

More generally,
the Zariski-lattice consisting of the radicals
of finitely generated ideals of a finitely presented $R$-algebra $A$,
coincides with the lattice of open subtypes.
This means, that internal to the Zariski-topos,
it is not neccessary to consider the full Zariski-lattice for a constructive treatment of schemes.

\begin{lemma}[using \axiomref{sqc}]%
  Let $A$ be a finitely presented $R$-algebra
  and let $f, g_1, \dots, g_n \in A$.
  Then we have $D(f) \subseteq D(g_1, \dots, g_n)$
  as subsets of $\Spec A$
  if and only if $f \in \sqrt{(g_1, \dots, g_n)}$.
\end{lemma}

\begin{proof}
  Since $D(g_1, \dots, g_n) = \{\, x \in \Spec A \mid x \notin V(g_1, \dots, g_n) \,\}$,
  the inclusion $D(f) \subseteq D(g_1, \dots, g_n)$
  can also be written as
  $D(f) \cap V(g_1, \dots, g_n) = \varnothing$, \rednote{(UNDEFINED NOTATION ``$V(g_1,\dots,g_m)$'')} that is,
  $\Spec((A/(g_1, \dots, g_n))[f^{-1}]) = \varnothing$.
  By (\axiomref{sqc})
  this means that the finitely presented $R$-algebra $(A/(g_1, \dots, g_n))[f^{-1}]$
  is zero.
  And this is the case if and only if $f$ is nilpotent in $A/(g_1, \dots, g_n)$,
  that is, if $f \in \sqrt{(g_1, \dots, g_n)}$, as stated.
\end{proof}

In particular,
we have $\Spec A = \bigcup_{i = 1}^n D(f_i)$
if and only if $(f_1, \dots, f_n) = (1)$.

\subsection{Pullbacks of affine schemes}

\begin{lemma}%
  \label{affine-product}
  The product of two affine schemes is again an affine scheme,
  namely
  $\Spec A \times \Spec B = \Spec (A \otimes_R B)$.
\end{lemma}

\begin{proof}
  By the universal property of the tensor product $A \otimes_R B$.
\end{proof}

More generally we have:

\begin{lemma}[using \axiomref{sqc}]%
  \label{affine-fiber-product}
  Let $X=\Spec A,Y=\Spec B$ and $Z=\Spec C$ be affine schemes
  with maps $f:X\to Z$, $g:Y\to Z$.
  Then the pullback of this diagram is an affine scheme given by $\Spec (A\otimes_C B)$.
\end{lemma}

\begin{proof}
  The maps $f:X\to Z$, $g:Y\to Z$ are induced by $R$-algebra homomorphisms $f^*:A\to R$ and $g^*:B\to R$.
  Let
  \[ (h,k,p) : \Spec A \times_{\Spec C} \Spec B \]
  with $p:h\circ f^*=k\circ g^* $.
  This defines a $R$-cocone on the diagram
  \[
    \begin{tikzcd}
      A & C\ar[r,"g^*"]\ar[l,"f^*",swap] & B
    \end{tikzcd}
  \]
  Since $A\otimes_C B$ is a pushout in $R$-algebras,
  there is a unique $R$-algebra homomorphism $A\otimes_C B \to R$ corresponding to $(h,k,p)$.
\end{proof}

\subsection{Boundedness of functions to $\N$}

While the axiom \axiomref{sqc}
describes functions on an affine scheme
with values in $R$,
we can generalize it to functions taking values
in another finitely presented $R$-algebra,
as follows.

\begin{lemma}[using \axiomref{sqc}]%
  \label{algebra-valued-functions-on-affine}
  For finitely presented $R$-algebras $A$ and $B$,
  the function
  \begin{align*}
    A \otimes B &\xrightarrow{\sim} (\Spec A \to B) \\
    c &\mapsto (\varphi \mapsto (\varphi \otimes B)(c))
  \end{align*}
  is a bijection.
\end{lemma}

\begin{proof}
  We recall $\Spec (A \otimes B) = \Spec A \times \Spec B$
  from \cref{affine-product}
  and calculate as follows.
  \begin{align*}
    A \otimes B
    &=& (\Spec (A \otimes B) \to R)
    &=& (\Spec A \times \Spec B \to R)
    &=& (\Spec A \to (\Spec B \to R))
    &=& (\Spec A \to B) \\
    c
    &\mapsto& (\chi \mapsto \chi(c))
    &\mapsto& ((\varphi, \psi) \mapsto (\varphi \otimes \psi)(c))
    &\mapsto& (\varphi \mapsto (\psi \mapsto (\varphi \otimes \psi)(c)))
    &\mapsto& (\varphi \mapsto (\varphi \otimes B)(c))
  \end{align*}
  The last step is induced by the identification
  $B = (\Spec B \to R),\, b \mapsto (\psi \mapsto \psi(b))$,
  and we use the fact that
  $\psi \circ (\varphi \otimes B) = \varphi \otimes \psi$.
\end{proof}

\begin{lemma}[using \axiomref{sqc}]%
  \label{eventually-vanishing-sequence-on-affine}
  Let $A$ be a finitely presented $R$-algebra
  and let $s : \Spec A \to (\N \to R)$
  be a family of sequences,
  each of which eventually vanishes:
  \[ \prod_{x : \Spec A} \propTrunc{\sum_{N : \N} \prod_{n \geq N} s(x)(n) = 0} \]
  Then there merely exists one number $N : \N$
  such that $s(x)(n) = 0$ for all $x : \Spec A$ and all $n \geq N$.
\end{lemma}

\begin{proof}
  The set of eventually vanishing sequences $\N \to R$
  is in bijection with the set $R[X]$ of polynomials,
  by taking the entries of a sequence as the coefficients of a polynomial.
  So the family of sequences $s$
  is equivalently a family of polynomials $s : \Spec A \to R[X]$.
  Now we apply \cref{algebra-valued-functions-on-affine} with $B = R[X]$
  to see that such a family corresponds to a polynomial $p : A[X]$.
  Note that for a point $x : \Spec A$,
  the homomorphism
  \[ x \otimes R[X] : A[X] = A \otimes R[X] \to R \otimes R[X] = R[X] \]
  simply applies the homomorphism $x$ to every coefficient of a polynomial,
  so we have $(s(x))_n = x(p_n)$.
  This concludes our argument,
  because the coefficients of $p$,
  just like any polynomial,
  form an eventually vanishing sequence.
\end{proof}

\begin{theorem}[using (\axiomref{loc}), (\axiomref{sqc})]%
  \label{boundedness}
  Let $A$ be a finitely presented $R$-algebra.
  Then every function $f : \Spec A \to \N$ is bounded:
  \[ \Pi_{f : \Spec A \to \N} \propTrunc{\Sigma_{N : \N} \Pi_{x : \Spec A} f(x) \le N}
     \rlap{.} \]
\end{theorem}

\begin{proof}
  Given a function $f : \Spec A \to \N$,
  we construct the family $s : \Spec A \to (\N \to R)$
  of eventually vanishing sequences
  given by
  \[
    s(x)(n) \colonequiv
    \begin{cases}
      \,1 &\text{if $n < f(x)$}\\
      \,0 &\text{else} \rlap{.}
    \end{cases}
  \]
  Since $0 \neq 1 : R$ by \axiomref{loc},
  we in fact have $s(x)(n) = 0$ if and only if $n \geq f(x)$.
  Then the claim follows from \cref{eventually-vanishing-sequence-on-affine}.
\end{proof}

If we also assume the axiom \axiomref{Z-choice},
we can formulate the following simultaneous strengthening
of \cref{eventually-vanishing-sequence-on-affine}
and \cref{boundedness}.

\begin{proposition}[using \axiomref{loc}, \axiomref{sqc}, \axiomref{Z-choice}]%
  \label{strengthened-boundedness}
  Let $A$ be a finitely presented $R$-algebra.
  Let $P : \Spec A \to (\N \to \Prop)$
  be a family of upwards closed, merely inhabited subsets of $\N$.
  Then the set
  \[ \bigcap_{x : \Spec A} P(x) \subseteq \N \]
  is merely inhabited.
\end{proposition}

\begin{proof}
  By \axiomref{Z-choice},
  there merely exists a cover
  $\Spec A = \bigcup_{i = 1}^n D(f_i)$
  and functions $p_i : D(f_i) \to \N$
  such that $p_i(x) \in P(x)$ for all $x : D(f_i)$.
  By \cref{boundedness},
  every $p_i : D(f_i) = \Spec A[f_i^{-1}] \to \N$
  is merely bounded by some $N_i : \N$,
  and then $\mathrm{max}(N_1, \dots, N_n) \in P(x)$ for all $x : \Spec A$.
\end{proof}

\subsection{Properties of the ring $R$}

We adopt the following definition from
\cite[Section IV.8]{lombardi-quitte}.

\begin{definition}%
  \label{zero-dimensional-ring}
  A ring $A$ is \notion{zero-dimensional}
  if for all $x : A$
  there exists $a : A$ and $k : \N$
  such that $x^k = a x^{k + 1}$.
\end{definition}

\begin{lemma}[using \axiomref{loc}, \axiomref{sqc}, \axiomref{Z-choice}]%
  \label{R-not-zero-dimensional}
  The ring $R$ is not zero-dimensional.
\end{lemma}

\begin{proof}
  Assume that $R$ is zero-dimensional,
  so for every $f : R$ there merely is some $k : \N$ with $f^k \in (f^{k + 1})$.
  We note that $R = \A^1$ is an affine scheme and
  that if $f^k \in (f^{k + 1})$,
  then we also have $f^{k'} \in (f^{k' + 1})$ for every $k' \geq k$.
  This means that we can apply \cref{strengthened-boundedness}
  and merely obtain a number $K : \N$
  such that $f^K \in (f^{K + 1})$ for all $f : R$.
  In particular, $f^{K + 1} = 0$ implies $f^K = 0$,
  so the canonical map
  $\Spec R[X]/(X^K) \to \Spec R[X]/(X^{K + 1})$
  is a bijection.
  But this is a contradiction,
  since the homomorphism $R[X]/(X^{K + 1}) \to R[X]/(X^K)$
  is not an isomorphism.
\end{proof}

The following lemma,
which is a variant of \cite{ingo-thesis}[Proposition 18.32],
shows that $R$ is in a weak sense algebraically closed.

\begin{lemma}[using \axiomref{loc}, \axiomref{sqc}]%
  \label{polynomials-notnot-decompose}
  Let $f : R[X]$ be a polynomial.
  Then it is not not the case that:
  either $f = 0$ or
  $f = \alpha \cdot {(X - a_1)}^{e_1} \dots {(X - a_n)}^{e_n}$
  for some $\alpha : R^\times$,
  $e_i \geq 1$ and pairwise distinct $a_i : R$.
\end{lemma}

\begin{proof}
  Let $f : R[X]$ be given.
  Since our goal is a proposition,
  we can assume we have a bound $n$ on the degree of $f$,
  so
  \[ f = \sum_{i = 0}^n c_i X^i \rlap{.} \]
  Since our goal is even double-negation stable,
  we can assume $c_n = 0 \lor c_n \neq 0$
  and by induction $f = 0$ (in which case we are done)
  or $c_n \neq 0$.
  If $n = 0$ we are done,
  setting $\alpha \colonequiv c_0$.
  Otherwise,
  $f$ is not invertible (using $0 \neq 1$ by (\axiomref{loc})),
  so $R[X]/(f) \neq 0$,
  which by (\axiomref{sqc}) means that
  $\Spec(R[X]/(f)) = \{ x : R \mid f(x) = 0 \}$
  is not empty.
  Using the double-negation stability of our goal again,
  we can assume $f(a) = 0$ for some $a : R$
  and factor $f = (X - a_1) f_{n - 1}$.
  By induction, we get $f = \alpha \cdot (X - a_1) \dots (X - a_n)$.
  Finally, we decide each of the finitely many propositions $a_i = a_j$,
  which we can assume is possible
  because our goal is still double-negation stable,
  to get the desired form
  $f = \alpha \cdot {(X - \widetilde{a}_1)}^{e_1} \dots {(X - \widetilde{a}_n)}^{e_n}$
  with distinct $\widetilde{a}_i$.
\end{proof}

\begin{example}[using \axiomref{loc}, \axiomref{sqc}, \axiomref{Z-choice}]%
  \label{non-existence-of-roots}
  It is not the case that
  every monic polynomial $f : R[X]$ with $\deg f \geq 1$ has a root.
  More specifically,
  if $U \subseteq \A^1$ is an open subset
  with the property that
  the polynomial $X^2 - a : R[X]$ merely has a root
  for every $a : U$,
  then $U = \emptyset$.
\end{example}

\begin{proof}
  Let $U \subseteq \A^1$ be as in the statement.
  Since we want to show $U = \emptyset$,
  we can assume a given element $a_0 : U$
  and now have to derive a contradiction.
  By \axiomref{Z-choice},
  there exists in particular a standard open $D(f) \subseteq \A^1$
  with $a_0 \in D(f)$
  and a function $g : D(f) \to R$
  such that ${(g(x))}^2 = x$ for all $x : D(f)$.
  By \axiomref{sqc},
  this corresponds to an element $\frac{p}{f^n} : R[X]_f$
  with ${(\frac{p}{f^n})}^2 = X : R[X]_f$.
  We use \cref{polynomial-with-regular-value-is-regular}
  together with the fact that $f(a_0)$ is invertible
  to get that $f : R[X]$ is regular,
  and therefore $p^2 = f^{2n}X : R[X]$.
  Considering this equation over $R^{\mathrm{red}} = R/\sqrt{(0)}$ instead,
  we can show by induction that all coefficients of $p$ and of $f^n$ are nilpotent,
  which contradicts the invertibility of $f(a_0)$.
\end{proof}

\begin{remark}
  \Cref{non-existence-of-roots} shows that
  the axioms we are using here
  are incompatible with a natural axiom that is true
  for the structure sheaf of the big étale topos,
  namely that $R$ admits roots for unramifiable monic polynomials.
  The polynomial $X^2 - a$ is even separable for invertible $a$,
  assuming that $2$ is invertible in $R$.
  To get rid of this last assumption,
  we can use the fact that either $2$ or $3$ is invertible in the local ring $R$
  and observe that the proof of \cref{non-existence-of-roots}
  works just the same for $X^3 - a$.
\end{remark}


\section{Topology of schemes}

\subsection{Closed subtypes}

\begin{definition}%
  \label{closed-proposition}\label{closed-subtype}
  \begin{enumerate}[(a)]
  \item
    A \notion{closed proposition} is a proposition
    which is merely of the form $x_1 = 0 \land \dots \land x_n = 0$
    for some elements $x_1, \dots, x_n \in R$.
  \item
    Let $X$ be a type.
    A subtype $U : X \to \Prop$ is \notion{closed}
    if for all $x : X$, the proposition $U(x)$ is closed.
  \item
    For $A$ a finitely presented $R$-algebra
    and $f_1, \dots, f_n : A$,
    we set
    $V(f_1, \dots, f_n) \colonequiv
    \{\, x : \Spec A \mid f_1(x) = \dots = f_n(x) = 0 \,\}$.
  \end{enumerate}
\end{definition}

Note that $V(f_1, \dots, f_n) \subseteq \Spec A$ is a closed subtype
and we have $V(f_1, \dots, f_n) = \Spec (A/(f_1, \dots, f_n))$.

\begin{proposition}[using \axiomref{sqc}]%
  There is an order-reversing isomorphism of partial orders
  \begin{align*}
    \text{f.g.-ideals}(R) &\xrightarrow{{\sim}} \Omega_{cl} \\
    I &\mapsto (I = (0))
  \end{align*}
  between the partial order of finitely generated ideals of $R$
  and the partial order of closed propositions.
\end{proposition}

\begin{proof}
  For a finitely generated ideal $I = (x_1, \dots, x_n)$,
  the proposition $I = (0)$ is indeed a closed proposition,
  since it is equivalent to $x_1 = 0 \land \dots \land x_n = 0$.
  It is also evident that we get all closed propositions in this way.
  What remains to show is that
  \[ I = (0) \Rightarrow J = (0)
     \qquad\text{iff}\qquad
     J \subseteq I
     \rlap{\text{.}}
  \]
  For this we use synthetic quasicoherence.
  Note that the set $\Spec R/I = \Hom_R(R/I, R)$ is a proposition
  (has at most one element),
  namely it is equivalent to the proposition $I = (0)$.
  Similarly, $\Hom_R(R/J, R/I)$ is a proposition
  and equivalent to $J \subseteq I$.
  But then our claim is just the equation
  \[ \Hom(\Spec R/I, \Spec R/J) = \Hom_R(R/J, R/I) \]
  which holds by \Cref{spec-embedding},
  since $R/I$ and $R/J$ are finitely presented $R$-algebras
  if $I$ and $J$ are finitely generated ideals.
\end{proof}

\begin{lemma}[using \axiomref{sqc}]%
  \label{ideals-embed-into-closed-subsets}
  We have $V(f_1, \dots, f_n) \subseteq V(g_1, \dots, g_m)$
  as subsets of $\Spec A$
  if and only if
  $(g_1, \dots, g_m) \subseteq (f_1, \dots, f_n)$
  as ideals of $A$.
\end{lemma}

\begin{proof}
  The inclusion $V(f_1, \dots, f_n) \subseteq V(g_1, \dots, g_m)$
  means a map $\Spec (A/(f_1, \dots, f_n)) \to \Spec (A/(g_1, \dots, g_m))$
  over $\Spec A$.
  By \Cref{spec-embedding}, this is equivalent to
  a homomorphism $A/(g_1, \dots, g_m) \to A/(f_1, \dots, f_n)$,
  which in turn means the stated inclusion of ideals.
\end{proof}

\begin{lemma}[using \axiomref{loc}, \axiomref{sqc}, \axiomref{Z-choice}]%
  \label{closed-subtype-affine}
  A closed subtype $C$ of an affine scheme $X=\Spec A$ is an affine scheme
  with $C=\Spec (A/I)$ for a finitely generated ideal $I\subseteq A$.
\end{lemma}

\begin{proof}
  By \axiomref{Z-choice} and boundedness,
  there is a cover $D(f_1),\dots,D(f_l)$, such that
  on each $D(f_i)$, $C$ is the vanishing set of functions
  \[ g_1,\dots,g_n:D(f_i)\to R\rlap{.} \]
  By \Cref{ideals-embed-into-closed-subsets},
  the ideals generated by these functions
  agree in $A_{f_i f_j}$,
  so by \Cref{fg-ideal-local-global},
  there is a finitely generated ideal $I\subseteq A$,
  such that $A_{f_i}\cdot I$ is $(g_1,\dots,g_n)$
  and $C=\Spec A/I$.
\end{proof}

\subsection{Open subtypes}

While we usually drop the prefix ``qc'' in the definition below,
one should keep in mind, that we only use a definition of quasi compact open subsets.
The difference to general opens does not play a role so far,
since we also only consider quasi compact schemes later.

\begin{definition}%
  \label{qc-open}
  \begin{enumerate}[(a)]
  \item A proposition $P$ is \notion{(qc-)open}, if there merely are $f_1,\dots,f_n:R$,
    such that $P$ is equivalent to one of the $f_i$ being invertible.
  \item Let $X$ be a type.
    A subtype $U:X\to\Prop$ is \notion{(qc-)open}, if $U(x)$ is an open proposition for all $x:X$.
  \end{enumerate}
\end{definition}

\begin{proposition}[using \axiomref{loc}, \axiomref{sqc}]%
  \label{open-iff-negation-of-closed}
  A proposition $P$ is open
  if and only if
  it is the negation of some closed proposition
  (\Cref{closed-proposition}).
\end{proposition}

\begin{proof}
  Indeed, by \Cref{generalized-field-property},
  the proposition $\inv(f_1) \lor \dots \lor \inv(f_n)$
  is the negation of ${f_1 = 0} \land \dots \land {f_n = 0}$.
\end{proof}

\begin{proposition}[using \axiomref{loc}, \axiomref{sqc}]%
  \label{open-union-intersection}
  Let $X$ be a type.
  \begin{enumerate}[(a)]
  \item The empty subtype is open in $X$.
  \item $X$ is open in $X$.
  \item Finite intersections of open subtypes of $X$ are open subtypes of $X$.
  \item Finite unions of open subtypes of $X$ are open subtypes of $X$.
  \item Open subtypes are invariant under pointwise double-negation.
  \end{enumerate}
  Axioms are only needed for the last statement.
\end{proposition}

In \Cref{open-subscheme} we will see that open subtypes of open subtypes of a scheme are open in that scheme.
Which is equivalent to open propositions being closed under dependent sums.

\begin{proof}[of \Cref{open-union-intersection}]
  For unions, we can just append lists.
  For intersections, we note that invertibility of a product
  is equivalent to invertibility of both factors.
  Double-negation stability
  follows from \Cref{open-iff-negation-of-closed}.
\end{proof}

\begin{lemma}%
  \label{preimage-open}
  Let $f:X\to Y$ and $U:Y\to\Prop$ open,
  then the \notion{preimage} $U\circ f:X\to\Prop$ is open.
\end{lemma}

\begin{proof}
  If $U(y)$ is an open proposition for all $y : Y$,
  then $U(f(x))$ is an open proposition for all $x : X$.
\end{proof}

\begin{lemma}[using \axiomref{loc}, \axiomref{sqc}]%
  \label{open-inequality-subtype}
  Let $X$ be affine and $x:X$, then the proposition
  \[ x\neq y \]
  is open for all $y:X$.
\end{lemma}

\begin{proof}
  We show a proposition, so we can assume $\iota: X\to \A^n$ is a subtype.
  Then for $x,y:X$, $x\neq y$ is equivalent to $\iota(x)\neq\iota(y)$.
  But for $x,y:\A^n$, $x\neq y$ is the open proposition that $x-y\neq 0$.
\end{proof}

The intersection of all open neighborhoods of a point in an affine scheme,
is the formal neighborhood of the point.
We will see in \Cref{intersection-of-all-opens}, that this also holds for schemes.

\begin{lemma}[using \axiomref{loc}, \axiomref{sqc}]%
  \label{affine-intersection-of-all-opens}
  Let $X$ be affine and $x:X$, then the proposition
  \[ \prod_{U:X\to \Open}U(x)\to U(y) \]
  is equivalent to $\neg\neg (x=y)$.
\end{lemma}

\begin{proof}
  By \Cref{open-union-intersection}, $\neg\neg (x=y)$ implies $\prod_{U:X\to \Open}U(x)\to U(y)$.
  For the other implication,
  $\neg (x=y)$ is open by \Cref{open-inequality-subtype}, so we get a contradiction.
\end{proof}

We now show that our two definitions (\Cref{affine-open}, \Cref{qc-open})
of open subtypes of an affine scheme are equivalent.

\begin{theorem}[using \axiomref{loc}, \axiomref{sqc}, \axiomref{Z-choice}]%
  \label{qc-open-affine-open}
  Let $X=\Spec A$ and $U:X\to\Prop$ be an open subtype,
  then $U$ is affine open, i.e. there merely are $h_1,\dots,h_n:X\to R$ such that
  $U=D(h_1,\dots,h_n)$.
\end{theorem}

\begin{proof}
  Let $L(x)$ be the type of finite lists of elements of $R$,
  such that one of them being invertible is equivalent to $U(x)$.
  By assumption, we know
  \[\prod_{x:X}\propTrunc{L(x)}\rlap{.}\]
  So by \axiomref{Z-choice}, we have $s_i:\prod_{x:D(f_i)}L(x)$.
  We compose with the length function for lists to get functions $l_i:D(f_i)\to\N$.
  By \Cref{boundedness}, the $l_i$ are bounded.
  Since we are proving a proposition, we can assume we have actual bounds $b_i:\N$.
  So we get functions $\tilde{s_i}:D(f_i)\to R^{b_i}$,
  by append zeros to lists which are too short,
  i.e. $\widetilde{s}_i(x)$ is $s_i(x)$ with $b_i-l_i(x)$ zeros appended.

  Then one of the entries of $\widetilde{s}_i(x)$ being invertible,
  is still equivalent to $U(x)$.
  So if we define $g_{ij}(x)\colonequiv \pi_j(\widetilde{s}_i(x))$,
  we have functions on $D(f_i)$, such that
  \[
    D(g_{i1},\dots,g_{ib_i})=U\cap D(f_i)
    \rlap{.}
  \]
  By \Cref{affine-open-trans}, this is enough to solve the problem on all of $X$.
\end{proof}

This allows us to transfer one important lemma from affine-opens to qc-opens.
The subtlety of the following is that while it is clear that the intersection of two
qc-opens on a type, which are \emph{globally} defined is open again, it is not clear,
that the same holds, if one qc-open is only defined on the other.

\begin{lemma}[using \axiomref{loc}, \axiomref{sqc}, \axiomref{Z-choice}]%
  \label{qc-open-trans}
  Let $X$ be a scheme, $U\subseteq X$ qc-open in $X$ and $V\subseteq U$ qc-open in $U$,
  then $V$ is qc-open in $X$.
\end{lemma}

\begin{proof}
  Let $X_i=\Spec A_i$ be a finite affine cover of $X$.
  It is enough to show, that the restriction $V_i$ of $V$ to $X_i$ is qc-open.
  $U_i\colonequiv X_i\cap U$ is qc-open in $X_i$, since $X_i$ is qc-open.
  By \Cref{qc-open-affine-open}, $U_i$ is affine-open in $X_i$,
  so $U_i=D(f_1,\dots,f_n)$.
  $V_i\cap D(f_j)$ is affine-open in $D(f_j)$, so by \Cref{affine-open-trans},
  $V_i\cap D(f_j)$ is affine-open in $X_i$.
  This implies $V_i\cap D(f_j)$ is qc-open in $X_i$ and so is $V_i=\bigcup_{j}V_i\cap D(f_j)$.
\end{proof}

\begin{lemma}[using \axiomref{loc}, \axiomref{sqc}, \axiomref{Z-choice}]%
  \label{qc-open-sigma-closed}
  \begin{enumerate}[(a)]
  \item qc-open propositions are closed under dependent sums:
    if $P : \Open$ and $U : P \to \Open$,
    then the proposition $\sum_{x : P} U(x)$ is also open.
  \item Let $X$ be a type. Any open subtype of an open subtype of $X$ is an open subtype of $X$.
  \end{enumerate}
\end{lemma}

\begin{proof}
  \begin{enumerate}[(a)]
  \item Apply \Cref{qc-open-trans} to the point $\Spec R$.
  \item Apply the above pointwise.
  \end{enumerate}
\end{proof}

\begin{remark}
  \Cref{qc-open-sigma-closed} means that
  the (qc-) open propositions constitute a \notion{dominance}
  in the sense of~\cite{rosolini-phd-thesis}.
\end{remark}

The following fact about the interaction of closed and open propositions
is due to David Wärn.

\begin{lemma}%
  \label{implication-from-closed-to-open}
  Let $P$ and $Q$ be propositions
  with $P$ closed and $Q$ open.
  Then $P \to Q$ is equivalent to $\lnot P \lor Q$.
\end{lemma}

\begin{proof}
  We can assume $P = (f_1 = \dots = f_n = 0)$
  and $Q = (\inv(g_1) \lor \dots \lor \inv(g_m))$.
  Then we have:
  \begin{align*}
    (P \to Q) &= \qquad
    \text{\Cref{generalized-field-property} for $g_1, \dots, g_m$}\\
    (P \to \lnot (g_1 = \dots = g_m = 0)) &= \\
    \lnot (f_1 = \dots = f_n = g_1 = \dots = g_m = 0) &= \qquad
    \text{\Cref{generalized-field-property} for $f_1, \dots, f_n, g_1, \dots, g_m$}\\
    (\inv(f_1) \lor \dots \lor \inv(f_n) \lor \inv(g_1) \lor \dots \lor \inv(g_m) &= \qquad
    \text{\Cref{generalized-field-property} for $f_1, \dots, f_n$}\\
    \lnot P \lor Q &
  \end{align*}
\end{proof}



\pagebreak

\section{Projective space}
We follow the notations and setting for Synthetic Algebraic Geometry \cite{draft}.
In particular, $R$ denotes the generic local ring and $R^\times$ is the multiplicative group of units of $R$.

In Synthetic Algebraic Geometry, a scheme is defined as a set satisfying some property \cite{draft}. In particular
the projective space $\bP^n$ can be defined to be the quotient of $R^{n+1}\setminus\{0\}$ by the
equivalence relation $a\sim b$ which expresses that $a$ and $b$ are proportional, %i.e. $a_ib_j=a_jb_i$,
which is equal to $\Sigma_{r:R^\times}ar = b$. We can then prove \cite{draft}
that this set is a scheme. This definition goes back to \cite{Kock74}.

 In this setting, a map of schemes is simply an arbitrary set theoretic map. An application of this work is to show
 that the maps $\bP^n\rightarrow \bP^m$ are given by $m+1$ homogeneous polynomials of the same degree in $n+1$ variables.

\medskip


There is another definition of $\bP^n$ which uses ``higher'' notions. Let $\KR$ be the delooping
of $R^\times$. It can be defined as the type of lines $\Sigma_{M:\Mod{R}}\|{M=R^1}\|$. Over $\KR$ we have the
family of sets
$$T_n(l) = l^{n+1}\setminus\{0\}$$
Note that we use the same notation for an element $l : \KR$,
its underlying $R$-module and its underlying set.
An equivalent definition of $\bP^n$ is then
$$
\bP^n = \sum_{l:\KR}T_n(l)
$$
That is, we replaced the quotient operation, here a set of orbits for a free group action, by a sum type over the delooping of this group
\cite{Sym}.

The standard line bundles (or twisting sheaves) on $\bP^n$ can then be constructed as follows.
We define $l^{\vee}\colonequiv \Hom_{\Mod{R}}(l,R^1)$
and $l^{\otimes n}:\KR$ for $l:\KR$ by
$l^{\otimes 0} = R^1$,
$l^{\otimes (n+1)} = l^{\otimes n}\otimes l$ for $n \geqslant 0$
and $l^{\otimes (n-1)} = l^{\otimes n}\otimes l^{\vee}$ for $n \leqslant 0$.
Then we define the line bundle $O(d):\bP^n\rightarrow \KR$ by $O(d)(l,s) = l^{\otimes d}$.

\medskip

 Connected to this definition of $\bP^n$, we will prove some equalities in the following.
 To prove these equalities, we will make use of the following lemma, which holds in synthetic algebraic geometry:
 
\begin{lemma}\label{invariant-implies-homogenous}
  Let $n,d:\N$ and $\alpha:R^n\to R$ be a map such that
  \[\alpha(\lambda x)=\lambda^d\alpha(x)\]
  then $\alpha$ is a homogenous polynomial of degree $d$.
\end{lemma}

\begin{proof}
  By duality, any map $\alpha:R^n\to R$ is a polynomial.
  To see it is homogenous of degree $d$, let us first note that any $P:R[\lambda]$ with $P(\lambda)=\lambda^d P(1)$
  for all $\lambda:R^\times$ also satisfies this equation for all $\lambda : R$ and is therefore homogenous of degree $d$.
  Then for $\alpha'_x:R[\lambda]$ given by $\alpha'_x(\lambda)\colonequiv \alpha(\lambda\cdot x)$
  we have $\alpha'_x(\lambda)=\lambda^d \alpha'_x(1)$. This means any coeffiecent of $\alpha'_x$
  of degree different from $d$ is 0. Since this means every monomial appearing in $\alpha$,
  which is not of degree $d$, is zero for all $x$ and therefore 0.   
\end{proof}

\begin{proposition}\label{end}
  $$\prod_{l:\KR}l^n\rightarrow l \;\;\;=\;\;\; \Hom(R^n,R)$$
\end{proposition}

\begin{proof}
We rewrite $\Hom(R^n,R)$, the set of $R$-module morphism, as
$$
\sum_{\alpha:R^n\rightarrow R}\prod_{\lambda:R^\times}\prod_{x:R^n}\alpha(\lambda x) = \lambda \alpha(x)
$$
using \Cref{invariant-implies-homogenous} with $d=1$.

\medskip

It is then a general fact that if we have a pointed connected groupoid $(A,a)$ and a family of
sets $T(x)$ for $x:A$, then $\prod_{x:A}T(x)$ is the set of fixedpoints of $T(a)$ for the $(a=a)$ action
\cite{Sym}.
\end{proof}

We will use the following remark, proved in \cite{draft}[Remark 6.2.5].

\begin{lemma}\label{ext}
  Any map $R^{n+1}\setminus\{0\}\rightarrow R$ can be uniquely extended to a map $R^{n+1}\rightarrow R$ for $n>0$.
\end{lemma}

We will also use the following proposition, already noticed in \cite{draft}.

\begin{proposition}\label{const}
  Any map from $\bP^n$ to $R$ is constant.
\end{proposition}

\begin{proof}
  Since $\bP^n$ is a quotient of $R^{n+1}\setminus\{0\}$, the set $\bP^n\rightarrow R$ is
  the set of maps $\alpha:R^{n+1}\setminus\{0\}\rightarrow R$
  such that $\alpha(\lambda x) = \alpha(x)$ for all $\lambda$ in $R^\times$.
  These are exactly the constant maps
  using \Cref{ext} and \Cref{invariant-implies-homogenous} with $d=0$.
\end{proof}

\begin{proposition}\label{aut}
  For all $n:\N$ we have:
$$\prod_{l:\KR}T_n(l)\rightarrow T_n(l) \;\;=\;\; GL_{n+1}$$
\end{proposition}

\begin{proof}
  For $n=0$, this is the direct computation that a Laurent-polynomial $\alpha:(R[X,1/X])^\times$ which satisfies
  $\alpha(\lambda x)=\lambda \alpha(x)$ is $\lambda\alpha(1)$ where $\alpha(1):R^\times=\GL_1$.
  
  \medskip
  
  For $n>0$, the proposition follows from two remarks.

  The first remark is that maps $T_n(R)\to T_n(R)$, which are invariant under the induced $\KR$ action, are linear.
  To prove this remark, we first map from $T_n(l)\to T_n(l)$ to $T_n(l)\to l^{n+1}$ by composing with the inclusion.
  Maps of the latter kind can be uniquely extended to maps $l^{n+1}\to l^{n+1}$, since by 
  \Cref{ext} the restriction map
$$
(l^{n+1}\rightarrow l)\rightarrow ((l^{n+1}\setminus\{0\})\rightarrow l)
$$
is a bijection for $n>0$ and all $l:\KR$.

\medskip

The second remark is that a linear map $u:R^{m}\rightarrow R^{m}$ such that
$$
x\neq 0~\rightarrow~u(x)\neq 0
$$
is exactly an element of $GL_{m}$.

We show this by induction on $m$. For $m=1$ we have $u(1)\neq 0$ iff $u(1)$ invertible.

For $m>1$, we look at $u(e_1) = \Sigma \alpha_ie_i$ with $e_1,\dots,e_m$ basis of $R^m$.
We have that some $\alpha_j$ is invertible.
By composing $u$ with an element in $GL_m$, we can then
assume that $u(e_1) = e_1+v_1$ and $u(e_i) = v_i$, for $i>1$, with $v_1,\dots,v_m$ in $Re_2+\dots+Re_m$.
We can then conclude by induction.
\end{proof}

We can generalize \Cref{end}
and get a result related to \Cref{aut} as follows.
 
\begin{lemma}\label{hom}
  \begin{enumerate}[(i)]
    \item
      \[  \prod_{l:\KR}l^n\rightarrow l^{\otimes d} \;\;=\;\; (R[X_1, \dots, X_n])_d \]
      That is,
      every element of the left-hand side is given by
      a unique homogeneous polynomial of degree $d$ in $n$ variables.
    \item
      An element in
      $$\prod_{l:\KR}T_n(l)\rightarrow T_m(l^{\otimes d})$$
      is given by $m+1$ homogeneous polynomials $p = (p_0,\dots,p_m)$ of degree $d$ such that
      $x\neq 0$ implies $p(x)\neq 0$.
  \end{enumerate}
\end{lemma}

\begin{proof}
We show the first item. Following \cite{Sym} again, this product is the set of maps $\alpha:R^n\rightarrow R^{\otimes d}$
which are invariant by the $R^\times$-action which in this case acts by mapping $\alpha$ to $r^d\alpha(r^{-1} x)$ for each $r:R^\times$.
So by \Cref{invariant-implies-homogenous} these are exactly the maps given by homogeneous polynomials of degree $d$.
\end{proof}


\ignore{

  (Z-choice), (bound) => qc-open is affine-open
}


\ignore{
  external justification of zariski-local-choice

  def qc-scheme
    example: qc-open proposition

  discuss alternative definitions (of open and scheme)

projective space
  functions on P^n are constant
 
  line bundles?

cohomology

  explanation: structure sheaf - why R?

  def: cohomology as 0-trunc((x:X) -> K(A_x,n))

  is-connected(R-torsors on Spec A)

  cohomology of twisting sheaves on P^n?

}

\section{Bundles and cohomology}
In non-synthetic algebraic geometry,
the structure sheaf~$\mathcal{O}_X$ is part of the data constituting a scheme~$X$.
In our internal setting,
a scheme is just a type satisfying a property.
When we want to consider the structure sheaf as an object in its own right,
we can represent it by the trivial bundle
that assigns to every point $x : X$ the set $R$.
Indeed, for an affine scheme $X = \Spec A$,
taking the sections of this bundle over a basic open $D(f) \subseteq X$
\[ \left(\prod_{x : D(f)} R\right) = (D(f) \to R) = A[f^{-1}] \]
yields the localizations of the ring $A$
expected from the structure sheaf $\mathcal{O}_X$.
More generally,
instead of sheaves of abelian groups, $\mathcal{O}_X$-modules, etc.,
we will consider bundles of abelian groups, $R$-modules, etc.,
in the form of maps from $X$ to the respective type of algebraic structures.

\subsection{Quasi-coherent bundles}

Sometimes we want to ``apply'' a bundle to a subtype,
like sheaves can be evaluated on open subspaces
and introduce the common notation ``$M(U)$'' for that below.
It is, however, not justified to expect, that this application
and the corresponding theory of ``sheaves'' is ``the same'' as the external one,
since the definition below uses the internal hom ``$\prod$''
-- where the corresponding external construction, would be the set of continuous sections of a bundle.

\begin{definition}
  \index{$M(U)$}
  \label{application-of-bundle-to-subtype}
  Let $X$ be a type and $M:X\to \Mod{R}$ a dependent module.
  Let $U\subseteq X$ be any subtype.
  \begin{enumerate}[(a)]
  \item We write:
    \[
      M(U)\colonequiv \prod_{x:U}M_x
      \rlap{.}
    \]
  \item With pointwise structure, $U\to R$ is an $R$-algebra
    and $M(U)$ is a $(U\to R)$-module.
  \end{enumerate}
\end{definition}

Somewhat surprisingly, localization of modules $M(U)$
can be done pointwise:

\begin{lemma}[using \axiomref{loc}, \axiomref{sqc}, \axiomref{Z-choice}]%
  \label{module-bundle-localization-pointwise}
  Let $X$ be a scheme and $M:X\to \Mod{R}$ a dependent module.
  For any $f:X\to R$, there is an equality
  \[
    M(X)_f=\prod_{x:X}(M_x)_{f(x)}
  \]
  of $(X\to R)$-modules.
\end{lemma}

\begin{proof}
First we construct a map, by realizing that the following is well-defined:
\[
  \frac{m}{f^k}\mapsto\left(x\mapsto \frac{m(x)}{f(x)^k}\right)
\]
So let $\frac{m}{f^k}=\frac{m'}{f^{k'}}$,
i.e. let there be an $l:\N$ such that $f^l(mf^{k'}-m'f^k)=0$.
But then we can choose the same $l:\N$ for each $x:X$
and apply the equation to each $x:X$.

We will now show, that the map we defined is an embedding.
So let $g,h:M(X)_f$ such that $p:\prod_{x:X}g(x)=_{(M_x)_{f(x)}}h(x)$.
Let $m_g,m_h:\prod_{x:X} M_x$ and $k_g,k_h:\N$ such that
\[
  g=\frac{m_g}{f^{k_g}} \quad\text{and}\quad h=\frac{m_h}{f^{k_h}}
  \rlap{.}
\]
From $p$ we know $\prod_{x:X}\exists_{k_x:\N}f(x)^{k_x}(m_g(x)f(x)^{k_h}-m_h(x)f(x)^{k_g})=0$.
By \Cref{strengthened-boundedness},
we find one $k : \N$ with
\[
  \prod_{x:X}f(x)^{k}(m_g(x)f(x)^{k_h}-m_h(x)f(x)^{k_g})=0
\]
--- which shows $g=h$.

It remains to show that the map is surjective.
So let $\varphi:\prod_{x:X}(M_x)_{f(x)}$ and
note that
\[
  \prod_{x:X}
  \exists_{k_x:\N,m_x:M_x}.
  \varphi(x)=\frac{m_x}{f(x)^{k_x}}
  \rlap{.}
\]
By \Cref{strengthened-boundedness} and \Cref{zariski-choice-scheme},
we get $k:\N$, an affine open cover $U_1,\dots,U_n$ of $X$ and $m_i:(x : U_i)\to M_x$
such that for each $i$ and $x:U_i$ we have
\[
  \varphi(x)=\frac{m_i(x)}{f(x)^{k}}
  \rlap{.}
\]
The problem is now to construct a global $m:(x:X)\to M_x$ from the $m_i$.
We have
\[
    \prod_{x:U_{ij}}\frac{m_i(x)}{f(x)^k}=\varphi(x)=\frac{m_j(x)}{f(x)^k}
\]
meaning there is pointwise an exponent $t_x:\N$,
such that $f(x)^{t_x}m_i(x)=f(x)^{t_x}m_j(x)$.
By \Cref{strengthened-boundedness},
we can find a single $t:\N$ with this property and define
\[
  \tilde{m}_i(x) \colonequiv f(x)^t m_i(x)
  \rlap{.}
\]
Then we have $\tilde{m}_i(x)=\tilde{m}_j(x)$ on all intersections $U_{ij}$,
which is what we need to get a global $m:(x:X)\to M_x$ from \Cref{kraus-glueing}.
Since $\varphi(x)=\frac{f(x)^t m_i(x)}{f(x)^{t+k}}=\frac{\tilde{m}_i(x)}{f(x)^{t+k}}$
for all $i$ and $x : U_i$,
we have found a preimage of $\varphi$ in $M(X)_f$.
\end{proof}

We will need the following algebraic observation:

\begin{remark}%
  \label{localization-to-module-if-non-zero}
  Let $M$ be an $R$-module and $A$ a finitely presented $R$-algebra,
  then there is an $R$-linear map
  \[
    M\otimes A\to M^{\Spec A}
  \]
  induced by mapping $m\otimes f$ to $x\mapsto x(f)\cdot m$.
  In particular, for any $f:R$, there is a
  \[
    M_f\to M^{D(f)}
    \rlap{.}
  \]
  The map $M\otimes A\to M^{\Spec A}$ is natural in $M$.
\end{remark}

\begin{lemma}[using \axiomref{loc}, \axiomref{sqc}, \axiomref{Z-choice}]%
  \label{localization-to-restriction}                    
  Let $X$ be a scheme, $M:X\to\Mod{R}$, $U\subseteq X$ open and $f:A$.
  Then there is an $R$-linear map
  \[
    M(U)_f \to M(D(f)) 
    \rlap{.}
  \]
\end{lemma}

\begin{proof}
  Combining \Cref{module-bundle-localization-pointwise}
  and pointwise application of \Cref{localization-to-module-if-non-zero} we get
  \[
    M(U)_f=\left(\prod_{x:U}(M_x)_{f(x)}\right)\to \left(\prod_{x:U}(M_x)^{D(f(x))}\right)
    =\left(\prod_{x:D(f)}M_x\right)
    =M(D(f))
  \]
\end{proof}

A characterization of quasi coherent sheaves in the little Zariski-topos was found with \cite{ingo-thesis}[Theorem 8.3].
This characterization is similar to our following definition of weak quasi-coherence,
which will provide us with an abelian subcategory of the $R$-module bundles over a scheme,
where we can show that higher cohomology vanishes if the scheme is affine.

\begin{definition}
  \label{weakly-quasi-coherent-module}
  An $R$-module $M$ is \notion{weakly quasi-coherent},
  if for all $f:R$, the canonical homomorphism
  \[
    M_f\to M^{D(f)}
  \]
  from \Cref{localization-to-module-if-non-zero} is an equivalence.
  We denote the type of weakly quasi-coherent $R$-modules
  with $\Mod{R}_{wqc}$\index{$\Mod{R}_{wqc}$}.
\end{definition}

\begin{lemma}
  \label{kernel-wqc}
  For any $R$-linear map $f:M\to N$ of weakly quasi-coherent modules $M$ and $N$,
  the kernel of $f$ is weakly quasi-coherent.
\end{lemma}

\begin{proof}
  Let $K\to M$ be the kernel of $f$.
  For any $f:R$, the map $K^{D(f)}\to M^{D(f)}$ is the kernel of $M^{D(f)}\to N^{D(f)}$.
  The latter map is equal to $M_f\to N_f$ by weak quasi-coherence of $M$ and $N$
  and $K_f\to M_f$ is the kernel of $M_f\to N_f$.
  Let the vertical maps in
  \begin{center}
    \begin{tikzcd}
      K_f\ar[r]\ar[d] & M_f\ar[r]\ar[d,"\simeq"] & N_f\ar[d,"\simeq"] \\
      K^{D(f)}\ar[r] & M^{D(f)}\ar[r] & N^{D(f)}
    \end{tikzcd}
  \end{center}
  be the canonical maps from \Cref{localization-to-module-if-non-zero}.
  The squares commute because of the naturality of the vertical maps.
  Then the map $K_f\to K^{D(f)}$ is an isomorphism,
  because by commutativity, it is equal to the induced map between the kernels $K_f$ and $K^{D(f)}$,
  which has to be an isomorphism, since it is induced by an isomorphism of diagrams.
\end{proof}

\begin{definition}%
  \label{weakly-quasi-coherent-bundle}
  Let $X$ be a scheme.
  A weakly quasi-coherent bundle on $X$, is a map $M:X\to \Mod{R}_{wqc}$.
\end{definition}

An immediate consequence is, that
weakly quasi coherent dependent modules have
the property that ``restricting is the same as localizing'':

\begin{lemma}[using \axiomref{loc}, \axiomref{sqc}, \axiomref{Z-choice}]
  \label{weakly-quasi-coherent-open-localization}
  Let $X$ be a scheme and $M:X\to \Mod{R}$ weakly quasi-coherent,
  then for all open $U\subseteq X$ and $f:U\to R$
  the canonical morphism
  \[
    M(U)_f\to M(D(f))
  \]
  is an equivalence.
\end{lemma}

\begin{proof}
  By construction of the canonical map from \Cref{localization-to-restriction}.
\end{proof}

Let us look at an example.

\begin{proposition}%
  \label{fp-algebra-bundle-is-quasi-coherent}
  Let $X$ be a scheme and $C:X\to \Alg{R}_{fp}$.
  Then $C$, as a bundle of $R$-modules, is weakly quasi coherent.
\end{proposition}

\begin{proof}
  Then for any $f:R$ and $x:X$, using \Cref{algebra-valued-functions-on-affine}, we have
  \[
    (C_x)_f=C_x\otimes_R R_f=(\Spec R_f \to C_x)=(D(f)\to C_x)={C_x}^{D(f)}
    \rlap{.}
  \]
\end{proof}

For examples of non weakly quasicoherent modules,
see \Cref{non-wqc-module-family}
and \Cref{RN-non-wqc}.

\begin{lemma}[using \axiomref{loc}, \axiomref{sqc}, \axiomref{Z-choice}]%
  \label{weakly-quasi-coherent-pi}
  Let $X$ be an affine scheme and $M_x$ a weakly quasi-coherent $R$-module for any $x:X$,
  then
  \[
    \prod_{x:X}M_x
  \]
  is a weakly quasi-coherent $R$-module.
\end{lemma}

\begin{proof}
  We need to show:
  \[
    \left(\prod_{x:X}M_x\right)_f=\left(\prod_{x:X}M_x\right)^{D(f)}
  \]
  for all $f:R$.
  By weak \Cref{module-bundle-localization-pointwise}, quasi-coherence
  and \Cref{weakly-quasi-coherent-open-localization}
  we know:
  \[
    \left(\prod_{x:X}M_x\right)_f
    =\prod_{x:X}\left(M_x\right)_{f(x)}
    =\prod_{x:X}\left(M_x\right)^{D(f)}
    =\left(\prod_{x:X}M_x\right)^{D(f)}
    \rlap{.}
  \]
\end{proof}

Quasi-coherent dependent modules turn out to have very good properties,
which are to be expected from what is known about their external counterparts.
We will show below, that quasi coherence is preserved by the following constructions:

\begin{definition}
  \label{pullback-push-forward}
  Let $X,Y$ be types and $f:X\to Y$ be a map.
  \begin{enumerate}[(a)]
  \item \index{$f^*M$} For any dependent module $N:Y\to\Mod{R}$,
    the \notion{pullback} or \notion{inverse image} is the dependent module
    \[
      f^*N\colonequiv (x:X) \mapsto M_{f(x)}\rlap{.}
    \]
  \item \index{$f_*M$} For any dependent module $M:X\to\Mod{R}$,
    the \notion{push-forward} or \notion{direct image} is the dependent module
    \[
      f_*M\colonequiv (y:Y) \mapsto \prod_{x:\fib_f(y)}M_{\pi_1(x)}\rlap{.}
    \]
  \end{enumerate}
\end{definition}

\begin{theorem}[using \axiomref{loc}, \axiomref{sqc}, \axiomref{Z-choice}]%
  \label{pullback-push-forward-qcoh}
  Let $X,Y$ be schemes and $f:X\to Y$ be a map.
  \begin{enumerate}[(a)]
  \item For any weakly quasi-coherent dependent module $N:Y\to\Mod{R}$,
    the inverse image $f^*N$ is weakly quasi-coherent.
  \item For any weakly quasi-coherent dependent module $M:X\to\Mod{R}$,
    the direct image $f_*M$ is weakly quasi-coherent.
  \end{enumerate}
\end{theorem}

\begin{proof}
  \begin{enumerate}[(a)]
  \item There is nothing to do, when we use the pointwise definition of weak quasi-coherence. 
  \item We need to show, that
    \[
      \prod_{x:\fib_f(y)}M_{\pi_1(x)}
    \]
    is a weakly quasi-coherent $R$-module.
    By \Cref{fiber-product-scheme},
    the type $\fib_f(y)$ is a scheme.
    So by \Cref{weakly-quasi-coherent-pi},
    the module in question is weakly quasi-coherent.
  \end{enumerate}
\end{proof}

With a non-cyclic forward reference to a cohomological result,
there is a short proof of the following:

\begin{proposition}[using \axiomref{loc}, \axiomref{sqc}, \axiomref{Z-choice}]%
  Let $f:M\to N$ be an $R$-linear map of weakly quasi-coherent $R$-modules $M$ and $N$,
  then the cokernel $N/M$ is weakly quasi-coherent.
\end{proposition}

\begin{proof}
  We will first show, that for an $R$-linear embedding $m:M\to N$
  of weakly quasi-coherent $R$-modules $M$ and $N$,
  the cokernel $N/M$ is weakly quasi-coherent.
  We need to show:
  \[
    (N/M)_f=(N/M)^{D(f)}.
  \]
  By algebra: $(N/M)_f=N_f/M_f$.
  This means we are done, if $(N/M)^{D(f)}=N^{D(f)}/{M^{D(f)}}$.
  To see this holds, let us consider $0\to M\to N\to N/M\to 0$ as a short exact sequence of dependent modules,
  over the subtype of the point $D(f)\subseteq 1=\Spec R$.
  Then, taking global sections, by \Cref{cohomology-les},
  we have an exact sequence
  \[
    0\to M^{D(f)}\to N^{D(f)}\to (N/M)^{D(f)}\to H^1(D(f),M)
  \]
  -- but $D(f)=\Spec R_f$ is affine,
  so the last term is 0 by \Cref{H1-wqc-module-affine-trivial}
  and $(N/M)^{D(f)}$ is the cokernel $N^{D(f)}/M^{D(f)}$.

  Now we will show the statement for a general $R$-linear map $f:M\to N$.
  By algebra, the cokernel of $f$ is the same as the cokernel of the induced map
  $M/K\to N$, where $K$ is the kernel of $f$.
  By \Cref{kernel-wqc}, $K$ is weakly quasi-coherent, so by the proof above,
  $M/K$ is weakly quasi-coherent.
  $M/K\to N$ is an embedding, so again by the proof above, its cokernel is weakly quasi-coherent.
\end{proof}

\subsection{Finitely presented bundles}

We now investigate the relationship between bundles of $R$-modules on $X = \Spec A$
and $A$-modules.

\begin{proposition}
  Let $A$ be a finitely presented $R$-algebra.
  There is an adjunction
  \[ \begin{tikzcd}[row sep=tiny]
    M \ar[r, mapsto] & {(M \otimes x)}_{x : \Spec A} \\
    \Mod{A} \ar[r, shift left=2] \ar[r, phantom, "\rotatebox{90}{$\vdash$}"] &
    \Mod{R}^{\Spec A} \ar[l, shift left=2] \\
    \prod_{x : \Spec A} N_x & N \ar[l, mapsto]
  \end{tikzcd} \]
  between the category of $A$-modules
  and the category of bundles of $R$-modules on $\Spec A$.
\end{proposition}

For an $A$-module $M$,
the unit of the adjunction is:
\begin{align*}
  \eta_M : M &\to \prod_{x : \Spec A} (M \otimes x) \\
  m &\mapsto (m \otimes 1)_{x : \Spec A}
\end{align*}

\begin{example}[using \axiomref{sqc}, \axiomref{loc}]
  It is not the case that
  for every finitely presented $R$-algebra $A$
  and every $A$-module $M$
  the map $\eta_M$ is injective.
\end{example}

\begin{proof}
  \cite{topology-draft}.
\end{proof}

\begin{theorem}%
  \label{fp-module}
  Let $X=\Spec(A)$ be affine and
  let a bundle of finitely presented $R$-modules $M : X\to \fpMod{R}$ be given.
  Then the $A$-module
  \[ \tilde{M}\coloneqq\prod_{x:X}M_x \]
  is finitely presented and for any $x:X$ the $R$-module $\tilde{M}\otimes_A R$ is $M_x$.
  Under this correspondence, localizing $\tilde{M}$ at $f:A$ corresponds to restricting $M$ to $D(f)$.
\end{theorem}

\subsection{Cohomology on affine schemes}

\begin{definition}%
  \label{torsor}
  Let $X$ be a type and $A:X\to \AbGroup$ a map to the type of abelian groups.
  For $x:X$ let $T_x$ be a set with an $A_x$ action.
  \begin{enumerate}[(a)]
  \item $T$ is an \notion{$A$-pseudotorsor}, if the action is free and transitive for all $x:X$.
  \item $T$ is an \notion{$A$-torsor}, if it is an $A$-pseudotorsor and
    \[ \prod_{x:X} \propTrunc{ T_x } \rlap{.}\]
  \item We write $\Tors{A}(X)$ for the type of $A$-torsors on $X$.
  \end{enumerate}
\end{definition}

Torsors on a point are a concrete implementation of first deloopings:

\begin{definition}
  \label{delooping}
  Let $n:\N$.
  A $n$-th \notion{delooping}\index{$K(A,n)$} of an abelian group $A$,
  is a pointed, $(n-1)$-connected, $n$-truncated type $K(A,n)$,
  such that $\Omega^nK(A,n)=_{\AbGroup}A$.
\end{definition}

For any abelian group and any $n$, a delooping $K(A,n)$ exists by \cite{licata-finster}.
Deloopings can be used to represent cohomology groups by mapping spaces.
This is usually done in homotopy type theory to study higher inductive types, such as spheres and CW-complexes,
but the same approach works for internally representing sheaf cohomology,
which is the intent of the following definition:

\begin{definition}
  \label{cohomology}
  Let $X$ be a type and $\mathcal F:X\to\AbGroup$ a dependent abelian group.
  The $n$-th cohomology group of $X$ with coefficients in $\mathcal F$ is
  \[
    H^n(X,\mathcal F)\colonequiv \left\propTrunc{\prod_{x:X}K(\mathcal F,n)\right}_0\rlap{.}
  \]
\end{definition}

\begin{theorem}%
  \label{cohomology-les}
  Let $\mathcal F,\mathcal G,\mathcal H:X\to \AbGroup$ be such that for all $x:X$,
  \[
    0\to \mathcal F_x\to\mathcal G_x\to\mathcal H_x\to 0
  \]
  is an exact sequence of abelian groups. Then there is a long exact sequence:
  \begin{center}
    \begin{tikzcd}
      & \dots\ar[r] & H^{n-1}(X,\mathcal H)\ar[dll] \\
      H^n(X,\mathcal F)\ar[r] & H^n(X,\mathcal G)\ar[r] & H^n(X,\mathcal H)\ar[dll] \\
      H^{n+1}(X,\mathcal F)\ar[r] & \dots &
    \end{tikzcd}
  \end{center}
\end{theorem}

\begin{proof}
  By applying the long exact homotopy fiber sequence.
\end{proof}

The following is an explicit formulation of the fact, that the Čech-Complex for an
$\mathcal{O}_X$-module sheaf on $X=\Spec(A)$ given by an $A$-module $M$ is exact in degree 1.
\begin{lemma}%
  \label{H1-algebra}
  Let $M$ be a module over a commutative ring $A$, $F_1,\dots,F_l$ a coprime system on $A$
  and for $i,j\in\{1,\dots,l\}$, let $s_{ij} : F_i^{-1} F_j^{-1} M$ such that:
  \[ s_{jk}-s_{ik}+s_{ij}=0 \rlap{.}\]
  Then there are $u_i:F_i^{-1}M$ such that $s_{ij}=u_j - u_i$.
\end{lemma}

\begin{proof}
  Let $s_{ij}=\frac{m_{ij}}{f_i f_j}$ with $m_{ij}:M$, $f_i:F_i$ and $f_j:F_j$ such that:
  \[ f_i\cdot m_{jk}-f_j\cdot m_{ik}+f_k\cdot m_{ij}=0 \rlap{.}\]
  Let $r_i$ such that $\sum r_i f_i =1$.
  Then for
  \[ u_i \coloneqq -\sum_{k=1}^l\frac{r_k}{f_i}m_{ik} \]
  we have:
  \begin{align*}
      u_j-u_i &= -\sum_{k=1}^l\frac{r_k}{f_j}m_{jk} + \sum_{k=1}^l\frac{r_k}{f_i}m_{ik} \\
              &= -\sum_{k=1}^l\frac{r_k}{f_j f_i}f_i m_{jk} + \sum_{k=1}^l\frac{r_k}{f_i f_j} f_j m_{ik} \\
              &= \sum_{k=1}^l\frac{r_k}{f_j f_i}(-f_i m_{jk} + f_j m_{ik}) \\
              &= \sum_{k=1}^l\frac{r_k}{f_j f_i}f_k m_{ij} \\
              &= \frac{m_{ij}}{f_i f_j}
  \end{align*}
  \ %
\end{proof}

\begin{theorem}[using \axiomref{loc}, \axiomref{sqc}, \axiomref{Z-choice}]%
  \label{H1-wqc-module-affine-trivial}
  For any affine scheme $X=\Spec(A)$ and coefficients $M: X\to \Mod{R}_{wqc}$, we have
  \[ H^1(X,M)=0 \rlap{.} \]
\end{theorem}

\begin{proof}
  We need to show, that any $M$-torsor $T$ on $X$ is merely equal to the trivial torsor $M$,
  or equivalently show the existence of a section of $T$.
  We have
  \[ \prod_{x:X}\propTrunc{ T_x }\]
  and therefore, by (\axiomref{Z-choice}),
  there merely are $f_1,\dots,f_l:A$,
  such that the $U_i\coloneqq \Spec(A_{f_i})$ cover $X$ and
  there are local sections
  \[ s_i:\prod_{x:U_i}T_x\]
  of $T$. Our goal is to construct a matching family from the $s_i$.
  On intersections, let $t_{ij}\coloneqq s_i-s_j$ be the difference, so $t_{ij}:(x : U_i\cap U_j) \to M_x$.
  By \Cref{weakly-quasi-coherent-open-localization} equivalently,
  we have $t_{ij}:M(U_{i}\cap U_j)_{f_i f_j}$.
  Since the $t_{ij}$ were defined as differences,
  the condition in \Cref{H1-algebra} is satisfied and we get
  $u_i:M(U_i)_{f_i}$, such that $t_{ij}=u_i-u_j$.
  So we merely have a matching family $\tilde{s}_i\coloneqq s_i-u_i$ and therefore, using Lemma \ref{kraus-glueing} merely a section of $T$.
\end{proof}

A similar result is provable for $H^2(X,M)$ using the same approach.
There is an extension of this result to general $n$ in work in progress \cite{chech-draft}.

\subsection{Čech-Cohomology}

In this section, let $X$ be a type, $U_1,\dots,U_n\subseteq X$ open subtypes that cover $X$
and $\mathcal F:X\to \AbGroup$ a dependent abelian group on $X$.
We start by repeating the classical definition of \v{C}hech-Cohomology groups for a given cover.

\begin{definition}%
  \label{chech-complex}
  \begin{enumerate}[(a)]
  \item \index{$\mathcal F(U)$} For open $U\subseteq X$, we use the notation from \Cref{application-of-bundle-to-subtype}:
    \[
      \mathcal F(U)\colonequiv \prod_{x:U}\mathcal F_x\rlap{.}
    \]
  \item For $s:\mathcal F(U)$ and open $V\subseteq U$ we use the notation $s\colonequiv s_{|V} \colonequiv (x:V)\mapsto s_x$.
  \item \index{$U_{i_1\dots i_l}$}For a selection of indices $i_1,...,i_l:\{1,\dots,n\}$, we use the notation
    \[
      U_{i_1\dots i_l}\colonequiv U_{i_1}\cap\dots\cap U_{i_l}\rlap{.}
    \]
  \item For a list of indices $i_1,\dots,i_l$, let $i_1,\dots,\hat{i_t},\dots,i_l$ be the same list with the $t$-th element removed.
  \item For $k:\Z$, the $k$-th \notion{Čech-boundary operator}\index{$\partial^k$} is the homomorphism
    \[
      \partial^k:\bigoplus_{i_0,\dots,i_k}\mathcal F(U_{i_0\dots i_k})\to \bigoplus_{i_0,\dots,i_{k+1}}\mathcal F(U_{i_0\dots i_{k+1}})
    \]
    given by $\partial^k(s)\colonequiv (l_0,\dots,l_{k+1}) \mapsto \sum_{j=0}^k (-1)^j s_{l_0,\dots,\hat{l_j},\dots,l_k|U_{l_0,\dots,l_{k+1}}}$.
  \item The $k$-th \notion{Čech-Cohomology group} for the cover $U_1,\dots,U_n$ with coefficients in $\mathcal F$ is
    \[
      \check{H}^k(\{U\},\mathcal F)\colonequiv \ker\partial^{k} / \im(\partial^{k-1})\rlap{.}
    \]
  \end{enumerate}
\end{definition}

It is possible to construct a torsor from a \v{C}ech cocycle:

\begin{lemma}%
  \label{deligne-construction}
  Let $A$ be an abelian group and $L$ a type with $\propTrunc{L}$.
  Let us call $c:(i,j:L)\to A$ a $L$-cocycle, if $c_{ij}+c_{jk}=c_{ik}$ for all $i,j,k:L$.
  Then there is a bijection:
  \[
    \left((T:\text{$A$-torsor})\times T^L\right) \to \text{$L$-cocycles}
    \rlap{.}
  \]
\end{lemma}

\begin{proof}
  Let us first check, that the left side is a set.
  Let $(T,u),(T',u'):(T:\text{$A$-torsor})\times T^L$,
  then $(T,u)=(T',u')$ is equivalent to $(e:T\cong T')\times ((i:L)\to e(u_i)=u'_i)$.
  But two maps $e$ with this property are equal,
  since a map between torsors is determined by the image of a single element and $L$ is inhabited.
  
  Assume now $(T,u):(T:\text{$A$-torsor})\times T^L$ to construct the map.
  Then $c_{ij}\colonequiv u_i-u_j$ defines an $L$-cocycle,
  because
  \[
    u_i-u_j + u_j-u_k = u_i-u_k
    \rlap{.}
  \]
  This defines an embedding: Assume $(T,u)$ and $(T',u')$ define the same $L$-cocycle,
  then $u_i-u_j=u'_i-u'_j$ for all $i,j:L$.
  We want to show a proposition, so we can assume there is $i:L$ and use that to get a map $e:T\to T'$
  that sends $u_i$ to $u'_i$.
  But then we also have
  \[
    e(u_j)=e(u_j-u_i+u_i)=e(u'_j-u'_i+u_i)=u'_j-u'_i+e(u_i)=u'_j-u'_i+u'_i=u'_j
  \]
  for all $j:L$, which means $(T,u)=(T',u')$.
    
  Now let $c$ be an $L$-cocycle.
  Following \cite{Deligne91}[Section 5.2], we can define a preimage-candidate:
  \[
    T_c\colonequiv \{u:A^L\mid u_i-u_j=c_{ij}\}
    \rlap{.}
  \]
  $A$ acts on $T_c$ pointwise, since $(a+u_i)-(a+u_j)=u_i-u_j=c_{ij}$ for all $a:A$.
  
  To show that $T_c$ is inhabited,
  we may assume $i_0:L$.
  Then we define $u_i\colonequiv -c_{i_0i}$ to get $u_i-u_j=-c_{i_0i}+c_{i_0j}=c_{ij}$.

  Now $c$ is of type $(A^L)^L=A^{L\times L}$, so we have an element of the left hand side.
  Applying the map constructed above yields a cocycle
  \[
    \tilde{c}_{ij}=(k\mapsto c_{ki})-(k\mapsto c_{kj})=(k\mapsto c_{ki}-c_{kj})=(k\mapsto c_{kj}+c_{ji}-c_{kj})=(k\mapsto c_{ji})
  \]
  -- so $(T_c,c)$ is a preimage of $c_{ij}$.
\end{proof}

\begin{definition}
  The cover $U_1,\dots,U_n$ is called \notion{r-acyclic} for $\mathcal F$,
  if we have the following triviality of higher (non Čech) cohomology groups:
  \[
    \forall l, r\geq l>0\ \forall i_0,\dots,i_{r-l}. H^l(U_{i_0,\dots,i_{r-l}},\mathcal F)=0\rlap{.}
  \]
\end{definition}

\begin{example}
  If $X$ is a scheme, $U_1,\dots,U_n$ a cover by affine open subtypes
  and $\mathcal F$ pointwise a weakly quasi coherent $R$-module,
  then $U_1,\dots,U_n$ is 1-acyclic for $\mathcal F$ by \Cref{H1-wqc-module-affine-trivial}.
\end{example}

\begin{theorem}[using \axiomref{Z-choice}]%
  If $U_1,\dots,U_n$ is a 1-acyclic cover for $\mathcal F$, then
  \[
    \check{H}^1(\{U\},\mathcal F)=H^1(X,\mathcal F)\rlap{.}
  \]
\end{theorem}

\begin{proof}
  Let $\pi$ be the projection map
  \[
    \pi :
    \left(
      \sum_{T:\Tors{\mathcal F}(X)}\prod_{i}\prod_{x:U_i}T_x
    \right)
    \to \Tors{\mathcal F}(X)\rlap{.}
  \]
  Let us abbreviate the left hand side with $T(\mathcal F,U)$.
  Since the cover is 1-acyclic, $\pi$ is surjective.
  With $L_x\colonequiv \sum_{i}U_i(x)$ and \Cref{deligne-construction} we get:
  \begin{align*}
    T(\mathcal F,U)&=\prod_{x:X}(T_x:\Tors{\mathcal F_x})\times T_x^{L_x} \\
                   &=\prod_{x:X}\text{$L_x$-cocycles}
                     \rlap{.}
  \end{align*}
  The latter is the type of \v{C}ech-1-cocycles (\Cref{chech-complex} (e))
  and in total the equality is given by the isomorphism
  \[
    (T,t) \mapsto (i,j\mapsto t_i - t_j) :
    T(\mathcal F,U)
    \to
    \ker(\partial^1)
    \subseteq
    \bigoplus_{i,j}\mathcal F(U_{ij})\rlap{.}
  \]

  Realizing, that $\im(\partial^0)$ corresponds to the subtype of $T(\mathcal F,U)$ of trivial torsors,
  we arrive at the following diagram:
  \begin{center}
    \begin{tikzcd}
      & \Tors{\mathcal F}(X)\ar[r,->>] & H^1(X,\mathcal F) \\
      \sum_{T:T(\mathcal F,U)}\propTrunc{\pi_1(T)=\mathcal F}\ar[r,hook] & T(\mathcal F,U)\ar[u,->>]\ar[d,equal] & \\
      \im{\partial^0}\ar[r,hook]\ar[u,equal] & \ker{\partial^1}\ar[r,->>] & \check{H}^1(\{U\},\mathcal F)
    \end{tikzcd}
  \end{center}
  The composed map $T(\mathcal F,U)\to H^1(X,\mathcal F)$ is a homomorphism
  and therefore by \Cref{surjective-abgroup-hom-is-cokernel} a cokernel.
  So the two cohomology groups are equal, since they are cokernels of the same diagram.
\end{proof}

It is possible to pass from torsors to gerbes,
which are the degree 2 analogue of torsors:

\begin{definition}
  \label{gerbe}
  Let $A:\AbGroup$ be an abelian group.
  An \notion{$A$-banded gerbe}, is a connected type $\mathcal G:\mathcal U$,
  together with, for all $y:\mathcal G$ an identification of groups $\Omega (\mathcal G,y)=A$.
\end{definition}

Analogous to the type of $A$-torsors, the type of $A$-banded gerbes is a second delooping of an abelian group $A$.
We can formulate a second degree version of \Cref{deligne-construction}:

\begin{theorem}
  \label{deligne-construction-gerbes}
  Let $A$ be an abelian group and $L$ a type with $\propTrunc{L}$.
  Let us call $c:(i,j,k : L)\to A$ a $L$-2-cocycle,
  if $c_{jkl}-c_{ikl}+c_{ijl}-c_{ijk}=0$ for all $i,j,k,l : L$.
  Then there is a bijection:
  \[
    \left((\mathcal G:\text{$A$-gerbe})\times (u:\mathcal G^{L})\times (i,j : L) \to u_i=u_j\right) \to \text{$L$-2-cocycle}
    \rlap{.}
  \]
\end{theorem}

This is provable, again, by translating Deligne's argument \cite{Deligne91}[Section 5.3].
Using this, the correspondence of Eilenberg-MacLane-Cohomology and \v{C}ech-Cohomology can be extended in the following way:

\begin{theorem}
  If $U_1,\dots,U_n$ is a 2-acyclic cover for $\mathcal F$, then
  \[
    \check{H}^2(\{U\},\mathcal F)=H^2(X,\mathcal F)\rlap{.}
  \]  
\end{theorem}

However, with this approach, we need versions of \Cref{kraus-glueing}, with increasing truncation level.
While this suggests, we can prove the correspondence for any cohomology group of \emph{external} degree $l$,
there is follow-up work in progress \cite{chech-draft},
which proves the correspondence for all \emph{internal} $l:\N$.
A stronger assumption is needed in that work, though: The cover $U_1,\dots,U_n$ needs to be acyclic in the sense,
that all higher cohomology groups of all intersections $U_{i_0}\cap \dots \cap U_{i_k}$, with $k:\N$, vanish.
In the same draft, there is also a version of the vanishing result for all internal $l$,
and it is shown that separated schemes, like projective space, have an acyclic cover.
This means that many of the usual, essential computations with \v{C}ech-Cohomology can be transferred to synthetic algebraic geometry.


\printindex

\end{document}
