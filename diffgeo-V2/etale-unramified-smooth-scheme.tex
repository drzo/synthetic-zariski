


\subsection{Formally unramified schemes}

\begin{lemma}\label{unramified-affine-characterisation}
Let $X$ be an affine scheme, the following are equivalent:
\begin{enumerate}[(i)]
\item $X$ is formally unramified.
\item Identity types in $X$ are decidable.
\item For all $x:X$, we have that $T_x(X)=0$.
\end{enumerate}
\end{lemma}

\begin{proof}
(i) implies (ii). By \cref{closed-and-etale-decidable}.

(ii) implies (i). Because decidable implies formally étale.

(ii) implies (iii). Assume given $x:X$ with $t:T_x(X)$, then for all $\epsilon:\D(1)$ we have $\neg\neg(\epsilon = 0)$ so that we have $\neg\neg t(\epsilon) = t(0)$ which implies $t(\epsilon) = t(0)$ since equality is assumed decidable. Therefore $t = 0$ in $T_x(X)$.

(iii) implies (ii). Indeed given $\epsilon:R$ such that $\epsilon^2=0$, assume $x,y:X$ such that $\epsilon=0 \to x=y$. Then $x\in N_1(y)$ and by \cref{duality-infinitesimal-tangent} and $T_y(X)=0$ we conclude $x=y$.
\end{proof}

\begin{lemma}
Let $X$ with $(U_i)_{i:I}$ a finite open cover of $X$. Then $X$ is formally unramified if and only if all the $U_i$ are formally unramified.
\end{lemma}

\begin{proof}
Any subtype of a formally unramified type is formally unramified by \cref{prop-are-unramified}.

Conversely, assume $X$ with such a cover, for all $x,y:X$ we there exists $i:I$ such that $x\in U_i$ and then:
\[x=_Xy \leftrightarrow \sum_{y\in U_i} x=_{U_i}y \]
which in formally étale because open propositions are formally étale by \cref{not-not-stable-prop-etale}.
\end{proof}

\begin{corollary}
Let $X$ be a scheme, the following are equivalent:
\begin{enumerate}[(i)]
\item $X$ is formally unramified.
\item Identity types in $X$ are open.
\item For all $x:X$, we have that $T_x(X)=0$.
\end{enumerate}
\end{corollary}

\begin{proof}
TODO
\end{proof}


\subsection{Formally unramified maps schemes}

Now we generalise this to maps between schemes.

\begin{proposition}
A map between schemes is unramified if and only if its differentials are injective. 
\end{proposition}

\begin{proof}
The map $df_x$ is injective if and only if its kernel is $0$. By \cref{kernel-is-tangent-of-fibers}, this means that $df_x$ is injective for all $x:X$ if and only if:
\[
\prod_{x:X}T_{(x,\refl_{f(x)})}(\mathrm{fib}_f(f(x)))=0
\]
On the other hand having fibers with trivial tangent space is equivalent to:
\[
\prod_{y:Y}\prod_{x:X}\prod_{p:f(x)=y} T_{(x,p)}(\mathrm{fib}_f(y)) = 0
\]
Both are equivalent by path elimination on $p$.
\end{proof}


\subsection{Smooth maps are submersions}

Here we do not find an equivalence, but just an implication. This might be possible to correct.

\begin{proposition}\label{smooth-sets-map}
Let $f:X\to Y$ be a map between sets. Assume that $X$ is formally smooth and that for all $x:X$, the induced map:
\[N_1(f) : N_1(x)\to N_1(f(x))\]
merely has a section sending $f(x)$ to $x$. Then $f$ is formally smooth.
\end{proposition}

\begin{proof}
Assume given $\epsilon:R$ such that $\epsilon^2=0$ and try to merely find a lift in:
 \begin{center}
      \begin{tikzcd}
        \epsilon=0\ar[r,"\phi"]\ar[d] & X\ar[d,"f"] \\
       1 \ar[r,"y"]\ar[ru,dashed] & Y
      \end{tikzcd}
    \end{center}
    Since $X$ is formally smooth we merely have an $x:X$ such that:
\[\prod_{p:\epsilon=0} \phi(p)=x\]
and therefore:
\[ \epsilon=0 \to y=f(x)\]
This means that we can factor the square:
 \begin{center}
      \begin{tikzcd}
        \epsilon=0\ar[r,"\phi"]\ar[d] & N_1(x) \ar[d]\ar[r]& X\ar[d,"f"] \\
       1 \ar[r,"y"]\ar[ru,dashed] & N_1(f(x)) \ar[r]&Y
      \end{tikzcd}
    \end{center}
    where we can find a lift because the middle arrow has a section sending $f(x)$ to $x$.
\end{proof}

Note that it is immediate from the definition of smoothness that smooth maps induce surjections on tangent spaces. We have a converse when the domain is smooth.

\begin{corollary}\label{smooth-schemes-iff-submersion}
Let $f:X\to Y$ be a map between schemes with $X$ smooth. Then the following are equivalent:
\begin{enumerate}[(i)] 
\item The map $f$ is smooth.
\item For all $x:X$, the induced map:
\[df : T_x(X)\to T_{f(x)}(Y)\]
is surjective.
\end{enumerate}
\end{corollary}

\begin{proof}
It is straightforward to prove that (i) implies (ii), even without any assumption on $X$ and $Y$, by considering the diagram:
 \begin{center}
      \begin{tikzcd}
        1\ar[r]\ar[d] & X\ar[d,"f"] \\
       \mathbb{D}(1) \ar[r] \ar[r]&Y
      \end{tikzcd}
    \end{center}
To prove that (ii) implies (i) we use \cref{smooth-sets-map} with \cref{neighborhood-tangent-correspondence-smooth}.
\end{proof}



\subsection{Smooth schemes are free tangent spaces}


\begin{lemma}\label{smooth-implies-smooth-tangent}
Assume $X$ is a smooth scheme. Then for any $x:X$ the type $T_x(X)$ is formally smooth.
\end{lemma}

\begin{proof}
Consider $T(X) = X^{\mathbb{D}(1)}$ the total tangent bundle of $X$. We have to prove that the map:
\[p:T(X)\to X\]
is formally smooth. Both source and target are schemes, and the source is formally smooth because $X$ is smooth and $\mathbb{D}(1)$ has choice. So by \cref{smooth-schemes-iff-submersion} it is enough to prove that for all $x:X$ and $v:T_x(X)$ the induced map:
\[dp:T_{(x,v)}(T(X))\to T_x(X)\]
is surjective. 

Consider $v':T_x(X)$. By unpacking the definition of tangent spaces, we see that merely finding $w:T_{(x,v)}(T(X))$ such that $dp(w) = v'$ means merely finding:
\[\phi : \mathbb{D}(1) \times \mathbb{D}(1) \to X\]
such that for all $t:\mathbb{D}(1)$ we have that:
\[\phi(0,t) = v(t)\]
\[\phi(t,0) = v'(t)\]

But we know that there exists a unique:
\[\psi : \mathbb{D}(2)\to X\]
such that:
\[\psi(0,t) = v(t)\]
\[\psi(t,0) = v'(t)\]
used for example to define $(v+w)(t) = \psi(t,t)$.

Then the fact that $X$ is smooth and that:
\[\mathbb{D}(2) \to\mathbb{D}(1) \times \mathbb{D}(1) \]
is a closed dense embedding means that there merely exists a lift of $\psi$ to $\mathbb{D}(1) \times \mathbb{D}(1)$, which gives us the $\phi$ we wanted.
\end{proof}

\begin{lemma}\label{smooth-kernel-decidable}
Assume given a linear map:
\[M:R^m\to R^n\] 
which has a formally smooth kernel. Then we can decide whether $M=0$.
\end{lemma}

\begin{proof}
Since $M=0$ is closed, it is enough to prove that it is $\neg\neg$-stable to conclude that it is decidable. Assume $\neg\neg(M=0)$, then for any $x:R^m$ we have a dotted lift in:
 \begin{center}
      \begin{tikzcd}
        M=0\ar[d] \ar[r,"\_\mapsto x"] & K \\
       1 \ar[dotted,ru] &
      \end{tikzcd}
\end{center}
because $K$ is formally smooth, so that we merely have $y:K$ such that: 
\[M=0\to x=y\]
which implies that $\neg\neg(x=y)$ since we assumed $\neg\neg(M=0)$.

Then considering a basis $(x_1,\cdots,x_n)$ of $R^m$, we get $(y_1,\cdots,y_n)$ such that for all $i$ we have that $M(y_i) = 0$ and $\neg\neg(y_i=x_i)$. But then we have that $(y_1,\cdots,y_n)$ is infinitesimally close to a basis and that being a basis is an open proposition, so that $(y_1,\cdots,y_n)$ is a basis and $M=0$.
\end{proof}

\begin{lemma}\label{smooth-corpresented-implies-free}
Assume that $K$ is a finitely copresented module that is also formally smooth. Then it is finite free.
\end{lemma}

\begin{proof}
Assume a finite copresentation:
\[0\to K\to R^m\overset{M}{\to} R^n\]
We proceed by induction on $m$. By \cref{smooth-kernel-decidable} we can decide whether $M=0$ or not.
\begin{itemize}
\item If $M=0$ then $K=R^m$ and we can conclude.
\item If $M\not=0$ then we can find a non-zero coefficient in the matrix corresponding to $M$, and so up to base change it is of the form:

\[
\begin{pmatrix}
1 & \begin{matrix}0&\cdots & 0\end{matrix}  \\
\begin{matrix}0\\ \vdots\\ 0\end{matrix} & \widetilde{M} \\
\end{pmatrix}
\]

But then we know that the kernel of $M$ is equivalent to the kernel of $\widetilde{M}$, and by applying the induction hypothesis we can conclude that it is finite free.
\end{itemize}
\end{proof}

\begin{proposition}\label{smooth-have-free-tangent}
Let $X$ be a smooth scheme. Then for any $x:X$ we have that $T_x(X)$ is finite free.
\end{proposition}

\begin{proof}
By \cref{smooth-implies-smooth-tangent} we have that $T_x(X)$ is formally smooth, so that we can conclude by \cref{smooth-corpresented-implies-free}.
\end{proof}

The dimension of $T_x(X)$ is called the dimension of $X$ at $x$.


\subsection{Smooth schemes are locally standard}

\begin{proposition}\label{smooth-are-locally-standard}
A scheme is smooth if and only if it has a finite open cover by standard smooth schemes.
\end{proposition}

\begin{proof}
We can assume the scheme $X$ affine, say of the form:
\[X = \Spec(R[X_1,\cdots,X_m]/P_1,\cdots,P_l)\]

By \cref{smooth-have-free-tangent}, for any $x:X$ we have that $dP_x$ has free kernel. We partition by the dimension $n$ of the kernel. Then by \cref{rank-equivalent-definitions} we know that $dP_x$ has rank $n$ for every $x$.

We cover $X$ according to which $n$-minor is invertible, so that up to a rearranging of variables and polynomials we can assume that:
\[X = \Spec(R[X_1,\cdots,X_n,Y_1,\cdots,Y_k]/P_1,\cdots,P_n,Q_1,\cdots, Q_l)\]
where we have:
\[dP_{x,y} = \begin{pmatrix}
\left(\frac{\partial P}{\partial X}\right)_{x,y} & \left(\frac{\partial P}{\partial Y}\right)_{x,y} \\
\left(\frac{\partial Q}{\partial X}\right)_{x,y} & \left(\frac{\partial Q}{\partial Y}\right)_{x,y} \\
\end{pmatrix}\]
where we used the notation:
\[\left(\frac{\partial P}{\partial X}\right)_{x,y} = \begin{pmatrix}\left(\frac{\partial P_i}{\partial X_j}\right)_{x,y}\end{pmatrix}_{i,j}\]
so that $\frac{\partial P}{\partial X}$ is invertible of size $n$. Moreover by \cref{rank-bloc-matrix} we get:
\[\left(\frac{\partial Q}{\partial Y}\right)_{x,y} = \left(\frac{\partial Q}{\partial X}\right)_{x,y}\left(\frac{\partial P}{\partial X}\right)_{x,y}^{-1} \left(\frac{\partial P}{\partial Y}\right)_{x,y} \]
which will be useful later.

Now we prove that for any $(x,y):R^{n+k}$ such that $P(x,y)=0$ it is decidable whether
\[Q(x,y)=0 \] 
To do this it is enough to prove that:
\[(Q_1(x,y),\cdots,Q_l(x,y))^2=0 \to (Q_1(x,y),\cdots,Q_l(x,y))=0\]
Assuming $(Q_1(x,y),\cdots,Q_l(x,y))^2=0$, by smoothness there is a dotted lifting in:
 \begin{center}
      \begin{tikzcd}
        R/(Q_1(x,y),\cdots,Q_l(x,y)) & \Spec(R[X_1,\cdots,X_n,Y_1,\cdots,Y_k]/P_1,\cdots,P_n,Q_1,\cdots, Q_l)\ar[l,swap,"(x{,}y)"] \ar[dotted,ld,"(x{'}{,}y{'})"]\\
       R\ar[u] & \\
      \end{tikzcd}
\end{center}
Let us prove that $Q(x,y) = 0$. Indeed we have $(x,y) \sim_1 (x',y')$ so that we have:
\[P(x,y) = P(x',y')+ \left(\frac{\partial P}{\partial X}\right)_{x',y'}(x-x') + \left(\frac{\partial P}{\partial Y}\right)_{x',y'}(y-y') \]
\[Q(x,y) = Q(x',y')+ \left(\frac{\partial Q}{\partial X}\right)_{x',y'}(x-x') + \left(\frac{\partial Q}{\partial Y}\right)_{x',y'}(y-y') \]
Then we have $P(x,y) = 0$, $P(x',y')=0$ and $Q(x',y') = 0$. From the first equality we get:
\[x-x' =  -\left(\frac{\partial P}{\partial X}\right)_{x',y'}^{-1}\left(\frac{\partial P}{\partial Y}\right)_{x',y'}(y-y')\]
so that from the second we get:
\[Q(x,y) = -\left(\frac{\partial Q}{\partial X}\right)_{x',y'}\left(\frac{\partial P}{\partial X}\right)_{x',y'}^{-1}\left(\frac{\partial P}{\partial Y}\right)_{x',y'}(y-y') + \left(\frac{\partial Q}{\partial Y}\right)_{x',y'}(y-y')\]
so that $Q(x,y)=0$ as we have seen previously that:
\[\left(\frac{\partial Q}{\partial Y}\right)_{x',y'} = \left(\frac{\partial Q}{\partial X}\right)_{x',y'}\left(\frac{\partial P}{\partial X}\right)_{x',y'}^{-1} \left(\frac{\partial P}{\partial Y}\right)_{x',y'} \]

From the decidability of $Q(x,y)=0$ we get that $X$ is an open in:
\[\Spec(R[X_1,\cdots,X_n,Y_1,\cdots,Y_k]/P_1,\cdots,P_n)\]
so it is of the form $D(G_1,\cdots,G_n)$, and we have an open cover of our scheme by pieces of the form:
    \[\Spec((R[X_1,\cdots,X_n,Y_1,\cdots,Y_k]/P_1,\cdots,P_n)_G)\]
    Where $P_i(x)=0$ for all $i$ and $G(x)\not=0$ implies:
    \[\mathrm{det}(\mathrm{Jac}(P_1,\cdots,P_n)_x)\not=0\]
    
    We write:
    \[F(x)=\mathrm{det}(\mathrm{Jac}(P_1,\cdots,P_n)_x)\]
    Then for all $x:\Spec(R[X_1,\cdots,X_n,Y_1,\cdots,Y_k]/P_1,\cdots,P_n)$ we have that:
    \[(G(x)\not=0) \to (F(x)\not=0)\]
    so that there exists $n$ such that:
    \[F(x) | G(x)^n\]
    and using boundedness we get $N$ such that for all $x:\Spec(R[X_1,\cdots,X_n,Y_1,\cdots,Y_k]/P_1,\cdots,P_n)$ we have:
    \[F(x) | G(x)^N\]
    and we conclude that $F$ divides $G^N$ in $R[X_1,\cdots,X_n,Y_1,\cdots,Y_k]/P_1,\cdots,P_n$. So by replacing $G$ by $G^N$, we get standard smooth pieces.
\end{proof}


\subsection{Formally étale schemes}

\begin{corollary}
A scheme is formally étale if and only if it has a cover by standard étale schemes.
\end{corollary}

\begin{proof}
TODO
\end{proof}

\begin{corollary}
Let $f:X\to Y$ be a map between schemes. Assume $X$ is smooth. Then the following are equivalent:
\begin{enumerate}[(i)]
\item The map $f$ is étale. 
\item For all $x:X$, the induced map:
\[df : T_x(X)\to T_{f(x)}(Y)\]
is an iso.
\end{enumerate}
\end{corollary}

\begin{proof}
TODO
\end{proof}



