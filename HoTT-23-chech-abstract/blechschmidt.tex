% latexmk -pdfxe -pvc blechschmidt.tex
% This is a template for submissions to HoTT 2023.
% The file was created on the basis of easychair.tex,v 3.5 2017/03/15 
%
% Before submitting, rename the resulting pdf file to "yourname.pdf" 
%
\documentclass[letterpaper]{../util/easychair}
\usepackage{doc}
\usepackage[expansion=true,protrusion=true]{microtype}
%
\newcommand{\easychair}{\textsf{easychair}}
\newcommand{\miktex}{MiK{\TeX}}
\newcommand{\texniccenter}{{\TeX}nicCenter}
\newcommand{\makefile}{\texttt{Makefile}}
\newcommand{\latexeditor}{LEd}

% some stuff from ../util/zarisky.cls:

\RequirePackage{amsmath,amssymb,mathtools}
% Referenzen
\RequirePackage[backend=biber,style=alphabetic, backref, backrefstyle=none]{biblatex}
\addbibresource{../util/literature.bib}


% content of ../util/zarisky.sty:
% names for types
\newcommand{\R}{\mathbb{R}}
\newcommand{\Z}{\mathbb{Z}}
\newcommand{\N}{\mathbb{N}}
\newcommand{\Bool}{\mathrm{Bool}}
\DeclareMathOperator{\Fin}{Fin}
\newcommand{\unit}{\mathbf{1}}
\newcommand{\two}{\mathbf{2}}
\newcommand{\isContr}{\mathrm{isContr}}
\newcommand{\isProp}{\mathrm{isProp}}
\newcommand{\isSet}{\mathrm{isSet}}
\newcommand{\isEquiv}{\mathrm{isEquiv}}
\newcommand{\qinv}{\mathrm{qinv}}
\newcommand{\mU}{\mathcal{U}}
\newcommand{\Eq}[1]{\mathrm{Eq}_{#1}}
\newcommand{\isNType}[1]{\mathrm{is}\mbox{-}{#1}\mbox{-}\mathrm{type}}
\newcommand{\nType}[1]{#1\mbox{-}\mathrm{Type}}
\newcommand{\Type}{\mathrm{Type}}
\newcommand{\Prop}{\mathrm{Prop}}
\newcommand{\Open}{\mathrm{Open}}
\newcommand{\propTrunc}[1]{\lVert #1 \rVert}

% names for terms
\newcommand{\id}{\mathrm{id}}
\newcommand{\refl}{\mathrm{refl}}
\newcommand{\pair}{\mathrm{pair}}
\newcommand{\FunExt}{\mathrm{FunExt}}
\newcommand{\transp}{\mathrm{tr}}
\newcommand{\transpconst}{\mathrm{tconst}}
\newcommand{\ua}{\mathrm{ua}}
\newcommand{\fib}{\mathrm{fib}}

% category theory
\newcommand{\Hom}{\mathrm{Hom}}
\newcommand{\Sh}{\mathrm{Sh}}
\newcommand{\yo}{\mathrm{y}}

% algebra
\newcommand{\inv}{\mathrm{inv}}
\newcommand{\divides}{\mathbin{|}}
\DeclareMathOperator{\AbGroup}{Ab}
\DeclareMathOperator{\im}{im}
\DeclareMathOperator{\coker}{coker}
\newcommand{\Tors}[1]{#1\text{-}\mathrm{Tors}}
\newcommand{\Mod}[1]{#1\text{-}\mathrm{Mod}}
\newcommand{\Vect}[2]{#1\text{-}\mathrm{Vect}_{#2}}
\newcommand{\fpMod}[1]{#1\text{-}\mathrm{Mod}_{\text{fp}}}
\newcommand{\Alg}[1]{#1\text{-}\mathrm{Alg}}

% algebraic geometry
\DeclareMathOperator{\Spec}{Spec}
\DeclareMathOperator{\Sch}{\mathrm{Sch}_{qc}}
\newcommand{\A}{\mathbb{A}}
\newcommand{\D}{\mathbb{D}}
\newcommand{\bP}{\mathbb{P}}


% misc
\newcommand{\notion}[1]{\emph{#1}\index{#1}}
\providecommand*\colonequiv{\vcentcolon\mspace{-1.2mu}\equiv}
\newcommand{\ignore}[1]{}
\newcommand{\rednote}[1]{{\color{red}(#1)}}


%
\title{\v{C}ech Cohomology in Homotopy Type Theory
}

% Authors are joined by \and. 
% Their affiliations are given by \inst, which indexes
% into the list defined using \institute
%
\author{
Ingo Blechschmidt \inst{1}
% uncomment the following for multiple authors.
\and 
 Felix Cherubini \inst{2}%
 \thanks{Speaker.}%
\and 
 David Wärn \inst{3}
}

% Institutes for affiliations are also joined by \and,
\institute{
  \email{iblech@speicherleck.de}
% uncomment the following for multiple authors.
\and
  University of Gothenburg\\
  \email{felix.cherubini@posteo.de}
\and
  University of Gothenburg\\
  \email{warnd@chalmers.se}
}

%  \authorrunning{} has to be set for the shorter version of the authors' names;
% otherwise a warning will be rendered in the running heads. When processed by
% EasyChair, this command is mandatory: a document without \authorrunning
% will be rejected by EasyChair
\authorrunning{Blechschmidt, Cherubini and Wärn}

% \titlerunning{} has to be set to either the main title or its shorter
% version for the running heads. When processed by
% EasyChair, this command is mandatory: a document without \titlerunning
% will be rejected by EasyChair
\titlerunning{\v{C}ech Cohomology in Homotopy Type Theory}

\begin{document}
\maketitle

This is a report on work in progress (incomplete draft: \cite{chech-draft}).

In pure mathematics,
it is a common practice to simplify questions about complicated objects
by assigning them more simple objects in a systematic way,
that faithfully represents some features of interest.
One particular, but still surprisingly broad applicable instantiation of this appraoch,
is the assignment of a sequence of abelian groups, the cohomology groups, to spaces, sheaves and other things.
Over the last century, cohomology was first discovered in concrete examples, then generalized and streamlined --
a process that culminated in the presentation of cohomology groups as truncated mapping spaces in higher toposes.%
\footnote{See, e.g. the introduction of \cite{lurie-htt}.}

This is a representation, we can easily and elementary use through the interpretation of homotopy type theory in higher toposes.
In \cite{evan-master-thesis} results about cohomology theories like the Mayer--Vietoris sequence were proven and computations were carried out,
in \cite{floris-thesis}, the Serre spectral sequence was constructed and used.
The latter also introduced cohomology with non-constant coefficients,
which are the right level of generality for the applications we have in mind.
We are particularly interested in computing cohomology groups of sheaves in algebraic geometry,
which can be done synthetically using the foundation laid out by \cite{draft}
building on work and ideas of Ingo Blechschmidt~\cite{ingo-thesis}, Anders Kock~\cite{kock-sdg}
and David Jaz Myers~\cite{myers-talk1}, \cite{myers-talk2}.

In this setup, the basic spaces in algebraic geometry, schemes,
are just sets with a particular property \cite[Definition~4.3.1]{draft},
and instead of sheaves of abelian groups on a type $X$,
we consider more generally maps $A:X\to\AbGroup$ to the type of abelian groups.
The cohomology groups are then defined as dependent function types with values in Eilenberg--MacLane spaces
\[ H^n(X,A)\colonequiv \|(x:X)\to K(A(x),n)\|_0,\]
a definition first suggested by Shulman~\cite{mike-blogpost}.
Due to its simplicity, this is very convenient to work with.
One common way to compute cohomology groups $H^n(X,\mathcal F)$ is to
use results about the cohomology of simple subspaces $U_i\subseteq X$.
A computational result on the case with two subspaces $U,V\subseteq X$ is known as the \notion{Mayer--Vietoris sequence}.
In general this sequence helps to compute the cohomology groups of a pushout
and was constructed for cohomology with constant coefficients in a group by Cavallo~\cite{evan-master-thesis}.
We generalize this result to non-constant coefficients
with a slick proof the second author learned in part from Urs Schreiber in the course of his PhD thesis.

\v{C}ech Cohomology, in the sense of this work,
is a generalization of the Mayer--Vietoris sequence in the case, where $U$ and~$V$ are actually subtypes of a set,
to a space $X$ which is the union of fintely many subtypes $U_i\subseteq X$, i.e. $\bigcup_{i}U_i=X$.
From the point of view of \emph{synthetic homotopy theory}, the cohomology groups of sets are not very interesting,
but it is fundamental to our intended applications in \emph{synthetic algebraic geometry}, where schemes are sets.
In the latter subject, it was unclear for a long time how one could set up a theory of cohomology,
since the classical treatment relies on injective resolutions or flabby resolutions which, depending on the application, require Zorn's lemma or the law of excluded middle \cite{blechschmidt-flabby}.

In \cite{draft} this problem is circumvented,
by using a constructively justified axiom which internally reflects the topology of the relevant topos
as a nonclassical choice principle
and, secondly, as mentioned above, by using higher types to define and work with cohomology.

We present two approaches to a proof of a sufficiently general isomorphism between \v{C}ech cohomology groups
and cohomology groups defined using higher types.
The first approach is more conceptual and makes use of the higher types we have available in HoTT.
It is also related to how one would produce a \v{C}ech Cohomology theorem in higher category theory:
the space is represented as a colimit, so mapping into the coefficients should yield a limit description of
the (untruncated) cohomology of the whole space.

The second approach very roughly follows the seminal treatments of Grothendieck~\cite{tohoku1957} and Buchsbaum~\cite{buchsbaum}, but employs Riehl's modern Kan extension framework~\cite{riehl-cathtpy} and language from the Stacks project~\cite[Tag~05S7]{stacks-project}.
Briefly, the idea is first to setup, as a bootstrapping step, \v{C}ech
cohomology on quasi-compact and separated schemes. A Buchsbaum-style effacement argument
then shows that \v{C}ech resolutions can be used to construct and
compute actual (abstract) cohomology. This approach not only yields the
cohomology groups, but also the total right derived functor $\mathbb{R}\Gamma$.
In the future, we would like to widen the scope to the étale and fppf
topology, employing hypercoverings as in Verdier's hypercovering theorem.

We aim to show that both cohomology defined using higher types and \v{C}ech cohomology satisfies the universal property
of a \notion{universal $\partial$-functor} in some particular but still quite relevant situations.
While this approach seems to work only for particular schemes,
it also seems to need far less involved calculations.
Furthermore, $\partial$-functors could be reusable as a framework for questions about $\mathcal{E}\!xt$-sheaves and group cohomology.

As a prerequisite to make both approaches to \v{C}ech Cohomology,
we show that some schemes of interest have acyclic covers.
More specifically, we show that for some particular class of coefficients, called \emph{weakly quasicoherent} modules,
higher cohomology groups always vanish on affine schemes.
This result builds on a result for first cohomology groups from \cite{draft}
and already makes use of higher types and their homotopy theory.

\printbibliography

\end{document}
