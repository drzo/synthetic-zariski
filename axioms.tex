\subsection{Statement of the axioms}
We always assume there is a commutative ring $R$.
Sometimes we will assume $R$ has additional properties, or, more generally,
axioms hold that involve $R$.
We will always mention which of these axioms are needed to prove each statement,
by listing the shorthands introduced in the axioms below.

\begin{axiom}[Loc]%
  \label{loc}\index{Loc}
  $R$ is a local ring.
\end{axiom}

\begin{axiom}[SQC]%
  \label{sqc}\index{sqc}
  For any finitely presented $R$-algebra $A$, the homomorphism
  \[ a \mapsto (\varphi\mapsto \varphi(a)) : A \to (\Spec A \to R)\]
  is an isomorphism of $R$-algebras.
\end{axiom}

\begin{axiom}[Z-choice]%
  \label{Z-choice}\index{Z-choice}
  Let $A$ be a finitely presented $R$-algebra
  and let $B : \Spec A \to \mU$ be a family of inhabited types.
  Then there merely exists
  a finite list of coprime elements $f_1, \dots, f_n \in A$
  together with dependent functions $s_i : \Pi_{x : D(f_i)} B(x)$.
  As a formula:
  \[ (\Pi_{x : \Spec A} \propTrunc{B(x)}) \to
     \propTrunc{ \Sigma_{n : \N} \Sigma_{f_1, \dots, f_n : A}
      ((f_1, \dots, f_n) = (1)) \times
      \Pi_i \Pi_{x : D(f_i)} B(x) }
     \rlap{.}
  \]
\end{axiom}

\subsection{First consequences}

\begin{proposition}[using \axiomref{sqc}]%
  \label{spec-embedding}
  For all finitely generated $R$-algebras $A$ and $B$ we have
  \[ (\Spec B \to \Spec A) =\Hom_{\Alg{R}}(A,B)\]
  -- where the equality is induced by exponentiation with $R$.
\end{proposition}

An important consequence, which may be called \notion{weak nullstellensatz}:

\begin{proposition}[using \axiomref{loc}, \axiomref{sqc}]%
  \label{weak-nullstellensatz}
  If $A$ is a finitely generated $R$-algebra, $\Spec A=\emptyset$, if and only if $A=0$.
\end{proposition}

\begin{proof}
  If $\Spec A = \emptyset$
  then $A = R^{\Spec A} = R^\emptyset = 0$
  by (\axiomref{sqc}).
  If $A = 0$
  then there are no homomorphisms $A \to R$
  since $1 \neq 0$ in $R$ by (\axiomref{loc}).
\end{proof}

The following are originally proven in 
\cite{ingo-thesis}[Section 18]:

\begin{proposition}[using \axiomref{sqc}, \axiomref{loc}]%
  \label{nilpotence-double-negation}\label{non-zero-invertible}\label{generalized-field-property}
  
  \begin{enumerate}[(a)]
  \item An element $x:R$ is nilpotent,
    if and only if $\neg \neg (x=0)$.
  \item An element $x:R$ is invertible,
    if and only if $x\neq 0$.
  \item A vector $x:R^n$ is non-zero,
    if and only if one of its entries is invertible.
  \end{enumerate}
\end{proposition}
