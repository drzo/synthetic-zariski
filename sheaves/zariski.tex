In this section we work in with Zariski sheaves, i.e. the usual synthetic algebraic geometry. A lot of it is direct reformulation of the previous section.

\subsection{Definition of sheaves}

\begin{definition}
A pre-topology $T$ is a class of affine schemes.
\end{definition}

\begin{definition}
A topology $T$ is a class of affine schemes such that:
\begin{itemize}
\item It is stable by dependent sums and contains the unit.
\item It is stable by Zariski cover, meaning that given a Zariski cover:
\[\sum_{i:I}U_i \to X\]
if $X\in T$ then $\sum_{i:I}U_i\in T$.
\end{itemize}
\end{definition}

There is a smallest topology generated by a pre-topology. 

\begin{definition}
Let $T$ be a pre-topology, then a type is a $T$-sheaf if it is $\propTrunc{X}$-local for all $X\in T$. 
\end{definition}

This agree with the usual definition of sheaves at least for sets (and presumably for all $n$-types for a finite $n$). We write $L_T$ for $T$-sheafification, which is a lex modality. 

\begin{lemma}
Let $T$ be a topology and let $T'$ be the smallest topology containing $T$. A type is a $T$-sheaf if and only if it is a $T'$-sheaf.
\end{lemma}

\begin{proof}
We proceed as in \cref{topology-pretopology-same-sheaves}. The only new result we need is that is that the class of affine schemes $X$ such that $L_T\propTrunc{X}$ is closed by Zariski cover. But if:
\[Y\to X\]
is a Zariski cover then:
\[\propTrunc{Y}\leftrightarrow\propTrunc{X}\]
so we can conclude.
\end{proof}

\subsection{Covers and choice}

\begin{lemma}\label{sheaf-proposition}
Let $T$ be a topology and $P$ be a proposition, then:
\[L_T(P)\leftrightarrow \exists X\in T.\ P^X\]
\end{lemma}

\begin{proof}
We proceed as in \cref{sheaf-replacement-proposition}, using Zariski-local choice and the stability of $T$ by Zariski cover instead of choice.
\end{proof}

\begin{definition}
Let $T$ be a pre-topology. A $T$-cover of an affine scheme is inductively defined by:
\begin{itemize}
\item The identity is a $T$-cover, and the composite of $T$-covers is a $T$-cover.
\item Given a Zariski cover $(U_i)_{i:I}$ of $X$ and for all $i$ is a $T$-cover of $U_i$, the induced map to $X$ is a $T$-cover.
\item A map which fibers are in $T$ is a $T$-cover.
\end{itemize}
\end{definition}

\begin{lemma}\label{cover-can-be-defined-pointwise}
Let $T$ be a pre-topology generating the topology $T'$. Then a map is a $T$-cover if and only if its fibers are in $T'$.
\end{lemma}

\begin{proof}
TODO
\end{proof}

\begin{lemma}\label{cover-local-choice}
Let $T$ be a pre-topology, assume given an affine scheme $\Spec(A)$ and a family of types $P(x)$ for $x:A$. Assume:
\[\prod_{x:\Spec(A)}L_T\propTrunc{P(x)}\]
then there exists a $T$-cover:
\[f:\Spec(B)\to \Spec(A)\]
such that:
\[\prod{y:\Spec(B)} P(f(y))\]
\end{lemma}

\begin{proof}
Same as \cref{sheaves-have-local-choice} considering $T'$ the topology generated by $T$, with \cref{cover-can-be-defined-pointwise} and Zariski-local choice rather than choice.
\end{proof}

\subsection{Sheaf models TODO}

This section is incomplete, we would need to study the sheaf interpretation more in details (TODO).

\begin{lemma}\label{cover-are-surjective}
Let $T$ be a pre-topology, then we have $L_T(\propTrunc{X})$ for all $X\in T$.
\end{lemma}

\begin{proof}
Immediate by \cref{sheaf-proposition}.
\end{proof}

\begin{theorem}
Let $T$ be a subcanonical pre-topology, then $T$-sheaves enjoys the following:
\begin{enumerate}[(i)]
\item For any $X\in T$ we have $\propTrunc{X}$.
\item Synthetic quasi-coherence.
\item Affine schemes have $T$-local choice (meaning choice using $T$-covers).
\end{enumerate}
\end{theorem}

\begin{proof}
For (i) we just use \cref{cover-are-surjective}

For (ii) we use that subcanonicity implies that the statement of synthetic quasi coherence is a $T$-sheaf.

For (iii) we use \cref{cover-are-surjective} to get a $T$-cover, then we need to check that it is a $T$-cover internally to $T$-sheaves. TODO
\end{proof}




