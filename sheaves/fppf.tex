This section is concerned with fppf sheaves in the Zariski topos.

\subsection{Definition}

\begin{definition}
The fppf topology is the topology generated by $\Spec(R[X]/g)$ for all monic $g:R[X]$.
\end{definition}

\subsection{Schemes are fppf sheaves}

\begin{lemma}\label{fppf-subcanonical}
The type $R$ is an fppf sheaf. In other words, the fppf topology is subcanonical.
\end{lemma}

\begin{proof}
Using \cref{set-sheaves-condition} we just need to prove that the map from $R$ to the equaliser of:
\[R[X]/g \rightrightarrows R[X]/g \otimes R[X]/g\]
is an equivalence. But since $g$ is monic we merely have:
\[R[X]/g \simeq R^n\]
and then we check that the induced map from $R$ to the equaliser of:
\[R^n \rightrightarrows R^n\otimes R^n\]
is an equivalence.
\end{proof}

\begin{corollary}
Any affine scheme is a fppf sheaf. Any f.p. $R$-algebra is an fppf sheaf.
\end{corollary}

\begin{lemma}\label{scheme-are-sheaf-from-affine}
Assume given a type $P$ such that:
\begin{itemize}
\item The type $R$ is $P$-local.
\item The type of open propositions is $P$-smooth.
\end{itemize}
Then any scheme is $P$-local.
\end{lemma}

\begin{proof}
Since $R$ is $P$-local we see that all affine schemes are $P$-local through stability under dependent products and identity types.

We check that any scheme $X$ is $P$-smooth. Assume given constant map:
\[f:P\to X\]
Take $(U_i)_{i:I}$ a finite cover of $X$ by affine scheme. Then for any $i:I$ we have that $f^{-1}(U_i)$ is a constant open in $P$, so since the type of open is $P$-smooth, we merely have an open proposition $V_i$ such that for all $x:P$, we have:
\[(x\in f^{-1}(U_i) )\leftrightarrow V_i\]
Since the $f^{-1}(U_i)$ cover $P$, we have that:
\[P\to \lor_{i:I} V_i\]
But open propositions are affine schemes and affine schemes are $P$-local, so we have:
\[ \lor_{i:I} V_i\]
Assume $k:I$ such that $V_k$ holds. Then $f^{-1}(U_k) = P$ and the map $f$ factors through the affine scheme $U_k$. Since affine schemes are $P$-local, we merely have a lift for $f$.

Now we conclude that any scheme is $P$-local by proving that its identity types are $P$-local. Indeed they are propositional schemes, so they are $P$-smooth propositions, so they are $P$-local.
\end{proof}

\begin{lemma}\label{roots-monic-proper}
For any monic $g:R[X]$, we have that $\Spec(R[X]/g)$ is projective. In particular it is compact, meaning that for any open $U$ in $\Spec(R[X]/g)$ the proposition:
\[\prod_{x:\Spec(R[X]/g)}U(x)\]
is open.
\end{lemma}

\begin{proof}
Assume that:
\[g=X^n+a_{n-1}X^{n-1}+\cdots+a_0\]
Then we consider the homogeneous polynomial:
\[f(X,Y) = X^n + a_{n-1}X^{n-1}Y+\cdots+a_0Y^n\]
We prove that:
\[\sum_{[x,y]:\bP^1}f(x,y) = 0\]
is equivalent to $\Spec(R[X]/g)$. Indeed for any $x,y:R$ such that $f(x,y)=0$, we have that $x\not=0$ implies $y\not=0$, so that $(x,y)\not=0$ implies $y\not=0$. Then:
\[\sum_{[x,y]:\bP^1}f(x,y) = 0\]
is equivalent to:
\[\sum_{x:R} f(x,1)=0\]
which is the type of roots of $g$. 

Now we conclude using the fact that 
\[\sum_{[x,y]:\bP^1}f(x,y) = 0\]
is closed in the compact scheme $\bP^1$, so that it is compact.
\end{proof}

\begin{proposition}\label{schemes-are-fppf-sheaves}
Any scheme is an fppf sheaf.
\end{proposition}

\begin{proof}
Assume given $g:R[X]$ monic, by \cref{scheme-are-sheaf-from-affine} it is enough to prove that $R$ is $\propTrunc{\Spec(R[X]/g)}$-local (this is \cref{fppf-subcanonical}) and that the type of open propositions is $\propTrunc{\Spec(R[X]/g)}$-smooth. 

Assume given a constant open $D(h_1,\cdots,h_n)$ in $\Spec(R[X]/g)$. Then for any $x:\Spec(R[X]/g)$ we have that:
\[x\in D(h_1,\cdots,h_n) \leftrightarrow \prod_{y:\Spec(R[X]/g)} y\in D(h_1,\cdots,h_n)\]
because the open $D(h_1,\cdots,h_n)$ is constant. But the right hand side is open by \cref{roots-monic-proper}, so we indeed have a lift.
\end{proof}

\subsection{Fppf covers are flat}

\begin{definition}
A map between affine schemes:
\[\Spec(B)\to \Spec(A)\]
is flat if for all injective morphism of f.p. $A$-module: 
\[M\to N\]
the induced map:
\[B\otimes_A M \to B\otimes_A N\]
is injective.
\end{definition}

\begin{lemma}\label{root-monic-flat}
Assume a f.p. $R$-algebra $A$ and $g:A[X]$ monic, then the induced map:
\[\Spec(A[X]/g)\to \Spec(A)\]
is flat.
\end{lemma}

\begin{proof}
Since $g$ is monic we have:
\[A[X]/g\simeq A^n\]
as $A$-modules. Then given an injective map of f.p. $A$-modules:
\[f:M\to N\]
the induced map;
\[A[X]/g\otimes_AM \to A[X]/g\otimes_AN\]
is of the form:
\[f^n:M^n\to N^n\]
which is injective.
\end{proof}

\begin{lemma}\label{localisation-is-flat}
For any f.p. $R$-algebra $A$ and any $f:A$, the map:
\[\Spec(A_f)\to\Spec(A)\]
is flat.
\end{lemma}

\begin{proof}
Assume given an injective map of f.p. $A$-modules:
\[g:M\to N\]
Assume given:
\[\frac{m}{f^i},\frac{n}{f^j}\]
in $A_f\otimes_AM$ such that:
\[g(\frac{m}{f^i}) = g(\frac{n}{f^j})\]
in $A_f\otimes_AN$. Then there is $k$ such that:
\[f^{j+k}g(m) = f^{i+k}g(n)\]
in $N$. But then since $g$ is $A$-linear we have:
\[g(f^{j+k}m) = g(f^{i+k}n)\]
in $N$, so from injectivity of $g$ we conclude:
\[f^{j+k}m = f^{i+k}n\]
which implies that:
\[\frac{m}{f^i} = \frac{n}{f^j}\]
in $A_f\otimes_AM$.
\end{proof}

We have a kind of converse:

\begin{lemma}\label{injectivity-is-zariski-local}
Let $A$ be a f.p. $R$ algebra and $f_1,\cdots,f_n:A$ such that $(f_1,\cdots,f_n) = A$ . Assume given a map between between f.p. $A$-modules:
\[g:M\to N\] 
such that for all $i$ the induced map:
\[A_{f_i}\otimes_AM \to A_{f_i}\otimes_AN\]
is injective. Then $g$ is injective.
\end{lemma}

\begin{proof}
Assume $m$ in $M$ such that $g(m)=0$ in $N$. Then $g(m) =0$ in $A_{f_i}\otimes_AN$ so by the assumed injectivity we have $m=0$ in $A_{f_i}\otimes_AM$ for all $i$.

This means that for all $i$ we have $k_i:\N$ such that $f_i^{k_i}m = 0$ in $M$. But then since the $D(f_i^{k_i})$ cover $\Spec(A)$ we know that $(f_1^{k_1},\cdots,f_n^{k_n}) = A$ and we can conclude that $m=0$ in $M$ as we needed.
\end{proof}

\begin{lemma}\label{flat-zariski-local}
Being flat is Zariski-local in the target. More precisely assume given a map between affine schemes:
\[f:\Spec(B)\to \Spec(A)\]
and a Zariski cover $(U_i)_{i:I}$ of $\Spec(A)$. If for all $i:I$ the map:
\[f : f^{-1}(U_i)\to U_i\]
is flat, then the map $f$ is flat.
\end{lemma}

\begin{proof}
Assume an injective map betweem f.p. $A$-module:
\[M\to N\]
For all $i$ we have an injective map between f.p. $A_{f_i}$-module:
\[A_{f_i}\otimes_AM\to A_{f_i}\otimes_AN\]
because localisation is flat by \cref{localisation-is-flat}. Since:
\[\Spec(B\otimes_AA_{f_i})\to \Spec(A_{f_i})\]
is flat, we know that:
\[A_{f_i}\otimes_AB\otimes_AM\to A_{f_i}\otimes_AB\otimes_AN\]
is injective for all $i$, and from \cref{injectivity-is-zariski-local} we conclude that the map:
\[B\otimes_AM\to B\otimes_AN\]
is injective.
\end{proof}

\begin{lemma}
Any fppf cover is flat.
\end{lemma}

\begin{proof}
We proceed by induction on the fppf cover:
\begin{itemize}
\item If the fppf cover is an identity or a composite of fppf cover, we just need to check that identity maps are flat and that flat maps are stable under composition.
\item If the fppf cover is Zariski-locally an fppf cover we use \cref{flat-zariski-local}.
\item If the fppf cover has fibers of the form $\Spec(R[x]/g)$ with $g$ monic, then we know that it is Zariski-locally of the form:
\[\Spec(B[X]/f)\to \Spec(B)\]
with $f$ monic and we conclude by \cref{flat-zariski-local} and \cref{root-monic-flat}.
\end{itemize}
\end{proof}

\subsection{The fppf sheaf model}

\begin{theorem}
The following holds when interpreted in fppf sheaves:
\begin{enumerate}[(i)]
\item The ring $R$ is local, and any monic polynomial in $R$ merely has a root.
\item Synthetic quasi coherence.
\item Affine schemes enjoys flat local choice, meaning that TODO.
\end{enumerate}
\end{theorem}

\begin{proof}
TODO
\end{proof}


