
TODO: This needs to be updated and made coherent.

An $n$-th delooping of a pointed space $A$ which is also $(n-1)$-connected is unique and usually written as $B^nA$ or $K(A,n)$ and called an \notion{Eilenberg-MacLane space}.
We will just write $A_n$ for an $n$-th delooping.

It is known, that in HoTT, a (0-truncated) abelian group can be delooped arbitrarily often (\cite{LicataFinster}).

Contents of this section are from Mike Shulman's posts on the HoTT-Blog about cohomology,
Floris van Doorn's thesis (\cite{floris-thesis})[section 5.3]
and common knowledge in the field that is not written up,
with the possible exception of the Mayer-Vietoris-Sequence with non-constant coefficients (\cref{mayer-vietoris-sequence}).

Suppose we have a pointed type $A$ with delooping $A_k$ for any $k:\N$.
Then, analogous to the definition of the $k$-th homotopy group
\[ \pi_k:\equiv\|\Omega^kA \|_0 \]
one could define homotopy groups of \emph{negative} degree $-k$ by:
\[ \pi_{-k}:\equiv\|A_k \|_0 \]
Note that these will be trivial for any Eilenberg-MacLane spectrum, since for those, $A_{k+1}$ is $k$-connected for $k:\N$.
In general, spectra with trivial homotopy groups in negative degree are called \emph{connective}.
The result in this article is concerned with Eilenberg-MacLane spectra.

We will use spectra varying over a space as coefficints for cohomology,
which corresponds to the classical concept of parametrized spectra.
We fix our terminology in the following definition.

\begin{definition}
  \begin{enumerate}[(a)]
  \item A \notion{spectrum} is a sequence of pointed types $(A_k)_{k:\N}$, together with pointed equivalences $A_k\simeq \Omega A_{k+1}$.
  \item A spectrum $(A_k)_{k:\N}$ is \notion{connective}, if $\|A_{k+1}\|_0\simeq 1$ for all $k:\N$.
  \item Let $X$ be a type. A \notion{parametrized spectrum} over $X$, is a dependent function, which assigns to any $x:X$, a spectrum $(A_{x,k})_{k:\N}$.
    For brevity, We will call a parametrized spectrum $A\equiv x\mapsto (A_{x,k})_{k:\N}$ over $X$ just \notion{spectrum over $X$}.
  \item A \notion{morphism of spectra} $A,A'$ over $X$, is given by a sequence of pointed maps $f_{x,k}:A_{x,k}\to A_{x,k}'$ for any $x:X$,
    such that $\Omega f_{x,k+1}=f_{x,k}$ (using the pointed equivalences).
  \end{enumerate}
\end{definition}

The connective spectra form a nice ``subcategory'':
We will need the following (coreflective) construction that turns a spectrum into a connective spectrum.
See \cref{def:connected-cover} for the definition of the $k$-connected cover ``$D^k_Xd$''.

\begin{definition}
  For a spectrum $A$,
  the following construction is called the \notion{connective cover}:
  \[ \hat{A}:\equiv k\mapsto D_{A,k}^{k-1}\]
  There is also a sequence of pointed maps $\varphi_k:\hat{A}_k\to A_k$, given by the projection from the connected covers.
\end{definition}

The following fact will be useful to us on various occations and can be proven using the uniqueness of Eilenberg-MacLane spaces:

\begin{lemma}
  \label{em-comm-pi}
  Let $X$ be a type and $A:X\to\AbGroup$ a dependent abelian group.
  If for all $0<l\leq n$ the type $(x:X)\to K(A_x,l)$ is connected, then
  \[
    ((x:X)\to K(A_x,n)) = K((x:X)\to A_x,n)
    \rlap{.}
  \]
\end{lemma}

\begin{definition}
  The $k$-th cohomology group of $X$ with coefficients in $A$ is the following:
  \[ H^k(X,A):\equiv\left\|(x:X)\to A_{x,k}\right\|_0 \]
\end{definition}

\begin{definition}
  Let $f:Y\to X$ be a map of types and $\mathcal F:X\to \AbGroup$ and $\mathcal G:Y\to \AbGroup$ dependent abelian groups.
  \begin{enumerate}[(a)]
  \item $f^\ast \mathcal F\colonequiv (y:Y)\mapsto \mathcal F_{f(y)}$ is called the \notion{pullback}\index{$f^\ast \mathcal F$} of $\mathcal F$ along $f$.
  \item $f_{\ast}\mathcal G\colonequiv (x:X)\mapsto \mathcal G^{\fib_f(x)}$ is called the \notion{push forward}\index{$f_{\ast}\mathcal G$} of $\mathcal G$ along $f$.
  \end{enumerate}
\end{definition}

Cohomology does not change along maps with cohomologically trivial fibers:

\begin{lemma}
  \label{cohomologically-trivial-fibers}
  Let $f:Y\to X$ and $\mathcal F:Y\to\AbGroup$ be such that $H^l(\fib_f(x),\pi_1^\ast\mathcal F)=0$,
  then
  \[
    H^n(Y,\mathcal F)=H^n(X,f_\ast\mathcal F)
    \rlap{.}
  \]
\end{lemma}

\begin{proof}
  By direct application of \Cref{em-comm-pi}.
\end{proof}

An important notion in abelian categories, is that of short exact sequences.
And it is important to us here, since for every short exact sequence (somewhere), there should be an induced long exact sequence on cohomology groups.
The cokernel of an exact sequence, corresponds to a cofiber of a map of spectra.

\begin{definition}
  Let $f:A\to A'$ be a map of spectra.
  \begin{enumerate}[(a)]
  \item The \notion{cofiber} of $f$ is given by the spectrum
    \[ C_{f,k}:\equiv \fib_{f_{k+1}}\]
    together with the map $c:A'\to C_f$, where $c_k$ is induced in the following diagram of pullback-squares:
    \begin{center}
      \begin{tikzcd}
        A_k\ar[r,"f_k"]\ar[d] & A'_k\ar[d,dashed]\ar[r] & 1\ar[d] \\
        1\ar[r] & C_{f,k}\ar[r]\ar[d] & A_{k+1}\ar[d] \\
        & 1\ar[r] & A'_{k+1}
      \end{tikzcd}
    \end{center}
  \item The \notion{fiber} of $f$ is given by the spectrum
    \[ \fib_{f,k}:\equiv \fib_{f_k}\]
  \end{enumerate}
\end{definition}

Note that $f:A\to A'$ is always the fiber of its cofiber and conversely, $f:A\to A'$ is always the cofiber of its fiber, which is very different from the situation in a general abelian category,
where for example not every map is the kernel of its cokernel.

\begin{definition}
  A sequence of morphisms of spectra over $X$
  \begin{center}
    \begin{tikzcd}
      A\ar[r,"f"] & A'\ar[r,"g"] & A''
    \end{tikzcd}
  \end{center}
  is a \notion{fiber sequence}, if the following equivalent statements hold:
  \begin{enumerate}[(a)]
  \item $f_x$ is the fiber of $g_x$ for all $x:X$
  \item $f_{x,k}$ is the fiber of $g_{x,k}$ for all $x:X$ and $k:\N$
  \end{enumerate}
  If all spectra involved are Eilenberg-MacLane spectra, we call the sequence \notion{exact}, and vice versa,
  if we speak of a \notion{short exact sequence} of spectra (over $X$), we assume all spectra involved are Eilenberg-MacLane and we have a fiber sequence.
\end{definition}

\begin{lemma}
  If $A\to A'\to A''$ is a fiber sequence, then the induced square:
  \begin{center}
    \begin{tikzcd}
      \prod_{x:X}A_{x,k}\ar[r]\ar[d] & \prod_{x:X}A'_{x,k}\ar[d] \\
      1\ar[r] & \prod_{x:X}A''_{x,k}
    \end{tikzcd}
  \end{center}
  is a pullback square for all $k:\N$.
\end{lemma}
\begin{proof}
  $\prod$ maps families of pullback squares to a pullback square.
\end{proof}

This is just tailored to prove the following proposition:

\begin{proposition}
  For any fiber sequence
  \[ A\to A'\to A'' \]
  of spectra over $X$, there is a long exact sequence of cohomology groups:
  \begin{center}
    \begin{tikzcd}
      & \dots\ar[r]& H^{n-1}(X,A'')\ar[dll]\\
      H^n(X,A)\ar[r] & H^n(X,A')\ar[r] & H^n(X,A'')\ar[lld] \\
      H^{n+1}(X,A)\ar[r] & H^{n+1}(X,A')\ar[r] & \dots
    \end{tikzcd}
  \end{center}
\end{proposition}

\begin{proof}
  Apply homotopy fiber sequence to last proposition for all $n:\N$.
\end{proof}

\begin{lemma}
  \label{mayer-vietoris-sequence}
  Let $\mathcal F$ be a spectrum on $X$ and assume we have a pushout square of spaces
  \begin{center}
    \begin{tikzcd}
      \mathrm{S}\arrow[r, "f"]\arrow[d, " g"] & U\arrow[d, " h"] \\
      \mathrm{V}\arrow[r, " k"] & X
    \end{tikzcd}
  \end{center}
  Then we have a Mayer-Vietoris sequence:
  \begin{center}
    \begin{tikzcd}
      & \dots\ar[r] & H^{n-1}(S, f^* h^* \mathcal F)\ar[dll] \\
      H^n(X,\mathcal F)\ar[r] & H^n(U, h^*\mathcal F)\oplus H^n(V, k^*\mathcal F)\ar[r] & H^n(S, f^* h^*\mathcal F)\ar[dll] \\
      H^{n+1}(X,\mathcal F)\ar[r] & \dots &
    \end{tikzcd}
  \end{center}
\end{lemma}
 
\begin{proof}
  The square
  \begin{center}
    \begin{tikzcd}
      \prod\mathcal F\arrow[d]\arrow[r] & \prod h^*\mathcal F\arrow[d] \\
      \prod k^*\mathcal F\arrow[r] & \prod f^* h^*\mathcal F
    \end{tikzcd}
  \end{center}
  is a pullback by \cite[Proposition 2.1.6]{egbert-thesis}.
  This can be transformed to the following pullback square:
  \begin{center}
    \begin{tikzcd}
      \prod\mathcal F\arrow[d]\arrow[r] & \prod f^* h^*\mathcal F\arrow[d,"\Delta"] \\
      \prod k^*\mathcal F\times \prod h^*\mathcal F\arrow[r,"\times"] & \prod f^* h^*\mathcal F\times\prod f^* h^*\mathcal F
    \end{tikzcd}
  \end{center}
  
  By \cite[Lemma 3.3.6]{wellen-thesis} and the weak group structure on $\prod f^* h^* \Omega^{-n}\mathcal F$,
  we have a pullback square for each $n:\N$:
  \begin{center}
    \begin{tikzcd}
      \prod f^* h^* \Omega^{-n}\mathcal F\arrow[r]\arrow[d] & 1\arrow[d] \\
      (\prod f^* h^* \Omega^{-n}\mathcal F) \times (\prod f^* h^* \Omega^{-n}\mathcal F) \arrow[r, "-"] & \prod f^* h^* \Omega^{-n}\mathcal F
    \end{tikzcd}
  \end{center}
  Pasting gives a fiber-square:
  \begin{center}
    \begin{tikzcd}
      (\prod k^*\Omega^{-n}\mathcal F)\times(\prod h^*\Omega^{-n}\mathcal F)\arrow[d]  & \prod \Omega^{-n}\mathcal F\arrow[l]\arrow[d] \\
      (\prod f^* h^* \Omega^{-n}\mathcal F) \times (\prod f^* h^* \Omega^{-n}\mathcal F) \arrow[d, "-"] & \prod f^* h^* \Omega^{-n}\mathcal F\arrow[l]\arrow[d] \\
      \prod f^* h^* \Omega^{-n}\mathcal F  & 1\arrow[l] 
    \end{tikzcd}
  \end{center}
  So we get the desired fiber long exact sequence again by taking the long exact sequence of homotopy groups.
\end{proof}
