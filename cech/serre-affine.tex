\rednote{Give some context, i.e. important theorem in the foundations of
algebraic geometry, useful for X, Y, Z, ...}

\rednote{Discuss refined theorem where we only requiring vanishing of $H^1$ for
wqc bundles of ideals (instead of wqc bundles of modules)}

\rednote{Discuss slight apparent mismatch to classical version regarding
finiteness hypotheses}

\rednote{Discuss relative version}

\begin{lemma}\label{enough-functions-affine}
  Let~$X$ be a scheme. If there are global functions~$f_1,\ldots,f_n : X \to R$
  such that
  \begin{enumerate}
    \item the open subschemes $D(f_i)$ are all affine and
    \item the functions~$f_i$ generate the unit ideal in the ring~$R^X$,
  \end{enumerate}
  then~$X$ is affine.
\end{lemma}

\begin{proposition}
  Let~$X$ be a scheme. If~$H^1(X, E) = 0$ for all wqc bundles~$E$ on~$X$, then
  there are functions as in~\cref{enough-functions-affine}.
\end{proposition}

\begin{lemma}
  Let~$X$ be a scheme. Assume that~$H^1(X, E) = 0$ for all wqc bundles~$E$
  on~$X$. Let~$X = U \cup V$ be an open covering with~$U$ affine. Then there
  merely is a function~$f : X \to R$ such that~$D(f) \subseteq U$ and such
  that~$X = D(f) \cup V$.
\end{lemma}

\begin{theorem}
  Let~$X$ be a scheme. If~$H^1(X, E) = 0$ for all wqc bundles~$E$
  on~$X$, then~$X$ is affine.
\end{theorem}
