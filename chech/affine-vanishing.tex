
Let $R$ be a fixed commutative ring,
serving as a base ring for the definitions from the preprint \cite{draft},
we will now import:

\begin{definition}
  Let $A$ be an $R$-algebra.
  \begin{enumerate}[(a)]
  \item For $f:R$ let
    \[
      D(f):\equiv \text{$f$ is invertible}
    \]
    be the proposition that $f$ has a multiplicative inverse.
  \item A subtype $U:X\to \Prop$ of any type $X$ is open,
    if for all $x:X$, there merely are $f_1,\dots,f_n$ such that $U(x)=D(f_1)\vee \dots \vee D(f_n)$.
  \item The type
    \[
      \Spec A :\equiv \Hom_{R}(A,R)
    \]
    of $R$-algebra homomorphisms is called the \notion{spectrum of $A$}
    and there is a correspondence with external affine spectra in the Zariski-topos.
  \item A scheme is a type $X$ which is covered by finitely many open affine subtypes.
    These schemes are expected to correspond to
    external quasi compact, quasi separated schemes, locally of finite type.
  \end{enumerate}
\end{definition}

\begin{definition}
  Let $M$ be an $R$-module.
  $M$ is \notion{weakly quasi coherent}, if the canonical $R$-linear map
  \[
    \frac{m}{f^k}\mapsto ((\_: f\text{ inv})\mapsto f^{-k}\frac{m}{f^k}):M_f\to M^{D(f)}
  \]
  is an isomorphism.
  We denote the type of weakly quasi coherent $R$-modules with $\Mod{R}_{\text{wqc}}$.
\end{definition}

\begin{theorem}
  \label{affine-vanishing}
Let $X = \Spec(A)$ be an affine scheme, 
$M : X \to \Mod{R}_{\text{wqc}}$ a family of weakly quasi-coherent $R$-modules,
and $n > 0$.
Then we have
	\[
		H^n(X;M) = 0.
		\]
\end{theorem}

\begin{proof}
We induct on $n$. The base case $n = 1$ is \cite{draft}[Theorem 8.3.6].
Thus suppose $n \ge 2$ and that the theorem holds for all $0 < l < n$ and
any $X,M$.

Let $\chi : (x : X) \to K(M_x, n)$ represent a cohomology class.
We wish to show $\propTrunc{\chi = 0}$, a proposition.
We know that $\propTrunc{\chi(x) = 0}$ for all $x$, since $K(M_x,n)$ is connected.
By Zariski choice, we obtain a covering
$X = \bigcup_{i \in [m]} U_i$,
such that $\chi(x) = 0$ for $x \in U_i$, and
such that the proposition $x \in U_i$ is standard open for each $x$, $i$.
For $x : X$, let $I_x \coloneqq (i : [m]) \times (x \in U_i)$.
Note that $I_x$ is an affine scheme, since affine schemes are closed under finite
coproducts.

Since $\chi(x) = 0$ when $x \in U_i$, the image of $\chi(x)$ under the diagonal
map $K(M_x,n) \to K(M_x,n)^{I_x}$ is zero.
This diagonal map can be factored as $K(M_x,n) \to K(M_x^{I_x},n) \to K(M_x,n)^{I_x}$,
where the first map is induced by the diagonal $\Delta_x : M_x \to M_x^{I_x}$,
and the second is given by the equivalence
$M_x^{I_x} \simeq \Omega^n(K(M_x,n)^{I_x})$.
We claim that $\chi(x)$ maps to zero already in $K(M_x^{I_x},n)$.
To this end, it suffices to show that
$K(M_x^{I_x},n) \to K(M_x,n)^{I_x}$ is an embedding.
Since the domain is connected, it suffices to show that this map becomes
an equivalence after applying $\Omega$.
So we need to show that the canonical map
$K(M_x^{I_x},n-1) \to K(M_x,n-1)^{I_x}$ is an equivalence.
It becomes an equivalence after applying $\Omega^{n-1}$, and
the domain is $(n-2)$-connected, so by Whitehead's principle it suffices 
to show that the codomain is also $(n-2)$-connected.
Since $\pi_j(K(M_x,n-1)^{I_x}) = H^{n-1-j}(I_x;M_x)$,
it suffices to show that $H^l(I_x;M_x) = 0$ for
$0 < l \le n-1$. This follows from induction hypothesis
(using that $I_x$ is an affine scheme).

From this we can conclude that $\chi$ maps to zero in
$H^n(X;M_x^{I_x})$.
Since $I_x$ is merely inhabited, $\Delta_x$ is an embedding.
Hence we have a short exact sequence
$0 \to M_x \to M_x^{I_x} \to \coker \Delta_x \to 0$.
This induces a long exact sequence on cohomology.
One part of this long exact sequence is
$H^{n-1}(X;\coker\Delta_x)) \to H^n(X;M_x) \to H^n(X;M_x^{I_x})$.
By inductive hypothesis, $H^{n-1}(X;\coker\Delta_x) = 0$
(using that weakly quasi-coherent modules are closed under
 cokernels of monomorphisms, finite products, and exponentiation
 with standard opens).
Hence $H^n(X;M_x)$ embeds in $H^n(X;M_x^{I_x})$, so
$\chi$ must already have been zero in $H^n(X;M_x)$, as needed.
\end{proof}

One should be able to follow the above reasoning to show also vanishing of
$H^1$, provided we know that $H^0$ is right exact.

