% latexmk -pdfxe -pvc main.tex
% latexmk -pdfxe -pvc -interaction=nonstopmode main.tex
\documentclass{../util/zariski}

\title{\v{C}ech Cohomology in Homotopy Type Theory}

\author{Ingo Blechschmidt, Felix Cherubini and David Wärn}

\begin{document}

\maketitle

\begin{center}
  \color{purple}
  \large{Press CRTL+F5 to clear cached versions}
  \large{(if you are viewing these notes online)}
\end{center}

\tableofcontents

\section*{Introduction}

Algebraic geometry is the study of solutions of non-linear equations using methods from geometry.
Most prominently, algebraic geometry was essential in the proof of Fermat's last theorem by Wiles and Taylor.
The central geometric objects in algebraic geometry are called \emph{schemes}.
Their basic building blocks are called \emph{affine schemes},
where, informally, an affine scheme corresponds to a solution sets of polynomial equations.
While this correspondence is clearly visible in the functorial approach to algebraic geometry and our synthetic approach,
it is somewhat obfuscated in the most commonly used, topological appraoch.

In recent years,
formalization of the intricate notion of affine schemes
received some attention as a benchmark problem
-- this is, however, \emph{not} a problem addressed by this work.
Instead, we use a synthetic approach to algebraic geometry,
very much alike to that of synthetic differential geometry.
This means, while a scheme in classical algebraic geometry is a complicated compound datum,
we work in a setting where schemes are types,
with an additional property that can be defined within our synthetic theory.

Following ideas of Ingo Blechschmidt and Anders Kock  (\cite{ingo-thesis}, \cite{kock-sdg}[I.12]),
we use a base ring $R$, which is local and satisfies an axiom reminiscent of the Kock-Lawvere axiom.
A more general axiom, is called \emph{synthetic quasi coherence (SQC)} by Blechschmidt and
a version quatifying over external algbras is called the \emph{comprehensive axiom}\footnote{
  In \cite{kock-sdg}[I.12], Kock's ``axiom $2_k$'' could equivalently be Theorem 12.2,
  which is exactly our synthetic quasi coherence axiom, except that it only quantifies over external algebras.
}
by Kock.
The exact concise form of the SQC axiom we use, was noted by David Jaz Myers in 2018 and communicated to the first author.

Before we state the SQC axiom, let us take a step back and look at the basic objects of study in algebraic geometry,
solutions of polynomial equations.
Given a system of polynomial equations
\begin{align*}
  p_1(X_1, \dots, X_n) &= 0\rlap{,} \\
  \vdots\quad\quad\;\;   \\
  p_m(X_1, \dots, X_n) &= 0\rlap{,}
\end{align*}
the solution set
$\{ x : R^n \mid \forall i.\; p_i(x_1, \dots, x_n) = 0 \}$
is in canonical bijection to the set of $R$-algebra homomorphisms
\[ \Hom_R(R[X_1, \dots, X_n]/(p_1, \dots, p_m), R) \]
by identifying a solution $(x_1,\dots,x_n)$ with the homomorphism that maps each $X_i$ to $x_i$.
Conversely, for any $R$-algebra $A$, which is merely of the form $R[X_1, \dots, X_n]/(p_1, \dots, p_m)$,
we define the \emph{spectrum} of $A$ to be
\[
  \Spec A \colonequiv \Hom_R(A, R)
  \rlap{.}
\]
In contrast to classical, non-synthetic algebraic geometry,
where this set needs to be equipped with additional structure,
we postulate axioms that will ensure that $\Spec A$ has the expected geometric properties.
Namely, SQC is the statement that, for all finitely presented $R$-algebras $A$, the canonical map
  \begin{align*}
    A&\xrightarrow{\sim} (\Spec A\to R) \\
    a&\mapsto (\varphi\mapsto \varphi(a))
  \end{align*}
is an equivalence.
A prime example of a spectrum is $\A^1\colonequiv \Spec R[X]$,
which turns out to be the underlying set of $R$.
With the SQC axiom,
\emph{any} function $f:\A^1\to \A^1$ is given as a polynomial with coefficients in $R$.
In fact, all functions between affine schemes are given by polynomials.
Furthermore, for any affine scheme $\Spec A$,
the axiom ensures that
the algebra $A$ can be reconstructed as the algebra of functions $\Spec A \to R$,
therefore establishing a duality between affine schemes and algebras.

The Kock-Lawvere axiom used in synthetic differential geometry,
might be stated as the SQC axiom restricted to (external) \emph{Weil-algebras},
whose spectra correspond to pointed infinitesimal spaces.
These spaces can be used in both, synthetic differential and algebraic geometry,
in very much the same way.

In the accompanying formalization \cite{formalization} of some basic results,
we use a setup which was already proposed by David Jaz Myers
in a conference talk (\cite{myers-talk1, myers-talk2}).
On top of Myers' ideas,
we were able to define schemes, develop some topological properties of schemes,
and construct projective space.

An important, not yet formalized result
is the construction of cohomology groups.
This is where the \emph{homotopy} type theory really comes to bear --
instead of the hopeless adaption of classical, non-constructive definitions of cohomology,
we make use of higher types,
for example the $k$-th Eilenberg-MacLane space $K(R,k)$ of the group $(R,+)$.
As an analogue of classical cohomology with values in the structure sheaf,
we then define cohomology with coefficients in the base ring as:
\[
  H^k(X,R):\equiv \|X\to K(R,k)\|_0
  \rlap{.}
\]
This definition is very convenient for proving abstract properties of cohomology.
For concrete calculations we make use of another axiom,
which we call \emph{Zariski-local choice}.
While this axiom was conceived of for exactly these kind of calculations,
it turned out to settle numerous questions with no apparent connection to cohomology.
One example is the equivalence of two notions of \emph{open subspace}.
A pointwise definition of openness was suggested to us by Ingo Blechschmidt and
is very convenient to work with.
However, classically, basic open subsets of an affine scheme are given
by functions on the scheme and the corresponding open is morally the collection of points where the function does not vanish.
With Zariski-local choice, we were able to show that these notions of openness agree in our setup.

Apart from SQC, locality of the base ring $R$ and Zariski-local choice,
we only use homotopy type theory, including univalent universes, truncations and some very basic higher inductive types.
Roughly, Zariski-local choice states, that any surjection into an affine scheme merely has sections on a \emph{Zariski}-cover.
The latter, internal, notion of cover corresponds quite directly to the covers in the site of the \emph{Zariski topos},
which we use to construct a model of homotopy type theory with our axioms.

More precisely, we can use the \emph{Zariski topos} over any base ring.
Toposes built using other Grothendieck topologies, like for example the étale topology, are not compatible with Zariski-local choice.
We did not explore whether an analogous setup can be used for derived algebraic geometry
\footnote{Here, the word ``derived'' refers to the rings the algebraic geometry is built up from -- instead of the 0-truncated rings we use, ``derived'' algebraic geometry would use simplicial or spectral rings.
  Sometimes, ``derived'' refers to homotopy types appearing in ``the other direction'', namely as the values of the sheaves that used.
  In that direction, our theory is already derived, since we use homotopy type theory.
  Practically that means that we expect no problems when expanding our theory of synthetic schemes to what classic algebraic geometers
  call ``stacks''.
}
-- meaning that the 0-truncated rings we used are replaced by higher rings.
This is only because for a derived approach, would have to work with higher monoids, which is currently infeasible
-- we are not aware of any obstructions for, say, an SQC axiom holding in derived algebraic geometry.

In total, the scope of our theory so far, includes quasi-compact, quasi-separated schemes of finite type over an arbitrary ring.
These are all finiteness assumptions, that were chosen for convenience and include examples like closed subspaces of projective space,
which we want to study in future work, as example applications.
So far, we know that basic internal constructions, like affine schemes, correspond to the correct classical external constructions.
This can be expanded using our model, which is of course also important to ensure the consistency of our setup.


\section*{Acknowledgements}
We use work from Ingo Blechschmidt's PhD thesis, section 18 as a basis.
This includes in particular the synthetic quasi-coherence axiom and the assumption that the base ring is local.
David Jaz Myers had the idea to use Blechschmidt's ideas in homotopy type theory
and presented his ideas 2019 at the workshop ``Geometry in Modal Homotopy Type Theory'' in Pittsburgh.
Myers ideas include the algebra-setup we used in our formalization.

In December 2022, there was a mini-workshop in Augsburg, which helped with the development of this work.
We thank Jonas Höfer and Lukas Stoll for spotting a couple of small errors.


\ignore{
  PROPOSED CONTENT:

  hott preliminaries
  the mayer vietoris sequence
  basic chech cohomological results
  * def of chech cohomology
  * vanishing of H^{n+1} for a space with acyclic cover with n pieces?
  example: synthetic algebraic geometry
  * repetition of basic facts on wqc modules
  * triviality of H^n
  direct chech cohomology
  * proof so far
  derived functors
  * universality of H^n on some schemes
  * universality of chech-H^n
  universality of H^n and chech-H^n/construction of local resolutions
  * for chech cohomology on a space with acyclic cover
  * for a scheme with cover with cohomologically trivial fibers and EM-Cohomology with values in wqc modules
  examples?
  * line bundles on P^n?
}


\section{The Mayer-Vietoris-Sequence}

TODO: This needs to be updated and made coherent. It might be good to remove connective spectra and the connective cover (not sure if they are still needed).

An $n$-th delooping of a pointed space $A$ which is also $(n-1)$-connected is unique and usually written as $B^nA$ or $K(A,n)$ and called an \notion{Eilenberg-MacLane space}.
We will just write $A_n$ for an $n$-th delooping.

It is known, that in HoTT, a (0-truncated) abelian group can be delooped arbitrarily often (\cite{LicataFinster}).

Contents of this section are from Mike Shulman's posts on the HoTT-Blog about cohomology,
Floris van Doorn's thesis (\cite{floris-thesis})[section 5.3]
and common knowledge in the field that is not written up,
with the possible exception of the Mayer-Vietoris-Sequence with non-constant coefficients (\cref{mayer-vietoris-sequence}).

Suppose we have a pointed type $A$ with delooping $A_k$ for any $k:\N$.
Then, analogous to the definition of the $k$-th homotopy group
\[ \pi_k:\equiv\|\Omega^kA \|_0 \]
one could define homotopy groups of \emph{negative} degree $-k$ by:
\[ \pi_{-k}:\equiv\|A_k \|_0 \]
Note that these will be trivial for any Eilenberg-MacLane spectrum, since for those, $A_{k+1}$ is $k$-connected for $k:\N$.
In general, spectra with trivial homotopy groups in negative degree are called \emph{connective}.
The result in this article is concerned with Eilenberg-MacLane spectra.

We will use spectra varying over a space as coefficients for cohomology,
which corresponds to the classical concept of parametrized spectra.
We fix our terminology in the following definition.

\begin{definition}
  \begin{enumerate}[(a)]
  \item A \notion{spectrum} is a sequence of pointed types $(A_k)_{k:\N}$, together with pointed equivalences $A_k\simeq \Omega A_{k+1}$.
  \item A spectrum $(A_k)_{k:\N}$ is \notion{connective}, if $\|A_{k+1}\|_0\simeq 1$ for all $k:\N$.
  \item Let $X$ be a type. A \notion{parametrized spectrum} over $X$, is a dependent function, which assigns to any $x:X$, a spectrum $(A_{x,k})_{k:\N}$.
    For brevity, We will call a parametrized spectrum $A\equiv x\mapsto (A_{x,k})_{k:\N}$ over $X$ just \notion{spectrum over $X$}.
  \item A \notion{morphism of spectra} $A,A'$ over $X$, is given by a sequence of pointed maps $f_{x,k}:A_{x,k}\to A_{x,k}'$ for any $x:X$,
    such that $\Omega f_{x,k+1}=f_{x,k}$ (using the pointed equivalences).
  \end{enumerate}
\end{definition}

The connective spectra form a nice ``subcategory'':
We will need the following (coreflective) construction that turns a spectrum into a connective spectrum.
See \cref{def:connected-cover} for the definition of the $k$-connected cover ``$D^k_Xd$''.

\begin{definition}
  For a spectrum $A$,
  the following construction is called the \notion{connective cover}:
  \[ \hat{A}:\equiv k\mapsto D_{A,k}^{k-1}\]
  There is also a sequence of pointed maps $\varphi_k:\hat{A}_k\to A_k$, given by the projection from the connected covers.
\end{definition}

The following fact will be useful to us on various occations and can be proven using the uniqueness of Eilenberg-MacLane spaces:

\begin{lemma}
  \label{em-comm-pi}
  Let $X$ be a type and $A:X\to\AbGroup$ a dependent abelian group.
  If for all $0<l\leq n$ the type $(x:X)\to K(A_x,l)$ is connected, then
  \[
    ((x:X)\to K(A_x,n)) = K((x:X)\to A_x,n)
    \rlap{.}
  \]
\end{lemma}

\begin{definition}
  The $k$-th cohomology group of $X$ with coefficients in $A$ is the following:
  \[ H^k(X,A):\equiv\left\|(x:X)\to A_{x,k}\right\|_0 \]
\end{definition}

\begin{remark}
  We could also directly construct an $R$-module structure on all cohomology groups,
  whenever the coefficients are $R$-modules.
  We will not do this, since the cases of interest will be covered with far less effort
  once the correspondence with \v{C}ech cohomology is established.
\end{remark}

\rednote{Add a disclaimer: pullback and push forward do not coincide with the classical constructions}

\begin{definition}
  Let $f:Y\to X$ be a map of types and $\mathcal F:X\to \AbGroup$ and $\mathcal G:Y\to \AbGroup$ dependent abelian groups.
  \begin{enumerate}[(a)]
  \item $f^\ast \mathcal F\colonequiv (y:Y)\mapsto \mathcal F_{f(y)}$ is called the \notion{pullback}\index{$f^\ast \mathcal F$} of $\mathcal F$ along $f$.
  \item $f_{\ast}\mathcal G\colonequiv (x:X)\mapsto (((y,\_):\fib_f(x)) \to \mathcal G_y)$ is called the \notion{push-forward}\index{$f_{\ast}\mathcal G$} of $\mathcal G$ along $f$.
  \end{enumerate}
\end{definition}

Cohomlogy commutes with finite coproducts:

\begin{lemma}
  \label{cohomology-copropduct-direct-sum}
  Let $Y_i$, $i:I\colonequiv \{1,\dots,l\}$ be types and $f_i:(i:I)\times Y_i\to X$ and $\mathcal F:((i:I)\times  Y_i)\to\AbGroup$. Then for all $n:\N$
  \[
    H^n((i:I)\times  Y_i, \mathcal F)=\bigoplus_i H^n(Y_i, f_i^\ast\mathcal F)
    \rlap{.}
  \]
  
\end{lemma}

\begin{proof}
  Direct by currying, using that $\|\_\|_0$ preserves finite products.
\end{proof}

Cohomology does not change under push-forward along maps with cohomologically trivial fibers:

\begin{lemma}
  \label{cohomologically-trivial-fibers}
  Let $f:Y\to X$ and $\mathcal F:Y\to\AbGroup$ be such that $H^l(\fib_f(x),\pi_1^\ast\mathcal F)=0$ for all $0<l\leq n$,
  then
  \[
    H^n(Y,\mathcal F)=H^n(X,f_\ast\mathcal F)
    \rlap{.}
  \]
\end{lemma}

\begin{proof}
  By direct application of \Cref{em-comm-pi}.
\end{proof}

An important notion in abelian categories, is that of short exact sequences.
And it is important to us here, since for every short exact sequence (somewhere), there should be an induced long exact sequence on cohomology groups.
The cokernel of an exact sequence, corresponds to a cofiber of a map of spectra.

\begin{definition}
  Let $f:A\to A'$ be a map of spectra.
  \begin{enumerate}[(a)]
  \item The \notion{cofiber} of $f$ is given by the spectrum
    \[ C_{f,k}:\equiv \fib_{f_{k+1}}\]
    together with the map $c:A'\to C_f$, where $c_k$ is induced in the following diagram of pullback-squares:
    \begin{center}
      \begin{tikzcd}
        A_k\ar[r,"f_k"]\ar[d] & A'_k\ar[d,dashed]\ar[r] & 1\ar[d] \\
        1\ar[r] & C_{f,k}\ar[r]\ar[d] & A_{k+1}\ar[d] \\
        & 1\ar[r] & A'_{k+1}
      \end{tikzcd}
    \end{center}
  \item The \notion{fiber} of $f$ is given by the spectrum
    \[ \fib_{f,k}:\equiv \fib_{f_k}\]
  \end{enumerate}
\end{definition}

Note that $f:A\to A'$ is always the fiber of its cofiber and conversely, $f:A\to A'$ is always the cofiber of its fiber, which is very different from the situation in a general abelian category,
where for example not every map is the kernel of its cokernel.

\begin{definition}
  A sequence of morphisms of spectra over $X$
  \begin{center}
    \begin{tikzcd}
      A\ar[r,"f"] & A'\ar[r,"g"] & A''
    \end{tikzcd}
  \end{center}
  is a \notion{fiber sequence}, if the following equivalent statements hold:
  \begin{enumerate}[(a)]
  \item $f_x$ is the fiber of $g_x$ for all $x:X$
  \item $f_{x,k}$ is the fiber of $g_{x,k}$ for all $x:X$ and $k:\N$
  \end{enumerate}
  If all spectra involved are Eilenberg-MacLane spectra, we call the sequence \notion{exact}, and vice versa,
  if we speak of a \notion{short exact sequence} of spectra (over $X$), we assume all spectra involved are Eilenberg-MacLane and we have a fiber sequence.
\end{definition}

\begin{lemma}
  If $A\to A'\to A''$ is a fiber sequence, then the induced square:
  \begin{center}
    \begin{tikzcd}
      \prod_{x:X}A_{x,k}\ar[r]\ar[d] & \prod_{x:X}A'_{x,k}\ar[d] \\
      1\ar[r] & \prod_{x:X}A''_{x,k}
    \end{tikzcd}
  \end{center}
  is a pullback square for all $k:\N$.
\end{lemma}
\begin{proof}
  $\prod$ maps families of pullback squares to a pullback square.
\end{proof}

This is just tailored to prove the following proposition:

\begin{proposition}
\label{EM-les}
  For any fiber sequence
  \[ A\to A'\to A'' \]
  of spectra over $X$, there is a long exact sequence of cohomology groups:
  \begin{center}
    \begin{tikzcd}
      & \dots\ar[r]& H^{n-1}(X,A'')\ar[dll]\\
      H^n(X,A)\ar[r] & H^n(X,A')\ar[r] & H^n(X,A'')\ar[lld] \\
      H^{n+1}(X,A)\ar[r] & H^{n+1}(X,A')\ar[r] & \dots
    \end{tikzcd}
  \end{center}
\end{proposition}

\begin{proof}
  Apply homotopy fiber sequence to last proposition for all $n:\N$.
\end{proof}

\begin{lemma}
  \label{mayer-vietoris-sequence}
  Let $\mathcal F$ be a spectrum on $X$ and assume we have a pushout square of spaces
  \begin{center}
    \begin{tikzcd}
      \mathrm{S}\arrow[r, "f"]\arrow[d, " g"] & U\arrow[d, " h"] \\
      \mathrm{V}\arrow[r, " k"] & X
    \end{tikzcd}
  \end{center}
  Then we have a Mayer-Vietoris sequence:
  \begin{center}
    \begin{tikzcd}
      & \dots\ar[r] & H^{n-1}(S, f^* h^* \mathcal F)\ar[dll] \\
      H^n(X,\mathcal F)\ar[r] & H^n(U, h^*\mathcal F)\oplus H^n(V, k^*\mathcal F)\ar[r] & H^n(S, f^* h^*\mathcal F)\ar[dll] \\
      H^{n+1}(X,\mathcal F)\ar[r] & \dots &
    \end{tikzcd}
  \end{center}
\end{lemma}
 
\begin{proof}
  The square
  \begin{center}
    \begin{tikzcd}
      \prod\mathcal F\arrow[d]\arrow[r] & \prod h^*\mathcal F\arrow[d] \\
      \prod k^*\mathcal F\arrow[r] & \prod f^* h^*\mathcal F
    \end{tikzcd}
  \end{center}
  is a pullback by \cite[Proposition 2.1.6]{egbert-thesis}.
  This can be transformed to the following pullback square:
  \begin{center}
    \begin{tikzcd}
      \prod\mathcal F\arrow[d]\arrow[r] & \prod f^* h^*\mathcal F\arrow[d,"\Delta"] \\
      \prod k^*\mathcal F\times \prod h^*\mathcal F\arrow[r,"\times"] & \prod f^* h^*\mathcal F\times\prod f^* h^*\mathcal F
    \end{tikzcd}
  \end{center}
  
  By \cite[Lemma 3.3.6]{wellen-thesis} and the weak group structure on $\prod f^* h^* \Omega^{-n}\mathcal F$,
  we have a pullback square for each $n:\N$:
  \begin{center}
    \begin{tikzcd}
      \prod f^* h^* \Omega^{-n}\mathcal F\arrow[r]\arrow[d] & 1\arrow[d] \\
      (\prod f^* h^* \Omega^{-n}\mathcal F) \times (\prod f^* h^* \Omega^{-n}\mathcal F) \arrow[r, "-"] & \prod f^* h^* \Omega^{-n}\mathcal F
    \end{tikzcd}
  \end{center}
  Pasting gives a fiber-square:
  \begin{center}
    \begin{tikzcd}
      (\prod k^*\Omega^{-n}\mathcal F)\times(\prod h^*\Omega^{-n}\mathcal F)\arrow[d]  & \prod \Omega^{-n}\mathcal F\arrow[l]\arrow[d] \\
      (\prod f^* h^* \Omega^{-n}\mathcal F) \times (\prod f^* h^* \Omega^{-n}\mathcal F) \arrow[d, "-"] & \prod f^* h^* \Omega^{-n}\mathcal F\arrow[l]\arrow[d] \\
      \prod f^* h^* \Omega^{-n}\mathcal F  & 1\arrow[l] 
    \end{tikzcd}
  \end{center}
  So we get the desired fiber long exact sequence again by taking the long exact sequence of homotopy groups.
\end{proof}


\section{\v{C}ech cohomology}

In this section, let $X$ be a type, $U_1,\dots,U_n\subseteq X$ open subtypes that cover $X$
and $\mathcal F:X\to \AbGroup$ a dependent abelian group on $X$.
We start by repeating the classical definition of \v{C}hech-Cohomology groups for a given cover.

\begin{definition}%
  \label{chech-complex}
  \begin{enumerate}[(a)]
  \item \index{$\mathcal F(U)$} For open $U\subseteq X$, we use the notation
    \[
      \mathcal F(U)\colonequiv \prod_{x:U}\mathcal F_x\rlap{.}
    \]
  \item For $s:\mathcal F(U)$ and open $V\subseteq U$ we use the notation $s\colonequiv s_{|V} \colonequiv (x:V)\mapsto s_x$.
  \item \index{$U_{i_1\dots i_l}$}For a selection of indices $i_1,...,i_l:\{1,\dots,n\}$, we use the notation
    \[
      U_{i_1\dots i_l}\colonequiv U_{i_1}\cap\dots\cap U_{i_l}\rlap{.}
    \]
  \item For a list of indices $i_1,\dots,i_l$, let $i_1,\dots,\hat{i_t},\dots,i_l$ be the same list with the $t$-th element removed.
  \item For $k:\Z$, the $k$-th \notion{Čech-boundary operator}\index{$\partial^k$} is the homomorphism
    \[
      \partial^k:\bigoplus_{i_0,\dots,i_k}\mathcal F(U_{i_0\dots i_k})\to \bigoplus_{i_0,\dots,i_{k+1}}\mathcal F(U_{i_0\dots i_{k+1}})
    \]
    given by $\partial^k(s)\colonequiv (l_0,\dots,l_{k+1}) \mapsto \sum_{j=0}^k (-1)^j s_{l_0,\dots,\hat{l_j},\dots,l_k|U_{l_0,\dots,l_{k+1}}}$.
  \item The $k$-th \notion{Čech-Cohomology group} for the cover $U_1,\dots,U_n$ with coefficients in $\mathcal F$ is
    \[
      \check{H}^k(\{U\},\mathcal F)\colonequiv \ker\partial^{k} / \im(\partial^{k-1})\rlap{.}
    \]
  \end{enumerate}
\end{definition}

\begin{definition}
  The cover $U_1,\dots,U_n$ is called \notion{acyclic} for $\mathcal F$,
  if for all $k:\N$ and $i_0,\dots,i_k$, we have that the higher (non Čech) cohomology groups are trivial:
  \[
    \forall l>0. H^l(U_{i_0,\dots,i_k},\mathcal F)=0\rlap{.}
  \]
\end{definition}


\section{Cohomology of affine schemes}

Let $R$ be a fixed commutative ring,
serving as a base ring for the definitions from the preprint \cite{draft},
we will now import:

\begin{definition}
  Let $A$ be an $R$-algebra.
  \begin{enumerate}[(a)]
  \item For $r:R$ let
    \[
      D(r):\equiv \text{$r$ is invertible}
    \]
    be the proposition that $r$ has a multiplicative inverse.
  \item A subtype $U:X\to \Prop$ of any type $X$ is open,
    if for all $x:X$, there merely are $r_1,\dots,r_n$ such that $U(x)=D(r_1)\vee \dots \vee D(r_n)$.
  \item The type
    \[
      \Spec A :\equiv \Hom_{R}(A,R)
    \]
    of $R$-algebra homomorphisms is called the \notion{spectrum of $A$}
    and there is a correspondence with external affine spectra in the Zariski-topos.
  \item A scheme is a type $X$ which is covered by finitely many open affine subtypes.
    These schemes are expected to correspond to
    external quasi-compact, quasi-separated schemes, locally of finite presentation.
  \end{enumerate}
\end{definition}

\begin{definition}
  Let $M$ be an $R$-module.
  $M$ is \notion{weakly quasi-coherent}, if the canonical $R$-linear map
  \[
    \frac{m}{r^k}\mapsto ((\_: r\text{ inv})\mapsto r^{-k}m):M_r\to M^{D(r)}
  \]
  is an isomorphism for all $r:R$.
  We denote the type of weakly quasi-coherent $R$-modules with $\Mod{R}_{\text{wqc}}$.
\end{definition}

\begin{remark}
  Let $M:X\to \Mod{R}_{\mathrm{wqc}}$.
  Then for any $f:X\to R$ there are isomorphisms of $R^X$-modules\index{$M(D(f))$}
  \[
  M(D(f))\colonequiv \prod_{x:D(f)}M_x  = \prod_{x:X}M_{f(x)} = \prod_{x:X}M^{D(f(x))}
  \rlap{.}
  \]
\end{remark}

\begin{proof}
  \cite{draft}[Lemma 7.1.4].
\end{proof}

\begin{theorem}
  \label{affine-vanishing}
Let $X = \Spec(A)$ be an affine scheme, 
$M : X \to \Mod{R}_{\text{wqc}}$ a family of weakly quasi-coherent $R$-modules,
and $n > 0$.
Then we have
	\[
		H^n(X;M) = 0.
		\]
\end{theorem}

\begin{proof}
We induct on $n$. The base case $n = 1$ is \cite{draft}[Theorem 8.3.6].
Thus suppose $n \ge 2$ and that the theorem holds for all $0 < l < n$ and
any $X,M$.

Let $\chi : (x : X) \to K(M_x, n)$ represent a cohomology class.
We wish to show $\propTrunc{\chi = 0}$, a proposition.
We know that $\propTrunc{\chi(x) = 0}$ for all $x$, since $K(M_x,n)$ is connected.
By Zariski choice, we obtain a covering
$X = \bigcup_{i \in [m]} U_i$,
such that $\chi(x) = 0$ for $x \in U_i$, and
such that the proposition $x \in U_i$ is standard open for each $x$, $i$.
For $x : X$, let $I_x \coloneqq (i : [m]) \times (x \in U_i)$.
Note that $I_x$ is an affine scheme, since affine schemes are closed under finite
coproducts.

Since $\chi(x) = 0$ when $x \in U_i$, the image of $\chi(x)$ under the diagonal
map $K(M_x,n) \to K(M_x,n)^{I_x}$ is zero.
This diagonal map can be factored as $K(M_x,n) \to K(M_x^{I_x},n) \to K(M_x,n)^{I_x}$,
where the first map is induced by the diagonal $\Delta_x : M_x \to M_x^{I_x}$,
and the second is given by the equivalence
$M_x^{I_x} \simeq \Omega^n(K(M_x,n)^{I_x})$ (\rednote{Cite David's preprint?}).
We claim that $\chi(x)$ maps to zero already in $K(M_x^{I_x},n)$.
To this end, it suffices to show that
$K(M_x^{I_x},n) \to K(M_x,n)^{I_x}$ is an embedding.
Since the domain is connected, it suffices to show that this map becomes
an equivalence after applying $\Omega$.
So we need to show that the canonical map
$K(M_x^{I_x},n-1) \to K(M_x,n-1)^{I_x}$ is an equivalence.
It becomes an equivalence after applying $\Omega^{n-1}$, and
the domain is $(n-2)$-connected, so by Whitehead's principle it suffices 
to show that the codomain is also $(n-2)$-connected.
Since $\pi_j(K(M_x,n-1)^{I_x}) = H^{n-1-j}(I_x;M_x)$,
it suffices to show that $H^l(I_x;M_x) = 0$ for
$0 < l \le n-1$. This follows from induction hypothesis
(using that $I_x$ is an affine scheme).

From this we can conclude that $\chi$ maps to zero in
$H^n(X;M_x^{I_x})$.
Since $I_x$ is merely inhabited, $\Delta_x$ is an embedding.
Hence we have a short exact sequence
$0 \to M_x \to M_x^{I_x} \to \coker \Delta_x \to 0$.
This induces a long exact sequence on cohomology.
One part of this long exact sequence is
$H^{n-1}(X;\coker\Delta_x) \to H^n(X;M_x) \to H^n(X;M_x^{I_x})$.
By inductive hypothesis, $H^{n-1}(X;\coker\Delta_x) = 0$
(using that weakly quasi-coherent modules are closed under
 cokernels of monomorphisms, finite products, and exponentiation
 with standard opens).
Hence $H^n(X;M_x)$ embeds in $H^n(X;M_x^{I_x})$, so
$\chi$ must already have been zero in $H^n(X;M_x)$, as needed.
\end{proof}

One should be able to follow the above reasoning to show also vanishing of
$H^1$, provided we know that $H^0$ is right exact.



\section{\v{C}ech cohomology of a join}
\begin{definition}
The join $X * Y$ of two types $X$, $Y$ is given by the following pushout.
\[
\begin{tikzcd}
X \times Y \arrow[r] \arrow[d]
	& Y \arrow[d] \\
	X \arrow[r] &
	X * Y 
	\arrow[ul, phantom, "\ulcorner" , very near start]
\end{tikzcd}
\]
\end{definition}

Let $n$ be a natural number and $P_1, \ldots, P_n$ types.
We define the join $P_1 * \cdots * P_n$ by induction on $n$,
so that it is empty if $n = 0$ 
and $P_1 * (P_2 * \cdots * P_n)$ if $n > 1$.
Our goal is to describe a precise sense in which
this join is built from the products
$\Pi_{i : I} P_i$ where $I \subseteq [n]$ ranges over detachable,
inhabited subsets.
Note that if $P_i$ are all propositions, then so is the join, with
$P_1 * \cdots * P_n = P_1 \vee \cdots \vee P_n$.

\begin{definition}
We define a sequence $J_{-1} \to J_0 \to J_1 \to \cdots$ of types.
If $n = 0$, we take $J_r = \varnothing$ for all $r$.
If $n > 0$, let $\widehat J_r$ be the sequence obtained recursively from
the types $P_2, \ldots, P_n$.
We take $J_{-1} = \varnothing$ and for $r \ge 0$ define $J_r$ by the following
pushout diagram.
\[
\begin{tikzcd}
P_1 \times \widehat J_{r-1} \arrow[r] \arrow[d]
	& P_1 \arrow[d] \\
	\widehat J_r \arrow[r] &
	J_r 
	\arrow[ul, phantom, "\ulcorner" , very near start]
\end{tikzcd}
\]
The map $J_r \to J_{r+1}$ is induced by functoriality of pushouts via the following
commutative diagram.
\[
\begin{tikzcd}
	\widehat J_{r-1} \arrow[d] &
	P_1 \times \widehat J_{r-2} \arrow[l] \arrow[d] \arrow[r] &
	P_1 \arrow[d] \\
	\widehat J_r &
	P_1 \times \widehat J_{r-1} \arrow[l] \arrow[r] &
	P_1 \\
\end{tikzcd}
\]
\end{definition}

\begin{lemma}
For $r \ge n-1$, the map $J_r \to J_{r+1}$ is an equivalence and
$J_r \simeq P_1 * \cdots * P_n$ is the join.
\end{lemma}
\begin{proof}
Direct by induction on $n$.
\end{proof}

\begin{definition}
For $r$ a natural number, let
$[n]^{(r)}$ denote the type of $r$-element subsets of $[n]$, and
define
\[
Z_r \coloneqq (I : [n]^{(r)}) \times (i : I) \to P_i.
\]
\end{definition}
\begin{lemma}\label{cw}
For $r \ge 0$, we have a pushout square of the following form.
\[
\begin{tikzcd}
Z_{r+1} \times S^{r-1} \arrow[r] \arrow[d]
	& Z_{r+1} \arrow[d] \\
	J_{r-1} \arrow[r] &
	J_r
	\arrow[ul, phantom, "\ulcorner" , very near start]
\end{tikzcd}
\]
\end{lemma}
That is, $J_r$ is obtained from $J_{r-1}$ by attaching $Z_{r+1}$-many $r$-cells.
\begin{proof}
We induct on $n$. For $n = 0$, $Z_{r+1}$ is empty and so there is nothing to prove.
For $r = 0$ the conclusion is also clear.
Thus suppose $n > 0$, $r > 0$ and that the lemma holds for the sequence $P_2, \ldots, P_n$.
Consider the following 3-by-3-diagram, with the pushouts of the rows and columns listed
at the bottom and to the right.
\[
\begin{tikzcd}
 P_1 &
 P_1 \times \widehat Z_r \arrow[l] \arrow[r,equal]&
 P_1 \times \widehat Z_r &
 P_1 \\
 P_1 \times \widehat J_{r-2} \arrow[u] \arrow[d] &
 P_1 \times \widehat Z_r \times S^{r-2} \arrow[l] \arrow[r] \arrow[u] \arrow[d] &
 P_1 \times \widehat Z_r  \arrow[u,equal] \arrow[d] &
 P_1 \times \widehat J_{r-1} \arrow[u] \arrow[d] \\
 \widehat J_{r-1} &
 \widehat Z_{r+1} \times S^{r-1} + P_1 \times \widehat Z_r  \arrow[l] \arrow[r] &
 \widehat Z_{r+1} + P_1 \times \widehat Z_r &
 \widehat J_r \\
	 J_{r-1} & 
	 Z_{r+1} \times S^{r-1} \arrow[l] \arrow[r] & 
	 Z_{r+1} & J_r
\end{tikzcd}
\]
The maps in this diagram are all guessable, and the commutativity of each of
the four squares is direct. We explain how to compute the pushout of each row and column.
The pushout of the first row is $P_1$, since
the pushout of any equivalence is an equivalence.
The pushout of the second row is $P_1 \times \widehat J_{r-1}$,
by inductive hypothesis and using that $P_1 \times -$ preserves pushouts.
The pushout of the third row is $\widehat J_r$, again using inductive hypothesis
as well as the observation that the $P_1 \times \widehat Z_r$-terms do not
affect the pushout.

The pushout of the first column is $J_{r-1}$ by definition.
To compute the pushout of the second column, we observe that the 
$\widehat Z_{r+1} \times S^{r-1}$-term does not interact with the rest of the column,
that the suspension of $S^{r-2}$ is $S^{r-1}$, and that $P_1 \times \widehat Z_r \times -$
preserves pushouts. All together, this shows that
the pushout is $\widehat Z_{r+1} \times S^{r-1} + P_1 \times \widehat Z_r \times S^{r-1}$,
i.e.\ $(\widehat Z_{r+1} + P_1 \times \widehat Z_r) \times S^{r-1}$,
i.e.\ $Z_{r+1} \times S^{r-1}$.
Finally, the third pushout is 
$\widehat Z_{r+1} + P_1 \times \widehat Z_r$ since the pushout of an equivalence
is an equivalence, i.e.\ $Z_{r+1}$.

The $3 \times 3$-lemma tells us that the the pushout of row-wise pushouts is equivalent
to the pushout of column-wise pushouts. That is, 
$J_r$ is a pushout of the desired form. (Here one should be careful to
		check that the maps are the ones we expect.)
\end{proof}

Now let $X$ be a type, $n$ a natural number, and $P_i$ a type family over
$X$ for each $i : [n]$. For any $x : X$,
$P_1(x), \ldots, P_n(x)$ is simply a list of types, to which we may apply
Lemma \ref{cw}.
Taking sigma over $x : X$ preserves pushouts (since it is a left adjoint),
	   so we obtain the following pushout square for each $r \ge 0$.
\[
\begin{tikzcd}
(x : X) \times Z_{r+1}(x) \times S^{r-1} \arrow[r] \arrow[d]
	& (x : X) \times Z_{r+1}(x) \arrow[d] \\
	(x : X) \times J_{r-1}(x) \arrow[r] &
	(x : X) \times J_r(x)
	\arrow[ul, phantom, "\ulcorner" , very near start]
\end{tikzcd}
\]

Now suppose $M$ is a family of abelian groups over $X$,
and consider the
corresponding Mayer--Vietoris sequence.
It can be seen directly by currying that
\[H^l(Y \times S^{r-1})
	\cong 
H^l(Y) \oplus
H^{l-r+1}(Y),\]
and that the action on cohomology of the projection $Y \times S^{r-1} \to Y$
corresponds to the inclusion of the left summand.
Moreover, $(x : X) \times Z_{r+1}$ is a finite coproduct, and
cohomology is additive in those, so we have
\[H^l((x : X) \times Z_{r+1}(x))
	\cong
	\bigoplus_{I : [n]^{(r+1)}} H^l((x : X) \times (i : I) \to P_i(x)).\]
Now suppose for simplicity that $P_i$ is acyclic with regard to $M$, so 
that \[H^l((x : X) \times (i: I) \to P_i(x)) = 0\] for $l > 0$ and any
$I : [n]^{(r)}$.
In this case, the Mayer--Vietoris sequence mostly degenerates, and
we are left with the following.

\begin{lemma}
In the situation above, we have
\[H^l((x : X) \times J_r(x)) \cong H^l((x : X) \times J_{r-1}(x))\]
for $l < r-1$ (where the map is induced by the map $J_{r-1}(x) \to J_r(x)$),
	\[H^{r-1}((x : X) \times J_r(x)) \cong \ker\,\delta\]
	and
	\[H^r((x : X) \times J_r(x)) \cong \coker\,\delta\]
	where $\delta : H^{r-1}((x : X) \times J_{r-1}(x)) \to H^0((x : X) \times Z_{r+1}(x))$
	is induced by the attaching map $Z_{r+1}(x) \times S^{r-1} \to J_r(x)$,
	and
\[H^l((x : X) \times J_r(x)) \cong 0\]
for $l > r$.
\end{lemma}
\begin{proof}
By induction on $r$ using Mayer--Vietoris.
\end{proof}
We expect one can show that the map $\delta$ is induced by the \v{C}ech boundary map
\[ \bigoplus_{I : [n]^{(r)}} H^0((x : X) \times (i : I) \to P_i(x))
	\to
	\bigoplus_{J : [n]^{(r+1)}} H^0((x : X) \times (j : J) \to P_j(x)). \]
Modulo this, we arrive at our main result.
\begin{theorem}
For any $P_1,\ldots,P_n$ which are acyclic with regard to $M$, the cohomology groups 
\[
H^l((x: X) \times P_1(x) * \cdots * P_n(x); M)
\]
of the fibrewise join with coefficients in $M$ agree with the cohomology of the \v{C}ech complex.
\end{theorem}
Note that if $P_i(x)$ are propositions, then the fibrewise join
is simply the union of subtypes, and the types 
$H^0((x : X) \times (i: I) \to P_i(x))$ appearing in the \v{C}ech complex
are simply intersections of subtypes.


\section{$\partial$-Functors}
Cohomology has the universal property of being a \notion{universal $\partial$-functor}.
In this section, we will construct a tool for proving this in some particular situations,
both for the cohomology defined using Eilenberg-MacLane spaces and \v{C}hech cohomology.

The following definition, from (\cite[2.1]{tohoku-translation}) and originally from (\cite{tohoku1957}), is specialized to our needs.
Grothendieck makes a definition for additive functors from an abelian category to a preadditive category.
We will only need the theory for functors from certain subcategories of dependent $R$-modules over a fixed type to abelian groups.
Also, some arguments are a lot more convenient when we can use elements of modules instead of abstract categorical language.
Therefore, we will state our definitions and results only for this particular situation.

\emph{Let $R$ be a fixed commutative ring and $\mathcal A$ be a subcategory of the category of dependent $R$-modules over a fixed type $X$, which is closed under finite direct sums.}

\begin{definition}
  An \notion{($l$-truncated) $\partial$-functor} is a collection of additve\footnote{
    The zero object and binary direct sums are preserved.
  }
  functors
  $T^i:\mathcal A\to \AbGroup$, where $0\leq i < l$ with $l\in\N\cup\{\infty\}$,
  together with a collection of connecting morphisms $\partial_{S,i}$ for any short exact sequence $S$ and $0\leq i\le l$, subject to the following conditions:
  \begin{enumerate}[(a)]
  \item Let $\mathcal{S}$ be a short exact sequence
    \[ 0\to A'\to A\to A''\to 0\]
    in $\mathcal A$. Applying the $T^i$ yields a complex, together with connecting morphisms $(\partial_{\mathcal{S},i})_{0\leq i<l-1}$:
    \begin{center}
      \begin{tikzcd}
        0\ar[r] & T^0(A')\ar[r] & T^0(A)\ar[r]  & T^0(A'')\ar[r,"\partial_{\mathcal{S},0}"]  & T^1(A')\ar[r]  & T^1(A)\ar[r]  & \dots
      \end{tikzcd}
    \end{center}
  \item For any homomorphism to a second short exact sequence
    \[ 0\to B'\to B\to B''\to 0\]
    and any $i<l-1$ the corresponding square commutes:
    \begin{center}
      \begin{tikzcd}
        T^i(A'')\ar[r,"\partial"]\ar[d] & T^{i+1}(A')\ar[d] \\
        T^i(B'')\ar[r,"\partial"] & T^{i+1}(B') \\
      \end{tikzcd}
    \end{center}
  \end{enumerate}
\end{definition}

\begin{definition}
  Let $l,k:\N$.
  The $l$-th \notion{truncation} of a $(l+k)$-truncated $\partial$-functor $T$
  is just the restriction of $(T^i)_{i<l+k}$ to $(T^i)_{i<l}$, together with a restriction of the $\partial$-maps
  and we denote the $l$-th truncation with \notion{$T^{\leq l}$}.
\end{definition}

\begin{definition}
  Let $T$ and $T'$ be $\partial$-functors defined for the same indices.
  
  A \notion{morphism of $\partial$-functors} $f:T\to T'$ is given by a natural transformation $f^i:T^i\to T'^i$ for each valid $i$,
  such that for any short exact sequence
  \[ 0\to A'\to A\to A''\to 0\]
  the following square commutes:
  \begin{center}
    \begin{tikzcd}
      T^i(A'')\ar[r,"\partial"]\ar[d,"f^i_{A''}"] & T^{i+1}(A')\ar[d,"f^{i+1}_{A'}"] \\
      T'^i(A'')\ar[r,"\partial"] & T'^{i+1}(A') 
    \end{tikzcd}
  \end{center}
\end{definition}

\begin{definition}
  A $\partial$-functor $T$ is called \notion{exact},
  if all values are exact complexes.
\end{definition}

\begin{definition}
  A $\partial$-functor $T$ is called \notion{universal}, if for any $T'$, defined for the same indices,
  any natural transformation $f^0:T^0\to T'^0$ extends uniquely to a morphism of $\partial$-functors $f:T\to T'$.
\end{definition}

To prove that some $\partial$-functor has this universal property,
we will \emph{extend} morphisms of $\partial$-functors, level by level.
By observing the diagram in the proof of the lemma below,
one can see that this is possible using exact sequences with the property,
that some particular element is zero in their middle term.
This property will appear often enough to deserve a name:

\begin{definition}
  Let $T$ be a $\partial$-functor, $i$ a valid index, $A:\mathcal A$ and $\chi:T^i(A)$.
  We say that a short exact sequence $\mathcal{S}=A\to R\to S$ \notion{resolves} $\chi$,
  if $\chi$ is mapped to zero in $T^i(R)$.
\end{definition}

In the classical approach with injective resolutions,
for a fixed $A:\mathcal A$ all elements of $T^i(A)$ for all $i>0$ would be resolved.
For our examples, where we can resolve elements of $H^i(X,A)$,
we will only be able to \emph{merely} resolve one $\chi:H^i(X,A)$ at a time.
So resolving all elements at once with the same construction, would require some form of choice.

We will now show,
how a short exact sequence resolving an element might be of use to extend morphisms of $\partial$-functors.

\begin{lemma}
  \label{lem:extend-map}
  Let $l\geq i>0$ and $T$ be an exact, $l$-truncated $\partial$-functor, $\mathcal{S}=A\to R_\chi\to S_\chi$ a short exact sequence in $\mathcal A$
  and $\chi:T^i(A)$
  \begin{center}
    \begin{tikzcd}
      A\ar[r,"r_\chi"] & R_\chi\ar[r,"s_\chi"] & S_\chi
    \end{tikzcd}
  \end{center}
  that resolves $\chi$, i.e. such that $T^i(r_\chi)(\chi)=0$.
  For an $l$-truncated $\partial$-functor $T'$
  and any morphism of $(i-1)$-truncated $\partial$-functors $f:T^{\leq (i-1)}\to T'^{\leq (i-1)}$,
  there is a unique\index{$\mathrm{ext}(f,\chi,\mathcal{S})$}
  \[\mathrm{ext}(f,\chi,\mathcal{S}) : T'^i(A)\]
  such that for any $x:T^{i-1}(S_\chi)$ with $\partial_{T,\mathcal{S},i-1}(x)=|\chi|$ we have $\partial_{T,S,i-1}(f^{i-1}(x))=\mathrm{ext}(f,\chi,\mathcal{S})$.
\end{lemma}

\begin{proof}
  The following diagram commutes:
  \begin{center}
    \begin{tikzcd}
      T^{i-1}(A)\ar[r]\ar[d,"f^{i-1}"]  & T^{i-1}(R_\chi)\ar[r]\ar[d,"f^{i-1}"]  & T^{i-1}(S_\chi)\ar[r,"\partial"]\ar[d,"f^{i-1}"] & T^i(A)\ar[r,"r_\chi^\ast"] & T^i(R_\chi)\dots \\
      T'^{i-1}(A)\ar[r]  & T'^{i-1}(R_\chi)\ar[r,"s_\chi^\ast"]  & T'^{i-1}(S_\chi)\ar[r,"\partial"] & T'^i(A)\ar[r] & T'^i(R_\chi)\dots 
    \end{tikzcd}
  \end{center}
  The upper row is exact and the lower row is a complex.

  Let $E(\chi,\mathcal{S})$ be the type of all possible values of $f^i$ in $T'^i(A)$,
  with which the dependent sum over all $y:T'^i(A)$ such that there merely is $x:T^{i-1}(S_\chi)$ with $\partial(x)=|\chi|$
  and $\partial(f^{i-1}(x))=y$.
  Then $E(\chi,\mathcal{S})$ is inhabited, since $r_\chi(|\chi|)=0$ and by exactness, there has to be a mere preimage under $\partial$.
  So we need to show, that $E(\chi,\mathcal{S})$ is a proposition.
  
  Let $x:T^{i-1}(S_\chi)$ such that $\partial(x)=|\chi|$.
  Then any other element with this property will be of the form $x+k$, with $k$ in the kernel of $\partial$.
  Any $k$ like that, has a mere preimage $k':T^{i-1}(R_\chi)$ and since the lower row is a complex, we have $\partial(s_\chi^\ast(f^{i-1}(k')))=0$.
  
  So for any extension $y:T'^{i}(A)$ we have
  \begin{align*}
    y &= \partial(f^{i-1}(x+k)) \\
      &= \partial(f^{i-1}(x))+\partial(f^{i-1}(k)) \\
      &= \partial(f^{i-1}(x))+\partial(s_\chi^\ast(f^{i-1}(k'))) \\
      &= \partial(f^{i-1}(x))
  \end{align*}
  This means we can define $\mathrm{ext}(f,\chi,\mathcal{S})$ to be the unique element of $E(\chi,\mathcal{S})$.
\end{proof}

While this shows, that existence of these special short exact sequences
is enough to extend a \emph{map} from one truncation level to the next,
it is not clear, that an extension constructed in this way,
is actually a morphism of truncated $\partial$-functors.

It is also unclear, if the construction even yields a well-defined map,
independent of the short exact sequence we chose in the construction.
A solution to these problems is essentially given by
requiring some ``functoriality'' of the short exact sequences we will use (\cref{local-resolution}) and
the following naturality result:

\begin{lemma}
  \label{lem:extension-welldefined}
  Let $T$ be an exact $\partial$-functor.
  Let $\chi:T^i(A)$ and
  \begin{center}
    \begin{tikzcd}
      A\ar[r,"r_\chi"]\ar[d,"\varphi"] & R_{\chi}\ar[r]\ar[d,"\varphi_R"] & S_{\chi}\ar[d,"\varphi_S"] \\
      A'\ar[r] & R_{\varphi(\chi)}\ar[r] & S_{\varphi(\chi)}
    \end{tikzcd}
  \end{center}
  be a morphism of short exact sequences ${\mathcal S}_\chi$ and $\mathcal{S}_{\varphi(\chi)}$ in $\mathcal A$,
  where $T^i(r_\chi)(\chi)=0$.
  Then, for the construction from \cref{lem:extend-map}, we have the following commutativity:
  \[ T^i(\varphi)(\mathrm{ext}(f,\chi,{\mathcal S}_\chi)) = \mathrm{ext}(f,\varphi(\chi),\mathcal{S}_{\varphi(\chi)}) \]
\end{lemma}

\begin{proof}[of \cref{lem:extension-welldefined}]
  Let $T'$ be another $\partial$-functor and $f:T^{\leq i-1}\to T'^{\leq i-1}$.
  Apply the $\partial$-Functors $T$ and $T'$ to the morphism of short exact sequences,
  to get the following diagram:
  \begin{center}
    \begin{tikzcd}
      T^{i-1}(R_\chi)\ar[r]\ar[dr]\ar[dd,"f^{i-1}"] &
      T^{i-1}(S_\chi)\ar[r]\ar[dr]\ar[dd, shift right=1ex, near start, "f^{i-1}"] &
      T^i(A)\ar[r]\ar[dr,"\varphi^\ast"] & T^i(R_\chi)\dots\ar[dr] & \\
      & T^{i-1}(R_{\varphi(\chi)})\ar[r]\ar[dd,"f^{i-1}",crossing over,near start] &
      T^{i-1}(S_{\varphi(\chi)})\ar[r]\ar[dd,crossing over, near start, "f^{i-1}"] & T^i(A')\ar[r] & T^i(R_{\varphi(\chi)})\dots \\
      T'^{i-1}(R_\chi)\ar[r] & T'^{i-1}(S_\chi)\ar[r]\ar[dr] & T'^i(A)\ar[r]\ar[dr,"T'^i(\varphi)"] & T'^i(R_\chi)\dots\ar[dr] & \\
      & T'^{i-1}(R_{\varphi(\chi)})\ar[r] & T'^{i-1}(S_{\varphi(\chi)})\ar[r] & T'^i(A')\ar[r] & T'^i(R_{\varphi(\chi)})\dots \\
      
      & a\ar[mapsto,r]\ar[dd,mapsto]\ar[rd,mapsto] & \chi\ar[r,mapsto]\ar[rd,mapsto] & 0\ar[rd,mapsto] & \\
      & & a'\ar[mapsto,r]\ar[dd,mapsto] & {\varphi(\chi)}\ar[r,mapsto] & 0 \\
      & b\ar[rd,mapsto]\ar[r,mapsto]& \mathrm{ext}(\chi,R_\chi)\ar[rd,mapsto] & & \\
      & & b'\ar[r,mapsto] & ? & 
    \end{tikzcd}
  \end{center}
  From exactness of the upper sequence, we get that there is a preimage $a$ of $\chi$.
  Let $a'$ denote the image of $a$ in $T^{i-1}(S_{\varphi(\chi)})$,
  then $a'$ will be a preimage of ${\varphi(\chi)}$ in the parallel sequence by commutativity.
  That means, that $b'$, the image of $a'$ in the lower sequence,
  will be mapped to $ \mathrm{ext}(f,\varphi(\chi),\mathcal{S}_{\varphi(\chi)})$,
  but by commutativity, $\mathrm{ext}(f,\chi,\mathcal{S}_\chi)$ will be mapped to the same thing by $T^i(\varphi)$.
  So:
  \[ T^i(\varphi)(\mathrm{ext}(f,\chi,\mathcal{S}_\chi))=\mathrm{ext}(f,\varphi(\chi),\mathcal{S}_{\varphi(\chi)})\]  
\end{proof}

We summarize the exact condition we found useful to prove universality of $\partial$-functors,
together with the existence of enough ``good'' short exact sequences in the following definition.

\begin{definition}
  \label{local-resolution}
  Let $T$ be a $\partial$-functor.
  We say that $\mathcal A$ \notion{has local resolutions for $T$}, if
  \begin{enumerate}[(i)]
  \item For any $i>0$, $A:\mathcal A$ and $\chi:T^i(A)$ there merely is a short exact sequence:
    \[
      \begin{tikzcd}
        0\ar[r] & A\ar[r,"m_\chi"] & M_\chi\ar[r] & C_\chi\ar[r] & 0
      \end{tikzcd}
    \]
    resolving $\chi$, i.e. such that $T^i(m_\chi)(\chi)=0$.
  \item For any short exact sequence $\mathcal{S}=A\to R\to S$ resolving $\chi$
    and any morphism $\varphi:A\to B$,
    there is a zig--zag of short exact sequences resolving $\chi$ or, respectively $\varphi(\chi)$,
    of the following shape:
    \begin{center}
      \begin{tikzcd}
        A\ar[r]\ar[d,equal] & R\ar[r]\ar[d] & S\ar[d] \\
        A\ar[r] & R_1\ar[r] & S_1 \\
        \vdots\ar[u,equal] & \vdots\ar[u] & \vdots\ar[u] \\
        A\ar[r]\ar[d,"\varphi"] & R_l\ar[r]\ar[d] & S_l\ar[d] \\
        B\ar[r] & M_{\varphi,\mathcal{S}}\ar[r] & C_{\varphi,\mathcal{S}} \\
        B\ar[r,"m_{\varphi(\chi)}"]\ar[u,equal] & M_{\varphi(\chi)}\ar[r]\ar[u] & C_{\varphi(\chi)}\ar[u]
      \end{tikzcd}
    \end{center}
  \end{enumerate}
\end{definition}


The following is provable by a constructive adaption of Prop 2.2.1 in \cite{tohoku-translation}:
\begin{theorem}
  \label{thm:universal}
  Let $X$ be a type.
  An exact $\partial$-functor $(T^i)_{i<l}$ from $\mathcal A$ to $\AbGroup$ is universal,
  if $\mathcal A$ has local resolutions for $T$.
\end{theorem}

\begin{proof}[of \cref{thm:universal}]
  To extend a given morphism $f^0$, 
  we will construct $f^i:T^i\to T'^i$ by induction on $i$ for $0<i<l$.
  So let $T'$ be a $\partial$-functor and assume, we already have a morphism for $i-1$ and lower indices.
  We start by constructing a group homomorphism $f^i:T^i(A)\to T'^i(A)$ for arbitrary $A:\mathcal A$.

  By \cref{lem:extend-map},
  we merely get $f^{i}(\chi)\colonequiv \mathrm{ext}(f,\chi,\mathcal{S}_\chi)$,
  for each $\chi:T^i(A)$ and their merely given local resolutions $\mathcal{S}_\chi$.
  To see that this yields an actual map, we have to check that the
  values $\mathrm{ext}(f,\chi,\mathcal{S}_\chi)$ are independent of the short exact sequence $\mathcal{S}_\chi$.
  For any other short exact sequence $\mathcal{S}'=A\to R_\chi\to S_\chi$
  that resolves $\chi$, we get a zig-zag by our requirement on local resolutions:
  \begin{center}
    \begin{tikzcd}
      A\ar[r]\ar[d,equal] & R_\chi\ar[r]\ar[d] & S_\chi\ar[d] \\
      A\ar[r] & R_1\ar[r] & S_1 \\
      \vdots\ar[u,equal] & \vdots\ar[u] & \vdots\ar[u] \\
      A\ar[r]\ar[d,"\id"] & R_l\ar[r]\ar[d] & S_l\ar[d] \\
      A\ar[r] & M_{\id,\mathcal{S}'}\ar[r] & C_{\id,\mathcal{S}'} \\
      A\ar[r,"m_{\chi}"]\ar[u,equal] & M_{\chi}\ar[r]\ar[u] & C_{\chi}\ar[u]
    \end{tikzcd}
  \end{center}
  Applying \cref{lem:extension-welldefined} to any of these morphisms $\mathcal{S}\to\mathcal{S}'$ of exact sequences
  gives us:
  \[ \mathrm{ext}(f,\chi,\mathcal{S}) = T^i(\id)(\mathrm{ext}(f,\chi,\mathcal{S}')) = \mathrm{ext}(f,\chi,\mathcal{S}') \]
  So we have a well-defined map $f^{i}:T^i(A)\to T'^i(A)$.
  We will show that it is a homomorphism of groups.
  First, note that $f^i(0)=0$,
  because $0$ has the identity as a local resolution, i.e. the sequence $0\to A\to A\to 0\to 0$.
  
  Now let $\xi,\chi,\xi+\chi:T^i(A)$.
  We need to show that $f^i(\xi)+f^i(\chi)=f^i(\xi+\chi)$.
  By additivity of the $T^i$,
  we can ``resolve'' these three elements at once, applying our construction to $(\xi,\chi,\xi+\chi):T^i(A\oplus A\oplus A)$.
  Again by \cref{lem:extension-welldefined} using the inclusions $A\to A\oplus A\oplus A$,
  we get $f^i(\xi,\chi,\xi+\chi)=(f^i(\xi),f^i(\chi),f^i(\xi+\chi))$.
  By using \cref{lem:extension-welldefined} on the map $a\colonequiv (x,y,z)\mapsto x+y-z:A\oplus A\oplus A\to A$, we get:
  \begin{align*}
    f^i(\xi)+f^i(\chi)-f^i(\xi+\chi) &=T^i(a)((f^i(\xi),f^i(\chi),f^i(\xi+\chi))) \\
                                     &=f^i(a(\xi,\chi,\xi+\chi)) \\
                                     &=f^i(0) \\
                                     &=0
                                       \rlap{.}
  \end{align*}
  This shows that $f^i$ is as homomorphism.
  
  Let $\mathcal{S}=A\to B\to C$ be an arbitrary exact sequence.
  To see that $f^i$ commutes with the connecting morphism $\partial_{\mathcal{S},i-1}$,
  let $x:T^{i-1}(C)$ and $\chi$ be the image of $x$ in $T^i(A)$.
  By exactness, $\chi$ will be mapped to 0 in $T^i(B)$, so $\mathcal{S}$
  resolves $\chi$ and therefore, the desired commutativity follows from the well-definedness proof for $f^i$.
  
  The only thing left to show
  is that $f^i$ is a natural transformation $T^i\to T'^i$.
  Let $\varphi:A\to B$ and $\chi:T^i(A)$.
  By our definition of local resolutions,
  there is a zig-zag:
  \begin{center}
    \begin{tikzcd}
      A\ar[r]\ar[d,equal] & R_\chi\ar[r]\ar[d] & S_\chi\ar[d] \\
      A\ar[r] & R_1\ar[r] & S_1 \\
      \vdots\ar[u,equal] & \vdots\ar[u] & \vdots\ar[u] \\
      A\ar[r]\ar[d,"\varphi"] & R_l\ar[r]\ar[d] & S_l\ar[d] \\
      B\ar[r] & M_{\varphi,\mathcal{S}'}\ar[r] & C_{\varphi,\mathcal{S}'} \\
      B\ar[r,"m_{\varphi(\chi)}"]\ar[u,equal] & M_{\varphi(\chi)}\ar[r]\ar[u] & C_{\varphi(\chi)}\ar[u]
    \end{tikzcd}
  \end{center}
  and therefore by applying \cref{lem:extension-welldefined} to the but last rectangle:
  \begin{align*}
    T^i(\varphi)(f^i(\chi)) &= T^i(\varphi)(\mathrm{ext}(f,\chi,\mathcal{S}_l)) \\
                            &=\mathrm{ext}(f,\varphi(\chi),\mathcal{M}) \\
                            &=f^i(\varphi(\chi))
                              \rlap{.}
  \end{align*}
\end{proof}


\section{Construction of local resolutions}

\rednote{META: This section is somewhat incomplete, but expected to work out with high level of confidence.}

\subsection{General local resolutions}

\rednote{TODO: For the vanishing result and join-based chech cohomology, we also use what is explained below.
That should somehow be consolidated/brought into the right order.}

Among 1-truncated $\partial$-functors, $H^0, H^1$, i.e. the zeroth and first cohomology groups defined in terms of Eilenberg-MacLane spaces, will always be universal.
We will show this, by constructing local resolutions for $(H^0,H^1)$ in the sense of \cref{local-resolution}.
The construction will follow a general pattern,
which we will also use in the following sections for all other resolutions.
What follows, is an explanation in more classical terms -- a reader not familiar with those,
can safely skip that explanation, since the construction we use, will be quite simple
and can be understood without the classical notions of torsors or fibre bundles.

Let $X$ be a type and $A:X\to \AbGroup$.
An element $\chi:H^1(X,A)$ can be merely represented by a map $T:(x:X)\to K(A_x,1)$.
If we use the particular implementation of the deloopings $K(A_x,1)$ as
$A_x$-torsors, $T_x$ will be a set with an action and it is natural to view $T$ as a bundle over $X$.
Let us relax the usual notion of fibre bundle a bit, to also admit the case of our $A$-torsors,
i.e. let the following be the type of $A$-fibre bundles over $X$:
\[ \sum_{T:X\to\mathcal U}\|T_x=A_x\|\]
Then, $A$-torsors will in particular also be $A$-fibre bundles.

A canonical trivialization for fibre bundles with constant prescribed fiber
is given in (\cite{cherubini-cartan})[Definition 4.9, Definition 4.11] -- but this works for the more general notion as well.
The canonical trivialization is given by
\[ V_T:\equiv \sum_{x:X}T_x=A_x \]
Then, from the definition above, we get that $\pi_1:V_T\to X$ is surjective
and the second projection will give a trivialization witness for the pullback of $T$ along $\pi_1$.

Now, since the local resolution has to be something over $X$, we have to push-forward the construction above.
But that really just means we have to take a dependent product instead of a sum.
So the resulting dependent group is just:
\[ x\mapsto A_x^{(T_x=\ast)} : X\to\AbGroup\]
The canonical map $A_x\to A_x^{(T_x=\ast)}$ maps an element $a_x:A_x$ to the function $p\mapsto a_x$.
Since $T_x=\ast$, this map will be an embedding.

For a more general $T:(x:X)\to K(A_x,l)$,
the type $\|T_x=\ast\|_0$ will be trivial if $l>1$ and therefore $A_x\to A_x^{(T_x=\ast)}$ will be an equivalence,
since $A_x$ is 0-truncated.
This means the same construction will \emph{not} work for cohomology groups above degree 1,
with coefficients in Eilenberg-MacLane spaces.
Another way to phrase the problem is,
that the spectrum $K(A_x,\_)^{(T_x=\ast)}$ fails to be an Eilenberg-MacLane spectrum.
This happens, if and only if, ${T_x=\ast}$ has non-trivial cohomology.
So one thing we can use to resolve higher cohomology classes, are covers of $X$ with cohomologically trivial fibers,
which we will do in the next section.
Now we will show, how we can use the general construction for degree 1 to get all requirements of \cref{local-resolution}:

\begin{lemma}
  \label{general-1-resolution-exists}
  Let $A:X\to\AbGroup$ and $\chi:H^1(X,A)$.
  Then there merely is a short exact sequence $\mathcal{S}_\chi$:
  \begin{center}
    \begin{tikzcd}
      0\ar[r] & A\ar[r,"m_\chi"] & M_\chi\ar[r] & C_\chi\ar[r] & 0
    \end{tikzcd}
  \end{center}
  such that for any other short exact sequence $\mathcal{S}=A\to R\to S$
  such that $\chi$ is mapped to zero in $H^1(X,R)$ and any morphism $f:A\to B$ we have
  a zig-zag:
  \begin{center}
    \begin{tikzcd}
      A\ar[r]\ar[d,"f"] & R\ar[r]\ar[d] & S\ar[d] \\
      B\ar[r] & M_{f,\mathcal{S}}\ar[r] & C_{f,\mathcal{S}} \\
      B\ar[r,"m_{f(\chi)}"]\ar[u,equal] & M_{f(\chi)}\ar[r]\ar[u] & C_{f(\chi)}\ar[u]
    \end{tikzcd}
  \end{center}
\end{lemma}

\begin{proof}
  As explained above, we merely have $T:(x:X)\to K(A_x,1)$ with $|T|_0=\chi$ and the definition
  $M_\chi\colonequiv ((x:X)\mapsto A_x^{(T_x=\ast)}$ together with the diagonal map $(a\mapsto (x, p\mapsto a_x)):A\to M_\chi$,
  gives a monomorphism $A\to M_\chi$.
  So we can take the cokernel, to a short exact sequence as required.
  $m_\chi^\ast(\chi)$ is zero by $x\mapsto \id_{(T_x=\ast)}:(x:X)\to T_x=\ast$.
  Note that this construction is well-defined in the sense that
  for another $T'$ with $|T'|=\chi$, we merely have $\|T=T'\|$ and therefire an isomorphism between the resulting sequences.

  Now, let $\mathcal{S}=A\to R\to S$ be a short exact sequence
  such that $\chi$ is mapped to zero in $H^1(X,R)$ and $\varphi:A\to B$ any morphism.
  For $T:(x:X)\to K(A_x,1)$ with $|T|=\chi$ and $T'\colonequiv (x\mapsto K(\varphi_x,1)(T_x))$ 
  a zig-zag can be constructed,
  whose maps we will describe below the diagram: 
  \begin{center}
    \begin{tikzcd}
      A\ar[r]\ar[d,equal] & R\ar[r]\ar[d,"\Delta"] & S\ar[d,dashed] \\
      A\ar[r,"\varphi\Delta"]\ar[d,equal] & (x\mapsto R_x^{T_x=\ast})\ar[r] & \coker(\varphi\Delta) \\
      A\ar[r,"\Delta"]\ar[d,"\varphi"] & (x\mapsto A_x^{T_x=\ast})\ar[r]\ar[d]\ar[u] & \coker(\Delta)\ar[d, dashed]\ar[u,dashed] \\
      B\ar[r,"\Delta'"] & (x\mapsto B_x^{T_x=\ast})\ar[r] & \coker(\Delta') \\
      B\ar[r,"\Delta''"]\ar[u,equal] & (x\mapsto B_x^{T'_x=\ast})\ar[r]\ar[u] & \coker(\Delta'')\ar[u,dashed]
    \end{tikzcd}
  \end{center}
  The maps in the middle column are all given by postcomposition with given maps,
  except for the last map, which is given by precomposition with a map
  $T_x=\ast\to T'_x=\ast$ given by using the pointed map $K(\varphi_x,1)$.
  All maps in the right column, are then induced by the universal property of cokernels.
  As noted above, the last row is isomorphic to any $\mathcal{S}_\chi$,
  so the zig-zag does indeed satisfy the specification from \cref{local-resolution}.
\end{proof}

\begin{theorem}
  $H^{\leq 1}$ is a 1-truncated, universal $\partial$-functor.
\end{theorem}

\begin{proof}
  By \cref{general-1-resolution-exists} we have local resolutions for $H^{\leq 1}$,
  we can apply \cref{thm:universal}.
\end{proof}

\subsection{Local resolutions for schemes}

\begin{definition}
  Let $X,Y$ be schemes.
  \begin{enumerate}[(a)]
  \item For $M:Y\to\Mod{R}$ and $f:X\to Y$ let $f^*M:\equiv (x:X)\mapsto M_{f(x)}$.
  \item For $M:X\to\Mod{R}$ and $f:X\to Y$ let $f_*M:\equiv (y:Y)\mapsto \prod_{x:\fib_f(y)}M_{\pi_1(x)}$.
  \end{enumerate}
\end{definition}

Both operations preserve weakly quasi-coherent modules by \cite{draft}[Theorem 9.1.11].
As defined in \cite{draft}, a \notion{scheme} is a type $X$, such that there merely is an open cover by affine schemes $U_i=\Spec A_i$.
As shown in \cref{affine-vanishing},
higher cohomology with coefficients in weakly quasi-coherent modules is trivial on affine schemes.
So we know that, for a general scheme, cohomology will be locally trivial.
A \notion{separated scheme}, is defined in \cite{draft}, as a scheme where equality of points is a \emph{closed} proposition
-- we will not explain that here and only mention that examples include projective and affine schemes.
The consequence of relevance here, is that for a separated scheme,
intersections of affine opens are affine.
This means the open affines form an acyclic cover:

\begin{remark}
  For a separated scheme $X$ and $M:X\to\Mod{R}_{wqc}$,
  any open affine cover $(U_i)_{i:I}$ is \notion{acyclic},
  i.e\footnote{Using notation from \cref{chech-complex}}.
  \[
    H^k(U_{i_0\dotsi_l},M)=0\quad \forall l>0, k>0 \text{ and } i_0,\dots,i_l:I
    \rlap{.}
  \]
\end{remark}

We will use these covers in a way similar to the last section.
For a cover $(U_i)_{i:I}$, we can view as a map from the coproduct $u:\coprod_iU_i\to X$.
Then pullback along $u$ trivializes all higher cohomology of $X$ with values in $M:X\to \Mod{R}_{wqc}$
and we can take the push-forward again to get a candidate for a sequence that resolves higher cohomology classes.

\begin{remark}
  \label{resolving-mono}
  For a separated scheme $X$ and $M:X\to\Mod{R}_{wqc}$.
  Then, for any finite affine open cover $(U_i)_{i:I}$, $x:X$
  and $u:\coprod_i U_i\to X$,
  the $R$-linear map of weakly quasi-coherent $R$-modules
  \[ \Delta\colonequiv m_x\mapsto (\_\mapsto m_x):M_x\to M_x^{\fib_u(x)}\]
  is an embedding and resolves any $\chi:H^k(X,M)$ with $k>0$.
\end{remark}

\begin{proof}
  $(U_i)_{i:I}$ is a cover, so $\fib_u(x)$ is inhabited and therefore $\Delta$ is an embedding.
  For the resolving-property, we will use that $\fib_u(x)$ is affine.
  To see that,
  we compute the fiber as an iterated pullback,
  starting with an inclusion $\iota_i:U_i\to X$ such that $U_i(x)$:
  \begin{center}
    \begin{tikzcd}
      \fib_u(x)\ar[r]\ar[d] & \coprod_j U_i\cap U_j \ar[r]\ar[d] & \coprod_i U_i\ar[d,"u"] \\
      1\ar[r] & U_i\ar[r] & X
    \end{tikzcd}
  \end{center}
  The right pullback, $\coprod_j U_i\cap U_j$, is affine, since it is a finite coproduct of the affine schemes $U_i\cap U_j$.
  The left square is a pullback by pasting, and as a pullback of affine schemes, $\fib_u(x)$ is affine.

  So we know that for any $k>0$, $H^k(\fib_u(x),M_x)=0$.
  Equivalently, the latter means $\fib_u(x)\to K(M_x,k)$ is connected.
  By the uniqueness of connected deloopings, this means that
  \[
    K(M_x,k)^{\fib_u(x)} = K(M_x^{\fib_u(x)},k)
    \rlap{.}
  \]
  And connectedness of the latter means that all higher cohomology classes are resolved.
\end{proof}

\begin{theorem}
  For a separated scheme $X$ and $M:X\to\Mod{R}_{wqc}$.
  Then the $H^k$, for $k\in\N$ form a universal $\partial$-functor with domain $X\to \Mod{R}_{wqc}$.
\end{theorem}

\begin{proof}
  The resolving sequences can be constructed from the monomorphism in \cref{resolving-mono}
  by taking cokernels, which are proven to be weakly quasi-coherent in \cite{draft}.
  The second property of local resolutions can be proven analogous to the last section.
\end{proof}

\subsection{Local resolutions for \v{C}ech-Cohomology}

Let $X$ be a fixed set and $\{U\}=(U_0,\dots,U_n)$ a fixed cover of $X$.
If $\{U\}$ is acyclic, then the \v{C}ech Cohomology of weakly quasi coherent modules on $X$
will be a universal $\partial$-functor.

The local resolutions can be constructed using the \notion{\v{C}ech-sheaf construction}.


\printindex

\printbibliography

\end{document}
