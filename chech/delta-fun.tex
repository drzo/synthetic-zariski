TODO: This has to be updated...

The following definition, from (\cite[2.1]{tohoku-translation}) and originally from (\cite{tohoku1957}), is specialized to our needs.
Grothendieck makes a definition for additive functors from an abelian category to a preadditive category.
We just state it for abelian categories and will later apply it only to functors from dependent abelian groups over a fixed type to abelian groups.
While we expect both of these types to admit the structure of an abelian category, we have proved neither and we will use neither in any proofs in this article.
Instead, we will only use basic facts about abelian groups.

\begin{definition}
  Let $C$ and $C'$ be abelian categories.
  
  A \notion{$\partial$-Functor} is a collection of functors $T^i:C\to C'$, where $0\leq i < a$ with $a\in\N\cup\{\infty\}$,
  together with a collection of connecting morphisms $\partial_{S,i}$ for any short exact sequence $S$ and $0\leq i\le a$, subject to the following conditions:
  \begin{enumerate}[(a)]
  \item Let $S$ be a short exact sequence
    \[ 0\to A'\to A\to A''\to 0\]
    in $C$. Applying the $T^i$ yields a complex, together with connecting morphisms $(\partial_{S,i})_{i\in\N}$:
    \begin{center}
      \begin{tikzcd}
        T^0(A')\ar[r] & T^0(A)\ar[r]  & T^0(A'')\ar[r,"\partial_{S,0}"]  & T^1(A')\ar[r]  & T^1(A)\ar[r]  & \dots
      \end{tikzcd}
    \end{center}
  \item For any homomorphism to a second short exact sequence
    \[ 0\to B'\to B\to B''\to 0\]
    and any valid $i$ the corresponding square commutes:
    \begin{center}
      \begin{tikzcd}
        T^i(A'')\ar[r,"\partial"]\ar[d] & T^{i+1}(A')\ar[d] \\
        T^i(B'')\ar[r,"\partial"] & T^{i+1}(B') \\
      \end{tikzcd}
    \end{center}
  \end{enumerate}
\end{definition}

We will only consider the case that $C$ is $X\to\mathcal A$, where $\mathcal A$ is some subtype of the type of abelian groups
and $C'$ is the type of abelian groups.

\begin{definition}
  Let $T$ and $T'$ be $\partial$-Functors defined for the same indices.
  
  A \notion{morphism of $\partial$-Functors} $f:T\to T'$ is given by a natural transformation $f^i:T^i\to T'^i$ for each valid $i$,
  such that for any short exact sequence
  \[ 0\to A'\to A\to A''\to 0\]
  the following square commutes:
  \begin{center}
    \begin{tikzcd}
      T^i(A'')\ar[r,"\partial"]\ar[d,"f^i_{A''}"] & T^{i+1}(A')\ar[d,"f^{i+1}_{A'}"] \\
      T'^i(A'')\ar[r,"\partial"] & T'^{i+1}(A') 
    \end{tikzcd}
  \end{center}
\end{definition}

\begin{definition}
  A $\partial$-Functor $T$ is called \notion{universal}, if for any $T'$, defined for the same indices,
  any natural transformation $f^0:T^0\to T'^0$ extends uniquely to a morphism of $\partial$-Functors $f:T\to T'$.
\end{definition}

The following is provable by a constructive adaption of Prop 2.2.1 in \cite{tohoku-translation}:
\begin{theorem}
  \label{thm:universal}
  Let $X$ be a type.
  An exact $\partial$-functor $(T^i)_i$ from $(X\to\mathcal A)$ to $\AbGroup$ is universal,
  if there are \notion{local resolutions} for $T$:
  \begin{enumerate}[(i)]
  \item For any $i>0$, $A:X\to\mathcal A$ and $\chi:T^i(A)$ there merely is a short exact sequence:
    \[
      \begin{tikzcd}
        0\ar[r] & A\ar[r,"m_\chi"] & M_\chi\ar[r] & C_\chi\ar[r] & 0
      \end{tikzcd}
    \]
    such that $T^i(m_\chi)(\chi)=0$.
  \item For any sequence $A\to R\to S$ such that $\chi$ also vanishes in $T^i(R)$ and any morphism $f:A\to B$,
    there is a zig--zag of the following shape:
    \begin{center}
      \begin{tikzcd}
        A\ar[r]\ar[d,"f"] & R\ar[r]\ar[d] & S\ar[d] \\
        B\ar[r] & M_{f,K}\ar[r] & C_{f,K} \\
        B\ar[r,"m_\chi"]\ar[u,equal] & M_{f(\chi)}\ar[r]\ar[u] & C_{f(\chi)}\ar[u]
      \end{tikzcd}
    \end{center}
  \end{enumerate}
\end{theorem}

TODO: The proof, which takes the remainder of the section needs to be refactored to match the statement of the theorem
(the proof is still about $(H^i)_i$ specifically, but we need it for $(H^i(X,\_))_i$ \emph{and} $(\v{H}^i(\{U\},\_))_i$ ).

Before proving the theorem, we will establish some lemmas about extending morphisms between $\partial$-Functors.
In the following, let $X$ be a type, $A$ a dependent abelian group over $X$, $T$ a $\partial$-Functor and $f$ a morphism of $\partial$-Functors, defined up to degree $i-1$.

\begin{lemma}
  \label{lem:extend-map}
  For any gerbe $\chi:(x:X)\to A_{x,k}$ let $S$ be a short exact sequence
  \begin{center}
    \begin{tikzcd}
      A\ar[r,"\iota_\chi"] & R_\chi\ar[r,"c_\chi"] & C_\chi
    \end{tikzcd}
  \end{center}
  of Eilenberg-MacLane spectra over $X$ resolving $\chi$.
  There is a unique
  \[\mathrm{ext}(\chi,S) : T^i(A)\]
  (with $T$ as in \cref{thm:universal}) such that for any $x:H^{i-1}(C_\chi)$ with $\partial_{H,S,i-1}(x)=|\chi|$ we have $\partial_{T,S,i-1}(f^{i-1}(x))=\mathrm{ext}(\chi,S)$.
\end{lemma}
\begin{proof}[of \cref{lem:extend-map}]
  Let
  \[ A \to R_\chi\to C_\chi\]
  be a short exact sequence of Eilenberg-MacLane spectra resolving $\chi$.
  The following diagram commutes:
  \begin{center}
    \begin{tikzcd}
      H^{i-1}(A)\ar[r]\ar[d,"f^{i-1}"]  & H^{i-1}(R_\chi)\ar[r]\ar[d,"f^{i-1}"]  & H^{i-1}(C_\chi)\ar[r,"\partial"]\ar[d,"f^{i-1}"] & H^i(A)\ar[r,"\hat{\iota}_\chi^\ast"] & H^i(R_\chi)\dots \\
      T^{i-1}(A)\ar[r]  & T^{i-1}(R_\chi)\ar[r,"c_\chi^\ast"]  & T^{i-1}(C_\chi)\ar[r,"\partial"] & T^i(A)\ar[r] & T^i(R_\chi)\dots 
    \end{tikzcd}
  \end{center}
  The upper row is exact and the lower row is a complex.

  Let $E(\chi,S)$ be the type of all possible values of $f^i$ in $T^i(A)$,
  with which we mean all $y:T^i(A)$ such that there merely is $x:H^{i-1}(C_\chi)$ with $\partial(x)=|\chi|$
  and $\partial(f^{i-1}(x))=y$.
  Then $E(\chi,S)$ is inhabited, since $\iota_\chi(|\chi|)=0$ and by exactness, there has to be a mere preimage under $\partial$.
  So we need to show, that $E(\chi,S)$ is a proposition.
  
  Let $x:H^{i-1}(C_\chi)$ such that $\partial(x)=|\chi|$.
  Then any other element with this property will be of the form $x+k$, with $k$ in the kernel of $\partial$.
  Any $k$ like that, has a mere preimage $k':H^{i-1}(R_\chi)$ and since the lower row is a complex, we have $\partial(c_\chi^\ast(f^{i-1}(k')))=0$.
  
  So for any extension $y:T^{i}(A)$ we have
  \begin{align*}
    y &= \partial(f^{i-1}(x+k)) \\
      &= \partial(f^{i-1}(x))+\partial(f^{i-1}(k)) \\
      &= \partial(f^{i-1}(x))+\partial(c_\chi^\ast(f^{i-1}(k'))) \\
      &= \partial(f^{i-1}(x))
  \end{align*}
  This means we can define $\mathrm{ext}(\chi,S)$ to be the unique element of $E(\chi,S)$.
\end{proof}

\begin{lemma}
  \label{lem:extension-welldefined}
  For any cohomology classes $\chi:(x:X)\to A_{x,k}$, $\xi:(x:X)\to A_{x,k}$ and any morphism
  \begin{center}
    \begin{tikzcd}
      A\ar[r]\ar[d,"\varphi"] & R_{\chi}\ar[r]\ar[d] & C_{\chi}\ar[d] \\
      A'\ar[r] & R_{\xi}\ar[r] & C_{\xi}
    \end{tikzcd}
  \end{center}
  of short exact resolving sequences $S_\chi$ and $S_\xi$ for $\chi$ and $\xi$
  in the sense of \cref{def:morph-resolve}, we have:
  \[ T^k(\varphi)(\mathrm{ext}(\chi,S_\chi)) = \mathrm{ext}(\xi,S_\xi) \]
\end{lemma}
\begin{proof}[of \cref{lem:extension-welldefined}]
  Apply the $\partial$-Functors $H$ and $T$ to the morphism of resolving sequences, to get the following diagram:
  \begin{center}
    \begin{tikzcd}
      H^{i-1}(R_\chi)\ar[r]\ar[dr]\ar[dd,"f^{i-1}"] & H^{i-1}(C_\chi)\ar[r]\ar[dr]\ar[dd, shift right=1ex, near start, "f^{i-1}"] & H^i(A)\ar[r]\ar[dr,"\varphi^\ast"] & H^i(R_\chi)\dots\ar[dr] & \\
      & H^{i-1}(R_\xi)\ar[r]\ar[dd,"f^{i-1}",crossing over,near start] & H^{i-1}(C_\xi)\ar[r]\ar[dd,crossing over, near start, "f^{i-1}"] & H^i(A')\ar[r] & H^i(R_\xi)\dots \\
      T^{i-1}(R_\chi)\ar[r] & T^{i-1}(C_\chi)\ar[r]\ar[dr] & T^i(A)\ar[r]\ar[dr,"T^i(\varphi)"] & T^i(R_\chi)\dots\ar[dr] & \\
      & T^{i-1}(R_\xi)\ar[r] & T^{i-1}(C_\xi)\ar[r] & T^i(A')\ar[r] & T^i(R_\xi)\dots \\
      
      & a\ar[mapsto,r]\ar[dd,mapsto]\ar[rd,mapsto] & \chi\ar[r,mapsto]\ar[rd,mapsto] & 0\ar[rd,mapsto] & \\
      & & a'\ar[mapsto,r]\ar[dd,mapsto] & \xi\ar[r,mapsto] & 0 \\
      & b\ar[rd,mapsto]\ar[r,mapsto]& \mathrm{ext}(\chi,S_\chi)\ar[rd,mapsto] & & \\
      & & b'\ar[r,mapsto] & ? & 
      
    \end{tikzcd}
  \end{center}
  From exactness of the upper sequence, we get that there is a preimage $a$ of $\chi$.
  Let $a'$ denote the image of $a$ in $H^{i-1}(C_\xi)$,
  then $a'$ will be a preimage of $\xi$ in the other sequence by commutativity.
  That means, $b'$ will be mapped to $ \mathrm{ext}(\xi,S_\xi)$,
  but by commutativity, $\mathrm{ext}(\chi,S_\chi)$ will be mapped to the same thing by $T^i(\varphi)$.
  So
  \[ T^i(\varphi)(\mathrm{ext}(\chi,S_\chi))=\mathrm{ext}(\xi,S_\xi)\]
  
\end{proof}

\begin{proof}[of \cref{thm:universal}]
  First of all, $H$ is a $\partial$-functor by \cref{prop:cohomology-les}.

  To extend a given morphism $f^0$, 
  we will construct $f^i:H^i\to T^i$ recursively for $i\in\N$.
  The construction of $f^i$ will be done pointwise, for each element of $H^i(X,A)$ using the recursion principle for 0-truncation.
  Since the codomain of $f^i$ is 0-truncated, the latter means, we can just assume that each element is of the form $|\chi|:H^i(X,A)$
  for some cohomology class $\chi:(x:X)\to A_{x,i}$.

  So assume $i\neq 0$ and let $\chi:(x:X)\to A_{x,i}$.
  By \cref{thm:resolution-exists} we have the standard short exact sequence $S_\chi$ resolving $\chi$,
  so we can use \cref{lem:extend-map} to construct an image $f^i(|\chi|):\equiv \mathrm{ext}(\chi,S_\chi)$.
  
  To see that a natural homomorphism of abelian groups $f^i$ can be constructed in this way,
  we will \cref{lem:extension-welldefined}.

  We can apply \cref{lem:extension-welldefined} to the morphism from \cref{rem:prod-resolve-sequence},
  to see that we could have define $f^i(|\chi|)$ as well by
  a resolution of the three classes $\chi, \xi, \chi+\xi:(x:X)\to A_{x,i}$.
  But this is already enough to conclude that $f^i(|\chi|+|\xi|)=f^i(|\chi)+f^i(\xi)$, by the homomorphism properties of the maps involved in the construction in \cref{lem:extend-map}.

  For a morphism $\varphi:A\to A'$ of spectra over $X$,
  we can use the morphism of resolving sequences from \cref{rem:cong-resolve-sequence}.
  Then \cref{lem:extension-welldefined} tells us directly, that $f^i$ is natural.
  
  For the commutativity of $f^i$ with connecting morphisms, let $x:X$ and consider a short exact sequence $S$:
  \begin{center}
    \begin{tikzcd}
      A_x\ar[r,"\psi_x"] & A'_x\ar[r] &  A''_x
    \end{tikzcd}
  \end{center}
  For any $\chi_x:A_{x,i }$, we can construct the following diagram:
  \begin{center}
    \begin{tikzcd}
      A_{x,i}\ar[r,"\psi_{x,i}"]\ar[d,equal] & A_{x,i}'\ar[r]\ar[d] & A_{x,i}''\ar[d,dashed] \\
      A_{x,i}\ar[r,"\widehat{\Delta\circ\psi_{x,i}}"]\ar[d,equal] & \widehat{\left((\chi_{x}=\ast)\to A_{x,i}'\right)}\ar[r]  & \mathrm{Cok}(\widehat{\Delta\circ\psi_{x,i}}) \\
      A_{x,i}\ar[r,"\widehat{\Delta}"] &  \widehat{\left((\chi_{x}=\ast)\to A_{x,i}\right)}\ar[r]\ar[u] & \mathrm{Cok}(\widehat{\Delta})\ar[u,dashed]
    \end{tikzcd}
  \end{center}
  If we know $\psi_{x,i}(\chi_{x})=\ast$, then $\chi_{x,i}$ will also equal $\ast$ in $\left((\chi_{x}=\ast)\to A_{x,i}'\right)$ and $\left((\chi_{x}=\ast)\to A_{x,i}\right)$.
  This is still true in connective covers.
  So we have morphisms between three resolving sequences and we
  can apply \cref{lem:extension-welldefined} twice to get:
  \[ \mathrm{ext}(\chi,S)=\mathrm{ext}(\chi,S_\chi)\equiv f^i(|\chi|)\]
  -- where $S$ is the given short exact sequence and $S_\chi$ the standard resolving sequence for $\chi$.

  What remains now, is to use this to show that the following diagram (from the LES for $S$) commutes:
  \begin{center}
    \begin{tikzcd}
      H^{i-1}(A'')\ar[r,"\partial"]\ar[d,"f^{i-1}"] & H^i(A)\ar[d,"f^i"] \\
      T^{i-1}(A'')\ar[r,"\partial"] & T^i(A)
    \end{tikzcd}
  \end{center}
  So let $|\xi|:H^{i-1}(A'')$ and $|\chi|:\equiv \partial(|\xi|)$.
  Then, by exactness, we know $\psi^\ast(|\chi|):H^i(A')$ is zero.
  This means the underlying class is merely zero and we can apply what we just proved,
  to (merely) get that
  \[ \partial(f^{i-1}(|\xi|))=\mathrm{ext}(\chi,S)=\mathrm{ext}(\chi,S_\chi)\equiv f^i(|\chi|)=f^i(\partial(|\xi|))\]
  Done.
\end{proof}
