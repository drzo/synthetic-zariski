TODO: This has to be updated...

Cohomology has the universal property of being a \notion{universal $\partial$-functor}.
In this section, we will construct a tool for proving this in some particular situations,
both for the cohomology defined using Eilenberg-MacLane spaces and \v{C}hech cohomology.

The following definition, from (\cite[2.1]{tohoku-translation}) and originally from (\cite{tohoku1957}), is specialized to our needs.
Grothendieck makes a definition for additive functors from an abelian category to a preadditive category.
We will only need the theory for functors from certain subcategories of dependent $R$-modules over a fixed type to abelian groups.
Also, some arguments are a lot more convenient when we can use elements of modules instead of abstract categorical language.
Therefore, we will state our definitions and results only for this particular situation.

\emph{Let $R$ be a fixed commutative ring and $\mathcal A$ be a subcategory of the category of dependent $R$-modules over a fixed type $X$, which is closed under finite direct sums.}

\begin{definition}
  An \notion{($l$-truncated) $\partial$-functor} is a collection of additve\footnote{
    The zero object and binary direct sums are preserved.
  }
  functors
  $T^i:\mathcal A\to \AbGroup$, where $0\leq i < l$ with $l\in\N\cup\{\infty\}$,
  together with a collection of connecting morphisms $\partial_{S,i}$ for any short exact sequence $S$ and $0\leq i\le l$, subject to the following conditions:
  \begin{enumerate}[(a)]
  \item Let $\mathcal{S}$ be a short exact sequence
    \[ 0\to A'\to A\to A''\to 0\]
    in $\mathcal A$. Applying the $T^i$ yields a complex, together with connecting morphisms $(\partial_{\mathcal{S},i})_{0\leq i<l-1}$:
    \begin{center}
      \begin{tikzcd}
        0\ar[r] & T^0(A')\ar[r] & T^0(A)\ar[r]  & T^0(A'')\ar[r,"\partial_{\mathcal{S},0}"]  & T^1(A')\ar[r]  & T^1(A)\ar[r]  & \dots
      \end{tikzcd}
    \end{center}
  \item For any homomorphism to a second short exact sequence
    \[ 0\to B'\to B\to B''\to 0\]
    and any $i<l-1$ the corresponding square commutes:
    \begin{center}
      \begin{tikzcd}
        T^i(A'')\ar[r,"\partial"]\ar[d] & T^{i+1}(A')\ar[d] \\
        T^i(B'')\ar[r,"\partial"] & T^{i+1}(B') \\
      \end{tikzcd}
    \end{center}
  \end{enumerate}
\end{definition}

\begin{definition}
  Let $l,k:\N$.
  The $l$-th \notion{truncation} of a $(l+k)$-truncated $\partial$-functor $T$
  is just the restriction of $(T^i)_{i<l+k}$ to $(T^i)_{i<l}$, together with a restriction of the $\partial$-maps
  and we denote the $l$-th truncation with \notion{$T^{\leq l}$}.
\end{definition}

\begin{definition}
  Let $T$ and $T'$ be $\partial$-functors defined for the same indices.
  
  A \notion{morphism of $\partial$-functors} $f:T\to T'$ is given by a natural transformation $f^i:T^i\to T'^i$ for each valid $i$,
  such that for any short exact sequence
  \[ 0\to A'\to A\to A''\to 0\]
  the following square commutes:
  \begin{center}
    \begin{tikzcd}
      T^i(A'')\ar[r,"\partial"]\ar[d,"f^i_{A''}"] & T^{i+1}(A')\ar[d,"f^{i+1}_{A'}"] \\
      T'^i(A'')\ar[r,"\partial"] & T'^{i+1}(A') 
    \end{tikzcd}
  \end{center}
\end{definition}

\begin{definition}
  A $\partial$-functor $T$ is called \notion{exact},
  if all values are exact complexes.
\end{definition}

\begin{definition}
  A $\partial$-functor $T$ is called \notion{universal}, if for any $T'$, defined for the same indices,
  any natural transformation $f^0:T^0\to T'^0$ extends uniquely to a morphism of $\partial$-functors $f:T\to T'$.
\end{definition}

To prove that some $\partial$-functor has this universal property,
we will \emph{extend} morphisms of $\partial$-functors, level by level.
By observing the diagram in the proof of the lemma below,
one can see that this is possible using exact sequences with the property,
that some particular element is zero in their middle term.
This is something that could be achieved with injective resolutions classically -- for all elements at once.

\begin{lemma}
  \label{lem:extend-map}
  Let $l\geq i>0$ and $T$ be an exact, $l$-truncated $\partial$-functor, $\mathcal{S}=A\to R_\chi\to S_\chi$ a short exact sequence in $\mathcal A$
  and $\chi:T^i(A)$
  \begin{center}
    \begin{tikzcd}
      A\ar[r,"r_\chi"] & R_\chi\ar[r,"s_\chi"] & S_\chi
    \end{tikzcd}
  \end{center}
  such that $T^i(r_\chi)(\chi)=0$.
  For an $l$-truncated $\partial$-functor $T'$
  and any morphism of $(i-1)$-truncated $\partial$-functors $f:T^{\leq (i-1)}\to T'^{\leq (i-1)}$, there is a unique\index{$\mathrm{ext}(f,\chi,\mathcal{S})$}
  \[\mathrm{ext}(f,\chi,\mathcal{S}) : T'^i(A)\]
  such that for any $x:T^{i-1}(S_\chi)$ with $\partial_{T,\mathcal{S},i-1}(x)=|\chi|$ we have $\partial_{T,S,i-1}(f^{i-1}(x))=\mathrm{ext}(f,\chi,\mathcal{S})$.
\end{lemma}

\begin{proof}
  The following diagram commutes:
  \begin{center}
    \begin{tikzcd}
      T^{i-1}(A)\ar[r]\ar[d,"f^{i-1}"]  & T^{i-1}(R_\chi)\ar[r]\ar[d,"f^{i-1}"]  & T^{i-1}(S_\chi)\ar[r,"\partial"]\ar[d,"f^{i-1}"] & T^i(A)\ar[r,"r_\chi^\ast"] & T^i(R_\chi)\dots \\
      T'^{i-1}(A)\ar[r]  & T'^{i-1}(R_\chi)\ar[r,"s_\chi^\ast"]  & T'^{i-1}(S_\chi)\ar[r,"\partial"] & T'^i(A)\ar[r] & T'^i(R_\chi)\dots 
    \end{tikzcd}
  \end{center}
  The upper row is exact and the lower row is a complex.

  Let $E(\chi,\mathcal{S})$ be the type of all possible values of $f^i$ in $T'^i(A)$,
  with which the dependent sum over all $y:T'^i(A)$ such that there merely is $x:T^{i-1}(S_\chi)$ with $\partial(x)=|\chi|$
  and $\partial(f^{i-1}(x))=y$.
  Then $E(\chi,\mathcal{S})$ is inhabited, since $r_\chi(|\chi|)=0$ and by exactness, there has to be a mere preimage under $\partial$.
  So we need to show, that $E(\chi,\mathcal{S})$ is a proposition.
  
  Let $x:T^{i-1}(S_\chi)$ such that $\partial(x)=|\chi|$.
  Then any other element with this property will be of the form $x+k$, with $k$ in the kernel of $\partial$.
  Any $k$ like that, has a mere preimage $k':T^{i-1}(R_\chi)$ and since the lower row is a complex, we have $\partial(s_\chi^\ast(f^{i-1}(k')))=0$.
  
  So for any extension $y:T'^{i}(A)$ we have
  \begin{align*}
    y &= \partial(f^{i-1}(x+k)) \\
      &= \partial(f^{i-1}(x))+\partial(f^{i-1}(k)) \\
      &= \partial(f^{i-1}(x))+\partial(s_\chi^\ast(f^{i-1}(k'))) \\
      &= \partial(f^{i-1}(x))
  \end{align*}
  This means we can define $\mathrm{ext}(f,\chi,\mathcal{S})$ to be the unique element of $E(\chi,\mathcal{S})$.
\end{proof}

While this shows, that existence of these special short exact sequences
is enough to extend a \emph{map} from one truncation level to the next,
it is not clear, that an extension constructed in this way,
is actually a morphism of truncated $\partial$-functors.

It is also unclear, if the construction even yields a well-defined map,
independent of the short exact sequence we chose in the construction.

A solution to these problems is essentially given by
requiring some ``functoriality'' of the short exact sequences we will use (\cref{local-resolution}) and
the following naturality result:

\begin{lemma}
  \label{lem:extension-welldefined}
  Let $T$ be an exact $\partial$-functor.
  Let $\chi:T^i(A)$ and
  \begin{center}
    \begin{tikzcd}
      A\ar[r,"r_\chi"]\ar[d,"\varphi"] & R_{\chi}\ar[r]\ar[d,"\varphi_R"] & S_{\chi}\ar[d,"\varphi_S"] \\
      A'\ar[r] & R_{\varphi(\chi)}\ar[r] & S_{\varphi(\chi)}
    \end{tikzcd}
  \end{center}
  be a morphism of short exact sequences ${\mathcal S}_\chi$ and $\mathcal{S}_{\varphi(\chi)}$ in $\mathcal A$,
  where $T^i(r_\chi)(\chi)=0$.
  Then, for the construction from \cref{lem:extend-map}, we have the following commutativity:
  \[ T^i(\varphi)(\mathrm{ext}(f,\chi,{\mathcal S}_\chi)) = \mathrm{ext}(f,\varphi(\chi),\mathcal{S}_{\varphi(\chi)}) \]
\end{lemma}

\begin{proof}[of \cref{lem:extension-welldefined}]
  Let $T'$ be another $\partial$-functor and $f:T^{\leq i-1}\to T'^{\leq i-1}$.
  Apply the $\partial$-Functors $T$ and $T'$ to the morphism of short exact sequences,
  to get the following diagram:
  \begin{center}
    \begin{tikzcd}
      T^{i-1}(R_\chi)\ar[r]\ar[dr]\ar[dd,"f^{i-1}"] &
      T^{i-1}(S_\chi)\ar[r]\ar[dr]\ar[dd, shift right=1ex, near start, "f^{i-1}"] &
      T^i(A)\ar[r]\ar[dr,"\varphi^\ast"] & T^i(R_\chi)\dots\ar[dr] & \\
      & T^{i-1}(R_{\varphi(\chi)})\ar[r]\ar[dd,"f^{i-1}",crossing over,near start] &
      T^{i-1}(S_{\varphi(\chi)})\ar[r]\ar[dd,crossing over, near start, "f^{i-1}"] & T^i(A')\ar[r] & T^i(R_{\varphi(\chi)})\dots \\
      T'^{i-1}(R_\chi)\ar[r] & T'^{i-1}(S_\chi)\ar[r]\ar[dr] & T'^i(A)\ar[r]\ar[dr,"T'^i(\varphi)"] & T'^i(R_\chi)\dots\ar[dr] & \\
      & T'^{i-1}(R_{\varphi(\chi)})\ar[r] & T'^{i-1}(S_{\varphi(\chi)})\ar[r] & T'^i(A')\ar[r] & T'^i(R_{\varphi(\chi)})\dots \\
      
      & a\ar[mapsto,r]\ar[dd,mapsto]\ar[rd,mapsto] & \chi\ar[r,mapsto]\ar[rd,mapsto] & 0\ar[rd,mapsto] & \\
      & & a'\ar[mapsto,r]\ar[dd,mapsto] & {\varphi(\chi)}\ar[r,mapsto] & 0 \\
      & b\ar[rd,mapsto]\ar[r,mapsto]& \mathrm{ext}(\chi,R_\chi)\ar[rd,mapsto] & & \\
      & & b'\ar[r,mapsto] & ? & 
    \end{tikzcd}
  \end{center}
  From exactness of the upper sequence, we get that there is a preimage $a$ of $\chi$.
  Let $a'$ denote the image of $a$ in $T^{i-1}(S_{\varphi(\chi)})$,
  then $a'$ will be a preimage of ${\varphi(\chi)}$ in the parallel sequence by commutativity.
  That means, that $b'$, the image of $a'$ in the lower sequence,
  will be mapped to $ \mathrm{ext}(f,\varphi(\chi),\mathcal{S}_{\varphi(\chi)})$,
  but by commutativity, $\mathrm{ext}(f,\chi,\mathcal{S}_\chi)$ will be mapped to the same thing by $T^i(\varphi)$.
  So:
  \[ T^i(\varphi)(\mathrm{ext}(f,\chi,\mathcal{S}_\chi))=\mathrm{ext}(f,\varphi(\chi),\mathcal{S}_{\varphi(\chi)})\]  
\end{proof}

We summarize the exact condition we found useful to prove universality of $\partial$-functors,
together with the existence of enough ``good'' short exact sequences in the following definition.

\begin{definition}
  \label{local-resolution}
  Let $T$ be a $\partial$-functor.
  We say that $\mathcal A$ \notion{has local resolutions for $T$}, if
  \begin{enumerate}[(i)]
  \item For any $i>0$, $A:\mathcal A$ and $\chi:T^i(A)$ there merely is a short exact sequence:
    \[
      \begin{tikzcd}
        0\ar[r] & A\ar[r,"m_\chi"] & M_\chi\ar[r] & C_\chi\ar[r] & 0
      \end{tikzcd}
    \]
    such that $T^i(m_\chi)(\chi)=0$.
  \item For any short exact sequence $\mathcal{S}=A\to R\to S$
    such that $\chi$ also vanishes in $T^i(R)$ and any morphism $f:A\to B$,
    there is a zig--zag of the following shape:
    \begin{center}
      \begin{tikzcd}
        A\ar[r]\ar[d,"f"] & R\ar[r]\ar[d] & S\ar[d] \\
        B\ar[r] & M_{f,\mathcal{S}}\ar[r] & C_{f,\mathcal{S}} \\
        B\ar[r,"m_{f(\chi)}"]\ar[u,equal] & M_{f(\chi)}\ar[r]\ar[u] & C_{f(\chi)}\ar[u]
      \end{tikzcd}
    \end{center}
  \end{enumerate}
\end{definition}


The following is provable by a constructive adaption of Prop 2.2.1 in \cite{tohoku-translation}:
\begin{theorem}
  \label{thm:universal}
  Let $X$ be a type.
  An exact $\partial$-functor $(T^i)_{i<l}$ from $\mathcal A$ to $\AbGroup$ is universal,
  if $\mathcal A$ has local resolutions for $T$.
\end{theorem}

\begin{proof}[of \cref{thm:universal}]
  To extend a given morphism $f^0$, 
  we will construct $f^i:T^i\to T'^i$ by induction on $i$ for $0<i<l$.
  So let $T'$ be a $\partial$-functor and assume, we already have a morphism for $i-1$ and lower indices.
  We start by constructing a group homomorphism $f^i:T^i(A)\to T'^i(A)$ for arbitrary $A:\mathcal A$.

  By \cref{lem:extend-map},
  we merely get $f^{i}(\chi)\colonequiv \mathrm{ext}(f,\chi,\mathcal{S}_\chi)$,
  for each $\chi:T^i(A)$ and their merely given local resolutions $\mathcal{S}_\chi$.
  To see that this yields an actual map, we have to check that the
  values $\mathrm{ext}(f,\chi,\mathcal{S}_\chi)$ are independent of the short exact sequence $\mathcal{S}_\chi$.
  For any other short exact sequence $\mathcal{S}'=A\to R_\chi\to S_\chi$
  such that $T^i(\dots)(\chi)=0$, we get a zig-zag by our requirement on local resolutions:
  \begin{center}
    \begin{tikzcd}
      A\ar[r]\ar[d,"\id"] & R_\chi\ar[r]\ar[d] & S_\chi\ar[d] \\
      A\ar[r] & M_{\id,\mathcal{S}'}\ar[r] & C_{\id,\mathcal{S}'} \\
      A\ar[r,"m_{\chi}"]\ar[u,equal] & M_{\chi}\ar[r]\ar[u] & C_{\chi}\ar[u]
    \end{tikzcd}
  \end{center}
  Let us call the middle sequence $\mathcal{M}$.
  Applying \cref{lem:extension-welldefined} first to the lower rectangle and then to the upper one gives us:
  \begin{align*}
    \mathrm{ext}(f,\chi,\mathcal{S}_\chi) &= T^i(\id)(\mathrm{ext}(f,\chi,\mathcal{S}_\chi)) \\
                                          &=\mathrm{ext}(f,\id(\chi),\mathcal{M}) \\
                                          &=T^i(\id)(\mathrm{ext}(f,\chi,\mathcal{S}')) \\
                                          &=\mathrm{ext}(f,\chi,\mathcal{S}')
                                            \rlap{.}
  \end{align*}
  So we have a well-defined map $f^{i}:T^i(A)\to T'^i(A)$.
  We will show that it is a homomorphism of groups.
  First, note that $f^i(0)=0$,
  because $0$ has the identity as a local resolution, i.e. the sequence $0\to A\to A\to 0\to 0$.
  
  Now let $\xi,\chi,\xi+\chi:T^i(A)$.
  We need to show that $f^i(\xi)+f^i(\chi)=f^i(\xi+\chi)$.
  By additivity of the $T^i$,
  we can ``resolve'' these three elements at once, applying our construction to $(\xi,\chi,\xi+\chi):T^i(A\oplus A\oplus A)$.
  Again by \cref{lem:extension-welldefined} using the inclusions $A\to A\oplus A\oplus A$,
  we get $f^i(\xi,\chi,\xi+\chi)=(f^i(\xi),f^i(\chi),f^i(\xi+\chi))$.
  By using \cref{lem:extension-welldefined} on the map $a\colonequiv (x,y,z)\mapsto x+y-z:A\oplus A\oplus A\to A$, we get:
  \begin{align*}
    f^i(\xi)+f^i(\chi)-f^i(\xi+\chi) &=T^i(a)((f^i(\xi),f^i(\chi),f^i(\xi+\chi))) \\
                                     &=f^i(a(\xi,\chi,\xi+\chi)) \\
                                     &=f^i(0) \\
                                     &=0
                                       \rlap{.}
  \end{align*}
  This shows that $f^i$ is as homomorphism and the only thing left to show,
  is that $f^i$ commutes with all connecting maps $\partial_{\mathcal{S},A,i}$.
  
  
  TODO: The proof, which takes the remainder of the section needs to be refactored to match the statement of the theorem
(the proof is still about $(H^i)_i$ specifically, but we need it for $(H^i(X,\_))_i$ \emph{and} $(\v{H}^i(\{U\},\_))_i$ ).

  For a morphism $\varphi:A\to A'$ of spectra over $X$,
  we can use the morphism of resolving sequences from \cref{rem:cong-resolve-sequence}.
  Then \cref{lem:extension-welldefined} tells us directly, that $f^i$ is natural.
  
  For the commutativity of $f^i$ with connecting morphisms, let $x:X$ and consider a short exact sequence $S$:
  \begin{center}
    \begin{tikzcd}
      A_x\ar[r,"\psi_x"] & A'_x\ar[r] &  A''_x
    \end{tikzcd}
  \end{center}
  For any $\chi_x:A_{x,i }$, we can construct the following diagram:
  \begin{center}
    \begin{tikzcd}
      A_{x,i}\ar[r,"\psi_{x,i}"]\ar[d,equal] & A_{x,i}'\ar[r]\ar[d] & A_{x,i}''\ar[d,dashed] \\
      A_{x,i}\ar[r,"\widehat{\Delta\circ\psi_{x,i}}"]\ar[d,equal] & \widehat{\left((\chi_{x}=\ast)\to A_{x,i}'\right)}\ar[r]  & \mathrm{Cok}(\widehat{\Delta\circ\psi_{x,i}}) \\
      A_{x,i}\ar[r,"\widehat{\Delta}"] &  \widehat{\left((\chi_{x}=\ast)\to A_{x,i}\right)}\ar[r]\ar[u] & \mathrm{Cok}(\widehat{\Delta})\ar[u,dashed]
    \end{tikzcd}
  \end{center}
  If we know $\psi_{x,i}(\chi_{x})=\ast$, then $\chi_{x,i}$ will also equal $\ast$ in $\left((\chi_{x}=\ast)\to A_{x,i}'\right)$ and $\left((\chi_{x}=\ast)\to A_{x,i}\right)$.
  This is still true in connective covers.
  So we have morphisms between three resolving sequences and we
  can apply \cref{lem:extension-welldefined} twice to get:
  \[ \mathrm{ext}(\chi,S)=\mathrm{ext}(\chi,S_\chi)\equiv f^i(|\chi|)\]
  -- where $S$ is the given short exact sequence and $S_\chi$ the standard resolving sequence for $\chi$.

  What remains now, is to use this to show that the following diagram (from the LES for $S$) commutes:
  \begin{center}
    \begin{tikzcd}
      H^{i-1}(A'')\ar[r,"\partial"]\ar[d,"f^{i-1}"] & H^i(A)\ar[d,"f^i"] \\
      T^{i-1}(A'')\ar[r,"\partial"] & T^i(A)
    \end{tikzcd}
  \end{center}
  So let $|\xi|:H^{i-1}(A'')$ and $|\chi|:\equiv \partial(|\xi|)$.
  Then, by exactness, we know $\psi^\ast(|\chi|):H^i(A')$ is zero.
  This means the underlying class is merely zero and we can apply what we just proved,
  to (merely) get that
  \[ \partial(f^{i-1}(|\xi|))=\mathrm{ext}(\chi,S)=\mathrm{ext}(\chi,S_\chi)\equiv f^i(|\chi|)=f^i(\partial(|\xi|))\]
  Done.
\end{proof}
