
\rednote{META: This section is somewhat incomplete, but expected to work out with high level of confidence.}

\subsection{General local resolutions}

\rednote{TODO: For the vanishing result and join-based chech cohomology, we also use what is explained below.
That should somehow be consolidated/brought into the right order.}

Among 1-truncated $\partial$-functors, $H^0, H^1$, i.e. the zeroth and first cohomology groups defined in terms of Eilenberg-MacLane spaces, will always be universal.
We will show this, by constructing local resolutions for $(H^0,H^1)$ in the sense of \cref{local-resolution}.
The construction will follow a general pattern,
which we will also use in the following sections for all other resolutions.
What follows, is an explanation in more classical terms -- a reader not familiar with those,
can safely skip that explanation, since the construction we use, will be quite simple
and can be understood without the classical notions of torsors or fibre bundles.

Let $X$ be a type and $A:X\to \AbGroup$.
An element $\chi:H^1(X,A)$ can be merely represented by a map $T:(x:X)\to K(A_x,1)$.
If we use the particular implementation of the deloopings $K(A_x,1)$ as
$A_x$-torsors, $T_x$ will be a set with an action and it is natural to view $T$ as a bundle over $X$.
Let us relax the usual notion of fibre bundle a bit, to also admit the case of our $A$-torsors,
i.e. let the following be the type of $A$-fibre bundles over $X$:
\[ \sum_{T:X\to\mathcal U}\|T_x=A_x\|\]
Then, $A$-torsors will in particular also be $A$-fibre bundles.

A canonical trivialization for fibre bundles with constant prescribed fiber
is given in (\cite{cherubini-cartan})[Definition 4.9, Definition 4.11] -- but this works for the more general notion as well.
The canonical trivialization is given by
\[ V_T:\equiv \sum_{x:X}T_x=A_x \]
Then, from the definition above, we get that $\pi_1:V_T\to X$ is surjective
and the second projection will give a trivialization witness for the pullback of $T$ along $\pi_1$.

Now, since the local resolution has to be something over $X$, we have to push-forward the construction above.
But that really just means we have to take a dependent product instead of a sum.
So the resulting dependent group is just:
\[ x\mapsto A_x^{(T_x=\ast)} : X\to\AbGroup\]
The canonical map $A_x\to A_x^{(T_x=\ast)}$ maps an element $a_x:A_x$ to the function $p\mapsto a_x$.
Since $T_x=\ast$, this map will be an embedding.

For a more general $T:(x:X)\to K(A_x,l)$,
the type $\|T_x=\ast\|_0$ will be trivial if $l>1$ and therefore $A_x\to A_x^{(T_x=\ast)}$ will be an equivalence,
since $A_x$ is 0-truncated.
This means the same construction will \emph{not} work for cohomology groups above degree 1,
with coefficients in Eilenberg-MacLane spaces.
Another way to phrase the problem is,
that the spectrum $K(A_x,\_)^{(T_x=\ast)}$ fails to be an Eilenberg-MacLane spectrum.
This happens, if and only if, ${T_x=\ast}$ has non-trivial cohomology.
So one thing we can use to resolve higher cohomology classes, are covers of $X$ with cohomologically trivial fibers,
which we will do in the next section.
Now we will show, how we can use the general construction for degree 1 to get all requirements of \cref{local-resolution}:

\begin{lemma}
  \label{general-1-resolution-exists}
  Let $A:X\to\AbGroup$ and $\chi:H^1(X,A)$.
  Then there merely is a short exact sequence $\mathcal{S}_\chi$:
  \begin{center}
    \begin{tikzcd}
      0\ar[r] & A\ar[r,"m_\chi"] & M_\chi\ar[r] & C_\chi\ar[r] & 0
    \end{tikzcd}
  \end{center}
  such that for any other short exact sequence $\mathcal{S}=A\to R\to S$
  such that $\chi$ is mapped to zero in $H¹(X,R)$ and any morphism $f:A\to B$ we have
  a zig-zag:
  \begin{center}
    \begin{tikzcd}
      A\ar[r]\ar[d,"f"] & R\ar[r]\ar[d] & S\ar[d] \\
      B\ar[r] & M_{f,\mathcal{S}}\ar[r] & C_{f,\mathcal{S}} \\
      B\ar[r,"m_{f(\chi)}"]\ar[u,equal] & M_{f(\chi)}\ar[r]\ar[u] & C_{f(\chi)}\ar[u]
    \end{tikzcd}
  \end{center}
\end{lemma}

\begin{proof}
  As explained above, we merely have $T:(x:X)\to K(A_x,1)$ with $|T|_0=\chi$ and the definition
  $M_\chi\colonequiv ((x:X)\mapsto A_x^{(T_x=\ast)}$ together with the diagonal map $(a\mapsto (x, p\mapsto a_x)):A\to M_\chi$,
  gives a monomorphism $A\to M_\chi$.
  So we can take the cokernel, to a short exact sequence as required.
  $m_\chi^\ast(\chi)$ is zero by $x\mapsto \id_{(T_x=\ast)}:(x:X)\to T_x=\ast$.
  Note that this construction is well-defined in the sense that
  for another $T'$ with $|T'|=\chi$, we merely have $\|T=T'\|$ and therefire an isomorphism between the resulting sequences.

  Now, let $\mathcal{S}=A\to R\to S$ be a short exact sequence
  such that $\chi$ is mapped to zero in $H¹(X,R)$ and $\varphi:A\to B$ any morphism.
  For $T:(x:X)\to K(A_x,1)$ with $|T|=\chi$ and $T'\colonequiv (x\mapsto K(\varphi_x,1)(T_x))$ 
  a zig-zag can be constructed,
  whose maps we will describe below the diagram: 
  \begin{center}
    \begin{tikzcd}
      A\ar[r]\ar[d,equal] & R\ar[r]\ar[d,"\Delta"] & S\ar[d,dashed] \\
      A\ar[r,"\varphi\Delta"]\ar[d,equal] & (x\mapsto R_x^{T_x=\ast})\ar[r] & \coker(\varphi\Delta) \\
      A\ar[r,"\Delta"]\ar[d,"\varphi"] & (x\mapsto A_x^{T_x=\ast})\ar[r]\ar[d]\ar[u] & \coker(\Delta)\ar[d, dashed]\ar[u,dashed] \\
      B\ar[r,"\Delta'"] & (x\mapsto B_x^{T_x=\ast})\ar[r] & \coker(\Delta') \\
      B\ar[r,"\Delta''"]\ar[u,equal] & (x\mapsto B_x^{T'_x=\ast})\ar[r]\ar[u] & \coker(\Delta'')\ar[u,dashed]
    \end{tikzcd}
  \end{center}
  The maps in the middle column are all given by postcomposition with given maps,
  except for the last map, which is given by precomposition with a map
  $T_x=\ast\to T'_x=\ast$ given by using the pointed map $K(\varphi_x,1)$.
  All maps in the right column, are then induced by the universal property of cokernels.
  As noted above, the last row is isomorphic to any $\mathcal{S}_\chi$,
  so the zig-zag does indeed satisfy the specification from \cref{local-resolution}.
\end{proof}

\begin{theorem}
  $H^{\leq 1}$ is a 1-truncated, universal $\partial$-functor.
\end{theorem}

\begin{proof}
  By \cref{general-1-resolution-exists} we have local resolutions for $H^{\leq 1}$,
  we can apply \cref{thm:universal}.
\end{proof}

\subsection{Local resolutions for schemes}

\begin{definition}
  Let $X,Y$ be schemes.
  \begin{enumerate}[(a)]
  \item For $M:Y\to\Mod{R}$ and $f:X\to Y$ let $f^*M:\equiv (x:X)\mapsto M_{f(x)}$.
  \item For $M:X\to\Mod{R}$ and $f:X\to Y$ let $f_*M:\equiv (y:Y)\mapsto \prod_{x:\fib_f(y)}M_{\pi_1(x)}$.
  \end{enumerate}
\end{definition}

Both operations preserve weakly quasi-coherent modules by \cite{draft}[Theorem 9.1.11].
As defined in \cite{draft}, a \notion{scheme} is a type $X$, such that there merely is an open cover by affine schemes $U_i=\Spec A_i$.
As shown in \cref{affine-vanishing},
higher cohomology with coefficients in weakly quasi-coherent modules is trivial on affine schemes.
So we know that, for a general scheme, cohomology will be locally trivial.
A \notion{separated scheme}, is defined in \cite{draft}, as a scheme where equality of points is a \emph{closed} proposition
-- we will not explain that here and only mention that examples include projective and affine schemes.
The consequence of relevance here, is that for a separated scheme,
intersections of affine opens are affine.
This means the open affines form an acyclic cover:

\begin{remark}
  For a separated scheme $X$ and $M:X\to\Mod{R}_{wqc}$,
  any open affine cover $(U_i)_{i:I}$ is \notion{acyclic},
  i.e\footnote{Using notation from \cref{chech-complex}}.
  \[
    H^k(U_{i_0\dotsi_l},M)=0\quad \forall l>0, k>0 \text{ and } i_0,\dots,i_l:I
    \rlap{.}
  \]
\end{remark}

We will use these covers in a way similar to the last section.
For a cover $(U_i)_{i:I}$, we can view as a map from the coproduct $u:\coprod_iU_i\to X$.
Then pullback along $u$ trivializes all higher cohomology of $X$ with values in $M:X\to \Mod{R}_{wqc}$
and we can take the push-forward again to get a candidate for a sequence that resolves higher cohomology classes.

\begin{remark}
  \label{resolving-mono}
  For a separated scheme $X$ and $M:X\to\Mod{R}_{wqc}$.
  Then, for any finite affine open cover $(U_i)_{i:I}$, $x:X$
  and $u:\coprod_i U_i\to X$,
  the $R$-linear map of weakly quasi-coherent $R$-modules
  \[ \Delta\colonequiv m_x\mapsto (\_\mapsto m_x):M_x\to M_x^{\fib_u(x)}\]
  is an embedding and resolves any $\chi:H^k(X,M)$ with $k>0$.
\end{remark}

\begin{proof}
  $(U_i)_{i:I}$ is a cover, so $\fib_u(x)$ is inhabited and therefore $\Delta$ is an embedding.
  For the resolving-property, we will use that $\fib_u(x)$ is affine.
  To see that,
  we compute the fiber as an iterated pullback,
  starting with an inclusion $\iota_i:U_i\to X$ such that $U_i(x)$:
  \begin{center}
    \begin{tikzcd}
      \fib_u(x)\ar[r]\ar[d] & \coprod_j U_i\cap U_j \ar[r]\ar[d] & \coprod_i U_i\ar[d,"u"] \\
      1\ar[r] & U_i\ar[r] & X
    \end{tikzcd}
  \end{center}
  The right pullback, $\coprod_j U_i\cap U_j$, is affine, since it is a finite coproduct of the affine schemes $U_i\cap U_j$.
  The left square is a pullback by pasting, and as a pullback of affine schemes, $\fib_u(x)$ is affine.

  So we know that for any $k>0$, $H^k(\fib_u(x),M_x)=0$.
  Equivalently, the latter means $\fib_u(x)\to K(M_x,k)$ is connected.
  By the uniqueness of connected deloopings, this means that
  \[
    K(M_x,k)^{\fib_u(x)} = K(M_x^{\fib_u(x)},k)
    \rlap{.}
  \]
  And connectedness of the latter means that all higher cohomology classes are resolved.
\end{proof}

\begin{theorem}
  For a separated scheme $X$ and $M:X\to\Mod{R}_{wqc}$.
  Then the $H^k$, for $k\in\N$ form a universal $\partial$-functor with domain $X\to \Mod{R}_{wqc}$.
\end{theorem}

\begin{proof}
  The resolving sequences can be constructed from the monomorphism in \cref{resolving-mono}
  by taking cokernels, which are proven to be weakly quasi-coherent in \cite{draft}.
  The second property of local resolutions can be proven analogous to the last section.
\end{proof}

\subsection{Local resolutions for \v{C}ech-Cohomology}

Let $X$ be a fixed set and $\{U\}=(U_0,\dots,U_n)$ a fixed cover of $X$.
If $\{U\}$ is acyclic, then the \v{C}ech Cohomology of weakly quasi coherent modules on $X$
will be a universal $\partial$-functor.

The local resolutions can be constructed using the \notion{\v{C}ech-sheaf construction}.