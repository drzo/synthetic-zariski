
BEWARE, SECTION IN PROGRESS, By Hugo and Felix.

Here we go toward the long road giving quasi-projective schemes stable under dependent sums. 


\subsection{Families of projective spaces being a projectivisation of a bundle}

In this section we prove that for any type $X$, if $H^2(X,\A^\times) = 0$ then any family of projective spaces over $X$ is the projectivisation of a bundle of finite free modules over $X$. Next two lemmas are about HoTT only, not SAG.

\begin{lemma}
Assume given a central extension:
\[0 \to A \to K \to G \to 0\]
Then the fibers of the induced map:
\[BK\to BG\]
are $BA$-torsors.
\end{lemma}

\begin{proof}
TODO, not very difficult
\end{proof}

\begin{lemma}\label{trivial-cohomology-lift}
Assume given a central extension:
\[0 \to A \to K \to G \to 0\]
and a type $X$ such that $H^2(X,A)=0$. Then any map from $X$ to $BG$ factors through $BK$
\end{lemma}

\begin{lemma}
TODO, follow from the previous remark.
\end{lemma}

\begin{proposition}\label{auto-projective-central}
We have a central extension:
\[0 \to \A^\times \to GL_{n+1} \to Aut(\bP^n) \to 0\]
\end{proposition}

\begin{proof}
TODO, difficult, says $Aut(\bP^n) = PGL_{n+1}$.
\end{proof}

A projective space is a type merely equal to $\bP^k$ for some $k$.

\begin{corollary}\label{projective-space-are-projectivisation-bundle}
Assume given $X$ such that $H^2(X,\A^\times) = 0$. Then any family of projective spaces over $X$ is the projectivisation of a bundle of finite free modules over $X$.
\end{corollary}

\begin{proof}
There is a well defined function from projective spaces to natural number giving the dimension, as if $\bP^m = \bP^n$ then we have $m=n$, for example by considering the tangent space at any chosen point. Given a family of projective space over $X$, we split $X$ as $\Sigma_{n:\N}X_n$ with the family having dimension $n$ on $X_n$. So we can assume the family having constant dimension.

Now a family of projective spaces of dimension $k$ over $X$ is a map $X\to BAut(\bP^n)$. By \cref{auto-projective-central} and \cref{trivial-cohomology-lift} we know this map lift through $BGL_{n+1}$, but the map:
\[BGL_{n+1} \to BAut(\bP^n)\]
though which we lift is precisely projectivisation, and we can conclude.
\end{proof}

\subsection{Projectivisation of a finite free line bundles over a projective space is projective scheme}

\begin{lemma}\label{projectivisation-line-bundle-over-Pn-projective}
Assume given a bundle $V$ of finite free module of $\bP^n$. Then:
\[\Sigma_{x:\bP^n}\bP(V_x)\]
is a projective scheme.
\end{lemma}

\begin{proof}
TODO, we presumably have to twist enough to get global sections.
\end{proof}

\subsection{Projective schemes are stable under dependent sum}

\begin{lemma}\label{projective-2-cohomology-Ax-vanish}
We have that:
\[H^2(\bP^n,\A^\times) = 0\]
\end{lemma}

\begin{proof}
TODO
\end{proof}

\begin{proposition}
Assume given $X$ a projective scheme and $Y_x$ a family of projective schemes for $x:X$. Then the scheme $\Sigma_{x:X}Y_x$ is projective.
\end{proposition}

\begin{proof}
If $X$ is a closed proposition this is clear because closed propositions have choice. Then it is enough to show that for any family of projective spaces over $\bP^n$ their total space is a projective scheme. This holds since by \cref{projective-2-cohomology-Ax-vanish} and \cref{projective-space-are-projectivisation-bundle} the family is a projectivisation of a bundle of finite free modules, and then we conclude by \cref{projectivisation-line-bundle-over-Pn-projective}.
\end{proof}

\subsection{Projectivisation of a finite free line bundle over an open proposition is a quasi-projective scheme}

TODO

\subsection{Quasi-projective schemes are stable under dependent sum}

TODO
