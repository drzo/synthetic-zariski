
The following is expected to be analogous to completeness in algebraic geometry.
Since it coincides with the defintion of compactness in sythetic topology,
we just call it compact:

\begin{definition}
  A type $X$ is \notion{compact},
  if for any open proposition $U:X\to\Prop$ on $X$,
  the type $(x:X)\to U(x)$ is open.
\end{definition}

\begin{example}
  All finite types are compact, since a conjuction of open propositions is open.
\end{example}

\begin{lemma}
  \label{dependent-sum-compact}
  If $A$ is compact and for each $x:A$ $B(x)$ is a compact type,
  then the dependent sum $(x:A)\times B(x)$ is compact.
\end{lemma}

\begin{proof}
  Let $U:((x:A)\times B(x))\to\Prop$ be open.
  We have to show $(y:(x:A)\times B(x))\to U(y)$ is open.
  By currying, this is $(x:A)\to (z:B(x))\to U(x,z)$.
  By compactness of each $B(x)$, the type $V_x\colonequiv (z:B(x)\to U((x,z))$ is open for all $x:A$.
  So we have to show $(x:A)\to V_x$ is open, but this is the case by compactness of $A$.
\end{proof}

\begin{lemma}
  \label{closed-proposition-compact}
  Any closed proposition is compact.
\end{lemma}

\begin{proof}
  Let $U:C\to\Open$.
  Then, by \cref{commute-open-in-closed}, there is $\tilde{U}:\Open$,
  such that $(x:C)\to U(x)$ is equivalent to $C\to\tilde{U}$.
  By \cref{closed-implies-open-to-or}, the latter is equivalent to $\neg C\vee \tilde{U}$,
  which is open.
\end{proof}

\begin{lemma}
  \label{closed-subtype-compact}
  A closed subtype of a compact type is compact.
\end{lemma}

\begin{proof}
  By \cref{closed-proposition-compact} and \cref{dependent-sum-compact}.
\end{proof}