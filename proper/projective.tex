The goal of this section is to prove that projective space $\bP^n$ is compact,
a classical result of elimination theory.
We will first deal with the case $n = 1$, using algebraic methods, and
then deduce the general case.

The following lemma can be understood as a version of the Euclidean algorithm
for univariate polynomials in the abscence of decidable equality.
\begin{proposition}
Let $p_1, \ldots, p_n : R[X]$. Then we can find propositions
$b_1,\ldots,b_r$, with each $b_i$ of the form $D(u) \wedge v_1 = \ldots = v_k = 0$,
such that $\neg \neg (b_1 \vee \ldots \vee b_r)$,
and for any $i$, we either have that $(p_1,\ldots,p_n) = 0$ if $b_i$ holds,
or we have a natural $d$, such that if $b_i$ holds,
then $(p_1,\ldots,p_n)$ is principal generated by a degree $d$ monic polynomial.
\end{proposition}
\begin{proof}[sketch]
We suppose each $p_i$ is represented by a list of coefficients.
If one of these lists is empty, we simply throw it away.
If there is no $p_i$ left,
then there is nothing left to prove: we take $r = 1$ and $b_1 = D(1)$,
and note that $(p_1,\ldots,p_n) = 0$.
Thus we may suppose $n \ge 1$, and take $i$ such that the formal degree
of $p_i$ is the smallest. Let $u$ be the leading coefficient of $p_i$.
We have $\neg \neg (D(u) \vee u = 0)$. 
In either case, we can make progress: if $D(u)$ and $n = 1$,
then $(p_1,\ldots,p_n)$ is $(p_i)$ with $p_i$ monic;
if $D(u)$ and $n > 1$, then we can divide the other $p_j$ by $p_i$,
decreasing their formal degrees;
and if $u = 0$ then we can reduce the formal degree of $p_i$.
\end{proof}
