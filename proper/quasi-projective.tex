Written by Hugo.

\begin{definition}
A scheme $X$ is quasi-projective if it merely is a closed in an open in a projective space. 
\end{definition}

We think we could equivalently define it as an intersection of an open and a closed. Note that quasi-projective schemes are separated. Are all separated schemes quasi-projective?

\begin{lemma}
Projective schemes and affine schemes are quasi-projective.
\end{lemma}

\begin{lemma}
A closed or open subscheme of a quasi-projective schemes is itself quasi-projective.
\end{lemma}

\begin{proof}
Remember open in closed is intersection of closed and open.
\end{proof}

\begin{proposition}
A quasi-projective scheme is compact if and only if it is projective.
\end{proposition}

\begin{proof}
Consider a quasi-projective compact scheme $X$, then we merely have an embedding:
\[i: X\subset \bP^n\]
which fibers are of the form $\Sigma_{x:U}C(x)$ with $U$ open and $C(x)$ closed for all $x$. Since $i$ goes from a compact type to a type with compact identity types (as closed proposition are compact), its fibers are compact (since compact types are closed by dependent sums). Then by \cref{compact-proposition-scheme-are-closed} we conclude that $i$ is actually a closed embedding, and $X$ is projective.
\end{proof}

Do we have that any proper (i.e. compact and separated) scheme is projective? Without separatedness this fails as the suspension of an open proposition is a compact compact scheme, but it is not always separated (so not always projective). 

