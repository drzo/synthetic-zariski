
\subsection{Reduced schemes}

\rednote{What follows is a not completely satisfactory candidate definition of reduced schemes. Marc Nieper-Wißkirchen and Fabian Endres were involved in finding this notion.}

There is a \emph{candidate} definition of reduced schemes.
The analogue to the classical definition, that an affine scheme is reduced,
if its algebra of functions is reduced, is expected to be useless in the synthetic setup.
We start with a notion which is only suitable for affine schemes\footnote{An example where this fails for general schemes, is $V(X^2)\subseteq\bP^2$.}.

\begin{definition}
  An affine scheme $X=\Spec A$ is \notion{reduced},
  if for all functions $f:A$, nilpotency implies $\neg\neg (f=0)$.
\end{definition}

An alternative, stronger criterion would be that if $f : A$ is nilpotent, then
$f = r_1 a_1 + \ldots + r_n a_n$ with $r_i : R$ nilpotent and $a_i : A$.

\begin{example}
  \begin{enumerate}[(a)]
  \item $\D(1)$ is not reduced.
    The algebra of functions is $R+\varepsilon R$ and we know that $\varepsilon$ is nilpotent and non-zero.
  \item $\A^1$ is reduced. To see this, let $f:R[X]$ be nilpotent.
    Then all coefficients of $f$ are nilpotent and since we proof a double-negation,
    we can assume they are zero.
  \item A basic open $D(f)$ of an affine reduced scheme is reduced:
    If $\left(\frac{a}{f^l}\right)^n=0$, we want to show $\neg\neg \frac{a}{f^l} = 0$.
    Since we want to show a double negated proposition, we can decide if $f$ is regular or nilpontent.
    If it is regular, $f^ka^n=0$ implies $\neg\neg a = 0$ and if it is nilpotent, every function on $D(f)$ is $0$ anyway.
  \item Any closed dense proposition, i.e.\ affine scheme of the form $\Spec (R/(\varepsilon_1,\dots,\varepsilon_n))$ is reduced.
  \item The cross with one infinitesimal axis $\Spec R[X,Y]/(XY,Y^2)$ is not reduced, since $Y$ is a nilpotent, non-zero function.
  \end{enumerate}
\end{example}

From a notion of ``reduced'' for general schemes, we would expect that it is closed under
\begin{enumerate}[(i)]
\item taking open subtypes
\item and finite open unions.
\end{enumerate}

For the notion of affine reduced scheme from above, (i) holds for basic opens of affines,
which is enough to make the following definition well-defined and fullfil the requirements above:

\begin{definition}
  A scheme $X$ is \notion{reduced} if there is a finite affine open cover $X=\bigcup_i U_i$ such that each $U_i$ is reduced.
\end{definition}

In fact, if there is one cover, any cover will be reduced.

\begin{remark}
  For a reduced scheme $X$ we have:
  \begin{enumerate}[(i)]
  \item Any open $U\subseteq X$ is reduced.
  \item For any finite open affine cover of $X=\bigcup U_i$, all $U_i$ are reduced.
  \end{enumerate}
\end{remark}

\begin{proof}
  Any open subscheme of an affine scheme is covered by basic opens, so it is reduced.
  For any cover $(U_i)_i$ of $X$ and a given reduced cover $(V_j)_j$, we have $U_i=\bigcup_j V_j\cap U_i$, so $U_i$ is reduced.
\end{proof}

\begin{example}
  $V(X^2)\subseteq \bP^2$ is not reduced.
\end{example}