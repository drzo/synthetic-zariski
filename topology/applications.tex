
\subsection{Topological Properties of Projective Space}

\begin{proposition}
  \label{projective-space-separated}
  $\bP^n$ is separated.
\end{proposition}

\begin{proof}
  We have to show that $x=y$ is closed for all $x,y:\bP^n$.
  Since we are proving a proposition, we may assume represantatives
  $[x_0:\cdots:x_n]=[y_0:\cdots:y_n]$ and an index $i$ such that $x_i$ is invertible.
  Let $\lambda\colonequiv \frac{y_i}{x_i}$, then $x=y$ is equivalent to
  \[
    \prod_j \lambda x_j=y_j
  \]
  -- which is closed.
\end{proof}

\begin{proposition}[using \axiomref{loc}, \axiomref{sqc}, \axiomref{Z-choice}]%
  \label{projective-space-irreducible}
  $\bP^n$ is irreducible.
\end{proposition}

\begin{proof}
  By \cref{A1-irreducible} and \cref{product-irreducible}, $\A^{n+1}$ is irreducible.
  $\A^{n+1}\setminus\{0\}$ is an open subtype of $\A^{n+1}$,
  so it is also irreducible by \cref{open-subtype-of-irred-is-irred}.
  Finally, the projection $\A^{n+1}\setminus\{0\}\to\bP^n$ is surjective,
  so by \cref{surjection-irreducible}, $\bP^n$ is irreducible.
\end{proof}

The following conclusion could also be drawn from our results about functions on $\bP^n$ in the next section.

\begin{corollary}[using \axiomref{loc}, \axiomref{sqc}, \axiomref{Z-choice}]%
  \label{projective-space-connected}
  $\bP^n$ is connected.
\end{corollary}

\begin{proof}
  Note first that $\bP^n$ is pointed by $[1:0:\dots:0]$.
  By \cref{projective-space-irreducible}, $\bP^n$ is irreducible and
  by \cref{irreducible-implies-connected} any irreducible pointed type is connected.
\end{proof}


\subsection{A Property of $R$}

\begin{theorem}[using \axiomref{sqc}, \axiomref{loc}, \axiomref{Z-choice}]%
  The ring $R$ is not coherent, i.e.\ it is not the case,
  that all finitely generated ideals in $R$ are finitely presented.
\end{theorem}

\begin{proof}
  We will show, that it is not the case,
  that any $R$-module map $R\to R$ has a finitely generated kernel.
  Every $R$-linear map $\varphi:R\to R$
  is of the form $\varphi_x(z) = xz$ for some $x : R$,
  namely $x\colonequiv \varphi(1)$.
  Assume it is always possible to find generators $y_1,\dots,y_n:R$ of the kernel of $\varphi_x$.
  That means there is a map
  \[
    c:\prod_{x:R}
    \left(
      \exists_{y_1,\dots,y_n:R}
      \prod_{z:R}
      \left(
        \varphi_x(z)=0 \text{ iff } \exists_{\lambda_1,\dots,\lambda_n:R}z=\sum_{i=1}^n \lambda_i y_i
      \right)
    \right)
    \rlap{.}
  \]
  By \axiomref{Z-choice} and boundedness (\cref{boundedness}),
  we translate the first ``$\exists$'' to a function $g : D(f) \to R^n$
  on a neighborhood $D(f) \subseteq R$ of $0:R$.
  We know that if $x$ is invertible, then $\ker(\varphi_x)=(0)$, which means $y_i=0$ for all $i$.
  So $g(x)$ must be the 0-vector for all $x:D(f)\cap D(X)$.
  Since $\A^1$ is irreducible by \cref{A1-irreducible},
  $D(X)$ and $D(f)$ are both already dense by being non-empty.
  By \cref{restrict-dense-to-open} $D(f)\cap D(X)$ is dense in $D(f)$,
  so by \cref{dense-closed-n-th-neighborhood}
  applied to $V(g_i) \subseteq D(f)$,
  the entries $g_i(x)$ of $g(x)$ must be nilpotent for all $x:D(f)$.
  But this is a contradiction, since for $x = 0$,
  the kernel of $\varphi_x$ is $R$ and there must be an invertible entry in $g(x)$.
\end{proof}
