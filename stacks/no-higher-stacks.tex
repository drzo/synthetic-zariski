In this section we show that algebraic $n$-stacks are actually algebraic $1$-stacks, when using the definition with cover. Whether this is a good thing is up for debate. I guess the key question is whether stacks defined using cover are well-behaved.

First we give a result from plain HoTT.

\begin{lemma}\label{covering-with-sets-total-space}
Assume $Y$ a type with a surjective map:
\[f:X\to Y\]
where $X$ is a set and $\fib_f(y)$ is a set for all $y:Y$. Then $Y$ is a $1$-type.
\end{lemma}

\begin{proof}
It is enough to prove that for all $y:Y$, we have that $\Omega^2(Y,y)$ is contractible. Indeed let $y:Y$, we can assume $x:X$ such that $f(x)=y$ by surjectivity. We have a fiber sequence:
\[F\to X\to Y\]
where $F$ and $X$ are sets, therefore we have a fiber sequence:
\[\Omega^2(X,x) \to \Omega^2(Y,y) \to \Omega^1(F,x)\]
so that we have a fiber sequence:
\[1\to \Omega^2(Y,y)\to 1\]
and $\Omega^2(Y,y)$ is indeed contractible.
\end{proof}

\begin{remark}
In this situation, $Y$ is not necessarily a set, as seen with:
\[1\to BG\]
for $G$ a group.
\end{remark}

Let $T$ be a topology. We adapt \cref{covering-with-sets-total-space} to $T$-sheaves.

\begin{lemma}\label{covering-with-sets-total-space-sheaf}
Assume $Y$ is a $T$-sheaf with a $T$-surjective map:
\[f:X\to Y\]
where $X$ is a set and $\fib_f(y)$ is a set for all $y:Y$. Then $Y$ is a $1$-type.
\end{lemma}

\begin{proof}
It is enough to prove that for all $y:Y$, we have that $\Omega^2(Y,y)$ is contractible. As $\Omega^2(Y,y)$ is a $T$-sheaf, it being contractible is a $T$-sheaf. So given $y:Y$ we can assume $x:X$ such that $f(x)=y$ by $T$-surjectivity. We have a fiber sequence:
\[F\to X\to Y\]
where $F$ and $X$ are sets, therefore we have a fiber sequence:
\[\Omega^2(X,x) \to \Omega^2(Y,y) \to \Omega^1(F,x)\]
so that we have a fiber sequence:
\[1\to \Omega^2(Y,y)\to 1\]
and $\Omega^2(Y,y)$ is indeed contractible.
\end{proof}

\begin{corollary}
Let $X$ be a algebraic $\infty$-stack defined using $T$-cover, then $X$ is a $1$-type.
\end{corollary}

\begin{proof}
By definition we have an affine étale-surjective map:
\[\Spec(A)\to X\]
 and then we use \cref{covering-with-sets-total-space-sheaf}.
\end{proof}

\begin{remark}
Are all Zariski-stacks schemes?
\end{remark}

%\begin{lemma}\label{n-stacks-affine-cover}
%Let $X$ be an algebraic $n$-stack in the old sense, and assume it satisifes local choice (defined using $T$-cover). Then there merely is a $T$-cover:
%\[\Spec(A)\to X\]
%\end{lemma}

%\begin{proof}
%We proceed by induction, the case $n=-2$ is clear, assume its true for $n-1$ and let $X$ be an $n$-stack.

%We merely have an atlas:
%\[u:\Spec(A)\to X\]
%which is equivalent to the projection:
%\[\big(\sum_{x:X}\sum_{y:\Spec(A)} u(y)=x\big) \to X\]
%but by induction we know for all $y:\Spec(A)$ there are surjective affine maps:
%\[\Spec(B_y)\to u(y)=x\] 
%So by Zariski local choice there is a Zariski cover:
%\[j:\Spec(A')\to \Spec(A)\]
%with a surjective affine map (NO, AS X DO NOT HAVE LOCAL CHOICE WITH AFFINE FIBERS):
%\[\sum_{x:X}\sum_{y:\Spec(A')} \Spec(B_{j(y)}) \to \sum_{x:X}\sum_{y:\Spec(A)} u(y)=x\]
%(as fibers of Zariski cover are affine schemes, and affine schemes are stable by dependent sums). The composite:
%\[\big(\sum_{x:X}\sum_{y:\Spec(A')} \Spec(B_y) \big) \to \Spec(A) \to X\]
%is affine and surjective, and its source is an affine scheme as it is a dependent sums of $\Spec(A)$ by affine schemes.
%\end{proof}


%\begin{remark}
%It is still open whether there are algebraic $\infty$-stack that are not algebraic $1$-stack. I would bet not, it would be somehow interesting if this is not provable. Perhaps it has something to do with hypercompleteness?
%\end{remark}


