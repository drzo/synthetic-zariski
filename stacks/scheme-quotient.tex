The main goal of this section is to show that not every algebraic space is a scheme. To do this we work with scheme quotients.

\subsection{Quotient of an affine scheme by a finite group action}

In all this section we assume $G$ a finite group acting on $\Spec(A)$, such that the algebra of invariant $A^G$ is finitely presented. Ou goal is to prove that:

\[f:\Spec(A)\to \Spec(A^G)\] 

is universal among $G$-invariant maps from $\Spec(A)$ to a scheme.

\begin{remark}
TODO why this hypothesis on $A^G$ f.p.?
\end{remark}

\begin{lemma}\label{injective-on-open}
Assume given $U,V:\OO(\Spec(A^G))$ such that $f^{-1}(U) \subset f^{-1}(V)$. Then $U\subset V$. 
\end{lemma}

\begin{proof}
TODO
\end{proof}

\begin{lemma}\label{surjective-on-open}
For all $U: \OO(\Spec(A))$ that is $G$-invariant, there exists $V:\OO(\Spec(A^G))$ such that $f^{-1}(V)=U$.
\end{lemma}

\begin{proof}
TODO
\end{proof}

\begin{lemma}\label{affine-scheme-quotient-on-open}
For all $V:\Spec(A^G)$ the map:
\[f : f^{-1}(V) \to V\]
is an affine scheme quotient by the $G$-action.
\end{lemma}

\begin{proof}
TODO
\end{proof}

\begin{lemma}\label{piece-wise-contractible}
Assume given $Y$ a set with a dependent set $P(y)$ for $y:Y$. Assume given an open cover $(V_i)_{i:I}$ such that:
\begin{itemize}
\item For all $i:I$ we have that:
\[\prod_{y:V_i} P(y)\]
is contractible.
\item For all $i,j:I$ we have that:
\[\prod_{y:V_i\cap V_j} P(y)\]
is contractible.
\end{itemize}
Then:
\[\prod_{y:Y} P(y)\]
is contractible.
\end{lemma}

\begin{proof}
TODO
\end{proof}

\begin{proposition}
The map:
\[f:\Spec(A)\to \Spec(A^G)\] 
is the scheme quotient of $\Spec(A)$ by the action of $G$.
\end{proposition}

\begin{proof}
Assume given a $G$-invariant map from $\Spec(A)$ to a scheme $X$, we want to prove there is a unique dotted lift in:
\begin{center}
\begin{tikzcd}
\Spec(A)\ar[r,"f"]\ar[rd,swap,"g"] & \Spec(A^G)\ar[d,dotted]\\
& X
\end{tikzcd}
\end{center}
We cover $X$ by affine schemes $U_i$. Then using \cref{surjective-on-open} for all $i$ we choose $V_i$ such that:
\[f^{-1}(V_i) = g^{-1}(U_i)\]
By \cref{injective-on-open} we know that the $V_i$ cover $\Spec(A^G)$. 

By \cref{piece-wise-contractible} it is enough to prove that there is a unique lifting over any $V_i$ and over any $V_i\cap V_j$ in order to conclude. 

\begin{itemize}

\item Let's prove this for $V_i$, assume given $h$ a liftings:
\begin{center}
\begin{tikzcd}
f^{-1}(V_i)\ar[r,"f"]\ar[rd,swap,"g"] & V_i\ar[d,"h'"]\\
& X
\end{tikzcd}
\end{center}
We check that:
\[h(V_i) \subset U_i\] 
and the same with $h'$. This is equivalent to:
\[V_i \subset h^{-1}(U_i)\]
which by \cref{injective-on-open} is equivalent to:
\[f^{-1}(V_i)\subset f^{-1}h^{-1}(U_i)\]
i.e.
\[f^{-1}(V_i)\subset g^{-1}(U_i)\]
which holds by definition. Then we have a triangle:
\begin{center}
\begin{tikzcd}
f^{-1}(V_i)\ar[r,"f"]\ar[rd,swap,"g"] & V_i\ar[d,"h"]\\
& U_i
\end{tikzcd}
\end{center}
with $U_i$ affine so that by \cref{affine-scheme-quotient-on-open} there is a unique such $h$.

\item Now for $V_i\cap V_j$, by the same reasoning we have:
\[h(V_i\cap V_j) \subset U_i\cap U_j\] 
but $U_i\cap U_j$ is affine so we can conclude.
\end{itemize}

\end{proof}

\subsection{Not all algebraic space are schemes}

We assume $0\not=2$, and consider the free action of $\Z/2\Z$ on $\A^\times$ induced by $x\mapsto -x$. Then:
\[\A^\times /_{\Z/2\Z} \]
is an algebraic space. We want to prove it is not a scheme.

\begin{lemma}\abel{scheme-quotient-Ax-Z2Z}
The scheme quotient of this action is the map:
\[\A^\times \to \A^\times\]
\[x\mapsto x^2\]
\end{lemma}

\begin{proof}
TODO
\end{proof}

\begin{proposition}
The algebraic space quotient is not a scheme.
\end{proposition}

\begin{proof}
For all $x,y:\A^\times$, we have that:
\begin{itemize}
\item In the algebraic space quotient $[x]=[y]$ if and only if:
\[\sum_{g: \Z/2\Z} gx = y\]
which is equivalent to:
\[x=y + x=-y\]
\item In the scheme quotient $[x]=[y]$ if and only if:
\[x^2 = y^2\]
by \cref{scheme-quotient-Ax-Z2Z}.
\end{itemize}
So if the algebraic space quotient was a scheme, for any $x,y:\A^\times$ we would have that:
\[x^2=y^2 \leftrightarrow x=y + x=-y\]
We need to prove this leads to a contradiciton. 

In particular this implies that for all $y:R$, we have that:
\[y^2=1 \leftrightarrow y=1\lor y=-1\]
TODO derive contradiction

\end{proof}





