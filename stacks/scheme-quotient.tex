The main goal of this section is to show that not every algebraic space is a scheme. To do this we work with scheme quotients.

\subsection{Quotient of an affine scheme by a finite group action}

In all this section we assume $G$ a finite group acting on $\Spec(A)$, such that the algebra of invariant $A^G$ is finitely presented. Our goal is to prove that:

\[f:\Spec(A)\to \Spec(A^G)\] 

is universal among $G$-invariant maps from $\Spec(A)$ to a scheme.

\begin{remark}
TODO why this hypothesis on $A^G$ f.p.?
\end{remark}

\begin{lemma}\label{injective-on-open}
Assume given $V:\OO(\Spec(A^G))$ such that $f^{-1}(V) = \Spec(A)$. Then $V=\Spec(A^G)$. 
\end{lemma}

\begin{proof}
TODO
\end{proof}

\begin{lemma}\label{surjective-on-open}
For all $U: \OO(\Spec(A))$ that is $G$-invariant, there exists $V:\OO(\Spec(A^G))$ such that $f^{-1}(V)=U$.
\end{lemma}

\begin{proof}
TODO
\end{proof}

\begin{lemma}\label{affine-scheme-quotient-on-open}
For all $V:\Spec(A^G)$ the map:
\[f : f^{-1}(V) \to V\]
is an affine scheme quotient by the $G$-action.
\end{lemma}

\begin{proof}
TODO
\end{proof}

\begin{proposition}
The map:
\[f:\Spec(A)\to \Spec(A^G)\] 
is the scheme quotient of $\Spec(A)$ by the action of $G$.
\end{proposition}

\begin{proof}
Assume given a $G$-invariant map from $\Spec(A)$ to a scheme $X$, we want to prove there is a unique dotted lift in:
\begin{center}
\begin{tikzcd}
\Spec(A)\ar[r,"f"]\ar[rd,swap,"g"] & \Spec(A^G)\ar[d,dotted]\\
& X
\end{tikzcd}
\end{center}
We cover $X$ by affine schemes $U_i$. Then using \cref{surjective-on-open} for all $i$ we choose $V_i$ such that:
\[f^{-1}(V_i) = g^{-1}(U_i)\]
By \cref{injective-on-open} we know that the $V_i$ cover $\Spec(A^G)$. 

By \cref{affine-scheme-quotient-on-open} we know that the type of lifts over any $V_i$ is contractible, as is the type of fillings over any $V_i\cap V_j$. From this we conclude that the type of lifts on the whole $\Spec(A)$ is contractible.
\end{proof}

\subsection{Not all algebraic space are schemes}

We give two different proofs.

INCOMPLETE SECTION We give two different proofs.

\begin{lemma}
Consider $f:\A^1\to X$ with $X$ a scheme, such that for all $x:\A^1\setminus\{0\}$ we have $f(x)=f(-x)$. Then for all $x:\A^1$ we have $f(x)=f(-x)$.
\end{lemma}

\begin{proof}
I CAN'T SEEM TO PROVE THIS.

Let us consider an affine $V$ open in $X$ such that $f(0)\in V$. Write $U=f^{-1}(0)$. 

Then:
\[\prod_{x:\A^1} (U(x)\times x\not=0) \to f(x)=_Uf(-x)\]
But $U$ is affine so $f(x)=_Uf(-x)$ is closed, and $U(x)\times x\not=0$ is open, so that we have:
\[\prod_{x:\A^1} \not(U(x)\times x\not=0) \lor f(x)=_Uf(-x)\]
Then any 
\end{proof}

\begin{corollary}
Assume $2\not=0$. If the quotient $X$ of $\A^1$ by the relation $x=-x$ when $x\not=0$ was a scheme, then the canonical map:
\[X\to \A^1\]
\[[x]\mapsto x^2\]
would be an equivalence.
\end{corollary}

\begin{proof}
TODO
\end{proof}

\begin{proposition}
Assume $2\not=0$. The quotient $X$ of $\A^1$ by the relation $x=-x$ when $x\not=0$ is an algebraic space, but it is not a scheme.
\end{proposition}

\begin{proof}
TODO
\end{proof}

