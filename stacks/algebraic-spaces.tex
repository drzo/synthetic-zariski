\subsection{Definition and basic properties}

\begin{definition}
  An algebraic space is a Deligne-Mumford $0$-stack.
\end{definition}

Map from an algebraic space to $\N$ are bounded, and algebraic spaces have étale-local choice.

\begin{remark}
In the traditional definition it is additionally required that identity types in an algebraic space are schemes. Maybe we will change this later.
\end{remark}

\subsection{Fundamental theorem of algebraic spaces}

In brief, algebraic spaces are quotients of schemes by Deligne-Mumford étale equivalence relations.

\begin{lemma}\label{quotient-by-equivalence-relation}
Assume given a set $X$ then the map:
\[ \sum_{R:X\to X\to \Prop} R\ \mathrm{equivalence\ relation} \to \sum_{Y:\mathrm{Set}} \sum_{p:X\to Y} p\ \mathrm{surjective}\]
\[R \mapsto (X/R,[\_])\]
is an equivalence with inverse the map:
\[(Y,p) \mapsto \lambda x,y. p(x)=p(y)\]
\end{lemma}

\begin{proof}
Plain HoTT, beware that we need to use the set-truncation to define the quotient.
\end{proof}

\begin{definition}
An equivalence relation $R$ on a type $X$ is called:
\begin{itemize}
\item Deligne-Mumford if for all $x,y:X$ the proposition $R(x,y)$ is Deligne-Mumford.
\item Étale if for any $x:X$ its fibers:
\[\sum_{x:X} R(x,y)\]
are formally étale.
\end{itemize}
\end{definition}

\begin{proposition}\label{fundamental-propriety-algebraic-spaces}
Assume given a set $X$, then the following types are equivalent:
\begin{itemize}
\item The type of Deligne-Mumford étale equivalence relation over $X$.
\item The type of sets $Y$ with Deligne-Mumford identity types and a surjective formally étale map from $X$ to $Y$.
\end{itemize}
\end{proposition}

\begin{proof}
By the equivalence in \cref{quotient-by-equivalence-relation}, it is enough to check that:
\begin{itemize}
\item The identity types in $X/R$ are Deligne-Mumford if and only if the relation $R$ is Deligne-Mumford. For any $x,y:X$ we know that:
\[R(x,y) \simeq [x] =_{X/R}[y]\]
so the direct direction is immediate. For the converse we use that being Deligne-Mumford is a proposition and that the map $[\_]:X\to X/R$ is surjective.
\item The fibers of: 
\[[\_]:X\to X/R\] 
are formally étale if and only if the relation $R$ is étale. For any $y:X$ we have that:
\[\sum_{x:X} R(x,y) \simeq \mathrm{fib}_{[\_]}([y])\]
so the direct direction is immediate. Here as well the converse follows from surjectivity of $[\_]$.
\end{itemize}
\end{proof}

\begin{theorem}
A type is an algebraic space if and only if it is merely the quotient of a scheme by a Deligne-Mumford étale equivalence relation.
\end{theorem}

\begin{proof}
This is a direct application of \cref{fundamental-propriety-algebraic-spaces}.
\end{proof}

\subsection{Stability for algebraic spaces}

\begin{lemma}\label{algebraic-space-sum}
Algebraic spaces are stable by dependent sums.
\end{lemma}

\begin{proof}
By \cref{infty-stacks-sum}.
\end{proof}

\begin{lemma}\label{algebraic-space-identity}
Algebraic spaces are stable by identity types.
\end{lemma}

\begin{proof}
By definition.
\end{proof}

By \cref{algebraic-space-sum} and \cref{algebraic-space-identity}, algebraic spaces are stable by finite limits.

\begin{lemma}
Algebraic spaces are stable by quotients by Deligne-Mumford étale equivalence relations.
\end{lemma}

\begin{proof}
Assume given an algebraic space $X$. By \cref{fundamental-propriety-algebraic-spaces} it is enough to check that for any set $Y$ which identity types are Deligne-Mumford with a formally étale surjection:
\[p:X\to Y\]
we have that $Y$ is an algebraic space. Composing a scheme cover for $X$ with $p$ gives a scheme cover for $Y$.
\end{proof}

\subsection{Examples}

\begin{example}
The scheme $\A^1$ quotiented by the relation which identifies $x$ and $-x$ when $x\not=0$ is an algebraic space.
\end{example}

\begin{proof}
We need to show that the equivalence relation $E$ generated by $E(x,-x)$ when $x\not=0$ is Deligne-Mumford and étale. This equivalence relation is:
\[E(x,y) = (x=y) + (x\not=0 \land x=-y)\]
It is clearly a scheme. To check that it is étale, for any $y:R$ we compute:
\[\sum_{x:X}(x=y) + (x\not=0 \land x=-y) \simeq 1 + (y\not=0)\]
which is indeed étale.
\end{proof}

\begin{example}
The scheme:
\[\sum_{x,y:R} xy=0\]
quotiented by the relation which identifies $(x,0)$ and $(0,x)$ when $x\not=0$ is an algebraic space.
\end{example}

\begin{proof}
We need to show that the equivalence relation $E$ generated by $E((x,0),(0,x))$ when $x\not=0$ is Deligne-Mumford and étale. This equivalence relation is:
\[E((x,y),(x',y')) = (x=x'\land y=y') + (x\not=0 \land x=y' \land x'=0)\]
as $x\not=0$ implies $y=0$ since $xy=0$. It is clearly a scheme. To check that it is étale, for any $x',y':R$ such that $x'y'=0$ we compute:
\[\sum_{x,y:R} xy=0 \land E((x,y),(x',y')) \simeq 1+ (y'\not=0)\]
which is indeed étale.
\end{proof}

\subsection{Algebraic spaces and group actions}

\begin{definition}
An action of a group $G$ on a type $X$ is free if for all $x,y:X$ the type:
\[\sum_{g:G}gx=y\]
 is a proposition.
\end{definition}

If $X$ is a set this is the same as asking that for all $x:X$ we have that $gx=x$ implies $g=1$.

\begin{lemma}
Let $G$ be an étale group scheme acting freely on an algebraic space $X$. Then:
\[x,y:X \mapsto \sum_{g:G}gx=_Xy\]
is a schematic étale equivalence relation.
\end{lemma}

\begin{proof}
The type:
\[\sum_{g:G}gx=_Xy\]
is a scheme because it is a dependent sum of schemes. For any $y:Y$ we have:
\[\sum_{x:X}\sum_{g:G}gx=y \simeq G\]
which is assumed étale.
\end{proof}

\begin{corollary}
Algebraic spaces are stabe by quotient by free action of étale group schemes. In particular quotient of schemes by free action of étale group scheme are algebraic spaces.
\end{corollary}

\begin{lemma}
Let $G$ be a finite group acting on an unramified scheme $X$, then $X/G$ is an algebraic space.
\end{lemma}

\begin{proof}
We are considering the quotient of $X$ by the equivalence relation:
\[R(x,y) = \exists (g:G). gx=_Xy\]
this is a schematic relation (even open) because $G$ is finite and identity types in $X$ are open propositions as $X$ is unramified. 

Now we need to show that for all $y:X$ the type:
\[\sum_{x:X}\exists (g:G). gx=_Xy\]
is formally étale:
\begin{itemize}
\item It is formally unramified because it is a subtype of $X$, which is assumed formally unramified.
\item It is formally smooth because we have a surjection:
\[G \simeq \sum_{x:X}\sum_{g:G} gx=_Xy \to \sum_{x:X}\exists (g:G). gx=_Xy\]
and $G$ is finite so it is smooth.
\end{itemize}
\end{proof}
