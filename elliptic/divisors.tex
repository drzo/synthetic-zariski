(This section is preliminary, it is not clear if the notions of divisors presented here are the best we can do synthetically...)


Weil divisors are an important classical tool.
They can be used to describe zero and pole orders at closed points or other closed subspaces. 
In the case of curves over a field in classical algebraic geometry
the set of Weil divisors may be defined as the free abelian group over the closed points of the curve.


This is unlikely to be a good way to go synthetically,
since it is unclear how to produce actual integers from, say, a rational function which describe its zero or pole orders.
As noted by David Jaz Myers \cite{myers-talk2},
it is already not possible to define a degree function with values in the natural numbers and one should therefore pass to upper naturals.
We will start by extending Myers' proposed course of action to generalized integers which can serve as ``coefficients'' for Weil divisors and see how we can assign a Weil divisor to a rational function on a curve.  


Following a definition of Weil divisors, we will define Cartier divisors as well and compare the two notions.


\subsection{Generalized Integers}

The upper naturals are ``increasing'' sequences of propositions.
They can be thought of as upward closed subsets of the naturals
and a upper natural is called bounded, if it is inhabited as a subset.

\begin{definition}
  The \notion{upper naturals}\index{$\N^{\downarrow}$} are the following type:
  \[
  \N^{\downarrow}\colonequiv (s:\N\to \Prop)\times \left((n:\N)\to s(n) \to s(n+1)\right)
  \rlap{.}
  \]
  An upper natural $s:\N^{\downarrow}$ is called \notion{bounded} if there exists an $n$ such that $s(n)$
  \footnote{We will suppress the second component of some sigma types here.}.
  We denote the type of bounded upper naturals with $\N^{\downarrow}_b$\index{$\N^{\downarrow}_b$}.
\end{definition}

As suggested by Myers' we can define addition and multiplication via enriched Day convolution.
The latter means, that we view $\N$ as a category given by the order and with monoidal structure given by $+$ or $\cdot$ and monoidally embedd $\N$ into the $\Prop$-valued functors $\N\to \Prop$.
This leads to the following definitions:

\begin{definition}
  Let $s,s':\N^\downarrow$. Then for all $n:\N$ we define:
  \begin{align*}
    (s+s')(n)&\colonequiv \exists_{(l,k):\N^2} s(l)\wedge s(k)\wedge (l+k\leq n) \\
    (s\cdot s')(n)&\colonequiv \exists_{(l,k):\N^2} s(l)\wedge s(k)\wedge (l\cdot k\leq n) 
  \end{align*}
\end{definition}

One might expect that this inherits the full structure of a semiring from $\N$,
but this is only true if one passes to bounded upper naturals.

\begin{proposition}
  \begin{enumerate}[(a)]
  \item $+$ and $\cdot$ preserve boundedness of upper naturals.
  \item $\N^\downarrow_b$ is a commutative semiring with respect to $+$ and $\cdots$.
  \end{enumerate}
\end{proposition}

\begin{proof}
  \cite{cubical-library} in \href{https://github.com/agda/cubical/blob/master/Cubical/Algebra/CommSemiring/Instances/UpperNat.agda}{Cubical/Algebra/CommSemiring/Instances/UpperNat}.
\end{proof}

The order $\leq$ on $\N$ is decidable and therefore open.
Since $+$ and $\cdot$ only use existence over finite sets and $\wedge$,
being pointwise open is preserved by these operations.
From this we group complete to get generalized integers, which we do not really know how to denote yet.

\begin{definition}
  Let $\N_{\mathcal O,b}^\downarrow\subseteq \N_b^\downarrow$
  be the sub-semiring of pointwise \notion{open bounded upper naturals}\index{$\N_{\mathcal O,b}^\downarrow$}.
  The \notion{generalized integers} are the group completion of $\N_{\mathcal O,b}^\downarrow$, which is again a commutative semiring and essentially small. We will denote the generalized integers with $\Z_{\mathcal O}$\index{$\Z_{\mathcal O}$}.
\end{definition}

\subsection{Weil divisors and rational functions}

We will now aim at definining Weil divisors and start with rational function on a curve.
The definition of curve we will use here is not tested much and therefore preliminary.

\begin{definition}
  A \notion{curve} is an irreducible, inhabited, projective scheme $C$ which is smooth of dimension 1.
\end{definition}

\begin{definition}
  The type of \notion{rational functions} on a type $X$ is the quotient of
  \[
  \mathcal K(X)\colonequiv \{(f,U)\mid U\subseteq X\text{  dense open, } f:U\to R \}
  \]
  by the relation $(f,U) \sim (g,V) \colonequiv (f_{\mid U\cap V}=g_{\mid U\cap V})$.
\end{definition}

\begin{lemma}
  Let $X$ be an irreducible scheme.
  \begin{enumerate}[(a)]
  \item For any $(f,U):\mathcal K(X)$, any point $x:X$ and affine open $V$ containing $x$,
    there is a $g:V\to R$ such that $(f,U)$ is equivalent to $(f_{D(g)},D(g))$ (and $D(g)\subseteq U$).
    In other words: Zariski-locally, $(f,U)$ is of the form $(f_{D(g)},D(g))$
  \item Any non-zero $f:\mathcal K(X)$ has a multiplicative inverse.
  \end{enumerate}
\end{lemma}

\begin{proof}
  Let  $(f,U):\mathcal K(X)$ be a rational function.

  If we assume a point $x:X$ contained in a chart $V$,
  and note $V\cap U = D(g_1,\dots,g_l)$,
  we have a dense open $D(g_i)$ containing $x$.

  So $(f,U)$ is equivalent to $(f_{D(g)},D(g))$ for $g\colonequiv g_i$.

  Now let $f$ be non-zero.
  Without loss of generality we can assume that $f$ is non-zero on all of $U$,
  since by irreducibility of $X$, the non-empty subset where $f \neq 0$ is dense open
  and the intersection of dense opens is dense open.
  But then $f$ is pointwise invertible on $U$ and therefore invertible.
\end{proof}

\begin{lemma}
  Let $X$ be an irreducible scheme smooth of dimension 1.
  If $U\subseteq X$ is dense open and $f:U\to R$ non-zero,
  then for every point $P:X$, there is a natural $k$ such that $f_{\|\D(P,k)}\neq 0$.
\end{lemma}

\begin{proof}
  \rednote{MISSING, Hope this can be done by knowing it for $\A^1$ and moving it around using that a smooth scheme is a manifold...}
\end{proof}

\begin{definition}
  Let $X$ be an inrreducible scheme smooth of dimension 1.
  Then for a non-zero function $f:X\to R$ and $P:X$ we define the zero-order\index{$n_P$} of $f$ at $P$ to be
  the bounded open upper natural number
  \[
  n_p(f)\colonequiv (k:\N)\mapsto f_{\mid \D(P,k+1)}\neq 0
  \rlap{.}
  \]
\end{definition}

\begin{lemma}
  Let $X$ be an inrreducible scheme smooth of dimension 1 and $f,g:X\to R$.
  Then
  \[
  n_P(f\cdot g)=n_P( f) + n_P (g)
  \]
  for all $P:X$.
\end{lemma}


\begin{definition}
  The type of Weil divisors $\mathrm{Cl}(C)$ on a curve $C$ is
  the type of functions $C\to \Z_{\mathcal O}$ vanishing on a open dense subset modulo the subgroup of
  divisors of the form $\mathrm{div}(f)$.
\end{definition}

\begin{theorem}
  Let $X$ be an irreducible scheme smooth of dimension 1.
  Then there is a function $\mathrm{div}:\mathcal K(C)^\times \to \mathrm{Div}(C)$ given locally by:
  \[
  \mathrm{div}(\frac{f}{g})(P)=n_P(f)-n_P(g)
  \]
  And we have:
  \begin{enumerate}[(i)]
  \item There is a dense open $U\subseteq X$ such that $\mathrm{div}(f)=0$.
  \item $\mathrm{div}(f\cdot g)=\mathrm{div}(f)+\mathrm{div}(g)$.
  \end{enumerate}
\end{theorem}

\begin{proof}
  Let $f:\mathcal K(X)^\times$.
  The local requirement defines a unique value $\mathrm{div}(f)$ by multiplicativity of $n_P$.
  \begin{enumerate}[(i)]
  \item This is locally the case since $f$ is of the form $\frac{h}{g}$ and both $D(h)$ and $D(g)$ are dense.
  \item By multiplicativity of $n_P$.
  \end{enumerate}
\end{proof}

\subsection{Cartier divisors}
