% latexmk -pdfxe -pvc main.tex
% latexmk -pdfxe -pvc -interaction=nonstopmode main.tex
\documentclass{../util/zariski}

\title{A Foundation for Synthetic Algebraic Geometry}
\author{Felix Cherubini, Thierry Coquand and Matthias Hutzler}

\begin{document}

\maketitle

\begin{center}
  \color{purple}
  \large{Press CRTL+F5 to clear cached versions}
  \large{(if you are viewing these notes online)}
\end{center}

\begin{abstract}
  The following is work in progress on a development of algebraic geometry, internal to the Zariski topos, building on the work of Kock and Blechschmidt (\cite{kock-sdg}[I.12], \cite{ingo-thesis}).
  The Zariski topos consists of sheaves on the site opposite to the category of finitely presented algebras over a fixed ring, with the Zariski topology, i.e. generating covers are given by localization maps $A\to A_{f_1}$ for finitely many elements $f_1,\dots,f_n$ that generate the ideal $(1)=A\subseteq A$.

  Since we use HoTT, we need a higher version of these sheaves.
  One innovation we present, is the use of higher types to define and reason about cohomology.
  Actually computing cohomology groups, seems to need a principle along the lines of our ``Zariski local choice'' axiom,
  which we justify using a cubical model of homotopy type theory.
\end{abstract}

\tableofcontents

\section*{Introduction}

Algebraic geometry is the study of solutions of non-linear equations using methods from geometry.
Most prominently, algebraic geometry was essential in the proof of Fermat's last theorem by Wiles and Taylor.
The central geometric objects in algebraic geometry are called \emph{schemes}.
Their basic building blocks are called \emph{affine schemes},
where, informally, an affine scheme corresponds to a solution sets of polynomial equations.
While this correspondence is clearly visible in the functorial approach to algebraic geometry and our synthetic approach,
it is somewhat obfuscated in the most commonly used, topological appraoch.

In recent years,
formalization of the intricate notion of affine schemes
received some attention as a benchmark problem
-- this is, however, \emph{not} a problem addressed by this work.
Instead, we use a synthetic approach to algebraic geometry,
very much alike to that of synthetic differential geometry.
This means, while a scheme in classical algebraic geometry is a complicated compound datum,
we work in a setting where schemes are types,
with an additional property that can be defined within our synthetic theory.

Following ideas of Ingo Blechschmidt and Anders Kock  (\cite{ingo-thesis}, \cite{kock-sdg}[I.12]),
we use a base ring $R$, which is local and satisfies an axiom reminiscent of the Kock-Lawvere axiom.
A more general axiom, is called \emph{synthetic quasi coherence (SQC)} by Blechschmidt and
a version quatifying over external algbras is called the \emph{comprehensive axiom}\footnote{
  In \cite{kock-sdg}[I.12], Kock's ``axiom $2_k$'' could equivalently be Theorem 12.2,
  which is exactly our synthetic quasi coherence axiom, except that it only quantifies over external algebras.
}
by Kock.
The exact concise form of the SQC axiom we use, was noted by David Jaz Myers in 2018 and communicated to the first author.

Before we state the SQC axiom, let us take a step back and look at the basic objects of study in algebraic geometry,
solutions of polynomial equations.
Given a system of polynomial equations
\begin{align*}
  p_1(X_1, \dots, X_n) &= 0\rlap{,} \\
  \vdots\quad\quad\;\;   \\
  p_m(X_1, \dots, X_n) &= 0\rlap{,}
\end{align*}
the solution set
$\{ x : R^n \mid \forall i.\; p_i(x_1, \dots, x_n) = 0 \}$
is in canonical bijection to the set of $R$-algebra homomorphisms
\[ \Hom_R(R[X_1, \dots, X_n]/(p_1, \dots, p_m), R) \]
by identifying a solution $(x_1,\dots,x_n)$ with the homomorphism that maps each $X_i$ to $x_i$.
Conversely, for any $R$-algebra $A$, which is merely of the form $R[X_1, \dots, X_n]/(p_1, \dots, p_m)$,
we define the \emph{spectrum} of $A$ to be
\[
  \Spec A \colonequiv \Hom_R(A, R)
  \rlap{.}
\]
In contrast to classical, non-synthetic algebraic geometry,
where this set needs to be equipped with additional structure,
we postulate axioms that will ensure that $\Spec A$ has the expected geometric properties.
Namely, SQC is the statement that, for all finitely presented $R$-algebras $A$, the canonical map
  \begin{align*}
    A&\xrightarrow{\sim} (\Spec A\to R) \\
    a&\mapsto (\varphi\mapsto \varphi(a))
  \end{align*}
is an equivalence.
A prime example of a spectrum is $\A^1\colonequiv \Spec R[X]$,
which turns out to be the underlying set of $R$.
With the SQC axiom,
\emph{any} function $f:\A^1\to \A^1$ is given as a polynomial with coefficients in $R$.
In fact, all functions between affine schemes are given by polynomials.
Furthermore, for any affine scheme $\Spec A$,
the axiom ensures that
the algebra $A$ can be reconstructed as the algebra of functions $\Spec A \to R$,
therefore establishing a duality between affine schemes and algebras.

The Kock-Lawvere axiom used in synthetic differential geometry,
might be stated as the SQC axiom restricted to (external) \emph{Weil-algebras},
whose spectra correspond to pointed infinitesimal spaces.
These spaces can be used in both, synthetic differential and algebraic geometry,
in very much the same way.

In the accompanying formalization \cite{formalization} of some basic results,
we use a setup which was already proposed by David Jaz Myers
in a conference talk (\cite{myers-talk1, myers-talk2}).
On top of Myers' ideas,
we were able to define schemes, develop some topological properties of schemes,
and construct projective space.

An important, not yet formalized result
is the construction of cohomology groups.
This is where the \emph{homotopy} type theory really comes to bear --
instead of the hopeless adaption of classical, non-constructive definitions of cohomology,
we make use of higher types,
for example the $k$-th Eilenberg-MacLane space $K(R,k)$ of the group $(R,+)$.
As an analogue of classical cohomology with values in the structure sheaf,
we then define cohomology with coefficients in the base ring as:
\[
  H^k(X,R):\equiv \|X\to K(R,k)\|_0
  \rlap{.}
\]
This definition is very convenient for proving abstract properties of cohomology.
For concrete calculations we make use of another axiom,
which we call \emph{Zariski-local choice}.
While this axiom was conceived of for exactly these kind of calculations,
it turned out to settle numerous questions with no apparent connection to cohomology.
One example is the equivalence of two notions of \emph{open subspace}.
A pointwise definition of openness was suggested to us by Ingo Blechschmidt and
is very convenient to work with.
However, classically, basic open subsets of an affine scheme are given
by functions on the scheme and the corresponding open is morally the collection of points where the function does not vanish.
With Zariski-local choice, we were able to show that these notions of openness agree in our setup.

Apart from SQC, locality of the base ring $R$ and Zariski-local choice,
we only use homotopy type theory, including univalent universes, truncations and some very basic higher inductive types.
Roughly, Zariski-local choice states, that any surjection into an affine scheme merely has sections on a \emph{Zariski}-cover.
The latter, internal, notion of cover corresponds quite directly to the covers in the site of the \emph{Zariski topos},
which we use to construct a model of homotopy type theory with our axioms.

More precisely, we can use the \emph{Zariski topos} over any base ring.
Toposes built using other Grothendieck topologies, like for example the étale topology, are not compatible with Zariski-local choice.
We did not explore whether an analogous setup can be used for derived algebraic geometry
\footnote{Here, the word ``derived'' refers to the rings the algebraic geometry is built up from -- instead of the 0-truncated rings we use, ``derived'' algebraic geometry would use simplicial or spectral rings.
  Sometimes, ``derived'' refers to homotopy types appearing in ``the other direction'', namely as the values of the sheaves that used.
  In that direction, our theory is already derived, since we use homotopy type theory.
  Practically that means that we expect no problems when expanding our theory of synthetic schemes to what classic algebraic geometers
  call ``stacks''.
}
-- meaning that the 0-truncated rings we used are replaced by higher rings.
This is only because for a derived approach, would have to work with higher monoids, which is currently infeasible
-- we are not aware of any obstructions for, say, an SQC axiom holding in derived algebraic geometry.

In total, the scope of our theory so far, includes quasi-compact, quasi-separated schemes of finite type over an arbitrary ring.
These are all finiteness assumptions, that were chosen for convenience and include examples like closed subspaces of projective space,
which we want to study in future work, as example applications.
So far, we know that basic internal constructions, like affine schemes, correspond to the correct classical external constructions.
This can be expanded using our model, which is of course also important to ensure the consistency of our setup.


\section*{Formalization}
There is a related formalization project, which, at the time of writing,
contains the construction of projective $n$-space $\bP^n$ as a scheme.
The code may be found here:
\begin{center}
  \url{https://github.com/felixwellen/synthetic-geometry}
\end{center}
It makes extensive use of the algebra part of the cubical-agda library:
\begin{center}
  \url{https://github.com/agda/cubical}
\end{center}
-- which contains many contributions, in particular,
on finitely presented algebras and related concepts,
which where made in the scope of that project.

\section*{Acknowlegements}
We use work from Ingo Blechschmidt's PhD thesis, section 18 as a basis.
This includes in particular the synthetic quasi-coherence axiom and the assumption that the base ring is local.
David Jaz Myers had the idea to use Blechschmidt's ideas in homotopy type theory
and presented his ideas 2019 at the workshop ``Geometry in Modal Homotopy Type Theory'' in Pittsburgh.
Myers ideas include the algebra-setup we used in our formalization.

In December 2022, there was a mini-workshop in Augsburg, which helped with the development of this work.
We thank Jonas Höfer and Lukas Stoll for spotting a couple of small errors.


\section{Preliminaries}
\subsection{Homotopy type theory}

\begin{lemma}%
  \label{kraus-glueing}
  Let $X$ and $I$ be types.
  For $(U_i:X \to \Prop)_{i:I}$ and $P:U_i\to \nType{0}$, we have the following glueing property: \\
  If for each $i:I$ there is a dependent function $s_i:(x:U_i)\to P(x)$ together with
  proofs of equality on intersections $p_{ij}:(x:U_i\cap U_j)\to (s_i(x)=s_j(y))$,
  then there is a globally defined dependent function $s:(x:X) \to P(x)$,
  such that for all $x:X$ and $i:I$ we have $U_i(x) \to s(x)=s_i(x)$
\end{lemma}

\begin{proof}
  We define $s$ pointwise.
  Let $x:X$.
  Using a Lemma of Kraus and the $p_{ij}$, we get a factorization
  \[ \begin{tikzcd}[row sep=0mm]
    \sum_{i:I} U_i(x) \ar[rr, "s_{\pi_1(\_)}(x)"]\ar[rd] & & P(x) \\
    & \|\sum_{i:I} U_i(x)\|_{-1}\ar[ru,dashed] &
  \end{tikzcd} \]
-- which defines a unique value $s(x):P(x)$.
\end{proof}

\subsection{Subtypes and Logic}

We use the notation $\exists_{x:X}P(x)\colonequiv \|\sum_{x:X}P(x)\|$.
We use $+$ for the coproduct of types and for types $A,B$ we write
\[ A\vee B\colonequiv \| A+B \|\rlap{.}\]

We will use subtypes extensively.

\begin{definition}
  Let $X$ be a type.
  A \notion{subtype} of $X$ is a function $U:X\to\Prop$ to the type of propositions.
  We write $U\subseteq X$ to indicate that $U$ is as above.
  If $X$ is a set, a subtype may be called \notion{subset} for emphasis.
\end{definition}

We will freely switch between subtypes $U:X\to\Prop$ and the corresponding embeddings
\[
  \begin{tikzcd}
    \sum_{x:X}U(x) \ar[r,hook] & X
  \end{tikzcd}
  \rlap{.}
\]
In particular, if we write $x:U$ for a subtype $U:X\to\Prop$, we mean that $x:\sum_{x:X}U(x)$ -- but we might silently project $x$ to $X$.

\begin{definition}
  Let $I$ and $X$ be types and $U_i:X\to\Prop$ a subtype for any $i:I$.
  \begin{enumerate}[(a)]
  \item The \notion{union} $\bigcup_{i:I}U_i$ is the subtype $(x:X)\mapsto \exists_{i:I}U_i(x)$.
  \item The \notion{intersection} $\bigcap_{i:I}U_i$ is the subtype $(x:X)\mapsto\prod_{i:I}U_i(x)$.
  \end{enumerate}
\end{definition}

We will use common notation for finite unions and intersections.
The following formula hold:

\begin{lemma}
  Let $I$, $X$ be types, $U_i:X\to\Prop$ a subtype for any $i:I$ and $V,W$ subtypes of $X$.
  \begin{enumerate}[(a)]
  \item Any subtype $P:V\to\Prop$ is a subtype of $X$ given by $(x:X)\mapsto\sum_{x:V}P(x)$.
  \item $V\cap \bigcup_{i:I} U_i=\bigcup (V\cap U_i)$.
  \item If $\bigcup_{i:I}U_i=X$ we have $V=\bigcup_{i:I}U_i\cap V$.
  \item If $\bigcup_{i:I}U_i=\emptyset$, then $U_i=\emptyset$ for all $i:I$.
  \end{enumerate}
\end{lemma}

\subsection{Algebra}

\begin{definition}%
  A commutative ring $R$ is \notion{local} if $1\neq 0$ in $R$ and
  if for all $x,y:R$ such that $x+y$ is invertible, $x$ is invertible or $y$ is invertible.
\end{definition}

\begin{definition}%
  Let $R$ be a commutative ring.
  A \notion{finitely presented} $R$-algebra is an $R$-algebra $A$,
  such that there merely are natural numbers $n,m$ and polynomials $f_1,\dots,f_m:R[X_1,\dots,X_n]$
  and an equivalence of $R$-algebras $A\simeq R[X_1,\dots,X_n]/(f_1,\dots,f_m)$.
\end{definition}

\begin{definition}%
  \label{regular-element}
  Let $A$ be a commutative ring.
  An element $r:A$ is \notion{regular},
  if the multiplication map $r\cdot\_:A\to A$ is injective.
\end{definition}

\begin{lemma}%
  \label{units-products-regular}
  Let $A$ be a commutative ring.
  \begin{enumerate}[(a)]
  \item All units of $A$ are regular.
  \item If $f$ and $g$ are regular, their product $fg$ is regular.
  \end{enumerate}
\end{lemma}

\begin{example}
  The monomials $X^k:A[X]$ are regular.
\end{example}

\begin{lemma}%
  \label{fg-ideal-local-global}
  Let $A$ be a commutative ring and $f_1,\dots,f_n:A$.
  For finitely generated ideals $I_i\subseteq A_{f_i}$,
  such that $A_{f_if_j}\cdot I_i=A_{f_if_j}\cdot I_j$ for all $i,j$,
  there is a finitely generated ideal $I\subseteq A$,
  such that $A_{f_i}\cdot I=I_i$ for all $i$.
\end{lemma}

\begin{proof}
  Choose generators 
  \[ \frac{g_{i1}}{1},\dots,\frac{g_{ik_i}}{1} \]
  for each $I_i$.
  These generators will still generate $I_i$, if we multiply any of them with any power of the unit $\frac{f_i}{1}$.
  Now
  \[ A_{f_if_j}\cdot I_i\subseteq A_{f_if_j}\cdot I_j \]
  means that for any $g_{ik}$, we have a relation
  \[ (f_if_j)^l g_{ik}=\sum_{l}h_{l}g_{jl}\]
  for some power $l$ and coefficients $h_{l}:A$.
  This means, that $f_i^lg_{ik}$ is contained in $I_j$.
  Multiplying $f_i^lg_{ik}$ with further powers of $f_i$ or multiplying $g_{jl}$ with powers of $f_j$ does not change that.
  So we can repeat this for all $i$ and $k$ to arrive at elements $\tilde{g_{ik}}:A$,
  which generate an ideal $I\subseteq A$ with the desired properties.
\end{proof}

\ignore{
    - injective/embedding/-1-truncated map
  pushouts:
    - inclusions are jointly surjective,
    - pushouts of embeddings between sets are sets
  subtypes:
    - embeddings (composition, multiple definitions, relation to injection)  
    - we freely switch between predicates and types
    - subtypes of subtypes are subtypes
  pullbacks:
    - pasting (reference)
    - pullback of subtype = composition
  algebra:
    - free comm algebras, quotients
    - other definitions of polynomials
    - fp closed under: quotients, adjoining variables, tensor products
}

\section{Axioms}
\subsection{Statement of the axioms}
We always assume there is a commutative ring $R$.
Sometimes we will assume $R$ has additional properties, or, more generally,
axioms hold that involve $R$.
We will always mention which of these axiom are needed to prove each statement,
by listing the shorthands introduced in the axioms below.

\begin{axiom}[Loc]%
  \label{loc}\index{Loc}
  $R$ is a local ring.
\end{axiom}

\begin{axiom}[SQC]%
  \label{sqc}\index{sqc}
  For any finitely presented $R$-algebra $A$, the homomorphism
  \[ a \mapsto (\varphi\mapsto \varphi(a)) : A \to (\Spec A \to R)\]
  is an isomorphism of $R$-algebras.
\end{axiom}

\begin{axiom}[Z-choice]%
  \label{Z-choice}\index{Z-choice}
  Let $A$ be a finitely presented $R$-algebra
  and let $B : \Spec A \to \mU$ be a family of inhabited types.
  Then there merely exists
  a finite list of coprime elements $f_1, \dots, f_n \in A$
  together with dependent functions $s_i : \Pi_{x : D(f_i)} B(x)$.
  As a formula:
  \[ (\Pi_{x : \Spec A} \propTrunc{B(x)}) \to
     \propTrunc{ \Sigma_{n : \N} \Sigma_{f_1, \dots, f_n : A}
      ((f_1, \dots, f_n) = (1)) \times
      \Pi_i \Pi_{x : D(f_i)} B(x) }
     \rlap{.}
  \]
\end{axiom}

\subsection{First consequences}

\begin{proposition}[using \axiomref{sqc}]%
  For all finitely generated $R$-algebras $A$ and $B$ we have
  \[ \Hom(\Spec B, \Spec A)=\Hom_{\Alg{R}}(A,B)\]
  -- where the equality is induced by exponentiation with $R$.
\end{proposition}

\begin{proposition}[using \axiomref{sqc}, \axiomref{loc}]%
  \label{nilpotence-double-negation}\label{non-zero-invertible}\label{generalized-field-property}
  
  \begin{enumerate}[(a)]
  \item An element $x:R$ is nilpotent,
    if and only if $\neg \neg (x=0)$.
  \item An element $x:R$ is invertible,
    if and only if $x\neq 0$.
  \item A vector $x:R^n$ is non-zero,
    if and only if one of its entries is invertible.
  \end{enumerate}
\end{proposition}


\section{Affine schemes}
\subsection{Affine-open subtypes}

We only talk about affine schemes of finite type, i.e. schemes of the form $\Spec A$ (\cref{spec}),
where $A$ is a finitely presented algebra.

\begin{definition}%
  A type $X$ is \notion{(qc-)affine},
  if there is a finitely presented $R$-algebra $A$, such that $X=\Spec A$.
\end{definition}

\begin{proposition}%
  Let $X$ be a type.
  The type of all finitely presented $R$-algebras $A$, such that $X=\Spec A$, is a proposition.
\end{proposition}

When we write ``$\Spec A$'' we implicitly assume $A$ is a finitely presented $R$-algebra.
Recall from \cref{standard-open-subset}
that the standard open subset $D(f) \subseteq \Spec A$
is given by $D(f)(x)\colonequiv \inv(f(x))$.

\begin{example}[using \axiomref{loc}, \axiomref{sqc}]
  For $a_1, \dots, a_n : R$, we have
  \[ D((X - a_1) \cdots (X - a_n)) = \A^1 \setminus \{ a_1, \dots, a_n \} \rlap{.}\]
  Indeed,
  for any $x : \A^1$,
  $((X - a_1) \dots (X - a_n))(x)$ is invertible if and only if
  $x - a_i$ is invertible for all $i$.
  But by \cref{non-zero-invertible}
  this means $x \neq a_i$ for all $i$.
\end{example}

\begin{definition}%
  \label{affine-open}
  Let $X=\Spec A$.
  A subtype $U:X\to\Prop$ is called \notion{affine-open},
  if one of the following logically equivalent statements holds:
  \begin{enumerate}[(i)]%
  \item $U$ is the union of finitely many affine standard opens.
  \item There are $f_1,\dots,f_n:A$ such that
    \[U(x) \Leftrightarrow \exists_{i} f_i(x)\neq 0 \]
  \end{enumerate}
\end{definition}

We sometimes write $D(f_1, \dots, f_n) \colonequiv D(f_1) \cup \dots \cup D(f_n)$
for a finite union of standard opens.
Note that in general, affine-open subtypes do not need to be affine
-- this is why we use the dash ``-''.

We will introduce a more general definition of open subtype in \cref{qc-open}
and show in \cref{qc-open-affine-open}, that the two notions agree on affine schemes.

\begin{proposition}
  Let $X = \Spec A$ and $f : A$.
  Then $D(f) = \Spec A[f^{-1}]$.
\end{proposition}

\begin{proof}
  \[ D(f) =
     \sum_{x : X} D(f)(x) =
     \sum_{x : \Spec A} \inv(f(x)) =
     \sum_{x : \Hom_{\Alg{R}}(A, R)} \inv(x(f)) =
     \Hom_{\Alg{R}}(A[f^{-1}], R) =
     \Spec A[f^{-1}]
     \]
\end{proof}

Affine-openness is transitive in the following sense:

\begin{lemma}%
  \label{affine-open-trans}
  Let $X=\Spec A$ and $D(f)\subseteq X$ be a standard open.
  Any affine-open subtype $U$ of $D(f)$ is also affine-open in $X$.
\end{lemma}

\begin{proof}
  It is enough to show the statement for $U=D(g)$, $g:A_f$.
  Then
  \[ g=\frac{h}{f^k}\rlap{.}\]
  Now $D(hf)$ is an affine-open in $X$,
  that coincides with $U$: \\
  Let $x:X$, then $(hf)(x)$ is invertible, if and only if both $h(x)$ and $f(x)$ are invertible.
  The latter means $x:D(f)$, so we can interpret $x$ as a homorphism from $A_f$ to $R$.
  Then $x:D(g)$ means $x(g)$ is invertible, which is equivalent to $x(h)$ being invertible,
  since $x(f)^k$ is invertible anyway.
\end{proof}

\begin{lemma}[using \axiomref{loc}, \axiomref{sqc}]%
  \label{standard-open-empty}
  Let $X=\Spec A$ be an affine scheme and $D(f)\subseteq X$ a standard open,
  then $D(f)=\emptyset$, if and only if, $f$ is nilpotent.
\end{lemma}

\begin{proof}
  Since $D(f)=\Spec A_f$, by \cref{weak-nullstellensatz}, we know $D(f)=\emptyset$,
  if and only if, $A_f=0$.
  The latter is equivalent to $f$ being nilpotent.
\end{proof}

More generally,
the Zariski-lattice consisting of the radicals
of finitely generated ideals of a finitely presented $R$-algebra $A$,
coincides with the lattice of open subtypes.
This means, that internal to the Zariski-topos,
it is not neccessary to consider the full Zariski-lattice for a constructive treatment of schemes.

\begin{lemma}[using \axiomref{sqc}]%
  Let $A$ be a finitely presented $R$-algebra
  and let $f, g_1, \dots, g_n \in A$.
  Then we have $D(f) \subseteq D(g_1, \dots, g_n)$
  as subsets of $\Spec A$
  if and only if $f \in \sqrt{(g_1, \dots, g_n)}$.
\end{lemma}

\begin{proof}
  Since $D(g_1, \dots, g_n) = \{\, x \in \Spec A \mid x \notin V(g_1, \dots, g_n) \,\}$,
  the inclusion $D(f) \subseteq D(g_1, \dots, g_n)$
  can also be written as
  $D(f) \cap V(g_1, \dots, g_n) = \varnothing$, \rednote{(UNDEFINED NOTATION ``$V(g_1,\dots,g_m)$'')} that is,
  $\Spec((A/(g_1, \dots, g_n))[f^{-1}]) = \varnothing$.
  By (\axiomref{sqc})
  this means that the finitely presented $R$-algebra $(A/(g_1, \dots, g_n))[f^{-1}]$
  is zero.
  And this is the case if and only if $f$ is nilpotent in $A/(g_1, \dots, g_n)$,
  that is, if $f \in \sqrt{(g_1, \dots, g_n)}$, as stated.
\end{proof}

In particular,
we have $\Spec A = \bigcup_{i = 1}^n D(f_i)$
if and only if $(f_1, \dots, f_n) = (1)$.

\subsection{Pullbacks of affine schemes}

\begin{lemma}%
  \label{affine-product}
  The product of two affine schemes is again an affine scheme,
  namely
  $\Spec A \times \Spec B = \Spec (A \otimes_R B)$.
\end{lemma}

\begin{proof}
  By the universal property of the tensor product $A \otimes_R B$.
\end{proof}

More generally we have:

\begin{lemma}[using \axiomref{sqc}]%
  \label{affine-fiber-product}
  Let $X=\Spec A,Y=\Spec B$ and $Z=\Spec C$ be affine schemes
  with maps $f:X\to Z$, $g:Y\to Z$.
  Then the pullback of this diagram is an affine scheme given by $\Spec (A\otimes_C B)$.
\end{lemma}

\begin{proof}
  The maps $f:X\to Z$, $g:Y\to Z$ are induced by $R$-algebra homomorphisms $f^*:A\to R$ and $g^*:B\to R$.
  Let
  \[ (h,k,p) : \Spec A \times_{\Spec C} \Spec B \]
  with $p:h\circ f^*=k\circ g^* $.
  This defines a $R$-cocone on the diagram
  \[
    \begin{tikzcd}
      A & C\ar[r,"g^*"]\ar[l,"f^*",swap] & B
    \end{tikzcd}
  \]
  Since $A\otimes_C B$ is a pushout in $R$-algebras,
  there is a unique $R$-algebra homomorphism $A\otimes_C B \to R$ corresponding to $(h,k,p)$.
\end{proof}

\subsection{Boundedness of functions to $\N$}

While the axiom \axiomref{sqc}
describes functions on an affine scheme
with values in $R$,
we can generalize it to functions taking values
in another finitely presented $R$-algebra,
as follows.

\begin{lemma}[using \axiomref{sqc}]%
  \label{algebra-valued-functions-on-affine}
  For finitely presented $R$-algebras $A$ and $B$,
  the function
  \begin{align*}
    A \otimes B &\xrightarrow{\sim} (\Spec A \to B) \\
    c &\mapsto (\varphi \mapsto (\varphi \otimes B)(c))
  \end{align*}
  is a bijection.
\end{lemma}

\begin{proof}
  We recall $\Spec (A \otimes B) = \Spec A \times \Spec B$
  from \cref{affine-product}
  and calculate as follows.
  \begin{align*}
    A \otimes B
    &=& (\Spec (A \otimes B) \to R)
    &=& (\Spec A \times \Spec B \to R)
    &=& (\Spec A \to (\Spec B \to R))
    &=& (\Spec A \to B) \\
    c
    &\mapsto& (\chi \mapsto \chi(c))
    &\mapsto& ((\varphi, \psi) \mapsto (\varphi \otimes \psi)(c))
    &\mapsto& (\varphi \mapsto (\psi \mapsto (\varphi \otimes \psi)(c)))
    &\mapsto& (\varphi \mapsto (\varphi \otimes B)(c))
  \end{align*}
  The last step is induced by the identification
  $B = (\Spec B \to R),\, b \mapsto (\psi \mapsto \psi(b))$,
  and we use the fact that
  $\psi \circ (\varphi \otimes B) = \varphi \otimes \psi$.
\end{proof}

\begin{lemma}[using \axiomref{sqc}]%
  \label{eventually-vanishing-sequence-on-affine}
  Let $A$ be a finitely presented $R$-algebra
  and let $s : \Spec A \to (\N \to R)$
  be a family of sequences,
  each of which eventually vanishes:
  \[ \prod_{x : \Spec A} \propTrunc{\sum_{N : \N} \prod_{n \geq N} s(x)(n) = 0} \]
  Then there merely exists one number $N : \N$
  such that $s(x)(n) = 0$ for all $x : \Spec A$ and all $n \geq N$.
\end{lemma}

\begin{proof}
  The set of eventually vanishing sequences $\N \to R$
  is in bijection with the set $R[X]$ of polynomials,
  by taking the entries of a sequence as the coefficients of a polynomial.
  So the family of sequences $s$
  is equivalently a family of polynomials $s : \Spec A \to R[X]$.
  Now we apply \cref{algebra-valued-functions-on-affine} with $B = R[X]$
  to see that such a family corresponds to a polynomial $p : A[X]$.
  Note that for a point $x : \Spec A$,
  the homomorphism
  \[ x \otimes R[X] : A[X] = A \otimes R[X] \to R \otimes R[X] = R[X] \]
  simply applies the homomorphism $x$ to every coefficient of a polynomial,
  so we have $(s(x))_n = x(p_n)$.
  This concludes our argument,
  because the coefficients of $p$,
  just like any polynomial,
  form an eventually vanishing sequence.
\end{proof}

\begin{theorem}[using (\axiomref{loc}), (\axiomref{sqc})]%
  \label{boundedness}
  Let $A$ be a finitely presented $R$-algebra.
  Then every function $f : \Spec A \to \N$ is bounded:
  \[ \Pi_{f : \Spec A \to \N} \propTrunc{\Sigma_{N : \N} \Pi_{x : \Spec A} f(x) \le N}
     \rlap{.} \]
\end{theorem}

\begin{proof}
  Given a function $f : \Spec A \to \N$,
  we construct the family $s : \Spec A \to (\N \to R)$
  of eventually vanishing sequences
  given by
  \[
    s(x)(n) \colonequiv
    \begin{cases}
      \,1 &\text{if $n < f(x)$}\\
      \,0 &\text{else} \rlap{.}
    \end{cases}
  \]
  Since $0 \neq 1 : R$ by \axiomref{loc},
  we in fact have $s(x)(n) = 0$ if and only if $n \geq f(x)$.
  Then the claim follows from \cref{eventually-vanishing-sequence-on-affine}.
\end{proof}

If we also assume the axiom \axiomref{Z-choice},
we can formulate the following simultaneous strengthening
of \cref{eventually-vanishing-sequence-on-affine}
and \cref{boundedness}.

\begin{proposition}[using \axiomref{loc}, \axiomref{sqc}, \axiomref{Z-choice}]%
  \label{strengthened-boundedness}
  Let $A$ be a finitely presented $R$-algebra.
  Let $P : \Spec A \to (\N \to \Prop)$
  be a family of upwards closed, merely inhabited subsets of $\N$.
  Then the set
  \[ \bigcap_{x : \Spec A} P(x) \subseteq \N \]
  is merely inhabited.
\end{proposition}

\begin{proof}
  By \axiomref{Z-choice},
  there merely exists a cover
  $\Spec A = \bigcup_{i = 1}^n D(f_i)$
  and functions $p_i : D(f_i) \to \N$
  such that $p_i(x) \in P(x)$ for all $x : D(f_i)$.
  By \cref{boundedness},
  every $p_i : D(f_i) = \Spec A[f_i^{-1}] \to \N$
  is merely bounded by some $N_i : \N$,
  and then $\mathrm{max}(N_1, \dots, N_n) \in P(x)$ for all $x : \Spec A$.
\end{proof}

\subsection{Properties of the ring $R$}

We adopt the following definition from
\cite[Section IV.8]{lombardi-quitte}.

\begin{definition}%
  \label{zero-dimensional-ring}
  A ring $A$ is \notion{zero-dimensional}
  if for all $x : A$
  there exists $a : A$ and $k : \N$
  such that $x^k = a x^{k + 1}$.
\end{definition}

\begin{lemma}[using \axiomref{loc}, \axiomref{sqc}, \axiomref{Z-choice}]%
  \label{R-not-zero-dimensional}
  The ring $R$ is not zero-dimensional.
\end{lemma}

\begin{proof}
  Assume that $R$ is zero-dimensional,
  so for every $f : R$ there merely is some $k : \N$ with $f^k \in (f^{k + 1})$.
  We note that $R = \A^1$ is an affine scheme and
  that if $f^k \in (f^{k + 1})$,
  then we also have $f^{k'} \in (f^{k' + 1})$ for every $k' \geq k$.
  This means that we can apply \cref{strengthened-boundedness}
  and merely obtain a number $K : \N$
  such that $f^K \in (f^{K + 1})$ for all $f : R$.
  In particular, $f^{K + 1} = 0$ implies $f^K = 0$,
  so the canonical map
  $\Spec R[X]/(X^K) \to \Spec R[X]/(X^{K + 1})$
  is a bijection.
  But this is a contradiction,
  since the homomorphism $R[X]/(X^{K + 1}) \to R[X]/(X^K)$
  is not an isomorphism.
\end{proof}

The following lemma,
which is a variant of \cite{ingo-thesis}[Proposition 18.32],
shows that $R$ is in a weak sense algebraically closed.

\begin{lemma}[using \axiomref{loc}, \axiomref{sqc}]%
  \label{polynomials-notnot-decompose}
  Let $f : R[X]$ be a polynomial.
  Then it is not not the case that:
  either $f = 0$ or
  $f = \alpha \cdot {(X - a_1)}^{e_1} \dots {(X - a_n)}^{e_n}$
  for some $\alpha : R^\times$,
  $e_i \geq 1$ and pairwise distinct $a_i : R$.
\end{lemma}

\begin{proof}
  Let $f : R[X]$ be given.
  Since our goal is a proposition,
  we can assume we have a bound $n$ on the degree of $f$,
  so
  \[ f = \sum_{i = 0}^n c_i X^i \rlap{.} \]
  Since our goal is even double-negation stable,
  we can assume $c_n = 0 \lor c_n \neq 0$
  and by induction $f = 0$ (in which case we are done)
  or $c_n \neq 0$.
  If $n = 0$ we are done,
  setting $\alpha \colonequiv c_0$.
  Otherwise,
  $f$ is not invertible (using $0 \neq 1$ by (\axiomref{loc})),
  so $R[X]/(f) \neq 0$,
  which by (\axiomref{sqc}) means that
  $\Spec(R[X]/(f)) = \{ x : R \mid f(x) = 0 \}$
  is not empty.
  Using the double-negation stability of our goal again,
  we can assume $f(a) = 0$ for some $a : R$
  and factor $f = (X - a_1) f_{n - 1}$.
  By induction, we get $f = \alpha \cdot (X - a_1) \dots (X - a_n)$.
  Finally, we decide each of the finitely many propositions $a_i = a_j$,
  which we can assume is possible
  because our goal is still double-negation stable,
  to get the desired form
  $f = \alpha \cdot {(X - \widetilde{a}_1)}^{e_1} \dots {(X - \widetilde{a}_n)}^{e_n}$
  with distinct $\widetilde{a}_i$.
\end{proof}

\begin{example}[using \axiomref{loc}, \axiomref{sqc}, \axiomref{Z-choice}]%
  \label{non-existence-of-roots}
  It is not the case that
  every monic polynomial $f : R[X]$ with $\deg f \geq 1$ has a root.
  More specifically,
  if $U \subseteq \A^1$ is an open subset
  with the property that
  the polynomial $X^2 - a : R[X]$ merely has a root
  for every $a : U$,
  then $U = \emptyset$.
\end{example}

\begin{proof}
  Let $U \subseteq \A^1$ be as in the statement.
  Since we want to show $U = \emptyset$,
  we can assume a given element $a_0 : U$
  and now have to derive a contradiction.
  By \axiomref{Z-choice},
  there exists in particular a standard open $D(f) \subseteq \A^1$
  with $a_0 \in D(f)$
  and a function $g : D(f) \to R$
  such that ${(g(x))}^2 = x$ for all $x : D(f)$.
  By \axiomref{sqc},
  this corresponds to an element $\frac{p}{f^n} : R[X]_f$
  with ${(\frac{p}{f^n})}^2 = X : R[X]_f$.
  We use \cref{polynomial-with-regular-value-is-regular}
  together with the fact that $f(a_0)$ is invertible
  to get that $f : R[X]$ is regular,
  and therefore $p^2 = f^{2n}X : R[X]$.
  Considering this equation over $R^{\mathrm{red}} = R/\sqrt{(0)}$ instead,
  we can show by induction that all coefficients of $p$ and of $f^n$ are nilpotent,
  which contradicts the invertibility of $f(a_0)$.
\end{proof}

\begin{remark}
  \Cref{non-existence-of-roots} shows that
  the axioms we are using here
  are incompatible with a natural axiom that is true
  for the structure sheaf of the big étale topos,
  namely that $R$ admits roots for unramifiable monic polynomials.
  The polynomial $X^2 - a$ is even separable for invertible $a$,
  assuming that $2$ is invertible in $R$.
  To get rid of this last assumption,
  we can use the fact that either $2$ or $3$ is invertible in the local ring $R$
  and observe that the proof of \cref{non-existence-of-roots}
  works just the same for $X^3 - a$.
\end{remark}


\section{Topology of schemes}

\subsection{Closed subtypes}

\begin{definition}%
  \label{closed-proposition}\label{closed-subtype}
  \begin{enumerate}[(a)]
  \item
    A \notion{closed proposition} is a proposition
    which is merely of the form $x_1 = 0 \land \dots \land x_n = 0$
    for some elements $x_1, \dots, x_n \in R$.
  \item
    Let $X$ be a type.
    A subtype $U : X \to \Prop$ is \notion{closed}
    if for all $x : X$, the proposition $U(x)$ is closed.
  \item
    For $A$ a finitely presented $R$-algebra
    and $f_1, \dots, f_n : A$,
    we set
    $V(f_1, \dots, f_n) \colonequiv
    \{\, x : \Spec A \mid f_1(x) = \dots = f_n(x) = 0 \,\}$.
  \end{enumerate}
\end{definition}

Note that $V(f_1, \dots, f_n) \subseteq \Spec A$ is a closed subtype
and we have $V(f_1, \dots, f_n) = \Spec (A/(f_1, \dots, f_n))$.

\begin{proposition}[using \axiomref{sqc}]%
  There is an order-reversing isomorphism of partial orders
  \begin{align*}
    \text{f.g.-ideals}(R) &\xrightarrow{{\sim}} \Omega_{cl} \\
    I &\mapsto (I = (0))
  \end{align*}
  between the partial order of finitely generated ideals of $R$
  and the partial order of closed propositions.
\end{proposition}

\begin{proof}
  For a finitely generated ideal $I = (x_1, \dots, x_n)$,
  the proposition $I = (0)$ is indeed a closed proposition,
  since it is equivalent to $x_1 = 0 \land \dots \land x_n = 0$.
  It is also evident that we get all closed propositions in this way.
  What remains to show is that
  \[ I = (0) \Rightarrow J = (0)
     \qquad\text{iff}\qquad
     J \subseteq I
     \rlap{\text{.}}
  \]
  For this we use synthetic quasicoherence.
  Note that the set $\Spec R/I = \Hom_R(R/I, R)$ is a proposition
  (has at most one element),
  namely it is equivalent to the proposition $I = (0)$.
  Similarly, $\Hom_R(R/J, R/I)$ is a proposition
  and equivalent to $J \subseteq I$.
  But then our claim is just the equation
  \[ \Hom(\Spec R/I, \Spec R/J) = \Hom_R(R/J, R/I) \]
  which holds by \Cref{spec-embedding},
  since $R/I$ and $R/J$ are finitely presented $R$-algebras
  if $I$ and $J$ are finitely generated ideals.
\end{proof}

\begin{lemma}[using \axiomref{sqc}]%
  \label{ideals-embed-into-closed-subsets}
  We have $V(f_1, \dots, f_n) \subseteq V(g_1, \dots, g_m)$
  as subsets of $\Spec A$
  if and only if
  $(g_1, \dots, g_m) \subseteq (f_1, \dots, f_n)$
  as ideals of $A$.
\end{lemma}

\begin{proof}
  The inclusion $V(f_1, \dots, f_n) \subseteq V(g_1, \dots, g_m)$
  means a map $\Spec (A/(f_1, \dots, f_n)) \to \Spec (A/(g_1, \dots, g_m))$
  over $\Spec A$.
  By \Cref{spec-embedding}, this is equivalent to
  a homomorphism $A/(g_1, \dots, g_m) \to A/(f_1, \dots, f_n)$,
  which in turn means the stated inclusion of ideals.
\end{proof}

\begin{lemma}[using \axiomref{loc}, \axiomref{sqc}, \axiomref{Z-choice}]%
  \label{closed-subtype-affine}
  A closed subtype $C$ of an affine scheme $X=\Spec A$ is an affine scheme
  with $C=\Spec (A/I)$ for a finitely generated ideal $I\subseteq A$.
\end{lemma}

\begin{proof}
  By \axiomref{Z-choice} and boundedness,
  there is a cover $D(f_1),\dots,D(f_l)$, such that
  on each $D(f_i)$, $C$ is the vanishing set of functions
  \[ g_1,\dots,g_n:D(f_i)\to R\rlap{.} \]
  By \Cref{ideals-embed-into-closed-subsets},
  the ideals generated by these functions
  agree in $A_{f_i f_j}$,
  so by \Cref{fg-ideal-local-global},
  there is a finitely generated ideal $I\subseteq A$,
  such that $A_{f_i}\cdot I$ is $(g_1,\dots,g_n)$
  and $C=\Spec A/I$.
\end{proof}

\subsection{Open subtypes}

While we usually drop the prefix ``qc'' in the definition below,
one should keep in mind, that we only use a definition of quasi compact open subsets.
The difference to general opens does not play a role so far,
since we also only consider quasi compact schemes later.

\begin{definition}%
  \label{qc-open}
  \begin{enumerate}[(a)]
  \item A proposition $P$ is \notion{(qc-)open}, if there merely are $f_1,\dots,f_n:R$,
    such that $P$ is equivalent to one of the $f_i$ being invertible.
  \item Let $X$ be a type.
    A subtype $U:X\to\Prop$ is \notion{(qc-)open}, if $U(x)$ is an open proposition for all $x:X$.
  \end{enumerate}
\end{definition}

\begin{proposition}[using \axiomref{loc}, \axiomref{sqc}]%
  \label{open-iff-negation-of-closed}
  A proposition $P$ is open
  if and only if
  it is the negation of some closed proposition
  (\Cref{closed-proposition}).
\end{proposition}

\begin{proof}
  Indeed, by \Cref{generalized-field-property},
  the proposition $\inv(f_1) \lor \dots \lor \inv(f_n)$
  is the negation of ${f_1 = 0} \land \dots \land {f_n = 0}$.
\end{proof}

\begin{proposition}[using \axiomref{loc}, \axiomref{sqc}]%
  \label{open-union-intersection}
  Let $X$ be a type.
  \begin{enumerate}[(a)]
  \item The empty subtype is open in $X$.
  \item $X$ is open in $X$.
  \item Finite intersections of open subtypes of $X$ are open subtypes of $X$.
  \item Finite unions of open subtypes of $X$ are open subtypes of $X$.
  \item Open subtypes are invariant under pointwise double-negation.
  \end{enumerate}
  Axioms are only needed for the last statement.
\end{proposition}

In \Cref{open-subscheme} we will see that open subtypes of open subtypes of a scheme are open in that scheme.
Which is equivalent to open propositions being closed under dependent sums.

\begin{proof}[of \Cref{open-union-intersection}]
  For unions, we can just append lists.
  For intersections, we note that invertibility of a product
  is equivalent to invertibility of both factors.
  Double-negation stability
  follows from \Cref{open-iff-negation-of-closed}.
\end{proof}

\begin{lemma}%
  \label{preimage-open}
  Let $f:X\to Y$ and $U:Y\to\Prop$ open,
  then the \notion{preimage} $U\circ f:X\to\Prop$ is open.
\end{lemma}

\begin{proof}
  If $U(y)$ is an open proposition for all $y : Y$,
  then $U(f(x))$ is an open proposition for all $x : X$.
\end{proof}

\begin{lemma}[using \axiomref{loc}, \axiomref{sqc}]%
  \label{open-inequality-subtype}
  Let $X$ be affine and $x:X$, then the proposition
  \[ x\neq y \]
  is open for all $y:X$.
\end{lemma}

\begin{proof}
  We show a proposition, so we can assume $\iota: X\to \A^n$ is a subtype.
  Then for $x,y:X$, $x\neq y$ is equivalent to $\iota(x)\neq\iota(y)$.
  But for $x,y:\A^n$, $x\neq y$ is the open proposition that $x-y\neq 0$.
\end{proof}

The intersection of all open neighborhoods of a point in an affine scheme,
is the formal neighborhood of the point.
We will see in \Cref{intersection-of-all-opens}, that this also holds for schemes.

\begin{lemma}[using \axiomref{loc}, \axiomref{sqc}]%
  \label{affine-intersection-of-all-opens}
  Let $X$ be affine and $x:X$, then the proposition
  \[ \prod_{U:X\to \Open}U(x)\to U(y) \]
  is equivalent to $\neg\neg (x=y)$.
\end{lemma}

\begin{proof}
  By \Cref{open-union-intersection}, $\neg\neg (x=y)$ implies $\prod_{U:X\to \Open}U(x)\to U(y)$.
  For the other implication,
  $\neg (x=y)$ is open by \Cref{open-inequality-subtype}, so we get a contradiction.
\end{proof}

We now show that our two definitions (\Cref{affine-open}, \Cref{qc-open})
of open subtypes of an affine scheme are equivalent.

\begin{theorem}[using \axiomref{loc}, \axiomref{sqc}, \axiomref{Z-choice}]%
  \label{qc-open-affine-open}
  Let $X=\Spec A$ and $U:X\to\Prop$ be an open subtype,
  then $U$ is affine open, i.e. there merely are $h_1,\dots,h_n:X\to R$ such that
  $U=D(h_1,\dots,h_n)$.
\end{theorem}

\begin{proof}
  Let $L(x)$ be the type of finite lists of elements of $R$,
  such that one of them being invertible is equivalent to $U(x)$.
  By assumption, we know
  \[\prod_{x:X}\propTrunc{L(x)}\rlap{.}\]
  So by \axiomref{Z-choice}, we have $s_i:\prod_{x:D(f_i)}L(x)$.
  We compose with the length function for lists to get functions $l_i:D(f_i)\to\N$.
  By \Cref{boundedness}, the $l_i$ are bounded.
  Since we are proving a proposition, we can assume we have actual bounds $b_i:\N$.
  So we get functions $\tilde{s_i}:D(f_i)\to R^{b_i}$,
  by append zeros to lists which are too short,
  i.e. $\widetilde{s}_i(x)$ is $s_i(x)$ with $b_i-l_i(x)$ zeros appended.

  Then one of the entries of $\widetilde{s}_i(x)$ being invertible,
  is still equivalent to $U(x)$.
  So if we define $g_{ij}(x)\colonequiv \pi_j(\widetilde{s}_i(x))$,
  we have functions on $D(f_i)$, such that
  \[
    D(g_{i1},\dots,g_{ib_i})=U\cap D(f_i)
    \rlap{.}
  \]
  By \Cref{affine-open-trans}, this is enough to solve the problem on all of $X$.
\end{proof}

This allows us to transfer one important lemma from affine-opens to qc-opens.
The subtlety of the following is that while it is clear that the intersection of two
qc-opens on a type, which are \emph{globally} defined is open again, it is not clear,
that the same holds, if one qc-open is only defined on the other.

\begin{lemma}[using \axiomref{loc}, \axiomref{sqc}, \axiomref{Z-choice}]%
  \label{qc-open-trans}
  Let $X$ be a scheme, $U\subseteq X$ qc-open in $X$ and $V\subseteq U$ qc-open in $U$,
  then $V$ is qc-open in $X$.
\end{lemma}

\begin{proof}
  Let $X_i=\Spec A_i$ be a finite affine cover of $X$.
  It is enough to show, that the restriction $V_i$ of $V$ to $X_i$ is qc-open.
  $U_i\colonequiv X_i\cap U$ is qc-open in $X_i$, since $X_i$ is qc-open.
  By \Cref{qc-open-affine-open}, $U_i$ is affine-open in $X_i$,
  so $U_i=D(f_1,\dots,f_n)$.
  $V_i\cap D(f_j)$ is affine-open in $D(f_j)$, so by \Cref{affine-open-trans},
  $V_i\cap D(f_j)$ is affine-open in $X_i$.
  This implies $V_i\cap D(f_j)$ is qc-open in $X_i$ and so is $V_i=\bigcup_{j}V_i\cap D(f_j)$.
\end{proof}

\begin{lemma}[using \axiomref{loc}, \axiomref{sqc}, \axiomref{Z-choice}]%
  \label{qc-open-sigma-closed}
  \begin{enumerate}[(a)]
  \item qc-open propositions are closed under dependent sums:
    if $P : \Open$ and $U : P \to \Open$,
    then the proposition $\sum_{x : P} U(x)$ is also open.
  \item Let $X$ be a type. Any open subtype of an open subtype of $X$ is an open subtype of $X$.
  \end{enumerate}
\end{lemma}

\begin{proof}
  \begin{enumerate}[(a)]
  \item Apply \Cref{qc-open-trans} to the point $\Spec R$.
  \item Apply the above pointwise.
  \end{enumerate}
\end{proof}

\begin{remark}
  \Cref{qc-open-sigma-closed} means that
  the (qc-) open propositions constitute a \notion{dominance}
  in the sense of~\cite{rosolini-phd-thesis}.
\end{remark}

The following fact about the interaction of closed and open propositions
is due to David Wärn.

\begin{lemma}%
  \label{implication-from-closed-to-open}
  Let $P$ and $Q$ be propositions
  with $P$ closed and $Q$ open.
  Then $P \to Q$ is equivalent to $\lnot P \lor Q$.
\end{lemma}

\begin{proof}
  We can assume $P = (f_1 = \dots = f_n = 0)$
  and $Q = (\inv(g_1) \lor \dots \lor \inv(g_m))$.
  Then we have:
  \begin{align*}
    (P \to Q) &= \qquad
    \text{\Cref{generalized-field-property} for $g_1, \dots, g_m$}\\
    (P \to \lnot (g_1 = \dots = g_m = 0)) &= \\
    \lnot (f_1 = \dots = f_n = g_1 = \dots = g_m = 0) &= \qquad
    \text{\Cref{generalized-field-property} for $f_1, \dots, f_n, g_1, \dots, g_m$}\\
    (\inv(f_1) \lor \dots \lor \inv(f_n) \lor \inv(g_1) \lor \dots \lor \inv(g_m) &= \qquad
    \text{\Cref{generalized-field-property} for $f_1, \dots, f_n$}\\
    \lnot P \lor Q &
  \end{align*}
\end{proof}



\section{Schemes}
In \cref{def:schemes} we defined a scheme to be a type $X$ such that $X$
may be covered by finitely many open affine subtypes.
In this section, we will present general properies of schemes and a couple of common constructions for schemes.


\subsection{General Properties}

\begin{lemma}[using \axiomref{sqc}, \axiomref{loc}]%
  \label{lem:intersection-of-all-opens}
  Let $X$ be a scheme and $x:X$, then for all $y:X$ the proposition
  \[ \prod_{U:X\to \Open}U(x)\to U(y) \]
  is equivalent to $\neg\neg (x=y)$.
\end{lemma}

\begin{proof}
  By \cref{prop:open-union-intersection},
  open proposition are always double-negation stable,
  which settles one implication.
  For the implication
  \[ \left(\prod_{U:X\to \Open}U(x)\to U(y)\right) \Rightarrow \neg\neg (x=y) \]
  we can assume that $x$ and $y$ are both inside an open affine $U$
  and use that the statement holds for affine schemes by \cref{lem:affine-intersection-of-all-opens}.
\end{proof}

\subsection{Glueing}

\begin{proposition}[using \axiomref{loc}, \axiomref{sqc}, \axiomref{Z-choice}]%
  Let $X,Y$ be schemes and $f:U\to X$, $g:U\to Y$ be embeddings with open images in $X$ and $Y$,
  then the pushout of $f$ and $g$ is a scheme.
\end{proposition}

\begin{proof}
  As we noted in \cref{MISSING}, such a pushout is always 0-truncated.
  Let $U_1,\dots,U_n$ be a cover of $X$ and $V_1,\dots,V_m$ be a cover of $Y$.
  By \cref{lem:qc-open-trans}, $U_i\cap U$ is open in $Y$,
  so we can use (large) pushout-recursion to construct a subtype $\tilde{U_i}$,
  which is open in the pushout and restricts to $U_i$ on $X$ and $U_i\cap U$ on $Y$.
  Symetrically we define $\tilde{V_i}$ and in total get an open finite cover of the pushout.
  The pieces of this new cover are equivalent to their counterparts in the covers of $X$ and $Y$,
  so they are affine as well.
\end{proof}

\subsection{Subschemes}

\begin{definition}
  Let $X$ be a scheme.
  A \notion{subscheme} of $X$ is a subtype $Y:X\to\Prop$,
  such that $\sum Y$ is a scheme.
\end{definition}

\begin{proposition}[using \axiomref{loc}, \axiomref{sqc}, \axiomref{Z-choice}]%
  \label{prop:open-subscheme}
  Any open subtype of a scheme is a scheme.
\end{proposition}

\begin{proof}
  Using \cref{thm:qc-open-affine-open}.
\end{proof}

\begin{proposition}[using \axiomref{sqc}, \axiomref{loc}, \axiomref{Z-choice}]%
  \label{prop:closed-subscheme}
  Any closed subtype $A:X\to \Prop$ of a scheme $X$ is a scheme.
\end{proposition}

\begin{proof}
  Any open subtype of $X$ is also open in $A$.
  So it is enough to show,
  that any affine open $U_i$ of $X$,
  has affine intersection with $A$.
  But $U_i\cap A$ is closed in $U_i$ and therefore affine by \cref{lem:closed-subtype-affine}.
\end{proof}

\subsection{Equality types}

\begin{lemma}%
  \label{lem:affine-equality-closed}
  Let $X$ be an affine scheme and $x,y:X$,
  then $x=_Xy$ is an affine scheme
  and $((x,y):X\times X)\mapsto x=_Xy$
  is a closed subtype of $X\times X$.
\end{lemma}

\begin{proof}
  Any affine scheme is merely embedded into $\A^n$ for some $n:\N$.
  The proposition $x=y$ for elements $x,y:\A^n$ is equivalent to $x-y=0$,
  which is equivalent to all entries of this vector being zero.
  The latter is a closed proposition.
\end{proof}

\begin{proposition}[using \axiomref{sqc}, \axiomref{loc}, \axiomref{Z-choice}]%
  \label{prop:equality-scheme}
  Let $X$ be a scheme.
  The equality type $x=_Xy$ is a scheme for all $x,y:X$.
\end{proposition}

\begin{proof}
  Let $x,y:X$ and
  $U\subseteq X$ be an affine open containing $x$.
  Then $U(y)\wedge x=y$ is equivalent to $x=y$, so it is enough to show that $U(y)\wedge x=y$ is a scheme.
  As a open subscheme of the point, $U(y)$ is a scheme and $(x:U(y))\mapsto x=y$ defines a closed subtype by \cref{lem:affine-equality-closed}.
  But this closed subtype is a scheme by \cref{prop:closed-subscheme}.
\end{proof}

\subsection{Dependent sums}

\begin{theorem}[using \axiomref{loc}, \axiomref{sqc}, \axiomref{Z-choice}]%
  \label{thm:sigma-scheme}
  Let $X$ be a scheme and for any $x:X$, let $Y_x$ be a scheme.
  Then the dependent sum
  \[ \left((x:X)\times Y_x\right)\equiv \sum_{x:X}Y_x\]
  is a scheme.
\end{theorem}

\begin{proof}
  We start with an affine $X=\Spec A$ and $Y_x=\Spec B_x$.
  Locally on $U_i = D(f_i)$, for a Zariski-cover $f_1,\dots,f_l$ of $X$,
  we have $B_x=\Spec R[X_1,\dots,X_{n_i}]/(g_{i,x,1},\dots,g_{i,x,m_i})$
  with polynomials $g_{i,x,j}$.
  In other words, $B_x$ is the closed subtype of $\A^{n_i}$
  where the functions $g_{i,x,1},\dots,g_{i,x,m_i}$ vanish.
  By \cref{lem:affine-fiber-product}, the product
  \[ V_i\colonequiv U_i\times \A^{n_i}\]
  is affine.
  The type $(x:U_i)\times \Spec B_x\subseteq V_i$ is affine,
  since it is the zero set of the functions
  \[ ((x,y):V_i)\mapsto g_{i,x,j}(y) \]
  Furthermore, $W_i\colonequiv (x:U_i)\times \Spec B_x$
  is open in $(x:X)\times Y_x$,
  since $W_i(x)$ is equivalent to $U_i(\pi_1(x))$,
  which is an open proposition.

  This settles the affine case.
  We will now assume, that
  $X$ and all $Y_x$ are general schemes.
  We pass again to a cover of $X$ by affine open $U_1,\dots,U_n$.
  We can choose the latter cover,
  such that for each $i$ and $x:U_i$, the $Y_{\pi_1(x)}$
  are covered by $l_i$ many open affine pieces $V_{i,x,1},\dots,V_{i,x,l_i}$
  (by \cref{thm:boundedness}).
  Then $W_{i,j}\colonequiv(x:U_i)\times V_{i,x,j}$ is affine by what we established above.
  It is also open.
  To see this, let $(x,y):((x:X)\times Y_x)$.
  We want to show, that $(x,y)$ being in $W_{i,j}$ is an open proposition.
  We have to be a bit careful, since the open proposition
  $V_{i,x,j}$ is only defined, for $x:U_i$.
  So the proposition we are after is $(z:U_i(x,y))\times V_{i,z,j}(y)$.
  But this proposition is open by \cref{lem:qc-open-sigma-closed}.
\end{proof}

\begin{corollary}
  \label{cor:scheme-map-classification}
  Let $X$ be a scheme.
  For any other scheme $Y$ and any map $f:Y\to X$,
  the fiber map
  $(x:X)\mapsto \fib_f(x)$
  has values in the type of schemes $\Sch$.
  Mapping maps of schemes to their fiber maps,
  is an equivalence of types
  \[ \left(\sum_{Y:\Sch}(Y\to X)\right)\simeq (X\to \Sch)\rlap{.}\]
\end{corollary}

\begin{proof}
  By univalence, there is an equivalence
  \[ \left(\sum_{Y:\Type}(Y\to X)\right)\simeq (X\to \Type)\rlap{.} \]
  From left to right, the equivalence is given by turning a $f:Y\to X$ into $x\mapsto \fib_f(x)$,
  from right to left is given by taking the depedent sum.
  So we just have to note, that both constructions preserve schemes.
  From left to right, this is \cref{prop:fiber-product-scheme}, from right to left,
  this is \cref{thm:sigma-scheme}.
\end{proof}

Subschemes are classified by propositional schemes:

\begin{corollary}
  Let $X$ be a scheme.
  $Y:X\to\Prop$ is a subscheme,
  if and only if $Y_x$ is a scheme for all $x:X$.
\end{corollary}

\begin{proof}
  Restriction of \cref{cor:scheme-map-classification}.
\end{proof}

\subsection{Pullbacks of Schemes}

In this section, we will show in two different ways,
that the pullback of a cospan of schemes is a scheme.
The first poof is very short and reuses what we proved about equality and sigma-types,
the second proof is more direct, uses the proof of the affine case \cref{lem:affine-fiber-product}
and is along the lines of what one might find in an algebraic geometry textbook.

\begin{theorem}[using \axiomref{loc}, \axiomref{sqc}, \axiomref{Z-choice}]%
  \label{thm:fiber-product-scheme}
  Let
  \[
    \begin{tikzcd}
      X\ar[r,"f"] & Z & Y\ar[l,swap,"g"]
    \end{tikzcd}
  \]
  be schemes, then the \notion{pullback} $X\times_Z Y$ is also a scheme.
\end{theorem}

\begin{proof}
  The type $X\times_Z Y$ is given as the following, interated dependent sum:
  \[ \sum_{x:X}\sum_{y:Y}f(x)=g(y)\rlap{.}\]
  The innermost type, $f(x)=g(y)$
  is the equality type in the scheme $Z$ and by \cref{prop:equality-scheme} a scheme.
  By applying \cref{thm:sigma-scheme} twice, we prove that the itereated dependent sum is a scheme.
\end{proof}

We conclude with a construction, analogous to the classical treatment:

\begin{proof}[alternative proof of \cref{thm:fiber-product-scheme}]
  Let $W_1,\dots,W_n$ be a finite affine cover of $Z$.
  The preimages of $W_i$ under $f$ and $g$ are open
  and covered by fintely many affine open $U_{ik}$ and $V_{ij}$ by \cref{prop:open-subscheme}.
  This leads to the following diagram:
  \begin{center}
    \begin{tikzcd}
      X\times_Z Y\ar[rrr]\ar[ddd] & & & Y\ar[ddd] & \\
      & P_{ij}\ar[hook,lu]\ar[rrr,crossing over] &&& V_{ij}\ar[hook,lu]\ar[ddd] \\
      &&&& \\
      X\ar[rrr] & & & Z & \\
      & U_{i}\ar[rrr]\ar[hook,lu]\ar[from=uuu,crossing over] & & & W_i\ar[hook,lu]
    \end{tikzcd}
  \end{center}
  where the front and bottom square are pullbacks by definition.
  By pullback-pasting, the top is also a pullback,
  so all diagonal maps are embeddings.
  
  $P_{ij}$ is open, since it is a preimage of $V_{ij}$ (\cref{lem:preimage-open}),
  which is open in $Y$ by \cref{lem:qc-open-trans}.
  It remains to show, that the $P_{ij}$ cover $X\times_Z Y$ and that $P_{ij}$ is a scheme.
  Let $x:X\times_Z Y$.
  For the image $w$ of $x$ in $W$, there merely is an $i$ such that $w$ is in $W_i$.
  The image of $x$ in $V_i$ merely lies in some $V_{ij}$,
  so $x$ is in $P_{ij}$.

  We proceed by showing that $P_{ij}$ is a scheme.
  Let $U_{ik}$ be a part of the finite affine cover of $U_i$.
  We repeat part of what we just did:
  \begin{center}
    \begin{tikzcd}
      P_{ij}\ar[rrr]\ar[ddd] & & & U_i\ar[ddd] & \\
      & P_{ijk}\ar[hook,lu]\ar[rrr,crossing over] &&& U_{ik}\ar[hook,lu]\ar[ddd] \\
      &&&& \\
      V_{ij}\ar[rrr] & & & W_i & \\
      & V_{ij}\ar[rrr]\ar[equal,lu]\ar[from=uuu,crossing over] & & & W_i\ar[equal,lu]
    \end{tikzcd}
  \end{center}

  So by \cref{lem:affine-fiber-product}, $P_{ijk}$ is affine.
  Repetition of the above shows, that the $P_{ijk}$ are open and cover $P_{ij}$.
\end{proof}


\section{Projective space}
We follow the notations and setting for Synthetic Algebraic Geometry \cite{draft}.
In particular, $R$ denotes the generic local ring and $R^\times$ is the multiplicative group of units of $R$.

In Synthetic Algebraic Geometry, a scheme is defined as a set satisfying some property \cite{draft}. In particular
the projective space $\bP^n$ can be defined to be the quotient of $R^{n+1}\setminus\{0\}$ by the
equivalence relation $a\sim b$ which expresses that $a$ and $b$ are proportional, %i.e. $a_ib_j=a_jb_i$,
which is equal to $\Sigma_{r:R^\times}ar = b$. We can then prove \cite{draft}
that this set is a scheme. This definition goes back to \cite{Kock74}.

 In this setting, a map of schemes is simply an arbitrary set theoretic map. An application of this work is to show
 that the maps $\bP^n\rightarrow \bP^m$ are given by $m+1$ homogeneous polynomials of the same degree in $n+1$ variables.

\medskip


There is another definition of $\bP^n$ which uses ``higher'' notions. Let $\KR$ be the delooping
of $R^\times$. It can be defined as the type of lines $\Sigma_{M:\Mod{R}}\|{M=R^1}\|$. Over $\KR$ we have the
family of sets
$$T_n(l) = l^{n+1}\setminus\{0\}$$
Note that we use the same notation for an element $l : \KR$,
its underlying $R$-module and its underlying set.
An equivalent definition of $\bP^n$ is then
$$
\bP^n = \sum_{l:\KR}T_n(l)
$$
That is, we replaced the quotient operation, here a set of orbits for a free group action, by a sum type over the delooping of this group
\cite{Sym}.

The standard line bundles (or twisting sheaves) on $\bP^n$ can then be constructed as follows.
We define $l^{\vee}\colonequiv \Hom_{\Mod{R}}(l,R^1)$
and $l^{\otimes n}:\KR$ for $l:\KR$ by
$l^{\otimes 0} = R^1$,
$l^{\otimes (n+1)} = l^{\otimes n}\otimes l$ for $n \geqslant 0$
and $l^{\otimes (n-1)} = l^{\otimes n}\otimes l^{\vee}$ for $n \leqslant 0$.
Then we define the line bundle $O(d):\bP^n\rightarrow \KR$ by $O(d)(l,s) = l^{\otimes d}$.

\medskip

 Connected to this definition of $\bP^n$, we will prove some equalities in the following.
 To prove these equalities, we will make use of the following lemma, which holds in synthetic algebraic geometry:
 
\begin{lemma}\label{invariant-implies-homogenous}
  Let $n,d:\N$ and $\alpha:R^n\to R$ be a map such that
  \[\alpha(\lambda x)=\lambda^d\alpha(x)\]
  then $\alpha$ is a homogenous polynomial of degree $d$.
\end{lemma}

\begin{proof}
  By duality, any map $\alpha:R^n\to R$ is a polynomial.
  To see it is homogenous of degree $d$, let us first note that any $P:R[\lambda]$ with $P(\lambda)=\lambda^d P(1)$
  for all $\lambda:R^\times$ also satisfies this equation for all $\lambda : R$ and is therefore homogenous of degree $d$.
  Then for $\alpha'_x:R[\lambda]$ given by $\alpha'_x(\lambda)\colonequiv \alpha(\lambda\cdot x)$
  we have $\alpha'_x(\lambda)=\lambda^d \alpha'_x(1)$. This means any coeffiecent of $\alpha'_x$
  of degree different from $d$ is 0. Since this means every monomial appearing in $\alpha$,
  which is not of degree $d$, is zero for all $x$ and therefore 0.   
\end{proof}

\begin{proposition}\label{end}
  $$\prod_{l:\KR}l^n\rightarrow l \;\;\;=\;\;\; \Hom(R^n,R)$$
\end{proposition}

\begin{proof}
We rewrite $\Hom(R^n,R)$, the set of $R$-module morphism, as
$$
\sum_{\alpha:R^n\rightarrow R}\prod_{\lambda:R^\times}\prod_{x:R^n}\alpha(\lambda x) = \lambda \alpha(x)
$$
using \Cref{invariant-implies-homogenous} with $d=1$.

\medskip

It is then a general fact that if we have a pointed connected groupoid $(A,a)$ and a family of
sets $T(x)$ for $x:A$, then $\prod_{x:A}T(x)$ is the set of fixedpoints of $T(a)$ for the $(a=a)$ action
\cite{Sym}.
\end{proof}

We will use the following remark, proved in \cite{draft}[Remark 6.2.5].

\begin{lemma}\label{ext}
  Any map $R^{n+1}\setminus\{0\}\rightarrow R$ can be uniquely extended to a map $R^{n+1}\rightarrow R$ for $n>0$.
\end{lemma}

We will also use the following proposition, already noticed in \cite{draft}.

\begin{proposition}\label{const}
  Any map from $\bP^n$ to $R$ is constant.
\end{proposition}

\begin{proof}
  Since $\bP^n$ is a quotient of $R^{n+1}\setminus\{0\}$, the set $\bP^n\rightarrow R$ is
  the set of maps $\alpha:R^{n+1}\setminus\{0\}\rightarrow R$
  such that $\alpha(\lambda x) = \alpha(x)$ for all $\lambda$ in $R^\times$.
  These are exactly the constant maps
  using \Cref{ext} and \Cref{invariant-implies-homogenous} with $d=0$.
\end{proof}

\begin{proposition}\label{aut}
  For all $n:\N$ we have:
$$\prod_{l:\KR}T_n(l)\rightarrow T_n(l) \;\;=\;\; GL_{n+1}$$
\end{proposition}

\begin{proof}
  For $n=0$, this is the direct computation that a Laurent-polynomial $\alpha:(R[X,1/X])^\times$ which satisfies
  $\alpha(\lambda x)=\lambda \alpha(x)$ is $\lambda\alpha(1)$ where $\alpha(1):R^\times=\GL_1$.
  
  \medskip
  
  For $n>0$, the proposition follows from two remarks.

  The first remark is that maps $T_n(R)\to T_n(R)$, which are invariant under the induced $\KR$ action, are linear.
  To prove this remark, we first map from $T_n(l)\to T_n(l)$ to $T_n(l)\to l^{n+1}$ by composing with the inclusion.
  Maps of the latter kind can be uniquely extended to maps $l^{n+1}\to l^{n+1}$, since by 
  \Cref{ext} the restriction map
$$
(l^{n+1}\rightarrow l)\rightarrow ((l^{n+1}\setminus\{0\})\rightarrow l)
$$
is a bijection for $n>0$ and all $l:\KR$.

\medskip

The second remark is that a linear map $u:R^{m}\rightarrow R^{m}$ such that
$$
x\neq 0~\rightarrow~u(x)\neq 0
$$
is exactly an element of $GL_{m}$.

We show this by induction on $m$. For $m=1$ we have $u(1)\neq 0$ iff $u(1)$ invertible.

For $m>1$, we look at $u(e_1) = \Sigma \alpha_ie_i$ with $e_1,\dots,e_m$ basis of $R^m$.
We have that some $\alpha_j$ is invertible.
By composing $u$ with an element in $GL_m$, we can then
assume that $u(e_1) = e_1+v_1$ and $u(e_i) = v_i$, for $i>1$, with $v_1,\dots,v_m$ in $Re_2+\dots+Re_m$.
We can then conclude by induction.
\end{proof}

We can generalize \Cref{end}
and get a result related to \Cref{aut} as follows.
 
\begin{lemma}\label{hom}
  \begin{enumerate}[(i)]
    \item
      \[  \prod_{l:\KR}l^n\rightarrow l^{\otimes d} \;\;=\;\; (R[X_1, \dots, X_n])_d \]
      That is,
      every element of the left-hand side is given by
      a unique homogeneous polynomial of degree $d$ in $n$ variables.
    \item
      An element in
      $$\prod_{l:\KR}T_n(l)\rightarrow T_m(l^{\otimes d})$$
      is given by $m+1$ homogeneous polynomials $p = (p_0,\dots,p_m)$ of degree $d$ such that
      $x\neq 0$ implies $p(x)\neq 0$.
  \end{enumerate}
\end{lemma}

\begin{proof}
We show the first item. Following \cite{Sym} again, this product is the set of maps $\alpha:R^n\rightarrow R^{\otimes d}$
which are invariant by the $R^\times$-action which in this case acts by mapping $\alpha$ to $r^d\alpha(r^{-1} x)$ for each $r:R^\times$.
So by \Cref{invariant-implies-homogenous} these are exactly the maps given by homogeneous polynomials of degree $d$.
\end{proof}


\section{Line bundles}

\subsection{Regular sections and regular closed subschemes}

In classical algebraic geometry,
there is the concept of a \notion{generic section} of a line bundle.
Informally, the generic sections have the smallest possible vanishing set.
The following definition corresponds to this notion:

\begin{definition}%
  \label{regular-section}
  Let $X$ be a type and $\mathcal L:X\to \Mod{R}$ a line bundle.
  A section
  \[ s:\prod_{x:X}\mathcal L_x \]
  is \notion{regular}, there merely is a trivializing affine cover $U_1=\Spec A_1,\dots,U_n=\Spec A_n$
  of $\mathcal L$, such that each trivialized restriction
  \[ s_i:\Spec A_i\to R \]
  is a regular element (\cref{regular-element}) of $(\Spec A_i\to R) = A_i$.
\end{definition}

\begin{lemma}%
  \label{regular-zariski-local}
  Let $s:\Spec A\to R$.
  $s$ being regular is Zariski-local, i.e.
  for all Zariski-covers $U_1,\dots,U_n$ of $\Spec A$,
  $s$ is regular, if and only if it is regular on all $U_i$.
\end{lemma}

\begin{proof}
  It is enough to check this for a localization at $f:A$.
  Let
  \[ \frac{s}{1}\cdot\frac{g}{f^k}=0\rlap{.} \]
  then $f^lsg=0$, which implies $f^lg=0$ by regularity of $s$ and therefore $\frac{g}{f^l}=0$.
\end{proof}

\begin{proposition}%
  The choice of trivializing cover in \cref{regular-section}
  is irrelevant.
\end{proposition}

\begin{proof}
  By \cref{regular-zariski-local}.
\end{proof}

From a line bundle together with a regular section,
we can produce a closed subtype of a special kind:

\begin{definition}%
  Let $X$ be a scheme.
  A \notion{regular closed subtype} of $X$ is a closed subtype
  $C:X\to \Prop$, such that there merely is an affine open cover $U_1=\Spec A_1,\dots,U_n=\Spec A_n$,
  and $C\cap U_i$ is $V(f_i)$ for a regular $f_i:A_i$.
\end{definition}

\begin{lemma}%
  Let $f,g:A$, $f$ be regular and $V(f)=V(g)$,
  then $g$ is regular and there is a unique unit $\alpha:A^\times$, such that $\alpha f=g$.
\end{lemma}

\begin{proof}
  $V(f)=V(g)$ implies there are $\alpha,\beta:A$ such that
  $\alpha f = g$ and $\beta g = f$.
  But then: $f=\beta g=\beta\alpha f$.
  So by regularity of $f$, $\beta\alpha=1$.
  By \cref{units-products-regular}, units are regular and products of regular elements are regular,
  so $g$ is regular.
  Uniqueness of $\alpha$ follows from regularity.
\end{proof}

\begin{theorem}[using \axiomref{Z-choice}]%
  Let $X$ be a scheme.
  For any regular closed subscheme $C$,
  there is a line bundle with regular section $(\mathcal L,s)$ on $X$,
  such that $C=V(s)$.
\end{theorem}

\begin{proof}
  Let $U_1=\Spec A_1,\dots,U_n=\Spec A_n$ be a cover by standard  affine opens such that we have
  regular $f_i$ with $C\cap U_i=V(f_i)$. 
  We define $\mathcal L$ to be the trivial line bundle $\_\mapsto R$ on each $U_i$
  and by giving automorphisms on the intersections $U_i\cap U_j\colonequiv U_{ij}=\Spec A_{ij}$.
  On $U_{ij}$, $C$ is given by $V(\frac{f_i}{1})$ and $V(\frac{f_j}{1})$ which are both regular.
  Therefore, there is a unit $\alpha:A_{ij}^\times$ such that $\alpha\frac{f_i}{1}=\frac{f_j}{1}$,
  which we can also view as a map $U_{ij}\to R^\times$ and since $R^\times$
  is equivalent to the automorphism group of $R$ as an $R$-module,
  this provides the identetification we need to construct $\mathcal L$.
  Under the identification, the local regular sections are identified, so we get a global section $s$ of $\mathcal L$,
  which is locally regular.
\end{proof}


\section{Bundles and cohomology}
In non-synthetic algebraic geometry,
the structure sheaf~$\mathcal{O}_X$ is part of the data constituting a scheme~$X$.
In our internal setting,
a scheme is just a type satisfying a property.
When we want to consider the structure sheaf as an object in its own right,
we can represent it by the trivial bundle
that assigns to every point $x : X$ the set $R$.
Indeed, for an affine scheme $X = \Spec A$,
taking the sections of this bundle over a basic open $D(f) \subseteq X$
\[ \left(\prod_{x : D(f)} R\right) = (D(f) \to R) = A[f^{-1}] \]
yields the localizations of the ring $A$
expected from the structure sheaf $\mathcal{O}_X$.
More generally,
instead of sheaves of abelian groups, $\mathcal{O}_X$-modules, etc.,
we will consider bundles of abelian groups, $R$-modules, etc.,
in the form of maps from $X$ to the respective type of algebraic structures.

\subsection{Quasi-coherent bundles}

Sometimes we want to ``apply'' a bundle to a subtype,
like sheaves can be evaluated on open subspaces
and introduce the common notation ``$M(U)$'' for that below.
It is, however, not justified to expect, that this application
and the corresponding theory of ``sheaves'' is ``the same'' as the external one,
since the definition below uses the internal hom ``$\prod$''
-- where the corresponding external construction, would be the set of continuous sections of a bundle.

\begin{definition}
  \index{$M(U)$}
  \label{application-of-bundle-to-subtype}
  Let $X$ be a type and $M:X\to \Mod{R}$ a dependent module.
  Let $U\subseteq X$ be any subtype.
  \begin{enumerate}[(a)]
  \item We write:
    \[
      M(U)\colonequiv \prod_{x:U}M_x
      \rlap{.}
    \]
  \item With pointwise structure, $U\to R$ is an $R$-algebra
    and $M(U)$ is a $(U\to R)$-module.
  \end{enumerate}
\end{definition}

Somewhat surprisingly, localization of modules $M(U)$
can be done pointwise:

\begin{lemma}[using \axiomref{loc}, \axiomref{sqc}, \axiomref{Z-choice}]%
  \label{module-bundle-localization-pointwise}
  Let $X$ be a scheme and $M:X\to \Mod{R}$ a dependent module.
  For any $f:X\to R$, there is an equality
  \[
    M(X)_f=\prod_{x:X}(M_x)_{f(x)}
  \]
  of $(X\to R)$-modules.
\end{lemma}

\begin{proof}
First we construct a map, by realizing that the following is well-defined:
\[
  \frac{m}{f^k}\mapsto\left(x\mapsto \frac{m(x)}{f(x)^k}\right)
\]
So let $\frac{m}{f^k}=\frac{m'}{f^{k'}}$,
i.e. let there be an $l:\N$ such that $f^l(mf^{k'}-m'f^k)=0$.
But then we can choose the same $l:\N$ for each $x:X$
and apply the equation to each $x:X$.

We will now show, that the map we defined is an embedding.
So let $g,h:M(X)_f$ such that $p:\prod_{x:X}g(x)=_{(M_x)_{f(x)}}h(x)$.
Let $m_g,m_h:\prod_{x:X} M_x$ and $k_g,k_h:\N$ such that
\[
  g=\frac{m_g}{f^{k_g}} \quad\text{and}\quad h=\frac{m_h}{f^{k_h}}
  \rlap{.}
\]
From $p$ we know $\prod_{x:X}\exists_{k_x:\N}f(x)^{k_x}(m_g(x)f(x)^{k_h}-m_h(x)f(x)^{k_g})=0$.
By \Cref{strengthened-boundedness},
we find one $k : \N$ with
\[
  \prod_{x:X}f(x)^{k}(m_g(x)f(x)^{k_h}-m_h(x)f(x)^{k_g})=0
\]
--- which shows $g=h$.

It remains to show that the map is surjective.
So let $\varphi:\prod_{x:X}(M_x)_{f(x)}$ and
note that
\[
  \prod_{x:X}
  \exists_{k_x:\N,m_x:M_x}.
  \varphi(x)=\frac{m_x}{f(x)^{k_x}}
  \rlap{.}
\]
By \Cref{strengthened-boundedness} and \Cref{zariski-choice-scheme},
we get $k:\N$, an affine open cover $U_1,\dots,U_n$ of $X$ and $m_i:(x : U_i)\to M_x$
such that for each $i$ and $x:U_i$ we have
\[
  \varphi(x)=\frac{m_i(x)}{f(x)^{k}}
  \rlap{.}
\]
The problem is now to construct a global $m:(x:X)\to M_x$ from the $m_i$.
We have
\[
    \prod_{x:U_{ij}}\frac{m_i(x)}{f(x)^k}=\varphi(x)=\frac{m_j(x)}{f(x)^k}
\]
meaning there is pointwise an exponent $t_x:\N$,
such that $f(x)^{t_x}m_i(x)=f(x)^{t_x}m_j(x)$.
By \Cref{strengthened-boundedness},
we can find a single $t:\N$ with this property and define
\[
  \tilde{m}_i(x) \colonequiv f(x)^t m_i(x)
  \rlap{.}
\]
Then we have $\tilde{m}_i(x)=\tilde{m}_j(x)$ on all intersections $U_{ij}$,
which is what we need to get a global $m:(x:X)\to M_x$ from \Cref{kraus-glueing}.
Since $\varphi(x)=\frac{f(x)^t m_i(x)}{f(x)^{t+k}}=\frac{\tilde{m}_i(x)}{f(x)^{t+k}}$
for all $i$ and $x : U_i$,
we have found a preimage of $\varphi$ in $M(X)_f$.
\end{proof}

We will need the following algebraic observation:

\begin{remark}%
  \label{localization-to-module-if-non-zero}
  Let $M$ be an $R$-module and $A$ a finitely presented $R$-algebra,
  then there is an $R$-linear map
  \[
    M\otimes A\to M^{\Spec A}
  \]
  induced by mapping $m\otimes f$ to $x\mapsto x(f)\cdot m$.
  In particular, for any $f:R$, there is a
  \[
    M_f\to M^{D(f)}
    \rlap{.}
  \]
  The map $M\otimes A\to M^{\Spec A}$ is natural in $M$.
\end{remark}

\begin{lemma}[using \axiomref{loc}, \axiomref{sqc}, \axiomref{Z-choice}]%
  \label{localization-to-restriction}                    
  Let $X$ be a scheme, $M:X\to\Mod{R}$, $U\subseteq X$ open and $f:A$.
  Then there is an $R$-linear map
  \[
    M(U)_f \to M(D(f)) 
    \rlap{.}
  \]
\end{lemma}

\begin{proof}
  Combining \Cref{module-bundle-localization-pointwise}
  and pointwise application of \Cref{localization-to-module-if-non-zero} we get
  \[
    M(U)_f=\left(\prod_{x:U}(M_x)_{f(x)}\right)\to \left(\prod_{x:U}(M_x)^{D(f(x))}\right)
    =\left(\prod_{x:D(f)}M_x\right)
    =M(D(f))
  \]
\end{proof}

A characterization of quasi coherent sheaves in the little Zariski-topos was found with \cite{ingo-thesis}[Theorem 8.3].
This characterization is similar to our following definition of weak quasi-coherence,
which will provide us with an abelian subcategory of the $R$-module bundles over a scheme,
where we can show that higher cohomology vanishes if the scheme is affine.

\begin{definition}
  \label{weakly-quasi-coherent-module}
  An $R$-module $M$ is \notion{weakly quasi-coherent},
  if for all $f:R$, the canonical homomorphism
  \[
    M_f\to M^{D(f)}
  \]
  from \Cref{localization-to-module-if-non-zero} is an equivalence.
  We denote the type of weakly quasi-coherent $R$-modules
  with $\Mod{R}_{wqc}$\index{$\Mod{R}_{wqc}$}.
\end{definition}

\begin{lemma}
  \label{kernel-wqc}
  For any $R$-linear map $f:M\to N$ of weakly quasi-coherent modules $M$ and $N$,
  the kernel of $f$ is weakly quasi-coherent.
\end{lemma}

\begin{proof}
  Let $K\to M$ be the kernel of $f$.
  For any $f:R$, the map $K^{D(f)}\to M^{D(f)}$ is the kernel of $M^{D(f)}\to N^{D(f)}$.
  The latter map is equal to $M_f\to N_f$ by weak quasi-coherence of $M$ and $N$
  and $K_f\to M_f$ is the kernel of $M_f\to N_f$.
  Let the vertical maps in
  \begin{center}
    \begin{tikzcd}
      K_f\ar[r]\ar[d] & M_f\ar[r]\ar[d,"\simeq"] & N_f\ar[d,"\simeq"] \\
      K^{D(f)}\ar[r] & M^{D(f)}\ar[r] & N^{D(f)}
    \end{tikzcd}
  \end{center}
  be the canonical maps from \Cref{localization-to-module-if-non-zero}.
  The squares commute because of the naturality of the vertical maps.
  Then the map $K_f\to K^{D(f)}$ is an isomorphism,
  because by commutativity, it is equal to the induced map between the kernels $K_f$ and $K^{D(f)}$,
  which has to be an isomorphism, since it is induced by an isomorphism of diagrams.
\end{proof}

\begin{definition}%
  \label{weakly-quasi-coherent-bundle}
  Let $X$ be a scheme.
  A weakly quasi-coherent bundle on $X$, is a map $M:X\to \Mod{R}_{wqc}$.
\end{definition}

An immediate consequence is, that
weakly quasi coherent dependent modules have
the property that ``restricting is the same as localizing'':

\begin{lemma}[using \axiomref{loc}, \axiomref{sqc}, \axiomref{Z-choice}]
  \label{weakly-quasi-coherent-open-localization}
  Let $X$ be a scheme and $M:X\to \Mod{R}$ weakly quasi-coherent,
  then for all open $U\subseteq X$ and $f:U\to R$
  the canonical morphism
  \[
    M(U)_f\to M(D(f))
  \]
  is an equivalence.
\end{lemma}

\begin{proof}
  By construction of the canonical map from \Cref{localization-to-restriction}.
\end{proof}

Let us look at an example.

\begin{proposition}%
  \label{fp-algebra-bundle-is-quasi-coherent}
  Let $X$ be a scheme and $C:X\to \Alg{R}_{fp}$.
  Then $C$, as a bundle of $R$-modules, is weakly quasi coherent.
\end{proposition}

\begin{proof}
  Then for any $f:R$ and $x:X$, using \Cref{algebra-valued-functions-on-affine}, we have
  \[
    (C_x)_f=C_x\otimes_R R_f=(\Spec R_f \to C_x)=(D(f)\to C_x)={C_x}^{D(f)}
    \rlap{.}
  \]
\end{proof}

For examples of non weakly quasicoherent modules,
see \Cref{non-wqc-module-family}
and \Cref{RN-non-wqc}.

\begin{lemma}[using \axiomref{loc}, \axiomref{sqc}, \axiomref{Z-choice}]%
  \label{weakly-quasi-coherent-pi}
  Let $X$ be an affine scheme and $M_x$ a weakly quasi-coherent $R$-module for any $x:X$,
  then
  \[
    \prod_{x:X}M_x
  \]
  is a weakly quasi-coherent $R$-module.
\end{lemma}

\begin{proof}
  We need to show:
  \[
    \left(\prod_{x:X}M_x\right)_f=\left(\prod_{x:X}M_x\right)^{D(f)}
  \]
  for all $f:R$.
  By weak \Cref{module-bundle-localization-pointwise}, quasi-coherence
  and \Cref{weakly-quasi-coherent-open-localization}
  we know:
  \[
    \left(\prod_{x:X}M_x\right)_f
    =\prod_{x:X}\left(M_x\right)_{f(x)}
    =\prod_{x:X}\left(M_x\right)^{D(f)}
    =\left(\prod_{x:X}M_x\right)^{D(f)}
    \rlap{.}
  \]
\end{proof}

Quasi-coherent dependent modules turn out to have very good properties,
which are to be expected from what is known about their external counterparts.
We will show below, that quasi coherence is preserved by the following constructions:

\begin{definition}
  \label{pullback-push-forward}
  Let $X,Y$ be types and $f:X\to Y$ be a map.
  \begin{enumerate}[(a)]
  \item \index{$f^*M$} For any dependent module $N:Y\to\Mod{R}$,
    the \notion{pullback} or \notion{inverse image} is the dependent module
    \[
      f^*N\colonequiv (x:X) \mapsto M_{f(x)}\rlap{.}
    \]
  \item \index{$f_*M$} For any dependent module $M:X\to\Mod{R}$,
    the \notion{push-forward} or \notion{direct image} is the dependent module
    \[
      f_*M\colonequiv (y:Y) \mapsto \prod_{x:\fib_f(y)}M_{\pi_1(x)}\rlap{.}
    \]
  \end{enumerate}
\end{definition}

\begin{theorem}[using \axiomref{loc}, \axiomref{sqc}, \axiomref{Z-choice}]%
  \label{pullback-push-forward-qcoh}
  Let $X,Y$ be schemes and $f:X\to Y$ be a map.
  \begin{enumerate}[(a)]
  \item For any weakly quasi-coherent dependent module $N:Y\to\Mod{R}$,
    the inverse image $f^*N$ is weakly quasi-coherent.
  \item For any weakly quasi-coherent dependent module $M:X\to\Mod{R}$,
    the direct image $f_*M$ is weakly quasi-coherent.
  \end{enumerate}
\end{theorem}

\begin{proof}
  \begin{enumerate}[(a)]
  \item There is nothing to do, when we use the pointwise definition of weak quasi-coherence. 
  \item We need to show, that
    \[
      \prod_{x:\fib_f(y)}M_{\pi_1(x)}
    \]
    is a weakly quasi-coherent $R$-module.
    By \Cref{fiber-product-scheme},
    the type $\fib_f(y)$ is a scheme.
    So by \Cref{weakly-quasi-coherent-pi},
    the module in question is weakly quasi-coherent.
  \end{enumerate}
\end{proof}

With a non-cyclic forward reference to a cohomological result,
there is a short proof of the following:

\begin{proposition}[using \axiomref{loc}, \axiomref{sqc}, \axiomref{Z-choice}]%
  Let $f:M\to N$ be an $R$-linear map of weakly quasi-coherent $R$-modules $M$ and $N$,
  then the cokernel $N/M$ is weakly quasi-coherent.
\end{proposition}

\begin{proof}
  We will first show, that for an $R$-linear embedding $m:M\to N$
  of weakly quasi-coherent $R$-modules $M$ and $N$,
  the cokernel $N/M$ is weakly quasi-coherent.
  We need to show:
  \[
    (N/M)_f=(N/M)^{D(f)}.
  \]
  By algebra: $(N/M)_f=N_f/M_f$.
  This means we are done, if $(N/M)^{D(f)}=N^{D(f)}/{M^{D(f)}}$.
  To see this holds, let us consider $0\to M\to N\to N/M\to 0$ as a short exact sequence of dependent modules,
  over the subtype of the point $D(f)\subseteq 1=\Spec R$.
  Then, taking global sections, by \Cref{cohomology-les},
  we have an exact sequence
  \[
    0\to M^{D(f)}\to N^{D(f)}\to (N/M)^{D(f)}\to H^1(D(f),M)
  \]
  -- but $D(f)=\Spec R_f$ is affine,
  so the last term is 0 by \Cref{H1-wqc-module-affine-trivial}
  and $(N/M)^{D(f)}$ is the cokernel $N^{D(f)}/M^{D(f)}$.

  Now we will show the statement for a general $R$-linear map $f:M\to N$.
  By algebra, the cokernel of $f$ is the same as the cokernel of the induced map
  $M/K\to N$, where $K$ is the kernel of $f$.
  By \Cref{kernel-wqc}, $K$ is weakly quasi-coherent, so by the proof above,
  $M/K$ is weakly quasi-coherent.
  $M/K\to N$ is an embedding, so again by the proof above, its cokernel is weakly quasi-coherent.
\end{proof}

\subsection{Finitely presented bundles}

We now investigate the relationship between bundles of $R$-modules on $X = \Spec A$
and $A$-modules.

\begin{proposition}
  Let $A$ be a finitely presented $R$-algebra.
  There is an adjunction
  \[ \begin{tikzcd}[row sep=tiny]
    M \ar[r, mapsto] & {(M \otimes x)}_{x : \Spec A} \\
    \Mod{A} \ar[r, shift left=2] \ar[r, phantom, "\rotatebox{90}{$\vdash$}"] &
    \Mod{R}^{\Spec A} \ar[l, shift left=2] \\
    \prod_{x : \Spec A} N_x & N \ar[l, mapsto]
  \end{tikzcd} \]
  between the category of $A$-modules
  and the category of bundles of $R$-modules on $\Spec A$.
\end{proposition}

For an $A$-module $M$,
the unit of the adjunction is:
\begin{align*}
  \eta_M : M &\to \prod_{x : \Spec A} (M \otimes x) \\
  m &\mapsto (m \otimes 1)_{x : \Spec A}
\end{align*}

\begin{example}[using \axiomref{sqc}, \axiomref{loc}]
  It is not the case that
  for every finitely presented $R$-algebra $A$
  and every $A$-module $M$
  the map $\eta_M$ is injective.
\end{example}

\begin{proof}
  \cite{topology-draft}.
\end{proof}

\begin{theorem}%
  \label{fp-module}
  Let $X=\Spec(A)$ be affine and
  let a bundle of finitely presented $R$-modules $M : X\to \fpMod{R}$ be given.
  Then the $A$-module
  \[ \tilde{M}\coloneqq\prod_{x:X}M_x \]
  is finitely presented and for any $x:X$ the $R$-module $\tilde{M}\otimes_A R$ is $M_x$.
  Under this correspondence, localizing $\tilde{M}$ at $f:A$ corresponds to restricting $M$ to $D(f)$.
\end{theorem}

\subsection{Cohomology on affine schemes}

\begin{definition}%
  \label{torsor}
  Let $X$ be a type and $A:X\to \AbGroup$ a map to the type of abelian groups.
  For $x:X$ let $T_x$ be a set with an $A_x$ action.
  \begin{enumerate}[(a)]
  \item $T$ is an \notion{$A$-pseudotorsor}, if the action is free and transitive for all $x:X$.
  \item $T$ is an \notion{$A$-torsor}, if it is an $A$-pseudotorsor and
    \[ \prod_{x:X} \propTrunc{ T_x } \rlap{.}\]
  \item We write $\Tors{A}(X)$ for the type of $A$-torsors on $X$.
  \end{enumerate}
\end{definition}

Torsors on a point are a concrete implementation of first deloopings:

\begin{definition}
  \label{delooping}
  Let $n:\N$.
  A $n$-th \notion{delooping}\index{$K(A,n)$} of an abelian group $A$,
  is a pointed, $(n-1)$-connected, $n$-truncated type $K(A,n)$,
  such that $\Omega^nK(A,n)=_{\AbGroup}A$.
\end{definition}

For any abelian group and any $n$, a delooping $K(A,n)$ exists by \cite{licata-finster}.
Deloopings can be used to represent cohomology groups by mapping spaces.
This is usually done in homotopy type theory to study higher inductive types, such as spheres and CW-complexes,
but the same approach works for internally representing sheaf cohomology,
which is the intent of the following definition:

\begin{definition}
  \label{cohomology}
  Let $X$ be a type and $\mathcal F:X\to\AbGroup$ a dependent abelian group.
  The $n$-th cohomology group of $X$ with coefficients in $\mathcal F$ is
  \[
    H^n(X,\mathcal F)\colonequiv \left\propTrunc{\prod_{x:X}K(\mathcal F,n)\right}_0\rlap{.}
  \]
\end{definition}

\begin{theorem}%
  \label{cohomology-les}
  Let $\mathcal F,\mathcal G,\mathcal H:X\to \AbGroup$ be such that for all $x:X$,
  \[
    0\to \mathcal F_x\to\mathcal G_x\to\mathcal H_x\to 0
  \]
  is an exact sequence of abelian groups. Then there is a long exact sequence:
  \begin{center}
    \begin{tikzcd}
      & \dots\ar[r] & H^{n-1}(X,\mathcal H)\ar[dll] \\
      H^n(X,\mathcal F)\ar[r] & H^n(X,\mathcal G)\ar[r] & H^n(X,\mathcal H)\ar[dll] \\
      H^{n+1}(X,\mathcal F)\ar[r] & \dots &
    \end{tikzcd}
  \end{center}
\end{theorem}

\begin{proof}
  By applying the long exact homotopy fiber sequence.
\end{proof}

The following is an explicit formulation of the fact, that the Čech-Complex for an
$\mathcal{O}_X$-module sheaf on $X=\Spec(A)$ given by an $A$-module $M$ is exact in degree 1.
\begin{lemma}%
  \label{H1-algebra}
  Let $M$ be a module over a commutative ring $A$, $F_1,\dots,F_l$ a coprime system on $A$
  and for $i,j\in\{1,\dots,l\}$, let $s_{ij} : F_i^{-1} F_j^{-1} M$ such that:
  \[ s_{jk}-s_{ik}+s_{ij}=0 \rlap{.}\]
  Then there are $u_i:F_i^{-1}M$ such that $s_{ij}=u_j - u_i$.
\end{lemma}

\begin{proof}
  Let $s_{ij}=\frac{m_{ij}}{f_i f_j}$ with $m_{ij}:M$, $f_i:F_i$ and $f_j:F_j$ such that:
  \[ f_i\cdot m_{jk}-f_j\cdot m_{ik}+f_k\cdot m_{ij}=0 \rlap{.}\]
  Let $r_i$ such that $\sum r_i f_i =1$.
  Then for
  \[ u_i \coloneqq -\sum_{k=1}^l\frac{r_k}{f_i}m_{ik} \]
  we have:
  \begin{align*}
      u_j-u_i &= -\sum_{k=1}^l\frac{r_k}{f_j}m_{jk} + \sum_{k=1}^l\frac{r_k}{f_i}m_{ik} \\
              &= -\sum_{k=1}^l\frac{r_k}{f_j f_i}f_i m_{jk} + \sum_{k=1}^l\frac{r_k}{f_i f_j} f_j m_{ik} \\
              &= \sum_{k=1}^l\frac{r_k}{f_j f_i}(-f_i m_{jk} + f_j m_{ik}) \\
              &= \sum_{k=1}^l\frac{r_k}{f_j f_i}f_k m_{ij} \\
              &= \frac{m_{ij}}{f_i f_j}
  \end{align*}
  \ %
\end{proof}

\begin{theorem}[using \axiomref{loc}, \axiomref{sqc}, \axiomref{Z-choice}]%
  \label{H1-wqc-module-affine-trivial}
  For any affine scheme $X=\Spec(A)$ and coefficients $M: X\to \Mod{R}_{wqc}$, we have
  \[ H^1(X,M)=0 \rlap{.} \]
\end{theorem}

\begin{proof}
  We need to show, that any $M$-torsor $T$ on $X$ is merely equal to the trivial torsor $M$,
  or equivalently show the existence of a section of $T$.
  We have
  \[ \prod_{x:X}\propTrunc{ T_x }\]
  and therefore, by (\axiomref{Z-choice}),
  there merely are $f_1,\dots,f_l:A$,
  such that the $U_i\coloneqq \Spec(A_{f_i})$ cover $X$ and
  there are local sections
  \[ s_i:\prod_{x:U_i}T_x\]
  of $T$. Our goal is to construct a matching family from the $s_i$.
  On intersections, let $t_{ij}\coloneqq s_i-s_j$ be the difference, so $t_{ij}:(x : U_i\cap U_j) \to M_x$.
  By \Cref{weakly-quasi-coherent-open-localization} equivalently,
  we have $t_{ij}:M(U_{i}\cap U_j)_{f_i f_j}$.
  Since the $t_{ij}$ were defined as differences,
  the condition in \Cref{H1-algebra} is satisfied and we get
  $u_i:M(U_i)_{f_i}$, such that $t_{ij}=u_i-u_j$.
  So we merely have a matching family $\tilde{s}_i\coloneqq s_i-u_i$ and therefore, using Lemma \ref{kraus-glueing} merely a section of $T$.
\end{proof}

A similar result is provable for $H^2(X,M)$ using the same approach.
There is an extension of this result to general $n$ in work in progress \cite{chech-draft}.

\subsection{Čech-Cohomology}

In this section, let $X$ be a type, $U_1,\dots,U_n\subseteq X$ open subtypes that cover $X$
and $\mathcal F:X\to \AbGroup$ a dependent abelian group on $X$.
We start by repeating the classical definition of \v{C}hech-Cohomology groups for a given cover.

\begin{definition}%
  \label{chech-complex}
  \begin{enumerate}[(a)]
  \item \index{$\mathcal F(U)$} For open $U\subseteq X$, we use the notation from \Cref{application-of-bundle-to-subtype}:
    \[
      \mathcal F(U)\colonequiv \prod_{x:U}\mathcal F_x\rlap{.}
    \]
  \item For $s:\mathcal F(U)$ and open $V\subseteq U$ we use the notation $s\colonequiv s_{|V} \colonequiv (x:V)\mapsto s_x$.
  \item \index{$U_{i_1\dots i_l}$}For a selection of indices $i_1,...,i_l:\{1,\dots,n\}$, we use the notation
    \[
      U_{i_1\dots i_l}\colonequiv U_{i_1}\cap\dots\cap U_{i_l}\rlap{.}
    \]
  \item For a list of indices $i_1,\dots,i_l$, let $i_1,\dots,\hat{i_t},\dots,i_l$ be the same list with the $t$-th element removed.
  \item For $k:\Z$, the $k$-th \notion{Čech-boundary operator}\index{$\partial^k$} is the homomorphism
    \[
      \partial^k:\bigoplus_{i_0,\dots,i_k}\mathcal F(U_{i_0\dots i_k})\to \bigoplus_{i_0,\dots,i_{k+1}}\mathcal F(U_{i_0\dots i_{k+1}})
    \]
    given by $\partial^k(s)\colonequiv (l_0,\dots,l_{k+1}) \mapsto \sum_{j=0}^k (-1)^j s_{l_0,\dots,\hat{l_j},\dots,l_k|U_{l_0,\dots,l_{k+1}}}$.
  \item The $k$-th \notion{Čech-Cohomology group} for the cover $U_1,\dots,U_n$ with coefficients in $\mathcal F$ is
    \[
      \check{H}^k(\{U\},\mathcal F)\colonequiv \ker\partial^{k} / \im(\partial^{k-1})\rlap{.}
    \]
  \end{enumerate}
\end{definition}

It is possible to construct a torsor from a \v{C}ech cocycle:

\begin{lemma}%
  \label{deligne-construction}
  Let $A$ be an abelian group and $L$ a type with $\propTrunc{L}$.
  Let us call $c:(i,j:L)\to A$ a $L$-cocycle, if $c_{ij}+c_{jk}=c_{ik}$ for all $i,j,k:L$.
  Then there is a bijection:
  \[
    \left((T:\text{$A$-torsor})\times T^L\right) \to \text{$L$-cocycles}
    \rlap{.}
  \]
\end{lemma}

\begin{proof}
  Let us first check, that the left side is a set.
  Let $(T,u),(T',u'):(T:\text{$A$-torsor})\times T^L$,
  then $(T,u)=(T',u')$ is equivalent to $(e:T\cong T')\times ((i:L)\to e(u_i)=u'_i)$.
  But two maps $e$ with this property are equal,
  since a map between torsors is determined by the image of a single element and $L$ is inhabited.
  
  Assume now $(T,u):(T:\text{$A$-torsor})\times T^L$ to construct the map.
  Then $c_{ij}\colonequiv u_i-u_j$ defines an $L$-cocycle,
  because
  \[
    u_i-u_j + u_j-u_k = u_i-u_k
    \rlap{.}
  \]
  This defines an embedding: Assume $(T,u)$ and $(T',u')$ define the same $L$-cocycle,
  then $u_i-u_j=u'_i-u'_j$ for all $i,j:L$.
  We want to show a proposition, so we can assume there is $i:L$ and use that to get a map $e:T\to T'$
  that sends $u_i$ to $u'_i$.
  But then we also have
  \[
    e(u_j)=e(u_j-u_i+u_i)=e(u'_j-u'_i+u_i)=u'_j-u'_i+e(u_i)=u'_j-u'_i+u'_i=u'_j
  \]
  for all $j:L$, which means $(T,u)=(T',u')$.
    
  Now let $c$ be an $L$-cocycle.
  Following \cite{Deligne91}[Section 5.2], we can define a preimage-candidate:
  \[
    T_c\colonequiv \{u:A^L\mid u_i-u_j=c_{ij}\}
    \rlap{.}
  \]
  $A$ acts on $T_c$ pointwise, since $(a+u_i)-(a+u_j)=u_i-u_j=c_{ij}$ for all $a:A$.
  
  To show that $T_c$ is inhabited,
  we may assume $i_0:L$.
  Then we define $u_i\colonequiv -c_{i_0i}$ to get $u_i-u_j=-c_{i_0i}+c_{i_0j}=c_{ij}$.

  Now $c$ is of type $(A^L)^L=A^{L\times L}$, so we have an element of the left hand side.
  Applying the map constructed above yields a cocycle
  \[
    \tilde{c}_{ij}=(k\mapsto c_{ki})-(k\mapsto c_{kj})=(k\mapsto c_{ki}-c_{kj})=(k\mapsto c_{kj}+c_{ji}-c_{kj})=(k\mapsto c_{ji})
  \]
  -- so $(T_c,c)$ is a preimage of $c_{ij}$.
\end{proof}

\begin{definition}
  The cover $U_1,\dots,U_n$ is called \notion{r-acyclic} for $\mathcal F$,
  if we have the following triviality of higher (non Čech) cohomology groups:
  \[
    \forall l, r\geq l>0\ \forall i_0,\dots,i_{r-l}. H^l(U_{i_0,\dots,i_{r-l}},\mathcal F)=0\rlap{.}
  \]
\end{definition}

\begin{example}
  If $X$ is a scheme, $U_1,\dots,U_n$ a cover by affine open subtypes
  and $\mathcal F$ pointwise a weakly quasi coherent $R$-module,
  then $U_1,\dots,U_n$ is 1-acyclic for $\mathcal F$ by \Cref{H1-wqc-module-affine-trivial}.
\end{example}

\begin{theorem}[using \axiomref{Z-choice}]%
  If $U_1,\dots,U_n$ is a 1-acyclic cover for $\mathcal F$, then
  \[
    \check{H}^1(\{U\},\mathcal F)=H^1(X,\mathcal F)\rlap{.}
  \]
\end{theorem}

\begin{proof}
  Let $\pi$ be the projection map
  \[
    \pi :
    \left(
      \sum_{T:\Tors{\mathcal F}(X)}\prod_{i}\prod_{x:U_i}T_x
    \right)
    \to \Tors{\mathcal F}(X)\rlap{.}
  \]
  Let us abbreviate the left hand side with $T(\mathcal F,U)$.
  Since the cover is 1-acyclic, $\pi$ is surjective.
  With $L_x\colonequiv \sum_{i}U_i(x)$ and \Cref{deligne-construction} we get:
  \begin{align*}
    T(\mathcal F,U)&=\prod_{x:X}(T_x:\Tors{\mathcal F_x})\times T_x^{L_x} \\
                   &=\prod_{x:X}\text{$L_x$-cocycles}
                     \rlap{.}
  \end{align*}
  The latter is the type of \v{C}ech-1-cocycles (\Cref{chech-complex} (e))
  and in total the equality is given by the isomorphism
  \[
    (T,t) \mapsto (i,j\mapsto t_i - t_j) :
    T(\mathcal F,U)
    \to
    \ker(\partial^1)
    \subseteq
    \bigoplus_{i,j}\mathcal F(U_{ij})\rlap{.}
  \]

  Realizing, that $\im(\partial^0)$ corresponds to the subtype of $T(\mathcal F,U)$ of trivial torsors,
  we arrive at the following diagram:
  \begin{center}
    \begin{tikzcd}
      & \Tors{\mathcal F}(X)\ar[r,->>] & H^1(X,\mathcal F) \\
      \sum_{T:T(\mathcal F,U)}\propTrunc{\pi_1(T)=\mathcal F}\ar[r,hook] & T(\mathcal F,U)\ar[u,->>]\ar[d,equal] & \\
      \im{\partial^0}\ar[r,hook]\ar[u,equal] & \ker{\partial^1}\ar[r,->>] & \check{H}^1(\{U\},\mathcal F)
    \end{tikzcd}
  \end{center}
  The composed map $T(\mathcal F,U)\to H^1(X,\mathcal F)$ is a homomorphism
  and therefore by \Cref{surjective-abgroup-hom-is-cokernel} a cokernel.
  So the two cohomology groups are equal, since they are cokernels of the same diagram.
\end{proof}

It is possible to pass from torsors to gerbes,
which are the degree 2 analogue of torsors:

\begin{definition}
  \label{gerbe}
  Let $A:\AbGroup$ be an abelian group.
  An \notion{$A$-banded gerbe}, is a connected type $\mathcal G:\mathcal U$,
  together with, for all $y:\mathcal G$ an identification of groups $\Omega (\mathcal G,y)=A$.
\end{definition}

Analogous to the type of $A$-torsors, the type of $A$-banded gerbes is a second delooping of an abelian group $A$.
We can formulate a second degree version of \Cref{deligne-construction}:

\begin{theorem}
  \label{deligne-construction-gerbes}
  Let $A$ be an abelian group and $L$ a type with $\propTrunc{L}$.
  Let us call $c:(i,j,k : L)\to A$ a $L$-2-cocycle,
  if $c_{jkl}-c_{ikl}+c_{ijl}-c_{ijk}=0$ for all $i,j,k,l : L$.
  Then there is a bijection:
  \[
    \left((\mathcal G:\text{$A$-gerbe})\times (u:\mathcal G^{L})\times (i,j : L) \to u_i=u_j\right) \to \text{$L$-2-cocycle}
    \rlap{.}
  \]
\end{theorem}

This is provable, again, by translating Deligne's argument \cite{Deligne91}[Section 5.3].
Using this, the correspondence of Eilenberg-MacLane-Cohomology and \v{C}ech-Cohomology can be extended in the following way:

\begin{theorem}
  If $U_1,\dots,U_n$ is a 2-acyclic cover for $\mathcal F$, then
  \[
    \check{H}^2(\{U\},\mathcal F)=H^2(X,\mathcal F)\rlap{.}
  \]  
\end{theorem}

However, with this approach, we need versions of \Cref{kraus-glueing}, with increasing truncation level.
While this suggests, we can prove the correspondence for any cohomology group of \emph{external} degree $l$,
there is follow-up work in progress \cite{chech-draft},
which proves the correspondence for all \emph{internal} $l:\N$.
A stronger assumption is needed in that work, though: The cover $U_1,\dots,U_n$ needs to be acyclic in the sense,
that all higher cohomology groups of all intersections $U_{i_0}\cap \dots \cap U_{i_k}$, with $k:\N$, vanish.
In the same draft, there is also a version of the vanishing result for all internal $l$,
and it is shown that separated schemes, like projective space, have an acyclic cover.
This means that many of the usual, essential computations with \v{C}ech-Cohomology can be transferred to synthetic algebraic geometry.


\section{External justification of axioms}
\subsection{Justification of \axiomref{Z-choice}}

\begin{lemma}
  Let $(C, J)$ be a site,
  where the Grothendieck topology $J$ is subcanonical.
  Let
  \[ f : E \twoheadrightarrow \yo(c) \]
  be an epimorphism in $\Sh(C, J)$ with representable codomain.
  Then there is a $J$-cover $(c_i \to c)_{i \in I}$ of $c$
  such that for every $i$,
  the pullback of $f$ along $\yo(c_i) \to \yo(c)$
  is a split epimorphism.
  \[ \begin{tikzcd}
    E_i \ar[r] \ar[d, two heads, "f_i"] \ar[dr, phantom, very near start, "\lrcorner"] &
    E \ar[d, two heads, "f"] \\
    \yo(c_i) \ar[r] \ar[u, bend left, dashed]&
    \yo(c)
  \end{tikzcd} \]
\end{lemma}

\begin{proof}
  By the Yoneda lemma,
  an epimorphism $E \twoheadrightarrow \yo(c)$ is split
  if and only if
  the particular element $\id_c \in \yo(c)(c)$
  is in the image of the map $E(c) \to \yo(c)(c)$.
  Applying the usual characterization of epimorphisms of sheaves
  \cite[Corollary III.7.5]{maclane-moerdjik}
  to the element $\id_c \in \yo(c)(c)$
  shows that there is a $J$-cover ${(c_i \xrightarrow{g_i} c)}_{i \in I}$
  such that for every $i \in I$,
  there is some $e_i \in E(c_i)$
  with $f_{c_i}(e_i) = g_i \in \yo(c)(c_i)$.
  But this means that $\id_{c_i}$ is in the image of ${(f_i)}_{c_i} : E_i(c_i) \to \yo(c_i)(c_i)$,
  as we can see by evaluating the pullback diagram at $c_i$.
  So $f_i$ is a split epimorphism.
\end{proof}

Let us formulate a version of the axiom \axiomref{Z-choice}
in infinitary first-order logic extended with
unbounded quantification over objects/sorts
($\exists A. \varphi$, $\forall A. \varphi$)
and quantification over functions
($\exists f : A \to B. \varphi$, $\forall f : A \to B. \varphi$)
as in Shulmans stack semantics \cite[Section 7]{shulman-stack-semantics}.

We also use the syntax $\{x : A \mid \varphi(x)\}$
for bounded set comprehension,
but this can be translated away.
TODO


\section{Type Theoretic justification of axioms}
\newcommand{\inc}{\mathsf{inc}}
\newcommand{\inl}{\mathsf{inl}}
\newcommand{\inr}{\mathsf{inr}}
%\newcommand{\UU}{\mathcal{U}}
\newcommand{\II}{\mathbf{I}}
\newcommand\norm[1]{\left\lVert #1 \right\rVert}


\newcommand{\Gm}{\mathsf{G_m}}
\newcommand{\ext}{\mathsf{ext}}
\newcommand{\patch}{\mathsf{patch}}
\newcommand{\cov}{\mathsf{cov}}
\newcommand{\isSheaf}{\mathsf{isSheaf}}
\newcommand{\Fib}{\mathsf{Fib}}

\newcommand{\BB}{\mathcal{B}}
\newcommand{\CC}{\mathcal{C}}
\newcommand{\UU}{\mathcal{U}}

In this section, we present models of the 3 axioms, models that are inspired by type theory. The first model is a sheaf model, and does not
take into account universe (and univalence). This model is built
as an {\em internal} model inside a presheaf model. Rather surprisingly, the same method
works if we start from a(n effective) model of univalence. We can localise at a family of open modalities, and  obtain
in this way a model of type theory with univalence satisfying the
3 axioms for Zariski topos. It is then natural to ask how the global section operation behaves for this model, and we show that
it satisfies a property similar to Zariski local choice. 

\subsection{$1$-topos model}

We first build a $1$-sheaf model of the 3 axioms. This model will be formulated as a model of {\em dependent type theory}.
This will be an {\em internal} model inside a presheaf model. While the presheaf model covers universes, this internal model
will not cover universes (and univalence), and we need a refined notion of model in the next section to cover these extensions.

For any small category $\CC$ we can form the presheaf model of type theory over the base category $\CC$.

A context is interpreted as a presheaf over $\CC$. A type over a context $\Gamma$ is interpreted as
a presheaf over the category of elements of $\Gamma$, and an element of this type as a global section
of this presheaf. For any set theoretic universe $\UU$, we have a corresponding presheaf which to
any object $X$ associates the set of $\UU$-small types over $Yo(X)$, and this gives the interpretion
of universes. 

\medskip

We look at the special case where $\CC$ is the opposite of the category of finitely presented $k$-algebra for a fixed
ring $k$.

    In this model we have a presheaf $R(A) = Hom(k[X],A)$ which has a ring structure.

    We have the truth value presheaf $\Omega$ with $\Omega(A)$ is the set of sieves on $A$.
    (There is a predicative version of this object if we are working in a predicative metatheory.)

    We have a dependent type $[p]$ for $p:\Omega$, with $[p]$ subpresheaf of the unit presheaf $\top$.

    We also have a presheaf $Cov$, where $Cov(A)$ is the set of finite sequences $f_1,\dots,f_n$ that are comaximal: $1$ is in the
    ideal generated by $f_1,\dots,f_n$.

    We define $Cov\rightarrow\Omega,~c\mapsto p_c$ by defining $p_c$ to be the proposition $\inv(f_1)\vee\dots\vee\inv(f_n)$
    for $c = f_1,\dots,f_c$. In particular, for $n=0$, this is the proposition $\perp$.

    We can then define the property $\isSheaf$ of being a sheaf for a type $T$ as being the fact that
    the constant map $T\rightarrow T^{[p_c]}$ are isomorphism for all $c:Cov$.

    \medskip

    One can check directly that $\Pi_{x:A}B$ is a sheaf if $B$ is a family of sheaves over $A$
    and that $\Sigma_{x:A}B$ is a sheaf if $A$ is a sheaf and $B$ is a family of sheaves. We hence get a model of type theory
    by interpreting a type as a sheaf. A context is still interpreted as a presheaf, while a type is now a dependent presheaf
    with a proof that it is {\em modal}. An element is a section of the underlying dependent presheaf.

    \medskip

    Any representable presheaf is a sheaf, and in this new model $R$ is a local ring. In particular, we have $\neg (0 = 1)$.
    
    \medskip

    However the type of sheaves in given universe $\Sigma_{X:\UU}[\isSheaf(X)]$ is not a sheaf.

    We present another model which both interprets universe and {\em univalence} and {\em propositional truncation}.

\subsection{Models of univalence}

The constructive models of univalence are presheaf models parametrised by an interval object $\II$
(presheaf with two global distinct elements $0$ and $1$ and which is tiny) and a classifier object
$\Phi$ for cofibrations.

 This is over a base category $\BB$.
 
 If we have another category $\CC$, we automatically get a new model of univalent type theory by
 changing $\BB$ to $\BB\times\CC$.

 A particular case is if $\CC$ is the opposite of the category of f.p. $k$-algebras, where $k$ is a
 fixed commutative ring.

 We have the presheaf $R$ defined by $R(J,A) = Hom(k[X],A)$ where $J$ object of $\BB$ and $A$ object of $\CC$.

  The presheaf $\Gm$ is defined by $\Gm(J,A) = Hom(k[X,1/X],A) = A^{\times}$ set of invertible elements of $A$.

\subsection{Propositional truncation}

    We start by giving a simpler interpretation of propositional truncation. This will simplify
    the proof of the validity of Zariski local choice in the model.

    We work in the presheaf model over a base category $\BB$ which interprets univalent type theory,
    with a presheaf $\Phi$ of cofibrations. The interpretation of the propositional
    truncation $\norm{T}$ {\em does not} require the use of the interval $\II$.

    We recall that in the models, to be contractible can be formulated as having an operation
    $\ext(\psi,v)$ which extends any partial element $v$ of extent $\psi$ to a total element.

    The (new) remark is then that to be a (h)proposition can be formulated as having instead
    an operation $\ext(u,\psi,v)$ which, now {\em given}
    an element $u$, extends any partial element $v$ of extent $\psi$ to a total element.

\medskip    

Propositional truncation is defined as follows. An element of $\norm{T}$ is either of the form
$\inc(a)$ with $a$ in $T$, or of the form $\ext(u,\psi,v)$ where $u$ is in $\norm{T}$ and $\psi$
in $\Phi$ and $v$ a partial element of extent $\psi$.

In this definition, the special constructor $\ext$ is a ``constructor with restrictions'' which
satisfies $\ext(u,\psi,v) = v$ on the extent $\psi$.

\subsection{Covering}

We want to force $e_c = \norm{\inv(f_1)+\dots+\inv(f_n)}$ for $f_1,\dots,f_n$ in $Cov$.

Note that we have $e_c\rightarrow p_c$, with $p_c$ is a {\em strict} proposition $\inv(f_1)\vee \dots\vee\inv(f_n)$.
If we work in an effective metatheory, without choices, we cannot expect to have $p_c\rightarrow e_c$.

\begin{lemma}
  The constant map $\sigma:R\rightarrow R^{e_c}$ is an isomorphism.
\end{lemma}

\begin{proof}
  It is enough to define a map $\patch:R^{e_c}\rightarrow R$ such that $\patch(\sigma(r)) = r$ since
  $e_c$ is a hproposition.

  If we are at a stage $A$ and $f_1,\dots,f_n$ comaximal in $A$, and $u:R^{e_c}(A)$ we define $\patch(u)$ as follows.
  Note that $e_c$ becomes inhabited in each $A[1/f_i]$. We get then a family of element $u_i$ in each $A[1/f_i]$, with $u_i$ compactible
  with $u_j$. We can then patch these elements together to an element in $A$.
\end{proof}

(Since $R$ is a strict hset, the canonical map $R^{p_c}\rightarrow R^{e_c}$ is an isomorphism.)

We now consider the model which, intuitively, forces all these hpropositions $e_c$ to be contractible.
This is obtained by restricting ourseleves to {\em modal types} $T$, the types
$T$ such that the constant map $T\rightarrow T^{e_c}$ is an {\em equivalence}.

\medskip

We have just seen that the strict hset $R$ is such a modal type.

The same reasoning actually shows that each representable presheaf, in particular $\Gm$, is a modal
(strict) hset.

\medskip

We get a model of type theory with univalence where a type is interpreted as a modal type.
In this model $\perp$ is interpreted as $1=0$. We can check that if $T$ is modal then
$T^{1=0}$ is contractible (using the covering with $n=0$), since this is equivalent to $(T^{\perp})^{1=0}$.

\medskip

If $P$ is a proposition, then $P$ is modal if, and only if,
we have $P^{e_c}\rightarrow P$. (Thus $\perp$ is not modal, since we don't have $\neg\neg(1=0)$; indeed we have instead $\neg(1=0)$ in
the original presheaf model.)
Since $P$ is a
proposition, $P^{e_c}$ is equivalent to $P^{\inv(f_1)}\times \dots\times P^{\inv(f_n)}$.

\medskip

We can use this characterisation of modal proposition to define the interpretation of
$\norm{T}(A)$ in the {\em sheaf} model.

An element in $\norm{T}_M(A)$ is
\begin{enumerate}
\item either $\inc(a)$ with $a$ in $T(A)$,
\item or of the form $\ext(u,\psi,v)$ with $u$ in $\norm{T}_M(A)$ and $\psi$ in $\Phi$ and
  $v$ in $\norm{T}_M(A)$ of extent $\psi$,
\item or of the form $\cov(f_1,u_1,\dots,f_n,u_n)$ with $f_1,\dots,f_n$ in $Cov(A)$ and $u_i$ in $\norm{T}(A[1/f_i])$.
\end{enumerate}

If $\alpha:A\rightarrow B$ we define the restriction operation inductively as follows.
\begin{enumerate}
\item $\inc(a)|\alpha = \inc(a|\alpha)$
\item $\ext(u,\psi,v)|\alpha = \ext(u|\alpha,\psi,v|\alpha)$
\item $\cov(f_1,u_1,\dots,f_n,u_n)|\alpha = \cov(\alpha f_1,u_1|\alpha_1,\dots,f_n,u_n|\alpha_n)$ where
  $\alpha_i$ is the composition $A\rightarrow B\rightarrow B[1/\alpha f_i]$.
\end{enumerate}

If $P$ is a modal proposition and $u:T\rightarrow P$ we can define a map $v:\norm{T}_M\rightarrow P$
such that $v\circ\inc = u$ as a strict equality.

We can then define $l:\Pi_{f_1,\dots,f_n:R}\norm{\inv(f_1)+\dots+\inv(f_n)}$.

\subsection{Zariski local choice}

If $c = f_1,\dots,f_n$ is a covering of $A$ and $P:Sp(A)\rightarrow \UU$ we define
$\Pi_c P$ to be $\Pi_{D(f_1)}P\times\Pi_{D(f_n)}P$. In this way, $\Pi_c$ is an operation
$\UU^{Sp(A)}\rightarrow\UU$.

We prove local choice: if $A$ is a f.p. algebra over $R$ then we have a map
$$
l:(\Pi_{x:Sp(A)}\norm{P})\rightarrow \norm{\Sigma_{c:Cov(A)}\Pi_cT}
$$
At a stage $B$ a f.p. algebra over $R$ is given by $B\rightarrow A$ and we have $Yo(B).Sp(A)$ isomorphic
to $Yo(A)$.

For defining the map $l$, we define $l(v)$ by induction on $v$.
The element $v$ is in $(\Pi_{x:Sp(A)}\norm{P})(B)$, which can be seen as
an element of $\norm{P}(A)$. If it is $\inc(u)$ we associate $\inc(1,u)$ with the covering $1$ of $A$.
If it is $\ext(u,\psi,v)$ the image is $\ext(l(u),\psi,l(v))$.
If it is $\cov(f_1,u_1,\dots,f_n,u_n)$ we have by induction $l(u_i)$ in $\norm{\Sigma_{c:Cov(A[1/f_i])}\Pi_c P}$.
We can then conclude using the law $\norm{A}\times \norm{B}\rightarrow\norm{A\times B}$.

\medskip

The fact that $R$ is local holds like in the $1$-topos case, and similarly for the fact that
$A\rightarrow R^{Sp(A)}$ is an isomorphism for $A$ f.p. over $R$.

\medskip

 This model is a  simplified version of the sheaf model considered in \cite{CRS21}. It contains however the expected notion
of descent data. The following Lemma illustrates what is going on. If $p$ is a proposition, a partial element of a type $T$ of
extent $p$ is an element of $T^p$.

\begin{lemma}
  Let $p_0,p_1,p_2$ be propositions. The type $T^{p_0\vee p_1\vee p_2}$ is equivalent to the type of tuples $u_0,u_1,u_2$
  where $u_i$ is a partial element of extent $p_i$ together with paths $u_{ij}:u_i =_T u_j$ of extent $p_i\wedge p_j$
  satisfying the cocycle condition $u_{01}\cdot u_{12} = u_{02}$ on the extend $p_0\wedge p_1\wedge p_2$.
\end{lemma}

We can generalize this as follows. If $c = f_1,\dots,f_n:Cov(A)$  and $T$ is a presheaf defined at level $A$, we define
$D_c(T)(A)$, a descent data for $T$ for the covering $c$, as the type of family $u_K(i):T[1/f_K]$ where $K$ is a nonempty
finite subset of $1,\dots,n$ and $i$ an element of $\II^K$ such that $\vee_p (i(p) = 1)$ and $u_K(i) = u_L(i_{|L})$
if $K = L,p$ and $i(p) = 0$. We can then check that $D_c(T)$ is equivalent to $T^{inv(f_1)\vee\dots\vee inv(f_n)}$.
So $T$ is a sheaf iff the canonical map $T\rightarrow D_c(T)$ is an equivalence.

\medskip

If $T$ is a hset ($0$-type), we recover the usual patching condition: if we have $u_i:T[1/f_i]$ with an equality $u_i = u_j$ on
$T[1/f_if_j]$ we can find $u$ in $T(A)$ such that $u = u_i$ on $T[1/f_i]$.

If $T$ is a $1$-type, we recover the usual patching condition: if we have $u_i:T[1/f_i]$ with an equality $e_{ij}:u_i = u_j$ on
$T[1/f_if_j]$, with the cocycle condition $e_{ij}\cdot e_{jk} = e_{ik}$,
we can find $u$ in $T(A)$ with $e_i:u = u_i$ on $T[1/f_i]$ such that $e_{ij}\cdot e_j = e_i$.

%% This follows from the following Lemma, which holds in any effective model of type theory.

%% \begin{lemma}
%%   Let $p_i,~i<n$ be propositions. The type $T^{\vee p_i}$ is equivalent to the type of descent data $u_{l_0\dots l_m}(i_0,\dots,i_m)$
%%   of extend $p_{l_0}\wedge\dots\wedge p_{l_m}$
%%   with $l_0<\dots<l_m$ satisfying the compatibility conditions of \cite{CRS21}.
%% \end{lemma}

%% For $n = 3$ we give $u_{021}(i_0,i_1,i_2)$ with $u_{012}(0,i_1,i_2) = u_{12}(i_1,i_2),~u_{012}(i_0,0,i_2) = u_{02}(i_0,i_2)$
%% and $u_{012}(i_0,i_1,0) = u_{01}(i_0,i_1)$ and $u_{pq}(1,0) = u_p(1)$ and $u_{pq}(0,1) = u_q(1)$. This is one way to express
%% the cocycle condition.


\subsection{Global sections and Zariski global choice}

If $T$ is a sheaf defined at level $A$, we let $\Box_A T$ the type of global sections.
If $c = f_1,\dots,f_n$ is in $Cov(A)$ we let $\Box_c T$ be the type $\Box_{A[1/f_1]}T\times\dots\times\Box_{A[1/f_n]}T$.

Using these notations, we can state the principle of Zariski global choice
$$
(\Box \norm{T})\leftrightarrow \norm{\Sigma_{c:Cov(k)}\Box_c T}
$$

This is valid in the present model.

Using this principle, we can show that $\Box K(\Gm,1)$ is equal to the type of projective modules of rank $1$ over $k$
and that each $\Box K(R,n)$ for $n>0$ is contractible.
                                                                                  




 


\appendix

\section{Negative results}
Here we collect some results of
the theory developed from the axioms
(\axiomref{loc}), (\axiomref{sqc}) and (\axiomref{Z-choice})
that are of a negative nature
and primarily serve the purpose of counterexamples.

We adopt the following definition from
\cite[Section IV.8]{lombardi-quitte}.

\begin{definition}%
  \label{zero-dimensional-ring}
  A ring $A$ is \notion{zero-dimensional}
  if for all $x : A$
  there exists $a : A$ and $k : \N$
  such that $x^k = a x^{k + 1}$.
\end{definition}

\begin{lemma}[using \axiomref{loc}, \axiomref{sqc}, \axiomref{Z-choice}]%
  \label{R-not-zero-dimensional}
  The ring $R$ is not zero-dimensional.
\end{lemma}

\begin{proof}
  Assume that $R$ is zero-dimensional,
  so for every $f : R$ there merely is some $k : \N$ with $f^k \in (f^{k + 1})$.
  We note that $R = \A^1$ is an affine scheme and
  that if $f^k \in (f^{k + 1})$,
  then we also have $f^{k'} \in (f^{k' + 1})$ for every $k' \geq k$.
  This means that we can apply \cref{strengthened-boundedness}
  and merely obtain a number $K : \N$
  such that $f^K \in (f^{K + 1})$ for all $f : R$.
  In particular, $f^{K + 1} = 0$ implies $f^K = 0$,
  so the canonical map
  $\Spec R[X]/(X^K) \to \Spec R[X]/(X^{K + 1})$
  is a bijection.
  But this is a contradiction,
  since the homomorphism $R[X]/(X^{K + 1}) \to R[X]/(X^K)$
  is not an isomorphism.
\end{proof}

\begin{example}[using \axiomref{loc}, \axiomref{sqc}, \axiomref{Z-choice}]%
  \label{non-existence-of-roots}
  It is not the case that
  every monic polynomial $f : R[X]$ with $\deg f \geq 1$ has a root.
  More specifically,
  if $U \subseteq \A^1$ is an open subset
  with the property that
  the polynomial $X^2 - a : R[X]$ merely has a root
  for every $a : U$,
  then $U = \emptyset$.
\end{example}

\begin{proof}
  Let $U \subseteq \A^1$ be as in the statement.
  Since we want to show $U = \emptyset$,
  we can assume a given element $a_0 : U$
  and now have to derive a contradiction.
  By \axiomref{Z-choice},
  there exists in particular a standard open $D(f) \subseteq \A^1$
  with $a_0 \in D(f)$
  and a function $g : D(f) \to R$
  such that ${(g(x))}^2 = x$ for all $x : D(f)$.
  By \axiomref{sqc},
  this corresponds to an element $\frac{p}{f^n} : R[X]_f$
  with ${(\frac{p}{f^n})}^2 = X : R[X]_f$.
  We use \cref{polynomial-with-regular-value-is-regular}
  together with the fact that $f(a_0)$ is invertible
  to get that $f : R[X]$ is regular,
  and therefore $p^2 = f^{2n}X : R[X]$.
  Considering this equation over $R^{\mathrm{red}} = R/\sqrt{(0)}$ instead,
  we can show by induction that all coefficients of $p$ and of $f^n$ are nilpotent,
  which contradicts the invertibility of $f(a_0)$.
\end{proof}

\begin{remark}
  \Cref{non-existence-of-roots} shows that
  the axioms we are using here
  are incompatible with a natural axiom that is true
  for the structure sheaf of the big étale topos,
  namely that $R$ admits roots for unramifiable monic polynomials.
  The polynomial $X^2 - a$ is even separable for invertible $a$,
  assuming that $2$ is invertible in $R$.
  To get rid of this last assumption,
  we can use the fact that either $2$ or $3$ is invertible in the local ring $R$
  and observe that the proof of \cref{non-existence-of-roots}
  works just the same for $X^3 - a$.
\end{remark}

\begin{proposition}[using ???]%
  \label{RN-non-wqc}
  The $R$-module $R^\N$ is not weakly quasi-coherent.
\end{proposition}

\begin{proof}
  TODO
\end{proof}

\begin{proposition}[using \axiomref{loc}, \axiomref{sqc}, \axiomref{Z-choice}]%
  \label{non-wqc-module-family}
  Not every $R$-module is weakly quasi-coherent
  in the sense of \cref{weakly-quasi-coherent-module}.
  {\color{red} NOTE: Does this add any value compared to \cref{RN-non-wqc}?
  It only uses injectivity of $M_f \to M^{D(f)}$?}
\end{proposition}

\begin{proof}
  We construct a family of $R$-modules,
  parametrized by the elements of $R$,
  and deduce a contradiction from the assumption that
  all modules of this family are weakly quasi-coherent.

  Given an element $f : R$,
  the $R$-module we want to consider is
  the countable product
  \[ M(f) \colonequiv \prod_{n : \N} R/(f^n) \rlap{.} \]
  If $f \neq 0$ then $M(f) = 0$
  (using \cref{non-zero-invertible}).
  This implies that the $R$-module $M(f)^{f \neq 0}$
  is trivial:
  any function $f \neq 0 \to M(f)$ can only assign the value $0$
  to any of the at most one witnesses of $f \neq 0$.
  If $M(f)$ is weakly quasi-coherent,
  then this means that $M(f)_f$ is also trivial.
  Noting that
  $M(f)$ is not only an $R$-module
  but even an $R$-algebra in a natural way,
  we have
  \begin{align*}
    M(f)_f = 0
    &\;\Leftrightarrow\;
    \exists k : \N.\; \text{$f^k = 0$ in $M(f)$} \\
    &\;\Leftrightarrow\;
    \exists k : \N.\; \forall n : \N.\; f^k \in (f^n) \subseteq R \\
    &\;\Leftrightarrow\;
    \exists k : \N.\; f^k \in (f^{k + 1}) \subseteq R
    \rlap{.}
  \end{align*}

  In summary,
  if the module $M(f)$ is weakly quasi-coherent
  for every $f : R$,
  then the ring $R$ is zero-dimensional
  in the sense of \cref{zero-dimensional-ring}.
  But this is not the case,
  as we saw in \cref{R-not-zero-dimensional}.
\end{proof}

\begin{example}[using \axiomref{loc}, \axiomref{sqc}]
  It is not the case that
  for any pair of lines $L, L' \subseteq \bP^2$,
  the $R$-algebra $R^{L \cap L'}$ is free of rank $1$.
\end{example}

\begin{proof}
  The $R$-algebra $R^{L \cap L'}$ is free of rank $1$
  if and only if the structure homomorphism
  $\varphi : R \to R^{L \cap L'}$ is bijective.
  We will show that it is not even always injective.

  Consider the lines
  \[ L = \{\, [x : y : z] : \bP^2 \mid z = 0 \,\} \]
  and
  \[ L' = \{\, [x : y : z] : \bP^2 \mid \varepsilon x + \delta y + z = 0 \,\}
     \rlap{,} \]
  where $\varepsilon$ and $\delta$ are elements of $R$
  with $\varepsilon^2 = \delta^2 = 0$.
  Consider the element $\varphi(\epsilon \delta) : R^{L \cap L'}$,
  which is the constant function $L \cap L' \to R$
  with value $\varepsilon \delta$.
  For any point $[x : y : z] : L \cap L'$,
  we have $z = 0$ and $\varepsilon x + \delta y = 0$.
  But also, by definition of $\bP^3$,
  we have $(x, y, z) \neq 0 : R^3$,
  so one of $x, y$ must be invertible.
  This implies $\delta \divides \varepsilon$ or $\varepsilon \divides \delta$,
  and in both cases we can conclude $\varepsilon \delta = 0$.
  Thus, $\varphi(\epsilon \delta) = 0 : R^{L \cap L'}$.

  If $\varphi$ was always injective
  then this would imply $\varepsilon \delta = 0$
  for any $\varepsilon, \delta : R$
  with $\varepsilon^2 = \delta^2 = 0$.
  In other words, the inclusion
  \[ \Spec R[X, Y]/(X^2, Y^2, XY) \hookrightarrow \Spec R[X, Y]/(X^2, Y^2) \]
  would be a bijection.
  But the corresponding $R$-algebra homomorphism is not an isomorphism.
\end{proof}


\printindex

\printbibliography

\end{document}
