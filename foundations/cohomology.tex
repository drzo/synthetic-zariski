In non-synthetic algebraic geometry,
the structure sheaf~$\mathcal{O}_X$ is part of the data constituting a scheme~$X$.
In our internal setting,
the scheme $X$ is just a set without any additional data,
but when we want to consider the structure sheaf as an object in its own right,
then we can represent it by the trivial bundle
that assings to every point $x : X$ the set $R$.
Indeed, for an affine scheme $X = \Spec A$,
taking the sections of this bundle over a basic open $D(f) \subseteq X$
\[ (\prod_{x : D(f)} R) = (D(f) \to R) = A[f^{-1}] \]
yields the localizations of the ring $A$
expected from the structure sheaf $\mathcal{O}_X$.
More generally,
instead of sheaves of abelian groups, $\mathcal{O}_X$-modules, etc.,
we will consider bundels of abelian groups, $R$-modules, etc.,
in the form of maps from $X$ to the respective type of algebraic structures.

\subsection{Quasi-coherent bundles}

Sometimes we want to ``apply'' a bundle to a subtype,
like sheaves can be evaluated on open subspaces
and introduce the common notation ``$M(U)$'' for that below.
It is, however, not justified to expect, that this application
and the corresponding theory of ``sheaves'' is ``the same'' as the external one,
since the definition below uses the internal hom ``$\prod$''
-- where the corresponding external construction, would be the set of continuous sections of a bundle.

\begin{definition}
  \index{$M(U)$}
  Let $X$ be a type and $M:X\to \Mod{R}$ a dependent module.
  Let $U\subseteq X$ be any subtype.
  \begin{enumerate}[(a)]
  \item We write:
    \[
      M(U)\colonequiv \prod_{x:U}M_x
      \rlap{.}
    \]
  \item With pointwise structure, $U\to R$ is an $R$-algebra
    and $M(U)$ is a $(U\to R)$-module.
  \end{enumerate}
\end{definition}

Somewhat surprisingly, localization of modules $M(U)$
can be done pointwise:

\begin{lemma}[using \axiomref{loc}, \axiomref{sqc}, \axiomref{Z-choice}]%
  \label{module-bundle-localization-pointwise}
  Let $X$ be a scheme and $M:X\to \Mod{R}$ a dependent module.
  For any $f:X\to R$, there is an equality
  \[
    M(X)_f=\prod_{x:X}(M_x)_{f(x)}
  \]
  of $(X\to R)$-modules.
\end{lemma}

\begin{proof}
First we construct a map, by realizing that the following is well-defined:
\[
  \frac{m}{f^k}\mapsto\left(x\mapsto \frac{m(x)}{f(x)^k}\right)
\]
So let $\frac{m}{f^k}=\frac{m'}{f^{k'}}$,
i.e. let there be an $l:\N$ such that $f^l(mf^{k'}-m'f^k)=0$.
But then we can choose the same $l:\N$ for each $x:X$
and apply the equation to each $x:X$.

We will now show, that the map we defined is an embedding.
So let $g,h:M(X)_f$ such that $p:\prod_{x:X}g(x)=_{(M_x)_{f(x)}}h(x)$.
Let $m_g,m_h:\prod_{x:X} M_x$ and $k_g,k_h:\N$ such that
\[
  g=\frac{m_g}{f^{k_g}} \quad\text{and}\quad h=\frac{m_h}{f^{k_h}}
  \rlap{.}
\]
From $p$ we know $\prod_{x:X}\exists_{k_x:\N}f(x)^{k_x}(m_g(x)f(x)^{k_h}-m_h(x)f(x)^{k_g})=0$.
By \cref{strengthened-boundedness},
we find one $k : \N$ with
\[
  \prod_{x:X}f(x)^{k}(m_g(x)f(x)^{k_h}-m_h(x)f(x)^{k_g})=0
\]
--- which shows $g=h$.

It remains to show that the map is surjective.
So let $\varphi:\prod_{x:X}(M_x)_{f(x)}$ and
note that
\[
  \prod_{x:X}
  \exists_{k_x:\N,m_x:M_x}.
  \varphi(x)=\frac{m_x}{f(x)^{k_x}}
  \rlap{.}
\]
By \cref{strengthened-boundedness} and \cref{zariski-choice-scheme},
we get $k:\N$, an affine open cover $U_1,\dots,U_n$ of $X$ and $m_i:(x : U_i)\to M_x$
such that for each $i$ and $x:U_i$ we have
\[
  \varphi(x)=\frac{m_i(x)}{f(x)^{k}}
  \rlap{.}
\]
The problem is now to construct a global $m:(x:X)\to M_x$ from the $m_i$.
We have
\[
    \prod_{x:U_{ij}}\frac{m_i(x)}{f(x)^k}=\varphi(x)=\frac{m_j(x)}{f(x)^k}
\]
meaning there is pointwise an exponent $t_x:\N$,
such that $f(x)^{t_x}m_i(x)=f(x)^{t_x}m_j(x)$.
By \cref{strengthened-boundedness},
we can find a single $t:\N$ with this property and define
\[
  \tilde{m}_i(x) \colonequiv f(x)^t m_i(x)
  \rlap{.}
\]
Then we have $\tilde{m}_i(x)=\tilde{m}_j(x)$ on all intersections $U_{ij}$,
which is what we need to get a global $m:(x:X)\to M_x$ from \cref{kraus-glueing}.
Since $\varphi(x)=\frac{f(x)^t m_i(x)}{f(x)^{t+k}}=\frac{\tilde{m}_i(x)}{f(x)^{t+k}}$
for all $i$ and $x : U_i$,
we have found a preimage of $\varphi$ in $M(X)_f$.
\end{proof}

We will need the following algebraic observation:

\begin{remark}%
  \label{localization-to-module-if-non-zero}
  Let $M$ be an $R$-module and $A$ a finitely presented $R$-algebra,
  then there is an $R$-linear map
  \[
    M\otimes A\to M^{\Spec A}
  \]
  induced by mapping $m\otimes f$ to $x\mapsto x(f)\cdot m$.
  In particular, for any $f:R$, there is a
  \[
    M_f\to M^{D(f)}
    \rlap{.}
  \]
  The map $M\otimes A\to M^{\Spec A}$ is natural in $M$.
\end{remark}

\begin{lemma}[using \axiomref{sqc}, \axiomref{loc}, \axiomref{Z-choice}]%
  \label{localization-to-restriction}                    
  Let $X$ be a scheme, $M:X\to\Mod{R}$, $U\subseteq X$ open and $f:A$.
  Then there is an $R$-linear map
  \[
    M(U)_f \to M(D(f)) 
    \rlap{.}
  \]
\end{lemma}

\begin{proof}
  Combining \cref{module-bundle-localization-pointwise}
  and pointwise application of \cref{localization-to-module-if-non-zero} we get
  \[
    M(U)_f=\left(\prod_{x:U}(M_x)_{f(x)}\right)\to \left(\prod_{x:U}(M_x)^{D(f(x))}\right)
    =\left(\prod_{x:D(f)}M_x\right)
    =M(D(f))
  \]
\end{proof}

The following definition was used in \cite{ingo-thesis},
to characterize quasi coherent sheaves in the little Zariski-topos.
We will use it to get a good subcategory of the $R$-module bundles over a scheme.

\begin{definition}
  \label{weakly-quasi-coherent-module}
  An $R$-module $M$ is \notion{weakly quasi-coherent},
  if for all $f:R$, the canonical homomorphism
  \[
    M_f\to M^{D(f)}
  \]
  from \cref{localization-to-module-if-non-zero} is an equivalence.
  We denote the type of weakly quasi-coherent $R$-modules
  with $\Mod{R}_{wqc}$\index{$\Mod{R}_{wqc}$}.
\end{definition}

\begin{lemma}
  \label{kernel-wqc}
  For any $R$-linear map $f:M\to N$ of weakly quasi-coherent modules $M$ and $N$,
  the kernel of $f$ is weakly quasi-coherent.
\end{lemma}

\begin{proof}
  Let $K\to M$ be the kernel of $f$.
  For any $f:R$, the map $K^{D(f)}\to M^{D(f)}$ is the kernel of $M^{D(f)}\to N^{D(f)}$.
  The latter map is equal to $M_f\to N_f$ by weak quasi-coherence of $M$ and $N$
  and $K_f\to M_f$ is the kernel of $M_f\to N_f$.
  Let the vertical maps in
  \begin{center}
    \begin{tikzcd}
      K_f\ar[r]\ar[d] & M_f\ar[r]\ar[d,"\simeq"] & N_f\ar[d,"\simeq"] \\
      K^{D(f)}\ar[r] & M^{D(f)}\ar[r] & N^{D(f)}
    \end{tikzcd}
  \end{center}
  be the canonical maps from \cref{localization-to-module-if-non-zero}.
  The squares commute because of the naturality of the vertical maps.
  Then the map $K_f\to K^{D(f)}$ is an isomorphism,
  because by commutativity, it is equal to the induced map between the kernels $K_f$ and $K^{D(f)}$,
  which has to be an isomorphism, since it is induced by an isomorphism of diagrams.
\end{proof}

\begin{definition}%
  \label{weakly-quasi-coherent-bundle}
  Let $X$ be a scheme.
  A weakly quasi-coherent bundle on $X$, is a map $M:X\to \Mod{R}_{wqc}$.
\end{definition}

An immediate consequence is, that
weakly quasi coherent dependent modules have
the property that ``restricting is the same as localizing'':

\begin{lemma}[using \axiomref{sqc}, \axiomref{loc}, \axiomref{Z-choice}]
  \label{weakly-quasi-coherent-open-localization}
  Let $X$ be a scheme and $M:X\to \Mod{R}$ weakly quasi-coherent,
  then for all open $U\subseteq X$ and $f:U\to R$
  the canonical morphism
  \[
    M(U)_f\to M(D(f))
  \]
  is an equivalence.
\end{lemma}

\begin{proof}
  By construction of the canonical map from \cref{localization-to-restriction}.
\end{proof}

Let us look at an example.

\begin{proposition}%
  \label{fp-algebra-bundle-is-quasi-coherent}
  Let $X$ be a scheme and $C:X\to \Alg{R}_{fp}$.
  Then $C$, as a bundle of $R$-modules, is weakly quasi coherent.
\end{proposition}

\begin{proof}
  Then for any $f:R$ and $x:X$, using \cref{algebra-valued-functions-on-affine}, we have
  \[
    (C_x)_f=C_x\otimes_R R_f=(\Spec R_f \to C_x)=(D(f)\to C_x)={C_x}^{D(f)}
    \rlap{.}
  \]
\end{proof}

\begin{proposition}[using \axiomref{loc}, \axiomref{sqc}, \axiomref{Z-choice}]%
  Not every $R$-module is weakly quasi-coherent
  in the sense of \cref{weakly-quasi-coherent-module}.
\end{proposition}

\begin{proof}
  We construct a family of $R$-modules,
  parametrized by the elements of $R$,
  and deduce a contradiction from the assumption that
  all modules of this family are quasi-coherent.

  Given an element $f : R$,
  the $R$-module we want to consider is
  the countable product
  \[ M(f) \colonequiv \prod_{n : \N} R/(f^n) \rlap{.} \]
  If $f \neq 0$ then $M(f) = 0$
  (using \cref{non-zero-invertible}).
  This implies that the $R$-module $M(f)^{f \neq 0}$
  is trivial:
  any function $f \neq 0 \to M(f)$ can only assign the value $0$
  to any of the at most one witnesses of $f \neq 0$.
  If $M(f)$ is quasi-coherent,
  then this means that $M(f)_f$ is also trivial.
  Noting that
  $M(f)$ is not only an $R$-module
  but even an $R$-algebra in a natural way,
  we have
  \begin{align*}
    M(f)_f = 0
    &\;\Leftrightarrow\;
    \exists k : \N.\; \text{$f^k = 0$ in $M(f)$} \\
    &\;\Leftrightarrow\;
    \exists k : \N.\; \forall n : \N.\; f^k \in (f^n) \subseteq R \\
    &\;\Leftrightarrow\;
    \exists k : \N.\; f^k \in (f^{k + 1}) \subseteq R
    \rlap{.}
  \end{align*}

  In summary,
  if the module $M(f)$ is quasi-coherent
  for every $f : R$,
  then the ring $R$ is zero-dimensional
  in the sense of \cref{zero-dimensional-ring}.
  But this is not the case,
  as we saw in \cref{R-not-zero-dimensional}.
\end{proof}

TODO: $R^\N$ is also not weakly quasi-coherent.

\begin{lemma}[using \axiomref{sqc}, \axiomref{loc}, \axiomref{Z-choice}]%
  \label{weakly-quasi-coherent-pi}
  Let $X$ be an affine scheme and $M_x$ a weakly quasi-coherent $R$-module for any $x:X$,
  then
  \[
    \prod_{x:X}M_x
  \]
  is a weakly quasi-coherent $R$-module.
\end{lemma}

\begin{proof}
  We need to show:
  \[
    \left(\prod_{x:X}M_x\right)_f=\left(\prod_{x:X}M_x\right)^{D(f)}
  \]
  for all $f:R$.
  By weak \cref{module-bundle-localization-pointwise}, quasi-coherence
  and \cref{weakly-quasi-coherent-open-localization}
  we know:
  \[
    \left(\prod_{x:X}M_x\right)_f
    =\prod_{x:X}\left(M_x\right)_{f(x)}
    =\prod_{x:X}\left(M_x\right)^{D(f)}
    =\left(\prod_{x:X}M_x\right)^{D(f)}
    \rlap{.}
  \]
\end{proof}

Quasi-coherent dependent modules turn out to have very good properties,
which are to be expected from what is known about their external counterparts.
We will show below, that quasi coherence is preserved by the following constructions:

\begin{definition}
  \label{pullback-push-forward}
  Let $X,Y$ be types and $f:X\to Y$ be a map.
  \begin{enumerate}[(a)]
  \item \index{$f^*M$} For any dependent module $N:Y\to\Mod{R}$,
    the \notion{pullback} or \notion{inverse image} is the dependent module
    \[
      f^*N\colonequiv (x:X) \mapsto M_{f(x)}\rlap{.}
    \]
  \item \index{$f_*M$} For any dependent module $M:X\to\Mod{R}$,
    the \notion{push-forward} or \notion{direct image} is the dependent module
    \[
      f_*M\colonequiv (y:Y) \mapsto \prod_{x:\fib_f(y)}M_{\pi_1(x)}\rlap{.}
    \]
  \end{enumerate}
\end{definition}

\begin{theorem}[using \axiomref{sqc}, \axiomref{loc}, \axiomref{Z-choice}]%
  \label{pullback-push-forward-qcoh}
  Let $X,Y$ be schemes and $f:X\to Y$ be a map.
  \begin{enumerate}[(a)]
  \item For any weakly quasi-coherent dependent module $N:Y\to\Mod{R}$,
    the inverse image $f^*N$ is weakly quasi-coherent.
  \item For any weakly quasi-coherent dependent module $M:X\to\Mod{R}$,
    the direct image $f_*M$ is weakly quasi-coherent.
  \end{enumerate}
\end{theorem}

\begin{proof}
  \begin{enumerate}[(a)]
  \item There is nothing to do, when we use the pointwise definition of weak quasi-coherence. 
  \item We need to show, that
    \[
      \prod_{x:\fib_f(y)}M_{\pi_1(x)}
    \]
    is a weakly quasi-coherent $R$-module.
    By \cref{fiber-product-scheme},
    the type $\fib_f(y)$ is a scheme.
    So by \cref{weakly-quasi-coherent-pi},
    the module in question is weakly quasi-coherent.
  \end{enumerate}
\end{proof}

With a non-cyclic forward reference to a cohomological result,
there is a short proof of the following:

\begin{proposition}[using \axiomref{sqc}, \axiomref{loc}, \axiomref{Z-choice}]%
  Let $f:M\to N$ be an $R$-linear map of weakly quasi-coherent $R$-modules $M$ and $N$,
  then the cokernel $N/M$ is weakly quasi-coherent.
\end{proposition}

\begin{proof}
  We will first show, that for an $R$-linear embedding $m:M\to N$
  of weakly quasi-coherent $R$-modules $M$ and $N$,
  the cokernel $N/M$ is weakly quasi-coherent.
  We need to show:
  \[
    (N/M)_f=(N/M)^{D(f)}.
  \]
  By algebra: $(N/M)_f=N_f/M_f$.
  This means we are done, if $(N/M)^{D(f)}=N^{D(f)}/{M^{D(f)}}$.
  To see this holds, let us consider $0\to M\to N\to N/M\to 0$ as a short exact sequence of dependent modules,
  over the subtype of the point $D(f)\subseteq 1=\Spec R$.
  Then, taking global sections, by \cref{cohomology-les},
  we have an exact sequence
  \[
    0\to M^{D(f)}\to N^{D(f)}\to (N/M)^{D(f)}\to H^1(D(f),M)
  \]
  -- but $D(f)=\Spec R_f$ is affine,
  so the last term is 0 by \cref{H1-wqc-module-affine-trivial}
  and $(N/M)^{D(f)}$ is the cokernel $N^{D(f)}/M^{D(f)}$.

  Now we will show the statement for a general $R$-linear map $f:M\to N$.
  By algebra, the cokernel of $f$ is the same as the cokernel of the induced map
  $M/K\to N$, where $K$ is the kernel of $f$.
  By \cref{kernel-wqc}, $K$ is weakly quasi-coherent, so by the proof above,
  $M/K$ is weakly quasi-coherent.
  $M/K\to N$ is an embedding, so again by the proof above, its cokernel is weakly quasi-coherent.
\end{proof}

\subsection{Finitely presented bundles}

We now investigate the relationship between bundles of $R$-modules on $X = \Spec A$
and $A$-modules.

\begin{proposition}
  Let $A$ be a finitely presented $R$-algebra.
  There is an adjunction
  \[ \begin{tikzcd}[row sep=tiny]
    M \ar[r, mapsto] & {(M \otimes x)}_{x : \Spec A} \\
    \Mod{A} \ar[r, shift left=2] \ar[r, phantom, "\rotatebox{90}{$\vdash$}"] &
    \Mod{R}^{\Spec A} \ar[l, shift left=2] \\
    \prod_{x : \Spec A} N_x & N \ar[l, mapsto]
  \end{tikzcd} \]
  between the category of $A$-modules
  and the category of bundles of $R$-modules on $\Spec A$.
\end{proposition}

For an $A$-module $M$,
the unit of the adjunction is:
\begin{align*}
  \eta_M : M &\to \prod_{x : \Spec A} (M \otimes x) \\
  m &\mapsto (m \otimes 1)_{x : \Spec A}
\end{align*}

\begin{example}[using \axiomref{sqc}, \axiomref{loc}]
  It is not the case that
  for every finitely presented $R$-algebra $A$
  and every $A$-module $M$
  the map $\eta_M$ is injective.
\end{example}

\begin{proof}
  Instead of giving a single counterexample,
  we construct a family of potential counterexamples,
  indexed by an element $f : R$.
  We set $A \colonequiv R/(f)$ and
  \[ M \colonequiv A^1 / \langle \{ 1 \mid f =_R 0 \} \rangle \rlap{.} \]
  Then we have $M \otimes x = 0$ for all $x : \Spec A$:
  an element $x : \Spec A$ is a witness that $f : R$ is invertible
  and if $f$ is invertible then $A = 0$, so $M = 0$, so $M \otimes x = 0$.
  This implies that
  if $\eta_M$ is injective
  then $M = 0$.
  But we have $M = 0$ if and only if
  $1_A \in \langle \{ 1 \mid f = 0 \} \rangle$
  if and only if
  $1_A$ a linear combination (of some length $n$)
  of elements of the set $\{ 1 \mid f = 0 \}$
  if and only if
  $f = 0$ ($n > 0$) or
  $1 =_A 0$, that is, $f$ is invertible ($n = 0$).
  In summary,
  if $\eta_M$ is injective for every choice of $f : R$,
  then every $f : R$ is zero or invertible.
  But this would be a contradiction to \cref{R-not-discrete}.
\end{proof}

\begin{theorem}%
  \label{fp-module}
  Let $X=\Spec(A)$ be affine and
  let a bundle of finitely presented $R$-modules $M : X\to \fpMod{R}$ be given.
  Then the $A$-module
  \[ \tilde{M}\coloneqq\prod_{x:X}M_x \]
  is finitely presented and for any $x:X$ the $R$-module $\tilde{M}\otimes_A R$ is $M_x$.
  Under this correspondence, localizing $\tilde{M}$ at $f:A$ corresponds to restricting $M$ to $D(f)$.
\end{theorem}

\subsection{Cohomology on affine schemes}

\begin{definition}%
  \label{torsor}
  Let $X$ be a type and $A:X\to \AbGroup$ a map to the type of abelian groups.
  For $x:X$ let $T_x$ be a set with an $A_x$ action.
  \begin{enumerate}[(a)]
  \item $T$ is an \notion{$A$-pseudotorsor}, if the action is free and transitive for all $x:X$.
  \item $T$ is an \notion{$A$-torsor}, if it is an $A$-pseudotorsor and
    \[ \prod_{x:X} \| T_x \| \rlap{.}\]
  \item We write $\Tors{A}(X)$ for the type of $A$-torsors on $X$.
  \end{enumerate}
\end{definition}

Torsors on a point are a concrete implementaion of first deloopings:

\begin{definition}
  \label{delooping}
  Let $n:\N$.
  A $n$-th \notion{delooping}\index{$K(A,n)$} of an abelian group $A$,
  is a pointed, $(n-1)$-connected, $n$-truncated type $K(A,n)$,
  such that $\Omega^nK(A,n)=_{\AbGroup}A$.
\end{definition}

For any abelian group and any $n$, a delooping $K(A,n)$ exists by \cite{licata-finster}.
Deloopings can be used to represent cohomology groups by mapping spaces.
This is usually done in homotopy type theory to study higher inductive types, such as spheres and CW-complexes,
but the same approach works for internally representing sheaf cohomology,
which is the intent of the following definition:

\begin{definition}
  \label{cohomology}
  Let $X$ be a type and $\mathcal F:X\to\AbGroup$ a dependent abelian group.
  The $k$-th cohomology group of $X$ with coefficients in $\mathcal F$ is
  \[
    H^k(X,\mathcal F)\colonequiv \left\|\prod_{x:X}K(\mathcal F,k)\right\|_0\rlap{.}
  \]
\end{definition}

\begin{theorem}%
  \label{cohomology-les}
  Let $\mathcal F,\mathcal G,\mathcal H:X\to \AbGroup$ be such that for all $x:X$,
  \[
    0\to \mathcal F_x\to\mathcal G_x\to\mathcal H_x\to 0
  \]
  is an exact sequence of abelian groups. Then there is a long exact sequence:
  \begin{center}
    \begin{tikzcd}
      & \dots\ar[r] & H^{k-1}(X,\mathcal H)\ar[dll] \\
      H^k(X,\mathcal F)\ar[r] & H^k(X,\mathcal G)\ar[r] & H^k(X,\mathcal H)\ar[dll] \\
      H^{k+1}(X,\mathcal F)\ar[r] & \dots &
    \end{tikzcd}
  \end{center}
\end{theorem}

\begin{proof}
  By applying the long exact homotopy fiber sequence.
\end{proof}

The following is an explicit formulation of the fact, that the Čech-Complex for an
$\mathcal{O}_X$-module sheaf on $X=\Spec(A)$ given by an $A$-module $M$ is exact in degree 1.
\begin{lemma}%
  \label{H1-algebra}
  Let $M$ be a module over a commutative ring $A$, $F_1,\dots,F_l$ a coprime system on $A$
  and for $i,j\in\{1,\dots,l\}$, let $s_{ij} : F_i^{-1} F_j^{-1} M$ such that:
  \[ s_{jk}-s_{ik}+s_{ij}=0 \rlap{.}\]
  Then there are $u_i:F_i^{-1}M$ such that $s_{ij}=u_j - u_i$.
\end{lemma}

\begin{proof}
  Let $s_{ij}=\frac{m_{ij}}{f_i f_j}$ with $m_{ij}:M$, $f_i:F_i$ and $f_j:F_j$ such that:
  \[ f_i\cdot m_{jk}-f_j\cdot m_{ik}+f_k\cdot m_{ij}=0 \rlap{.}\]
  Let $r_i$ such that $\sum r_i f_i =1$.
  Then for
  \[ u_i \coloneqq -\sum_{k=1}^l\frac{r_k}{f_i}m_{ik} \]
  we have:
  \begin{align*}
      u_j-u_i &= -\sum_{k=1}^l\frac{r_k}{f_j}m_{jk} + \sum_{k=1}^l\frac{r_k}{f_i}m_{ik} \\
              &= -\sum_{k=1}^l\frac{r_k}{f_j f_i}f_i m_{jk} + \sum_{k=1}^l\frac{r_k}{f_i f_j} f_j m_{ik} \\
              &= \sum_{k=1}^l\frac{r_k}{f_j f_i}(-f_i m_{jk} + f_j m_{ik}) \\
              &= \sum_{k=1}^l\frac{r_k}{f_j f_i}f_k m_{ij} \\
              &= \frac{m_{ij}}{f_i f_j}
  \end{align*}
  \ %
\end{proof}

\begin{theorem}[using \axiomref{Z-choice}, \axiomref{loc}, \axiomref{sqc}]%
  \label{H1-wqc-module-affine-trivial}
  For any affine scheme $X=\Spec(A)$ and coefficients $M: X\to \Mod{R}_{wqc}$, we have
  \[ H^1(X,M)=0 \rlap{.} \]
\end{theorem}

\begin{proof}
  We need to show, that any $M$-torsor $T$ on $X$ is merely equal to the trivial torsor $M$,
  or equivalently show the existence of a section of $T$.
  We have
  \[ \prod_{x:X}\| T_x \|\]
  and therefore, by (\axiomref{Z-choice}),
  there merely are $f_1,\dots,f_l:A$,
  such that the $U_i\coloneqq \Spec(A_{f_i})$ cover $X$ and
  there are local sections
  \[ s_i:\prod_{x:U_i}T_x\]
  of $T$. Our goal is to construct a matching family from the $s_i$.
  On intersections, let $t_{ij}\coloneqq s_i-s_j$ be the difference, so $t_{ij}:(x : U_i\cap U_j) \to M_x$.
  By \cref{weakly-quasi-coherent-open-localization} equivalently,
  we have $t_{ij}:M(U_{i}\cap U_j)_{f_i f_j}$.
  Since the $t_{ij}$ were defined as differences,
  the condition in \cref{H1-algebra} is satisfied and we get
  $u_i:M(U_i)_{f_i}$, such that $t_{ij}=u_i-u_j$.
  So we merely have a matching family $\tilde{s}_i\coloneqq s_i-u_i$ and therefore, using Lemma \ref{kraus-glueing} merely a section of $T$.
\end{proof}

A similar result is provable for $H^2(X,M)$ using the same approach.
There is an extension of this result to general $n$ in work in progress \cite{chech-draft}.

\subsection{Čech-Cohomology}

In this section, let $X$ be a type, $U_1,\dots,U_n\subseteq X$ open subtypes that cover $X$
and $\mathcal F:X\to \AbGroup$ a dependent abelian group on $X$.
We start by repeating the classical definition of \v{C}hech-Cohomology groups for a given cover.

\begin{definition}%
  \label{chech-complex}
  \begin{enumerate}[(a)]
  \item \index{$\mathcal F(U)$} For open $U\subseteq X$, we use the notation {\color{purple} DUPLICATION}
    \[
      \mathcal F(U)\colonequiv \prod_{x:U}\mathcal F_x\rlap{.}
    \]
  \item For $s:\mathcal F(U)$ and open $V\subseteq U$ we use the notation $s\colonequiv s_{|V} \colonequiv (x:V)\mapsto s_x$.
  \item \index{$U_{i_1\dots i_l}$}For a selection of indices $i_1,...,i_l:\{1,\dots,n\}$, we use the notation
    \[
      U_{i_1\dots i_l}\colonequiv U_{i_1}\cap\dots\cap U_{i_l}\rlap{.}
    \]
  \item For a list of indices $i_1,\dots,i_l$, let $i_1,\dots,\hat{i_t},\dots,i_l$ be the same list with the $t$-th element removed.
  \item For $k:\Z$, the $k$-th \notion{Čech-boundary operator}\index{$\partial^k$} is the homomorphism
    \[
      \partial^k:\bigoplus_{i_0,\dots,i_k}\mathcal F(U_{i_0\dots i_k})\to \bigoplus_{i_0,\dots,i_{k+1}}\mathcal F(U_{i_0\dots i_{k+1}})
    \]
    given by $\partial^k(s)\colonequiv (l_0,\dots,l_{k+1}) \mapsto \sum_{j=0}^k (-1)^j s_{l_0,\dots,\hat{l_j},\dots,l_k|U_{l_0,\dots,l_{k+1}}}$.
  \item The $k$-th \notion{Čech-Cohomology group} for the cover $U_1,\dots,U_n$ with coefficients in $\mathcal F$ is
    \[
      \check{H}^k(\{U\},\mathcal F)\colonequiv \ker\partial^{k} / \im(\partial^{k-1})\rlap{.}
    \]
  \end{enumerate}
\end{definition}

It is possible to construct a torsor from a \v{c}ech cocycle:

\begin{lemma}%
  \label{deligne-construction}
  Let $A$ be an abelian group and $L$ a set with decidable equality and $\|L\|$.
  Let us call $c:(i,j:L)\to A$ a $L$-cocycle, if $c_{ij}+c_{jk}=c_{ik}$ for all $i,j:L$.
  Then there is a bijection:
  \[
    \left((T:\text{$A$-torsor})\times T^L\right) \to \text{$L$-cocylces}
    \rlap{.}
  \]
\end{lemma}

\begin{proof}
  Let us first check, that the left side is a set.
  Let $(T,u),(T',u'):(T:\text{$A$-torsor})\times T^L$,
  then $(T,u)=(T',u')$ is equivalent to $(e:T\cong T')\times ((i:L)\to e(u_i)=e(u'_i))$.
  But two maps $e$ with this property are equal,
  since a map between torsors is determined by the image of a single element and $L$ is inhabited.
  
  Assume now $(T,u):(T:\text{$A$-torsor})\times T^L$ to construct the map.
  Then $c_{ij}\colonequiv u_i-u_j$ defines an $L$-cocycle,
  because
  \[
    u_i-u_j + u_j-u_k = u_i-u_k
    \rlap{.}
  \]
  This defines an embedding: Assume $(T,u)$ and $(T',u')$ define the same $L$-cocycle,
  then $u_i-u_j=u'_i-u'_j$ for all $i,j:L$.
  We want to show a proposition, so we can assume there is $i:L$ and use that to get a map $e:T\to T'$
  that sends $u_i$ to $u'_i$.
  But then we also have
  \[
    e(u_j)=e(u_j-u_i+u_i)=e(u'_j-u'_i+u_i)=u'_j-u'_i+e(u_i)=u'_j-u'_i+u'_i=u'_j
  \]
  for all $j:L$, wich means $(T,u)=(T',u')$.
    
  Now let $c$ be an $L$-cocylce.
  Following \cite{Deligne91}, we can define a preimage-candidate:
  \[
    T_c\colonequiv \{u:A^L\mid u_i-u_j=c_{ij}\}
    \rlap{.}
  \]
  $A$ acts on $T_c$ pontwise, since $(a+u_i)-(a+u_j)=u_i-u_j=c_{ij}$ for all $a:A$.
  
  To show that $T_c$ is inhabited,
  we may assume $i_0:L$.
  Then we define $u_i\colonequiv -c_{i_0i}$ to get $u_i-u_j=-c_{i_0i}+c_{i_0j}=c_{ij}$.

  Now $c$ is of type $(A^L)^L=A^{L\times L}$, so we have an element of the left hand side.
  Applying the map constructed above yields a cocycle
  \[
    \tilde{c}_{ij}=(k\mapsto c_{ki})-(k\mapsto c_{kj})=(k\mapsto c_{ki}-c_{kj})=(k\mapsto c_{kj}+c_{ji}-c_{kj})=(k\mapsto c_{ji})
  \]
  -- so $(T_c,c)$ is a preimage of $c_{ij}$.
\end{proof}

\begin{definition}
  The cover $U_1,\dots,U_n$ is called \notion{(r-)acyclic} for $\mathcal F$,
  if for all $k:\N$ and $i_0,\dots,i_k$, we have that the higher (non Čech) cohomology groups are trivial:
  \[
    \forall (r\geq)l>0. H^l(U_{i_0,\dots,i_k},\mathcal F)=0\rlap{.}
  \]
\end{definition}

\begin{example}
  If $X$ is a scheme, $U_1,\dots,U_n$ a cover by affine open subtypes
  and $\mathcal F$ pointwise a weakly quasi coherent $R$-module,
  then $U_1,\dots,U_n$ is 1-acyclic for $\mathcal F$ by \cref{H1-wqc-module-affine-trivial}.
\end{example}

\begin{theorem}[using \axiomref{Z-choice}]%
  If $U_1,\dots,U_n$ is a 1-acyclic cover for $\mathcal F$, then
  \[
    \check{H}^1(\{U\},\mathcal F)=H^1(X,\mathcal F)\rlap{.}
  \]
\end{theorem}

\begin{proof}
  Let $\pi$ be the projection map
  \[
    \pi :
    \left(
      \sum_{T:\Tors{\mathcal F}(X)}\prod_{i}\prod_{x:U_i}T_x
    \right)
    \to \Tors{\mathcal F}(X)\rlap{.}
  \]
  Let us abbreviate the left hand side with $T(\mathcal F,U)$.
  Since the cover is 1-acyclic, $\pi$ is surjective.
  With $L_x\colonequiv \sum_{i}U_i(x)$ and \cref{deligne-construction} we get:
  \begin{align*}
    T(\mathcal F,U)&=\prod_{x:X}\sum_{T_x:\Tors{\mathcal F_x}}T_x^{L_x} \\
                   &=\prod_{x:X}\text{$L_x$-cocycles}
                     \rlap{.}
  \end{align*}
  The latter is the type of \v{c}ech-1-cocycles (\cref{chech-complex} (e))
  and in total the equality is given by the isomorphism
  \[
    \iota \colonequiv
    (T,t) \mapsto (i,j\mapsto t_i - t_j) :
    T(\mathcal F,U)
    \to
    \ker(\partial^1)
    \subseteq
    \bigoplus_{i,j}\mathcal F(U_{ij})\rlap{.}
  \]

  Realizing, that $\im(\partial^0)$ corresponds to the subtype of $T(\mathcal F,U)$ of trivial torsors,
  we arrive at the following diagram:
  \begin{center}
    \begin{tikzcd}
      & \Tors{\mathcal F}(X)\ar[r,->>] & H^1(X,\mathcal F) \\
      \sum_{T:T(\mathcal F,U)}\|\pi_1(T)=\mathcal F\|\ar[r,hook] & T(\mathcal F,U)\ar[u,->>]\ar[d,equal] & \\
      \im{\partial^0}\ar[r,hook]\ar[u,equal] & \ker{\partial^1}\ar[r,->>] & \check{H}^1(\{U\},\mathcal F)
    \end{tikzcd}
  \end{center}
  The composed map $T(\mathcal F,U)\to H^1(X,\mathcal F)$ is a homomorphism
  and therefore by \cref{surjective-abgroup-hom-is-cokernel} a cokernel.
  So the two cohomology groups are equal, since they are cokernels of the same diagram.
\end{proof}

There is follow-up work in progress \cite{chech-draft},
that answers the question if \v{C}ech cohomology with respect to an acyclic cover agrees with our definition of cohomology in general positively.