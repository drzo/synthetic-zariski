
\subsection{Definition of schemes}

The following definition \emph{does not} define schemes in general,
but something which is expected to correspond to quasi-compact, quasi separated schemes,
locally of finite type externally.

\begin{definition}%
  \label{schemes}
  A type $X$ is a \notion{(qc-)scheme}\index{scheme}
  if there merely is a cover by finitely many open subtypes $U_i:X\to\Prop$,
  such that each of the $U_i$ is affine.
\end{definition}

\begin{definition}
  \label{type-of-schemes}
  We denote the \notion{type of schemes} with $\Sch$\index{$\Sch$}.
\end{definition}

Zariski-choice \axiomref{Z-choice} extends to schemes:

\begin{proposition}[using \axiomref{Z-choice}]%
  \label{zariski-choice-scheme}
  Let $X$ be a scheme and $P:X\to \Type$ with $\prod_{x:X}\propTrunc{P(x)}$,
  then there merely is a cover $U_i$ by standard opens of the affine parts of $X$,
  such that there are $s_i:\prod_{x:U_i}P(x)$ for all $i$.
\end{proposition}

\subsection{General Properties}

\begin{lemma}[using \axiomref{loc}, \axiomref{sqc}]%
  \label{intersection-of-all-opens}
  Let $X$ be a scheme and $x:X$, then for all $y:X$ the proposition
  \[ \prod_{U:X\to \Open}U(x)\to U(y) \]
  is equivalent to $\neg\neg (x=y)$.
\end{lemma}

\begin{proof}
  By \Cref{open-union-intersection},
  open proposition are always double-negation stable,
  which settles one implication.
  For the implication
  \[ \left(\prod_{U:X\to \Open}U(x)\to U(y)\right) \Rightarrow \neg\neg (x=y) \]
  we can assume that $x$ and $y$ are both inside an open affine $U$
  and use that the statement holds for affine schemes by \Cref{affine-intersection-of-all-opens}.
\end{proof}

\subsection{Glueing}

\begin{proposition}[using \axiomref{loc}, \axiomref{sqc}, \axiomref{Z-choice}]%
  Let $X,Y$ be schemes and $f:U\to X$, $g:U\to Y$ be embeddings with open images in $X$ and $Y$,
  then the pushout of $f$ and $g$ is a scheme.
\end{proposition}

\begin{proof}
  As is shown for example \href{https://github.com/agda/cubical/blob/310a0956bb45ea49a5f0aede0e10245292ae41e0/Cubical/HITs/Pushout/KrausVonRaumer.agda#L170-L176}{here},
  such a pushout is always 0-truncated.
  Let $U_1,\dots,U_n$ be a cover of $X$ and $V_1,\dots,V_m$ be a cover of $Y$.
  By \Cref{qc-open-trans}, $U_i\cap U$ is open in $Y$,
  so we can use (large) pushout-recursion to construct a subtype $\tilde{U_i}$,
  which is open in the pushout and restricts to $U_i$ on $X$ and $U_i\cap U$ on $Y$.
  Symmetrically we define $\tilde{V_i}$ and in total get an open finite cover of the pushout.
  The pieces of this new cover are equivalent to their counterparts in the covers of $X$ and $Y$,
  so they are affine as well.
\end{proof}

\subsection{Subschemes}

\begin{definition}
  Let $X$ be a scheme.
  A \notion{subscheme} of $X$ is a subtype $Y:X\to\Prop$,
  such that $\sum Y$ is a scheme.
\end{definition}

\begin{proposition}[using \axiomref{loc}, \axiomref{sqc}, \axiomref{Z-choice}]%
  \label{open-subscheme}
  Any open subtype of a scheme is a scheme.
\end{proposition}

\begin{proof}
  Using \Cref{qc-open-affine-open}.
\end{proof}

\begin{proposition}[using \axiomref{loc}, \axiomref{sqc}, \axiomref{Z-choice}]%
  \label{closed-subscheme}
  Any closed subtype $A:X\to \Prop$ of a scheme $X$ is a scheme.
\end{proposition}

\begin{proof}
  Any open subtype of $X$ is also open in $A$.
  So it is enough to show,
  that any affine open $U_i$ of $X$,
  has affine intersection with $A$.
  But $U_i\cap A$ is closed in $U_i$ and therefore affine by \Cref{closed-subtype-affine}.
\end{proof}

\subsection{Equality types}

\begin{lemma}%
  \label{affine-equality-closed}
  Let $X$ be an affine scheme and $x,y:X$,
  then $x=_Xy$ is an affine scheme
  and $((x,y):X\times X)\mapsto x=_Xy$
  is a closed subtype of $X\times X$.
\end{lemma}

\begin{proof}
  Any affine scheme is merely embedded into $\A^n$ for some $n:\N$.
  The proposition $x=y$ for elements $x,y:\A^n$ is equivalent to $x-y=0$,
  which is equivalent to all entries of this vector being zero.
  The latter is a closed proposition.
\end{proof}

\begin{proposition}[using \axiomref{loc}, \axiomref{sqc}, \axiomref{Z-choice}]%
  \label{equality-scheme}
  Let $X$ be a scheme.
  The equality type $x=_Xy$ is a scheme for all $x,y:X$.
\end{proposition}

\begin{proof}
  Let $x,y:X$ and
  $U\subseteq X$ be an affine open containing $x$.
  Then $U(y)\wedge x=y$ is equivalent to $x=y$, so it is enough to show that $U(y)\wedge x=y$ is a scheme.
  As a open subscheme of the point, $U(y)$ is a scheme and $(x:U(y))\mapsto x=y$ defines a closed subtype by \Cref{affine-equality-closed}.
  But this closed subtype is a scheme by \Cref{closed-subscheme}.
\end{proof}

\subsection{Dependent sums}

\begin{theorem}[using \axiomref{loc}, \axiomref{sqc}, \axiomref{Z-choice}]%
  \label{sigma-scheme}
  Let $X$ be a scheme and for any $x:X$, let $Y_x$ be a scheme.
  Then the dependent sum
  \[ \left((x:X)\times Y_x\right)\equiv \sum_{x:X}Y_x\]
  is a scheme.
\end{theorem}

\begin{proof}
  We start with an affine $X=\Spec A$ and $Y_x=\Spec B_x$.
  Locally on $U_i = D(f_i)$, for a Zariski-cover $f_1,\dots,f_l$ of $X$,
  we have $B_x=\Spec R[X_1,\dots,X_{n_i}]/(g_{i,x,1},\dots,g_{i,x,m_i})$
  with polynomials $g_{i,x,j}$.
  In other words, $B_x$ is the closed subtype of $\A^{n_i}$
  where the functions $g_{i,x,1},\dots,g_{i,x,m_i}$ vanish.
  By \Cref{affine-fiber-product}, the product
  \[ V_i\colonequiv U_i\times \A^{n_i}\]
  is affine.
  The type $(x:U_i)\times \Spec B_x\subseteq V_i$ is affine,
  since it is the zero set of the functions
  \[ ((x,y):V_i)\mapsto g_{i,x,j}(y) \]
  Furthermore, $W_i\colonequiv (x:U_i)\times \Spec B_x$
  is open in $(x:X)\times Y_x$,
  since $W_i(x)$ is equivalent to $U_i(\pi_1(x))$,
  which is an open proposition.

  This settles the affine case.
  We will now assume, that
  $X$ and all $Y_x$ are general schemes.
  We pass again to a cover of $X$ by affine open $U_1,\dots,U_n$.
  We can choose the latter cover,
  such that for each $i$ and $x:U_i$, the $Y_{\pi_1(x)}$
  are covered by $l_i$ many open affine pieces $V_{i,x,1},\dots,V_{i,x,l_i}$
  (by \Cref{boundedness}).
  Then $W_{i,j}\colonequiv(x:U_i)\times V_{i,x,j}$ is affine by what we established above.
  It is also open.
  To see this, let $(x,y):((x:X)\times Y_x)$.
  We want to show, that $(x,y)$ being in $W_{i,j}$ is an open proposition.
  We have to be a bit careful, since the open proposition
  $V_{i,x,j}$ is only defined, for $x:U_i$.
  So the proposition we are after is $(z:U_i(x,y))\times V_{i,z,j}(y)$.
  But this proposition is open by \Cref{qc-open-sigma-closed}.
\end{proof}

It can be shown, that if $X$ is affine and for $Y:X\to\Sch$,
$Y_x$ is affine for all $x:X$,
then $(x:X)\times Y_x$ is affine.
An easy proof using cohomology is \href{https://felix-cherubini.de/random.pdf}{here}. 

\begin{corollary}
  \label{scheme-map-classification}
  Let $X$ be a scheme.
  For any other scheme $Y$ and any map $f:Y\to X$,
  the fiber map
  $(x:X)\mapsto \fib_f(x)$
  has values in the type of schemes $\Sch$.
  Mapping maps of schemes to their fiber maps,
  is an equivalence of types
  \[ \left(\sum_{Y:\Sch}(Y\to X)\right)\simeq (X\to \Sch)\rlap{.}\]
\end{corollary}

\begin{proof}
  By univalence, there is an equivalence
  \[ \left(\sum_{Y:\Type}(Y\to X)\right)\simeq (X\to \Type)\rlap{.} \]
  From left to right, the equivalence is given by turning a $f:Y\to X$ into $x\mapsto \fib_f(x)$,
  from right to left is given by taking the dependent sum.
  So we just have to note, that both constructions preserve schemes.
  From left to right, this is \Cref{fiber-product-scheme}, from right to left,
  this is \Cref{sigma-scheme}.
\end{proof}

Subschemes are classified by propositional schemes:

\begin{corollary}
  Let $X$ be a scheme.
  $Y:X\to\Prop$ is a subscheme,
  if and only if $Y_x$ is a scheme for all $x:X$.
\end{corollary}

\begin{proof}
  Restriction of \Cref{scheme-map-classification}.
\end{proof}

We will conclude now,
that the pullback of a cospan of schemes is a scheme.

\begin{theorem}[using \axiomref{loc}, \axiomref{sqc}, \axiomref{Z-choice}]%
  \label{fiber-product-scheme}
  Let
  \[
    \begin{tikzcd}
      X\ar[r,"f"] & Z & Y\ar[l,swap,"g"]
    \end{tikzcd}
  \]
  be schemes, then the \notion{pullback} $X\times_Z Y$ is also a scheme.
\end{theorem}

\begin{proof}
  The type $X\times_Z Y$ is given as the following iterated dependent sum:
  \[ \sum_{x:X}\sum_{y:Y}f(x)=g(y)\rlap{.}\]
  The innermost type, $f(x)=g(y)$
  is the equality type in the scheme $Z$ and by \Cref{equality-scheme} a scheme.
  By applying \Cref{sigma-scheme} twice, we prove that the iterated dependent sum is a scheme.
\end{proof}
