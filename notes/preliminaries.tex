\subsection{Homotopy type theory}

\begin{lemma}%
  \label{kraus-glueing}
  Let $X$ and $I$ be types.
  For $(U_i:X \to \Prop)_{i:I}$ and $P:U_i\to \nType{0}$, we have the following glueing property: \\
  If for each $i:I$ there is a dependent function $s_i:(x:U_i)\to P(x)$ together with
  proofs of equality on intersections $p_{ij}:(x:U_i\cap U_j)\to (s_i(x)=s_j(y))$,
  then there is a globally defined dependent function $s:(x:X) \to P(x)$,
  such that for all $x:X$ and $i:I$ we have $U_i(x) \to s(x)=s_i(x)$
\end{lemma}

\begin{proof}
  We define $s$ pointwise.
  Let $x:X$.
  Using a Lemma of Kraus and the $p_{ij}$, we get a factorization
  \[ \begin{tikzcd}[row sep=0mm]
    \sum_{i:I} U_i(x) \ar[rr, "s_{\pi_1(\_)}(x)"]\ar[rd] & & P(x) \\
    & \|\sum_{i:I} U_i(x)\|_{-1}\ar[ru,dashed] &
  \end{tikzcd} \]
-- which defines a unique value $s(x):P(x)$.
\end{proof}

Similarly we can prove.

\begin{lemma}%
  \label{kraus-glueing-1-type}
  Let $X$ and $I$ be types.
  For $(U_i:X \to \Prop)_{i:I}$ and $P:U_i\to \nType{1}$, we have the following glueing property: \\
  If for each $i:I$ there is a dependent function $s_i:(x:U_i)\to P(x)$ together with
  proofs of equality on intersections $p_{ij}:(x:U_i\cap U_j)\to (s_i(x)=s_j(y))$ satsfying the cocycle
  condition $p_{ij}\cdot p_{jk} = p_{ik}$.
  then there is a globally defined dependent function $s:(x:X) \to P(x)$,
  such that for all $x:X$ and $i:I$ we have $p_i:U_i(x) \to s(x)=s_i(x)$ such that $p_i\cdot p_{ij} = p_j$.
\end{lemma}

This can be generalized to $\nType{k}$ for each {\em external} $k$.

The condition for $\nType{0}$ can be seen as an internal version of the usual patching {\em sheaf} condition.
The condition for $\nType{0}$ is then the internal version of the usual patching {\em $1$-stack} condition.

\subsection{Subtypes and Logic}

We use the notation $\exists_{x:X}P(x)\colonequiv \|\sum_{x:X}P(x)\|$.
We use $+$ for the coproduct of types and for types $A,B$ we write
\[ A\vee B\colonequiv \| A+B \|\rlap{.}\]

We will use subtypes extensively.

\begin{definition}
  \index{$\subseteq$}
  Let $X$ be a type.
  A \notion{subtype} of $X$ is a function $U:X\to\Prop$ to the type of propositions.
  We write $U\subseteq X$ to indicate that $U$ is as above.
  If $X$ is a set, a subtype may be called \notion{subset} for emphasis.
  For subtypes $A,B\subseteq X$, we write $A\subseteq B$ as a shorthand for pointwise implication.
\end{definition}

We will freely switch between subtypes $U:X\to\Prop$ and the corresponding embeddings
\[
  \begin{tikzcd}
    \sum_{x:X}U(x) \ar[r,hook] & X
  \end{tikzcd}
  \rlap{.}
\]
In particular, if we write $x:U$ for a subtype $U:X\to\Prop$, we mean that $x:\sum_{x:X}U(x)$ -- but we might silently project $x$ to $X$.

\begin{definition}
  Let $I$ and $X$ be types and $U_i:X\to\Prop$ a subtype for any $i:I$.
  \begin{enumerate}[(a)]
  \item The \notion{union} $\bigcup_{i:I}U_i$\index{$\bigcup_{i:I}U_i$} is the subtype $(x:X)\mapsto \exists_{i:I}U_i(x)$.
  \item The \notion{intersection} $\bigcap_{i:I}U_i$\index{$\bigcap_{i:I}U_i$} is the subtype $(x:X)\mapsto\prod_{i:I}U_i(x)$.
  \end{enumerate}
\end{definition}

We will use common notation for finite unions and intersections.
The following formula hold:

\begin{lemma}
  Let $I$, $X$ be types, $U_i:X\to\Prop$ a subtype for any $i:I$ and $V,W$ subtypes of $X$.
  \begin{enumerate}[(a)]
  \item Any subtype $P:V\to\Prop$ is a subtype of $X$ given by $(x:X)\mapsto\sum_{x:V}P(x)$.
  \item $V\cap \bigcup_{i:I} U_i=\bigcup (V\cap U_i)$.
  \item If $\bigcup_{i:I}U_i=X$ we have $V=\bigcup_{i:I}U_i\cap V$.
  \item If $\bigcup_{i:I}U_i=\emptyset$, then $U_i=\emptyset$ for all $i:I$.
  \end{enumerate}
\end{lemma}

\begin{definition}
  Let $X$ be a type.
  \begin{enumerate}[(a)]
  \item $\emptyset\colonequiv (x:X)\mapsto \emptyset$. \index{$\emptyset$}
  \item For $U\subseteq X$, let $\neg U\colonequiv (x:X)\mapsto \neg U(x)$. \index{$\neg U$}
  \item For $U\subseteq X$, let $\neg\neg U\colonequiv (x:X)\mapsto \neg\neg U(x)$. \index{$\neg\neg U$}
  \end{enumerate}
\end{definition}

\begin{lemma}
  $U=\emptyset$ if and only if $\neg\left(\exists_{x:X}U(x)\right)$.
\end{lemma}

\subsection{Algebra}

\begin{definition}%
  A commutative ring $R$ is \notion{local} if $1\neq 0$ in $R$ and
  if for all $x,y:R$ such that $x+y$ is invertible, $x$ is invertible or $y$ is invertible.
\end{definition}

\begin{definition}%
  Let $R$ be a commutative ring.
  A \notion{finitely presented} $R$-algebra is an $R$-algebra $A$,
  such that there merely are natural numbers $n,m$ and polynomials $f_1,\dots,f_m:R[X_1,\dots,X_n]$
  and an equivalence of $R$-algebras $A\simeq R[X_1,\dots,X_n]/(f_1,\dots,f_m)$.
\end{definition}

\begin{definition}%
  \label{regular-element}
  Let $A$ be a commutative ring.
  An element $r:A$ is \notion{regular},
  if the multiplication map $r\cdot\_:A\to A$ is injective.
\end{definition}

\begin{lemma}%
  \label{units-products-regular}
  Let $A$ be a commutative ring.
  \begin{enumerate}[(a)]
  \item All units of $A$ are regular.
  \item If $f$ and $g$ are regular, their product $fg$ is regular.
  \end{enumerate}
\end{lemma}

\begin{example}
  The monomials $X^k:A[X]$ are regular.
\end{example}

\begin{lemma}%
  \label{polynomial-with-regular-value-is-regular}
  Let $f : A[X]$ be a polynomial
  and $a : A$ an element
  such that $f(a) : A$ is regular.
  Then $f$ is regular as an element of $A[X]$.
\end{lemma}

\begin{proof}
  TODO
\end{proof}

\begin{lemma}%
  \label{fg-ideal-local-global}
  Let $A$ be a commutative ring and $f_1,\dots,f_n:A$.
  For finitely generated ideals $I_i\subseteq A_{f_i}$,
  such that $A_{f_if_j}\cdot I_i=A_{f_if_j}\cdot I_j$ for all $i,j$,
  there is a finitely generated ideal $I\subseteq A$,
  such that $A_{f_i}\cdot I=I_i$ for all $i$.
\end{lemma}

\begin{proof}
  Choose generators 
  \[ \frac{g_{i1}}{1},\dots,\frac{g_{ik_i}}{1} \]
  for each $I_i$.
  These generators will still generate $I_i$, if we multiply any of them with any power of the unit $\frac{f_i}{1}$.
  Now
  \[ A_{f_if_j}\cdot I_i\subseteq A_{f_if_j}\cdot I_j \]
  means that for any $g_{ik}$, we have a relation
  \[ (f_if_j)^l g_{ik}=\sum_{l}h_{l}g_{jl}\]
  for some power $l$ and coefficients $h_{l}:A$.
  This means, that $f_i^lg_{ik}$ is contained in $I_j$.
  Multiplying $f_i^lg_{ik}$ with further powers of $f_i$ or multiplying $g_{jl}$ with powers of $f_j$ does not change that.
  So we can repeat this for all $i$ and $k$ to arrive at elements $\tilde{g_{ik}}:A$,
  which generate an ideal $I\subseteq A$ with the desired properties.
\end{proof}

The following definition also appears as \cite{ingo-thesis}[Definition 18.5]
and a version restricted to external finitely presented algebras was already used by Anders Kock in \cite{kock-sdg}[I.12]:

\begin{definition}
  \label{spec}
  The \notion{(synthetic) spectrum}\index{$\Spec A$} of a finitely presented $R$-algebra $A$
  is the set of $R$-algebra homomorphisms from $A$ to $R$:
  \[ \Spec A \colonequiv \Hom_{\Alg{R}}(A, R) \]
\end{definition}

\begin{definition}%
  \label{standard-open-subset}
  Let $A$ be a finitely presented $R$-algebra.
  For $f:A$, the \notion{standard open subset} given by $f$,
  is the subtype 
  \[
    D(f)\colonequiv (x:\Spec A)\mapsto (x(f)\text{ is invertible})
    \rlap{.}
  \]
\end{definition}

\begin{definition}
  $\AbGroup$\index{$\AbGroup$} denotes the type of abelian groups.
\end{definition}

\begin{lemma}%
  \label{surjective-abgroup-hom-is-cokernel}
  Let $A,B:\AbGroup$ and $f:A\to B$ be a homomorphism of abelian groups.
  Then $f$ is surjective, if and only if, it is a cokernel.
\end{lemma}

\begin{proof}
  A cokernel is a set-quotient by an effective relation,
  so the projection map is surjective.
  On the other hand, if $f$ is surjective and we are in the situation:
  \begin{center}
    \begin{tikzcd}
      \ker(f)\ar[r,hook]\ar[dr] & A\ar[r,"f",->>]\ar[dr,"g"] & B \\
      & 0\ar[r] & C
    \end{tikzcd}
  \end{center}
  then we can construct a map $\varphi:B\to C$ as follows.
  For $x:B$, we define the type of possible values $\varphi(x)$ in $C$ as
  \[
    \sum_{z:C}\exists_{y:A}(f(y)=x)
  \]
  which is a proposition by algebraic calculation.
  By surjectivity of $f$, this type is inhabited and therefore contractible.
  So we can define $\varphi(x)$ as its center of contraction.
\end{proof}

\ignore{
    - injective/embedding/-1-truncated map
  pushouts:
    - inclusions are jointly surjective,
    - pushouts of embeddings between sets are sets
  subtypes:
    - embeddings (composition, multiple definitions, relation to injection)  
    - we freely switch between predicates and types
    - subtypes of subtypes are subtypes
  pullbacks:
    - pasting (reference)
    - pullback of subtype = composition
  algebra:
    - free comm algebras, quotients
    - other definitions of polynomials
    - fp closed under: quotients, adjoining variables, tensor products
}
