We use definitions and results from \cite{draft} and \cite{proper-draft}.

There are a couple of equivalent definitions of finite schemes, which we will introduce and show to be equivalent in this section.

\begin{definition}
  A type $X$ is a \notion{finite scheme} if it is of the form $X=\Spec A$ for a finitely presented $R$-algebra which is finitely generated as an $R$-module.
\end{definition}

\begin{example}
  \begin{enumerate}[(a)]
  \item Finite types.
  \item Infinitesimal disks $\D_k(n)\subseteq\A^n$ of order $k$.
  \end{enumerate}
\end{example}

\begin{theorem}
  Let $X$ be a scheme, then the following are equivalent:
  \begin{enumerate}[(i)]
  \item $X$ is a finite scheme.
  \item There is an $R$-algebra $A$, such that $X=\Spec A$ where $A$ is a finitely presented $R$-module.
  \item $X$ is projective and affine.
  \item $X$ is compact and affine.
  \end{enumerate}
\end{theorem}

\begin{proof}
  (i) $\Leftrightarrow$ (ii): By constructive reading of Tag 0564 in the Stacks Project (TODO: turn into reference).
  
  (i) $\Rightarrow$ (iii): We consider the projective space $\mathbb PA^\star$ associated with the R-linear dual of $A$.
  This is $\bP^{n-1}$ after choosing a basis of $A$.
  Given $[\varphi] : \bP A^\star$ we consider the proposition $C([\varphi])$ that $\varphi(1) \varphi(xy) = \varphi(x) \varphi(y)$ for all $x, y  : A$.
  This is well-defined and a closed proposition because it suffices to check it for basis elements of $A$.
  $C([\varphi])$ implies $\varphi(1) \ne 0$ because otherwise $\varphi(x)^2 = 0$ for all $x : A$ and then $\varphi$ is not-not zero (projective space contains only non-zero vectors).
  So $x \mapsto \varphi(x) / \varphi(1)$ determines a point of Spec A for $[\varphi] : \bP A^\star$ such that $C([\varphi])$ holds (and one can go in the reverse direction, and verify that the two maps are inverse to each other).

  (iv) $\Rightarrow$ (i): (TODO: copy from \#6)
\end{proof}
