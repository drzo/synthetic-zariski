We use definitions and results from \cite{draft} and \cite{proper-draft}.

There are a couple of equivalent definitions of finite schemes, which we will introduce and show to be equivalent in this section.

\begin{definition}
  A type $X$ is a \notion{finite scheme} if it is of the form $X=\Spec A$ for a finitely presented $R$-algebra which is finitely generated as an $R$-module.
\end{definition}

\begin{theorem}
  Let $X$ be a scheme, then the following are equivalent:
  \begin{enumerate}[(i)]
  \item $X$ is a finite scheme.
  \item There is an $R$-algebra $A$, such that $X=\Spec A$ where $A$ is a finitely presented $R$-module.
  \item $X$ is projective and affine.
  \item $X$ is compact and affine.
  \end{enumerate}
\end{theorem}

\begin{proof}
  (i) $\Leftrightarrow$ (ii):
  
\end{proof}
