
\begin{definition}A proposition~$p$ holds \emph{foo-locally} if and only if
there are numbers~$a_1,\ldots,a_n : R$ such that for every
partition~$\{1,\ldots,n\} = I \mathop{\dot\cup} J$, if all the~$a_i$ with~$i
\in I$ are zero and all the~$a_j$ with~$j \in J$ are invertible, then~$p$
holds.
\end{definition}

\begin{definition}A scheme is \emph{finite} if and only if it of the
form~$\Spec(A)$ where~$A$ is finitely presented as an~$R$-algebra and finitely
generated as an~$R$-module.\end{definition}

\begin{definition}A scheme is \emph{quasi-finite} if and only if foo-locally,
it is finite.
\end{definition}

\begin{proposition}Finite schemes are quasi-finite and
compact.\end{proposition}

\begin{proof}Compactness is by Proposition~\ref{finite-compact} and
quasi-finiteness is immediate.
\end{proof}

XXX Question: Does the converse hold? Classically it is
\href{https://math.stackexchange.com/questions/4674878/does-quasi-finite-and-\%C3\%A9tale-locally-closed-enough-to-imply-finite}{well-known}.
Need to check issue \#6.
