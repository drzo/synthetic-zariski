\begin{definition}
The join $X * Y$ of two types $X$, $Y$ is given by the following pushout.
\[
\begin{tikzcd}
X \times Y \arrow[r] \arrow[d]
	& Y \arrow[d] \\
	X \arrow[r] &
	X * Y 
	\arrow[ul, phantom, "\ulcorner" , very near start]
\end{tikzcd}
\]
\end{definition}

Let $n$ be a natural number and $P_1, \ldots, P_n$ types.
We define the join $P_1 * \cdots * P_n$ by induction on $n$,
so that it is empty if $n = 0$ 
and $P_1 * (P_2 * \cdots * P_n)$ if $n > 1$.
Our goal is to describe a precise sense in which
this join is built from the products
$\Pi_{i : I} P_i$ where $I \subseteq [n]$ ranges over detachable,
inhabited subsets.
Note that if $P_i$ are all propositions, then so is the join, with
$P_1 * \cdots * P_n = P_1 \vee \cdots \vee P_n$.

\begin{definition}
We define a sequence $J_{-1} \to J_0 \to J_1 \to \cdots$ of types.
If $n = 0$, we take $J_r = \varnothing$ for all $r$.
If $n > 0$, let $\widehat J_r$ be the sequence obtained recursively from
the types $P_2, \ldots, P_n$.
We take $J_{-1} = \varnothing$ and for $r \ge 0$ define $J_r$ by the following
pushout diagram.
\[
\begin{tikzcd}
P_1 \times \widehat J_{r-1} \arrow[r] \arrow[d]
	& P_1 \arrow[d] \\
	\widehat J_r \arrow[r] &
	J_r 
	\arrow[ul, phantom, "\ulcorner" , very near start]
\end{tikzcd}
\]
The map $J_r \to J_{r+1}$ is induced by functoriality of pushouts via the following
commutative diagram.
\[
\begin{tikzcd}
	\widehat J_{r-1} \arrow[d] &
	P_1 \times \widehat J_{r-2} \arrow[l] \arrow[d] \arrow[r] &
	P_1 \arrow[d] \\
	\widehat J_r &
	P_1 \times \widehat J_{r-1} \arrow[l] \arrow[r] &
	P_1 \\
\end{tikzcd}
\]
\end{definition}

\begin{lemma}\label{eventually-join}
For $r \ge n-1$, the map $J_r \to J_{r+1}$ is an equivalence and
$J_r \simeq P_1 * \cdots * P_n$ is the join.
\end{lemma}
\begin{proof}
Direct by induction on $n$.
\end{proof}

\begin{definition}
For $r$ a natural number, let
$[n]^{(r)}$ denote the type of $r$-element subsets of $[n]$, and
define
\[
Z_r \coloneqq (I : [n]^{(r)}) \times (i : I) \to P_i.
\]
\end{definition}
\begin{lemma}\label{cw}
For $r \ge 0$, we have a pushout square of the following form.
\[
\begin{tikzcd}
Z_{r+1} \times S^{r-1} \arrow[r] \arrow[d]
	& Z_{r+1} \arrow[d] \\
	J_{r-1} \arrow[r] &
	J_r
	\arrow[ul, phantom, "\ulcorner" , very near start]
\end{tikzcd}
\]
\end{lemma}
That is, $J_r$ is obtained from $J_{r-1}$ by attaching $Z_{r+1}$-many $r$-cells.
\begin{proof}
We induct on $n$. For $n = 0$, $Z_{r+1}$ is empty and so there is nothing to prove.
For $r = 0$ the conclusion is also clear.
Thus suppose $n > 0$, $r > 0$ and that the lemma holds for the sequence $P_2, \ldots, P_n$.
Consider the following 3-by-3-diagram, with the pushouts of the rows and columns listed
at the bottom and to the right.
\[
\begin{tikzcd}
 P_1 &
 P_1 \times \widehat Z_r \arrow[l] \arrow[r,equal]&
 P_1 \times \widehat Z_r &
 P_1 \\
 P_1 \times \widehat J_{r-2} \arrow[u] \arrow[d] &
 P_1 \times \widehat Z_r \times S^{r-2} \arrow[l] \arrow[r] \arrow[u] \arrow[d] &
 P_1 \times \widehat Z_r  \arrow[u,equal] \arrow[d] &
 P_1 \times \widehat J_{r-1} \arrow[u] \arrow[d] \\
 \widehat J_{r-1} &
 \widehat Z_{r+1} \times S^{r-1} + P_1 \times \widehat Z_r  \arrow[l] \arrow[r] &
 \widehat Z_{r+1} + P_1 \times \widehat Z_r &
 \widehat J_r \\
	 J_{r-1} & 
	 Z_{r+1} \times S^{r-1} \arrow[l] \arrow[r] & 
	 Z_{r+1} & J_r
\end{tikzcd}
\]
The maps in this diagram are all guessable, and the commutativity of each of
the four squares is direct. We explain how to compute the pushout of each row and column.
The pushout of the first row is $P_1$, since
the pushout of any equivalence is an equivalence.
The pushout of the second row is $P_1 \times \widehat J_{r-1}$,
by inductive hypothesis and using that $P_1 \times -$ preserves pushouts.
The pushout of the third row is $\widehat J_r$, again using inductive hypothesis
as well as the observation that the $P_1 \times \widehat Z_r$-terms do not
affect the pushout.

The pushout of the first column is $J_{r-1}$ by definition.
To compute the pushout of the second column, we observe that the 
$\widehat Z_{r+1} \times S^{r-1}$-term does not interact with the rest of the column,
that the suspension of $S^{r-2}$ is $S^{r-1}$, and that $P_1 \times \widehat Z_r \times -$
preserves pushouts. All together, this shows that
the pushout is $\widehat Z_{r+1} \times S^{r-1} + P_1 \times \widehat Z_r \times S^{r-1}$,
i.e.\ $(\widehat Z_{r+1} + P_1 \times \widehat Z_r) \times S^{r-1}$,
i.e.\ $Z_{r+1} \times S^{r-1}$.
Finally, the third pushout is 
$\widehat Z_{r+1} + P_1 \times \widehat Z_r$ since the pushout of an equivalence
is an equivalence, i.e.\ $Z_{r+1}$.

The $3 \times 3$-lemma tells us that the the pushout of row-wise pushouts is equivalent
to the pushout of column-wise pushouts. That is, 
$J_r$ is a pushout of the desired form. (Here one should be careful to
		check that the maps are the ones we expect.)
\end{proof}

Now let $X$ be a type, $n$ a natural number, and $P_i$ a type family over
$X$ for each $i : [n]$. For any $x : X$,
$P_1(x), \ldots, P_n(x)$ is simply a list of types, to which we may apply
Lemma \ref{cw}.
Taking sigma over $x : X$ preserves pushouts (since it is a left adjoint),
	   so we obtain the following pushout square for each $r \ge 0$.
\begin{equation}\label{sigma-po}
\begin{tikzcd}
(x : X) \times Z_{r+1}(x) \times S^{r-1} \arrow[r] \arrow[d]
	& (x : X) \times Z_{r+1}(x) \arrow[d] \\
	(x : X) \times J_{r-1}(x) \arrow[r] &
	(x : X) \times J_r(x)
	\arrow[ul, phantom, "\ulcorner" , very near start]
\end{tikzcd}
\end{equation}

We could now use Mayer--Vietoris, % and maybe we should!
but instead let us
consider the following lemma which explains how to compute the cohomology of such a pushout.
It is in the spirit of cellular cohomology. 

\begin{lemma}\label{cw-fibre}
Let $n \ge 0$ a natural number, and suppose given the
following pushout square.
\[
\begin{tikzcd}
I \times S^n \arrow[r] \arrow[d]
	& I \arrow[d] \\
	C \arrow[r] &
	C'
	\arrow[ul, phantom, "\ulcorner" , very near start]
\end{tikzcd}
\]
Then for any parametrised spectrum $A$ over $C'$ we have the following fibre
sequence of spectra, where $A^C$ denotes the cohomology spectrum of $C$ with
coefficients in $A$:
\[
	A^{C'} \to A^C \to \Omega^n A^I.
\]
\end{lemma}
\begin{proof}
By the universal property of $A^-$, it turns colimits into limits, so we have the following
pullback square in spectra.
\[
\begin{tikzcd}
A^{C'} \arrow[r] \arrow[d] \arrow[dr, phantom, "\lrcorner" , very near start]
	& A^I \arrow[d] \\
	A^C \arrow[r] &
	A^{I \times S^n}
\end{tikzcd}
\]
We compute the bottom-right term:
\[A^{I \times S^n} \simeq (A^{S^n})^I \simeq (A \oplus \Omega^n A)^I \simeq A^I \oplus \Omega^n A^I,\]
using that $(S^n \to B) \simeq B \times \Omega^n B$ for any homogeneous pointed type $B$.
Here $\oplus$ denotes biproduct of spectra.
It can be seen that the map $A^I \to A^{I \times S^n}$ corresponds to the left inclusion into
this direct sum. We can then paste this pullback diagram with
\[
\begin{tikzcd}
A^I \arrow[r] \arrow[d] \arrow[dr, phantom, "\lrcorner" , very near start]
	& 0 \arrow[d] \\
	A^I \oplus \Omega^n A \arrow[r] &
	\Omega^n A^I
\end{tikzcd}
\]
to get the desired fibre sequence.
\end{proof}

Now suppose $M$ is a family of abelian groups over $X$, and we are interested in
cohomology with coefficients in $M$.
Note that $(x : X) \times Z_{r+1}$ is a finite coproduct, and
cohomology is additive in those, so we have
\[H^l((x : X) \times Z_{r+1}(x))
	\cong
	\bigoplus_{I : [n]^{(r+1)}} H^l((x : X) \times (i : I) \to P_i(x)).\]

Let us write $C_{r+1}$ for this group in the case where $l = 0$.

\begin{lemma}\label{acyclic-rec}
If $P_i$ is acyclic with regard to $M$ in the sense that
$H^l((x : X) \times (i : I) \to P_i(x)) = 0$ for $l > 0$ and any
$r \ge 1$, $I : [n]^{(r)}$, then we have the following description of
$H^l_r \coloneqq H^l((x: X) \times J_r(x))$:
\[H^l_r \cong H^l_{r-1}\]
for $l < r-1$ (where the map is induced by the map $J_{r-1}(x) \to J_r(x)$),
	\[H^{r-1}_r \cong \ker\,\delta\]
	and
	\[H^r_r \cong \coker\,\delta\]
	where $\delta : H^{r-1}_{r-1} \to C_{r+1}$
	is induced by the attaching map $Z_{r+1}(x) \times S^{r-1} \to J_r(x)$,
	and
\[H^l_r \cong 0\]
for $l > r$.
\end{lemma}
\begin{proof}
We induct on $r$. For $r = -1$ there is nothing to prove. For $r \ge 0$ we start with
the pushout square (1). %could not get referencing to work properly
To this we apply Lemma \ref{cw-fibre}, obtaining
a fibre sequence of cohomology spectra. Consider the associated long exact sequence
on homotopy groups. By assumption, one of the spectra in our fibre sequence has
its cohomology concentrated in degree $r-1$, where it is
$C_{r+1}$, leading to the desired result.
\end{proof}
From Lemma \ref{acyclic-rec}, we have that $H^r_r$ is a quotient of $C_{r+1}$
for all $r$. In particular, the map $\delta : H^{r-1}_{r-1} \to C_{r+1}$ appearing in the statement of Lemma \ref{acyclic-rec}
composes with the quotient map to give a map $C_r \to C_{r+1}$.
We expect this map to be the usual boundary map appearing in the \v{C}ech complex.
Modulo this gap, we arrive at our main result.
\begin{theorem}
For any $P_1,\ldots,P_n$ which are acyclic with regard to $M$, the cohomology groups 
\[
H^l((x: X) \times P_1(x) * \cdots * P_n(x); M)
\]
of the fibrewise join with coefficients in $M$ agree with the cohomology of the \v{C}ech complex,
i.e.\ the kernel of the boundary map $C_{l+1} \to C_{l+2}$ modulo the image of the boundary
map $C_l \to C_{l+1}$.
\end{theorem}
\begin{proof}
Combining Lemmas \ref{eventually-join} and \ref{acyclic-rec}. (Modulo the gap about describing
		the boundary map $C_r \to C_{r+1}$.)
\end{proof}
Note that if $P_i(x)$ are propositions, then the fibrewise join
is simply the union of subtypes, and the types 
$(x : X) \times (i: I) \to P_i(x)$ appearing in the \v{C}ech complex
are simply intersections of subtypes.
