
\begin{lemma}%
  \label{lem:kraus-glueing}
  Let $X$ and $I$ be types.
  For $(U_i:X \to \Prop)_{i:I}$ and $P:U_i\to \nType{0}$, we have the following glueing property: \\
  If for each $i:I$ there is a dependent function $s_i:(x:U_i)\to P(x)$ together with
  proofs of equality on intersections $p_{ij}:(x:U_i\cap U_j)\to (s_i(x)=s_j(y))$,
  then there is a globally defined dependent function $s:(x:X) \to P(x)$,
  such that for all $x:X$ and $i:I$ we have $U_i(x) \to s(x)=s_i(x)$
\end{lemma}
\begin{proof}
  We define $s$ pointwise.
  Let $x:X$.
  Using a Lemma of Kraus and the $p_{ij}$, we get a factorization
  \[ \begin{tikzcd}[row sep=0mm]
    \sum_{i:I} U_i(x) \ar[rr, "s_{\pi_1(\_)}(x)"]\ar[rd] & & P(x) \\
    & \|\sum_{i:I} U_i(x)\|_{-1}\ar[ru,dashed] &
  \end{tikzcd} \]
-- which defines a unique value $s(x):P(x)$.
\end{proof}

The following is in conflict with the usual use of the word ``connected'' in homotopy type theory.

\begin{definition}%
  \label{def:connected}
  A pointed type $X$ is called \notion{connected},
  if any function $X\to \Bool$ is constant.
\end{definition}

\begin{proposition}[\axiomref{sqc}, \axiomref{loc}]%
  \label{proposition:A1-connected}
  The set $\A^1$ is connected, that is,
  every function $f : \A^1 \to \Bool$ is constant.
\end{proposition}

\begin{proof}
  We embed $\Bool$ into $R$ as the subset $\{0, 1\} \subseteq R$.
  (We have $0 \neq 1$ in $R$ by (\axiomref{loc}).)
  Then we have a function $\widetilde{f} : \A^1 \to R$
  and we can assume $\widetilde{f}(0) = 0$.
  Note that $\widetilde{f}$ is an idempotent element of the algebra $R^{\A^1}$,
  since all its values are idempotent elements of $R$.
  By (\axiomref{sqc}),
  $\widetilde{f}$ is given by an idempotent polynomial $p \in R[X]$
  with $p(0) = 0$.
  But from this follows $p = 0$:
  we can factorize $p = X q$
  and then calculate $p = p^n = X^n q^n$
  to see that all coefficients of $p$ are zero.
\end{proof}

A connected type, that is covered by its point and everything except the point,
is already trivial.

\begin{corollary}[\axiomref{Z-choice}]%
  \label{cor:connected-to-contractible}
  Let $X$ be connected and 
  \[\prod_{x:X}x=* \vee x\neq *.\]
  Then $X$ is contractible.
\end{corollary}

\begin{proof}
  Assume $\prod_{x:X}x=* \vee x\neq *$.
  By \axiomref{Z-choice} we get $s_i:(x:D(f_i))\to x=* + x\neq *$
  for a Zariski-cover $D(f_1),\dots,D(f_n)$ of $X$. 
  On intersections $D(f_if_j)$, we have $s_i(x)=s_j(x)$
  since types of the form $P+\neg P$ for a proposition $P$ are propositions.
  This means we can use \cref{lem:kraus-glueing} to construct a map $X\to \Bool$,
  which decides if a general $x:X$ is the point $*$ or not.
  By connectedness of $X$, this map is constant, but we know $*=*$,
  so $x=*$ for all $x$.
\end{proof}

\begin{corollary}[\axiomref{sqc}, \axiomref{loc}, \axiomref{Z-choice}]%
  $\neg(\prod_{x:\A^1}x=0 \vee x\neq 0)$
\end{corollary}

\begin{proof}
  By \cref{cor:connected-to-contractible} and by the connectedness of $\A^1$ (\cref{proposition:A1-connected}),
  we can show from $\prod_{x:\A^1}x=0 \vee x\neq 0$ that $\A^1$ is contractible.
  This contradicts $1\neq 0$.
\end{proof}


\begin{definition}
  A \notion{closed proposition} is a proposition
  which is merely of the form $x_1 = 0 \land \dots \land x_n = 0$
  for some elements $x_1, \dots, x_n \in R$.
\end{definition}

\begin{proposition}[\axiomref{sqc}]
  There is an order-reversing isomorphism of partial orders
  \begin{align*}
    \text{f.g.-ideals}(R) &\xrightarrow{{\sim}} \Omega_{cl} \\
    I &\mapsto (I = (0))
  \end{align*}
  between the partial order of finitely generated ideals of $R$
  and the partial order of closed propositions.
\end{proposition}

\begin{proof}
  For a finitely generated ideal $I = (x_1, \dots, x_n)$,
  the proposition $I = (0)$ is indeed a closed proposition,
  since it is equivalent to $x_1 = 0 \land \dots \land x_n = 0$.
  It is also evident that we get all closed propositions in this way.
  What remains to show is that
  \[ I = (0) \Rightarrow J = (0)
     \qquad\text{iff}\qquad
     J \subseteq I
     \rlap{\text{.}}
  \]
  For this we use synthetic quasicoherence.
  Note that the set $\Spec R/I = \Hom_R(R/I, R)$ is a proposition
  (has at most one element),
  namely it is equivalent to the proposition $I = (0)$.
  Similarly, $\Hom_R(R/J, R/I)$ is a proposition
  and equivalent to $J \subseteq I$.
  But then our claim is just the equation
  \[ \Hom(\Spec R/I, \Spec R/J) = \Hom_R(R/J, R/I) \]
  which holds by Lemma~\ref{MISSING},
  since $R/I$ and $R/J$ are finitely presented $R$-algebras
  if $I$ and $J$ are finitely generated ideals.
\end{proof}

\begin{lemma}[\axiomref{sqc}]
  Let $A$ be a finitely presented $R$-algebra
  and let $f, g_1, \dots, g_n \in A$.
  Then we have $D(f) \subseteq D(g_1, \dots, g_n)$
  as subsets of $\Spec A$
  if and only if $f \in \sqrt{(g_1, \dots, g_n)}$.
\end{lemma}

\begin{proof}
  Since $D(g_1, \dots, g_n) = \{\, x \in \Spec A \mid x \notin V(g_1, \dots, g_n) \,\}$,
  the inclusion $D(f) \subseteq D(g_1, \dots, g_n)$
  can also be written as
  $D(f) \cap V(g_1, \dots, g_n) = \varnothing$, that is,
  $\Spec((A/(g_1, \dots, g_n))[f^{-1}]) = \varnothing$.
  By (\axiomref{sqc})
  this means that the finitely presented $R$-algebra $(A/(g_1, \dots, g_n))[f^{-1}]$
  is zero.
  And this is the case if and only if $f$ is nilpotent in $A/(g_1, \dots, g_n)$,
  that is, if $f \in \sqrt{(g_1, \dots, g_n)}$, as stated.
\end{proof}
