
\begin{definition}%
  A type $X$ is \notion{(qc-)affine},
  if the there is a finitely presented $R$-algebra $A$, such that $X=\Spec A$. 
\end{definition}

\begin{proposition}%
  Let $X$ be a type.
  The type of all finitely presented $R$-algebra $A$, such that $X=\Spec A$, is a proposition.
\end{proposition}

When we write ``$\Spec A$'' we implicitly assume $A$ is a finitely presented $R$-algebra.

\begin{definition}%
  Let $X=\Spec A$.
  A \notion{(affine) standard open} is the subtype $D(f):X\to\Prop$
  for some $f:\Spec A\to R$ given by $D(f)(x)\colonequiv f(x)\neq 0$.
\end{definition}

\begin{definition}%
  Let $X=\Spec A$.
  A subtype $U:X\to\Prop$ is called \notion{affine-open},
  if one of the following logically equivalent statements holds:
  \begin{enumerate}[(i)]%
  \item $U$ is the intersection of fintely many affine standard opens.
  \item There are $f_1,\dots,f_n:A$ such that
    \[U(x) \Leftrightarrow \exists_{i}f_i\neq 0 \]
  \end{enumerate}
\end{definition}