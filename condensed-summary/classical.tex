A proposition is {\em open} iff it is of the form $\exists_n a_n$ with $a_n$ decidable.
It is {\em closed} if it is of the form $\forall_n a_n$ with $a_n$ decidable.
We write $\Open$ (resp. $\Closed$) the set of open (resp. closed) propositions.


\begin{theorem}[The negation of the weak limited principle of omniscience]
   It is not the case that for all $\alpha:\N\to 2$ we can decide whether $\forall (n:\N).\ \alpha(n)=0$.
\end{theorem}

\begin{theorem}[Markov's principle]
  For all $\alpha:\N\to 2$, we have that:
   \[\neg(\forall (n:\N).\ \alpha(n)=0) \to \exists (n:\N).\ \alpha(n)=1\]
\end{theorem}

This can be rephrased as: the negation of a closed proposition is open. It is direct that the negation
of an open proposition is closed.

It follows that both open and closed propositions are not not stable. (They are not decidable in general.)

\begin{definition}
Let $\Noo$ be the type of sequence:
\[\alpha:\N\to 2\]
where $\alpha$ has value $1$ at most one. We have $\Noo:\Stone$.
\end{definition}

\begin{theorem}[The lesser limited principle of omniscience (LLPO)]
  For $\alpha:\mathbb \Noo$, 
  we have that 
  \begin{equation}\label{eqnLLPO}
    \left(\forall (k:\mathbb N).\ \alpha(2k) = 0 \right) \vee \left(\forall (k:\mathbb N).\ \alpha(2k+1) = 0\right)
  \end{equation}
\end{theorem}

This can be rephrased as the fact that the map $\Noo+\Noo\rightarrow \Noo$ sending $\inl(\alpha)$ to $\lambda_k \alpha(2k)$
and $\inr(\alpha)$ to $\lambda_k \alpha(2k+1)$ is {\em surjective}.

\medskip

Since this map has no section, this shows that $\Noo$ is {\em not} projective.
(David W\"arn has noticed that $\mathbb{Z}[\Noo]$ is {\em not} internally projective in the category of
Abelian groups.)

\medskip

 Yet another formulation of LLPO is that the disjunction of two closed propositions is closed.

 
