\begin{theorem}
Let $S$ be a Stone space, then for all $i>0$ we have \[H^i(S,\ints) = 0\]
\end{theorem}

If $X:\CHaus$ and $S:\Stone$ with $S\rightarrow X$ surjective, of fiber $S_x$ for $x:X$,  we can consider the cochain complex
\[\Pi_{x:X}\ints^{S_x}\rightarrow \Pi_{x:X}\ints^{S_x\times S_x}\rightarrow \Pi_{x:X}\ints^{S_x\times S_x\times S_x}\rightarrow\dots\]
then the cohomology groups $H^n(X,\ints)$, in the sense of univalent type theory, are exactly the cohomology groups of this
cochain complex. In particular, we have the following.

\begin{proposition}
For all $i>0$ we have \[H^i([0,1],\ints) = 0\]
\end{proposition}

Similar to real-cohesion, we can construct a shape modality, which we expect to map finite CW-complexes to their fundamental $\infty$-groupoids:

\begin{definition}
  $\shape$ is the modality given by nullification at the interval $[0,1]$.
\end{definition}

\begin{proposition}
  The shape of the topological circle $\bS^1\colonequiv \R/\Z$ is the higher inductive circle: 
  \[
  \shape(\bS^1)=\mathrm{K}(\Z,1)=S^1
  \]
  Since $K(\Z,n)$ is $\shape$-modal, we also have:
  \[
  H^n(\bS^1,\Z) = H^n(\shape \bS^1,\Z) = H^n(S^1,\Z) = \Z
  \]
\end{proposition}

%We expect to be able to prove that cohomology with value in overtly discrete abelian groups is well-behaved on CW complex.
