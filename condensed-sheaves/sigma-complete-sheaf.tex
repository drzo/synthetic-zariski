We work in a presheaf topos on a $\sigma$-complete boolean algebra.


\subsection{Sheaves}

\begin{definition}
A type $X$ is a sheaf if for all fundamental system of idempotent $e_1,\cdots,e_n:B$ in $B$, we have that $X$ is $(e_1=1\lor\cdots\lor e_n=1)$-local.
\end{definition}

\begin{lemma}\label{sigma-complete-sheaf-simpler-aux}
A type $X$ is a sheaf if and only if $0=_B1$ implies $X$ contractible and for all $e:B$ we have that $X$ is $(e=0\lor e=1)$-local.
\end{lemma}

\begin{proof}
Consider $e_1,\cdots,e_n$ a fundamental system of idempotents. If $n=0$ then $0=_B1$ and $X$ is contractible. 

Otherwise we proceed by induction on $n$. If $n=1$ it is clear. If $n\geq 2$ we want to check that the localisation of $e_1=1\lor\cdots\lor e_n=1$ at $e=0\lor e=1$ is contractible. We know it is a proposition so it is enough to check that it is inhabited. But we have that:
\[(e_1=1\lor e_1=0) \to (e_1=1\lor\cdots\lor e_n=1)\]
as if $e_1=1$ it is clear, and if $e_1=0$ then $e_2,\cdots,e_n$ is a fundamental system of idempotent and we conclude by induction.

The converse is immediate.
\end{proof}

\begin{lemma}\label{sigma-complete-sheaf-simpler}
Given a type $X$ such that $0=_B1$, the following are equivalent:
\begin{enumerate}[(i)]
\item $X$ is a sheaf.
\item For all $e:B$, we have that $X$ is $(e=0\lor e=1)$-local.
\item For all $e:B$, we have that $X$ is$(e=0 + e=1)$-local.
\end{enumerate}
\end{lemma}

\begin{proof}
We have that (i) $\Leftrightarrow$ (ii) by \cref{sigma-complete-sheaf-simpler-aux}.

To prove (ii) $\Leftrightarrow$ (iii), it is enough to check that for all $e:B$ the canonical map:
\[X^{e=0\lor e=1} \to X^{e=0+ e=1}\]
is an equivalence. From \cref{join-zariski-equiv-binary} we just need to prove that $X^{e=0\land e=1}$ is contractible, but $0=1$ implies $X$ contractible by hypothesis so we can conclude.
\end{proof}

We will study the topos of such sheaves.


\subsection{Interpretation in the sheaf topos}

\begin{lemma}\label{prop-trunc-sheaf-sigma-complete}
Let $X$ be a sheaf, then $\propTrunc{X}$ is a sheaf. 
\end{lemma}

\begin{proof}
We use \cref{sigma-complete-sheaf-simpler}. If $0=1$ implies $X$ contractible, then it implies $\propTrunc{X}$ is contractible.

Given $e:B$ we check that $\propTrunc{X}$ is $(e=0 + e=1)$-local. It is clear that $\propTrunc{X}$ is $(e=0+e=1)$-separated because it is a proposition. Given a map:
\[(e=0+e=1) \to \propTrunc{X}\]
we can merely fill the a diagram:
\begin{center}
\begin{tikzcd}
e=0+e=1\ar[d]\ar[r]\ar[rd,dotted] & \propTrunc{X}\\
1\ar[dotted,r] & X\ar[u]\\
\end{tikzcd}
\end{center}
where the dotted diagonal arrow comes from $(e=0+e=1)$ having choice and the dotted down arrow comes from $X$ being a sheaf.
\end{proof}

This means that the propositional truncation can be interpreted as itself in the sheaf topos.

We write $\hat{X}$ for the intepretation of $X$ in the sheaf topos. 

\begin{lemma}
The boolean algebra $B$ is a sheaf.
\end{lemma}

\begin{proof}
We use \cref{sigma-complete-sheaf-simpler}. It is clear that if $0=_B1$ then $B$ is contractible.

Assume $e:B$, we check that the map:
\[B\to B^{e=0+e=1}\]
is an equivalence. This map is equivalent to the map:
\[B\to B^{1-e=1}\times B^{e=1}\]
which by the assumed duality is equivalent to the map:
\[B\to B_{1-e}\times B_e\]
We can check this is an equivalence by direct algebraic computations.
\end{proof}

This means that we can define $\hat{B}$ in the sheaf topos simply as $B$ with this proof that it is a sheaf.

\begin{lemma}\label{bot-sheaf-sigma-complete}
We have that $\hat{\bot}$ is equivalent to $0=_B1$.
\end{lemma}

\begin{proof}
It is clear that $\hat{\bot}$ is a proposition and that:
\[0=_B1 \to \hat{\bot}\]
as $0=_B1$ implies all sheaves contractible. On the other hand since $B$ is a sheaf we have that $0=_B 1$ us a sheaf, so to prove that:
\[\hat{\bot}\to 0=_B1\]
it is enough to prove that:
\[\bot\to 0=_B1\]
\end{proof}

So we can interpret $\bot$ as $0=_B1$. We have a similar result for $\hat{\N}$, indeed:

\begin{lemma}\label{N-sheafification-sigma-complete}
We have that $\hat{\N}$ is equivalent to the sheafification of $\N$,
\end{lemma}

\begin{proof}
We just check by direct computation. Omitted, holds for any lex modality.
\end{proof}

It should be noted that $\hat{N}$ and the sheafification of $\N$ do not compute the same. Nevertheless when proving the interpretation of a formula we can assume $\hat{\N}$ is the sheafification of $\N$, as long as the formula does not depend on the computation rules for $\N$.


\subsection{Weak limited principle of omniscience}

\begin{lemma}\label{B-is-2-sigma-complete}
The interpretation of $B=2$ in the sheaf topos holds.
\end{lemma}

\begin{proof}
First we prove the interpretation of:
\[0\not=_B1\] 
in the sheaf topos. This is true as $0=_B1$ is interpreted as itself, and $\bot$ as $0=_B1$ by \cref{bot-sheaf-sigma-complete}.

It is clear that the interpretation of:
\[\forall(e:B).\ e=1 \lor e=0\]
holds in the sheaf topos.

These two properties imply $B=2$. 
\end{proof}

The proof of the next theorem is more verbose than needed. The key point is that by \cref{N-sheafification-sigma-complete} we have that $\hat{\N}$ is equivalent to the sheafification of $\N$. 

\begin{theorem}
In the sheaf topos, we have that $2$ is $\sigma$-complete.
\end{theorem}

\begin{proof}
By \cref{B-is-2-sigma-complete} we know it is enough to show that $B$ is $\sigma$-complete in the sheaf topos. 

By \cref{N-sheafification-sigma-complete}, we can assume $\hat{\N}$ the interpretation of $\N$ in the sheaf topos is the sheafification of $\N$.
\begin{itemize}
\item We need to define an inhabitant of:
\[(\hat{\N} \to B) \to B\]
but this type is actually equivalent to:
\[(\N\to B) \to B\]
so we can just use $B$ being $\sigma$-complete.
\item We need to prove that:
\[\forall (\alpha:\hat{\N}\to B)(n:\hat{\N}).\ \phi(n)\leq \hat{\lor}\alpha\]
but this type is actually equivalent to:
\[\forall (\alpha:\N\to B)(n:\N).\ \phi(n)\leq \lor\alpha\]
as $B$ and inequalites in $B$ are sheaves.
\item Similarly we need to prove:
\[\forall (e:B)(\alpha:\hat{\N}\to B).\ \big((\forall(n:\hat{\N}). \alpha(n)\leq e) \to \hat{\lor} \alpha \leq e\big)\]
but this type is equivalent to:
\[\forall (e:B)(\alpha:\N\to B).\ \big((\forall(n:\N). \alpha(n)\leq e) \to \lor \alpha \leq e\big)\]
as inequalities in $B$ are sheaves.
\end{itemize}
\end{proof}

\begin{corollary}[WLPO]
In the sheaf model, given $\alpha:\N\to 2$ we have that:
\[\forall(n:\N).\ \alpha_n=0\]
is decidable.
\end{corollary}

\begin{proof}
We have that $\lor \alpha = 0$ if and only $\forall(n:\N).\ \alpha_n = 0$, and $\lor \alpha = 0$ is decidable.
\end{proof}


\subsection{Dependent choice}

\begin{theorem}[DC]
Dependent choice holds in the sheaf topos.
\end{theorem}

\begin{proof}
Recall that by \cref{prop-trunc-sheaf-sigma-complete} the propositional truncation can be interpreted as itself, and by \cref{N-sheafification-sigma-complete} we have that $\N$ can be interpreted as the sheafification on $\N$.

Then we see that the interpretation of dependent choice in the sheaf topos is equivalent to dependent choice for sheaves in the presheaf topos. We omit the details.
\end{proof}