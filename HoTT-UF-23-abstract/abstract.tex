% latexmk -pdfxe -pvc -interaction=nonstopmode abstract.tex
\documentclass{../util/zariski}

\title{A Foundation for Synthetic Algebraic Geometry}
\author{Felix Cherubini, Thierry Coquand and Matthias Hutzler}

\begin{document}
\maketitle

Algebraic geometry is the study of solutions of non-linear equations using methods from geometry.
Most prominently, algebraic geometry was essential in the proof of Fermat's last theorem by Wiles and Taylor.
The central geometric objects in algebraic geometry are called \emph{schemes}.
Their basic building blocks are called \emph{affine schemes},
where, informally, an affine scheme corresponds to a solution sets of polynomial equations.
While this correspondence is clearly visible in the functorial approach to algebraic geometry and our synthetic approach,
it is somewhat obfuscated in the most commonly used, topological appraoch.

In recent years,
formalization of the intricate notion of affine schemes
received some attention as a benchmark problem
-- this is, however, \emph{not} a problem addressed by this work.
Instead, we use a synthetic approach to algebraic geometry,
very much alike to that of synthetic differential geometry.
This means, while a scheme in classical algebraic geometry is a complicated compound datum,
we work in a setting where schemes are types,
with an additional property that can be defined within our synthetic theory.

In our work in progress \cite{draft},
following ideas of Ingo Blechschmidt and Anders Kock  (\cite{ingo-thesis}, \cite{kock-sdg}[I.12]),
we use a base ring $R$, which is local and satisfies an axiom reminiscent of the Kock-Lawvere axiom.
A more general axiom, is called \emph{synthetic quasi coherence (SQC)} by Blechschmidt and
a version quatifying over external algbras is called the \emph{comprehensive axiom} by Kock.\ 
\footnote{In \cite{kock-sdg}[I.12], Kock's ``axiom $2_k$'' could equivalently be thoerem 12.2,
  which is exactly our synthetic quasi coherence axiom, except that it only quantifies over external algebras.}
However, the exact concise form of the SQC axiom we use, was noted by David Jaz Myers in 2018 and communicated to the first author.

Before we state the SQC axiom, let us take a step back and look at the basic objects of study in algebraic geometry,
solutions of polynomial equations.
Given a system of polynomial equations
\begin{align*}
  p_1(X_1, \dots, X_n) &= 0\rlap{,} \\
  \vdots\quad\quad\;\;   \\
  p_m(X_1, \dots, X_n) &= 0\rlap{,}
\end{align*}
the solution set
$\{ x : R^n \mid \forall i.\; p_i(x_1, \dots, x_n) = 0 \}$
is in canonical bijection to the set of $R$-algebra homomorphisms
\[ \Hom_R(R[X_1, \dots, X_n]/(p_1, \dots, p_m), R) \]
by identifying a solution $(x_1,\dots,x_n)$ with the homomorphism that maps each $X_i$ to $x_i$.
Conversely, for any $R$-algebra $A$, which is merely of the form $R[X_1, \dots, X_n]/(p_1, \dots, p_m)$,
we define the \emph{spectrum} of $A$ to be
\[
  \Spec A \colonequiv \Hom_R(A, R)
  \rlap{.}
\]
In contrast to classical, non-synthetic algebraic geometry,
where this set needs to be equipped with additional structure,
we postulate axioms that will ensure that $\Spec A$ has the expected geometric properties.
Namely, SQC is the statement that, for all finitely presented $R$-algebras $A$, the canonical map
  \begin{align*}
    A&\xrightarrow{\sim} (\Spec A\to R) \\
    a&\mapsto (\varphi\mapsto \varphi(a))
  \end{align*}
is an equivalence.
A prime example of a spectrum is $\A^1\colonequiv \Spec R[X]$,
which turns out to be the underlying set of $R$.
With the SQC axiom,
\emph{any} function $f:\A^1\to \A^1$ is given as a polynomial with coefficients in $R$.
In fact, all functions between affine schemes are given by polynomials.
Furthermore, for any affine scheme $\Spec A$,
the axiom ensures that
the algebra $A$ can be reconstructed as the algebra of functions $\Spec A \to R$,
therefore establishing a duality between affine schemes and algebras.

In the accompanying formalization \cite{formalization} of some basic results,
we use a setup which was already propose by David Jaz Myers
in a conference talk (\cite{myers-talk1, myers-talk2}).
On top of Myers' ideas,
we were able to define schemes, develop some topological properties of schemes,
and construct projective space.

An important, not yet formalized result
is the construction of cohomology groups.
This is where the \emph{homotopy} type theory really comes to bear --
instead of the hopeless adaption of classical, non-constructive definitions of cohomology,
we make use of higher types,
for example the $k$-th Eilenberg-MacLane space $K(R,k)$ of the group $(R,+)$.
As an analogue of classical cohomology with values in the structure sheaf,
we then define cohomology with coefficients in the base ring as:
\[
  H^k(X,R):\equiv \|X\to K(R,k)\|_0
  \rlap{.}
\]
This definition is very convenient for proving abstract properties of cohomology.
For concrete calculations we make use of another Axiom,
which we call \emph{Zariski-local choice}.
While this axiom was conceived of for exactly these kind of calculations,
it turned out to settle numerous questions with no apparent connection to cohomology.

A model for the axioms we use is given by the \emph{Zariski topos} over any base.
Apart from ensuring consistency,
we also expect that we can use this model for concrete applications.


\printbibliography

\end{document}

