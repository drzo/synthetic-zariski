% latexmk -pdfxe -pvc -interaction=nonstopmode abstract.tex
\documentclass{../util/zariski}

\begin{document}

\begin{center}
  \LARGE{Synthetic Algebraic Geometry}
\end{center}

Algebraic geometry is the study of solutions of non-linear equations using methods from geometry.
Most prominently, algebraic geometry was essential in the proof of Fermat's last theorem by Wiles.
The central geometric objects in algebraic geometry are called \emph{schemes}.
Their basic building blocks are solution sets of polynomial equations, called \emph{affine schemes}.
In recent years,
formalization of the intricate notion of affine schemes
received some attention as a benchmark problem
-- this is, however, \emph{not} a problem addressed by this work.
Instead, we use a synthetic approach to algebraic geometry,
very much alike to that of synthetic differential geometry.
This means, while a scheme in classical algebraic geometry is a complicated compound datum,
we work in a setting where schemes are types,
with an additional property we can define using axioms.

Following ideas of Ingo Blechschmidt and Anders Kock  (\cite{kock-sdg}[I.12], \cite{ingo-thesis}),
we work with a base ring $R$, which is local and ...

\printbibliography

\end{document}

