% latexmk -pdfxe -pvc -interaction=nonstopmode abstract.tex
\documentclass{../util/zariski}

\begin{document}

\begin{center}
  \LARGE{A Foundation for Synthetic Algebraic Geometry}
\end{center}

Algebraic geometry is the study of solutions of non-linear equations using methods from geometry.
Most prominently, algebraic geometry was essential in the proof of Fermat's last theorem by Wiles.
The central geometric objects in algebraic geometry are called \emph{schemes}.
Their basic building blocks are solution sets of polynomial equations, called \emph{affine schemes}.
In recent years,
formalization of the intricate notion of affine schemes
received some attention as a benchmark problem
-- this is, however, \emph{not} a problem addressed by this work.
Instead, we use a synthetic approach to algebraic geometry,
very much alike to that of synthetic differential geometry.
This means, while a scheme in classical algebraic geometry is a complicated compound datum,
we work in a setting where schemes are types,
with an additional property we can define using axioms.

Following ideas of Ingo Blechschmidt and Anders Kock  (\cite{ingo-thesis}, \cite{kock-sdg}[I.12]),
we work with a base ring $R$, which is local and satisfies an axiom reminiscent of the Kock-Lawvere axioms,
which is called \emph{synthetic quasi coherence (SQC)} by Blechschmidt and \emph{comprehensive axiom} by Kock.
Before we state the SQC axiom, let us take a step and look at the basic objects of study in algebraic geometry,
solutions of polynomial equations. 
Given a system of polynomial equations
\begin{align*}
  p_1(X_1, \dots, X_n) &= 0\rlap{,} \\
  \vdots  \\
  p_m(X_1, \dots, X_n) &= 0\rlap{,}
\end{align*}
the solution set
$\{ x : R^n \mid \forall i.\; p_i(x_1, \dots, x_n) = 0 \}$
is in canonical bijection to the set of $R$-algebra homomorphisms
\[ \Hom_R(R[X_1, \dots, X_n]/(p_1, \dots, p_m), R) \]
by identifying a solution $(x_1,\dots,x_n)$ with the homomorphism that maps each $X_i$ to $x_i$.
Conversely, for any $R$-algebra $A$, which is merely of the form $R[X_1, \dots, X_n]/(p_1, \dots, p_m)$,
we define the \emph{spectrum} of $A$ to be
\[
  \Spec A \colonequiv \Hom_R(A, R)
  \rlap{.}
\]
In contrast to classical, non-synthetic algebraic geometry,
where this set needs to be equipped with additional structure,
we postulate axioms that will ensure that $\Spec A$ has the expected geometric properties.
Namely, SQC is the statement that the canonical map
\[
  A\xrightarrow{\sim} (\Spec A\to R)
\]
given by mapping $a:A$ to the homomorphism evaluating at $a$, is an equivalence.

\printbibliography

\end{document}

