% latexmk -pdflatex='xelatex %O %S' -pvc -pdf slides.tex
\documentclass{beamer}

\beamertemplatenavigationsymbolsempty

% für xelatex:
\usepackage{xltxtra}
\usepackage{unicode-math}

% literatur
\usepackage[backend=biber,style=alphabetic]{biblatex}

\addbibresource{../util/literature.bib}

\usepackage{../util/zariski}

\usepackage{csquotes}
\usepackage{hyperref}
\usepackage{tikz}
\usetikzlibrary{cd,arrows,shapes,calc,through,backgrounds,matrix,trees,decorations.pathmorphing,positioning,automata}
\usepackage{graphicx}
\usepackage{color}

\usepackage{mathpartir}
\newcommand{\yields}{\vdash}
\newcommand{\cbar}{\, | \,}


% für tabellen
\usepackage{booktabs}

\title[ConCoh]
{\v{C}ech Cohomology in Homotopy Type Theory}
\author[Author, Anders] 
{Ingo Blechschmidt, Felix Cherubini, David Wärn}

\begin{document}

\date{}
\begin{frame}
  \titlepage
\end{frame}

\begin{frame}
  \frametitle{Acknowledgements and Subprojects}
  The approach to synthetic algebraic geometry is based on work by  \textbf{Ingo Blechschmidt}, \textbf{Anders Kock} and \textbf{David Jaz Myers}.

  \pause
  \vspace{0.25cm}
  The approach to cohomology we use was proposed by \textbf{Michael Shulman} in 2013 and was later worked out by \textbf{Floris van Doorn}.
  It is known to many people in the field.

  \pause
  \vspace{0.5cm}
  Present work is joint with
  Ingo Blechschmidt,
  Hugo Moeneclaey,
  Josselin Poiret, and
  David Wärn.

  \vspace{0.25cm}
  \begin{tabular}{lp{7.5cm}}
    \textit{Foundations} & (Felix, Matthias, Thierry) \\
    \textit{Proper Schemes} & (David, Felix, Matthias, Thierry) \\
    \textit{Differential Geometry} & (David, Felix, Hugo, Matthias) \\
    \textit{\v{C}ech Cohomology} & (David, Felix, Ingo) \\
    \textit{Formalization} & (Felix, Josselin, Matthias) \\
  \end{tabular}

  \vspace{0.25cm}
  \url{https://github.com/felixwellen/synthetic-zariski/}
\end{frame}

\begin{frame}
  \vspace{1cm}
  \begin{center}
    \begin{tikzpicture} 
      [ 
      node distance=.8cm, 
      circled/.style={draw, ellipse, ultra thick, fill=blue!12},
      edge from parent/.style={very thick,draw=black,-latex},
      plaintext/.style={}
      ]

      \tikzstyle{level 1}=[sibling angle=81,level distance=4cm]

      \node[circled] (hott) {HoTT+Axioms} [counterclockwise from=207]
      child { 
        node[circled] (grp) {$\infty$-Groupoids} 
        edge from parent [-latex] node (grp) 
        {
          \begin{tikzpicture}[scale=0.8, rotate=25]
            \draw[-, red, ultra thick] (0,0) to (1,1);
            \draw[-, red, ultra thick] (1,0) to (0,1);
          \end{tikzpicture}
        } 
      }
      child { node[matrix, circled, inner sep=0pt] (smgrp) 
        {
          \node[circled, minimum height=0.7cm, minimum width=3cm] (mfd) {Schemes};
          \node[below of=mfd, text width=3cm,
          node distance=0.9cm] 
          {``Cubical Zariski-sheaves''}; \\
        }
      }
      ;
    \end{tikzpicture}
  \end{center}
  \vspace{1cm}
  {\footnotesize * Schemes = quasi-compact, quasi-separated schemes of finite type}
\end{frame}

\begin{frame}
  \frametitle{Synthetic algebraic geometry}
  \textbf{Axiom:} We have a local, commutative ring $R$. \\

  \pause
  \vspace{0.25cm}
  For a finitely presented $R$-algebra $A$, define:
  \[ \Spec(A):\equiv \Hom_{R\text{-algebra}}(A,R)\]
  \pause
  \textbf{Axiom (synthetic quasi-coherence (SQC)):} \\
  For any finitely presented $R$-algebra $A$, the map
  \[ a\mapsto (\varphi\mapsto \varphi(a)):A \xrightarrow{\sim} R^{\Spec(A)} \]
  is an equivalence.

  \pause
  \vspace{0.25cm}
  \textbf{Example:}
  $\Spec(R[X]) = R$.
  \pause
  Thus:
  \[ R[X] \xrightarrow{\sim} R^R \]
  \[ \text{polynomials} = \text{functions}\rlap{\quad\text{!!}} \]
\end{frame}

\begin{frame}
  \frametitle{Towards schemes}
  For $f:A$ define:
  \[ D(f):\equiv \{\, x:\Spec(A) \mid x(f)\text{ is invertible} \,\}\]
  \pause
  For $A = R$, we get \emph{open propositions}:
  \[ \mathrm{OpenProp} :\equiv
    \{\,(r_1 \text{ inv.}) \lor \dots \lor (r_n \text{ inv.}) \mid r_i : R \,\}
  \]
  \pause
  \textbf{Lemma:}
  There is an embedding:
  \[ \{\,D(f_1) \cup \dots \cup D(f_n) \mid f_i : A \,\}
     \quad\hookrightarrow\quad
     \mathrm{OpenProp}^{\Spec(A)}
  \]
  But is it an equivalence??\\
  \pause
  Yes, using \emph{Zariski-local choice}!
\end{frame}

\begin{frame}
  \frametitle{Zariski-local choice}
  \textbf{Axiom (Zariski-local choice):}\\
  For every surjective $\pi$, there merely exist local sections $s_i$
  \[ \begin{tikzcd}[ampersand replacement=\&, column sep=small]
    \& E \ar[d, two heads, "\pi"] \\
    D(f_i) \ar[r, hook] \ar[ur, bend left, dashed, "s_i"] \& \Spec(A)
  \end{tikzcd} \]
  with $f_1, \dots, f_n : A$ coprime.
\end{frame}

\begin{frame}
  \frametitle{Some more results}
  \begin{itemize}
    \item
      $\mathrm{OpenProp}^{\Spec(A)} \cong \{\text{f.g. radical ideals of $A$}\}$
    \item
      $\mathrm{OpenProp}$ is closed under $\Sigma$-types.
    \item
      All functions $\Spec A \to \mathbb{N}$ are bounded.
    \item
      The type of schemes is closed under $\Sigma$-types.
  \end{itemize}

  \pause
  \vspace{0.25cm}
  For $A : X \to \mathrm{Ab}$, define \emph{cohomology} as:
  \[ H^n(X, A) :\equiv \Big\| \prod_{x:X}K(A_x,n) \Big\|_{\mathrm{set}} \]
  \begin{itemize}
    \item
      $H^n$ coincides with \v{C}ech-Cohomology (for \emph{separated} schemes).
    \item
      A scheme $X$ is affine if and only if
      \[ H^n(X, M) = 0 \]
      for all $M : X \to R\text{-}\mathrm{Mod}_{\mathrm{wqc}}$ and $n > 0$.
  \end{itemize}
\end{frame}

\begin{frame}
  \centering
  Thank you!
\end{frame}

\begin{frame}
  \frametitle{Cohomology of sheaves}
  Let $X$ be a type and $\mathcal F:X\to \mathrm{Ab}$ a dependent abelian group on $X$. \\
  \pause
  The $n$-th cohomology group of $\mathcal F$ is
  \[ H^n(X,\mathcal F):\equiv\left\|\prod_{x:X}K(\mathcal F_x,n)\right\|_0 \]
  \pause
  Properties: \\
  \pause
  The $H^n(X,\mathcal F)$ are all abelian groups. \\
  \pause
  Functoriality, covariant in $\mathcal F$, contravariant in $X$. \\
  \pause
  Some long exact sequence for coefficients. \\
  \pause
  We have a Mayer-Vietoris-Lemma and more generally correspondence with \v{C}ech-Cohomology, for nice enough spaces. \\
\end{frame}

\begin{frame}
  \frametitle{Zariski-Choice and Cohomology}  
  Let $X=\Spec(A)$ and $M:X\to R\text{-Mod}$ such that $((x:D(f))\to M_x)=((x:X)\to M)_f$, then
  \[ H^1(X,M)=0 \]
  \pause
  \textbf{Proof:} Let $|T|: H^1(X,M)\equiv \|(x:X)\to K(M_x,1)\|_0$ and from that $(x:X)\to \|T_x=M_x\|$.
  \pause
  Our third axiom, \textbf{Zariski-local choice}, merely gives us coprime $f_1,\dots,f_n:A$, such that for each $i$ we have
  \[
    s_i:(x:D(f_i)) \to T_x=M_x
    \rlap{.}
  \]
  \pause
  So for $t_{ij}(x):\equiv s_j(x)^{-1}\cdot s_i(x)$ we have $t_{ij}+t_{jk}=t_{ik}$.
  \pause
  By algebra, we get $u_i:(x:D(f_i))\to M_x$ with $t_{ij}=u_i-u_j$.
  Then the $\tilde{s}_i:\equiv s_i-u_i$ glues to a global trivialization.
\end{frame}

\end{document}
