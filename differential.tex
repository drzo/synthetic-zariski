\subsection{The Tangent Space}

We will use the concept of tangent spaces from synthetic differential geometry.
More concretely, we follow \cite{david-orbifolds}[Section 4]
on the subject, which also uses homotopy type theory as a basis.

\begin{definition}
  The \notion{first order disk}\index{$\D(n)$} of dimension $n$ is the type
  \[ \D(n)\colonequiv \{x:R^n \mid xx^T = 0 : R^{n \times n}\}\rlap{.}\]
\end{definition}

By definition, the first order disks are affine, since
\[
  \{x:R^n \mid xx^T = 0 : R^{n \times n}\} = \Spec R[X_1,\dots,X_n]/(X\cdot X^T)
\]
-- with $X=(X_1,\dots,X_n)$.
We will describe these algebras more explicitly.

\begin{lemma}%
  For all $n:\N$,
  \[
    R[X_1,\dots,X_n]/(X\cdot X^T) = R+\varepsilon_1R+\dots+\varepsilon_nR
  \]
  where $\varepsilon_i\varepsilon_j=0$ for all $i,j$ and
  where the equality is in the type of $R$-algebras and the right hand side is a direct sum of $R$-modules,
\end{lemma}

The first order disks are quite convenient to work with,
as shown in the following example of a tangent bundle\index{tangent bundle},
which we will define below.

\begin{example}[using \axiomref{sqc}]%
  \label{example-tangent-bundle}
  Let $S\colonequiv\Spec R[X,Y]/(X^2+Y^2-1)$.
  We will see that $S^{\D(1)}$ is affine by the following calculation:
  \begin{align*}
    S^{\D(1)} &= \Spec R[\varepsilon] \to \Spec R[X,Y]/(X^2+Y^2-1) \\
              &= \Hom_R(R[X,Y]/(X^2+Y^2-1), R[\varepsilon]) \\
              &= \{{z,w:R[\varepsilon]} \mid z^2+w^2=1 \} \\
              &= \{x+\varepsilon u, y+\varepsilon v:R+\varepsilon R \mid (x+\varepsilon u)^2+(y+\varepsilon v)^2=1 \} \\
              &= \{x,y,u,v:R \mid x^2+y^2+\varepsilon (2xu+2yv)=1 \} \\
              &=\{x,y,u,v:R \mid (x^2+y^2=1)\times(2xu+2yv=0)\} \\
    &=\Spec R[X,Y,U,V]/(X^2+Y^2-1,2XU+2YV)
  \end{align*}
\end{example}

\begin{definition}
  We write $R[\varepsilon]$ for the $R$-algebra $R[X]/(X^2)$ and its generator $\varepsilon\colonequiv X+(X^2)$
  in the context of $\D(1)=\Spec R[\varepsilon]$\index{$R[\varepsilon]$}.
\end{definition}

\begin{lemma}[using \axiomref{sqc}]%
  \label{evaluate-on-first-order-disk}
  Let $f:\A^n\to R$ be a function or equivalently, $f:R[X_1,\dots,X_n]$.
  Then, for all $x=(x_1,\dots,x_n):R[\varepsilon]^n$,
  \[
    f((x_1+\varepsilon u_1, \dots , x_n+\varepsilon u_n))=f(x_1,\dots,x_n)+\varepsilon(\partial_{X_1}f\cdot u_1+\dots+\partial_{X_n}f\cdot u_n)
    \rlap{.}
  \]
\end{lemma}

\begin{proof}
  TODO, maybe reference to code.
\end{proof}

More generally than first order disks, we can consider infinitesimal varieties:

\begin{definition}
  \begin{enumerate}[(a)]
  \item A \notion{Weil algebra} over $R$ is a finitely presented $R$-algebra
    $W$ together with a homomorphism $\pi : W \to R$, such that the kernel of $\pi$ is a nilpotent ideal.
  \item An \notion{infinitesimal variety} is a pointed type $D$,
    such that $D = (\Spec W, \pi)$ for a Weil algebra $(W, \pi)$.
  \end{enumerate}
\end{definition}

Note that the kernel of $\pi$ is a finitely generated ideal,
as the kernel of a surjective homomorphism between finitely presented algebras.
To ask that $\ker \pi$ is nilpotent as an ideal
is therefore the same as to ask that each of its elements is nilpotent.

\begin{lemma}[using \axiomref{loc}, \axiomref{sqc}]%
  \label{characterization-infinitesimal-variety}
  A pointed affine scheme $(\Spec A, \pi)$
  is an infinitesimal variety
  if and only if,
  for every $x : \Spec A$ we have $\lnot \lnot (x = \pi)$.
\end{lemma}

\begin{proof}
  First note that we can choose a finite presentation
  $A = R[X_1, \dots, X_n]/(r_1, \dots, r_m)$
  of $A$
  such that $\pi(X_i) = 0$ for all $i$
  (see~\cite{david-orbifolds}[Lemma 4.1.3])
  and therefore $\ker \pi = (X_1, \dots, X_n)$.

  Assume $(\Spec A, \pi)$ is an infinitesimal variety.
  By (\axiomref{sqc}), this means that $(A, \pi)$ itself is a Weil algebra,
  so every $X_i$ is nilpotent in $A$.
  Now if $x : \Spec A$ is any homomorphism $A \to R$,
  then $x(X_i)$ is also nilpotent in $R$,
  meaning, by \cref{nilpotence-double-negation}, that $\lnot \lnot (x(X_i) = 0)$.
  Since we have this for all $i = 1, \dots, n$
  and double negation commutes with finite conjunctions,
  we have $\lnot \lnot (x = \pi)$.

  Now assume $\lnot \lnot (x = \pi)$ for all $x : \Spec A$.
  To show that $(A, \pi)$ is a Weil algebra,
  let $f : A$ be given with $\pi(f) = 0$.
  Then in particular we have $\lnot \lnot (x(f) = 0)$
  for every $x : \Spec A$.
  But this means $D(f) = 0$
  (using (\axiomref{loc}) for $\inv(x(f)) \Rightarrow x(f) \neq 0$),
  so $f$ is nilpotent by \cref{standard-open-empty}.
\end{proof}

The following lemma allows us to reduce
maps from infinitesimal varieties to schemes
to the affine case:

\begin{lemma}%
  \label{affine-opens-infinitesimal-closed}
  Let $X$ be a scheme, $V$ an infinitesimal variety and $p:X$.
  Then for all affine open $U\subseteq X$
  containing $p$, there is an equivalence
  of pointed mapping types:
  \[ V\to^. (X, p) \quad\cong\quad V\to^. (U, p) \]
\end{lemma}

\begin{proof}
  By \cref{characterization-infinitesimal-variety},
  all points in $V$ are not not equal,
  so all points in the image of a pointed map
  \[ V \to^. (X, p) \]
  will be not not equal to $p$.
  Since $p \in U$ and open propositions are double-negation stable
  (\cref{open-union-intersection}),
  the image is contained in $U$
  and the map factors uniquely over $(U, p)$.
\end{proof}

\begin{definition}
  Let $X$ be a type.
  \begin{enumerate}[(a)]
  \item For $p:X$ a point in $X$.
    The \notion{tangent space}\index{$T_pX$} of $X$ at $p$, is the type
    \[ T_pX\colonequiv \{d:\D(1)\to X\vert d(0)=p \}\rlap{.}\]
  \item The \notion{tangent bundle} of $X$ is the type $X^{\D(1)}$.
  \end{enumerate}
\end{definition}

We start our analysis of the tangent bundle,
by computing it for affine schemes:

\begin{lemma}[using \axiomref{sqc}]%
  \label{affine-tangent-bundle-scheme}
  If $X=\Spec R[X_1,\dots,X_n]/(f_1,\dots,f_l)$,
  then the tangent bundle of $X$ is an affine scheme and we have
  \[
    X^{\D(1)}=\Spec R[X_1,\dots,X_n,U_1,\dots,U_n]/(f_1,\dots,f_n,J_f U^T)
  \]
  -- where $J_f$ is the Jacobi-matrix of $f=(f_1,\dots,f_n)$:
  \[
    J_f=
    \left(
      \begin{matrix}
        \partial_{X_1}f_1 & \dots & \partial_{X_n}f_1 \\
        \vdots &  & \vdots \\
        \partial_{X_1}f_n & \dots & \partial_{X_n}f_n 
      \end{matrix}
    \right)
    \rlap{.}
  \]
\end{lemma}

\begin{proof}
  We calculate as in \cref{example-tangent-bundle}, using \cref{evaluate-on-first-order-disk}:
  \begin{align*}
    X^{\D(1)}&= \Spec R[\epsilon] \to \Spec R[X_1,\dots,X_n]/(f_1,\dots,f_l)\\
             &= \Hom_R(R[X_1,\dots,X_n]/(f_1,\dots,f_l),R[\epsilon]) \\
             &= \{ x:R[\epsilon]^n \mid \forall i. f_i(x)=0 \} \\
             &= \{ (x_1+\varepsilon u_1, \dots , x_n+\varepsilon u_n):R[\epsilon]^n \mid \forall i. f_i(x_1,\dots,x_n)+\varepsilon(\partial_{X_1}f_i\cdot u_1+\dots+\partial_{X_n}f_i\cdot u_n) =0 \} \\
             &= \{ (x,u):R^{n+n}\mid (\forall i . f_i(x)=0) \times (J_f\cdot u^T=0) \} \\
             &= \Spec R[X_1,\dots,X_n,U_1,\dots,U_n]/(f_1,\dots,f_n,J_f U^T)
               \rlap{.}
  \end{align*}
\end{proof}

\begin{theorem}[using \axiomref{sqc}]%
  Let $X$ be a scheme,
  then the tangent bundle, $X^{\D(1)}$
  is a scheme.
\end{theorem}

\begin{proof}
  Let $U_1,\dots,U_n$ be an open affine cover of $X$.
  By \cref{affine-tangent-bundle-scheme}, $U_i^{\D(1)}$ is an affine scheme $T_i\subseteq X^{\D(1)}$.
  Since we have a projection $T_i\to U_i$, each $T_i$ is open in $X^{\D(1)}$.
  For each tangent vector $v:X^{\D(1)}$,
  $v(0):U_i$ for some $i$,
  so by \cref{affine-opens-infinitesimal-closed},
  $v$ is contained in $T_i$.
  This means the $T_i$ cover $X^{\D(1)}$.
\end{proof}

We transfer a result of Myers \cite{david-orbifolds}[Theorem 4.2.19] to schemes:

\begin{theorem}[using \axiomref{sqc}]%
  Let $X$ be a scheme.
  There is a coherent $R$-module structure on the tangent spaces $T_pX$,
  i.e. there is a map
  \[ (p:X)\to \text{is-$R$-module}(T_pX)\rlap{.}\]
\end{theorem}

\begin{proof}
  Following the proof of \cite{david-orbifolds}[Theorem 4.2.19],
  it is enough to show, that any scheme is infinitesimal linear,
  which amount to showing that
  \[
    \begin{tikzcd}
      X^{\D(n+m)}\ar[r]\ar[d] & X^{\D(n)}\ar[d] \\
      X^{\D(m)}\ar[r]         & X
    \end{tikzcd}
  \]
  is a pullback for all $n,m:\N$.
  This is equivalent to:
  For any $p:X$, we have an equivalence of types of pointed maps
  \begin{align*}
    &\D(n+m)\to^.X \\
    = &\left(\D(n)\to^.X\right) \times \left({\D(m)}\to^.X\right).
  \end{align*}
  Let $U=\Spec A$ be an affine open containing $p$.
  By \cref{affine-opens-infinitesimal-closed},
  we only need to show:
  \begin{align*}
    &\D(n+m)\to^.U \\
    = &\left(\D(n)\to^.U\right) \times \left({\D(m)}\to^.U\right).
  \end{align*}
  \emph{tbc}
\end{proof}
