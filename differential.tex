\subsection{The Tangent Space}

We will use the concept of tangent spaces from synthetic differential geometry.
More conretely, we follow \cite{david-orbifolds}[Section 4]
on the subject, which also uses homotopy type theory as a basis.

\begin{definition}
  The \notion{first order disk}\index{$\D(n)$} of dimension $n$ is the type
  \[ \D(n)\colonequiv \{x:R^n\vert x^Tx=0\}\rlap{.}\]
\end{definition}

More genrally, we can consider infinitesimal varieties:

\begin{definition}
  \begin{enumerate}[(a)]
  \item A \notion{Weil-algebra} over $R$ is a finitely presented $R$-algebra
    $W$ together with a homomorphism $a:W\to R$, such that the kernel of $a$ is a nilpotent ideal.
  \item An \notion{infinitesimal variety}, is a type $D$, such that $D=\Spec W$ for a Weil-algebra $W$.
  \end{enumerate}
\end{definition}

The following lemma is hopefully correct and let's us reduce
from a scheme
to the affine case:

\begin{lemma}
  \label{affine-opens-infinitesimal-closed}
  Let $X$ be a scheme, $V$ an infinitesimal variety and $p:X$.
  Then for all affine open $U\subseteq X$
  containing $p$, there is an equivalence
  of pointed mapping types:
  \[ V\to^. X \cong V\to^. U\]
\end{lemma}

\begin{definition}
  Let $X$ be a type and $p:X$ a point in $X$.
  The \notion{tangent space}\index{$T_pX$} of $X$ at $p$, is the type
  \[ T_pX\colonequiv \{d:\D(1)\to X\vert d(0)=p \}\rlap{.}\]
\end{definition}

We transfer a result of Myers \cite{david-orbifolds}[Theorem 4.2.19] to schemes:

\begin{theorem}%
  Let $X$ be a scheme.
  There is a coherent $R$-module structure on the tangent spaces $T_pX$,
  i.e. there is a map
  \[ (p:X)\to \text{is-$R$-module}(T_pX)\rlap{.}\]
\end{theorem}

\begin{proof}
  Following the proof of \cite{david-orbifolds}[Theorem 4.2.19],
  it is enough to show, that any scheme is infinitesimal linear,
  which amount to showing that
  \[
    \begin{tikzcd}
      X^{\D(n+m)}\ar[r]\ar[d] & X^{\D(n)}\ar[d] \\
      X^{\D(m)}\ar[r]         & X
    \end{tikzcd}
  \]
  is a pullback for all $n,m:\N$.
  This is equivalent to:
  For any $p:X$, we have an equivalence of types of pointed maps
  \[ \left({\D(n+m)}\to^.X\right) = \left({\D(n)}\to^.X \times {\D(m)}\to^.X\right)\rlap{.} \]
  Let $U=\Spec A$ be an affine open containing $p$.
  By \cref{affine-opens-infinitesimal-closed},
  we only need to show:
  \[ \left({\D(n+m)}\to^.U\right) = \left({\D(n)}\to^.U \times {\D(m)}\to^.U\right)\rlap{.} \]
  \emph{tbc}
\end{proof}