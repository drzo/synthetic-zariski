
We give two definitions of projective space, which differ only in size.

\begin{definition}%
  \begin{enumerate}[(a)]
  \item An $n$-dimensional $R$-\notion{vector space} is an $R$-module $V$,
    such that $\| V = R^n \|$. 
  \item We write $\Vect{R}{n}$ for the type of these vector spaces and $V\setminus\{0\}$ for the type
    \[ \sum_{x:V}x\neq 0\]
  \item A \notion{vector bundle} on a type $X$ is a map $V:X\to \Vect{R}{n}$. 
  \end{enumerate}
\end{definition}

The following defines projective space as the space of lines in a vector space.
This is a large type.
We will see below, that there is also a small definition of the same type.

\begin{definition}%
  \begin{enumerate}[(a)]
  \item   A \notion{line} in a $R$-vector space $V$ is a subtype $\mathcal L:V\to \Prop$,
    such that there exists an $x:V\setminus\{0\}$ with
    \[ \prod_{y:V}\left(\mathcal L (y) \Leftrightarrow \exists c:R.y=c\cdot x\right)\]
  \item The space of all lines in a fixed $n$-dimensional vector space $V$ is the projectivization of $V$:
    \[ \bP(V)\colonequiv \sum_{\mathcal L:V\to \Prop} \mathcal L \text{ is a line}  \]
  \item \notion{Projective $n$-space} is the projectivization of $\bA^{n+1}$,
    $\bP^n \colonequiv \bP(\bA^{n+1})$.
  \end{enumerate}
\end{definition}

Lines are closed subschemes (not defined yet):

\begin{proposition}%
  For any line $\mathcal L : \bA^{2}\to \Prop$, there merely is a degree one polynomial $P\in R[X_0,X_1]$ such that
  for all $x:\bA^{2}$, $\mathcal L(x)$ is equivalent to $P(x)=0$.
\end{proposition}
\begin{proof}
  For $x\equiv(a,b)$ a point on $\mathcal L$,
  let $P$ be the polynomial, given by inner product with $(b,-a)$. 
\end{proof}

\ignore{
\begin{definition}
  Let $n:\NN$. Projective $n$-space is the type given 
\end{definition}
}

\begin{theorem}[using \axiomref{sqc},\axiomref{loc}]
  $\bP^n$ is a scheme.
\end{theorem}

\begin{proof}
  \dots
\end{proof}

\begin{lemma}
  All functions $\bP^1 \to R$ are constant.
\end{lemma}

\begin{proof}
  \dots
\end{proof}

\begin{lemma}[using (\axiomref{sqc}), (\axiomref{loc})]
  Let $p \neq q \in \bP^n$ be given.
  Then there exists a map $f : \bP^1 \to \bP^n$
  such that $f([0 : 1]) = p$, $f([1 : 0]) = q$.
\end{lemma}

\begin{proof}
  What we want to prove is a proposition,
  so we can assume chosen $a, b \in \bA^{n+1} \setminus \{0\}$
  with $p = [a]$, $q = [b]$.
  Then we set
  \[ f([x, y]) \colonequiv [xa + yb] \rlap{.}\]
  Let us check that $xa + yb \neq 0$.
  By \dots,
  we have that $x$ or $y$ is invertible
  and both $a$ and $b$ have at least one invertible entry.
  If $xa = - yb$
  then it follows that $x$ and $y$ are both invertible
  and therefore $a$ and $b$ would be linearly equivalent,
  contradicting the assumption $p \neq q$.
  Of course $f$ is also well-defined
  with respect to linear equivalence in the pair $(x, y)$.
\end{proof}

\begin{lemma}
  Let $n \geq 1$.
  For every point $p \in \bP^n$,
  we have $p \neq [1 : 0 : 0 : \dots]$
  or $p \neq [0 : 1 : 0 : \dots]$.
\end{lemma}

\begin{proof}
  Let $p = [a]$ with $a \in \bA^{n+1} \setminus \{0\}$.
  By \dots,
  there is an $i \in \{0, \dots, n\}$ with $a_i \neq 0$.
  If $i = 0$ then $p \neq [0 : 1 : 0 : \dots]$,
  if $i \geq 1$ then $p \neq [1 : 0 : 0 : \dots]$.
\end{proof}

\begin{theorem}
  All functions $\bP^n \to R$ are constant,
  that is,
  \[ H^0(\bP^n, R) \colonequiv (\bP^n \to R) = R \rlap{.} \]
\end{theorem}

\begin{proof}
  \dots
\end{proof}
