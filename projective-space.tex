
\subsection{Construction of Projective Spaces}
We give two definitions of projective space, which differ only in size.

\begin{definition}%
  \begin{enumerate}[(a)]
  \item An $n$-dimensional $R$-\notion{vector space} is an $R$-module $V$,
    such that $\| V = R^n \|$. 
  \item We write $\Vect{R}{n}$ for the type of these vector spaces and $V\setminus\{0\}$ for the type
    \[ \sum_{x:V}x\neq 0\]
  \item A \notion{vector bundle} on a type $X$ is a map $V:X\to \Vect{R}{n}$. 
  \end{enumerate}
\end{definition}

The following defines projective space as the space of lines in a vector space.
This is a large type.
We will see below, that there is also a small definition of the same type.

\begin{definition}%
  \begin{enumerate}[(a)]
  \item   A \notion{line} in a $R$-vector space $V$ is a subtype $L:V\to \Prop$,
    such that there exists an $x:V\setminus\{0\}$ with
    \[ \prod_{y:V}\left(L (y) \Leftrightarrow \exists c:R.y=c\cdot x\right)\]
  \item The space of all lines in a fixed $n$-dimensional vector space $V$ is the \notion{projectivization} of $V$:
    \[ \bP(V)\colonequiv \sum_{L:V\to \Prop} L \text{ is a line}  \]
  \item \notion{Projective $n$-space} $\bP^n \colonequiv \bP(\A^{n+1})$ is the projectivization of $\A^{n+1}$.
  \end{enumerate}
\end{definition}

\begin{proposition}%
  \label{lines-are-one-dimensional}
  For any vector space $V$ and line $L\subseteq V$,
  $L$ is 1-dimensional in the sense that $\|L=_{\Mod{R}}R\|$.
\end{proposition}

\begin{proof}
  Let $L$ be a line.
  We merely have $x:V\setminus\{0\}$ such that 
  \[ \prod_{y:V}\left(L (y) \Leftrightarrow \exists c:R.y=c\cdot x\right)\]
  We may replace the ``$\exists$'' with a ``$\sum$'',
  since $c$ is uniquely determined for any $x,y$.
  This means we can construct the map $\alpha\mapsto \alpha\cdot x:R\to L$ and it is an equivalence.
\end{proof}

Lines are closed subschemes (at least in $\A^2$):

\begin{proposition}%
  For any line $L : \A^{2}\to \Prop$, there merely is a degree one polynomial $P\in R[X_0,X_1]$ such that
  for all $x:\A^{2}$, $L(x)$ is equivalent to $P(x)=0$.
\end{proposition}

\begin{proof}
  For $x\equiv(a,b)$ a point on $L$,
  let $P$ be the polynomial, given by inner product with $(b,-a)$. 
\end{proof}

We now give the small construction:

\begin{definition}[using \axiomref{loc}, \axiomref{sqc}]%
  \label{projective-space-hit}
  Let $n:\N$.
  \notion{Projective $n$-space}\index{$\bP^n$} $\bP^n$ is the setquotient of the type $\A^{n+1}\setminus\{0\}$ by the relation
  \[
    x \sim y \colonequiv \sum_{\lambda : R} \lambda x=y\rlap{.}
  \]
  By \cref{generalized-field-property}, the non-zero vector $y$ has an invertible entry,
  so that the right hand side is a proposition and $\lambda$ is a unit.
  We write $[x_0:\dots:x_n]:\bP^n$ for the equivalence class of $(x_0,\dots,x_n):\A^{n+1}\setminus\{0\}$.
\end{definition}

\begin{theorem}[using \axiomref{sqc}, \axiomref{loc}]
  $\bP^n$ is a scheme.
\end{theorem}

\begin{proof}
  \dots
\end{proof}

\subsection{Functions on $\bP^n$}

\begin{example}[using \axiomref{loc}]
  Let $s:\bP^1\to \bP^1$ be given by $s([x:y])\colonequiv [x^2:y^2]$
  (see \cref{projective-space-hit} for notation).
  Let us compute some fibers of $s$. The fiber $\fib_s([0:1])$ is
  \[
    \sum_{[x:y]:\bP^1}[x^2:y^2]=[0:1]\rlap{.}
  \]
  So for any $x:R$ with $x^2=0$, $[x,1]:\fib_s([0:1])$  and
  any other point $(x,y)$ such that $[x:y]$ is in $\fib_s([0:1])$,
  already yields an equivalent point, since $y$ has to be invertible.

  This shows that the fiber over $[0:1]$ is a first order disk, i.e. $\D(1)=\{x:R|x^2=0\}$.
  The same applies to the point $[1:0]$.
  To analyze $\fib_s([1:1])$, let us assume $2\neq 0$ (in $R$).
  Then we know, the two points $[1:-1]$ and $[1:1]$ are in $\fib_s([1:1])$ and they are different.
  Now any $x:R$ with $x^2=1$ is already the $1$ or $-1$,
  since those are apart by \cref{separated-inequality-apartness}, which means that $(x-1)$ or $(x+1)$ is invertible.  
\end{example}

\begin{lemma}
  All functions $\bP^1 \to R$ are constant.
\end{lemma}

\begin{proof}
  \dots
\end{proof}

\begin{lemma}[using \axiomref{sqc}, \axiomref{loc}]
  Let $p \neq q \in \bP^n$ be given.
  Then there exists a map $f : \bP^1 \to \bP^n$
  such that $f([0 : 1]) = p$, $f([1 : 0]) = q$.
\end{lemma}

\begin{proof}
  What we want to prove is a proposition,
  so we can assume chosen $a, b \in \A^{n+1} \setminus \{0\}$
  with $p = [a]$, $q = [b]$.
  Then we set
  \[ f([x, y]) \colonequiv [xa + yb] \rlap{.}\]
  Let us check that $xa + yb \neq 0$.
  By \dots,
  we have that $x$ or $y$ is invertible
  and both $a$ and $b$ have at least one invertible entry.
  If $xa = - yb$
  then it follows that $x$ and $y$ are both invertible
  and therefore $a$ and $b$ would be linearly equivalent,
  contradicting the assumption $p \neq q$.
  Of course $f$ is also well-defined
  with respect to linear equivalence in the pair $(x, y)$.
\end{proof}

\begin{lemma}
  Let $n \geq 1$.
  For every point $p \in \bP^n$,
  we have $p \neq [1 : 0 : 0 : \dots]$
  or $p \neq [0 : 1 : 0 : \dots]$.
\end{lemma}

\begin{proof}
  Let $p = [a]$ with $a \in \A^{n+1} \setminus \{0\}$.
  By \dots,
  there is an $i \in \{0, \dots, n\}$ with $a_i \neq 0$.
  If $i = 0$ then $p \neq [0 : 1 : 0 : \dots]$,
  if $i \geq 1$ then $p \neq [1 : 0 : 0 : \dots]$.
\end{proof}

\begin{theorem}
  All functions $\bP^n \to R$ are constant,
  that is,
  \[ H^0(\bP^n, R) \colonequiv (\bP^n \to R) = R \rlap{.} \]
\end{theorem}

\begin{proof}
  \dots
\end{proof}

\subsection{Line Bundles on $\bP^n$}

We will construct Serre's twisting sheaves in this section,
starting with the ``minus first''.
The following works because of \cref{lines-are-one-dimensional}.

\begin{definition}
  The \notion{tautological bundle} is the line bundle $\mathcal O_{\bP^n}(-1):\bP^n\to \Mod{R}$,
  given by
  \[ (L:\bP^n)\mapsto L\rlap{.}\]
\end{definition}

\begin{definition}
  The \notion{dual}\index{$\mathcal L^\vee$} $\mathcal L^\vee$ of a line bundle $\mathcal L:\bP^n\to \Mod{R}$,
  is the line bundle given by
  \[ (x:\bP^n)\mapsto \Hom_{\Mod{R}}(\mathcal L_x,R)\rlap{.}\]
\end{definition}
