
We give two definitions of projective space, which differ only in size.

\begin{definition}
  \begin{enumerate}
  \item An $n$-dimensional $R$-\notion{vector space} is an $R$-module $V$,
    such that $\| V = R^n \|$. 
  \item We write $\Vect{R}{n}$ for the type of these vector spaces and $V\setminus\{0\}$ for the type
    \[ \sum_{x:V}x\neq 0\]
  \item A \notion{vector bundle} on a type $X$ is a map $V:X\to \Vect{R}{n}$. 
  \end{enumerate}
\end{definition}

The following defines projective space as the space of lines in a vector space.
This is a large type.
We will see below, that there is also a small definition of the same type.

\begin{definition}
  \begin{enumerate}
  \item   A \notion{line} in a $R$-vector space $V$ is a subtype $\mathcal L:V\to \Prop$,
    such that there exists an $x:V\setminus\{0\}$ with
    \[ \prod_{y:V}\left(\mathcal L (y) \Leftrightarrow \exists c:R.y=c\cdot x\right)\]
  \item The space of all lines in a fixed $n$-dimensional vector space $V$ is the projectivization of $V$:
    \[ \bP(V)\colonequiv \sum_{\mathcal L:V\to \Prop} \mathcal L \text{ is a line}  \]
  \item The \notion{projective $n$-space} \notion{$\bP^n$} is the projectivization of $\bA^{n+1}$.
  \end{enumerate}
\end{definition}

Lines are closed subschemes (not defined yet):

\begin{proposition}
  For any line $\mathcal L : \bA^{2}\to \Prop$, there is a degree one polynomial $P\in R[X_0,X_1]$ such that
  for all $x:\bA^{2}$, $\mathcal L(x)$ is equivalent to $P(x)=0$.
\end{proposition}
\begin{proof}
  Take $P$ to be the polynomial, given by inner product 
\end{proof}

\ignore{
\begin{definition}
  Let $n:\NN$. Projective $n$-space is the type given 
\end{definition}
}

\begin{theorem}[\axiomref{sqc},\axiomref{loc}]
  $\bP^n$ is a scheme.
\end{theorem}
